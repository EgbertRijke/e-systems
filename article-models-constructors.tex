\section{Introducing the type constructors}
We will now describe the rules for the type constructors that we don't assume
in our type theory by default. The first of these are the dependent function
types. We also discuss $\tfW$-types, suspensions, truncations. More general higher inductive
types will have to wait until we have introduced models, for the models of the
basic type theory because we will use them as index categories of the diagrams.

\subsection{The universal property of dependent pair types}
This section shouldn't be about extension anymore.

Using weakening and substitution we are able to state the universal property
for extension. It looks a bit more involved, since we cannot directly refer
to the variables in the contexts. On the other hand, we can now plainly see
were there were secretly weakenings going on.

We begin with stating the universal property of the extension $\ctxext{\Gamma}{A}$.
In these rules we assume we have $\jtype{\Gamma}{A}$ in the hypotheses.
\begin{align}
& \inference{}
{\jterm{\ctxext{\Gamma}{A}}{\ctxwk{A}{{\Gamma}{\ctxext{\Gamma}{A}}}}{\pair_A}}\\
& \inference{
  \jtype{\ctxext{\Gamma}{A}}{P}
  \qquad
  \jterm{\ctxext{\Gamma}{A}}{\subst{\pair_A}{\ctxwk{A}{{\Gamma}{P}}}}{f}}
  {\jterm{\ctxext{\Gamma}{A}}{P}{\ind{\tfext_\Gamma(A)}(f)}}\\
& \inference{
  \jtype{\ctxext{\Gamma}{A}}{P}
  \qquad
  \jterm{\ctxext{\Gamma}{A}}{\subst{\pair_A}{\ctxwk{A}{{\Gamma}{P}}}}{f}}
  {\jtermeq{\ctxext{\Gamma}{A}}{\subst{\pair_A}{\ctxwk{A}{{\Gamma}{P}}}}{\subst{\pair_A}{\ctxwk{A}{{\Gamma}{\ind{\tfext_\Gamma(A)}(f)}}}}{f}}
\end{align}

Under the hypothesis that $\jtypeeq{\Gamma}{A}{A'}$
we will also have the rules

\begin{align}
& \inference{}{\jtermeq{\ctxext{\Gamma}{A}}{\ctxwk{A}{{\Gamma}{\ctxext{\Gamma}{A}}}}{\pair_A}{\pair_{A'}}}\\
& \inference{\jtype{\ctxext{\Gamma}{A}}{P}\qquad\jterm{\ctxext{\Gamma}{A}}{\subst{\pair_A}{\ctxwk{A}{{\Gamma}{P}}}}{f}}
{\jtermeq{\ctxext{\Gamma}{A}}{P}{\ind{\tfext_\Gamma(A)}(f)}{\ind{\tfext_\Gamma(A')}(f)}}
\end{align}

Note that we don't need the notion of terms for contexts to state the universal
property of context extension (which is a good thing, for we don't assume to have them).

Next, we give the universal property of the extension $\ctxext{A}{P}$ in context
$\Gamma$.
In all of the following inference rules we assume that $\jtype{\ctxext{\Gamma}{A}}{P}$
is among the hypothesis. The induction principle for extension consists of three
inference rules:

\begin{align}
& \inference{}
{\jterm{\ctxext{{\Gamma}{A}}{P}}{\ctxwk{P}{{A}{\ctxext{A}{P}}}}{\pair_P}}\\
& \inference{
  \jtype{\ctxext{\Gamma}{{A}{P}}}{Q}
  \qquad
  \jterm{\ctxext{{\Gamma}{A}}{P}}{\subst{\pair_P}{\ctxwk{P}{{A}{Q}}}}{f}}
  {\jterm{\ctxext{\Gamma}{{A}{P}}}{Q}{\ind{\tfext_A(P)}(f)}}\\
& \inference{
  \jtype{\ctxext{\Gamma}{{A}{P}}}{Q}
  \qquad
  \jterm{\ctxext{{\Gamma}{A}}{P}}{\subst{\pair_P}{\ctxwk{P}{{A}{Q}}}}{f}}
  {\jtermeq{\ctxext{{\Gamma}{A}}{P}}{\subst{\pair_P}{\ctxwk{P}{{A}{Q}}}}{\subst{\pair_P}{\ctxwk{P}{{A}{\ind{\tfext_A(P)}(f)}}}}{f}}
\end{align}

As with context extension, we shall require two more inference rules stating that
$\pair_P$ and $\ind{\tfext_A(P)}$ are invariant under judgmental equality.

\subsection{The unit type}
Since we don't have a notion of terms of a context, we just say that the context
$\unit$ is the terminal context.

\begin{align}
& \inference{}{\jctx{\unit}}\\
& \inference{\jctx{\Gamma}}{\jhom{\Gamma}{\unit}{\tounit{\Gamma}}}\\
& \inference{\jctx{\Gamma}\qquad\jhom{\Gamma}{\unit}{f}}{\jhomeq{\Gamma}{\unit}{f}{\tounit{\Gamma}}}
\end{align}

Note that we don't have to require that $\jhomeq{\Gamma}{\unit}{\tounit{\Gamma}}{\tounit{\Delta}}$
whenever we have a judgmental equality $\jctxeq{\Gamma}{\Delta}$, since this already follows from the third rule.

When the context $\unit$ is present, we may use the expression $\jtype{}{\Gamma}$
as a shorthand for the judgment $\jtype{\unit}{\ctxwk{\unit}{\Gamma}}$. Likewise,
we may use the expression $\jtermc{\Gamma}{i}$ as a shorthand
for the judgment $\jhom{\unit}{\Gamma}{i}$. If we have a type $A$ in context
$\Gamma$, we may use the expression $\jtype{}{\subst{i}{A}}$ to stand for
the judgment $\jtype{\unit}{\subst{i}{\ctxwk{\unit}{A}}}$. We see that in every
respect, contexts are types in the empty context.

We have created a strictly terminal object $\unit$. This is not necessary when
we're working in a context. We introduce the unit type $\unit_\Gamma$ in context
$\Gamma$ in the familiar type theoretical way.

\begin{align}
& \inference{\jctx{\Gamma}}{\jtype{\Gamma}{\unit_\Gamma}}\\
& \inference{\jctx{\Gamma}}{\jterm{\Gamma}{\unit_\Gamma}{\ttt_\Gamma}}\\
& \inference{\jtype{\ctxext{\Gamma}{\unit_\Gamma}}{P}\qquad\jterm{\Gamma}{\subst{\ttt_\Gamma}{P}}{u}}
          {\jterm{\ctxext{\Gamma}{\unit_\Gamma}}{P}{\ind{\unit_\Gamma}(u)}}\\
& \inference{\jtype{\ctxext{\Gamma}{\unit_\Gamma}}{P}\qquad\jterm{\Gamma}{\subst{\ttt_\Gamma}{P}}{u}}
          {\jtermeq{\Gamma}{\subst{\ttt_\Gamma}{P}}{\subst{\ttt_\Gamma}{\ind{\unit_\Gamma}(u)}}{u}}
\end{align}

\subsection{Subterminal types}
The subterminal types we're about to present are strict, so they're more like \verb+Prop+
in \Coq. We can define subterminal types in two ways: the first equalizes all elements
of the subject type; the second is universal with the property that for every constant
map factors through it.

\subsubsection{Equalizing all terms}

\subsubsection{Factorizing constant maps}
Let $\jhom{\Gamma}{\Delta}{f}$. We can weaken $f$ by $\Gamma$ to obtain a term
$\jterm{\ctxext{\Gamma}{\ctxwk{\Gamma}{\Gamma}}}{\ctxwk{\Gamma}{{\Gamma}{\Delta}}}{\ctxwk{\Gamma}{f}}$.
This term is `like the function $\lam{x}{y}f(y)$'. Likewise, we can weaken $f$
by $\ctxwk{{\Gamma}{\Gamma}}$ to obtain a term
$\jterm{\ctxext{\Gamma}{\ctxwk{\Gamma}{\Gamma}}}{\ctxwk{\Gamma}{{\Gamma}{\Delta}}}{\ctxwk{{\Gamma}{\Gamma}}{f}}$,
which is `like the function $\lam{x}{y}f(x)$'. Since we have
the judgmental equality 
$\jtypeeq{\ctxext{\Gamma}{\ctxwk{\Gamma}{\Gamma}}}{\ctxwk{{\Gamma}{\Gamma}}{{\Gamma}{\Delta}}}{\ctxwk{\Gamma}{{\Gamma}{\Delta}}}$ we can consider the judgmental equality between $\ctxwk{\Gamma}{f}$
and $\ctxwk{{\Gamma}{\Gamma}}{f}$, which is exactly what we'll do in the definition
of judgmentally constant.

\begin{defn}
A term $\jhom{\Gamma}{\Delta}{f}$ is said to be \emph{judgmentally constant} if
the judgment
\begin{equation*}
\jtermeq{\ctxext{\Gamma}{\ctxwk{\Gamma}{\Gamma}}}{\ctxwk{{\Gamma}{\Gamma}}{{\Gamma}{\Delta}}}{\ctxwk{\Gamma}{f}}{\ctxwk{{\Gamma}{\Gamma}}{f}}
\end{equation*}
can be derived.
\end{defn}

\begin{defn}
\begin{align}
& \inference{\jctx{\Gamma}}{\jctx{\tau\Gamma}}\\
& \inference{\jctx{\Gamma}}{\jhom{\Gamma}{\tau\Gamma}{t}}\\
& \inference{\jctx{\Gamma}}{\jtermeq{\ctxext{\Gamma}{\ctxwk{\Gamma}{\Gamma}}}{\ctxwk{{\Gamma}{\Gamma}}{{\Gamma}{\Delta}}}{\ctxwk{\Gamma}{t}}{\ctxwk{{\Gamma}{\Gamma}}{t}}}\\
& \inference{\jhom{\Gamma}{\Delta}{f}\qquad\jtermeq{\ctxext{\Gamma}{\ctxwk{\Gamma}{\Gamma}}}{\ctxwk{{\Gamma}{\Gamma}}{{\Gamma}{\Delta}}}{\ctxwk{\Gamma}{f}}{\ctxwk{{\Gamma}{\Gamma}}{f}}}{\jhom{\tau\Gamma}{\Delta}{\tilde{f}}}\\
& \inference{\jhom{\Gamma}{\Delta}{f}\qquad\jtermeq{\ctxext{\Gamma}{\ctxwk{\Gamma}{\Gamma}}}{\ctxwk{{\Gamma}{\Gamma}}{{\Gamma}{\Delta}}}{\ctxwk{\Gamma}{f}}{\ctxwk{{\Gamma}{\Gamma}}{f}}}
{\jtermeq{\Gamma}{\ctxwk{\Gamma}{\Delta}}{\jcomp{\Gamma}{t}{\tilde{f}}}{f}}
\end{align}
\end{defn}

\begin{lem}
Any term $\jhom{\unit}{\Gamma}{i}$ is judgmentally constant.
\end{lem}

\begin{proof}

\end{proof}

\subsection{Product types}

\subsubsection{Products}
\begin{align}
& \inference{\jctx{\Gamma}\qquad\jctx{\Delta}}{\jctx{\product{\Gamma}{\Delta}}}\\
& \inference{\jctx{\Gamma}\qquad\jctx{\Delta}}{\jhom{{\Gamma}{\Delta}}{\product{\Gamma}{\Delta}}{\pair}}
\end{align}

\subsubsection{Strict products}
As with the unit type, we may use the categorical description of the product
for our type theoretical definition. If we do that, we get strict products.

\begin{align}
& \inference{\jctx{\Gamma}\qquad\jctx{\Delta}}{\jctx{\product{\Gamma}{\Delta}}}\\
& \inference{\jctx{\Gamma}\qquad\jctx{\Delta}}{\jhom{\product{\Gamma}{\Delta}}{\Gamma}{\proj1}}\\
& \inference{\jctx{\Gamma}\qquad\jctx{\Delta}}{\jhom{\product{\Gamma}{\Delta}}{\Delta}{\proj2}}\\
& \inference{\jhom{\greek{E}}{\Gamma}{f}\qquad\jhom{\greek{E}}{\Delta}{g}}
          {\jhom{\greek{E}}{\product{\Gamma}{\Delta}}{\pairp{f,g}}}\\
& \inference{\jhom{\greek{E}}{\Gamma}{f}\qquad\jhom{\greek{E}}{\Delta}{g}}
          {\jhomeq{\greek{E}}{\Gamma}{\jcomp{\greek{E}}{\pairp{f,g}}{\proj1}}{f}}\\
& \inference{\jhom{\greek{E}}{\Gamma}{f}\qquad\jhom{\greek{E}}{\Delta}{g}}
          {\jhomeq{\greek{E}}{\Delta}{\jcomp{\greek{E}}{\pairp{f,g}}{\proj2}}{g}}
\end{align}

\begin{equation}
\inference{\jhom{\greek{E}}{\product{\Gamma}{\Delta}}{h}
           \qquad
           \jhomeq{\greek{E}}{\Gamma}{\jcomp{E}{h}{\proj1}}{f}
           \qquad
           \jhomeq{\greek{E}}{\Delta}{\jcomp{E}{h}{\proj2}}{g}}
          {\jhomeq{\greek{E}}{\product{\Gamma}{\Delta}}{h}{\pairp{f,g}}}
\end{equation}

\subsection{Equalizer types}
Now that we have introduced a terminal object and products the categorical way,
we may just continue and present (strict) equalizers and pullbacks too, just to
see where we get. These notions are probably just not very useful in a univalent
setting until we got a good computational interpretation.

\begin{align}
& \inference{\jhom{\Gamma}{\Delta}{f}\qquad\jhom{\Gamma}{\Delta}{g}}{\jctx{\jequalizer{\Gamma}{f}{g}}}\\
& \inference{\jhom{\Gamma}{\Delta}{f}\qquad\jhom{\Gamma}{\Delta}{g}}{\jhom{\jequalizer{\Gamma}{f}{g}}{\Gamma}{\jequalizerin{f}{g}}}\\
& \inference{\jhom{\greek{E}}{\Gamma}{h}\qquad\jhomeq{\greek{E}}{\Delta}{\jcomp{\greek{E}}{h}{f}}{\jcomp{\greek{E}}{h}{g}}}{\jhom{\greek{E}}{\jequalizer{\Gamma}{f}{g}}{\jequalizer{h}{f}{g}}}\\
& \inference{
  {\begin{array}{l}
    \jhom{\greek{E}}{\Gamma}{h}\\
    \jhomeq{\greek{E}}{\Delta}{\jcomp{\greek{E}}{h}{f}}{\jcomp{\greek{E}}{h}{g}}
  \end{array}}
  \qquad
  {\begin{array}{l}
    \jhom{\greek{E}}{\jequalizer{\Gamma}{f}{g}}{k}\\
    \jhomeq{\greek{E}}{\Gamma}{\jcomp{\greek{E}}{k}{\jequalizerin{f}{g}}}{h}
  \end{array}}
}
  {\jhom{\greek{E}}{\jequalizer{\Gamma}{f}{g}}{\jequalizer{h}{f}{g}}}
\end{align}

\subsection{Pullback types}

\subsection{Dependent function types}
\begin{align}
& \inference{\jtype{\ctxext{\Gamma}{A}}{P}}{\jtype{\Gamma}{\mprd{A}{P}}}\\
& \inference{\jterm{\ctxext{\Gamma}{A}}{P}{f}}{\jterm{\Gamma}{\mprd{A}{P}}{\lambda(f)}}\\
& \inference{\jterm{\Gamma}{\mprd{A}{P}}{g}}{\jterm{\ctxext{\Gamma}{A}}{P}{\tfev(g)}}\\
& \inference{\jterm{\ctxext{\Gamma}{A}}{P}{f}}{\jtermeq{\Gamma}{A}{\tfev(\lambda(f))}{f}}\\
& \inference{\jtermeq{\ctxext{\Gamma}{A}}{P}{f}{f'}}{\jtermeq{\Gamma}{\mprd{A}{P}}{\lambda(f)}{\lambda(f')}}\\
& \inference{\jtermeq{\Gamma}{\mprd{A}{P}}{g}{g'}}{\jtermeq{\ctxext{\Gamma}{A}}{P}{\tfev(g)}{\tfev(g')}}\\
& \inference{\jtypeeq{\ctxext{\Gamma}{A}}{P}{P'}}{\jtypeeq{\Gamma}{\mprd{A}{P}}{\mprd{A}{P'}}}\\
\end{align}

With these rules we will not get the weak $\eta$-rule when identity types are present.
So it might be better to state that $\lambda$ is a trivial cofibration.

\subsection{Subterminal types}
Let $\jhom{\Gamma}{\Delta}{f}$. Then we may also consider the type $\ctxwk{{\Gamma}{\Gamma}}{{\Gamma}{\Delta}}$
in context $\ctxwk{\Gamma}{\Gamma}$ and we have the terms
\begin{align*}
& \jhom{\ctxext{\Gamma}{\ctxwk{\Gamma}{\Gamma}}}{\ctxwk{\Gamma}{A}}{\ctxwk{\Gamma}{f}}\\
& \jhom{\ctxext{\Gamma}{\ctxwk{\Gamma}{\Gamma}}}{\ctxwk{\Gamma}{A}}{\ctxwk{{\Gamma}{\Gamma}}{f}}
\end{align*}

\subsection{The empty type}

\subsection{Coproduct types}

\subsection{The natural numbers}

\section{A categorical explanation of the variable-free type theory}

The categorical interpretation of type theory is a category $\catfont{C}$ of
contexts, for each object $\Gamma$ of $\catfont{C}$ a class $\catfont{type}(\Gamma)$
of morphisms into $\Gamma$ and for each $A\in\catfont{type}(\Gamma)$ a category
$\catfont{term}(A)$ of sections of $A$. 

\begin{itemize}
\item The context extension $\ctxext{\Gamma}{A}$ is the domain of $A$.
\item There is a choice of $\Gamma\times\Delta$ for each $\Gamma$ and $\Delta$
      for which the first projection $\pi_1:\Gamma\times\Delta\to\Gamma$ is in $\catfont{type}(\Gamma)$;
      this is the weakening $\ctxwk{\Gamma}{\Delta}$.
\item For each $A\in\catfont{type}(\Gamma)$ and $x:A$ there is a choice of
      $\pi_1:\Gamma\times_A P\to\Gamma$ which is in $\catfont{type}(\Gamma)$.
\end{itemize}

\begin{comment}
\section{Binary trees}
We prove a little meta theorem about the type system we have so far. It is not
really of importance to the development of the theory. 

Let's say
that a binary tree of contexts is a pair $\pair{T,f}$ consisting of a binary 
tree $T$ together with a function $f$ assigning to each leaf a context. The set
of all binary trees of contexts is denoted by $B$. Such
binary trees may be defined inductively: $(\unit,f):B$ for any $f:\unit\to ctx$
and given any $(T_1,f_1)$ and $(T_2,f_2)$ in $B$ we have $(T_1,f_1)*(T_2,f_2)$
in $B$.

We simultaneously define
the following two functions:
\begin{align*}
\trext & : B\to ctx\\
\trwk_0 & : \prd{X,Y:B} typ(\trext(X))\\
\trwk_1 & : \prd{X:B}{\Gamma~ctx} typ(\Gamma)\to typ(\ctxext{\trext(X)}{\trwk_0(X,\Gamma)})
\end{align*}
Both these functions are defined by induction on binary trees. We set
\begin{align*}
\trext((\unit,f)) & \defeq f(\unit)\\
\trext((T_1,f_1)*(T_2,f_2)) & \defeq \ctxext{\trext((T_1,f_1))}
{\trwk_0((T_1,f_1),\trext((T_2,f_2)))}
\end{align*}
and for any context $\Gamma$
\begin{align*}
\trwk_0((\unit,f),(\unit,g)) & \defeq \ctxwk{f(\unit)}{g(\unit)}\\
\trwk_0((T_1,f_1)*(T_2,f_2),\Gamma) 
  & \defeq \trwk_1((T_1,f_1),\trwk_0((T_2,f_2)),\Gamma))
\end{align*}
and for any type $A$ in context $\Gamma$
\begin{align*}
\trwk_1((\unit,f),A) & \defeq \ctxwk{f(\unit)}{A}\\
\trwk_1((T_1,f_1)*(T_2,f_2),A) & \defeq
  \trwk_1((T_1,f_1),\trwk_1((T_2,f_2)),A))
\end{align*}

\begin{lem}
$\trwk_0((T,f),\Gamma)$ is a type in context $\trext((T,f))$ and
$\trwk_1((T,f),A)$ is a type in context $\ctxext{\trext((T,f))}{\trwk_0((T,f),\Gamma)}$
for any type $A$ in context $\Gamma$.
\end{lem}

\begin{proof}
It is immediate that $\trwk_0((\unit,f),\Gamma)$ is a type in context $\trext((\unit,f))$
and that $\trwk_1((\unit,f),A)$ is a type in context $\ctxext{\trext((\unit,f))}
{\trwk_0((\unit,f),\Gamma)}$ for any type $A$ in context $\Gamma$.

Suppose that $(T_1,f_1)$ and $(T_2,f_2)$ are binary trees of contexts such that
$\trwk_0((T_i,f_i),\Gamma)$ is a type in context
$\trext((T_i,f_i))$ for $i\jdeq 1$ and $i\jdeq 2$, and such that
$\trwk_1((T_i,f_1),A)$ is a type in context $\ctxext{\trext((T_i,f_i))}{\trwk_0((T_i,f_i),\Gamma)}$
for $i\jdeq 1$ and $i\jdeq 2$. Then
\begin{equation*}
\trwk_0((T_1,f_1)*(T_2,f_2),\Gamma) 
\jdeq \trwk_1((T_1,f_1),\trwk_0((T_2,f_2),\Gamma))
\end{equation*}
is a type in context
\begin{equation*}
\ctxext{\trext((T_1,f_1))}{\trwk_0((T_1,f_1),\trext((T_2,f_2)))}
\jdeq \trext((T_1,f_1)*(T_2,f_2)).
\end{equation*}
Also,
\begin{equation*}
\trwk_1((T_1,f_1)*(T_2,f_2),A) \jdeq \trwk_1((T_1,f_1),\trwk_1((T_2,f_2),A))
\end{equation*}
is a type in context
\begin{equation*}
\ctxext{\trext((T_1,f_1))}{\trwk_0((T_1,f_1),\ctxext{\trext((T_2,f_2))}{\trwk_0((T_2,f_2),\Gamma))}}
\end{equation*}
\end{proof}
\end{comment}

