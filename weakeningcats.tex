\section{E-systems}\label{sec:esys_defn}

\subsection{Weakening systems}

\begin{defn}
A \emph{pre-weakening structure on a pre-category $\cat{F}$} consists of a functor
$W_A:\cat{F}/\Gamma\to \cat{F}/\ctxext{\Gamma}{A}$ for every $A:\ctxext{\Gamma}{A}\to \Gamma$ in $\cat{F}$ such 
that
\begin{enumerate}
\item $W_{\catid{\Gamma}}\jdeq \catid{\cat{F}/\Gamma}$ for every object $\Gamma\in\cat{F}$.
\item $W_{\ctxext{A}{P}}\jdeq W_P\circ W_A$ for every $P\in\cat{F}/\ctxext{\Gamma}{A}$ and $A\in\cat{F}/\Gamma$.
\item $W_A$ preserves the final object, i.e.~$W_A(\catid{\Gamma})\jdeq \catid{\ctxext{\Gamma}{A}}$.
\end{enumerate}
A \emph{pre-weakening system} is a pre-category together with a pre-weakening structure.
\end{defn}

\begin{defn}
A \emph{pre-weakening homomorphism} $F:\cat{F}\to\cat{D}$ between pre-weakening systems
is a functor $F:\cat{F}\to\cat{D}$ such that the square
\begin{equation*}
\begin{tikzcd}[column sep=large]
\cat{F}/\ctxext{\Gamma}{A}
  \arrow[r,"F/\ctxext{\Gamma}{A}"]
  &
\cat{D}/F(\ctxext{\Gamma}{A})
  \\
\cat{F}/\Gamma
  \arrow[u,"W_A"]
  \arrow[r,swap,"F/\Gamma"]
  &
\cat{D}/F(\Gamma)
  \arrow[u,swap,"W_{F(A)}"]
\end{tikzcd}
\end{equation*}
commutes for any $A\in\cat{F}/\Gamma$.
\end{defn}

\begin{lem}
Let $\cat{F}$ be a pre-weakening system. Then for any object $\Gamma$ of $\cat{F}$,
the slice pre-category $\cat{F}/\Gamma$ is also a pre-weakening system, with the weakening
structure given by $W(\cat{F}/\Gamma)_P\jdeq W(\cat{F})_P$ for each $P\in\cat{F}/\ctxext{\Gamma}{A}$
and each $A\in\cat{F}/\Gamma$.
\end{lem}

\begin{defn}
A \emph{weakening system} is a pre-weakening system for which every $W_f$ is a
pre-weakening homomorphism. A \emph{weakening homomorphism} is a pre-weakening 
homomorphism between weakening systems.
\end{defn}

\begin{rmk}
In other words, a weakening system is a pre-weakening system with a weakening
structure $W$ for which the square
\begin{equation*}
\begin{tikzcd}[column sep=huge]
\cat{F}/\ctxext{{\Gamma}{B}}{Q}
  \arrow[r,"W_A/\ctxext{B}{Q}"]
  &
\cat{F}/\ctxext{{\Gamma}{A}}{W_A(\ctxext{B}{Q})}
  \\
\cat{F}/\ctxext{\Gamma}{B}
  \arrow[u,"W_Q"]
  \arrow[r,swap,"W_A/B"]
  &
\cat{F}/\ctxext{{\Gamma}{A}}{W_A(B)}
  \arrow[u,swap,"W_{W_A(Q)}"]
\end{tikzcd}
\end{equation*}
for each $A,B\in\cat{F}/\Gamma$ and $Q:\cat{F}/\ctxext{\Gamma}{B}$. The objects part of this 
property asserts that for any $k\in \cat{F}/\greek{E}$, the dotted arrows in the diagram
\begin{equation*}
\begin{tikzcd}[column sep=tiny]
\bullet \arrow[dr,densely dotted,swap,near start,"(W_A/\ctxext{B}{Q})(W_Q(R))"] & & \bullet \arrow[dl,densely dotted,near start,"W_{(W_A/B)(Q)}((W_A/B)(R))"] & & & & & & & & & & \bullet \arrow[dl,"W_Q(R)"] \\
& \bullet \arrow[dr,swap,"(W_A/B)(Q)"] & & \bullet \arrow[dl,"(W_A/B)(R)"] & & & & & & & & \bullet \arrow[dr,swap,"Q"] & & \bullet \arrow[dl,"R"] \\
& & \bullet \arrow[d,swap,"W_A(B)"] & & & & & & & & & & \ctxext{\Gamma}{B} \arrow[d,"B"] \\
& & \ctxext{\Gamma}{A} \arrow[rrrrrrrrrr,swap,"A"] & & & & & & & & & & \Gamma
\end{tikzcd}
\end{equation*}
are equal.

A useful special case of this property is where $B\jdeq \catid{\Gamma}$. Thus, if $W$ is
a weakening system, then the diagram
\begin{equation*}
\begin{tikzcd}[column sep=huge]
\cat{F}/\ctxext{\Gamma}{C}
  \arrow[r,"W_A/C"]
  &
\cat{F}/\ctxext{{\Gamma}{A}}{W_A(C)}
  \\
\cat{F}/\Gamma
  \arrow[u,"W_C"]
  \arrow[r,swap,"W_A"]
  &
\cat{F}/\ctxext{\Gamma}{A}
  \arrow[u,swap,"W_{W_A(C)}"]
\end{tikzcd}
\end{equation*}
commutes for every $A,C\in\cat{F}/\Gamma$. In particular, we see
that $W_A(W_C(D))\jdeq W_{W_A(C)}(W_A(D))$ for any $D\in\cat{F}/\Gamma$,
i.e.~that weakening is a self-distributive operation. 
\end{rmk}

\begin{cor}
For any object $\Gamma$ of a weakening system $\cat{F}$, the slice pre-category
$\cat{F}/\Gamma$ is a weakening system.
\end{cor}

\begin{eg}
If the underlying pre-category is a group, the axioms of a weakening system
obtain the following form: There is a group homomorphism $W_x:G\to G$ for
every $x\in G$, in such a way that $W_e\jdeq \catid{G}$ and $W_{xy}\jdeq W_y\circ W_x$
for every $x,y\in G$, and for any $x,y\in G$ the diagram
\begin{equation*}
\begin{tikzcd}
G \arrow[r,"W_y"]
  &
G 
  \\
G \arrow[u,"W_x"]
  \arrow[r,swap,"W_y"]
  &
G \arrow[u,swap,"W_{W_y(x)}"]
\end{tikzcd}
\end{equation*}
is required to commute. Thus, a weakening system on a group $G$ consists of
a group homomorphism $W:\op G\to\mathrm{Aut}(G)$ such that $W_a\circ W_b\jdeq 
W_{W_a(b)}\circ W_a$. The latter requirement is satisfied if $b\cdot a\jdeq a\cdot W_a(b)$
for every $a,b\in G$, or equivalently if $[b,a]\jdeq b^{-1}\cdot W_a(b)$.
\end{eg}

\begin{lem}
For any group, conjugation is a weakening structure. Thus, any group is
canonically a weakening system.
\end{lem}

\begin{proof}
Note that $e^{-1}xe\jdeq x$ and $(xy)^{-1}z(xy)\jdeq y^{-1}(x^{-1}zx)y$ for any $x,y,z\in G$.
Also, the diagram
\begin{equation*}
\begin{tikzcd}
G \arrow[r,"\mathrm{conj}_y"]
  &
G 
  \\
G \arrow[u,"\mathrm{conj}_x"]
  \arrow[r,swap,"\mathrm{conj}_y"]
  &
G \arrow[u,swap,"\mathrm{conj}_{y^{-1}xy}"]
\end{tikzcd}
\end{equation*}
commutes.
\end{proof}

\begin{comment}
\begin{rmk}
Similarly, every groupoid has a canonical weakening structure given by
conjugation.
\end{rmk}
\end{comment}

\begin{comment}
\begin{lem}
Let $G$ be a graph, and let $\cat{F}_G$ be the free pre-category generated by
$G$. A pre-weakening structure on $\cat{F}_G$ is determined by functors
$W_e:\cat{F}_G/\Gamma\to\cat{F}_G/\Delta$ for every edge $e:\Delta\to \Gamma$ in $G$, with the
property that $W_e(\catid{\Gamma})\jdeq \catid{\Delta}$. 

A weakening structure on $\cat{F}_G$ is
a pre-weakening structure on $\cat{F}_G$ such that the diagram
\begin{equation*}
\begin{tikzcd}[column sep=huge]
\cat{F}_G/\greek{Z} \arrow[r,"W_e/(g\circ h)"] & \cat{F}_G/\mathrm{dom}(W_e(g\circ h)) \\
\cat{F}_G/\greek{E} \arrow[u,"W_h"] \arrow[r,swap,"W_e/g"] & \cat{F}_G/\mathrm{dom}(W_e(g)) \arrow[u,swap,"W_{W_e(h)}"]
\end{tikzcd}
\end{equation*}
commutes for every arrow $g:\greek{E}\to \Gamma$ in $\cat{F}_G$ and every two edges $e:\Delta\to \Gamma$ and
$h:\greek{Z}\to \greek{E}$ in $G$.
\end{lem}

\begin{eg}
Let $G$ be the graph consisting of one vertex $\ast$ and one edge $e:\ast\to\ast$.
Then $\cat{F}_G/\ast\jdeq \N$. Then the only weakening structure on $\cat{F}/G$ is
given by $W_e\jdeq \catid{\N}$. The reason is that 
\end{eg}
\end{comment}

\begin{lem}
Let $P$ be a pre-order. A pre-weakening structure on $P$ is determined by order
preserving maps $W_{a,b}:{\downarrow}(b)\to{\downarrow}(a)$ satisfying
$W_{a,b}(b)\jdeq a$, for every $a\leq b$, where
${\downarrow}(x)$ is the set $\{y\in P\mid y\leq x\}$.

A weakening structure on $P$ is a pre-weakening structure on $P$ such that the
diagram
\begin{equation*}
\begin{tikzcd}[column sep=huge]
{\downarrow}(c') \arrow[r,"W_{a,b}|_{{\downarrow}(c')}"] & {\downarrow}(W_{a,b}(c')) \\
{\downarrow}(c) \arrow[u,"W_{c',c}"] \arrow[r,swap,"W_{a,b}|_{{\downarrow}(c)}"] & {\downarrow}(W_{a,b}(c)) \arrow[u,swap,"W_{W_{a,b}(c'),W_{a,b}(c)}"]
\end{tikzcd}
\end{equation*}
for any $a\leq b$ and $c'\leq c\leq b$.
\end{lem}

\begin{eg}
In a poset with meets, the functions $W_{a,b}:{\downarrow}(b)\to{\downarrow}(a)$
given by $W_{a,b}(x)\defeq a\land x$ assemble a weakening structure, because for
any $a\leq b$ and $x,c'\leq c\leq b$ we have $a\land(c'\land x)\jdeq (a\land c')\land
(a\land x)$.  
\end{eg}

\begin{eg}
Consider the poset $(\N,\geq)$, which has a morphism from $a$ to $b$ if $a\geq b$.
On this poset, we may consider the weakening structure defined by $W_{a,b}(x)\jdeq
a+(x-b)$.
\end{eg}

\subsection{Weakening systems with term structure}
\begin{defn}
A \emph{category with a term structure} is a pre-category $\cat{F}$ with a set $T(A)$ for every
$A\in\cat{F}/\Gamma$ for every $\Gamma\in\cat{F}$. A \emph{functor with term structure} from $\cat{F}$ to $\cat{D}$
is a functor $F:\cat{F}\to\cat{D}$ with a function $T(A)\to T(F(A))$ for
every $A\in\cat{F}/\Gamma$, for every $\Gamma\in\cat{F}$.
\end{defn}

\begin{defn}
Suppose $\cat{F}$ is a pre-category with term structure, and $\Gamma$ is an object of $\cat{F}$. 
Then $\cat{F}/\Gamma$ is a pre-category with term structure, where $T_{\cat{F}/\Gamma}(A)\jdeq T_\cat{F}(A)$.
\end{defn}

\begin{defn}
A \emph{weakening system with term term structure} is the same thing as a weakening system,
except that the underlying pre-category and the functors $W_A$ are required to
be functors with term term structure. The equalities are also required to hold on the term structure.
\end{defn}

\begin{cor}
For any object $\Gamma$ of a weakening system with term structure, the slice pre-category
$\cat{F}/\Gamma$ is a weakening system with term structure.
\end{cor}

\begin{eg}
%For any group $G$ with weakening structure $W:\op{G}\to\mathrm{Aut}(G)$, a term 
%structure is determined by a presheaf on $G/\ker(W)$.

Let $G$ be a group, with the weakening structure given by conjugation. A term
structure on $G$ is given by a set $X$ with a right $G$-action $\mu:\op G\times X\to X$.
We take $T(a)\defeq X$ for every $a\in G$.
\end{eg}

\subsection{Projection systems}
\begin{defn}
A \emph{pre-projection system} is a pre-weakening system with term structure with an element
$\tfid{A}\in T(W_A(A))$ for every $A\in\cat{F}/\Gamma$ and $\Gamma\in\cat{F}$. 
\end{defn}

\begin{defn}
A \emph{pre-projection homomorphism} $F:\cat{F}\to\cat{D}$ is a pre-weakening homomorphism for which
\begin{equation*}
F(\tfid{A})\jdeq \tfid{F(A)}
\end{equation*}
for every $A\in\cat{F}/\Gamma$ and $\Gamma\in\cat{F}$.
\end{defn}

\begin{lem}
For any object $\Gamma$ in a pre-projection system $\cat{F}$, the slice pre-category
$\cat{F}/\Gamma$ is a pre-projection system.
\end{lem}

\begin{proof}
Straightforward.
\end{proof}

\begin{defn}
A \emph{projection system} is a pre-projection system for which every $W_A$ is
a pre-projection homomorphism. A \emph{projection homomorphism} is a pre-projection homomorphism
between projection systems.
\end{defn}

\begin{cor}
For any object $\Gamma$ of a projection system $\cat{F}$, the slice pre-category $\cat{F}/\Gamma$
is a projection system.
\end{cor}

%\begin{eg}
%Consider a group $G$ with a right $G$-set $X$. A projection structure on $(G,X)$
%consists of an element $\tfid{a}\in X$ for every $a\in G$, so that $\tfid{a}\cdot b=
%\tfid{b^{-1}ab}$ for every $a,b\in G$. 
%For instance, this is the case if we take $X$ to be a pointed set with $\ast\in X$,
%and take $\tfid{a}:=\ast$ for all $a\in G$.
%\end{eg}

\subsection{Substitution systems}

\begin{defn}
A \emph{pre-substitution structure} on a pre-category with term structure $\cat{F}$ consists
of a functor with term structure $S_x:\cat{F}/\ctxext{\Gamma}{A}\to\cat{F}/\Gamma$ for every $x\in T(A)$ and
$A\in\cat{F}/\Gamma$, such that $S_x(\catid{\ctxext{\Gamma}{A}})\jdeq\catid{\Gamma}$.

A \emph{pre-substitution system} is a pre-category with term structure together with a
pre-substitution structure. 
\end{defn}

\begin{defn}
A \emph{pre-substitution homomorphism} $F:\cat{F}\to\cat{D}$ is a functor with term structure for
which the diagram
\begin{equation*}
\begin{tikzcd}[column sep=large]
\cat{F}/\ctxext{\Gamma}{A}
  \arrow[r,"F/\ctxext{\Gamma}{A}"]
  \arrow[d,swap,"S_x"]
  &
\cat{D}/F(\ctxext{\Gamma}{A})
  \arrow[d,"S_{F(x)}"]
  \\
\cat{F}/\Gamma
  \arrow[r,swap,"F/\Gamma"]
  &
\cat{D}/F(\Gamma)
\end{tikzcd}
\end{equation*}
commutes for every $x\in T(A)$ and $A\in\cat{F}/\Gamma$.
\end{defn}

\begin{lem}
For any object $\Gamma$ of a pre-substitution system $\cat{F}$, the slice pre-category
$\cat{F}/\Gamma	$ is a pre-substitution system.
\end{lem}

\begin{defn}
A \emph{substitution system} is a pre-substitution system for which each
$S_t$ is a pre-substitution homomorphism. A \emph{substitution homomorphism} is a
pre-substitution homomorphism between substitution systems.
\end{defn}

\begin{cor}
For any object $\Gamma$ of a substitution system, the slice pre-category $\cat{F}/\Gamma$
is a substitution system.
\end{cor}

\begin{rmk}
The condition that every $S_x$ is a substitution homomorphism, asserts that
the diagram
\begin{equation*}
\begin{tikzcd}[column sep=large]
\cat{F}/\ctxext{{{\Gamma}{A}}{P}}{Q}
  \arrow[r,"S_x/\ctxext{P}{Q}"]
  \arrow[d,swap,"S_y"]
  &
\cat{F}/\ctxext{{\Gamma}{S_x(P)}}{S_x(Q)}
  \arrow[d,"S_{S_x(y)}"]
  \\
\cat{F}/\ctxext{{\Gamma}{A}}{P}
  \arrow[r,swap,"S_x/P"]
  &
\cat{F}/\ctxext{\Gamma}{S_x(P)}
\end{tikzcd}
\end{equation*}
commutes for every $y\in T(Q)$.
\end{rmk}

\begin{eg}
Let $G$ be a group and let $H$ be a set, forming a category with term 
structure of which $T(a)\jdeq H$ for every $a\in G$. A substitution
system on $(G,H)$ consists of a function $\sigma_G : H\times G\to G$ and
a function $\sigma_H:H\times H\to H$, for which the diagrams
\begin{equation*}
\begin{tikzcd}[column sep=large]
G \arrow[r,"\sigma_G(h)"]
  \arrow[d,swap,"\sigma_G(k)"]
  &
G \arrow[d,"{\sigma_G(\sigma_H(h,k))}"]
  \\
G \arrow[r,swap,"\sigma_G(h)"] 
& G
\end{tikzcd}
\end{equation*}
and
\begin{equation*}
\begin{tikzcd}[column sep=large]
H \arrow[r,"\sigma_H(h)"]
  \arrow[d,swap,"\sigma_H(k)"]
  &
H \arrow[d,"{\sigma_H(\sigma_H(h,k))}"]
  \\
H \arrow[r,swap,"\sigma_H(h)"]
  &
H
\end{tikzcd}
\end{equation*}
commute for every $h,k\in H$. 
\end{eg}

\subsection{The definition of E-systems}

Although it is now possible to state the definition of an E-system without
any difficulty, it shall be convenient later to also have a category of
pre-E-systems at our disposal. Pre-E-systems shall by default be non-unital,
i.e.~they do not come with a pre-projection system, while E-systems shall
always be assumed to be unital. In the present context, we shall not need
unital pre-E-systems or non-unital E-systems.

\begin{defn}
A \emph{pre-E-system} is a pre-category $\cat{F}$ with term structure
with a pre-weakening structure $W$, a pre-substitution structure $S$ and a terminal
object. A \emph{unital pre-E-system}, or simply \emph{pre-E-system}, is a non-unital pre-E-system
which also has the structure of a pre-projection system. 
\end{defn}

\begin{defn}
A \emph{pre-E-homomorphism} from $\mathbb{E}\to\mathbb{D}$ is a functor
$H:\mathbb{C}_\mathbb{E}\to\mathbb{C}_\mathbb{D}$ between the underlying categories
with term structure, which is a pre-weakening homomorphism, a pre-substitution
homomorphism, and which preserves the terminal object. 

A \emph{unital} pre-E-homomorphism is a non-unital pre-E-homomorphism which is 
also a pre-projection homomorphism. 
\end{defn}

\begin{defn}
\label{defn:esystems}
An \emph{E-system} is a unital pre-E-system in which 
\begin{enumerate}
\item each $W_A$ is a unital pre-E-homomorphism,
\item each $S_x$ is a unital pre-E-homomorphism,
\item \label{tTf:StWf_id} weakened families are constant, i.e.~$S_x\circ W_A\jdeq \catid{\cat{F}/\Gamma}$ for any $x\in T(A)$ and $A\in\cat{F}/\Gamma$,
\item $S_x(\tfid{A})\jdeq x$ for any $x\in T(A)$ and $A\in\cat{F}/\Gamma$, and
\item \label{f:SidfWff_id} $S_{\tfid{A}}\circ W_A/A\jdeq \catid{\cat{F}/\ctxext{\Gamma}{A}}$ for any $A\in\cat{F}/\Gamma$.
\end{enumerate}
An \emph{E-homomorphism} $H:\mathbb{E}\to\mathbb{D}$ is a functor from
$\cat{F}_{\mathbb{E}}$ to $\cat{F}_{\mathbb{D}}$ which is both a projection
homomorphism and a substitution homomorphism.
\end{defn}

\begin{rmk}
The condition that each $W_A$ is a substitution homomorphism asserts that
the diagram
\begin{equation*}
\begin{tikzcd}[column sep=huge]
\cat{F}/\ctxext{{\Gamma}{B}}{Q}
  \arrow[r,"W_A/\ctxext{B}{Q}"]
  \arrow[d,swap,"S_y"]
  &
\cat{F}/\ctxext{{{\Gamma}{A}}{W_A(B)}}{W_A(Q)}
  \arrow[d,"S_{W_A(y)}"]
  \\
\cat{F}/\ctxext{\Gamma}{B}
  \arrow[r,swap,"W_A/B"]
  &
\cat{F}/\ctxext{{\Gamma}{A}}{W_A(B)}
\end{tikzcd}
\end{equation*}
of functors with term structure commutes for every 
$Q\in\cat{F}/\ctxext{\Gamma}{B}$, $B\in\cat{F}/\Gamma$ and each $y\in T(Q)$.

Likewise, the condition that each $S_x$ is a weakening homomorphism
asserts that the diagram
\begin{equation*}
\begin{tikzcd}[column sep=huge]
\cat{F}/\ctxext{{\Gamma}{A}}{P}
  \arrow[r,"S_x/P"]
  \arrow[d,swap,"W_Q"]
  &
\cat{F}/\ctxext{\Gamma}{S_x(P)}
  \arrow[d,"W_{S_x(Q)}"]
  \\
\cat{F}/\ctxext{{{\Gamma}{A}}{P}}{Q}
  \arrow[r,swap,"S_x/\ctxext{P}{Q}"]
  &
\cat{F}/\ctxext{{\Gamma}{S_x(P)}}{S_x(Q)}
\end{tikzcd}
\end{equation*}
of functors with term structure commutes for every 
$Q\in\cat{F}/\ctxext{{\Gamma}{A}}{P}$.
\end{rmk}

\begin{cor}
For any object $\Gamma$ of an E-system $\cat{F}$, the slice pre-category $\cat{F}/\Gamma$
is an E-system.
\end{cor}

\begin{comment}
\begin{rmk}[Needs more precision]
Consider the category $E$ generated by the graph consisting of the objects of 
the underlying category of an $E$-system $\mathcal{E}$ as edges, and the morphisms
$S_t:X\to Y$ and $W_A:Y\to X$ as edges, subject to the axioms of an $E$-systems.
 
Any morphism of $E$ can be rewritten in a unique way
as a composition $\tilde S\circ \tilde W$, where $\tilde S$ is a morphism in
substitution-normal form, $\tilde W$ is in weakening normal form, every term
expression is in normal form, and where
each occurence of $S_x\circ W_A$ and $S_{\tfid{A}}\circ W_A/A$ is eliminated.
\end{rmk}
\end{comment}

\section{The category of internal morphisms of an E-system}\label{sec:esys_props}

The theory of E-systems could be seen as a sort of category theory with only
projections and their sections. We elaborate on this point of view by defining
the category of internal morphisms, which Lawvere might have called the
category of terms. 
In the category of internal morphisms over a context $\Gamma$,
which we define in \autoref{subsec:im_cat},
a morphism $f:A\to B$ is a term of the constant family $W_A(B)$ over 
$\ctxext{\Gamma}{A}$. Rather than directly defining composition, we will define
for every $f:A\to B$ in context $\Gamma$, a pre-composition operation
$f_\ast$, which is going to be an E-homomorphism from 
$\cat{F}/\ctxext{\Gamma}{B}$ to $\cat{F}/\ctxext{\Gamma}{A}$. This allows us
to consider diagrams of the following form:
\begin{equation*}
\begin{tikzcd}
P \arrow[d,->>] \arrow[r,"g"] & Q \arrow[d,->>] \\
A \arrow[r,swap,"f"] & B
\end{tikzcd}
\end{equation*}
where we consider the morphism $g$ over $f$ to be an internal morphism from 
$P$ to $f_\ast(Q)$ in context $\ctxext{\Gamma}{A}$, capturing the
essence of being a morphism from the family $P$ over $A$ to the family
$Q$ over $B$, reindexed by $f$. 
Whenever we draw such diagrams, a double headed arrow will always
refer to a family, and an ordinary arrow will refer to an internal morphism,
possibly over another internal morphism.

A morphism $g$ over $f$ can be vertically composed with $f$ so
that we obtain a morphism $\jvcomp{P}{f}{g}:\ctxext{A}{P}\to\ctxext{B}{Q}$ in 
context $\Gamma$, and it can be horizontally composed with a morphism $k:Q\to R$
over $h:B\to C$ so that we obtain a morphism $\jfcomp{A}{f}{P}{g}{k}:P\to R$ 
over $\jcomp{A}{f}{h}:A\to C$. 
In \autoref{sec:interchange}, we will show interchange laws between these
sorts of composition. The interchange laws have as a consequence that if we
assert the commutativity of a large composed diagram.

The category of internal morphisms forms a category of which the families form
a subcategory. This, together with the pre-composition operation described
earlier, gives rise to the auxiliary notion of E'-systems, which we define
in \autoref{sec:E'}. The thread running through this whole section is proving
that the notions of E-system and E'-system are the same. In \autoref{sec:E'}
we show directly that any E'-system is an E-system. We end this section by
showing that the converse also holds.

Before we begin this section, we introduce some more convenient notation.

\begin{defn}
Let $A\in\cat{F}/\Gamma$. Recall that $W_A:\cat{F}/\Gamma\to\cat{F}/\ctxext{\Gamma}{A}$ acts on objects,
morphisms and terms. We introduce the infix form of weakening by $A\in\cat{F}/\Gamma$ to be
$\ctxwk{A}{\blank}$. Thus, we will write
\begin{align*}
\ctxwk{A}{B} & \defeq W_A(B) & & \text{for $B\in\cat{F}/\Gamma$} \\
\ctxwk{A}{Q} & \defeq W_A(Q) & & \text{for $B\in\cat{F}/\Gamma$ and $Q\in\cat{F}/\ctxext{\Gamma}{B}$}\\
\ctxwk{A}{g} & \defeq W_A(g) & & \text{for $B\in\cat{F}/\Gamma$, $Q\in\cat{F}/\ctxext{\Gamma}{B}$ and $g\in T(Q)$}
\end{align*}
\end{defn}

\begin{defn}
Let $x\in T(A)$ for a family $A\in\cat{F}/\Gamma$. The infix form of substitution
by $x$ is taken to be $\subst{x}{\blank}$. Thus, we will write
\begin{align*}
\subst{x}{P} & \defeq S_x(P) & & \text{for $P\in\cat{F}/\ctxext{\Gamma}{A}$} \\
\subst{x}{Q} & \defeq S_x(Q) & & \text{for $P\in\cat{F}/\ctxext{\Gamma}{A}$ and $Q\in\cat{F}/\ctxext{{\Gamma}{A}}{P}$} \\
\subst{x}{g} & \defeq S_x(g) & & \text{for $P\in\cat{F}/\ctxext{\Gamma}{A}$, $Q\in\cat{F}/\ctxext{{\Gamma}{A}}{P}$ and $g\in T(Q)$}
\end{align*}
\end{defn}

\subsection{E'-systems}\label{sec:E'}

E'-systems can be thought of as strict versions of tribes%
\footnote{Tribes are an invention of Joyal, see \url{http://ncatlab.org/homotopytypetheory/files/Joyal.pdf}}.

\begin{defn}\label{defn:E'sys}
An \emph{E'-system} consists of a category $\cat{C}$ and a subcategory $\cat{F}$ with
the same objects as $\cat{C}$, such that
\begin{enumerate}
\item $\cat{F}$ has a terminal object.
\item For any $f:\Delta\to\Gamma$ in $\cat{C}$ and any $A\in\cat{F}/\Gamma$, a
functorial choice of a pullback square 
\begin{equation*}
\begin{tikzcd}[column sep=large]
\ctxext{\Delta}{f_\ast(A)} \arrow[r,"{\pi_2(f,A)}"] \arrow[d,fib,swap,"{f_\ast(A)}"] & \ctxext{\Gamma}{A} \arrow[d,fib,"A"] \\
\Delta \arrow[r,"f"] & \Gamma
\end{tikzcd}
\end{equation*}
such that $f_\ast(A)\in\cat{F}/\Delta$. Explicitly, the functoriality
requirement is that
\begin{enumerate}
\item For any $f:\Delta\to\Gamma$, one has 
$f_\ast(\catid{\Gamma})\jdeq\catid{\Delta}$ and
\begin{equation*}
\pi_2(f,\catid{\Gamma})\jdeq f.
\end{equation*}
\item \label{defn:E'sys:b} For any $A\in\cat{F}/\Gamma$, one has $(\catid{\Gamma})_\ast(A)\jdeq A$ and
\begin{equation*}
\pi_2(\catid{\Gamma},A)\jdeq\catid{\ctxext{\Gamma}{A}}
\end{equation*}
\item \label{defn:E'sys:c} For any $f:\Delta\to\Gamma$, $g:X\to\Delta$ and $A\in\cat{F}/\Gamma$, one
has $(f\circ g)_\ast(A)\jdeq g_\ast(f_\ast(A))$ and 
\begin{equation*}
\pi_2(f\circ g,A)\jdeq \pi_2(f,A)\circ\pi_2(g,f_\ast(A))
\end{equation*}
\item For any $P\in\cat{F}/\ctxext{\Gamma}{A}$ and $f:\Delta\to\Gamma$, one has
$f_\ast(P)\jdeq\pi_2(f,A)_\ast(P)$ and 
\begin{equation*}
\pi_2(f,\ctxext{A}{P}) \jdeq \pi_2(\pi_2(f,a),P)
\end{equation*}
\end{enumerate}
\end{enumerate}
For any $f:\Delta\to\Gamma$ we shall write $f_\ast$ for the induced functor
$\cat{F}/\Gamma\to\cat{F}/\Delta$. 
\end{defn}

\begin{rmk}
Without a terminal object in $\cat{F}$, the category $\cat{C}$ does not correspond
directly to anything in an E-system.
\end{rmk}

\begin{rmk}
For E'-systems, we do not require that every pullback of $A\in\cat{F}/\Gamma$ along $f:\Delta\to\Gamma$
is in $\cat{F}/\Delta$, just that there is a choice of such a pullback which is
in $\cat{F}/\Delta$. Thus, the definition of E'-system is slightly different from
Taylor's notion of display map category. Furthermore, the class of display maps in
a display map category is not required to be closed under composition, and neither
is there assumed to exist an object $T$ such that for every object $X$ there is a 
unique display map $X\to T$.
\end{rmk}

\begin{rmk}
Also, we do not require that $\cat{C}$ has a terminal object, and neither do
we require that $\cat{F}$ contains all the isomorphisms of $\cat{C}$. These are
essential differences with Joyal's notion of tribes.   
\end{rmk}

\begin{defn}
For any $\Gamma\in\cat{C}$, define $\cat{C}_\cat{F}/\Gamma$ to be the full
subcategory of $\cat{C}/\Gamma$ with the same objects as $\cat{F}/\Gamma$. 
\end{defn}

\begin{lem}\label{lem:pb_selfdistributive}
Let $\mathbb{E}$ be an E'-system. Then $f_\ast$ can be extended to a functor from
$\cat{C}_\cat{F}/\Gamma$ to $\cat{C}_\cat{F}/\Delta$, for each $f:\Delta\to\Gamma$, and moreover the diagram
\begin{equation*}
\begin{tikzcd}[column sep=large]
\cat{C}_\cat{F}/\ctxext{\Gamma}{B} \arrow[r,"{f_\ast/B}"] \arrow[d,swap,"{g_\ast}"]
& \cat{C}_\cat{F}/\ctxext{\Delta}{f_\ast(B)} \arrow[d,"{(f_\ast(g))_\ast}"] \\
\cat{C}_\cat{F}/\ctxext{\Gamma}{A} \arrow[r,swap,"{f_\ast/A}"]
& \cat{C}_\cat{F}/\ctxext{\Delta}{f_\ast(A)}
\end{tikzcd}
\end{equation*}
commutes for every $g:A\to B$ in $\cat{C}_\cat{F}/\Gamma$.
\end{lem}

\begin{proof}
It is straightforward to use the universal property of the pullback to extend
$f_\ast$ to a functor from $\cat{C}_\cat{F}/\Gamma$ to $\cat{C}_\cat{F}/\Delta$.
To see that the diagram commutes, we calculate
\begin{align*}
f_\ast/A\circ g_\ast
& \jdeq
\pi_2(f,A)_\ast\circ g_\ast \\
& \jdeq
(g\circ\pi_2(f,A))_\ast \\
& \jdeq
(\pi_2(f,B)\circ f_\ast(g))_\ast \\
& \jdeq
(f_\ast(g))_\ast\circ\pi_2(f,B)_\ast \\
& \jdeq
(f_\ast(g))_\ast\circ f_\ast/B.\qedhere
\end{align*}
\end{proof}

\begin{thm}
An E-system is the same thing as an E'-system.
\end{thm}

\begin{proof}
We define for any $A\in\cat{F}/\Gamma$ the set
\begin{equation*}
T(A)\defeq\{x:\Gamma\to\ctxext{\Gamma}{A}\mid A\circ x\jdeq\catid{\Gamma}\}.
\end{equation*}
Since each functor $f_\ast$ is extended to all of $\cat{C}_{\cat{F}}/\Gamma$,
it follows immediately that those are functors with term structure.

We define for any $A\in\cat{F}/\Gamma$, the functor $W_A\defeq A_\ast:\cat{F}/\Gamma\to\cat{F}/\ctxext{\Gamma}{A}$.
Likewise, we define for any $x\in T(A)$, the functor $S_x\defeq x_\ast:\cat{F}/\ctxext{\Gamma}{A}\to\cat{F}/\Gamma$.
We also define $\tfid{A}:T(W_A(A))$ by the universal property of pullbacks:
\begin{equation*}
\begin{tikzcd}[column sep=large]
\ctxext{\Gamma}{A} \arrow[drr,bend left=15,equals] \arrow[ddr,bend right=15,equals] \arrow[dr,densely dotted,near end,"{\tfid{A}}"] \\
& \ctxext{{\Gamma}{A}}{W_A(A)} \arrow[r,swap,"{\pi_2(A,A)}"] \arrow[d,fib] & \ctxext{\Gamma}{A} \arrow[d,fib] \\
& \ctxext{\Gamma}{A} \arrow[r,swap,"A"] & \Gamma
\end{tikzcd}
\end{equation*}
As an immediate consequence of \autoref{lem:pb_selfdistributive}, we get that
each functor $W_A$ and $S_x$ is both a weakening functor and a substitution functor.
Since each $W_A$ and $S_x$ preserves identity morphisms, it also preserves the
terms $\tfid{A}$. 

It remains to verify the rest of the conditions of E-systems.
\begin{itemize}
\item To show that weakened families are constant, note that
\begin{equation*}
S_x\circ W_A\jdeq x_\ast\circ A_\ast\jdeq (A\circ x)_\ast\jdeq (\catid{\Gamma})_\ast\jdeq \catid{\cat{C}_{\cat{F}}/\Gamma}.
\end{equation*}
\item The identity terms are neutral for pre-composition:
\begin{equation*}
S_{\tfid{A}}\circ W_A/A \jdeq S_{\tfid{A}}\circ \pi_2(A,A)_\ast \jdeq (\pi_2(A,A)\circ\tfid{A})_\ast \jdeq
(\catid{\ctxext{\Gamma}{A}})_\ast \jdeq \catid{\cat{C}_{\cat{F}}/\ctxext{\Gamma}{A}} 
\end{equation*}
\item The identity terms behave like identity functions: by the universal property,
$S_x(\tfid{A})$ is the unique section of $A$ such that the square
\begin{equation*}
\begin{tikzcd}[column sep=6em]
\Gamma \arrow[d,swap,"{S_x(\tfid{A})}"] \arrow[r,"{\pi_2(x,\catid{\ctxext{\Gamma}{A}})}"] & \ctxext{\Gamma}{A} \arrow[d,"{\tfid{A}}"] \\
\ctxext{\Gamma}{A} \arrow[r,swap,"{\pi_2(x,W_A(A))}"] & \ctxext{{\Gamma}{A}}{W_A(A)}
\end{tikzcd}
\end{equation*}
commutes. Thus, it suffices to show that this square also commutes with $x$ in the place of
$S_x(\tfid{A})$. Note that $\pi_2(x,\catid{\ctxext{\Gamma}{A}})\jdeq x$. 
Since $\ctxext{{\Gamma}{A}}{W_A(A)}$ is itself a pullback, it suffices
and it is straightforward to verify the equalities
\begin{align*}
W_A(A)\circ\pi_2(x,W_A(A))\circ x & \jdeq W_A(A)\circ\idtm{A}\circ x \\
\pi_2(A,A)\circ\pi_2(x,W_A(A))\circ x & \jdeq \pi_2(A,A)\circ\idtm{A}\circ x.\qedhere
\end{align*}
\end{itemize}
\end{proof}



\begin{comment}
We then demonstrate how to define a category internal to an E-system. The main
point of this lies mainly in demonstrating how the operations of E-systems can
be used to define a category, because it is no surprise that a category can be
defined in type theory. Nevertheless, it is noteworthy that we need no further 
conditions on E-systems.
This is unlike the situation in category theory, where we need
to assume that the ambient category has finite limits in order to say what an
internal category is. In fact, it is possible to define an E-system internal
to an E-system, just using the algebraic structure of an E-system, and actually
defining an internal category was the hardest part of that definition.
\end{comment}

\subsection{The category of internal morphisms}\label{subsec:im_cat}

\begin{defn}
For every $A,B\in\cat{F}/\Gamma$ we define the set 
\begin{equation*}
\thom{A}{B}\defeq T(\ctxwk{A}{B}).
\end{equation*}
An element $f\in \thom{A}{B}$ is called an \emph{internal morphism in context
$\Gamma$}. We sometimes write $\jhom{\Gamma}{A}{B}{f}$ to indicate that $f$ is
an internal morphism over $\Gamma$, or we may draw a diagram of the form
\begin{equation*}
\begin{tikzcd}[column sep=tiny]
A \arrow[rr,"f"] \arrow[dr,fib] & & B \arrow[dl,fib] \\
& \Gamma
\end{tikzcd}
\end{equation*}
but we shall generally omit the arrows down to $\Gamma$ and say instead that we have
a diagram in context $\Gamma$. 
\end{defn}

\begin{rmk}
Note that $\thom{\catid{\Gamma}}{A}\jdeq T(A)$ for any $A\in\cat{F}/\Gamma$, 
because we have 
$W_{\catid{\ctxext{\Gamma}{A}}}\jdeq \catid{\cat{F}/\ctxext{\Gamma}{A}}$.

Note also that $\thom{\ctxext{A}{P}}{B}\jdeq \thom{P}{\ctxwk{A}{B}}$ 
for any $P\in\cat{F}/\ctxext{\Gamma}{A}$ and $B\in\cat{F}/\Gamma$,
because $W_{\ctxext{A}{P}}\jdeq W_P\circ W_A$.
Once we have established a category of which the morphisms are given by 
$\thom{\blank}{\blank}$, we therefore get that 
\begin{equation*}
\ctxext{A}{(\blank)}\dashv W_A.
\end{equation*}
The right adjoint to weakening by $A$, if it exists, will be the dependent
product $\Pi_A$. 
\end{rmk}

\begin{defn}
Let $A,B\in\cat{F}/\Gamma$.
For any $f\in\thom{A}{B}$ we define the pre-composition E-homomorphism
\begin{equation*}
f_\ast \defeq S_f\circ W_A/B : \cat{F}/\ctxext{\Gamma}{B}\to\cat{F}/\ctxext{\Gamma}{A}.
\end{equation*}
Furthermore, for $g\in\thom{B}{C}$ we may consider $W_B(C)\in\cat{F}/\ctxext{\Gamma}{B}$ as a morphism from $W_B(C)$ to $\catid{\ctxext{\Gamma}{B}}$. Therefore, $f_\ast$ acts on the terms of
$W_B(C)$, so it makes sense to define $g\circ f\defeq f_\ast(g)$. In fact, it makes
sense to let $(\blank)\circ f$ be the infix notation for $f_\ast$. 
\end{defn}

\begin{rmk}
Note that for $\jhom{\Gamma}{B}{C}{g}$, 
we have $g\circ f\in T(S_f(W_A/B(W_B(C))))$, whereas we would like that
$g\circ f\in\thom{A}{C}$. More generally, we can show that
\begin{equation*}
S_f\circ (W_A/B)\circ W_B\jdeq W_A.
\end{equation*}
Since weakening is a weakening homomorphism, we have
\begin{equation*}
S_f\circ (W_A/B)\circ W_B\jdeq S_f\circ W_{W_A(B)}\circ W_A
\end{equation*}
By \autoref{tTf:StWf_id} of \autoref{defn:esystems} it we get that
\begin{equation*}
S_f\circ W_{W_A(B)}\circ W_A\jdeq W_A.
\end{equation*}
\end{rmk}

\begin{rmk}
To see how this extended notion of precomposition works, suppose we have a diagram
\begin{equation*}
\begin{tikzcd}[column sep=tiny]
Q \arrow[rr,"g"] \arrow[dr,fib] & & R \arrow[dl,fib] \\
& B
\end{tikzcd}
\end{equation*}
in context $\Gamma$, i.e.~$\jhom{\ctxext{\Gamma}{B}}{Q}{R}{g}$, and let $\jhom{\Gamma}{A}{B}{f}$. Precomposing by $f$ gives
us a diagram
\begin{equation*}
\begin{tikzcd}[column sep=tiny]
\jcomp{A}{f}{Q} \arrow[rr,"\jcomp{A}{f}{g}"] \arrow[dr,fib] & & \jcomp{A}{f}{R} \arrow[dl,fib] \\
& A
\end{tikzcd}
\end{equation*}
in context $\Gamma$.
\end{rmk}

\begin{rmk}
Note that \autoref{f:SidfWff_id} of \autoref{defn:esystems} asserts precisely
that ${(\tfid{A})}_\ast\jdeq \catid{\cat{F}/\ctxext{\Gamma}{A}}$ for any $A\in\cat{F}/\Gamma$. In
particular, it follows that $g\circ\tfid{A}\jdeq g$ for any $g\in\thom{A}{B}$
\end{rmk}

\begin{lem}\label{lem:compcomp}
For any $f\in\thom{A}{B}$ and $g\in\thom{B}{C}$ we have $f_\ast\circ g_\ast\jdeq (g\circ f)_\ast$. 
\end{lem}

\begin{proof}
\begin{align*}
f_\ast\circ g_\ast & \jdeq S_f\circ (W_A/B) \circ S_g \circ (W_B/C)\\
& \jdeq S_f\circ S_{W_A(g)}\circ(W_A/\ctxext{B}{W_B(C)})\circ (W_B/C) \\
& \jdeq S_{S_f(W_A(g))}\circ (S_f/W_A(W_B(C)))\circ (W_A/\ctxext{B}{W_B(C)}) \circ W_B/C \\
& \jdeq S_{S_f(W_A(g))}\circ ((S_f\circ (W_A/B) \circ W_B)/C) \\
& \jdeq S_{S_f(W_A(g))}\circ ((S_f\circ W_{W_A(B)}\circ W_A)/C) \\
& \jdeq S_{S_f(W_A(g))}\circ W_A/C \\
& \jdeq (g\circ f)_\ast.\qedhere
\end{align*}
\end{proof}

\begin{thm}
For each object $\Gamma$ of an E-system $\mathbb{E}$, we have a category
$\mathbb{C}_I(\mathbb{E},\Gamma)$ of internal morphisms of $\mathbb{E}$,
consisting of the same objects as $\cat{F}/\Gamma$ and the hom-sets 
$\thom{A}{B}$. 
\end{thm}

\begin{proof}
The fact that composition is associative is a direct corollary of
\autoref{lem:compcomp}. The axiom $(\tfid{A})_\ast\jdeq\catid{\ctxext{\Gamma}{A}}$
implies that the identity morphisms satisfy the right identity law. It remains
to show that $\tfid{B}\circ f\jdeq f$. This is a simple calculation:
\begin{equation*}
\tfid{B}\circ f 
  \jdeq
\subst{f}{\ctxwk{A}{\tfid{B}}}
  \jdeq
\subst{f}{\tfid{\ctxwk{A}{B}}}
  \jdeq
f.\qedhere
\end{equation*}
\end{proof}

\begin{rmk}
Since any E-homomorphism is both a substitution homomorphism and a weakening
homomorphism, it follows immediately that any E-homomorphism preserves any
$f_\ast$, i.e.~for any $f\in\thom{A}{B}$ in $\mathcal{E}$ and any E-homomorphism
$F:\mathcal{E}\to\mathcal{D}$, the diagram
\begin{equation*}
\begin{tikzcd}[column sep=large]
\mathcal{E}/\ctxext{\Gamma}{A} \arrow[r,"F/\ctxext{\Gamma}{A}"] & \mathcal{D}/F(\ctxext{\Gamma}{A})
  \\
\mathcal{E}/\ctxext{\Gamma}{B} \arrow[u,"f_\ast"] \arrow[r,swap,"F/\ctxext{\Gamma}{B}"] & \mathcal{D}/F(\ctxext{\Gamma}{B}) \arrow[u,swap,"F(f)_\ast"]
\end{tikzcd}
\end{equation*}
commutes. This applies to weakening, substitution and pre-composition, and it
follows that for any E-homomorphism $F:\mathcal{E}\to\mathcal{D}$ and any
$\Gamma\in\mathcal{E}$, we have a functor $\mathbb{C}_I(F,\Gamma):
\mathbb{C}_I(\mathcal{E},\Gamma)\to\mathbb{C}_I(\mathcal{D},F(\Gamma))$.
\end{rmk}

The following fact will be useful:

\begin{lem}\label{lem:compW_W}
Let $f\in\thom{A}{B}$ be an internal morphism in context $\Gamma$. Then one has
\begin{equation*}
f_\ast\circ W_B\jdeq W_A.
\end{equation*}
\end{lem}

\begin{proof}
The proof is a simple calculation:
\begin{equation*}
f_\ast\circ W_B\jdeq S_f\circ W_A/B\circ W_B\jdeq S_f\circ W_{W_A(B)}\circ W_A
\jdeq W_A.\qedhere
\end{equation*}
\end{proof}

\subsection{Pairing and the projections}

The composition $\ctxext{A}{P}$ of $A\in\cat{F}/\Gamma$ and $P\in\cat{F}/\ctxext{\Gamma}{A}$
behaves like a strict $\Sigma$-type. We may define the pairing term
$\typefont{pair}^{A,P}\defeq\tfid{\ctxext{A}{P}}\in T(W_P(W_A(\ctxext{A}{P})))$
and the projections and prove several useful properties about them. The strictness
is found, among other things, in the fact that we can prove judgmental $\eta$-equality,
and that pairing is strictly associative.

In this section we will make more extensive use of the infix form of the
weakening and substitution operations.

\begin{defn}
Let $x\in T(A)$ and $u\in T(S_x(P))$ for $A\in\cat{F}/\Gamma$ and $P\in\cat{F}/\ctxext{\Gamma}{A}$. 
We define the \emph{term extension of $x$ and $u$} to be
\begin{equation*}
\tmext{x}{u}\defeq \subst{u}{{x}{\tfid{\ctxext{A}{P}}}}\in T(\ctxext{A}{P}).
\end{equation*}
\end{defn}

To prove anything about the term $\tmext{x}{u}$, we need the following property.

\begin{thm}\label{subst_by_tmext}
Let $x\in T(A)$ and $u\in T(S_x(P))$ for $A\in\cat{F}/\Gamma$ and $P\in\cat{F}/\ctxext{\Gamma}{A}$.
Then we have
\begin{equation*}
S_{\tmext{x}{u}}\jdeq S_u\circ (S_x/P):\cat{F}/\ctxext{\Gamma}{{A}{P}}\to\cat{F}/\Gamma
\end{equation*}
\end{thm}

\begin{proof}
\begin{align*}
S_{\tmext{x}{u}} & \jdeq S_{S_u(S_x(\tfid{\ctxext{A}{P}}))} \\
& \jdeq S_{S_u(S_x(\tfid{\ctxext{A}{P}}))}\circ (S_u\circ W_{S_x(P)})\circ (S_x\circ W_{A}) \\
& \jdeq S_u \circ S_{S_x(\tfid{\ctxext{A}{P}})} \circ W_{S_x(P)}\circ S_x\circ W_A \\
& \jdeq S_u \circ (S_x/P) \circ S_{\tfid{\ctxext{A}{P}}} \circ W_P\circ W_A \\
& \jdeq S_u \circ (S_x/P) \circ S_{\tfid{\ctxext{A}{P}}} \circ W_{\ctxext{A}{P}}\\
& \jdeq S_u \circ (S_x/P).\qedhere
\end{align*}
\end{proof}

\begin{defn}
Let $A\in\cat{F}/\Gamma$ and $P\in\cat{F}/\ctxext{\Gamma}{A}$. We define
\begin{align*}
\cprojfstf{A}{P} & \defeq \ctxwk{P}{\tfid{A}}\in\thom{\ctxext{A}{P}}{A}\\
\cprojsndf{A}{P} & \defeq \tfid{P}\in\thomd{\cprojfstf{A}{P}}{\ctxext{A}{P}}{P}
\end{align*}
\end{defn}

\begin{cor}\label{cor:pairing_ehom}
Let $F:\mathcal{E}\to\mathcal{D}$ be an E-homomorphism and let $x\in T(A)$ and
$u\in T(S_x(P))$. Then we have
\begin{equation*}
F(\tmext{x}{u})\jdeq \tmext{F(x)}{F(u)}.
\end{equation*}
Also, we have
\begin{align*}
F(\cprojfstf{A}{P}) & \jdeq \cprojfstf{F(A)}{F(P)} \\
F(\cprojsndf{A}{P}) & \jdeq \cprojsndf{F(A)}{F(P)}.
\end{align*}
\end{cor}

\begin{thm}\label{thm:famext_up}
For any $A\in\cat{F}/\Gamma$ and $P\in\cat{F}/\ctxext{\Gamma}{A}$,
the map
\begin{equation*}
(x,u)\mapsto \tmext{x}{u}:\big(\bigsqcup\nolimits_{x\in T(A)} T(\subst{x}{P})\big)\to T(\ctxext{A}{P}).
\end{equation*}
is an isomorphism.  
\end{thm}

\begin{proof}
Lets write $\varphi$ for the indicated map. The inverse $\psi$ of $\varphi$ is defined
by $w\mapsto (\subst{w}{\cprojfstf{A}{P}},\subst{w}{\cprojsndf{A}{P}})$. We have
to show that for any $x\in T(A)$ and $u\in T(\subst{x}{P})$, one has
\begin{align*}
\subst{\tmext{x}{u}}{\cprojfstf{A}{P}} & \jdeq x\\
\subst{\tmext{x}{u}}{\cprojsndf{A}{P}} & \jdeq u,
\end{align*}
and that for any $w\in T(\ctxext{A}{P})$, one has
\begin{equation*}
\tmext{\subst{w}{\cprojfstf{A}{P}}}{\subst{w}{\cprojsndf{A}{P}}} \jdeq w.
\end{equation*}
To show that $\subst{\tmext{x}{u}}{\cprojfstf{A}{P}} \jdeq x$, we use that
$S_{\tmext{x}{u}}\jdeq S_u\circ S_x/P$ to show that
\begin{align*}
\subst{\tmext{x}{u}}{\cprojfstf{A}{P}}
  & \jdeq 
\subst{\tmext{x}{u}}{\ctxwk{P}{\tfid{A}}} \\
  & \jdeq
\subst{u}{{x}{\ctxwk{P}{\tfid{A}}}} \tag{By \autoref{subst_by_tmext}}\\
  & \jdeq
\subst{u}{\ctxwk{\subst{x}{P}}{\subst{x}{\tfid{A}}}} \\
  & \jdeq
\subst{x}{\tfid{A}} \\
  & \jdeq
x
\end{align*}
To show that $\subst{\tmext{x}{u}}{\cprojsndf{A}{P}}\jdeq u$, note that
\begin{equation*}
\subst{\tmext{x}{u}}{\cprojsndf{A}{P}}
  \jdeq
\subst{u}{{x}{\tfid{P}}}
  \jdeq
\subst{u}{\tfid{\subst{x}{P}}}
  \jdeq
u
\end{equation*}
Therefore, $\psi$ is a left-inverse of $\varphi$. To show that it is also a
right inverse, note that \autoref{cor:pairing_ehom} gives us that
\begin{equation*}
\tmext{\cprojfst{A}{P}{w}}{\cprojsnd{A}{P}{w}}
  \jdeq
\subst{w}{(\tmext{\cprojfstf{A}{P}}{\cprojsndf{A}{P}})}.
\end{equation*}
Thus, it suffices to show that $\tmext{\cprojfstf{A}{P}}{\cprojsndf{A}{P}}\jdeq
\tfid{\ctxext{A}{P}}$, which is straightforward.
\end{proof}

\begin{lem}\label{lem:tmext_assoc}
For any $x\in T(A)$, $u\in T(S_x(P))$ and $v\in T(S_{\tmext{x}{u}}(Q))$ we have
\begin{equation*}
\tmext{{x}{u}}{v}\jdeq \tmext{x}{{u}{v}}\in T(\ctxext{{A}{P}}{Q}).
\end{equation*}
\begin{comment}
Also, we have
\begin{align*}
\cprojfstf{A}{P}\circ\cprojfstf{\ctxext{A}{P}}{Q} & \jdeq\cprojfstf{A}{\ctxext{P}{Q}} \\
\cprojsndf{A}{P}\circ\cprojfstf{\ctxext{A}{P}}{Q} & \jdeq \cprojfstf{P}{Q}\circ\cprojsndf{A}{\ctxext{P}{Q}} \\
\cprojsndf{\ctxext{A}{P}}{Q} & \jdeq \cprojsndf{P}{Q}\circ\cprojsndf{A}{\ctxext{P}{Q}}.
\end{align*}
\end{comment}
\end{lem}

\begin{proof}
By \autoref{subst_by_tmext}, we have $S_v\circ S_{\tmext{x}{u}}\jdeq S_v\circ S_u\circ S_x \jdeq S_{\tmext{u}{v}}\circ S_x$,
so associativity of term extension follows.
\end{proof}

\begin{comment}
\begin{lem}
Let $f\in\thom{A}{\ctxext{B}{Q}}$ Then we have
\begin{equation*}
f \jdeq \tmext{\jcomp{A}{f}{\cprojfstf{B}{Q}}}{\jcomp{A}{f}{\cprojsndf{B}{Q}}}.
\end{equation*}
Alternatively, when $f_0\in\thom{A}{B}$ and $f_1\in\thom{A}{\jcomp{A}{f}{Q}}$,
then we have $\tmext{f_0}{f_1}\in\thom{A}{\ctxext{B}{Q}}$ and
\begin{align*}
\jcomp{A}{\tmext{f_0}{f_1}}{\cprojfstf{B}{Q}} & \jdeq f_0 \\
\jcomp{A}{\tmext{f_0}{f_1}}{\cprojsndf{B}{Q}} & \jdeq f_1.
\end{align*}
\end{lem}

\begin{proof}
Straightforward
\end{proof}
\end{comment}

We end this section by showing that weakening is the same operation as pre-composition
with a first projection.

\begin{thm}\label{precomp_by_proj}
Precomposition with $\cprojfstf{A}{P}$ is weakening by $P$, i.e.
\begin{equation*}
(\cprojfstf{A}{P})_\ast \jdeq W_P.
\end{equation*}
\end{thm}

\begin{proof}
\begin{align*}
(\cprojfstf{A}{P})_\ast & \jdeq S_{\cprojfstf{A}{P}}\circ W_{\ctxext{A}{P}} \\
& \jdeq S_{W_P(\tfid{A})}\circ W_P\circ W_A \\
& \jdeq W_P\circ S_{\tfid{A}}\circ W_A \\
& \jdeq W_P.\qedhere 
\end{align*}
\end{proof}

\subsection{The interchange laws}\label{sec:interchange}

We are now in the position to define vertical and horizontal composition, and
prove properties of them.

\begin{defn}
Let $\jhom{\Gamma}{A}{B}{f}$ and $\jfhom{\Gamma}{A}{B}{f}{P}{Q}{F}$. Then we
define 
\begin{equation*}
\jhomdefn{\Gamma}{{A}{P}}{{B}{Q}}{\jvcomp{P}{f}{F}}{\unfold{\jvcomp{P}{f}{F}}}.
\end{equation*}
Whenever we say that we have a diagram of the form
\begin{equation*}
\begin{tikzcd}
R \arrow[r,"f_2"] \arrow[d,fib] &
S \arrow[d,fib] \\
P \arrow[r,"f_1"] \arrow[d,fib] &
Q \arrow[d,fib] \\
A \arrow[r,"f_0"] &
B
\end{tikzcd}
\end{equation*}
we mean that we have
$f_0\in\thom{A}{B}$, $f_1\in\thomd{f_0}{P}{Q}$ and 
$f_2\in\thomd{\jvcomp{P}{f_0}{f_1}}{R}{S}$.
\end{defn}

\begin{cor}
Let $\jhom{\Gamma}{A}{B}{f}$ and $\jfhom{\Gamma}{A}{B}{f}{P}{Q}{F}$. Then we
have
\begin{equation*}
H(\jvcomp{P}{f}{F})\jdeq \jvcomp{H(P)}{H(f)}{H(F)}
\end{equation*}
for any E-homomorphism $H:\mathbb{E}\to\mathbb{D}$.
\end{cor}

\begin{lem}
Vertical composition is associative.
\end{lem}

\begin{proof}
Consider the diagram
\begin{equation*}
\begin{tikzcd}
R \arrow[r,"f_2"] \arrow[d,fib] &
S \arrow[d,fib] \\
P \arrow[r,"f_1"] \arrow[d,fib] &
Q \arrow[d,fib] \\
A \arrow[r,"f_0"] &
B
\end{tikzcd}
\end{equation*}
in context $\Gamma$.
Because weakening distributes over term extension, and term extension is
associative, we have
\begin{align*}
\jvcomp{R}{\jvcomp{P}{f_0}{f_1}}{f_2}
  & \jdeq
\tmext{\ctxwk{R}{\tmext{\ctxwk{P}{f_0}}{f_1}}}{f_2}
  \\
  & \jdeq
\tmext{\tmext{\ctxwk{R}{{P}{f_0}}}{\ctxwk{R}{f_1}}}{f_2}
  \\
  & \jdeq
\tmext{\ctxwk{\ctxext{P}{R}}{f_0}}{\tmext{\ctxwk{R}{f_1}}{f_2}}
  \tag{By \autoref{lem:tmext_assoc}}
  \\
  & \jdeq
\jvcomp{{P}{R}}{f_0}{\jvcomp{R}{f_1}{f_2}}.\qedhere
\end{align*}
\end{proof}

\begin{defn}
Let $\jhom{\Gamma}{A}{B}{f}$ and $\jfhom{\Gamma}{A}{B}{f}{P}{Q}{F}$. Then we
define the E-homomorphism 
\begin{equation*}
F_\circledast\defeq F_\ast\circ (f_\ast/Q):
\mathbb{E}/\ctxext{{\Gamma}{B}}{Q}\to\mathbb{E}/\ctxext{{\Gamma}{A}}{P}.
\end{equation*}
The infix notation of $F_\circledast$ is taken to be $\jfcomp{\blank}{\blank}{\blank}{F}{\blank}$.
\end{defn}

\begin{lem}\label{lem:three-composition}
Let $\jhom{\Gamma}{A}{B}{f}$ and $\jfhom{\Gamma}{A}{B}{f}{P}{Q}{F}$. Then we
have the equality
\begin{equation*}
F_\circledast\jdeq (\jvcomp{P}{f}{F})_\ast.
\end{equation*}
\end{lem}

\begin{proof}
\begin{align*}
F_\ast\circ (f_\ast/Q)
  & \jdeq
S_F\circ W_P\circ S_f/(W_A(Q))\circ W_A/\ctxext{B}{Q}
  \\
  & \jdeq
S_F\circ S_{W_P(f)}/W_P(W_A(Q))\circ W_P/W_A(\ctxext{B}{Q})\circ W_A/\ctxext{B}{Q}
  \\
  & \jdeq
S_F\circ S_{W_P(f)}/W_P(W_A(Q))\circ W_{\ctxext{A}{P}}/\ctxext{B}{Q}
  \\
  & \jdeq
S_{\tmext{W_P(f)}{F}}\circ W_{\ctxext{A}{P}}/\ctxext{B}{Q}
  \tag{By \autoref{subst_by_tmext}}
  \\
  & \jdeq
(\jvcomp{P}{f}{F})_\ast.\qedhere
\end{align*}
\end{proof}

We are now in position to state and prove the interchange law of horizontal
and vertical composition.

\begin{thm}\label{thm:interchange}
Consider the diagram
\begin{equation*}
\begin{tikzcd}
P \arrow[r,"F"] \arrow[d,fib] &
Q \arrow[r,"G"] \arrow[d,fib] &
R \arrow[d,fib] \\
A \arrow[r,"f"] &
B \arrow[r,"g"] &
C
\end{tikzcd}
\end{equation*}
in context $\Gamma$. Then the equality
\begin{equation*}
{\jcomp{{A}{P}}{\jvcomp{P}{f}{F}}{\jvcomp{Q}{g}{G}}}
  \jdeq
{\jvcomp{P}{\jcomp{A}{f}{g}}{\jfcomp{A}{f}{P}{F}{G}}}
\end{equation*}
of morphisms from $\ctxext{A}{P}$ to $\ctxext{C}{R}$ in context $\Gamma$ holds.
\end{thm}

\begin{proof}
By \autoref{lem:three-composition}, we have
\begin{align*}
\jcomp{{A}{P}}{\jvcomp{P}{f}{F}}{\jvcomp{Q}{g}{G}}
  & \jdeq
\jcomp{P}{F}{\jcomp{A}{f}{\tmext{\ctxwk{Q}{g}}{G}}}
  \tag{By \autoref{lem:three-composition}}
  \\
  & \jdeq
\jcomp{P}{F}{\tmext{\ctxwk{\jcomp{A}{f}{Q}}{\jcomp{A}{f}{g}}}{\jcomp{A}{f}{G}}}
  \\
  & \jdeq
\tmext{\jcomp{P}{F}{\ctxwk{\jcomp{A}{f}{Q}}{\jcomp{A}{f}{g}}}}{\jcomp{P}{F}{{A}{f}{G}}}
  \\
  & \jdeq
\tmext{\jcomp{P}{F}{\ctxwk{\jcomp{A}{f}{Q}}{\jcomp{A}{f}{g}}}}{\jfcomp{A}{f}{P}{F}{G}}
  \\
  & \jdeq
\tmext{\ctxwk{P}{\jcomp{A}{f}{g}}}{\jfcomp{A}{f}{P}{F}{G}}
  \tag{By \autoref{lem:compW_W}}
  \\
  & \jdeq
\jvcomp{P}{\jcomp{A}{f}{g}}{\jfcomp{A}{f}{P}{F}{G}}.\qedhere
\end{align*}
\end{proof}

\begin{thm}
Consider the diagram
\begin{equation*}
\begin{tikzcd}
P \arrow[r,"F"] \arrow[d,fib] &
Q \arrow[r,"G"] \arrow[d,fib] &
R \arrow[d,fib] \\
A \arrow[r,"f"] &
B \arrow[r,"g"] &
C
\end{tikzcd}
\end{equation*}
in context $\Gamma$. 
Then $F_\circledast\circ G_\circledast\jdeq (\jfcomp{A}{f}{P}{F}{G})_\circledast$.
In other words the composition $\jfcomp{}{}{}{\blank}{\blank}$ is associative.
\end{thm}

\begin{proof}
\begin{align*}
F_\circledast\circ G_\circledast
  & \jdeq
(\jvcomp{P}{f}{F})_\ast\circ(\jvcomp{Q}{g}{G})_\ast 
  \tag{By \autoref{lem:three-composition}}\\
  & \jdeq
(\jcomp{\ctxext{A}{P}}{\jvcomp{P}{f}{F}}{\jvcomp{Q}{g}{G}})_\ast 
  \tag{By \autoref{lem:compcomp}}\\
  & \jdeq
(\jvcomp{}{\jcomp{A}{f}{g}}{\jfcomp{A}{f}{P}{F}{G}})_\ast 
  \tag{By \autoref{thm:interchange}} \\
  & \jdeq
(\jfcomp{A}{f}{P}{F}{G})_\circledast.
  \tag{By \autoref{lem:three-composition}}
\end{align*}
\end{proof}

\begin{lem}
Let $\jhom{\Gamma}{A}{B}{f}$ and $\jfhom{\Gamma}{A}{B}{f}{P}{Q}{F}$. Then
$\jvcomp{P}{f}{F}$ is the unique morphism from $\ctxext{A}{P}$ to $\ctxext{B}{Q}$
with the property that both the diagram
\begin{equation*}
\begin{tikzcd}
\ctxext{A}{P}
  \ar{r}{\jvcomp{P}{f}{F}}
  \ar{d}[swap]{\cprojfstf{A}{P}}
& \ctxext{B}{Q}
  \ar{d}{\cprojfstf{B}{Q}}
  \\
A \ar{r}[swap]{f}
& B
\end{tikzcd}
\end{equation*}
commutes, and $\jcomp{}{\jvcomp{P}{f}{F}}{\cprojsndf{B}{Q}}\jdeq F$.
\end{lem}

\begin{proof}
We first note that
\begin{align*}
\jcomp{}{\jvcomp{P}{f}{F}}{\cprojfstf{B}{Q}}
  & \jdeq
\jcomp{P}{F}{\jcomp{A}{f}{\cprojfstf{B}{Q}}} 
  \tag{By \autoref{lem:three-composition}} \\
  & \jdeq
\jcomp{P}{F}{\ctxwk{\jcomp{A}{f}{Q}}{\jcomp{A}{f}{\idtm{B}}}} \\
  & \jdeq
\jcomp{P}{F}{\ctxwk{\jcomp{A}{f}{Q}}{f}} \\
  & \jdeq
\ctxwk{P} f \tag{By \autoref{lem:compW_W}} \\
  & \jdeq
\jcomp{\ctxext{A}{P}}{\cprojfstf{A}{P}}{f}.
  \tag{By \autoref{precomp_by_proj}}
\end{align*}
Also, we have
\begin{align*}
\jcomp{}{\jvcomp{P}{f}{F}}{\cprojsndf{B}{Q}}
  & \jdeq
\jcomp{P}{F}{\jcomp{A}{f}{\idtm{Q}}}
  \tag{By \autoref{lem:three-composition}}
  \\
  & \jdeq
\jcomp{P}{F}{\idtm{\jcomp{A}{f}{Q}}}
  \\
  & \jdeq
F.
\end{align*}
Thus, we conclude that $\jvcomp{P}{f}{F}$ has indeed the stated property. For
the uniqueness, let $G:\ctxext{A}{P}\to\ctxext{B}{Q}$ be a morphism such that
$\jcomp{}{G}{\cprojfstf{B}{Q}}\jdeq\jcomp{}{\cprojfstf{A}{P}}{f}$ and
$\jcomp{}{G}{\cprojsndf{B}{Q}}\jdeq F$. Then it follows that
\begin{equation*}
G \jdeq \tmext{\jcomp{}{\cprojfstf{A}{P}}{f}}{F}
  \jdeq \tmext{\ctxwk{P}{f}}{F}
  \jdeq \jvcomp{P}{f}{F}.\qedhere
\end{equation*}
\end{proof}

\subsection{E-systems are E'-systems}

\begin{lem}
Consider $A\in\cat{F}/\Gamma$ and $Q\in\cat{F}/\ctxext{\Gamma}{B}$. Then there
is an isomorphism
\begin{equation*}
\{h\in\thom{A}{\ctxext{B}{Q}}\mid \jcomp{A}{h}{\cprojfstf{B}{Q}}\jdeq f\}\cong T(\jcomp{A}{f}{Q}).
\end{equation*}
for any $f\in\thom{A}{B}$.
\end{lem}

\begin{proof}
We have the isomorphism
\begin{align*}
\thom{A}{\ctxext{B}{Q}} 
& \jdeq T(\ctxwk{A}{\ctxext{B}{Q}}) \\
& \jdeq T(\ctxext{\ctxwk{A}{B}}{\ctxwk{A}{Q}}) \\
& \cong \bigsqcup\nolimits_{(f\in T(\ctxwk{A}{B}))}T(\subst{f}{\ctxwk{A}{Q}})
\tag{By \autoref{thm:famext_up}} \\
& \jdeq \bigsqcup\nolimits_{(f\in\thom{A}{B})}T(\jcomp{A}{f}{Q}).
\end{align*}
Also, we find $\cprojfst{\ctxwk{A}{B}}{\ctxwk{A}{Q}}{h}\jdeq\subst{h}{\ctxwk{A}{\cprojfstf{B}{Q}}}\jdeq
\jcomp{A}{h}{\cprojfstf{B}{Q}}$.
\end{proof}

\begin{thm}
E-systems are the same things as E'-systems.
\end{thm}

\begin{proof}
For any $f:A\to B$ in context $\Gamma$ and any $Q\in\cat{F}/B$, we define
\begin{equation*}
\pi_2(f,Q)\defeq \jvcomp{\jcomp{A}{f}{Q}}{f}{\tfid{\jcomp{A}{f}{Q}}}
  : \ctxext{A}{\jcomp{A}{f}{Q}}\to\ctxext{B}{Q}.
\end{equation*}
Then we have the commuting diagram
\begin{equation*}
\begin{tikzcd}[column sep=huge]
\ctxext{A}{\jcomp{A}{f}{Q}} \arrow[r,"{\pi_2(f,Q)}"] \arrow[d,swap,"{\cprojfstf{A}{\jcomp{A}{f}{Q}}}"] & \ctxext{B}{Q} \arrow[d,"{\cprojfstf{B}{Q}}"] \\
A \arrow[r,"f"] & B
\end{tikzcd}
\end{equation*}
in context $\Gamma$. To show that this is a pullback diagram, consider a
morphism $g:X\to A$ in context $\Gamma$. Then we have the isomorphisms
\begin{equation*}
\{h\in\thom{X}{\ctxext{B}{Q}}\mid \jcomp{X}{h}{\cprojfstf{B}{Q}}\jdeq\jcomp{X}{g}{f}\}
  \cong
T((\jcomp{X}{g}{f})_\ast(Q))
  \cong
\{u\in\thom{X}{\ctxext{A}{\jcomp{A}{f}{Q}}}\mid\jcomp{X}{u}{\cprojfstf{A}{\jcomp{A}{f}{Q}}}\jdeq g\}
\end{equation*}
Thus, we find for every $h:X\to\ctxext{B}{Q}$ satisfying
$\jcomp{X}{h}{\cprojfstf{B}{Q}}\jdeq\jcomp{X}{g}{f}$, a unique morphism
$u:X\to\ctxext{A}{\jcomp{A}{f}{Q}}$ satisfying
$\jcomp{X}{u}{\cprojfstf{A}{\jcomp{A}{f}{Q}}}\jdeq g$. It is easy to verify
that $\jcomp{X}{u}{\pi_2(f,Q)}\jdeq h$, so the universal property of pullbacks
holds. The functoriality conditions follow immediately from the interchange
laws proven in \autoref{sec:interchange}.

We have described functors from E'-systems to E-systems and vice versa. It is
a straightforward check that these are mutual inverses.
\end{proof}

\begin{comment}
\subsection{Categories internal to E-systems}
The definition of internal category we're about to give, has a type of objects
and a type of morphisms, such that the codomain function is a projection. It might
be more appropriate to call such structures `right categories', and the corresponding
structures where the domain function is a projection might be called `left
categories'. In particular, the opposite of a right category will be a left
category. There is also a symmetric notion of a category.

\begin{defn}
A category-framework $\cftalg{A}$ in context $\Gamma$ in an E-system $\mathbb{E}$ consists of
\begin{enumerate}
\item A family $\cftalgc{\cftalg{A}}\in\cat{F}/\Gamma$, representing the type of
objects of the category framework;
\item A family $\cftalgf{\cftalg{A}}\in\cat{F}/\ctxext{\Gamma}{\cftalgc{\cftalg{A}}}$,
representing the type of morphisms of the category framework, where a term of
$\subst{X}{\cftalgf{\cftalg{A}}}$ is considered to be a morphism into $X\in T(\cftalgc{\cftalg{A}})$;
\item A morphism $\jhom{\Gamma}{\ctxext{\cftalgc{\cftalg{A}}}{\cftalgf{\cftalg{A}}}}{\cftalgc{\cftalg{A}}}{\cftctxext}$, representing the domain function;
\item A morphism $\jhom{{\Gamma}{\cftalgc{\cftalg{A}}}}{{\cftalgf{\cftalg{A}}}{\jcomp{}{\cftctxext}{\cftalgf{\cftalg{A}}}}}{\cftalgf{\cftalg{A}}}{\cftfamext}$, representing composition;
\item A term $\jterm{{\Gamma}{\cftalgc{\cftalg{A}}}}{\cftalgf{\cftalg{A}}}{\cftempf{\cftalg{A}}}$,
representing the identity morphisms.
\end{enumerate}
\end{defn}

\begin{rmk}
Instead of looking at $\cftfamext$ as a morphism from 
$\ctxext{\cftalgf{\cftalg{A}}}{\jcomp{}{\cftctxext}{\cftalgf{\cftalg{A}}}}$
to $\cftalgf{\cftalg{A}}$ in context $\ctxext{\Gamma}{\cftalgc{\cftalg{A}}}$, 
one could also look at $\cftfamext$ as a 
morphism \emph{over $\cprojfstf{\cftalgc{\cftalg{A}}}{\cftalgf{\cftalg{A}}}$} 
in context $\Gamma$, as indicated in the following diagram:
\begin{equation*}
\begin{tikzcd}[column sep=large]
\cftalgf{\cftalg{A}}
  \ar[fib]{d}
& \jcomp{}{\cftctxext}{\cftalgf{\cftalg{A}}}
  \ar[fib]{d}
  \ar{l}[swap]{\cftfamext}
  \ar{r}
& \cftalgf{\cftalg{A}}
  \ar[fib]{d}
  \\
\cftalgc{\cftalg{A}}
& \ctxext{A}{\cftalgf{\cftalg{A}}}
  \ar{l}{\cprojfstf{A}{\cftalgf{\cftalg{A}}}}
  \ar{r}[swap]{\cftctxext}
& \cftalgc{\cftalg{A}}
\end{tikzcd}
\end{equation*}
From this point of view, we see that $\cftfamext$ has an object $X$ and a 
morphism $f$ into $X$ as parameters, 
then takes a morphism into the domain of $f$ as an argument, 
and outputs a morphism into $X$. The projection maps $(X,f)$ to $X$, so we see
that composition is indeed an operation of this kind.
\end{rmk}

A category-framework is not a category unless several axioms of the theory of
categories are satisfied. The domain of a composed arrow $g\circ f$ must be the
domain of $f$, composition must be associative, the domains of the identity
morphisms must be the same as their codomains, and identities must satisfy
the left and right identity laws. However, the axioms expressing these 
conditions are not very insightful when they're written down directly. Thus, we
take the usual way and define the notion of functor before we describe the
actual conditions of being a category.

\begin{defn}
Let $\cftalg{A}$ and $\cftalg{B}$ be category-frameworks in context $\Gamma$,
in an E-system $\mathbb{E}$.
An \emph{internal functor $\cfthom{F}$ from $\extalg{A}$ to
$\extalg{B}$ in context $\Gamma$} is a pair 
$(\cfthomc{\cfthom{F}},\cfthomf{\cfthom{F}})$ consisting of
\begin{align*}
\jalign\jhom
  {\Gamma}
  {\cftalgc{\cftalg{A}}}
  {\cftalgc{\cftalg{B}}}
  {\cfthomc{\cfthom{F}}}
  \\
\jalign\jfhom
  {\Gamma}
  {\cftalgc{\cftalg{A}}}
  {\cftalgc{\cftalg{B}}}
  {\cfthomc{\cfthom{F}}}
  {\cftalgf{\cftalg{A}}}
  {\cftalgf{\cftalg{B}}}
  {\cfthomf{\cfthom{F}}}
\end{align*}
for which the diagrams
\begin{equation}\label{eq:exthom1}
\begin{tikzcd}
\ctxext{\cftalgc{\cftalg{A}}}{\cftalgf{\cftalg{A}}}
  \ar{r}{\jvcomp{}{\cfthomc{\cfthom{F}}}{\cfthomf{\cfthom{F}}}}
  \ar{d}[swap]{\cftctxext[\cftalg{A}]}
& \ctxext{\cftalgc{\cftalg{B}}}{\cftalgf{\cftalg{B}}}
  \ar{d}{\cftctxext[\cftalg{B}]}
  \\
\cftalgc{\cftalg{A}}
  \ar{r}[swap]{\cfthomc{\cfthom{F}}}
& \cftalgc{\cftalg{B}}
\end{tikzcd}
\end{equation}
and
\begin{equation}\label{eq:exthom2}
\begin{tikzcd}[column sep=huge]
\ctxext{\cftalgf{\cftalg{A}}}{\jcomp{}{\cftctxext[\cftalg{A}]}{\cftalgf{\cftalg{A}}}}
  \ar{r}{\jvcomp{}{\cfthomf{\cfthom{F}}}{\jcomp{}{\cftctxext[\cftalg{A}]}{\cfthomf{\cfthom{F}}}}}
  \ar{d}[swap]{\cftfamext[\cftalg{A}]}
& \jcomp{}{\cfthomc{\cfthom{F}}}{\ctxext{\cftalgf{\cftalg{B}}}{\jcomp{}{\cftctxext[\cftalg{B}]}{\cftalgf{\cftalg{B}}}}}
  \ar{d}{\jcomp{}{\cfthomc{\cfthom{F}}}{\cftfamext[\cftalg{B}]}}
  \\
\cftalgf{\cftalg{A}}
  \ar{r}[swap]{\cfthomf{\cfthom{F}}}
& \jcomp{}{\cfthomc{\cfthom{F}}}{\cftalgf{\cftalg{B}}}
\end{tikzcd}
\end{equation}
commute, and for which $\subst{\cftempf{\cftalg{A}}}{\cfthomf{\cfthom{F}}}
\jdeq\jcomp{}{\cfthomc{\cfthom{F}}}{\cftempf{\cftalg{B}}}$.
\end{defn}

\begin{defn}
Let $\cftalg{A}$ be a category-framework. We define the category-framework
\begin{equation*}
\cftfamalg{\cftalg{A}}\defeq(\cftalgf{\cftalg{A}},\jcomp{}{\cftctxext}%
{\cftalgf{\cftalg{A}}},\cftfamext,\jcomp{}{\cftctxext}{\cftfamext},
\jcomp{}{\cftctxext}{\cftempf{\cftalg{A}}}),
\end{equation*}
When $\cftalg{P}$ is a category-framework in context $\ctxext{\Gamma}{A}$, then
and we define the category-framework
\begin{equation*}
\ctxext{\cftalgc{\cftalg{A}}}{\cftalg{P}}
  \defeq
( \ctxext{A}{\cftalgc{\cftalg{P}}},
  \cftalgf{\cftalg{P}},
  \jvcomp{}{\idtm{A}}{\cftctxext[\cftalg{P}]},
  \cftfamext[\cftalg{P}],
  \cftempf{\cftalg{P}}).
\end{equation*}
\end{defn}

\begin{defn}
An internal category $\cftalg{A}$ in an E-system $\mathbb{E}$ consists of a
category-framework $\cftalg{A}$ satisfying the following conditions:
\begin{enumerate}
\item \emph{The domain condition.} The pair 
$(\cftctxext,\idtm{\jcomp{}{\cftctxext}{\cftalgf{\cftalg{A}}}})$ is a
functor from $\ctxext{\cftalgc{\cftalg{A}}}{\cftfamalg{\cftalg{A}}}$ to
$\cftalg{A}$.
\item \emph{The associativity law.} The pair
$(\cftfamext,\idtm{\jcomp{}{\cftfamext}{\cftalgf{\cftfamalg{\cftalg{A}}}}})$
is a functor from $\ctxext{\cftalgf{\cftalg{A}}}{\cftfamalg{\cftfamalg{\cftalg{A}}}}$ to
$\cftfamalg{\cftalg{A}}$.
\item \emph{The identity domain condition.} The domain of the identity morphism
is the same as its codomain:
\begin{equation*}
\jtermeq
  {{\Gamma}{\cftalgc{\cftalg{A}}}}
  {\ctxwk{\cftalgc{\cftalg{A}}}{\cftalgc{\cftalg{A}}}}
  {\subst{\cftempf{\cftalg{A}}}{\cftctxext[\cftalg{A}]}}
  {\idtm{\cftalgc{\cftalg{A}}}}.
\end{equation*}
\item \emph{The left identity law.} A morphism precomposed by the identity morphism
is equal to the original morphism:
\begin{equation*}
\jtermeq
  {{{\Gamma}{\cftalgc{\cftalg{A}}}}{\cftalgf{\cftalg{A}}}}
  {\ctxwk{\cftalgf{\cftalg{A}}}{\cftalgf{\cftalg{A}}}}
  {\subst{\cftempf{\cftalg{A}}}{\cftfamext}}
  {\idtm{\cftalgf{\cftalg{A}}}}.
\end{equation*}
\item \emph{The right identity law.} A morphism postcomposed by the identity morphism
is equal to the original morphism:
\begin{equation*}
\jtermeq
  {{{\Gamma}{\cftalgc{\cftalg{A}}}}{\cftalgf{\cftalg{A}}}}
  {\ctxwk{\cftalgf{\cftalg{A}}}{\cftalgf{\cftalg{A}}}}
  {\subst{\jcomp{}{\cftctxext}{\cftempf{\cftalg{A}}}}{\cftfamext}}
  {\idtm{\cftalgf{\cftalg{A}}}}.
\end{equation*}
\end{enumerate}
\end{defn}

\begin{rmk}
Since we have only used operations of E-systems to define internal categories,
it immediately follows that if $\cftalg{A}$ is an internal category in
$\mathbb{E}$ and if $F:\mathbb{E}\to\mathbb{D}$ is an E-homomorphism, then
$F(\cftalg{A})$ is an internal category in $\mathbb{D}$. In particular, if
$\cftalg{Q}$ is an internal category in context $\ctxext{\Gamma}{B}$ and
$f:A\to B$, then the change of base $f_\ast(\cftalg{Q})$ is an internal category
in context $\ctxext{\Gamma}{A}$. Thus, it is immediate that the notion of
internal category is stable under change of base.
\end{rmk}
\end{comment}

\begin{comment}
\section{Dependent morphisms}

\begin{defn}
Let $f_0\in\thom{A}{B}$ be an internal morphism and let $P\in\cat{F}/\ctxext{\Gamma}{A}$ and
$Q\in\cat{F}/\ctxext{\Gamma}{B}$. We define
\begin{equation*}
\thomd{f_0}{P}{Q} \defeq \thom{P}{{f_0}_\ast(Q)}
\end{equation*}
We indicate that $f_1\in\thomd{f_0}{P}{Q}$ by drawing the diagram
\begin{equation*}
\begin{tikzcd}
P \arrow[r,"f_1"] \arrow[d,->>] & Q \arrow[d,->>] \\
A \arrow[r,swap,"f_0"] & B.
\end{tikzcd}
\end{equation*}
\end{defn}

\begin{defn}
Consider the diagram
\begin{equation*}
\begin{tikzcd}
P \arrow[r,"f_1"] \arrow[d,->>] & Q \arrow[d,->>] \arrow[r,"g_1"] & R \arrow[d,->>] \\
A \arrow[r,swap,"f_0"] & B \arrow[r,"g_0"] & C.
\end{tikzcd}
\end{equation*}
We define
\begin{align*}
\jvcomp{P}{f_0}{f_1} & \defeq\unfold\jvcomp{P}{f_0}{f_1} & & \in \thom{\ctxext{A}{P}}{\ctxext{B}{Q}} \\
\jfcomp{A}{f_0}{P}{f_1}{g_1} & \defeq \unfold\jfcomp{A}{f_0}{P}{f_1}{g_1} & & \in \thomd{\jcomp{A}{f_0}{g_0}}{P}{R}.
\end{align*}
\end{defn}

\begin{rmk}
Consider
\begin{equation*}
\begin{tikzcd}
P \arrow[r,"f_1"] \arrow[d,->>] & Q \arrow[d,->>] \\
A \arrow[r,swap,"f_0"] & B.
\end{tikzcd}
\end{equation*}
Then the diagram
\begin{equation*}
\begin{tikzcd}[column sep=large]
\ctxext{A}{P} \arrow[r,"\jvcomp{P}{f_0}{f_1}"] \arrow[d,swap,"\cprojfstf{A}{P}"] & \ctxext{B}{Q} \arrow[d,"\cprojfstf{B}{Q}"] \\
A \arrow[r,swap,"f_0"] & B
\end{tikzcd}
\end{equation*}
commutes. Also, we have
\begin{equation*}
S_{\tmext{x}{u}}(\jvcomp{P}{f_0}{f_1}) \jdeq \tmext{S_x(f_0)}{S_{\tmext{x}{u}}(f_1)}
\end{equation*}
for each $x\in T(A)$ and $u\in T(S_x(P))$.
\end{rmk}

\section{Natural transformations of E-homomorphisms}
\begin{rmk}
When $\mathcal{E}$ is an E-system with empty context, then we have the category
structure on the contexts given by $\thom{\Gamma}{\Delta}= T(W_\Gamma(\Delta))$.
\end{rmk}

\begin{defn}
A \emph{natural transformation} of E-homomorphisms from $F:\mathcal{E}\to\mathcal{D}$
to $G:\mathcal{E}\to\mathcal{D}$ preserving the empty context consists of
\begin{enumerate}
\item A t-morphism $\tau0_A\in\thom{F(\Gamma)}{G(\Gamma)}$ for every $\Gamma\in\mathcal{E}$.
\item A commuting square square
\begin{equation*}
\begin{tikzcd}
F(A) \arrow[r,"\tau1_A"] \arrow[d,twoheadrightarrow] & G(A) \arrow[d,twoheadrightarrow] \\
F(\Gamma) \arrow[r,swap,"\tau0_\Gamma"] & G(\Gamma)
\end{tikzcd}
\end{equation*}
for every family $P$ over $A$ of solid arrows, such that the diagram
\begin{equation*}
\begin{tikzcd}
F(A) \arrow[r,"\tau1_A"] & G(A) \\
F(\Gamma) \arrow[r,swap,"\tau0_\Gamma"] \arrow[densely dotted,u,"F(x)"] & G(\Gamma) \arrow[densely dotted,u,swap,"G(x)"]
\end{tikzcd}
\end{equation*}
commutes for every $x\in T(A)$.
\end{enumerate}
such that
\begin{enumerate}
\item the square
\begin{equation*}
\begin{tikzcd}[column sep=huge]
W_{F(A)}(F(B)) \arrow[r,"W_{F(A)}(\tau1_B)"] \arrow[d,->>] & W_{G(A)}(G(B)) \arrow[d,->>] \\
F(A) \arrow[r,"\tau1_A"] & G(A)
\end{tikzcd}
\end{equation*}
commutes for every $A,B\in\mathcal{E}/\Gamma$. Note that 
\begin{align*}
W_{F(A)}(\tau1_B)
  & \in 
T(W_{F(A)}(W_{F(B)}(G(B))))
  \\
  & =
T(W_{W_{F(A)}(F(B))}(W_{F(A)}(G(B))))
\end{align*}
\end{enumerate}
\end{defn}

\section{Indexed E-systems}
\begin{defn}
Let $\cat{A}$ be a category. An $\cat{A}$-indexed E-system is an internal
E-system in the category $Psh(\cat{A})$.
\end{defn}

\section{Examples of adjunctions}

\begin{lem}
For each $A\in\mathcal{E}/\Gamma$, $W_A$ is left adjoint to $\Pi_A$. 
\end{lem}

\begin{proof}
The unit $\eta:1_{\cat{F}/\ctxext{\Gamma}{A}}\Rightarrow W_A\circ\Pi_A$ is given by
\begin{equation*}
\eta_P:\lam{f}{x} ev(\lambda(f),x)
\end{equation*}
$\eta_P\in\thom{P}{W_A(\Pi_A(P))}$
\end{proof}

\begin{lem}
Suppose $\cat{D}$ is a retract of a pre-category $\cat{F}$, i.e. there are functors
$S:\cat{D}\to\cat{F}$ and $R:\cat{F}\to\cat{D}$ with a natural isomorphism
$\tau:R\circ S \simeq \catid{\cat{D}}$, and suppose that $\cat{F}$ is a weakening
category. Then $\cat{D}$ is a weakening system too.
\end{lem}

\begin{proof}
Let $f:\Delta\to \Gamma$ be a morphism in $\cat{D}$. Then we define $W_f$ to be the
composition
\begin{equation*}
\begin{tikzcd}[column sep=3.2em]
\cat{D}/\Gamma
  \arrow[r,"S/\Gamma"]
  &
\cat{F}/S(\Gamma)
  \arrow[r,"W_{S(f)}"]
  &
\cat{F}/S(\Delta)
  \arrow[r,"R/S(\Delta)"]
  &
\cat{D}/R(S(\Delta))
  \arrow[r,"\tau_\Delta\circ{-}"]
  &
\cat{D}/\Delta
\end{tikzcd}
\end{equation*}
We first verify whether this assembles a pre-weakening structure on $\cat{D}$.
\begin{enumerate}
\item Let $\Delta$ be an object of $\cat{D}$. Then $W_{\catid{\Delta}}$ is the composition
\begin{equation*}
\begin{tikzcd}[column sep=3.2em]
\cat{D}/\Delta
  \arrow[r,"S/\Delta"]
  &
\cat{F}/S(\Delta)
  \arrow[r,"R/S(\Delta)"]
  &
\cat{D}/R(S(\Delta))
  \arrow[r,"{-}\circ\tau_\Delta"]
  &
\cat{D}/\Delta
\end{tikzcd}
\end{equation*}
This functor maps $g:\Delta'\to \Delta$ consecutively to $S(g)$, to $R(S(g))$, to
$\tau_\Delta\circ R(S(g))$. Note that $\tau_\Delta\circ R(S(g))=g\circ \tau_{\Delta'}$ by
naturality.
\end{enumerate}
\end{proof}

\section{Weak E-systems}

\begin{rmk}
Note that $W$ is a strict $2$-functor frm $\cat{F}$ to $\mathbb{Cat}$ of which
the value at an object $\Gamma$ is specified to be $\cat{F}/\Gamma$, and which
is required to preserve the final object. We could have considered $W$ to be a 
pseudo-functor from $\cat{F}$ to $\mathbb{Cat}$ instead. A coherence theorem
by Power \cite{Power89} then allows for the strictification of this pseudo-functor
into a (strict) weakening structure. (Does this map $\Gamma$ to $\cat{F}/\Gamma$
and final to final?). Thus, we know what a weak pre-weakening system has to be.
The following questions need to be answered:
\begin{itemize}
\item What is a weak weakening/projection/substitution system.
\item What is a weak E-system? I.e.~what are the coherence rules for something
to be a weakening/projection/substitution homomorphism?
\item Can they all be strictified? Are there more examples of weak E-systems?
\end{itemize}
\end{rmk}
\end{comment}

\section{E-systems compared to other systems of type dependency}\label{sec:esys_compared}
In an effort to formalize the simplicial set model of Martin-L\"of type theory
with the univalence axiom, Voevodsky has studied various notions of models of
type theory, taking Cartmell's \emph{contextual categories} as a starting
point. Voevodsky's definition of a \emph{C-system} is equivalent to that of Cartmell
\cite{VV_C-systems_quotients}. There are several techniques of defining 
C-systems: in \cite{VV_Csys_univ}, C-systems are constructed out of a universe
in a category; in \cite{VV_C-systems_quotients}, sub-C-systems and regular
quotients of C-systems are defined; in \cite{VV_C-systems_monad}, a C-system
is constructed out of a monad $R$ on the category of sets and a left $R$-module
with values in $\mathbf{Set}$. 

In \cite{VV_B-systems}, Voevodsky introduced his notion of \emph{B-systems}, which is
an essentially algebraic theory of type dependency with infinitely many sorts 
and operations, and subsequently he showed that there is a full and faithful
functor from the category of C-systems to the category of B0-systems, of which
the image is contained in the subcategory of B-systems. This fact is important at least
for a theoretical reason: it tells us that the conditions of being a B-system
are necessary to axiomatize type dependency. In other words,
this is a theorem that tells us that we need no further axioms. It is
unknown whether the categories of C-systems and B-systems are actually equivalent.
This question would settle the sufficiency of the conditions of being a B-system.

The category of B-systems will be shown to be equivalent to the category of
stratified E-systems. However, we do not arrive at the fact that the functor
$\mathcal{E}:\mathbf{Bsys}\to\mathbf{Esys}$ is full without imposing an additional
condition on the E-homomorphisms. 
To ensure that the category of stratified E-systems with stratified 
E-homomorphisms between them is a full subcategory of the category 
$\mathbf{Esys}$, we will require that the underlying functor of each E-homomorphism
lifts factorizations. It is
important to note that this condition is also algebraic, so that the category
of E-systems remains algebraic.

\subsection{Lifting factorizations}
Recall
that for any morphism $f:X\to Y$ in a category $\cat{C}$ there is a category
$\mathbf{fact}(f)$ of factorizations of $f$. Also, any functor $F:\cat{C}\to\cat{D}$
determines a functor $F_{\mathbf{fact}(f)}:\mathbf{fact}(f)\to\mathbf{fact}(H(f))$
for any morphism $f$ of $\cat{C}$.

\begin{defn}
A functor $F:\cat{C}\to\cat{D}$ is said to
\emph{lift factorizations} if for any $f\in\cat{C}/X$, the functor
$F_{\mathbf{fact}(f)}:\mathbf{fact}(f)\to\mathbf{fact}(F(f))$ is an isomorphism
of categories. 
\end{defn}

\begin{comment}
\begin{rmk}
We may choose the property of lifting factorizations
to involve either an equivalence or an isomorphism of categories. The
isomorphism-version, would say that for every factorization $h'\circ g'$
of $F(f)$ there is a unique factorization $f\jdeq h\circ g$ in $\cat{C}$ such
that $F(g)\jdeq g'$ and $F(h)\jdeq h'$. In the version with equivalences,
the uniqueness is replaced by uniqueness up to isomorphism, and the equalites
are replaced by an isomorphism in $\mathbf{fact}(F(f))$. 

It is easier to state the isomorphism version of lifting factorizations with
inference rules and it might be easier to explain this condition on type theoretical
grounds, although the equivalence version has the advantage of being
categorical (i.e.~invariant under equivalence of categories).

In the present
context, it doesn't matter very much which one we pick: the categories in which
we're interested are all posets.
\end{rmk}
\end{comment}

\begin{eg}
For any slice category $\cat{C}/X$, the forgetful functor $\cat{C}/X\to\cat{C}$
lifts factorizations.
\end{eg}

\begin{defn}
We write $\mathbf{Esys}$ for the category of E-systems where the underlying
functor of each E-homomorphism lifts factorizations.
\end{defn}

\subsection{Stratified E-systems}

In this subsection we define the notion of stratified E-system. Then we show
that the category $\mathbf{Esys_s}$ of stratified E-systems with homomorphisms preserving the
stratification, is a full subcategory of $\mathbf{Esys}$. 

\begin{defn}
A category $\cat{C}$ with terminal object is said to be \emph{stratified} if there exists a 
\emph{stratification functor}
\begin{equation*}
L : \cat{C}\to (\mathbb{N},\geq)
\end{equation*}
such that
\begin{enumerate}
\item $L(X)\jdeq 0$ if and only if $X$ is terminal, and for any $f:X\to Y$ we have
$L(X)\jdeq L(Y)$ if and only if $X\jdeq Y$ and $f\jdeq\catid{X}$. 
\item every morphism $f:X\to Y$ in $\cat{C}_\mathbb{E}$, where $L(X)\jdeq
n+m+1$ and $L(Y)\jdeq n$, has a unique factorization 
\begin{equation*}
\begin{tikzcd}
X \arrow[r,"f_m"] & X_m \arrow[r,"f_{m-1}"] & \cdots \arrow[r,"f_1"] & X_1 \arrow[r,"f_0"] & Y
\end{tikzcd}
\end{equation*}
where $L(X_i)\jdeq n+i$.
\end{enumerate}
A functor $F:\cat{C}\to\cat{D}$ between stratified categories is said to be stratified 
if $L_{\cat{C}}\jdeq L_{\cat{D}}\circ F$. 
\end{defn}

%\begin{rmk}
%The categorical version of being stratified, is that there exists a factofibration
%$L:\cat{C}\to (N,\geq)$ which preserves the terminal object.
%\end{rmk}

\begin{lem}
Any stratified category is a rooted tree.
\end{lem}

\begin{proof}
Let $\cat{C}$ be a stratified category. The root of the tree is going to be the terminal
object of $\cat{C}$. Note that objects of $\cat{C}$ have no automorphisms other
than the identity morphisms. Therefore, to show that $\cat{C}$ is a tree,
it suffices to show that for every object $X$ with $L(X)>0$ there is
a unique object $\eft(X)$ with $L(\eft(X))\jdeq L(X)-1$, with a morphism $X\to \eft(X)$.
The existence follows from the existence of the factorization of $X\to 1$. The 
uniqueness follows from the observation that for any arrow
$X\to Y$ with $L(Y)\jdeq L(X)-1$, we can factorize $X\to 1$ through $Y$. Since
factorizations are unique, we will have $Y\jdeq \eft(X)$. 
\end{proof}

\begin{lem}
The category of small stratified categories is a full subcategory of the category
of small categories with terminal objects and between them functors which lift 
factorizations and preserve the terminal object.
\end{lem}

\begin{proof}
Let $F:\cat{C}\to\cat{D}$ be a functor between stratified categories which lifts
factorizations and 
which preserves terminal objects. We show by induction that $F$ is stratified.
Since $F$ preserves the terminal object, $L(Y)\jdeq 0$ implies $L(F(Y))\jdeq 0$.
Now suppose that for $n\in\mathbb{N}$, we have that $L(\Gamma)\jdeq n$ implies
$L(F(Y))\jdeq n$, and let $X$ be such that $L(X)\jdeq n+1$ with
$p_X:X\to\eft(X)$. Then the poset $\mathbf{fact}(p_X)$ is isomorphic to the
poset $\mathbf{2}$ with two objects, one of which is smaller than the other. By the
assumption that $F$ lifts factorizations, the poset $\mathbf{fact}(F(p_X))$
is also isomorphic to $\mathbf{2}$. Thus, the only factorization
of $F(X)\to\unit$ starts with $F(p_X):H(X)\to F(\eft(X))$, which implies that
$L(F(X))\jdeq n+1$. 
\end{proof}

\begin{rmk}
It is easy to show that a category can only be stratified in at most one way,
and that if a category is stratified, then so are any of its slices. Therefore,
it makes sense to ask for the property that the weakening and substitution
functors in a pre-weakening or pre-substitution system are stratified.
\end{rmk}

\begin{defn}
A pre-E-system is said to be \emph{stratified} if its underlying category is
stratified and if each $W_A$ and $S_x$ is a stratified functor. The category of
stratified pre-E-systems with stratified E-homomorphisms between them is denoted by
$\mathbf{Esys_s}$.
\end{defn}

\begin{cor}\label{lem:strat_full}
The category of stratified pre-E-systems is a full subcategory of the category of
pre-E-systems.
\end{cor}

\subsection{The embedding of B0-systems into pre-E-systems}

Voevodsky distinguishes between pre-B-systems, which is an essentially algebraic theory with
only operations and no equalities; B0-systems, which is an extension of the
essentially algebraic theory of pre-B-systems which ensures that all the operations behave
well with two sets of the operations ($\eft$ and $\ebd$); and B-systems, which extends the 
theory of B0-systems with equalities ensuring that the rest of the operations 
are compatible with each other. Voevodsky, furthermore distinguishes
non-unital and unital versions of these, and noted that not all B-homomorphisms
between unital B-systems are unital. The same distinguishment could be made for 
E-systems by forgetting the projection structure, but we have not been able to construct a
functor from unital B0-systems to unital pre-E-systems. Therefore, we shall only
consider non-unital B0-systems and unital B-systems. In this section we will
recall the definition of non-unital B0-systems and construct a full and faithful
functor from them to the category of pre-E-systems. 

One reason for introducing B-systems by first introducing pre-B-systems is that
it is clear in the definition of pre-B-systems that there are only countably
many operations. Since we are interested in embedding the B-systems into E-systems,
we shall organize the definition of B-systems a bit differently, but of course
the resulting definition of B0-systems and B-systems will be (trivially) 
equivalent.

\begin{defn}
A \emph{B-framework} is a collection of data of the following form:
\begin{enumerate}
\item for all $n\in\mathbb{N}$ two sets $B_n$ and $\tilde{B}_n$. 
\item for all $n\in\mathbb{N}$ maps of the form
\begin{align*}
\eft[n] & : B_{n+1}\to B_n \\
\ebd[n] & : \tilde{B}_n\to B_n.
\end{align*}
For $m,n\in\mathbb{N}$, we denote the composition $\eft[n]\circ\cdots\circ\eft[n+m]:B_{n+m+1}\to B_n$ by $\eft[n]^m$. 
\item $B_0$ is a singleton $\{\pt\}$.
\end{enumerate} 
\end{defn}

\begin{rmk}
Since the indices $n,m\in\mathbb{N}$ can be infered, we will usually omit them
in the infix form of the operators.
\end{rmk}

\begin{defn}
A homomorphism $H:\mathbb{B}\to\mathbb{A}$ of B-frameworks consists of maps
$H_n:B_n\to A_n$ and $\tilde{H}_n:\tilde{B}_n\to\tilde{A}_n$ such that
\begin{align*}
\eft(H(X)) & \jdeq H(\eft(X)) \\
\ebd(\tilde{H}(x)) & \jdeq H(\ebd(x))
\end{align*}
for any $X\in B_n$ and $x\in\tilde{B}_n$. The category of B-frameworks is
denoted by $\mathbf{Bfr}$. 
\end{defn}

\begin{defn}
For every B-framework $\mathbb{B}$ and any $X\in B_n$, there is a B-framework
$\mathbb{B}/X$ given by
\begin{align*}
(B/X)_{m} & \jdeq \{Y\in B_{n+m}\mid\eft^{m}(Y)\jdeq X\}\\
(\tilde{B}/X)_m & \jdeq \{y\in \tilde{B}_{n+m}\mid\eft^m(\ebd(y))\jdeq X\}.
\end{align*}
Also, for any homomorphism $H:\mathbb{B}\to\mathbb{A}$ of B-frameworks and any
$X\in B_n$, there is a homomorphism $H/X:\mathbb{B}/X\to\mathbb{A}/H(X)$
defined in the obvious way.
\end{defn}

\begin{rmk}
Note that for $X\in B_n$ and $Y\in B_{n+m}$ such that $\eft^m(Y)\jdeq X$, 
we have $(\mathbb{B}/X)/Y\cong B/Y$.
\end{rmk}

\begin{defn}
A B0-system $\mathbb{B}$ consists of a B-framework $\mathbb{B}$ and homomorphisms
\begin{align*}
W_{X} & : \mathbb{B}/\eft(X)\to\mathbb{B}/X\\
S_{x} & : \mathbb{B}/\ebd(x)\to\mathbb{B}/\eft(\ebd(x))
\end{align*}
of B-frameworks, for any $X\in B_{n+1}$ and any $x\in\tilde{B}_{n+1}$. 
\end{defn}

\begin{lem}
When $\mathbb{B}$ is a B0-system, then so is each $\mathbb{B}/X$.
\end{lem}

\begin{proof}
Straightforward.
\end{proof}

\begin{defn}
A B0-homomorphism $H:\mathbb{B}\to\mathbb{A}$ is a homomorphism of B-frameworks
for which the diagrams
\begin{equation*}
\begin{tikzcd}[column sep=huge]
\mathbb{B}/X \arrow[r,"H/X"] & \mathbb{A}/H(X) \\
\mathbb{B}/\eft(X) \arrow[u,"W_X"] \arrow[r,swap,"H/\eft(X)"] & \mathbb{A}/\eft(H(X)) \arrow[u,swap,"W_{H(X)}"]
\end{tikzcd}
\end{equation*}
and
\begin{equation*}
\begin{tikzcd}[column sep=huge]
\mathbb{B}/\ebd(x) \arrow[r,"H/\ebd(x)"] \arrow[d,swap,"S_x"] & \mathbb{A}/\ebd(H(X)) \arrow[d,"S_{\tilde{H}(x)}"] \\
\mathbb{B}/\eft(\ebd(x)) \arrow[r,swap,"H/\eft(\ebd(x))"] & \mathbb{A}/\eft(\ebd(H(X)))
\end{tikzcd}
\end{equation*}
of homomorphisms of B-frameworks commute for every 
$X\in B_n$ and $x\in\tilde{B}_n$. The category of B0-systems is denoted
by $\mathbf{B0sys}$.
\end{defn}

Now that we have defined the category of B0systems, we can start defining the
embedding into pre-E-systems.

\begin{defn}
Let $\mathbb{B}$ be a B-framework. Then $\mathbb{B}$ determines
a stratified category $U(\mathbb{B})$. Also, any homomorphism of B-frameworks
determines a stratified functor.
\end{defn}

\begin{constr}
The set of objects of $U(\mathbb{B})$ is taken to be the set
$\bigsqcup_{(n\in\mathbb{N})}B_n$. Now we define
$(k,X)\leq (n,Y)$ if and only if $n\leq k$ and $\eft[n]^{k-n}(X)\jdeq Y$.
In other words, we only have $(n+m,X)\leq (n,\eft^m(X))$. It is immediate from
this definition that for any $X\in B_{n+1}$ there is a unique $\Gamma\in
B_n$ such that $(n+1,X)\leq (n,\Gamma)$, so $U(\mathbb{B})$ is stratified.

To show that the functor $U(H):U(\mathbb{B})\to U(\mathbb{A})$ is stratified for
any homomorphism $H:\mathbb{B}\to\mathbb{A}$ of B-frameworks, note that
by \autoref{lem:strat_full} it suffices to show that each $U(H)$ 
lifts factorizations.
Let $H:\mathbb{B}\to\mathbb{A}$ be a homomorphism of B-frameworks, and let
$(m,X):(n+m,X)\leq (n,\Gamma)$. Then the category $\mathbf{fact}_{(m,X)}$ of
factorizations of $(m,X)$ is a finite linear order with
$m+1$ elements, and so is the category $\mathbf{fact}_{(m,H(X))}$. 
\end{constr}

\begin{rmk} 
It is in fact the case that $U(\mathbb{B})$ is the free category generated by the
graph with $\bigsqcup_{(n\in\mathbb{N})}B_n$ as the set of vertices, and
for each object of the form $(n+1,X)$ an edge to $(n,\eft[n](X))$. 
Essentially all $U$ does, is forgetting about the term structure 
$\tilde{B}$ of $\mathbb{B}$.

Furthermore, it is useful to note that for any $X\in B_n$, we get an 
isomorphism $U(\mathbb{B}/X)\cong  U(\mathbb{B})/(n,X)$ of rooted trees,
natural in $X$. Therefore we will usually not distinguish between $U(\mathbb{B})/X$
and $U(\mathbb{B})/(n,X)$. 
\end{rmk}

\begin{defn}
Let $\mathbb{B}$ be a B0-system. Then $U(\mathbb{B})$ can be given the structure
of a pre-E-system. We write $\mathcal{E}(\mathbb{B})$ for the resulting
pre-E-system. Thus we get a functor $\mathcal{E}:\mathbf{B0sys}\to\mathbf{Esys_s}$.
\end{defn}

\begin{constr}
We begin by simultaneously defining the term structure and the pre-substitution
structure on $U(\mathbb{B})$ by induction on the natural numbers. More precisely,
for any $m\in\mathbb{N}$ we will define, for any $(n+m,X)\leq (n,\Gamma)$ a set
$T(m,X)$ and for any $x\in T(m,X)$ a homomorphism 
$S_x:\mathbb{B}/X\to\mathbb{B}/\Gamma$ of B-frameworks. Then it follows
immediately that we get stratified functors $U(S_x):U(\mathbb{B})/X\to U(\mathbb{B})/\Gamma$
defining the pre-substitution structure of $\mathcal{E}(\mathbb{B})$.

For $(n,X)\in U(\mathbb{B})$, we define $T(0,X)\defeq
\{\pt[X]\}$ and $S_{\pt[X]}\defeq \catid{\mathbb{B}/X}$. We also define for
$(n+1,X)\leq (n,\Gamma)$ the set $T(1,X)\jdeq\{x\in B_{n+1}\mid \ebd(x)\jdeq X\}$.
Note that for $x\in T(1,X)$, the homomorphism $S_x:\mathbb{B}/X \to\mathbb{B}/\Gamma$
is already assumed to exist, because $\mathbb{B}$ is a B0-system.

For the inductive step, suppose that for $m\in\mathbb{N}$ we have sets $T(m,X)$
for any $(n+m,X)\leq (n,\Gamma)$ and homomorphisms $S_x:\mathbb{B}/X \to \mathbb{B}/\Gamma$
of B-frameworks, for any $x\in T(m,X)$. Then we define
\begin{equation*}
T(m+1,X)\defeq\bigsqcup_{x\in T(m,\eft(X))} T(1,S_x(X)).
\end{equation*}
For $(x,u)\in T(m+1,X)$ we define the homomorphism $S_{(x,u)}:\mathbb{B}/X\to
\mathbb{B}/\Gamma$ as
\begin{equation*}
S_{(x,u)}\defeq S_u\circ S_x/X.
\end{equation*}
It remains to define a pre-weakening structure on $U(\mathbb{B})$. To do this,
it suffices to define homomorphisms $W_{(m,X)}:\mathbb{B}/\Gamma \to\mathbb{B}/X$
of B-frameworks, for any $(n+m,X)\leq (n,\Gamma)$. We define
\begin{equation*}
W_{(m,X)}\defeq W_X\circ\cdots\circ W_{\eft^{m-1}(X)}.\qedhere
\end{equation*}
\end{constr}

\begin{thm}
The functor $\mathcal{E}$ is full and faithful.
\end{thm}

\begin{proof}
We have to show that the map $\mathrm{hom}(\mathbb{B},\mathbb{A})\to
\mathrm{hom}(\mathcal{E}(\mathbb{B}),\mathcal{E}(\mathbb{A}))$ is a bijection
for any two B0-systems $\mathbb{B}$ and $\mathbb{A}$.

To show that this map is injective, suppose $F,G:\mathbb{B}\to\mathbb{A}$ are 
B0-homomorphisms such that $\mathcal{E}(F)\jdeq\mathcal{E}(G)$. Then we have
$F_n(X)\jdeq\mathcal{E}(F)(n,X)\jdeq \mathcal{E}(G)(n,X)\jdeq G_n(X)$ for each
object $x\in B_n$. Also, we have $\tilde{F}_n(x)\jdeq T_{\mathcal{E}(F)}(n,x)
\jdeq T_{\mathcal{E}(G)}(n,x)\jdeq \tilde{G}_n(x)$ for each
$x\in\tilde{B}_n$. This shows that $F\jdeq G$, so we conclude that $\mathcal{E}$
is faithful.  

To show that the map $\mathrm{hom}(\mathbb{B},\mathbb{A})\to
\mathrm{hom}(\mathcal{E}(\mathbb{B}),\mathcal{E}(\mathbb{A}))$ 
is surjective, let $H:\mathcal{E}(\mathbb{B})\to
\mathcal{E}(\mathbb{A})$. Since both $\mathcal{E}(\mathbb{B})$ and
$\mathcal{E}(\mathbb{A})$ are stratified, we get from
\autoref{lem:strat_full} that $H$ is a stratified pre-E-homomorphism. 
Because $H$ is stratified, we get maps $H_n:B_n\to A_n$ for any $n\in\mathbb{N}$.
The term structure of $H$ gives maps $\tilde{H}_n:
\tilde{B}_n\to\tilde{A}_n$. This defines a B0-homomorphism since $H$ 
preserves the pre-weakening and pre-substitution structures of $\mathcal{E}(\mathbb{B})$.
It is immediate from the definition of $H_n$ and
$\tilde{H}_n$ that this B0-homomorphism is mapped back to $H$, so we conclude
that $\mathcal{E}$ is full.
\end{proof}

\subsection{B-systems are stratified E-systems}
There are two properties which hold for pre-E-systems of the form $\mathcal{E}
(\mathbb{B})$ for some B0-system $\mathbb{B}$, 
which stand in the way of proving that the category of stratified
pre-E-systems is equivalent to the category of B0-systems. The first is that
for any $A\in U(\mathbb{B})/\Gamma$ and $P\in U(\mathbb{B})/\ctxext{\Gamma}{A}$,
we have an isomorphism $T(\ctxext{A}{P})\cong\bigsqcup_{x\in T(A)}T(S_x(P))$.
The second, which depends on the first, is that for any $(x,u)\in T(\ctxext{A}{P})$
we have $S(x,u)\jdeq S_u\circ S_x/P$. In other words, both the term structure
and the pre-substitution structure of $\mathcal{E}(\mathbb{B})$ are compatible
with the categorical structure of the underlying category $U(\mathbb{B})$.
However, to prove the corresponding facts for E-systems, we needed the full
set of conditions on unital E-systems: to prove that the term structure of
$U(\mathbb{B})$ is compatible with composition we needed units, and to prove 
that the operations $S_x$ are compatible with the pairing function we needed
that they are E-homomorphisms. 

Presumably, we can assume these compatibility conditions on pre-E-systems in order
to prove the desired equivalence of B0-systems and stratified pre-E-systems. However,
that seems somewhat unnatural, especially given that they automatically hold
for E-systems. Thus, for the remainder of this section we set out to show that
the category of unital B-systems is equivalent to the category of stratified 
unital E-systems.

\begin{defn}
A unital B0-system is a B0-system with maps
\begin{equation*}
\delta_n : B_{n+1}\to \tilde{B}_{n+2}.
\end{equation*}
such that $\ebd(\delta_n(X))\jdeq W_{X}(X)$ for any $X\in B_n$. 
A B0-homomorphism $H$ is said to be unital if one has
\begin{equation*}
H(\delta(X))\jdeq\delta(H(X))
\end{equation*}
for any $X\in B_n$.  
\end{defn}

\begin{rmk}
Voevodsky showed in \cite{VV_B-systems} that the category of unital B0-systems
with unital B0-homomorphisms between them is not a full
subcategory of $\mathbf{B0sys}$. 
\end{rmk}

\begin{defn}
A \emph{(unital) B-system} is a B0-system for which the following conditions
hold:
\begin{enumerate}
\item Every $W_X$ and $S_x$ is a unital B0-homomorphism.
\item For every $x\in \tilde{B}_{n+1}$ one has $S_x\circ W_{\ebd(x)}\jdeq
\catid{\mathbb{B}/\eft(\ebd(x))}$. 
\item For every $x\in\tilde{B}_{n+1}$ one has $S_x(\delta(\ebd(x)))\jdeq x$.
\item For every $X\in B_{n+1}$ one has $S_{\delta(X)}\circ W_X/X\jdeq
\catid{\mathbb{B}/X}$. 
\end{enumerate}
\emph{B-homomorphisms} are unital B0-homomorphisms between B-systems.
We denote the category of B-systems by $\mathbf{Bsys}$. 
\end{defn}

\begin{lem}
For any non-unital B-system $\mathbb{B}$, $\mathcal{E}(\mathbb{B})$ is a 
stratified non-unital E-system. 
\end{lem}

\begin{proof}
Straightforward.
\end{proof}

\begin{defn}
For any unital B-system $\mathbb{B}$, $\mathcal{E}(\mathbb{B})$ has a
projection structure. We thus obtain a functor 
$\mathcal{E}:\mathbf{Bsys}\to\mathbf{Esys}$ from unital B-systems to E-systems.
\end{defn}

\begin{constr}
We first need to extend the functor $\mathcal{E}:\mathbf{B0sys}\to\mathbf{Esys_p}$
from non-unital B0-systems to non-unital pre-E-systems, to a functor from
unital B-systems to E-systems.

We need to give $\mathcal{E}(\mathbb{B})$ the structure of a unital
pre-E-system. We construct by induction on $m\in \mathbb{N}$, a term of type 
$T(m,W_{(m,X)}(X))$ with the property that that 
\begin{equation*}
S_{\tfid{(m,X)}}\circ W_{(m,X)}/X\jdeq\catid{U(\mathbb{B})/X},
\end{equation*} 
for every $(n+m,X)\leq (n,\Gamma)$ by induction on $m$. The extra
condition is part of the inductive construction, because we need it on the way.

For $m\jdeq 0$ we have but one
choice, and for $m\jdeq 1$ we can directly use the unital structure of
$\mathbb{B}$. For the inductive step, recall that
\begin{equation*}
T(m+1,W_{(m+1,X)}(X))\jdeq\bigsqcup_{x\in T(m,W_{(m+1,X)}(\eft(X)))}T(1,S_x(W_{(m+1,X)}(X))).
\end{equation*}
We have the term $x\defeq \tilde{W}_X(\tfid{(m,\eft(X))})\in T(m,W_{(m+1,X)}(\eft(X)))$.
Now we may observe that 
\begin{align*}
S_x(W_{(m+1,X)}(X))
  & \jdeq 
S_{\tilde{W}_X(\tfid{(m,\eft(X))})}(W_{(m+1,X)}(X))\\
  & \jdeq 
S_{\tilde{W}_X(\tfid{(m,\eft(X))})}(W_X(W_{(m,\eft(X))}(X))) \\
  & \jdeq
W_{X}(S_{\tfid{(m,\eft(X))}}(W_{(m,\eft(X))}(X))) \\
  & \jdeq 
W_{X}(X).
\end{align*}
Therefore, we see that $T(1,S_x(W_{(m+1,X)}(X)))\jdeq T(1,W_X(X))$, where we can
choose the term $\delta(X)$. Thus, we define
\begin{equation*}
\tfid{(m+1,X)}\defeq (\tilde{W}_X(\tfid{(m,\eft(X))}),\delta(X))
\end{equation*} 
for $m\geq 1$. Now we need to check that $S_{\tfid{(m+1,X)}}\circ 
W_{(m+1,X)}/X\jdeq \catid{U(\mathbb{B})/X}$. This is indeed the case:
\begin{align*}
S_{\tfid{(m+1,X)}}\circ W_{(m+1,X)}/X
  & \jdeq
S_{\delta(X)}\circ S_{\tilde{W}_X(\tfid{(m,\eft(X))})}\circ W_{(m+1,X)}/X \\
  & \jdeq
S_{\delta(X)}\circ W_X/X \\
  & \jdeq
\catid{U(\mathbb{B})/X}.
\end{align*}
This finishes the definition of the pre-projection structure on $\mathcal{E}(\mathbb{B})$.
Straightforward proves by induction give that each $W_{(m,X)}$ and $S_x$ is a 
pre-projection homomorphism and that $S_x(\tfid{(m,X)})\jdeq x$, for any 
$x\in T(m,X)$.
\end{constr}

\begin{thm}
The functor $\mathcal{E}:\mathbf{Bsys}\to\mathbf{Esys}$ lifts to an equivalence
to the subcategory $\mathbf{Esys_s}$ of stratified E-systems: i.e.~we have a
commuting diagram
\begin{equation*}
\begin{tikzcd}
& \mathbf{Esys_s} \arrow[dr,"\subseteq"]\\
\mathbf{Bsys} \arrow[ur,"\simeq"] \arrow[rr,swap,"\mathcal{E}"] & & \mathbf{Esys}
\end{tikzcd}
\end{equation*}
\end{thm}

\begin{proof}
We already have shown that $\mathcal{E}$ factors through the stratified
E-systems and that it is full and faithful, so it remains
to show that every stratified E-system is in the image of $\mathcal{E}$. In
other words, for every stratified E-system $\mathbb{E}$ we need to construct
a B0-system $\mathbb{B}$ such that $\mathcal{E}(\mathbb{B})\cong\mathbb{E}$. 

Let $\mathbb{E}$ be a stratified E-system. From the stratification functor
$L:\mathbb{C}_{\mathbb{E}}\to (\mathbb{N},\geq)$ we obtain the pre-image
presheaf $B(\mathbb{E}):\op{(\mathbb{N},\leq)}\to\mathbf{Set}$ given by
\begin{align*}
B(\mathbb{E})_n & \defeq \{X\in\mathrm{Ob}(\mathbb{C}_{\mathbb{E}})\mid L(X)\jdeq n\}
\end{align*}
and $\eft(X)\defeq\mathrm{cod}(p_X)$. We define the term structure of $B(\mathbb{E})_n$
to be given by
\begin{equation*}
\tilde{B}(\mathbb{E})_n \defeq \bigsqcup_{X\in B(\mathbb{E})_n} T(p_X)
\end{equation*}
with the projection map $\ebd[n]\defeq\pi_1:\tilde{B}(\mathbb{E})_n\to B(\mathbb{E})_n$.
This defines the B-framework $B(\mathbb{E})$, and we obtain a functor
$B:\mathbf{Esys_s}\to\mathbf{B0sys}$. It is immediate that we get isomorphisms
$U(B(\mathbb{E}))\cong\cat{F}_{\mathbb{E}}$, natural in $\mathbb{E}$. Also, we
get natural isomorphisms $B(\mathbb{E}/\Gamma)\cong B(\mathbb{E})/\Gamma$. 

The weakening and substitution structures on
$B(\mathbb{E})$ are given by $W_X\defeq B(W_{p_X})$ and $S_{(X,x)}\defeq B(S_x)$,
respectively. Since these functors are assumed to be stratified, they define
homomorphisms of B-frameworks. Because a weakening system of an E-system is
required to be functorial, i.e.~$W_{\ctxext{A}{P}}\jdeq W_P\circ W_A$ and
$W_{\catid{\Gamma}}\jdeq\catid{\cat{F}_{\mathbb{E}}/\Gamma}$, it follows
immediately that the weakening structure on $U(B(\mathbb{E}))$ is isomorphic to
that of $\mathbb{E}$. 
\end{proof}

\begin{comment}
In the following definition $\Delta(X)$ denotes the discrete category on the set
$X$ and $\mathbf{2}$ is the category $s\to t$.
\begin{defn}
Define $\mathbb{S}$ to be the pushout
\begin{equation*}
\begin{tikzcd}[column sep=large]
\Delta(\mathbb{N}) \arrow[r,"{n\mapsto(n,s)}"] \arrow[d,swap,"i"] & \Delta(\mathbb{N})\times\mathbf{2} \arrow[d] \\
(\mathbb{N},\leq) \arrow[r] & \mathbb{S}
\end{tikzcd}
\end{equation*}
in $\mathbf{Cat}$.
\end{defn}

Note that for any $n\in\mathbb{N}$, we have the map $S^n := (m\mapsto m+n)$ which
is monotone. By the universal property of pushouts, we obtain a functor $\sigma^n:\mathbb{S}\to\mathbb{S}$
given by
\begin{equation*}
\begin{tikzcd}[column sep=huge]
\Delta(\mathbb{N}) 
  \arrow[r,"{n\mapsto(n,s)}"] 
  \arrow[d,swap,"i"] 
  & 
\Delta(\mathbb{N})\times\mathbf{2} 
  \arrow[d]
  \arrow[r,"{\Delta(S^n)\times\catid{\mathbf{2}}}"]
  &
\Delta(\mathbb{N})\times\mathbf{2}
  \arrow[dd]
  \\
(\mathbb{N},\leq) 
  \arrow[r] 
  \arrow[d,swap,"{(S^n,\leq)}"]
  & 
\mathbb{S}
  \arrow[dr,densely dotted,yshift=.5ex,"\sigma^n"]
  \\
(\mathbb{N},\leq)
  \arrow[rr] & &
\mathbb{S}
\end{tikzcd}
\end{equation*}

\begin{defn}
A non-unital pre-B0-system is a presheaf $B$ on $\mathbb{S}$ such that $B(0)$ is
a singleton, and the following additional structure:
\begin{enumerate}
\item For any $n\in\mathbb{N}$ and $X\in B(n)$, write $B/X$ for the sub-presheaf
of $B\circ\sigma^n$ given by $B/X(0)=\{X\}$ and $B/X(n+1)=B_{n\leq n+1}^{-1}(B/X(n))$. 
We require that for any $n\in\mathbb{N}$ and any $A\in B(n+1)$, there is a 
natural transformation $W_A:B/(B_{n\leq n+1}(A))\Rightarrow B/A$.
\end{enumerate}
\end{defn}
\end{comment}

\subsection{From categories with families and conceptual categories to E-systems}

In this section we recall Dybjer's notion of categories with families and we
introduce the equivalent notion of D-systems. After having proved that categories with families
are the same things as D-systems, we will focus on constructing a functor from
D-systems to E-systems without empty context. However, due to the existence of pathological examples
of categories with families, this functor is not going to be faithful. We then
introduce the notion of well-founded D-systems, and show that the
functor from D-systems to E'-systems without empty context restricts to an
embedding from the well-founded D-systems to E'-systems. 

\begin{defn}
A \emph{category with families} consists of
\begin{enumerate}
\item A category $\cat{C}$ with a terminal object $1$.
\item A presheaf $\mathrm{Ty}:\op{\cat{C}}\to\mathbf{Set}$. We will write
$f_\ast(A)$ for $\mathrm{Ty}(f,A)$.
\item For any $A\in\mathrm{Ty}(\Gamma)$, a morphism
\begin{equation*}
p_A : \ctxext{\Gamma}{A}\twoheadrightarrow \Gamma
\end{equation*}
in $\cat{C}$.
\item a preasheaf $\mathrm{Tm}:\op{(\int_\cat{C}\mathrm{Ty})}\to\mathbf{Set}$.
The action of a morphism $f:(\Delta,f_\ast(A))\to(\Gamma,A)$ of $\int_\cat{C}\mathrm{Ty}$
on a term $x\in \mathrm{Tm}(A)$ is also written by $f_\ast(x)$.
\item a term $\delta_A\in\mathrm{Tm}(\ctxext{\Gamma}{A},(p_A)_\ast(A))$ for every
$A\in\mathrm{Ty}(\Gamma)$, 
\end{enumerate}
with the universal property that for every $f:\Delta\to\Gamma$ in $\cat{C}$, 
the function
\begin{equation*}
\theta\mapsto\theta_\ast(\delta_A)
  : \{\theta:\Delta\to\ctxext{\Gamma}{A}\mid p_A\circ\theta\jdeq f\}\to
    \mathrm{Tm}(f_\ast(A))
\end{equation*}
is a bijection. We denote the inverse of this map by $\mu_{A,f}$. 
\end{defn}

There are of course many more sensible examples of categories with families,
but the following class of examples causes us to doubt whether it is possible
at all to have a faithful functor from the category of categories
with families to the category of E-systems. 

\begin{eg}
Every category $\cat{C}$ with a terminal object and the empty presheaf 
$\mathrm{Ty}$, is a category with families.
\end{eg}

\begin{rmk}
Although a terminal object is usually assumed to exist, not every object 
of a D-system is assumed to be `fibrant', i.e.~it is
not the case that every $X\to 1$ is a composition of projections $p_A$.
We could only hope that there is a functor from D-sytems to E-systems without
terminal objects.
\end{rmk}

There is essentially only one choice for the presheaf $\mathrm{Tm}$ and the
term $\delta_A$, so the extra structure on a setup for categories with families
is determined essentially uniquely once it exists. In other words, being a
category with families is a property on setups for categories with families.

\begin{defn}\label{lem:cwf_to_dsys}
Let $\mathbb{D}$ be a category with families. Then there is a natural isomorphism
\begin{equation*}
\varphi_A:\mathrm{Tm}(A)\cong\{\theta:\Gamma\to\ctxext{\Gamma}{A}\mid p_A\circ\theta\jdeq\catid{\Gamma}\}
\end{equation*}
in the presheaf category $\mathrm{Psh}(\int_{\cat{C}}\mathrm{Ty})$. 
\begin{comment}
The
natural isomorphism $\varphi$ is such that the term $\delta_A\in\mathrm{Tm}((p_A)_\ast(A))$
corresponds to the unique arrow $\tilde{q}_A$ which fits in the diagram
\begin{equation*}
\begin{tikzcd}
\ctxext{\Gamma}{A} \arrow[ddr,bend right=15,equals] \arrow[drr,bend left=15,equals] \arrow[dr,"{\tilde \delta_A}"] \\
& \ctxext{{\Gamma}{A}}{(p_A)_\ast(A)} \arrow[d,fib] & \ctxext{\Gamma}{A} \arrow[d,fib] \\
& \ctxext{\Gamma}{A} \arrow[r,"{p_A}"] & \Gamma
\end{tikzcd}
\end{equation*}
\end{comment}
\end{defn}

\begin{constr}
We define the inverse of $\varphi_A$ by $\theta\mapsto\theta_\ast(\delta_A)$. Verifying
that these form a natural isomorphism is straightforward.
\end{constr}

\begin{defn}
A \emph{D-system} consists of
\begin{enumerate}
\item A category $\cat{C}$ with a terminal object $1$.
\item A presheaf $\mathrm{Ty}:\op{\cat{C}}\to\mathbf{Set}$. 
\item A commuting triangle
\begin{equation*}
\begin{tikzcd}
\int_{\cat{C}}\mathrm{Ty} \arrow[rr,"p"] \arrow[dr,"\pi"] & & \cat{C}^\square \arrow[dl,"\mathrm{cod}"] \\
& \cat{C}
\end{tikzcd}
\end{equation*}
of functors, where $\cat{C}^\square$ is the sub-pre-category of $\cat{C}^{\to}$ of
which the morphisms are pullback squares. For any $A\in\mathrm{Ty}(\Gamma)$,
the morphism $p(\Gamma,A)$ will be denoted by $p_A$; for any
$(f,A):(\Delta,f_\ast(A))\to(\Gamma,A)$ in $\int_{\cat{C}}\mathrm{Ty}$, the morphism
$p(f,A)$ will be denoted by $\pi_2(f,A)$.
\end{enumerate}
\end{defn}

\begin{rmk}
For any $f:\Delta\to\Gamma$, $g:X\to\Delta$ and $A\in\mathrm{Ty}(\Gamma)$ in a
D-system $\mathbb{D}$, one has 
$\pi_2(\catid{\Gamma},A)\jdeq\catid{\ctxext{\Gamma}{A}}$ and
$\pi_2(f\circ g,A)\jdeq \pi_2(f,A)\circ\pi_2(g,f_\ast(A))$.
\end{rmk}

\begin{defn}
A \emph{D-homomorphism from $\mathbb{D}\to\mathbb{D}'$} consists of a functor 
$\mathcal{F}:\cat{C}\to\cat{C}'$ and a natural transformation
$\mathrm{Ty}\Rightarrow\mathrm{Ty}'\circ\mathcal{F}$, such that 
$\mathcal{F}(p_A)\jdeq p_{\mathcal{F}(A)}$ for any
$A\in\mathrm{Ty}(\Gamma)$.
\end{defn}

\begin{lem}
In the definition of D-system, the assumed structure beyond a choice of
projections $p_A:\ctxext{\Gamma}{A}\twoheadrightarrow\Gamma$ is property-like.
\end{lem}

\begin{thm}\label{thm:DD}
A category with families is the same thing as a D-system.
\end{thm}

\begin{proof}
Let $\mathbf{CWF}$ be a category with families. We show that $\mathbf{CWF}$ is
a D-system in five steps.
\begin{enumerate}
\item We first show that
\begingroup\it for every $f:\Delta\to\Gamma$ and $A\in\mathrm{Ty}(\Gamma)$, there is an
isomorphism
\begin{equation*}
\zeta_{A,f}:\{\theta:\Delta\to\ctxext{\Gamma}{A}\mid p_A\circ\theta\jdeq f\}
  \cong
\{\eta:\Delta\to\ctxext{\Delta}{f_\ast(A)}\mid p_{f_\ast(A)}\circ\eta\jdeq\catid{\Delta}\}
\end{equation*}
natural in $f$ in the sense that the equalities
\begin{align*}
\zeta_{A,\catid{\Gamma}}(\theta) & \jdeq \theta \\
\zeta_{A,f\circ g}(\theta\circ g) & \jdeq \zeta_{f_\ast(A),g}(\zeta_{A,f}(\theta)\circ g)
\end{align*}
hold for any $f:\Delta\to\Gamma$ and $g:X\to\Delta$, and any $\theta:\Delta\to\ctxext{\Gamma}{A}$
such that $p_A\circ\theta\jdeq f$.

Furthermore, we get $\zeta_{A,p_A}(\catid{\ctxext{\Gamma}{A}})\jdeq\varphi_{(p_A)_\ast(A)}(\delta_A)$%
\endgroup
\medskip

\renewcommand\qedsymbol{\(\blacksquare\)}
By \autoref{lem:cwf_to_dsys} we may define
\begin{equation*}
\zeta_{A,f}(\theta)\defeq \varphi_{f_\ast(A)}(\theta_\ast(\delta_A))
\end{equation*}
for any $\theta\in\{\theta:\Delta\to\ctxext{\Gamma}{A}\mid p_A\circ\theta\jdeq f\}$. 
Since $\zeta_{A,f}$ is a composition of isomorphism, it is itself an isomorphism.
Also, we get immediately that $\zeta_{A,\catid{\Gamma}}(\theta)\jdeq\theta$. To
show the remaining identity:
\begin{align*}
\zeta_{A,f\circ g}(\theta\circ g)
& \jdeq \varphi_{g_\ast(f_\ast(A))}((\theta\circ g)_\ast(\delta_A)) \\
& \jdeq \varphi_{g_\ast(f_\ast(A))}(g_\ast(\theta_\ast(\delta_A))) \\
& \jdeq \varphi_{g_\ast(f_\ast(A))}(g_\ast(\varphi_{f_\ast(A)}(\theta_\ast(\delta_A))_\ast(\delta_A))) \\
& \jdeq \varphi_{g_\ast(f_\ast(A))}(g_\ast(\zeta_{A,f}(\theta)_\ast(\delta_A))) \\
& \jdeq \varphi_{g_\ast(f_\ast(A))}((\zeta_{A,f}(\theta)\circ g)_\ast(\delta_A)) \\
& \jdeq \zeta_{f_\ast(A),g}(\zeta_{A,f}(\theta)\circ g).
\end{align*}
This completes the proof that the isomorphisms $\zeta_{A,f}$ satisfy the required
equalities.

\item
For any $A\in\mathrm{Ty}(\Gamma)$, we define
$W_A\defeq A_\ast:\cat{F}/\Gamma\to\cat{F}/\ctxext{\Gamma}{A}$. Then we get
for any $A,B\in\cat{F}/\Gamma$ an isomorphism
\begin{equation*}
\mathrm{hom}_{\cat{C}/\Gamma}(A,B)
  \cong
T(W_A(B)).
\end{equation*}
In particular, we get a term $\idtm{A}\in T((p_A)_\ast(A))$. 
\medskip

Let $A,B\in\cat{F}/\Gamma$. Then we have the isomorphism
\begin{align*}
T(W_A(B)) & \jdeq \{\eta:\ctxext{\Gamma}{A}\to\ctxext{{\Gamma}{A}}{W_A(B)}\mid W_A(B)\circ\eta\jdeq\catid{\ctxext{\Gamma}{A}}\} \\
  & \cong
\{\theta:\ctxext{\Gamma}{A}\to\ctxext{\Gamma}{B}\mid B\circ\theta\jdeq A\},
\end{align*}
so we find $\idtm{A}\defeq\zeta_{A,A}(\catid{\ctxext{\Gamma}{A}})\in T(W_A(A))$.

\item Now we show that 
\begingroup\it for any $A\in\mathrm{Ty}(\Gamma)$ and $f:\Delta\to\Gamma$ 
there is a morphism $\pi_2(f,A):\ctxext{\Delta}{f_\ast(A)}\to\ctxext{\Gamma}{A}$
such that $p_A\circ \pi_2(f,A)\jdeq f\circ p_{f_\ast(A)}$.
\endgroup
\medskip

Note that we have the isomorphism
\begin{align*}
& \{ \theta : \ctxext{\Delta}{f_\ast(A)}\to\ctxext{\Gamma}{A}\mid p_A\circ\theta\jdeq f\circ p_{f_\ast(A)} \} \\
  & \cong
\{ \eta : \ctxext{\Delta}{f_\ast(A)}\to\ctxext{{\Delta}{f_\ast(A)}}{(f\circ p_{f_\ast(A)})_\ast(A)}
  \mid (f\circ p_{f_\ast(A)})_\ast(A)\circ\eta\jdeq\catid{\ctxext{\Delta}{f_\ast(A)}}\} \\
  & \jdeq
\{ \eta : \ctxext{\Delta}{f_\ast(A)}\to\ctxext{{\Delta}{f_\ast(A)}}{(p_{f_\ast(A)})_\ast(f_\ast(A))}
  \mid (f\circ p_{f_\ast(A)})_\ast(A)\circ\eta\jdeq\catid{\ctxext{\Delta}{f_\ast(A)}}\} \\
  & \jdeq
T((p_{f_\ast(A)})_\ast(f_\ast(A))).
\end{align*}
Since we have $\delta_{f_\ast(A)}\in T((p_{f_\ast(A)})_\ast(f_\ast(A)))$, we obtain 
\begin{equation*}
\pi_2(f,A)\defeq \zeta_{A,f\circ p_{f_\ast(A)}}^{-1}(\delta_{f_\ast(A)}):\ctxext{\Delta}{f_\ast(A)}\to\ctxext{\Gamma}{A}
\end{equation*}
satisfying $p_A\circ\pi_2(f,A)\jdeq f\circ p_{f_\ast(A)}$.
\item Next, we show that
\begingroup\it
for any $f:\Delta\to\Gamma$, $g:X\to\Delta$ and $A\in\mathrm{Ty}(\Gamma)$, one has
\begin{equation*}
\pi_2(f,A)\circ\pi_2(g,f_\ast(A))\jdeq\pi_2(f\circ g,A).
\end{equation*}
\endgroup
\medskip

It suffices to show that
\begin{equation*}
\zeta_{A,f\circ g\circ p_{g_\ast(f_\ast(A))}}(\pi_2(f,A)\circ\pi_2(g,f_\ast(A)))\jdeq\delta_{g_\ast(f_\ast(A))}.
\end{equation*}
Because $g\circ p_{g_\ast(f_\ast(A))}\jdeq p_{f_\ast(A)}\circ\pi_2(g,f_\ast(A))$, we get
to use the naturality of $\zeta$:
\begin{align*}
& \zeta_{A,f\circ g\circ p_{g_\ast(f_\ast(A))}}(\pi_2(f,A)\circ\pi_2(g,f_\ast(A))) \\
& \jdeq
\zeta_{A,f\circ p_{f_\ast(A)}\circ\pi_2(g,f_\ast(A))}(\pi_2(f,A)\circ\pi_2(g,f_\ast(A))) \\
& \jdeq
\zeta_{(p_{f_\ast(A)})_\ast(f_\ast(A)),\pi_2(g,f_\ast(A))}(\zeta_{A,f\circ p_{f_\ast(A)}}(\pi_2(f,a))\circ\pi_2(g,f_\ast(A))) \\
& \jdeq
\zeta_{(p_{f_\ast(A)})_\ast(f_\ast(A)),\pi_2(g,f_\ast(A))}(\delta_{f_\ast(A)}\circ\pi_2(g,f_\ast(A))) \\
& \jdeq
\zeta_{(p_{f_\ast(A)})_\ast(f_\ast(A)),\pi_2(g,f_\ast(A))}(\zeta_{f_\ast(A),p_{f_\ast(A)}}(\catid{\ctxext{\Delta}{f_\ast(A)}})\circ\pi_2(g,f_\ast(A))) \\
& \jdeq
\zeta_{f_\ast(A),p_{f_\ast(A)}\circ\pi_2(g,f_\ast(A))}(\catid{\ctxext{\Delta}{f_\ast(A)}}\circ\pi_2(g,f_\ast(A))) \\
& \jdeq
\zeta_{f_\ast(A),p_{f_\ast(A)}\circ\pi_2(g,f_\ast(A))}(\pi_2(g,f_\ast(A))) \\
& \jdeq
\zeta_{f_\ast(A),g\circ p_{g_\ast(f_\ast(A))}}(\pi_2(g,f_\ast(A)))\\
& \jdeq
\delta_{g_\ast(f_\ast(A))}.
\end{align*}
\item
\begingroup\it
Let $f:\Delta\to\Gamma$ and $A\in\mathrm{Ty}(\Gamma)$. Then one has 
\begin{equation*}
\pi_2(f,A)\circ\zeta_{A,f}(\theta)\jdeq\theta
\end{equation*}
for any $\theta:\Delta\to\ctxext{\Gamma}{A}$ satisfying $A\circ\theta\jdeq f$.
\endgroup
\medskip

We will write $\tilde\theta\defeq\zeta_{A,f}(\theta)$. Note that
\begin{equation*}
p_A\circ\pi_2(f,A)\circ\tilde\theta\jdeq f\circ p_{f_\ast(A)}\circ\tilde\theta\jdeq f.
\end{equation*} 
Therefore, we can prove that $\pi_2(f,A)\circ\tilde\theta\jdeq\theta$ by showing
that $\zeta_{A,f}(\pi_2(f,A)\circ\tilde\theta)\jdeq\tilde\theta$. This is
shown using the naturality of $\zeta$:
\begin{align*}
\zeta_{A,f}(\pi_2(f,A)\circ\tilde\theta)
  & \jdeq
\zeta_{A,f\circ p_{f_\ast(A)}\circ\tilde\theta}(\pi_2(f,A)\circ\tilde\theta)
  \\
  & \jdeq
\zeta_{(p_{f_\ast(A)})_\ast(f_\ast(A)),\tilde{\theta}}(\zeta_{A,f\circ p_{f_\ast(A)}}(\pi_2(f,A))\circ\tilde\theta)
  \\
  & \jdeq
\zeta_{(p_{f_\ast(A)})_\ast(f_\ast(A)),\tilde{\theta}}(\idtm{f_\ast(A)}\circ\tilde\theta)
  \\
  & \jdeq
\zeta_{(p_{f_\ast(A)})_\ast(f_\ast(A)),\tilde{\theta}}(\zeta_{f_\ast(A),p_{f_\ast(A)}}(\catid{\ctxext{\Delta}{f_\ast(A)}})\circ\tilde\theta)
  \\
  & \jdeq
\zeta_{f_\ast(A),p_{f_\ast(A)}\circ\tilde\theta}(\catid{\ctxext{\Delta}{f_\ast(A)}}\circ\tilde\theta) \\
  & \jdeq
\zeta_{f_\ast(A),\catid{\Delta}}(\tilde\theta) \\
  & \jdeq 
\theta.
\end{align*}
\item 
\begingroup\it
The diagram
\begin{equation*}
\begin{tikzcd}[column sep=large]
\ctxext{\Delta}{f_\ast(A)} \arrow[d,fib] \arrow[r,"{\pi_2(f,A)}"] & \ctxext{\Gamma}{A} \arrow[d,fib] \\
\Delta \arrow[r,"f"] & \Gamma
\end{tikzcd}
\end{equation*}
is a pullback square.
\endgroup
\medskip

To verify the universal mapping property of the pullback square, consider a
commuting diagram
\begin{equation*}
\begin{tikzcd}
X \arrow[d,swap,"g"] \arrow[r,"h"] & \ctxext{\Gamma}{A} \arrow[d,fib] \\
\Delta \arrow[r,"f"] & \Gamma
\end{tikzcd}
\end{equation*}
Then $h$ corresponds uniquely to a section $k$ of $(f\circ g)_\ast(A)$. Since
$(f\circ g)_\ast(A)\jdeq g_\ast(f_\ast(A))$, the section $k$ fits in the
commuting diagram
\begin{equation*}
\begin{tikzcd}[column sep=huge]
\ctxext{X}{g_\ast(f_\ast(A))} \arrow[d,xshift=.7ex,fib] \arrow[r,"{\pi_2(g,f_\ast(A))}"]
& \ctxext{\Delta}{f_\ast(A)} \arrow[d,fib] \arrow[r,"{\pi_2(f,A)}"]
& \ctxext{\Gamma}{A} \arrow[d,fib] \\
X \arrow[r,swap,"g"] \arrow[u,xshift=-.7ex,"k"] & \Delta \arrow[r,swap,"f"] & \Gamma
\end{tikzcd}
\end{equation*}
so we obtain $u\defeq\pi_2(g,f_\ast(A))\circ k:X\to\ctxext{\Delta}{f_\ast(A)}$. 
Then we have $p_{f_\ast(A)}\circ u\jdeq g\circ p_{g_\ast(f_\ast(A))}\circ k\jdeq g$ and
$\pi_2(f,A)\circ u\jdeq \pi_2(f\circ g,A)\circ k\jdeq h$. Note that
$u$ corresponds uniquely to $k$, which corresponds uniquely to $h$, so the
uniqueness of $u$ is automatic.
\end{enumerate}
Now let $\mathbb{D}$ be a D-system. We define the presheaf $\mathrm{Tm}:
\op{(\int_{\cat{C}}\mathrm{Ty})}\to\mathbf{Set}$ by
\begin{equation*}
\mathrm{Tm}(A) \defeq \{x:\Gamma\to\ctxext{\Gamma}{A}\mid p_A\circ x\jdeq \catid{\Gamma}\}.
\end{equation*}
For $f:\Delta\to\Gamma$ and $A\in\mathrm{Ty}(\Gamma)$, we get a function
$x\mapsto f_\ast(x):\mathrm{Tm}(A)\to\mathrm{Tm}(f_\ast(A))$ by defining $f_\ast(x)$
to be the unique morphism such that the diagram
\begin{equation*}
\begin{tikzcd}[column sep=huge] 
\Delta \arrow[ddr,bend right=15,equals] \arrow[drr,bend left=15,"x\circ f"] \arrow[dr,densely dotted,near end,"{f_\ast(x)}"] \\
& \ctxext{\Delta}{f_\ast(A)} \arrow[d,fib] \arrow[r,swap,"{\pi_2(f,A)}"] & \ctxext{\Gamma}{A} \arrow[d,fib] \\
& \Delta \arrow[r,swap,"f"] & \Gamma
\end{tikzcd}
\end{equation*}
commutes. A combination of the assumed uniqueness and the assumed equalities on
$\pi_2(\blank,\blank)$ gives the functoriality of $\mathrm{Tm}$. 

The term $\delta_A\in\mathrm{Tm}((p_A)_\ast(A))$ is defined to be the unique morphism
such that the diagram
\begin{equation*}
\begin{tikzcd}[column sep=large]
\ctxext{\Gamma}{A} \arrow[drr,bend left=15,equals] \arrow[ddr,bend right=15,equals] \arrow[dr,densely dotted,near end,"{\delta_{A}}"] \\
& \ctxext{{\Gamma}{A}}{(p_A)_\ast(A)} \arrow[r,swap,"{\pi_2(A,A)}"] \arrow[d,fib] & \ctxext{\Gamma}{A} \arrow[d,fib] \\
& \ctxext{\Gamma}{A} \arrow[r,swap,"A"] & \Gamma
\end{tikzcd}
\end{equation*}
commutes.

The universality assumption of categories with families is a consequence of
the universal property of pullbacks: the map $x\mapsto\pi_2(f,X)\circ x$
from $\{x:\Delta\to\ctxext{\Delta}{f_\ast(A)}\mid p_{f_\ast(A)}\circ x\jdeq\catid{\Delta}\}$
to $\{\theta:\Delta\to\ctxext{\Gamma}{A}\mid p_A\circ\theta\jdeq f\}$ is a bijection.
\end{proof}

\begin{defn}
Let $\mathbb{D}$ be a D-system with underlying category $\cat{C}$. We define
$\cat{F}\subseteq\cat{C}$ to be the subcategory with the same objects as
$\cat{C}$, where the morphisms are generated by the projections $p_A$. Note that
there is no reason why $\cat{F}$ should possess a terminal object.
\end{defn}

To be able to even construct an E'-system out of a D-system, we need to impose
a rather restricting conditon on the D-system.

\begin{defn}
Let $\mathbb{D}$ be a D-system. Then we define for any $\Gamma\in\mathbb{D}$ 
the category $\Lambda(\Gamma)$ of which the objects are finite sequences
of the form $(A_1,\ldots,A_n)$ where
\begin{equation*}
A_1\in\mathrm{Ty}(\Gamma),\ A_2\in\mathrm{Ty}(\ctxext{\Gamma}{A_1}),\ \ldots,\ 
A_n\in\mathrm{Ty}(\ctxext{{{\Gamma}{A_1}}{\cdots}}{A_{n-1}})
\end{equation*}
where the morphisms are generated by $(A_1,\ldots,A_{n+1})\leq (A_1,\ldots,A_{n})$.
\end{defn}

\begin{defn}
Let $\mathbb{D}$ be a D-system and let $f:\Delta\to\Gamma$ in $\mathbb{D}$.
Then there is a functor $\Lambda(f):\Lambda(\Gamma)\to\Lambda(\Delta)$, with
morphisms $\pi_2(f,A):\ctxext{\Delta}{p_{\Lambda(f,A)}}\to
\ctxext{\Gamma}{p_A}$ in $\mathbb{D}$, such that the square
\begin{equation*}
\begin{tikzcd}[column sep=large]
\ctxext{\Delta}{\Lambda(f,A)} \arrow[r,"{\pi_2(f,A)}"] \arrow[d,fib] & \ctxext{\Gamma}{A} \arrow[d,fib] \\
\Delta \arrow[r,swap,"f"] & \Gamma
\end{tikzcd}
\end{equation*}
is a pullback square.
\end{defn}

\begin{constr}
We construct the asserted structure by
induction on the length of the arrow $A$ into $\Gamma$. We define
$\Lambda(f,\emptyf[\Gamma])\defeq \emptyf[\Delta]$, and $\pi_2(f,\emptyf[\Gamma])\defeq f:\ctxext{\Delta}{\catid{\Delta}}\to\ctxext{\Gamma}{\catid{\Gamma}}$. The resulting square is trivially
a pullback square. Now suppose we have a pullback square as above.
and let $B\in\mathrm{Ty}(\ctxext{\Gamma}{A})$. Then we define
\begin{equation*}
\Lambda(f,[A,B])\defeq [\Lambda(f,A),\pi_2(f,A)_\ast(B)].
\end{equation*}
By \autoref{thm:DD}, the top square in the diagram
\begin{equation*}
\begin{tikzcd}[column sep=6em]
\ctxext{{\Delta}{\Lambda(f,A)}}{(\pi_2(f,A)_\ast(B))} \arrow[d,fib] \arrow[r,"{\pi_2(\pi_2(f,A),B)}"] & \ctxext{{\Gamma}{A}}{B} \arrow[d,fib] \\
\ctxext{\Delta}{\Lambda(f,A)} \arrow[d,fib] \arrow[r,"{\pi_2(f,A)}"] & \ctxext{\Gamma}{A} \arrow[d,fib] \\
\Delta \arrow[r,"f"] & \Gamma
\end{tikzcd}
\end{equation*}
is a pullback square.
Since the bottom square is assumed to be a pullback, we get that the outer rectangle
is a pullback, as desired. 
\end{constr}

\begin{rmk}
The construction above does not help us define a functor $f_\ast:\cat{F}/\Gamma
\to\cat{F}/\Delta$ without further assumptions on the D-system $\mathbb{D}$.
The functor $f_\ast$ would have the property that the square
\begin{equation*}
\begin{tikzcd}
\Lambda(\Gamma) \arrow[d,swap,"p"] \arrow[r,"\Lambda(f)"] & \Lambda(\Delta) \arrow[d,"p"] \\
\cat{F}/\Gamma \arrow[r,densely dotted,"{f_\ast}"] & \cat{F}/\Delta
\end{tikzcd}
\end{equation*}
commutes. The above square can only commute, if one has
\begin{equation*}
\Lambda(f)/A\jdeq\Lambda(f)/A'
\end{equation*} 
for any $A,A'\in\Lambda(\Gamma)$ with $p_A\jdeq p_{A'}$. This would entail that
$\pi_2(f,A)\jdeq\pi_2(f,A')$, a property that seems to not hold generally.
\end{rmk}

\begin{defn}
A D-system is said to be coherent if for any $f:\Delta\to\Gamma$ and any
$A,A'\in\Lambda(\Gamma)$ satisfying $p_A\jdeq p_{A'}$, one has
\begin{equation*}
\pi_2(f,A)\jdeq \pi_2(f,A').
\end{equation*}
\end{defn}

\begin{defn}
We say that a D-system $\cat{D}$ is \emph{locally well-founded} if the obvious functor
$p:\Lambda(\mathbb{D})\to\cat{F}/\Gamma$ is an isomorphism of pre-categories.
\end{defn}

\begin{rmk}
In other words, a D-system is locally well-founded if and only if
for any $A\in\cat{F}$ there is a unique sequence
$A_1\in\mathrm{Ty}(\Gamma)$, $A_2\in\mathrm{Ty}(\ctxext{\Gamma}{A_1})$, \ldots,
$A_n\in\mathrm{Ty}(\ctxext{{{\Gamma}{A_1}}{\cdots}}{A_{n-1}})$ such that
$A\jdeq p_{A_1}\circ\cdots\circ p_{A_n}$. Locally well-founded D-systems 
are examples of coherent D-systems.
\end{rmk}

\begin{defn}
For any locally well-founded D-system $\mathbb{D}$ there is 
an E'-system $U(\mathbb{D})$ without terminal object in 
$\cat{F}$. Thus, we get a functor $\mathbf{Dsys}\to\mathbf{E'sys}_{\circ}$
\end{defn}

\begin{constr}
Let $f:\Delta\to\Gamma$. By the assumption that $\mathbb{D}$ is locally well-founded,


To verify that this gives indeed a functor from $\cat{F}/\Gamma$ to
$\cat{F}/\Delta$, we need to show that for any two objects or $A,A'\in\Lambda(\Gamma)$
which get mapped to the same object of $\cat{F}/\Gamma$, get mapped to the
same object of $\cat{F}/\Delta$. Let $A,A'\in\Lambda(\Gamma)$ be such that
$p_A\jdeq p_{A'}$. It suffices to show that $\pi_2(f,A)\jdeq\pi_2(f,A')$

The inductive proofs to verify properties
\autoref{defn:E'sys:b} and \autoref{defn:E'sys:c} of \autoref{defn:E'sys} are
completely straightforward, so $U(\mathbb{D})$ is an E'-system without terminal
object in $\cat{F}$.
\begin{comment}
Finally, we need to verify that the structure we've defined satisfies the properties
of being an E'-system without terminal object. We go through them systematically,
following the same order as in \autoref{defn:E'sys}:
\begin{enumerate}[label=(\alph*)]
\item This property holds by definition.
\item Let $\Gamma\in\cat{C}$. We will prove by induction on $A\in\cat{F}/\Gamma$
that $(\catid{\Gamma})_\ast(A)\jdeq A$ and that $\pi_2(\catid{\Gamma},A)\jdeq \catid{\ctxext{\Gamma}{A}}$.
Note that these properties hold by definition for $A\jdeq\catid{\Gamma}$. For the
inductive step, assume that $A\in\cat{F}/\Gamma$ is such that
$(\catid{\Gamma})_\ast(A)\jdeq A$ and $\pi_2(\catid{\Gamma},A)\jdeq
\catid{\ctxext{\Gamma}{A}}$ and let $B\in\mathrm{Ty}(\ctxext{\Gamma}{A})$. Then
we have
\begin{align*}
(\catid{\Gamma})_\ast(A\circ p_B) & \jdeq (\catid{\Gamma})_\ast(A)\circ \pi_2(\catid{\Gamma},A)_\ast(p_B) \\
& \jdeq A\circ p_B
\end{align*}
Also, we have
\begin{align*}
\pi_2(\catid{\Gamma},A\circ p_B) & \jdeq \pi_2(\pi_2(\catid{\Gamma},A),p_B) \\
& \jdeq \pi_2(\catid{\ctxext{\Gamma}{A}},p_B) \\
& \jdeq \catid{\ctxext{{\Gamma}{A}}{B}}
\end{align*}
\item Let $f:\Delta\to \Gamma$ and $g:X\to \Delta$. We will prove by induction
on $A\in\cat{F}/\Gamma$ that $(f\circ g)_\ast(A)\jdeq g_\ast(f_\ast(A))$ and
that $\pi_2(f\circ g,A)\jdeq \pi_2(f,A)\circ \pi_2(g,f_\ast(A))$. Note that
these properties hold by definition for $A\jdeq\catid{\Gamma}$. For the inductive
step, assume that $A\in\cat{F}/\Gamma$ is such that $(f\circ g)_\ast(A)\jdeq g_\ast(f_\ast(A))$ and
$\pi_2(f\circ g,A)\jdeq \pi_2(f,A)\circ \pi_2(g,f_\ast(A))$, and let
$B\in\mathrm{Ty}(\ctxext{\Gamma}{A})$. Then we have
\begin{align*}
(f\circ g)_\ast(A\circ p_B) & \jdeq \pi_2(f\circ g,A)_\ast(p_B) \\
& \jdeq (\pi_2(f,A)\circ\pi_2(g,f_\ast(A)))_\ast(p_B) \\
& \jdeq \pi_2(g,f_\ast)_\ast(\pi_2(f,A)_\ast(p_B)) \\
& \jdeq g_\ast(f_\ast(A\circ p_B))
\end{align*}

\item This property holds by definition.
\end{enumerate}
\end{comment}
\end{constr}

\begin{thm}
The functor $U:\mathbf{Dsys}\to\mathbf{E'sys_{\circ}}$ is full and faithful.
\end{thm}

\begin{proof}
The functor $U$ is faithful, because for any morphism 
$\mathcal{F}:\mathbb{D}\to\mathbb{D}'$ of locally well-founded D-systems, 
$U(\mathcal{F})$ and $\mathcal{F}$ have the same underlying functor
from $\cat{C}$ to $\cat{C}'$. 

It remains to show that $U$ is full. Let
$\mathbb{D}$ and $\mathbb{D}$ be D-systems and let $\mathcal{G}:U(\mathbb{D})
\to U(\mathbb{D}')$. Then $\mathcal{G}$ restricted to $\cat{F}$ lifts factorizations.
We  first have to define a natural transformation
$\mathrm{Ty}\Rightarrow \mathrm{Ty}'\circ\mathcal{G}$. Let
$\Gamma\in\cat{C}$ and $A\in\mathrm{Ty}(\Gamma)$. Then we have
$\mathcal{G}(p_A):\mathcal{G}(\ctxext{\Gamma}{A})\to \mathcal{G}(\Gamma)$.
The category of factorizations of $p_A$ in $\cat{F}$ is a isomorphic to $\mathbf{2}$,
so the category of factorizations of $\mathcal{G}(p_A)$ is also isomorphic to
$\mathbf{2}$. This implies that $\mathcal{G}(p_A)$ is of the form $p_B$ for
some $B\in\mathrm{Ty}'(\mathcal{G}(\Gamma))$. This $B$ is called $\mathcal{G}(A)$.

To verify the naturality, let $f:\Delta\to\Gamma$. We need to show that the
square
\begin{equation*}
\begin{tikzcd}
\mathrm{Ty}(\Delta) \arrow[r,"{\mathcal{G}_\Delta}"] & \mathrm{Ty}'(\mathcal{G}(\Delta)) \\
\mathrm{Ty}(\Gamma) \arrow[u,"\mathrm{Ty}(f)"] \arrow[r,"{\mathcal{G}_\Gamma}"] &
\mathrm{Ty}'(\mathcal{G}(\Gamma)) \arrow[u,swap,"{\mathrm{Ty}'(\mathcal{G}(f))}"]
\end{tikzcd}
\end{equation*}
commutes. Let $A\in\mathrm{Ty}(\Gamma)$. Then we need to show that
$\mathcal{G}(f)_\ast(\mathcal{G}(A))\jdeq \mathcal{G}(f_\ast(A))$. By the uniqueness
assumption of well-founded D-systems, it suffices to show that
$p_{\mathcal{G}(f)_\ast(\mathcal{G}(A))}\jdeq p_{\mathcal{G}(f_\ast(A))}$. To
see this, we do a simple calculation
\begin{align*}
p_{\mathcal{G}(f)_\ast(\mathcal{G}(A))}
& \jdeq \mathcal{G}(f)_\ast(p_{\mathcal{G}(A)}) \\
& \jdeq \mathcal{G}(f)_\ast(\mathcal{G}(p_A)) \\
& \jdeq \mathcal{G}(f_\ast(p_A)) \\
& \jdeq \mathcal{G}(p_{f_\ast(A)}) \\
& \jdeq p_{\mathcal{G}(f_\ast(A))} \qedhere
\end{align*}
\end{proof}

\begin{thm}
The image of $U$ is the subcategory of locally stratified E'-systems without
empty context. 
\end{thm}

\begin{proof}
For any $\Gamma\in\mathbb{D}$, there is an obvious stratification functor
$\Lambda(\Gamma)\to (\N,\geq)$, so $U(\mathbb{D})$ is a locally stratified
E'-system. 

Now let $\mathbb{E}$ be a locally stratified E'-system without empty context.
For $\Gamma\in\cat{C}$, we define $\mathrm{Ty}(\Gamma)$ to be the set of
families of length one. Since any $f_\ast$ is stratified, it follows that
$\mathrm{Ty}$ is a presheaf on $\cat{C}$. The choice of pullback squares
\begin{equation*}
\begin{tikzcd}[column sep=large]
\ctxext{\Delta}{f_\ast(A)} \arrow[r,"{\pi_2(f,A)}"] \arrow[d,fib] & \ctxext{\Gamma}{A} \arrow[d,fib] \\
\Delta \arrow[r,swap,"f"] & \Gamma
\end{tikzcd}
\end{equation*}
is functorial by assumption, so we have defined a category with families. It
is straigtforward to verify that this category with families is locally well-founded,
and that its construction is inverse to $U$. 
\end{proof}

\begin{comment}
\subsection{The equivalence of E-systems and E'-systems}

The definition of D-system suggests a more general structure related to E-systems. 

\begin{defn}
An \emph{E'-system} consists of a category $\cat{C}$ with a subcategory $\cat{F}$ with the
same objects as $\cat{C}$, with
the following structure
\begin{enumerate}
\item $\cat{F}$ has a terminal object.
\item For any $f:\Delta\to\Gamma$ in $\cat{C}$, a functor $f_\ast:\cat{F}/\Gamma\to
\cat{F}/\Delta$ %and
%for any $Q\in\cat{F}/\ctxext{\Gamma}{B}$ and any section $g$ of $Q$ a section
%$f_\ast(g)$ of $f_\ast(Q)$, 
so that $f_\ast$ is contravariant in $f$ and preserves the terminal object. 
\item For any $A\in\cat{F}/\Gamma$ and $f:\Delta\to\Gamma$, lifts of
$f$ to $A$ are naturally the same thing as sections of
$f_\ast(A)$, i.e. there is an isomorphism
\begin{equation*}
\zeta_{A,f}%\defeq\theta\mapsto f_\ast(\theta)
:\{\theta:\Delta\to\ctxext{\Gamma}{A}\mid A\circ\theta\jdeq f\}
  \cong
\{\eta:\Delta\to\ctxext{\Delta}{(f_\ast(A))}\mid f_\ast(A)\circ\eta\jdeq\catid{\Delta}\}
\end{equation*}
which are natural in $A$ and $f$, in the sense that
\begin{enumerate}
\item For any $P\in\cat{F}/\ctxext{\Gamma}{A}$, $f:\Delta\to\Gamma$ and
$\theta:\Delta\to\ctxext{{\Gamma}{A}}{P}$ such that $(\ctxext{A}{P})\circ\theta\jdeq f$,
one hase
\begin{equation*}
\zeta_{A,f}(P\circ\theta)\jdeq f_\ast(P)\circ \zeta_{\ctxext{A}{P},f}(\theta).
\end{equation*}
\item For any $A\in\cat{F}/\Gamma$, $f:\Delta\to\Gamma$, $g:X\to\Delta$
and $\theta:\Delta\to\ctxext{\Gamma}{A}$ such that $A\circ\theta\jdeq f$, one
has
\begin{align*}
\zeta_{A,\catid{\Gamma}}(\theta) & \jdeq \theta \\
\zeta_{A,f\circ g}(\theta\circ g) & \jdeq\zeta_{f_\ast(A),g}(\zeta_{A,f}(\theta)\circ g).
\end{align*}
\end{enumerate}
\end{enumerate}
\end{defn}

We can give the category $\cat{F}$ a term structure as follows:

\begin{defn}
For any E'-system $\mathbb{E}$, the category $\cat{F}$ has a term structure
where the terms of $A$ are sections of $A$, i.e.
\begin{equation*}
T(A)\defeq\{\theta:\Gamma\to\ctxext{\Gamma}{A}\mid A\circ\theta\jdeq\catid{\Gamma}\}.
\end{equation*}
\end{defn}

\begin{lem}\label{lem:ptm}
Let $\mathbb{E}$ be an E'-system. Then there is a function $T(A)\to T(f_\ast(A))$
for any $f:\Delta\to\Gamma$ in $\cat{C}$ and $A\in\cat{F}/\Gamma$, contravariant
in $f$. 
\end{lem}

\begin{proof}
Using the natural isomorphism $\zeta_{A,f}$, we see that
\begin{align*}
T(f_\ast(A)) & \jdeq \{\eta : \Delta\to\ctxext{\Delta}{f_\ast(A)}\mid f_\ast(A)\circ\eta\jdeq\catid{\Delta}\}
  \\
  & \cong
\{\theta : \Delta\to \ctxext{\Gamma}{A}\mid A\circ\theta\jdeq f\}.
\end{align*}
Therefore, we have the function $x\mapsto \zeta_{A,f}(x\circ f)$ from $T(A)$ to
$T(f_\ast(A))$. 
\end{proof}

\begin{defn}
For any $A\in\cat{F}/\Gamma$, we define
$W_A\defeq A_\ast:\cat{F}/\Gamma\to\cat{F}/\ctxext{\Gamma}{A}$. Then we get
for any $A,B\in\cat{F}/\Gamma$ an isomorphism
\begin{equation*}
\mathrm{hom}_{\cat{C}/\Gamma}(A,B)
  \cong
T(W_A(B)).
\end{equation*}
In particular, we get a term $\idtm{A}\in T(W_A(A))$. 
\end{defn}

\begin{constr}
Let $A,B\in\cat{F}/\Gamma$. Then we have the isomorphism
\begin{align*}
T(W_A(B)) & \jdeq \{\eta:\ctxext{\Gamma}{A}\to\ctxext{{\Gamma}{A}}{W_A(B)}\mid W_A(B)\circ\eta\jdeq\catid{\ctxext{\Gamma}{A}}\} \\
  & \cong
\{\theta:\ctxext{\Gamma}{A}\to\ctxext{\Gamma}{B}\mid B\circ\theta\jdeq A\},
\end{align*}
so we find $\idtm{A}\defeq\zeta_{A,A}(\catid{\ctxext{\Gamma}{A}})\in T(W_A(A))$.
\end{constr}

\begin{defn}
For any $P\in\cat{F}/\ctxext{\Gamma}{A}$ and $f:\Delta\to\Gamma$ 
there is a morphism $\pi_2(f,A):\ctxext{\Delta}{f_\ast(A)}\to\ctxext{\Gamma}{A}$
such that $A\circ \pi_2(f,A)\jdeq f\circ f_\ast(A)$.
\end{defn}

\begin{constr}
\begin{align*}
& \{ \theta : \ctxext{\Delta}{f_\ast(A)}\to\ctxext{\Gamma}{A}\mid A\circ\theta\jdeq f\circ f_\ast(A) \} \\
  & \cong
\{ \eta : \ctxext{\Delta}{f_\ast(A)}\to\ctxext{{\Delta}{f_\ast(A)}}{(f\circ f_\ast(A))_\ast(A)}
  \mid (f\circ f_\ast(A))_\ast(A)\circ\eta\jdeq\catid{\ctxext{\Delta}{f_\ast(A)}}\} \\
  & \jdeq
\{ \eta : \ctxext{\Delta}{f_\ast(A)}\to\ctxext{{\Delta}{f_\ast(A)}}{(f_\ast(A))_\ast(f_\ast(A))}
  \mid (f\circ f_\ast(A))_\ast(A)\circ\eta\jdeq\catid{\ctxext{\Delta}{f_\ast(A)}}\} \\
  & \jdeq
T(W_{f_\ast(A)}(f_\ast(A))).
\end{align*}
Since we have $\tfid{f_\ast(A)}\in T(W_{f_\ast(A)}(f_\ast(A)))$, we obtain 
\begin{equation*}
\pi_2(f,A)\defeq \zeta_{A,f\circ f_\ast(A)}^{-1}(\tfid{f_\ast(A)}):\ctxext{\Delta}{f_\ast(A)}\to\ctxext{\Gamma}{A}
\end{equation*}
satisfying $A\circ\pi_2(f,A)\jdeq f\circ f_\ast(A)$.
\end{constr}

\begin{lem}
For any $f:\Delta\to\Gamma$, $g:X\to\Delta$ and $A\in\cat{F}/\Gamma$, one has
\begin{equation*}
\pi_2(f,A)\circ\pi_2(g,f_\ast(A))\jdeq\pi_2(f\circ g,A).
\end{equation*}
\end{lem}

\begin{proof}
It suffices to show that
\begin{equation*}
\zeta_{A,f\circ g\circ g_\ast(f_\ast(A))}(\pi_2(f,A)\circ\pi_2(g,f_\ast(A)))\jdeq\tfid{g_\ast(f_\ast(A))}.
\end{equation*}
Because $g\circ g_\ast(f_\ast(A))\jdeq f_\ast(A)\circ\pi_2(g,f_\ast(A))$, we get
to use the naturality of $\zeta$:
\begin{align*}
& \zeta_{A,f\circ g\circ g_\ast(f_\ast(A))}(\pi_2(f,A)\circ\pi_2(g,f_\ast(A))) \\
& \jdeq
\zeta_{A,f\circ f_\ast(A)\circ\pi_2(g,f_\ast(A))}(\pi_2(f,A)\circ\pi_2(g,f_\ast(A))) \\
& \jdeq
\zeta_{f_\ast(A)_\ast(f_\ast(A)),\pi_2(g,f_\ast(A))}(\zeta_{A,f\circ f_\ast(A)}(\pi_2(f,a))\circ\pi_2(g,f_\ast(A))) \\
& \jdeq
\zeta_{f_\ast(A)_\ast(f_\ast(A)),\pi_2(g,f_\ast(A))}(\tfid{f_\ast(A)}\circ\pi_2(g,f_\ast(A))) \\
& \jdeq
\zeta_{f_\ast(A)_\ast(f_\ast(A)),\pi_2(g,f_\ast(A))}(\zeta_{f_\ast(A),f_\ast(A)}(\catid{\ctxext{\Delta}{f_\ast(A)}})\circ\pi_2(g,f_\ast(A))) \\
& \jdeq
\zeta_{f_\ast(A),f_\ast(A)\circ\pi_2(g,f_\ast(A))}(\catid{\ctxext{\Delta}{f_\ast(A)}}\circ\pi_2(g,f_\ast(A))) \\
& \jdeq
\zeta_{f_\ast(A),f_\ast(A)\circ\pi_2(g,f_\ast(A))}(\pi_2(g,f_\ast(A))) \\
& \jdeq
\zeta_{f_\ast(A),g\circ g_\ast(f_\ast(A))}(\pi_2(g,f_\ast(A)))\\
& \jdeq
\tfid{g_\ast(f_\ast(A))}.\qedhere
\end{align*}
\end{proof}

\begin{lem}
Let $f:\Delta\to\Gamma$ and $A\in\cat{F}/\Gamma$. Then one has 
\begin{equation*}
\pi_2(f,A)\circ\zeta_{A,f}(\theta)\jdeq\theta
\end{equation*}
for any $\theta:\Delta\to\ctxext{\Gamma}{A}$ satisfying $A\circ\theta\jdeq f$.
\end{lem}

\begin{proof}
We will write $\tilde\theta\defeq\zeta_{A,f}(\theta)$. Note that
\begin{equation*}
A\circ\pi_2(f,A)\circ\tilde\theta\jdeq f\circ f_\ast(A)\circ\tilde\theta\jdeq f.
\end{equation*} 
Therefore, we can prove that $\pi_2(f,A)\circ\tilde\theta\jdeq\theta$ by showing
that $\zeta_{A,f}(\pi_2(f,A)\circ\tilde\theta)\jdeq\tilde\theta$. This is
shown using the naturality of $\zeta$:
\begin{align*}
\zeta_{A,f}(\pi_2(f,A)\circ\tilde\theta)
  & \jdeq
\zeta_{A,f\circ f_\ast(A)\circ\tilde\theta}(\pi_2(f,A)\circ\tilde\theta)
  \\
  & \jdeq
\zeta_{(f_\ast(A))_\ast(f_\ast(A)),\tilde{\theta}}(\zeta_{A,f\circ f_\ast(A)}(\pi_2(f,A))\circ\tilde\theta)
  \\
  & \jdeq
\zeta_{(f_\ast(A))_\ast(f_\ast(A)),\tilde{\theta}}(\idtm{f_\ast(A)}\circ\tilde\theta)
  \\
  & \jdeq
\zeta_{(f_\ast(A))_\ast(f_\ast(A)),\tilde{\theta}}(\zeta_{f_\ast(A),f_\ast(A)}(\catid{\ctxext{\Delta}{f_\ast(A)}})\circ\tilde\theta)
  \\
  & \jdeq
\zeta_{f_\ast(A),f_\ast(A)\circ\tilde\theta}(\catid{\ctxext{\Delta}{f_\ast(A)}}\circ\tilde\theta) \\
  & \jdeq
\zeta_{f_\ast(A),\catid{\Delta}}(\tilde\theta) \\
  & \jdeq
\theta.\qedhere
\end{align*}
\end{proof}

\begin{lem}\label{lem:EE_pb}
The diagram
\begin{equation*}
\begin{tikzcd}[column sep=large]
\ctxext{\Delta}{f_\ast(A)} \arrow[d,fib] \arrow[r,"{\pi_2(f,A)}"] & \ctxext{\Gamma}{A} \arrow[d,fib] \\
\Delta \arrow[r,"f"] & \Gamma
\end{tikzcd}
\end{equation*}
is a pullback square.
\end{lem}


\begin{proof}
To verify the universal mapping property of the pullback square, consider a
commuting diagram
\begin{equation*}
\begin{tikzcd}
X \arrow[d,swap,"g"] \arrow[r,"h"] & \ctxext{\Gamma}{A} \arrow[d,fib] \\
\Delta \arrow[r,"f"] & \Gamma
\end{tikzcd}
\end{equation*}
Then $h$ corresponds uniquely to a section $k$ of $(f\circ g)_\ast(A)$. Since
$(f\circ g)_\ast(A)\jdeq g_\ast(f_\ast(A))$, the section $k$ fits in the
commuting diagram
\begin{equation*}
\begin{tikzcd}[column sep=huge]
\ctxext{X}{g_\ast(f_\ast(A))} \arrow[d,xshift=.7ex,fib] \arrow[r,"{\pi_2(g,f_\ast(A))}"]
& \ctxext{\Delta}{f_\ast(A)} \arrow[d,fib] \arrow[r,"{\pi_2(f,A)}"]
& \ctxext{\Gamma}{A} \arrow[d,fib] \\
X \arrow[r,swap,"g"] \arrow[u,xshift=-.7ex,"k"] & \Delta \arrow[r,swap,"f"] & \Gamma
\end{tikzcd}
\end{equation*}
so we obtain $u\defeq\pi_2(g,f_\ast(A))\circ k:X\to\ctxext{\Delta}{f_\ast(A)}$. 
Then we have $f_\ast(A)\circ u\jdeq g\circ g_\ast(f_\ast(A))\circ k\jdeq g$ and
$\pi_2(f,A)\circ u\jdeq \pi_2(f\circ g,A)\circ k\jdeq h$. Note that
$u$ corresponds uniquely to $k$, which corresponds uniquely to $h$, so the
uniqueness of $u$ is automatic.
\end{proof}

\begin{lem}
For any $f:\Delta\to\Gamma$ and $P\in\cat{F}/\ctxext{\Gamma}{A}$, there is a unique isomorphism
$\varphi_{f,P}:\pi_2(f,A)_\ast(P)\cong (f_\ast/A)(P)$ in 
$\mathbb{C}/\ctxext{\Delta}{f_\ast(A)}$ such that
\begin{equation*}
\pi_2(f,\ctxext{A}{P})\jdeq \pi_2(\pi_2(f,A),P)\circ\varphi_{f,P}.
\end{equation*}
Consequently, we get isomorphisms
\begin{align*}
\{\theta:\ctxext{\Delta}{f_\ast(A)}\to\ctxext{{\Gamma}{A}}{P}\mid P\circ\theta\jdeq \pi_2(f,A)\}
&  \cong
\{\eta:\ctxext{\Delta}{f_\ast(A)}\to\ctxext{{\Delta}{(f_\ast(A))}}{f_\ast(P)}\mid f_\ast(P)\circ\eta\jdeq\catid{\ctxext{\Delta}{f_\ast(A)}}\}
\end{align*}
natural in $P$ and $f$.
\end{lem}

\begin{proof}
Note that both the lower square and the outer rectangle in the diagram
\begin{equation*}
\begin{tikzcd}[column sep=huge]
\ctxext{{\Delta}{f_\ast(A)}}{f_\ast(P)} \arrow[d,fib] \arrow[r,"{\pi_2(f,\ctxext{A}{P})}"]
& \ctxext{{\Gamma}{A}}{P} \arrow[d,fib] \\
\ctxext{\Delta}{f_\ast(A)} \arrow[d,fib] \arrow[r,"{\pi_2(f,A)}"] 
& \ctxext{\Gamma}{A} \arrow[d,fib] \\
\Delta \arrow[r,swap,"f"] & \Gamma
\end{tikzcd}
\end{equation*}
are pullback squares. By the pasting lemma for pullbacks, it follows that the
top square must also be a pullback. On the other hand, we have the pullback
square
\begin{equation*}
\begin{tikzcd}
\ctxext{{\Delta}{f_\ast(A)}}{\pi_2(f,A)_\ast(P)} \arrow[d,fib] \arrow[r]
& \ctxext{{\Gamma}{A}}{P} \arrow[d,fib] \\
\ctxext{\Delta}{f_\ast(A)} \arrow[r,swap,"{\pi_2(f,A)}"]
& \ctxext{\Gamma}{A}
\end{tikzcd}
\end{equation*}
Since pullbacks are uniquely unique, we get the desired isomorphism $\varphi_{f,P}$.

It remains to prove the asserted isomorphism. 
\begin{align*}
& \{\theta:\ctxext{\Delta}{f_\ast(A)}\to\ctxext{{\Gamma}{A}}{P}\mid P\circ\theta\jdeq \pi_2(f,A)\} \\
&  \cong
\{\eta : \ctxext{\Delta}{f_\ast(A)}\to\ctxext{{\Delta}{f_\ast(A)}}{\pi_2(f,A)_\ast(P)} \mid
  \pi_2(f,A)_\ast(P)\circ\eta\jdeq\catid{}\}
  \\
  & \cong
\{\eta:\ctxext{\Delta}{f_\ast(A)}\to\ctxext{{\Delta}{(f_\ast(A))}}{f_\ast(P)}\mid f_\ast(P)\circ\eta\jdeq\catid{\ctxext{\Delta}{f_\ast(A)}}\}
\end{align*}
\end{proof}

\begin{lem}
Let $\mathbb{E}$ be an E'-system. Then there is a function $T(P)\to T(f_\ast(P))$
for any $f:\Delta\to\Gamma$ in $\cat{C}$ and $P\in\cat{F}/\ctxext{\Gamma}{A}$, contravariant
in $f$. 
\end{lem}

\begin{proof}
Since $f_\ast(P)$ is isomorphic to $\pi_2(f,A)_\ast(P)$ in $\cat{C}/\ctxext{\Gamma}{A}$,
it follows that the sets $T(f_\ast(P))$ and $T(\pi_2(f,A)_\ast(P))$ are isomorphic
via composition with $\varphi_{f,P}$. By \autoref{lem:ptm}, we have a function
from $T(P)$ to $T(\pi_2(f,A)_\ast(P))$.
\end{proof}

\begin{defn}
For any $\Gamma\in\cat{C}$, define $\cat{C}_\cat{F}/\Gamma$ to be the full
subcategory of $\cat{C}/\Gamma$ with the same objects as $\cat{F}/\Gamma$. 
\end{defn}

\begin{thm}
Let $\mathbb{E}$ be an E'-system. Then $f_\ast$ can be extended to a functor
$\cat{C}_\cat{F}/\Gamma\to\cat{C}_\cat{F}/\Delta$, for each $f:\Delta\to\Gamma$, and moreover the diagram
\begin{equation*}
\begin{tikzcd}[column sep=large]
\cat{C}_\cat{F}/\ctxext{\Gamma}{B} \arrow[r,"{f_\ast/B}"] \arrow[d,swap,"{g_\ast}"]
& \cat{C}_\cat{F}/\ctxext{\Delta}{f_\ast(B)} \arrow[d,"{(f_\ast(g))_\ast}"] \\
\cat{C}_\cat{F}/\ctxext{\Gamma}{A} \arrow[r,swap,"{f_\ast/A}"]
& \cat{C}_\cat{F}/\ctxext{\Delta}{f_\ast(A)}
\end{tikzcd}
\end{equation*}
commutes for every $g:A\to B$ in $\cat{C}_\cat{F}/\Gamma$.
\end{thm}

\begin{proof}
For any $A,B\in\cat{F}/\Gamma$, we have an isomorphism
$\mathrm{hom}_{\cat{C}/\Gamma}(A,B)\cong T(W_A(B))$. We also get a function
$T(W_A(B))\to T(f_\ast(W_A(B)))$. Therefore, it suffices to show that
$f_\ast(W_A(B))\cong W_{f_\ast(A)}(f_\ast(B))$. 
\begin{align*}
f_\ast(W_A(B))
& \jdeq f_\ast(A_\ast(B)) \\
& \cong \pi_2(f,A)_\ast(A_\ast(B)) \\
& \jdeq (A\circ\pi_2(f,A))_\ast(B) \\
& \jdeq (f\circ f_\ast(A))_\ast(B) \\
& \jdeq f_\ast(A)_\ast(f_\ast(B)) \\
& \jdeq W_{f_\ast(A)}(f_\ast(B)). \qedhere
\end{align*}
\end{proof}

\begin{thm}
E-systems are the same things as E'-systems.
\end{thm}

\begin{proof}
Given an E-system $\mathbb{E}$, we have the category $\cat{C}$ of internal
morphisms in the empty context, with the subcategory $\cat{F}$ of projections $\cprojfstf{A}{P}$. 
The functors $f_\ast:\cat{F}/\Gamma\to\cat{F}/\Delta$ are contravariant in
$f$, so we have to verify whether we have the stated isomorphisms.

Note first that for any $A\in\cat{F}/\Gamma$, $W_A$ is
contravariant in $A$ and that it preserves the terminal object. 

Now let $\theta:\Delta\to\ctxext{\Gamma}{A}$ so that $A\circ\theta\jdeq f$
for some $f:\Delta\to\Gamma$. Then we have $\zeta_{A,f}(\theta)\in T(f_\ast(A))$.  
We wish to define $\theta_\ast(\idtm{A}):\Delta\to\ctxext{\Delta}{f_\ast(A)}$

for any $A\in\cat{F}/\Gamma$ and $Q\in\cat{F}/\ctxext{\Gamma}{B}$, we have
\begin{align*}
T(B) 
  & \jdeq 
\{ \eta : \Gamma \to\ctxext{\Gamma}{B}
        \mid B\circ\eta\jdeq\catid{\Gamma}\}
  \\
T(W_A(B))
  & \jdeq
\{ \eta : \ctxext{\Gamma}{A}\to\ctxext{{\Gamma}{A}}{W_A(B)}
        \mid W_A(B)\circ\eta\jdeq\catid{\ctxext{\Gamma}{A}}\}
  \\
  & \cong
\{ \theta : \ctxext{\Gamma}{A}\to\ctxext{\Gamma}{B}\mid B\circ\theta\jdeq A\}
\end{align*}
Therefore, $\eta\mapsto \zeta_{B,A}(\eta\circ A)$ defines a function from
$T(B)$ to $T(W_A(B))$. 
\end{proof}
\end{comment}

\subsection{Comprehension categories}

\begin{defn}
Let $p:\cat{E}\to\cat{B}$ be a functor and let $g:A\to B$ be a morphism in $\cat{B}$.
A morphism $f:X\to Y$ in $\cat{E}$ is said to be \emph{cartesian over $g$}, if
\begin{enumerate}
\item $p(f)\jdeq g$
\item For all $f':X'\to Y$ with $p(f')\jdeq g$, there is a unique $u:X'\to X$
such that $f'=f\circ u$ and $p(u)\jdeq\catid{A}$. 
\end{enumerate}
A cartesian morphism over $g$ is also called a \emph{cartesian lift of $g$}.
\end{defn}

\begin{defn}
A functor $p:\cat{E}\to\cat{B}$ is called a \emph{Grothendieck fibration} if
every morphism into an object of the form $p(X)$ has a cartesian lift, 
and cartesian morphisms are closed under composition.
\end{defn}

\begin{defn}
A Grothendieck fibration $p:\cat{E}\to\cat{B}$ is said to be \emph{split}, if
there is a functorial choice of cartesian lifts, i.e.~for every $g:A\to B$ in $\cat{B}$
there is a cartesian lift $j(g):j(A)\to j(B)$, such that $j(\catid{A})\jdeq\catid{j(A)}$
and $j(h\circ g)\jdeq j(h)\circ j(g)$.
\end{defn}

\begin{defn}
A comprehension category $\mathcal{P}$ consists of categories $\cat{E}$, $\cat{C}$ and a
commuting triangle
\begin{equation*}
\begin{tikzcd}
\cat{E} \arrow[rr,"\mathcal{P}"] \arrow[dr,swap,"p"] & & \cat{C}^{\to} \arrow[dl,"\mathrm{cod}"] \\
& \cat{C}
\end{tikzcd}
\end{equation*}
of functors, such that $p$ is a Grothendieck fibration, and $\mathcal{P}$ maps
cartesian morphisms in $\cat{E}$ to pullback squares in $\cat{C}$. We say that
$\mathcal{P}$ is a full comprehension category if the functor $\mathcal{P}:
\cat{E}\to\cat{C}^{\to}$ is full and faithful; and we say that
$\mathcal{P}$ is split if $p$ is split.
\end{defn}

\begin{defn}
There is a functor $\mathcal{V}:\mathbf{Comp}\to\mathbf{E'sys_{\circ}}$, from
the category $\mathbf{Comp}$ of split full comprehension categories to
E'-systems without empty context.
\end{defn}

\begin{constr}
Let $\mathcal{P}$ be a comprehension category with underlying category
$\cat{C}$. We define $\cat{F}$ to be the sub-pre-category of $\cat{C}$ generated
by the morphisms $\mathcal{P}(A):\ctxext{\Gamma}{A}\to\Gamma$ for $A\in\cat{E}$.  

For any $f:\Delta\to\Gamma$, we define a functor $f_\ast:\cat{F}/\Gamma\to
\cat{F}/\Delta$ by induction on the length of the morphism
Let $f:\Delta\to\Gamma$ and $p(A)\jdeq\Gamma$. Then we get a pullback square
\begin{equation*}
\begin{tikzcd}[column sep=large]
\ctxext{\Delta}{f_\ast(A)} \arrow[d,fib] \arrow[r,"{\pi_2(f,A)}"] & \ctxext{\Gamma}{A} \arrow[d,fib] \\
\Delta \arrow[r,"f"] & \Gamma
\end{tikzcd}
\end{equation*}
There seems to be no way to show that
$\pi_2(f,\ctxext{A}{P})\jdeq \pi_2(\pi_2(f,A),P)$.
\end{constr}

\begin{conj}
An E'-system is the same thing as a split full comprehension category in which
$\cat{E}$ has a terminal object.
\end{conj}

\begin{proof}
Let $\mathcal{E}$ be an E'-system. Define $\cat{E}$ to be the full subcategory
of $\cat{C}^\to$ whose objects are the morphisms of $\cat{F}$. Thus, for any
$\Gamma\in\cat{C}$, the category $p^{-1}(\Gamma)$ is the category of internal
morphisms the E-system $\mathcal{E}/\Gamma$. 

We first show that $p$ is a Grothendieck fibration. Let $f:\Delta\to\Gamma$ be
a morphism in $\cat{C}$ and let $A\in\cat{F}/\Gamma$. Then we have $\catid{\Delta},\catid{\Gamma}\in\cat{E}$,
and $(f,\pi_2(f,A)):f_\ast(A)\to A$. It is routine to verify that $(f,\pi_2(f,A))$
is a cartesian lift of $f$, and that this choice of catesian lifts is functorial. 

Now let $g:X\to\Delta$ and $f:\Delta\to \Gamma$, and let $\tilde{f}:\tilde{\Delta}\to\tilde{\Gamma}$ and
$\tilde{g}:\tilde{X}\to\tilde{\Delta}$ be cartesian lifts of $f$ and $g$, respectively. We have to show
that $\tilde{f}\circ \tilde{g}$ is a cartesian lift of $f\circ g$. Let $C\in\cat{F}/X$ and
let $h : C\to \tilde{\Gamma}$ be a morphism such that $\mathrm{cod}(h)\jdeq f\circ g$. 
\end{proof}

\begin{comment}
\subsection{Quotients of E-systems}

\begin{defn}
A regular congruence relation $\sim$ on a category $\cat{F}$ with term structure,
is a triple $(\sim_c,\sim_f,\sim_t)$ consisting of
\begin{enumerate}
\item an equivalence relation $\sim_c$ on the set of objects of $\cat{F}$,
\item an equivalence relation $\sim_f$ on the set of morphisms of $\cat{F}$,
\item an equivalence relation $\sim_t$ on the set of terms of $\cat{F}$,
\end{enumerate}
which are compatible with $\mathrm{cod}$, $\mathrm{dom}$, $\catid{}$, $\circ$
and $\ebd$
\end{defn}
\end{comment}
