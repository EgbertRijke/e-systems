\section{Derived notions of the theory of contexts, families and terms}
\label{digging_deeper}

In this section we use the framework we have developed in \autoref{tt}
to derive new notions and their properties. In particular, we will
develop the notion of \emph{extension on terms} together with the projection
maps from an extension to the `base' context of family, the 
\emph{family pullback} which is a version of pullbacks for families and thirdly
the \emph{inductive morphisms} which are morphisms of type theory that allow
to find terms of families over the codomain context by pulling them back to
the domain context and finding a term there.
\emph{This section contains no new assumptions.}

In \autoref{extension-on-terms} on the extension operation on terms we will
derive all the compatibility rules that one would expect to hold for extension
on terms. The key to most of these results is the currying operation, which
could be seen as the missing feature in the table above. The extension on terms
operation depends in an essential way on the substitution operation, on the
identity terms and therefore indirectly also on the weakening operation. Thus,
we will see here all of the features of the theory we develop in
\autoref{tt} come to the 
surface.

Next, we introduce the inclusion of the fibers $\subst{x}{P}$ into the extension
$\ctxext{A}{P}$ as a morphism in context $\Gamma$. As was the case with
extension on terms and with projections, there will be a ton of compatibility
properties which we will prove about these inclusions. 

It should be kept in mind though that in the current formulation there is no
sealed deal establishing a relationship between families over a context
and any kind of morphisms -- neither with morphisms having the `base' of the
family as its codomain nor with families into a universe (universes will be
introduced in \autoref{universes}). The only thing we know here is
that a family $P$ over $\ctxext{\Gamma}{A}$ determines a context morphism
from $\ctxext{A}{P}$ to $A$ in context $\Gamma$, the projection. 
We do not see this as
a shortcoming of the theory of contexts families and terms. Rather, such a
correspondence is a feature of a theory which does incorporate universes. The
fact that we're lacking a clear connection between families and (a specified
class of) morphisms, however, does show up in our treatment of the notion we
called familie pullbacks. For instance, we can't show that a square of families
is a pullback precisely when the corresponding square of projections is a
pullback: only the backwards direction holds. 
The discrepancies continue: ordinary pullbacks do not always exist
whereas family pullbacks do but the composition of two family pullback squares
need not be a family pullback square again whereas the pasting lemma of
ordinary pullbacks holds as usually.
We feel that pointing out what we can and can't do in the current setting is
an important aspect of developing an intuition with the system and therefore
we include this subsection even though the theory of family pullbacks
might feel a bit different than the usual theory of pullbacks.

In the last subsection we give a treatment of inductive morphisms. These
morphisms appear also in the introduction of the many inductive type
constructors in \autoref{tt_constructors} and therefore a general treatment of
the subject is insightful.

\subsection{Morphisms}\label{morphisms}
Using the rules of the compatibility of substitution with weakening and of the
compatibility of weakening with itself, we see that we can show

\begin{lem}\label{lem:prehom}
The inference rule\begin{equation*}
\inference
  { \jfam{\Gamma}{A}
    \jfam{\Gamma}{B}
    \jfam{\Gamma}{C}
    \jhom{\Gamma}{A}{B}{f}
    }
  { \jfameq
    {{\Gamma}{A}}
    {\subst{f}{\ctxwk{A}{{B}{C}}}}
    {\ctxwk{A}{C}}
    }
\end{equation*}
is valid.
\end{lem}

\begin{proof}
Let $\jfam{\Gamma}{A}$, $\jfam{\Gamma}{B}$, $\jfam{\Gamma}{C}$ and $\jhom{\Gamma}{A}{B}{f}$.
Then we have the judgmental equalities
\begin{align*}
\subst{f}{\ctxwk{A}{{B}{C}}}
& \jdeq 
  \subst{f}{\ctxwk{{A}{B}}{{A}{C}}}
  \tag{by \autoref{comp-ww-f}}\\
& \jdeq 
  \ctxwk{A}{C}.
  \tag{by \autoref{cancellation-ws-t}}
\end{align*}
\end{proof}

It follows that for $\jterm{{\Gamma}{B}}{\ctxwk{B}{C}}{g}$ we can compose $f$
with $g$ to obtain a term of $\ctxwk{A}{C}$ in context $\ctxext{\Gamma}{A}$.
In the following definition, we work with in a slightly greater generality.

\begin{defn}
We define the judgment\begin{equation*}
\jhom{\Gamma}{A}{B}{f},
\end{equation*}
which is pronounced as `$f$ is a morphism from $A$ to $B$ in context $\Gamma$',
to be the judgment\begin{equation*}
\unfold{\jhom{\Gamma}{A}{B}{f}}.
\end{equation*}
Likewise, we define the judgment\begin{equation*}
\jhomeq{\Gamma}{A}{B}{f}{f'}
\end{equation*}
to be the judgment\begin{equation*}
\unfold{\jhomeq{\Gamma}{A}{B}{f}{f'}}.
\end{equation*}
\end{defn}

\begin{defn}
Let $\jhom{\Gamma}{A}{B}{f}$ and consider a family $\jfam{{\Gamma}{B}}{Q}$,
a family $\jfam{{{\Gamma}{B}}{Q}}{R}$ and a term $\jterm{{{\Gamma}{B}}{Q}}{R}{h}$.
We define\begin{align*}
\jfamdefn*
  {{\Gamma}{A}}
  {\jcomp{A}{f}{Q}}
  {\unfold{\jcomp{A}{f}{Q}}}
  \\
\jfamdefn*
  {{{\Gamma}{A}}{\jcomp{A}{f}{Q}}}
  {\jcomp{A}{f}{R}}
  {\unfold{\jcomp{A}{f}{R}}}
  \\
\jtermdefn*
  {{{\Gamma}{A}}{\jcomp{A}{f}{Q}}}
  {\jcomp{A}{f}{R}}
  {\jcomp{A}{f}{h}}
  {\unfold{\jcomp{A}{f}{h}}}.
\end{align*}
\end{defn}

\begin{lem}\label{lem:jcomp-emp}
Let $\jhom{\Gamma}{A}{B}{f}$. Then the inference rules
\begin{align*}
& \inference
  { %
    }
  { \jfameq{{\Gamma}{A}}{\jcomp{A}{f}{\emptyf}}{\emptyf}
    }
  \\
& \inference
  { \jfam{{\Gamma}{B}}{Q}
    }
  { \jfameq{{{\Gamma}{B}}{\jcomp{A}{f}{Q}}}{\jcomp{A}{f}{\emptyf}}{\emptyf}
    }
\end{align*}
are valid.
\end{lem}

\begin{proof}
For the first, note that we have the judgmental equalities
\begin{align*}
\jcomp{A}{f}{\emptyf}
& \jdeq
  \unfold{\jcomp{A}{f}{\emptyf}}
  \tag{by definition}
  \\
& \jdeq
  \subst{f}{\emptyf}
  \tag{by \autoref{comp-w0-c}}
  \\
& \jdeq
  \emptyf.
  \tag{by \autoref{comp-s0-c}}
\end{align*}
The second judgmental equality is proven similarly.
\end{proof}

\begin{rmk}
Recall that we can treat a family $\jfam{{\Gamma}{B}}{Q}$ as a family
$\jfam{{{\Gamma}{B}}{\emptyf}}{Q}$ and that $\jcomp{A}{f}{\emptyf}\jdeq
\emptyf$. Thus we can apply composition with $f$ to terms 
$\jterm{{\Gamma}{B}}{Q}{g}$. We get
\begin{equation*}
\jtermeq
  {{\Gamma}{B}}
  {\jcomp{A}{f}{Q}}
  {\jcomp{A}{f}{g}}
  {\unfold{\jcomp{A}{f}{g}}}.
\end{equation*}
In the particular situation where we take $Q$ to be a weakened family
$\ctxwk{B}{C}$, we see that we can apply composition with $f$ to morphisms
from $B$ to $C$ and we can use \autoref{lem:prehom} to see that we get
\begin{equation*}
\jhomeq{\Gamma}{A}{C}{\jcomp{A}{f}{g}}{\unfold{\jcomp{A}{f}{g}}}
\end{equation*}
for $\jhom{\Gamma}{B}{C}{g}$. 

One might argue that the notation for composition should be reserved to only
this special case, to not confuse with common intuition of composition. It is
however very convenient to see composition as an operation of the theory of
contexts, families and terms. This allows us to follow the scheme of
compatibility rules which are provable for this form of composition. 
\end{rmk}

We have lots of compatibility properties for composition. On the one hand we
have the properties that composition with a morphism $f$ is compatible with
the empty family, extension, weakening, substitution and identity terms. On
the other hand, there are compatibility properties saying what happens when
we substitute by a composition, or when we weaken or substitute a composition.
Proving all these compatibility properties is the content of the rest of this
subsection. None of it is very difficult.

\begin{lem}\label{lem:jcomp-ext}
We have the following inference rule expressing that composition with $f$ is
compatible with extension:
\begin{equation*}
\inference
  { \jfam{{{{\Gamma}{B}}{Q}}{R}}{S}
    }
  { \jfameq
      {{{\Gamma}{A}}{\jcomp{A}{f}{Q}}}
      {\jcomp{A}{f}{\ctxext{R}{S}}}
      {\ctxext{\jcomp{A}{f}{R}}{\jcomp{A}{f}{S}}}
    }
\end{equation*}
\end{lem}

\begin{proof}
Let $\jfam{{{{\Gamma}{B}}{Q}}{R}}{S}$. Then we have the judgmental equalities
\begin{align*}
\jcomp{A}{f}{\ctxext{R}{S}}
& \jdeq
  \unfold{\jcomp{A}{f}{\ctxext{R}{S}}}
  \tag{by definition}
  \\
& \jdeq
  \subst{f}{\ctxext{\ctxwk{A}{R}}{\ctxwk{A}{S}}}
  \tag{by \autoref{comp-we-f}}
  \\
& \jdeq
  \unfoldall{\ctxext{\jcomp{A}{f}{R}}{\jcomp{A}{f}{S}}}
  \tag{by \autoref{comp-se-f}}
  \\
& \jdeq
  \ctxext{\jcomp{A}{f}{R}}{\jcomp{A}{f}{S}}.
  \tag{by definition}
\end{align*}
\end{proof}

\begin{lem}\label{lem:jcomp-wk}
We have the following inference rules expressing that composition with $f$ is
compatible with weakening:
\begin{align*}
& \inference
  { \jfam{{{\Gamma}{B}}{Q}}{R}
    \jfam{{{\Gamma}{B}}{Q}}{S}
    }
  { \jfameq
      {{{{\Gamma}{A}}{\jcomp{A}{f}{Q}}}{\jcomp{A}{f}{R}}}
      {\jcomp{A}{f}{\ctxwk{R}{S}}}
      {\ctxwk{\jcomp{A}{f}{R}}{\jcomp{A}{f}{S}}}
    }
  \\
& \inference
  { \jfam{{{\Gamma}{B}}{Q}}{R}
    \jterm{{{\Gamma}{B}}{Q}}{S}{k}
    }
  { \jtermeq
      {{{{\Gamma}{A}}{\jcomp{A}{f}{Q}}}{\jcomp{A}{f}{R}}}
      {\jcomp{A}{f}{\ctxwk{R}{S}}}
      {\jcomp{A}{f}{\ctxwk{R}{k}}}
      {\ctxwk{\jcomp{A}{f}{R}}{\jcomp{A}{f}{k}}}
    }
\end{align*}
\end{lem}

\begin{proof}
Consider the families $\jfam{{{\Gamma}{B}}{Q}}{R}$ and 
$\jfam{{{\Gamma}{B}}{Q}}{S}$. Then we have the judgmental equalities
\begin{align*}
\jcomp{A}{f}{\ctxwk{R}{S}}
& \jdeq
  \unfold{\jcomp{A}{f}{\ctxwk{R}{S}}}
  \tag{by definition}
  \\
& \jdeq
  \subst{f}{\ctxwk{{A}{R}}{{A}{S}}}
  \tag{by \autoref{comp-ww-f}}
  \\
& \jdeq
  \unfoldall{\ctxwk{\jcomp{A}{f}{R}}{\jcomp{A}{f}{S}}}
  \tag{by \autoref{comp-sw-f}}
  \\
& \jdeq
  \ctxwk{\jcomp{A}{f}{R}}{\jcomp{A}{f}{S}}.
  \tag{by definition}
\end{align*}
The proof of the second property is similar.
\end{proof}

\begin{lem}\label{lem:jcomp-subst}
We have the following inference rules expressing that composition with $f$ is
compatible with substitution:
\begin{align*}
& \inference
  { \jterm{{{\Gamma}{B}}{Q}}{R}{h}
    \jfam{{{{\Gamma}{B}}{Q}}{R}}{S}
    }
  { \jfameq
      {{{\Gamma}{A}}{\jcomp{A}{f}{Q}}}
      {\jcomp{A}{f}{\subst{h}{S}}}
      {\subst{\jcomp{A}{f}{h}}{\jcomp{A}{f}{S}}}
    }
  \\
& \inference
  { \jterm{{{\Gamma}{B}}{Q}}{R}{h}
    \jterm{{{{\Gamma}{B}}{Q}}{R}}{S}{k}
    }
  { \jtermeq
      {{{\Gamma}{A}}{\jcomp{A}{f}{Q}}}
      {\jcomp{A}{f}{\subst{h}{S}}}
      {\jcomp{A}{f}{\subst{h}{k}}}
      {\subst{\jcomp{A}{f}{h}}{\jcomp{A}{f}{k}}}
    }
\end{align*}
\end{lem}

\begin{proof}
Let $\jterm{{{\Gamma}{B}}{Q}}{R}{h}$ and $\jfam{{{{\Gamma}{B}}{Q}}{R}}{S}$.
Then we have the judgmental equalities
\begin{align*}
\jcomp{A}{f}{\subst{h}{S}}
& \jdeq
  \unfold{\jcomp{A}{f}{\subst{h}{S}}}
  \tag{by definition}
  \\
& \jdeq
  \subst{f}{{\ctxwk{A}{h}}{\ctxwk{A}{S}}}
  \tag{by \autoref{comp-ws-f}}
  \\
& \jdeq
  \unfoldall{\subst{\jcomp{A}{f}{h}}{\jcomp{A}{f}{S}}}
  \tag{by \autoref{comp-ss-f}}
  \\
& \jdeq
  \subst{\jcomp{A}{f}{h}}{\jcomp{A}{f}{S}}.
  \tag{by definition}
\end{align*}
The proof of the second inference rule is similar.
\end{proof}

\begin{lem}\label{lem:jcomp-idtm}
We have the following inference rule expressing that composition with $f$ is
compatible with identity terms:
\begin{equation*}
\inference
  { \jfam{{{\Gamma}{B}}{Q}}{R}
    }
  { \jtermeq
      {{{{\Gamma}{A}}{\jcomp{A}{f}{Q}}}{\jcomp{A}{f}{R}}}
      {\ctxwk{\jcomp{A}{f}{R}}{\jcomp{A}{f}{R}}}
      {\jcomp{A}{f}{\idtm{R}}}
      {\idtm{\jcomp{A}{f}{R}}}
    }
\end{equation*}
\end{lem}

\begin{proof}
Let $\jfam{{{\Gamma}{B}}{Q}}{R}$. We have the following judgmental equalities:
\begin{align*}
\jcomp{A}{f}{\idtm{R}}
& \jdeq
  \unfold{\jcomp{A}{f}{\idtm{R}}}
  \tag{by definition}
  \\
& \jdeq
  \subst{f}{\idtm{\ctxwk{A}{R}}}
  \tag{by \autoref{comp-wi-t}}
  \\
& \jdeq
  \unfoldall{\idtm{\jcomp{A}{f}{R}}}
  \tag{by \autoref{comp-si-t}}
  \\
& \jdeq
  \idtm{\jcomp{A}{f}{R}}.
  \tag{by definition}
\end{align*}
\end{proof}

\begin{lem}\label{lem:jcomp-jcomp}
We have the following inference rules about the situation where something is
substituted by a composition:\begin{align*}
& \inference
  { \jhom{\Gamma}{A}{B}{f}
    \jhom{\Gamma}{B}{C}{g}
    \jfam{{{\Gamma}{A}}{\ctxwk{A}{C}}}{R}
    }
  { \jfameq
      {{\Gamma}{A}}
      {\subst{\jcomp{A}{f}{g}}{R}}
      {\subst{f}{{\ctxwk{A}{g}}{\ctxwk{{A}{B}}{R}}}}
    }
  \\
& \inference
  { \jhom{\Gamma}{A}{B}{f}
    \jhom{\Gamma}{B}{C}{g}
    \jterm{{{\Gamma}{A}}{\ctxwk{A}{C}}}{R}{h}
    }
  { \jfameq
    {{\Gamma}{A}}
    {\subst{\jcomp{A}{f}{g}}{h}}
    {\subst{f}{{\ctxwk{A}{g}}{\ctxwk{{A}{B}}{h}}}}
    }
\end{align*}
We also have the following related inference rules, asserting that composition
is strictly associative:\begin{align*}
& \inference
  { \jhom{\Gamma}{A}{B}{f}
    \jhom{\Gamma}{B}{C}{g}
    \jfam{{\Gamma}{C}}{R}
    }
  { \jfameq
      {{\Gamma}{A}}
      {\jcomp{A}{{A}{f}{g}}{R}}
      {\jcomp{A}{f}{{B}{g}{R}}}
    }
  \\
& \inference
    { \jhom{\Gamma}{A}{B}{f}
      \jhom{\Gamma}{B}{C}{g}
      \jterm{{\Gamma}{C}}{R}{h}
      }
    { \jtermeq
        {{\Gamma}{A}}
        {\jcomp{A}{{A}{f}{g}}{R}}
        {\jcomp{A}{{A}{f}{g}}{h}}
        {\jcomp{A}{f}{{B}{g}{h}}}
      }
\end{align*}
\end{lem}

\begin{proof}
Consider family morphisms $\jhom{\Gamma}{A}{B}{f}$ and $\jhom{\Gamma}{B}{C}{g}$
and a family $\jfam{{{\Gamma}{A}}{\ctxwk{A}{C}}}{R}$. Then we have the judgmental
equalities\begin{align*}
\subst{\jcomp{A}{f}{g}}{R} 
& \jdeq 
  \subst{{f}{\ctxwk{A}{g}}}{R}
  \tag{by definition}
  \\
& \jdeq 
  \subst{{f}{\ctxwk{A}{g}}}{\subst{f}{\ctxwk{{A}{B}}{R}}}
  \tag{by \autoref{cancellation-ws-f}}
  \\
& \jdeq 
  \subst{f}{{\ctxwk{A}{g}}{\ctxwk{{A}{B}}{R}}}.
  \tag{by \autoref{comp-ss-f}}
\end{align*}
The proof that 
$\subst{\jcomp{A}{f}{g}}{h}\jdeq\subst{f}{{\ctxwk{A}{g}}{\ctxwk{{A}{B}}{h}}}$
is similar.

Now suppose that $\jfam{{\Gamma}{C}}{R}$ instead. Then we have\begin{align*}
\jcomp{A}{{A}{f}{g}}{R} 
& \jdeq 
  \subst{\jcomp{A}{f}{g}}{\ctxwk{A}{R}}
  \tag{by definition}
  \\
& \jdeq 
  \subst{{f}{\ctxwk{A}{g}}}{\ctxwk{A}{R}}
  \tag{by definition}
  \\
& \jdeq
  \subst{{f}{\ctxwk{A}{g}}}{{f}{\ctxwk{{A}{B}}{{A}{R}}}}
  \tag{by \autoref{cancellation-ws-t}}
  \\
& \jdeq 
  \subst{f}{{\ctxwk{A}{g}}{\ctxwk{{A}{B}}{{A}{R}}}}
  \tag{by \autoref{comp-ss-f}}
  \\
& \jdeq 
  \subst{f}{{\ctxwk{A}{g}}{\ctxwk{A}{{B}{R}}}}
  \tag{by \autoref{comp-ww-f}}
  \\
& \jdeq 
  \subst{f}{\ctxwk{A}{\subst{g}{\ctxwk{B}{R}}}}
  \tag{by \autoref{comp-ws-f}}
  \\
& \jdeq 
  \subst{f}{\ctxwk{A}{\jcomp{B}{g}{R}}}
  \tag{by definition}
  \\
& \jdeq 
  \jcomp{A}{f}{{B}{g}{R}}.
  \tag{by definition}
\end{align*}
Again, the proof is similar for terms $h$ of $R$ in context $\ctxext{\Gamma}{C}$.
\end{proof}

We established already in \autoref{lem:jcomp-wk} that composition is compatible
with weakening. However, in that lemma we did not consider the possibility
of weakening a composition and neither did we consider the possibility of
composition by or with a constant morphism. We do this in the following lemma.

\begin{lem}\label{lem:jcomp-const}
The inference
\begin{align*}
& \inference
  { \jhom{\Gamma}{A}{B}{f}
    \jhom{\Gamma}{B}{C}{g}
    \jfam{{\Gamma}{A}}{P}
    }
  { \jhomeq
      {\Gamma}
      {{A}{P}}
      {C}
      {\ctxwk{P}{\jcomp{A}{f}{g}}}
      {\jcomp{{A}{P}}{\ctxwk{P}{f}}{g}}
    }
\end{align*}
tells what happens when we weaken a composition, which is a term of 
$\ctxwk{A}{C}$, by a family $P$ over $\ctxext{\Gamma}{P}$. We also have the following
inference rules expressing that compositions by or with constant morphisms are
again constant morphisms.
\begin{align*}
& \inference
  { \jterm{\Gamma}{B}{y}
    \jhom{\Gamma}{B}{C}{g}
    }
  { \jhomeq
      {\Gamma}
      {A}
      {C}
      {\jcomp{A}{\ctxwk{A}{y}}{g}}
      {\ctxwk{A}{\subst{y}{g}}}
    }
  \\
& \inference
  { \jhom{\Gamma}{A}{B}{f}
    \jterm{\Gamma}{C}{z}
    }
  { \jhomeq
      {\Gamma}
      {A}
      {C}
      {\jcomp{A}{f}{\ctxwk{B}{z}}}
      {\ctxwk{A}{z}}
    }
\end{align*}
\end{lem}

\begin{proof}
Let $\jhom{\Gamma}{A}{B}{f}$, $\jhom{\Gamma}{B}{C}{g}$ and $\jfam{{\Gamma}{A}}{P}$.
Then we have the judgmental equalities\begin{align*}
\ctxwk{P}{\jcomp{A}{f}{g}} 
& \jdeq 
  \ctxwk{P}{\subst{f}{\ctxwk{A}{g}}}
  \tag{by definition}
  \\
& \jdeq 
  \subst{\ctxwk{P}{f}}{\ctxwk{P}{{A}{g}}}
  \tag{by \autoref{comp-ws-t}}
  \\
& \jdeq 
  \subst{\ctxwk{P}{f}}{\ctxwk{\ctxext{A}{P}}{g}}
  \tag{by \autoref{comp-ew-t}}
  \\
& \jdeq 
  \jcomp{{A}{P}}{\ctxwk{P}{f}}{g}.
  \tag{by definition}
\end{align*}
Now let $\jterm{\Gamma}{B}{y}$ and $\jhom{\Gamma}{B}{C}{g}$. Then we have the
judgmental equalities\begin{align*}
\jcomp{A}{\ctxwk{A}{y}}{g}
& \jdeq 
  \subst{\ctxwk{A}{y}}{\ctxwk{A}{g}}
  \tag{by definition}
  \\
& \jdeq 
  \ctxwk{A}{\subst{y}{g}}.
  \tag{by \autoref{comp-ws-t}}
\end{align*}
For the third assertion, let $\jhom{\Gamma}{A}{B}{f}$ and $\jterm{\Gamma}{C}{z}$.
Then we have the judgmental equalities\begin{align*}
\jcomp{A}{f}{\ctxwk{B}{z}} 
& \jdeq 
  \subst{f}{\ctxwk{A}{{B}{z}}}
  \tag{by definition}
  \\
& \jdeq 
  \subst{f}{\ctxwk{{A}{B}}{{A}{z}}}
  \tag{by \autoref{comp-ww-t}}
  \\
& \jdeq 
  \ctxwk{A}{z}.
  \tag{by \autoref{cancellation-ws-t}}
\end{align*}
\end{proof}

Likewise, we can substitute a term $\jterm{\Gamma}{A}{x}$ in a composed morphism
$\jhom{\Gamma}{A}{C}{\jcomp{A}{f}{g}}$. What we get is the \emph{value}
$\subst{{x}{f}}{g}$. This is the content of one part of the following lemma.

There is a second, related case where we can substitute: when
we consider morphisms $\jhom{{\Gamma}{A}}{P}{Q}{f}$ and 
$\jhom{{\Gamma}{A}}{Q}{R}{g}$ we can restrict them to the fibers, obtaining
the inference rule
\begin{equation*}
\inference
  { \jhom{{\Gamma}{A}}{P}{Q}{f}
    \jhom{{\Gamma}{A}}{Q}{R}{g}
    \jterm{\Gamma}{A}{x}
    }
  { \jhomeq
      {\Gamma}
      {\subst{x}{P}}
      {\subst{x}{R}}
      {\subst{x}{\jcomp{P}{f}{g}}}
      {\jcomp{\subst{x}{P}}{\subst{x}{f}}{\subst{x}{g}}}
    }
\end{equation*}
In the following lemma we prove the validity of a more general version of this
inference rule.

\begin{lem}\label{lem:jcomp-fiber}
The following inference rules are valid:
\begin{align*}
& \inference
  { \jhom{{\Gamma}{A}}{P}{Q}{f}
    \jfam{{{\Gamma}{A}}{Q}}{R}
    \jterm{\Gamma}{A}{x}
    }
  { \jfameq
      {{\Gamma}{\subst{x}{P}}}
      {\subst{x}{\jcomp{P}{f}{R}}}
      {\jcomp{\subst{x}{P}}{\subst{x}{f}}{\subst{x}{R}}}
    }
  \\
& \inference
  { \jhom{{\Gamma}{A}}{P}{Q}{f}
    \jfam{{{{\Gamma}{A}}{Q}}{R}}{S}
    \jterm{\Gamma}{A}{x}
    }
  { \jfameq
      {{{\Gamma}{\subst{x}{P}}}{\subst{x}{\jcomp{P}{f}{R}}}}
      {\subst{x}{\jcomp{P}{f}{S}}}
      {\jcomp{\subst{x}{P}}{\subst{x}{f}}{\subst{x}{S}}}
    }
  \\
& \inference
  { \jhom{{\Gamma}{A}}{P}{Q}{f}
    \jterm{{{{\Gamma}{A}}{Q}}{R}}{S}{k}
    \jterm{\Gamma}{A}{x}
    }
  { \jtermeq
      {{{\Gamma}{\subst{x}{P}}}{\subst{x}{\jcomp{P}{f}{R}}}}
      {\subst{x}{\jcomp{P}{f}{S}}}
      {\subst{x}{\jcomp{P}{f}{k}}}
      {\jcomp{\subst{x}{P}}{\subst{x}{f}}{\subst{x}{k}}}
    }
\end{align*}
Also, the following inference rule computing the value at $x$ of a composed
morphism is valid:
\begin{align*}
& \inference
  { \jhom{\Gamma}{A}{B}{f}
    \jhom{\Gamma}{B}{C}{g}
    \jterm{\Gamma}{A}{x}
    }
  { \jtermeq
      {\Gamma}
      {C}
      {\subst{x}{\jcomp{A}{f}{g}}}
      {\subst{{x}{f}}{g}}
    }
\end{align*}
\end{lem}

\begin{proof}
Of the first three inference rules, we only prove the first.
Let $\jhom{{\Gamma}{A}}{P}{Q}{f}$, $\jfam{{{\Gamma}{A}}{Q}}{R}$ and 
$\jterm{\Gamma}{A}{x}$.
Then we have the judgmental equalities\begin{align*}
\subst{x}{\jcomp{A}{f}{R}}
& \jdeq 
  \subst{x}{{f}{\ctxwk{P}{R}}}
  \tag{by definition}
  \\
& \jdeq 
  \subst{{x}{f}}{{x}{\ctxwk{P}{R}}}
  \tag{by \autoref{comp-ss-f}}
  \\
& \jdeq 
  \subst{{x}{f}}{\ctxwk{\subst{x}{P}}{\subst{x}{R}}}
  \tag{by \autoref{comp-sw-f}}
  \\
& \jdeq 
  \jcomp{\subst{x}{P}}{\subst{x}{f}}{\subst{x}{R}}.
  \tag{by definition}
\end{align*}
Now let $\jhom{\Gamma}{A}{B}{f}$, $\jhom{\Gamma}{B}{C}{g}$ and $\jterm{\Gamma}{A}{x}$.
Then we have the judgmental equalities\begin{align*}
\subst{x}{\jcomp{A}{f}{g}}
& \jdeq 
  \subst{x}{{f}{\ctxwk{A}{g}}}
  \tag{by definition}
  \\
& \jdeq 
  \subst{{x}{f}}{{x}{\ctxwk{A}{g}}}
  \tag{by \autoref{comp-ss-t}}
  \\
& \jdeq 
  \subst{{x}{f}}{g}.
  \tag{by \autoref{cancellation-ws-t}}
\end{align*}
\end{proof}

\subsection{Projections and extension on terms}\label{extension-on-terms}
In this subsection we consider the notion of extension on terms, which has now
become definable inside our theory. Moreover, every compatibility rule one may
dream of is provable as well, using the compatibility rules we have introduced
earlier.

\begin{defn}
When $\jterm{\Gamma}{A}{x}$ and $\jterm{\Gamma}{\subst{x}{P}}{u}$ are terms,
we define 
\begin{equation*}
\jtermdefn
  {\Gamma}
  {\ctxext{A}{P}}
  {\tmext{A}{P}{x}{u}}
  {\unfold{\tmext{A}{P}{x}{u}}}.
\end{equation*} 
\end{defn}

Thus, the term $\tmext{A}{P}{x}{u}$ is the pairing of $x$ and $u$. Note that because
we have the judgmental equality 
$\ctxwk{P}{{A}{\ctxext{A}{P}}}\jdeq\ctxwk{\ctxext{A}{P}}{\ctxext{A}{P}}$ in the
context $\ctxext{{\Gamma}{A}}{P}$, the
pairing function could just be defined as $\idtm{\ctxext{A}{P}}$. 

When we substitute by an extended term we get an equal result as when we
substitute two consecutive times, like the way currying works.

\begin{lem}\label{comp-es}
The following inference rules are valid:
\begin{align*}
& \inference
  { \jterm{\Gamma}{A}{x}
    \jterm{\Gamma}{\subst{x}{P}}{u}
    \jfam{{{\Gamma}{A}}{P}}{Q}
    }
  { \jfameq
      {\Gamma}
      {\subst{\tmext{A}{P}{x}{u}}{Q}}
      {\subst{u}{{x}{Q}}}
    }
  \\
& \inference
  { \jterm{\Gamma}{A}{x}
    \jterm{\Gamma}{\subst{x}{P}}{u}
    \jfam{{{{\Gamma}{A}}{P}}{Q}}{R}
    }
  { \jfameq
      {{\Gamma}{\subst{u}{{x}{Q}}}}
      {\subst{\tmext{A}{P}{x}{u}}{R}}
      {\subst{u}{{x}{R}}}
    }
  \\
& \inference
  { \jterm{\Gamma}{A}{x}
    \jterm{\Gamma}{\subst{x}{P}}{u}
    \jterm{{{{\Gamma}{A}}{P}}{Q}}{R}{t}
    }
  { \jtermeq
      {{\Gamma}{\subst{u}{{x}{Q}}}}
      {\subst{u}{{x}{R}}}
      {\subst{\tmext{A}{P}{x}{u}}{h}}
      {\subst{u}{{x}{h}}}
    }
\end{align*}
\end{lem}

\begin{proof}
We prove only the first judgmental equality. All the others are similar.
Let $\jterm{\Gamma}{A}{x}$ and $\jterm{\Gamma}{\subst{x}{P}}{u}$
be terms and let $\jfam{{{\Gamma}{A}}{P}}{Q}$ be a family. Then we have
\begin{align*}
\subst
  {\tmext{A}{P}{x}{u}}
  {Q} 
& \jdeq 
  \subst
    {{u}{{x}{\idtm{\ctxext{A}{P}}}}}
    {Q}
  \tag{by definition}\\
& \jdeq 
  \subst
    {{u}{{x}{\idtm{\ctxext{A}{P}}}}}
    {{x}{\ctxwk{A}{Q}}}
  \tag{by \autoref{cancellation-ws-f}}\\
& \jdeq 
  \subst
    {{u}{{x}{\idtm{\ctxext{A}{P}}}}}
    {{u}{\ctxwk{\subst{x}{P}}{\subst{x}{\ctxwk{A}{Q}}}}}
  \tag{by \autoref{cancellation-ws-f}}\\
& \jdeq 
  \subst
    {{u}{{x}{\idtm{\ctxext{A}{P}}}}}
    {{u}{{x}{\ctxwk{P}{{A}{Q}}}}}
  \tag{by \autoref{comp-sw-f}}\\
& \jdeq 
  \subst
    {u}
    {{{x}{\idtm{\ctxext{A}{P}}}}{{x}{\ctxwk{{P}{{A}{Q}}}}}}
  \tag{by \autoref{comp-ss-f}}\\
& \jdeq 
  \subst
    {u}
    {{x}{{\idtm{\ctxext{A}{P}}}{\ctxwk{P}{{A}{Q}}}}}
  \tag{by \autoref{comp-ss-f}}\\
& \jdeq 
  \subst
    {u}
    {{x}{{\idtm{\ctxext{A}{P}}}{\ctxwk{\ctxext{A}{P}}{Q}}}}
  \tag{by \autoref{comp-ew-f}}\\
& \jdeq 
  \subst
    {u}
    {{x}{Q}}
  \tag{by \autoref{idfunc-wk-defn}}
\end{align*}
\end{proof}

We have seen above that the pairing function into $\ctxext{A}{P}$ is just the identity term on
$\ctxext{A}{P}$. To analyze the pairing functin a little further, we will also
need the projection maps from $\ctxext{A}{P}$ to $A$ and from $\ctxext{A}{P}$
to $P$. We will now define these and see that the identity term of an
extended family is the extension (or pairing) of the identity
functions on the components in the apropriate way.

To find out what the
apropriate way is, note that
\begin{align*}
\ctxwk{\ctxext{A}{P}}{\ctxext{A}{P}}
& \jdeq
  \ctxext{\ctxwk{\ctxext{A}{P}}{A}}{\ctxwk{\ctxext{A}{P}}{P}}
  \tag{by \autoref{comp-we-f}}
  \\
& \jdeq
  \ctxext{\ctxwk{P}{{A}{A}}}{\ctxwk{P}{{A}{P}}}
  \tag{by \autoref{comp-ew-f}}
\end{align*}
We have the term $\jterm{{\Gamma}{{A}{P}}}{\ctxwk{P}{A}}{\ctxwk{P}{\idtm{A}}}$.
Thus we need to find out what $\subst{\ctxwk{P}{\idtm{A}}}{\ctxwk{P}{{A}{P}}}$ is:
\begin{align*}
\subst{\ctxwk{P}{\idtm{A}}}{\ctxwk{P}{{A}{P}}} 
& \jdeq 
  \ctxwk{P}{\subst{\idtm{A}}{\ctxwk{A}{P}}}
  \tag{by \autoref{comp-ws-f}}
  \\
& \jdeq 
  \ctxwk{P}{P},
  \tag{by \autoref{idfunc-wk-defn}}
\end{align*}
where we find the term $\idtm{P}$. Therefore we define:

\begin{defn}
Let $\jfam{\Gamma}{A}$ and $\jfam{{\Gamma}{A}}{P}$ be families. We define
\begin{align*}
\jhomdefn*
  {\Gamma}
  {{A}{P}}
  {A}
  {\cprojfstf{A}{P}}
  {\unfold{\cprojfstf{A}{P}}}
  \\
\jtermdefn*
  {\ctxext{\Gamma}{{A}{P}}}
  {\ctxwk{P}{P}}
  {\cprojsndf{A}{P}}
  {\unfold{\cprojsndf{A}{P}}}
\end{align*}
\end{defn}

The constructions of the terms $\tmext{A}{P}{x}{u}$ and $\cprojfst{A}{P}{w}$ and
$\cprojsnd{A}{P}{w}$ are subject to various rules, with all of them being
consequences of earlier introduced inference rules.

\begin{lem}\label{lem:tmext-basic}
The following inference rules expressing that pairing is a strict
inverse to the combination of decompositions, are valid:
\begin{align*}
& \inference
  { \jterm{\Gamma}{\ctxext{A}{P}}{w}
    }
  { \jtermeq
      {\Gamma}
      {\ctxext{A}{P}}
      {\tmext{A}{P}{\cprojfst{A}{P}{w}}{\cprojsnd{A}{P}{w}}}
      {w}
    }
  \\
& \inference
  { \jterm{\Gamma}{A}{x}
    \jterm{\Gamma}{\subst{x}{P}}{u}
    }
  { \jtermeq
      {\Gamma}
      {A}
      {\cprojfst{A}{P}{\tmext{A}{P}{x}{u}}}
      {x}
    }
  \\
& \inference
  { \jterm{\Gamma}{A}{x}
    \jterm{\Gamma}{\subst{x}{P}}{u}
    }
  { \jtermeq
      {\Gamma}
      {\subst{x}{P}}
      {\cprojsnd{A}{P}{\tmext{A}{P}{x}{u}}}
      {u}
    }
\end{align*}
\end{lem}

\begin{proof}
To prove the first judgmental equality, note that
\begin{align*}
w 
& \jdeq 
  \subst{w}{\idtm{\ctxext{A}{P}}} 
  \tag{by \autoref{cancellation-si}}\\
& \jdeq 
  \subst
    { w}
    { { \idtm{P}}
      { { \ctxwk{P}{\idtm{A}}}
        { \idtm{\ctxwk{\ctxext{A}{P}}{\ctxext{A}{P}}}}
        }
      }
  \tag{by \autoref{idfunc-ext-comp}}\\
& \jdeq 
  \subst
    { {w}
      {\idtm{P}}
      }
    { {w}
      { { \ctxwk{P}{\idtm{A}}
          }
        { \idtm{\ctxwk{\ctxext{A}{P}}{\ctxext{A}{P}}}
          }
        }
      }
  \tag{by \autoref{comp-ss-t}}\\
& \jdeq 
  \subst
    { {w}
      {\idtm{P}}
      }
    { { {w}
        {\ctxwk{P}{\idtm{A}}}
        }
      { {w}
        {\idtm{\ctxwk{\ctxext{A}{P}}{\ctxext{A}{P}}}}
        }
      }
  \tag{by \autoref{comp-ss-t}}\\
& \jdeq 
  \subst
    { {w}
      {\idtm{P}}
      }
    { { {w}
        {\ctxwk{P}{\idtm{A}}}
        }
      { {w}
        {\ctxwk{\ctxext{A}{P}}{\idtm{\ctxext{A}{P}}}}
        }
      }
  \tag{by \autoref{comp-wi-t}}\\
& \jdeq 
  \subst
    { {w}
      {\idtm{P}}
      }
    { { {w}
        {\ctxwk{P}{\idtm{A}}}
        }
      { \idtm{\ctxext{A}{P}}
        }
      }
  \tag{by \autoref{cancellation-ws-t}}\\
& \jdeq 
  \tmext{A}{P}{\cprojfst{A}{P}{w}}{\cprojsnd{A}{P}{w}}
  \tag{by definition}
\end{align*}
To prove the second judgmental equality, let $\jterm{\Gamma}{A}{x}$ and
$\jterm{\Gamma}{\subst{x}{P}}{u}$. Then we have
\begin{align*}
\cprojfst{A}{P}{\tmext{A}{P}{x}{u}}
& \jdeq 
  \subst{\tmext{A}{P}{x}{u}}{\ctxwk{P}{\idtm{A}}}
  \tag{by definition}
  \\
& \jdeq 
  \subst{u}{{x}{\ctxwk{P}{\idtm{A}}}} 
  \tag{by \autoref{comp-es}}
  \\
& \jdeq 
  \subst{u}{\ctxwk{\subst{x}{P}}{\subst{x}{\idtm{A}}}}
  \tag{by \autoref{comp-sw-t}}
  \\
& \jdeq 
  \subst{x}{\idtm{A}}
  \tag{by \autoref{cancellation-ws-t}}
  \\
& \jdeq 
  x.
  \tag{by \autoref{cancellation-si}}
\end{align*}
To prove the third judgmental equality, note that
\begin{align*}
\cprojsnd{A}{P}{\tmext{A}{P}{x}{u}}
& \jdeq 
  \subst{\tmext{A}{P}{x}{u}}{\idtm{P}}
  \tag{by definition}
  \\
& \jdeq 
  \subst{u}{{x}{\idtm{P}}}
  \tag{by \autoref{comp-es}}
  \\
& \jdeq 
  \subst{u}{\idtm{\subst{x}{P}}}
  \tag{by \autoref{comp-si-t}}
  \\
& \jdeq 
  u.
  \tag{by \autoref{cancellation-si}}
\end{align*}
\end{proof}

In \autoref{lem:tmext-emp,lem:tmext-ext,lem:tmext-wk,lem:tmext-subst,lem:tmext-id}
we show that term extension and the projections are
compatible with the empty families, extension,
weakening, substitution and the identity terms (in that order). \autoref{lem:tmext-id}
is in fact a generalization of the above \autoref{lem:tmext-basic}.

\begin{lem}\label{lem:tmext-emp}
The following compatibility rules for extensions of the term of the empty family
are valid:
\begin{align*}
& \inference
  { \jterm{\Gamma}{A}{x}
    }
  { \jtermeq{\Gamma}{A}{\tmext{\emptytm}{x}}{x}
    }
& & \inference
  { \jterm{\Gamma}{A}{x}
    }
  { \jtermeq{\Gamma}{A}{\tmext{x}{\emptytm}}{x}
    }
  \\
& \inference
  { \jfam{\Gamma}{A}
    }
  { \jtermeq
      {{\Gamma}{A}}
      {\emptyf}
      {\cprojfstf{\emptyf}{A}}
      {\emptytm}
    }
& & \inference
  { \jfam{\Gamma}{A}
    }
  { \jhomeq
      {\Gamma}
      {A}
      {A}
      {\cprojfstf{A}{\emptyf}}
      {\idtm{A}}
    }
  \\
& \inference
  { \jfam{\Gamma}{A}
    }
  { \jhomeq
      {\Gamma}
      {A}
      {A}
      {\cprojsndf{\emptyf}{A}}
      {\idtm{A}}
    }
& & \inference
  { \jfam{\Gamma}{A}
    }
  { \jtermeq
      {{\Gamma}{A}}
      {\emptyf}
      {\cprojsndf{A}{\emptyf}}
      {\emptytm}
    }
\end{align*}
\end{lem}

\begin{proof}
These equalities are very easy to verify. We only display a proof of the first:
\begin{equation*}
\tmext{\emptytm}{x}
\jdeq \unfold{\tmext{\emptyf}{A}{\emptytm}{x}}
\jdeq \subst{x}{\idtm{{\emptyf}{A}}}
\jdeq \subst{x}{\idtm{A}}
\jdeq x.\qedhere
\end{equation*}
\end{proof}

\begin{lem}\label{lem:tmext-ext}
The following compatibility rules for two consecutive term extensions are valid:
\begin{align*}
& \inference
  { \jterm{\Gamma}{A}{x}
    \jterm{\Gamma}{\subst{x}{P}}{u}
    \jterm{\Gamma}{\subst{\tmext{A}{P}{x}{u}}{Q}}{v}
    }
  { \jtermeq
      {\Gamma}
      {\ctxext{{A}{P}}{Q}}
      {\tmext{A}{{P}{Q}}{x}{{\subst{x}{P}}{\subst{x}{Q}}{u}{v}}}
      {\tmext{{A}{P}}{Q}{{A}{P}{x}{u}}{v}}
    }
  \\
& \inference
  { \jfam{{{\Gamma}{A}}{P}}{Q}
    }
  { \jhomeq
      {\Gamma}
      {\ctxext{{A}{P}}{Q}}
      {A}
      {\jcomp{}{\cprojfstf{{A}{P}}{Q}}{\cprojfstf{A}{P}}}
      {\cprojfstf{A}{{P}{Q}}}
    }
  \\
& \inference
  { \jfam{{{\Gamma}{A}}{P}}{Q}
    }
  { \jhomeq
      {{\Gamma}{A}}
      {{P}{Q}}
      {P}
      {\jcomp{}{\cprojfstf{{A}{P}}{Q}}{\cprojsndf{A}{P}}}
      {\jcomp{}{\cprojsndf{A}{{P}{Q}}}{\cprojfstf{P}{Q}}}
    }
  \\
& \inference
    { \jfam{{{\Gamma}{A}}{P}}{Q}
      }
    { \jhomeq
        {{{\Gamma}{A}}{P}}
        {Q}
        {Q}
        {\cprojsndf{{A}{P}}{Q}}
        {\jcomp{}{\cprojsndf{A}{{P}{Q}}}{\cprojsndf{P}{Q}}}
      }
\end{align*}
\end{lem}

\begin{proof}
Consider terms $\jterm{\Gamma}{A}{x}$, $\jterm{\Gamma}{\subst{x}{P}}{u}$ and
$\jterm{\Gamma}{\subst{u}{{x}{Q}}}{v}$. Then we have
\begin{align*}
\tmext{x}{{u}{v}}
& \jdeq 
  \subst
    {\tmext{u}{v}}{{x}{\idtm{\ctxext{A}{{P}{Q}}}}}
  \\
& \jdeq 
  \subst{v}{{u}{{x}{\idtm{\ctxext{A}{{P}{Q}}}}}}
  \\
& \jdeq 
  \subst{v}{{u}{{x}{\idtm{\ctxext{{A}{P}}{Q}}}}}
  \\
& \jdeq 
  \subst{v}{{\tmext{x}{u}}{\idtm{\ctxext{{A}{P}}{Q}}}}
  \\
& \jdeq 
  \tmext{{x}{u}}{v}.
\end{align*}
To prove the judmental equality
\begin{equation*}
\jhomeq
  {\Gamma}
  {\ctxext{{A}{P}}{Q}}
  {A}
  {\jcomp{}{\cprojfstf{{A}{P}}{Q}}{\cprojfstf{A}{P}}}
  {\cprojfstf{A}{{P}{Q}}}
\end{equation*}
note that we have the judgmental equalities
\begin{align*}
\jcomp{{{A}{P}}{Q}}{\cprojfstf{{A}{P}}{Q}}{\cprojfstf{A}{P}}
& \jdeq 
  \unfoldall{\jcomp{{{A}{P}}{Q}}{\cprojfstf{{A}{P}}{Q}}{\cprojfstf{A}{P}}}
  \\
& \jdeq 
  \subst
    {\ctxwk{Q}{\idtm{{A}{P}}}}
    {\ctxwk{Q}{{\ctxext{A}{P}}{{P}{\idtm{A}}}}}
  \\
& \jdeq
  \ctxwk{Q}{\subst{\idtm{{A}{P}}}{\ctxwk{\ctxext{A}{P}}{{P}{\idtm{A}}}}}
  \\
& \jdeq
  \ctxwk{Q}{{P}{\idtm{A}}}
  \\
& \jdeq
  \ctxwk{\ctxext{P}{Q}}{\idtm{A}}
  \\
& \jdeq
  \cprojfstf{A}{{P}{Q}}
\end{align*}
To prove the judgmental equality
\begin{equation*}
\jhomeq
  {{\Gamma}{A}}
  {{P}{Q}}
  {P}
  {\jcomp{}{\cprojfstf{{A}{P}}{Q}}{\cprojsndf{A}{P}}}
  {\jcomp{}{\cprojsndf{A}{{P}{Q}}}{\cprojfstf{P}{Q}}}
\end{equation*}
note that we have the judgmental equalities
\begin{align*}
\jcomp{{{A}{P}}{Q}}{\cprojfstf{{A}{P}}{Q}}{\cprojsndf{A}{P}}
& \jdeq 
  \unfoldall{\jcomp{{{A}{P}}{Q}}{\cprojfstf{{A}{P}}{Q}}{\cprojsndf{A}{P}}}
  \\
& \jdeq
  \subst{\ctxwk{Q}{\idtm{{A}{P}}}}{\ctxwk{Q}{{\ctxext{A}{P}}{\idtm{P}}}}
  \\
& \jdeq
  \ctxwk{Q}{\subst{\idtm{{A}{P}}}{\ctxwk{\ctxext{A}{P}}{\idtm{P}}}}
  \\
& \jdeq
  \ctxwk{Q}{\idtm{P}}
  \\
& \jdeq
  \unfoldall{\jcomp{{P}{Q}}{\cprojsndf{A}{{P}{Q}}}{\cprojfstf{P}{Q}}}
  \\
& \jdeq
  \jcomp{{P}{Q}}{\cprojsndf{A}{{P}{Q}}}{\cprojfstf{P}{Q}}
\end{align*}
To prove the judgmental equality
\begin{equation*}
\jhomeq
  {{{\Gamma}{A}}{P}}
  {Q}
  {Q}
  {\cprojsndf{{A}{P}}{Q}}
  {\jcomp{}{\cprojsndf{A}{{P}{Q}}}{\cprojsndf{P}{Q}}}
\end{equation*}
note that we have the judgmental equalities
\begin{align*}
\cprojsndf{{A}{P}}{Q}
& \jdeq
  \unfoldall{\cprojsndf{{A}{P}}{Q}}
  \\
& \jdeq 
  \unfoldall{\jcomp{{P}{Q}}{\cprojsndf{A}{{P}{Q}}}{\cprojsndf{P}{Q}}}
  \\
& \jdeq
  \jcomp{}{\cprojsndf{A}{{P}{Q}}}{\cprojsndf{P}{Q}}\qedhere
\end{align*}
\begin{comment}
%%%% This was a proof of an old version of the statement
Now consider a term $\jterm{\Gamma}{\ctxext{A}{{P}{Q}}}{w}$. Then we have
\begin{align*}
w 
& \jdeq 
  \tmext
    {A}
    {{P}{Q}}
    {\cprojfst{A}{\ctxext{P}{Q}}{w}}
    {\cprojsnd{A}{\ctxext{P}{Q}}{w}}
  \\
& \jdeq 
  \tmext
    {A}
    {{P}{Q}}
    {\cprojfst{A}{\ctxext{P}{Q}}{w}}
    { {P}
      {Q}
      {\cprojfst{P}{Q}{\cprojsnd{A}{\ctxext{P}{Q}}{w}}}
      {\cprojsnd{P}{Q}{\cprojsnd{A}{\ctxext{P}{Q}}{w}}}
      }
  \\
& \jdeq 
  \tmext
    {{A}{P}}
    {Q}
    { {} % need to provide base and family, but there's no unfold.
      {}
      {\cprojfst{A}{\ctxext{P}{Q}}{w}}
      {\cprojfst{P}{Q}{\cprojsnd{A}{\ctxext{P}{Q}}{w}}}
      }
    { \cprojsnd{P}{Q}{\cprojsnd{A}{\ctxext{P}{Q}}{w}}
      }
\end{align*}
Thus we see that 
\begin{align*}
\cprojfst{\ctxext{A}{P}}{Q}{w} 
& \jdeq 
  \tmext
    {A}
    {P}
    {\cprojfst{A}{\ctxext{P}{Q}}{w}}
    {\cprojfst{P}{Q}{\cprojsnd{A}{\ctxext{P}{Q}}{w}}}
  \\ 
\cprojsnd{\ctxext{A}{P}}{Q}{w} 
& \jdeq 
  \cprojsnd{P}{Q}{\cprojsnd{A}{\ctxext{P}{Q}}{w}},
\end{align*}
proving the fourth judgmental equality, and therefore also that
\begin{align*}
\cprojfst{A}{P}{\cprojfst{\ctxext{A}{P}}{Q}{w}} 
& \jdeq 
  \cprojfst{A}{\ctxext{P}{Q}}{w}
  \\
\cprojsnd{A}{P}{\cprojfst{\ctxext{A}{P}}{Q}{w}} 
& \jdeq 
  \cprojfst{P}{Q}{\cprojsnd{A}{\ctxext{P}{Q}}{w}},
\end{align*}
proving the second and the third judgmental equalities.
\end{comment}
\end{proof}

\begin{lem}\label{lem:tmext-wk}\label{comp-we-t}
When we weaken a term $\tmext{B}{Q}{y}{v}$ of $\ctxext{B}{Q}$ in context 
$\Gamma$ by a family $A$, the term that we get is 
$\tmext{\ctxwk{A}{B}}{\ctxwk{A}{Q}}{\ctxwk{A}{y}}{\ctxwk{A}{v}}$. More
precisely, the following inference rules are valid:
\begin{align*}
& \inference
  { \jterm{{\Gamma}{B}}{Q}{g}
    \jterm{{\Gamma}{B}}{\subst{g}{R}}{t}
    }
  { \jtermeq
      {{{\Gamma}{A}}{\ctxwk{A}{B}}}
      {\ctxwk{A}{\ctxext{Q}{R}}}
      {\ctxwk{A}{\tmext{Q}{R}{g}{t}}}
      {\tmext{\ctxwk{A}{Q}}{\ctxwk{A}{R}}{\ctxwk{A}{g}}{\ctxwk{A}{t}}}
    }
  \\
& \inference
  { \jfam{{{\Gamma}{B}}{Q}}{R}
    }
  { \jhomeq
      {{{\Gamma}{A}}{\ctxwk{A}{B}}}
      {{\ctxwk{A}{Q}}{\ctxwk{A}{R}}}
      {\ctxwk{A}{Q}}
      {\cprojfstf{\ctxwk{A}{Q}}{\ctxwk{A}{R}}}
      {\ctxwk{A}{\cprojfstf{Q}{R}}}
    }
  \\
& \inference
  { \jfam{{{\Gamma}{B}}{Q}}{R}
    }
  { \jhomeq
      {{{{\Gamma}{A}}{\ctxwk{A}{B}}}{\ctxwk{A}{Q}}}
      {\ctxwk{A}{R}}
      {\ctxwk{A}{R}}
      {\cprojsndf{\ctxwk{A}{Q}}{\ctxwk{A}{R}}}
      {\ctxwk{A}{\cprojsndf{Q}{R}}}
    }
\end{align*}
\end{lem}

\begin{proof}
Consider $\jterm{{\Gamma}{B}}{Q}{g}$ and $\jterm{{\Gamma}{B}}{\subst{g}{R}}{t}$.
Then we have the judgmental equalities
\begin{align*}
\ctxwk{A}{\ctxext{Q}{R}{g}{t}}
& \jdeq 
  \ctxwk{A}{\subst{t}{{g}{\idtm{\ctxext{Q}{R}}}}}
  \\
& \jdeq 
  \subst{\ctxwk{A}{t}}{\ctxwk{A}{\subst{g}{\idtm{\ctxext{Q}{R}}}}}
  \\
& \jdeq 
  \subst{\ctxwk{A}{t}}{{\ctxwk{A}{g}}{\ctxwk{A}{\idtm{\ctxext{Q}{R}}}}}
  \\
& \jdeq 
  \subst{\ctxwk{A}{t}}{{\ctxwk{A}{g}}{\idtm{\ctxwk{A}{\ctxext{Q}{R}}}}}
  \\
& \jdeq 
  \subst{\ctxwk{A}{t}}{{\ctxwk{A}{g}}{\idtm{\ctxext{\ctxwk{A}{Q}}{\ctxwk{A}{R}}}}}
  \\
& \jdeq 
  \tmext{\ctxwk{A}{Q}}{\ctxwk{A}{R}}{\ctxwk{A}{g}}{\ctxwk{A}{t}}
\end{align*}
Next, we want to prove the judgmental equality
\begin{equation*}
\jhomeq
  {{{\Gamma}{A}}{\ctxwk{A}{B}}}
  {{\ctxwk{A}{Q}}{\ctxwk{A}{R}}}
  {\ctxwk{A}{Q}}
  {\cprojfstf{\ctxwk{A}{Q}}{\ctxwk{A}{R}}}
  {\ctxwk{A}{\cprojfstf{Q}{R}}}
\end{equation*}
Note that we have the judgmental equalities
\begin{align*}
\cprojfstf{\ctxwk{A}{Q}}{\ctxwk{A}{R}}
& \jdeq
  \unfoldall{\cprojfstf{\ctxwk{A}{Q}}{\ctxwk{A}{R}}}
  \\
& \jdeq
  \ctxwk{{A}{R}}{{A}{\idtm{Q}}}
  \\
& \jdeq
  \unfoldall{\ctxwk{A}{\cprojfstf{Q}{R}}}
  \\
& \jdeq
  \ctxwk{A}{\cprojfstf{Q}{R}}.
\end{align*}
Finally, we want to prove the judgmental equality
\begin{equation*}
\jhomeq
  {{{{\Gamma}{A}}{\ctxwk{A}{B}}}{\ctxwk{A}{Q}}}
  {\ctxwk{A}{R}}
  {\ctxwk{A}{R}}
  {\cprojsndf{\ctxwk{A}{Q}}{\ctxwk{A}{R}}}
  {\ctxwk{A}{\cprojsndf{Q}{R}}}
\end{equation*}
Note that we have the judgmental equalities
\begin{align*}
\cprojsndf{\ctxwk{A}{Q}}{\ctxwk{A}{R}}
& \jdeq
  \unfoldall{\cprojsndf{\ctxwk{A}{Q}}{\ctxwk{A}{R}}}
  \\
& \jdeq
  \unfoldall{\ctxwk{A}{\cprojsndf{Q}{R}}}
  \\
& \jdeq
  \ctxwk{A}{\cprojsndf{Q}{R}}.
  \qedhere
\end{align*}
\end{proof}

\begin{lem}\label{lem:tmext-subst}\label{comp-se-t}
When we substitute an extended term $\tmext{P}{Q}{f}{g}$ of $\ctxext{P}{Q}$ by a term
$x$ of $A$, the term that we get is $\tmext{\subst{x}{P}}{\subst{x}{Q}}{\subst{x}{f}}{\subst{x}{g}}$.
More precisely, the following inference rules are valid:
\begin{align*}
& \inference
  { \jterm{\Gamma}{A}{x}
    \jterm{{{\Gamma}{A}}{P}}{Q}{g}
    \jterm{{{\Gamma}{A}}{P}}{\subst{g}{R}}{t}
    }
  { \jtermeq
      {{\Gamma}{\subst{x}{P}}}
      {\ctxext{\subst{x}{Q}}{\subst{x}{R}}}
      {\subst{x}{\tmext{Q}{R}{g}{t}}}
      {\tmext{\subst{x}{Q}}{\subst{x}{R}}{\subst{x}{g}}{\subst{x}{t}}}
    }
  \\
& \inference
  { \jterm{\Gamma}{A}{x}
    \jfam{{{{\Gamma}{A}}{P}}{Q}}{R}
    }
  { \jhomeq
      {{\Gamma}{\subst{x}{P}}}
      {{\subst{x}{Q}}{\subst{x}{R}}}
      {\subst{x}{Q}}
      {\cprojfstf{\subst{x}{Q}}{\subst{x}{R}}}
      {\subst{x}{\cprojfstf{Q}{R}}}
    }
  \\
& \inference
  { \jterm{\Gamma}{A}{x}
    \jfam{{{{\Gamma}{A}}{P}}{Q}}{R}
    }
  { \jhomeq
      {{{\Gamma}{\subst{x}{P}}}{\subst{x}{Q}}}
      {\subst{x}{R}}
      {\subst{x}{R}}
      {\cprojsndf{\subst{x}{Q}}{\subst{x}{R}}}
      {\subst{x}{\cprojsndf{Q}{R}}}
    }
\end{align*}
\end{lem}

\begin{proof}
Consider $\jterm{{\Gamma}{B}}{Q}{g}$ and $\jterm{{\Gamma}{B}}{\subst{g}{R}}{t}$.
Then we have the judgmental equalities
\begin{align*}
\subst{x}{\tmext{Q}{R}{g}{t}}
& \jdeq 
  \subst{x}{{t}{{g}{\idtm{\ctxext{Q}{R}}}}}
  \\
& \jdeq 
  \subst{{x}{t}}{{x}{{g}{\idtm{\ctxext{Q}{R}}}}}
  \\
& \jdeq 
  \subst{{x}{t}}{{{x}{g}}{{x}{\idtm{\ctxext{Q}{R}}}}}
  \\
& \jdeq 
  \subst{{x}{t}}{{{x}{g}}{\idtm{\subst{x}{\ctxext{Q}{R}}}}}
  \\
& \jdeq 
  \subst{{x}{t}}{{{x}{g}}{\idtm{\ctxext{\subst{x}{Q}}{\subst{x}{R}}}}}
  \\
& \jdeq 
  \tmext{\subst{x}{Q}}{\subst{x}{R}}{\subst{x}{g}}{\subst{x}{t}}.
\end{align*}
Next, we want to prove the judgmental equality
\begin{equation*}
\jhomeq
  {{\Gamma}{\subst{x}{P}}}
  {{\subst{x}{Q}}{\subst{x}{R}}}
  {\subst{x}{Q}}
  {\cprojfstf{\subst{x}{Q}}{\subst{x}{R}}}
  {\subst{x}{\cprojfstf{Q}{R}}}
\end{equation*}
Note that we have the judgmental equalities
\begin{align*}
\cprojfstf{\subst{x}{Q}}{\subst{x}{R}}
& \jdeq
  \unfoldall{\cprojfstf{\subst{x}{Q}}{\subst{x}{R}}}
  \\
& \jdeq
  \ctxwk{\subst{x}{R}}{\subst{x}{\idtm{Q}}}
  \\
& \jdeq
  \unfoldall{\subst{x}{\cprojfstf{Q}{R}}}
  \\
& \jdeq
  \subst{x}{\cprojfstf{Q}{R}}.
\end{align*}
And finally we want to prove the judgmental equality
\begin{equation*}
\jhomeq
  {{{\Gamma}{\subst{x}{P}}}{\subst{x}{Q}}}
  {\subst{x}{R}}
  {\subst{x}{R}}
  {\cprojsndf{\subst{x}{Q}}{\subst{x}{R}}}
  {\subst{x}{\cprojsndf{Q}{R}}}
\end{equation*}
Note that we have the judgmental equalities
\begin{align*}
\cprojsndf{\subst{x}{Q}}{\subst{x}{R}}
& \jdeq
  \unfoldall{\cprojsndf{\subst{x}{Q}}{\subst{x}{R}}}
  \\
& \jdeq
  \unfoldall{\subst{x}{\cprojsndf{Q}{R}}}
  \\
& \jdeq
  \subst{x}{\cprojsndf{Q}{R}}.
  \qedhere
\end{align*}
\end{proof}

We find the following inference rule, which expresses that the identity term
is compatible with extension:

\begin{lem}\label{lem:tmext-id}\label{comp-ie}
For any $\jfam{\Gamma}{A}$ and $\jfam{{\Gamma}{A}}{P}$ we have
\begin{equation}\label{idfunc-ext-comp}
\inference
  { \jfam{\Gamma}{A}
    \jfam{{\Gamma}{A}}{P}
    }
  { \jhomeq
      {\Gamma}
      {{A}{P}}{{A}{P}}
      {\idtm{\ctxext{A}{P}}}
      { \tmext
          {\ctxwk{\ctxext{A}{P}}{A}}
          {\ctxwk{\ctxext{A}{P}}{P}}
          {\cprojfstf{A}{P}}
          {\cprojsndf{A}{P}}
        }
    }
\end{equation}
\end{lem}

\begin{proof}
Consider the families $\jfam{\Gamma}{A}$ and $\jfam{{\Gamma}{A}}{P}$. Then
we have the judgmental equalities
\begin{align*}
\tmext
  {\ctxwk{\ctxext{A}{P}}{A}}
  {\ctxwk{\ctxext{A}{P}}{P}}
  {\cprojfstf{A}{P}}
  {\cprojsndf{A}{P}}
& \jdeq 
  \unfold
  { \tmext
      {\ctxwk{\ctxext{A}{P}}{A}}
      {\ctxwk{\ctxext{A}{P}}{P}}
      {\cprojfstf{A}{P}}
      {\cprojsndf{A}{P}}
    }
  \\
& \jdeq 
  \subst
    { \idtm{P}
      }
    { {\ctxwk{P}{\idtm{A}}}
      {\idtm{\ctxwk{\ctxext{A}{P}}{\ctxext{A}{P}}}}
      }
  \\
& \jdeq 
  \subst
    { \idtm{P}
      }
    { {\ctxwk{P}{\idtm{A}}}
      {\ctxwk{\ctxext{A}{P}}{\idtm{\ctxext{A}{P}}}}
      }
  \\
& \jdeq 
  \subst
    { \idtm{P}
      }
    { {\ctxwk{P}{\idtm{A}}}
      {\ctxwk{P}{{A}{\idtm{\ctxext{A}{P}}}}}
      }
  \\
& \jdeq 
  \subst
    { \idtm{P}
      }
    {\ctxwk
      {P}
      { \subst
        {\idtm{A}}
        {\ctxwk{A}{\idtm{\ctxext{A}{P}}}}
        }
      }
  \\
& \jdeq 
  \subst
    { \idtm{A}
      }
    { \ctxwk{A}{\idtm{\ctxext{A}{P}}}
      }
  \\
& \jdeq 
  \idtm{\ctxext{A}{P}}.
  \qedhere
\end{align*}
\end{proof}

\subsection{Family morphisms}
In this section we are going to develop dependent morphisms and analyze various
sorts of composition. The notion of family morphism is the notion of `a 
morphism over a context morphism'. When we have established all the associativity
and interchange laws for all the compositions that come with the theory of
family morphisms, we will also be able to consider \emph{family diagrams}. Those
are graphical displays of situations in the theory of contexts, families and
terms, but unlike diagrams in category theory they can contain families (which
are strictly speaking not morphisms) and terms (which can be seen as morphisms,
but not in the way we're about to display them).

For the following lemma, recall that the judgment $\jhom{\Gamma}{A}{{B}{Q}}{f}$
unfolds as
\begin{equation*}
\unfold{\jhom{\Gamma}{A}{{B}{Q}}{f}}
\end{equation*}
and that we have the judgmental equality 
$ \jfameq
    {{\Gamma}{A}}
    {\ctxwk{A}{\ctxext{B}{Q}}}
    {\ctxext{\ctxwk{A}{B}}{\ctxwk{A}{Q}}}.
  $
Therefore, each morphism into an extended family can itself be described as
an extended term. The following lemma explains how this goes.

\begin{lem}
Let $\jhom{\Gamma}{A}{{B}{Q}}{f}$ be a morphism from $A$ to $\ctxext{B}{Q}$
in a context $\Gamma$. Then we have
\begin{equation*}
\jhomeq
  {\Gamma}
  {A}
  {{B}{Q}}
  {f}
  {\tmext{\jcomp{}{f}{\cprojfstf{B}{Q}}}{\jcomp{}{f}{\cprojsndf{B}{Q}}}}.
\end{equation*}
Alternatively, when $\jhom{\Gamma}{A}{B}{f_0}$ and 
$\jterm{{\Gamma}{A}}{\jcomp{}{f_0}{Q}}{f_1}$ we obtain a morphism
$\jhom{\Gamma}{A}{{B}{Q}}{\tmext{f_0}{f_1}}$ with the property that
\begin{align*}
\jhomeq*{\Gamma}{A}{B}{\jcomp{}{\tmext{f_0}{f_1}}{\cprojfstf{B}{Q}}}{f_0}\\
\jtermeq*{{\Gamma}{A}}{\jcomp{}{f_0}{Q}}{\jcomp{}{\tmext{f_0}{f_1}}{\cprojsndf{B}{Q}}}{f_1}.
\end{align*}
\end{lem}

\begin{proof}
Let $\jhom{\Gamma}{A}{{B}{Q}}{f}$ be a morphism in context $\Gamma$. Then we
have the judgmental equalities
\begin{align*}
\cprojfst{\ctxwk{A}{B}}{\ctxwk{A}{Q}}{f}
& \jdeq
  \subst{f}{\ctxwk{A}{\cprojfstf{B}{Q}}}
  \\
& \jdeq
  \jcomp{}{f}{\cprojfstf{B}{Q}}
  \\
\cprojsnd{\ctxwk{A}{B}}{\ctxwk{A}{Q}}{f}
& \jdeq
  \subst{f}{\ctxwk{A}{\cprojsndf{B}{Q}}}
  \\
& \jdeq
  \jcomp{}{f}{\cprojsndf{B}{Q}}
\end{align*}
The alternative formulation of the statement is a direct corollary.
\end{proof}

We also have the following lemma about the compatibility of pairing and composition:

\begin{lem}\label{lem:tmext-jcomp}
The following inference rule is valid
\begin{align*}
& \inference
  { \jhom{\Gamma}{A}{B}{f}
    \jhom{\Gamma}{B}{C}{g}
    \jfam{{\Gamma}{C}}{R}
    \jterm{{\Gamma}{B}}{\subst{g}{\ctxwk{B}{R}}}{w}
    }
  { \jhomeq
      {\Gamma}
      {A}
      {{C}{R}}
      {\jcomp{A}{f}{\tmext{\ctxwk{B}{C}}{\ctxwk{B}{R}}{g}{w}}}
      {\tmext{\ctxwk{A}{C}}{\ctxwk{A}{R}}{\jcomp{A}{f}{g}}{\jcomp{A}{f}{w}}}
    }
\end{align*}
\end{lem}

\begin{proof}
Let $\jhom{\Gamma}{A}{B}{f}$, $\jhom{\Gamma}{B}{C}{g}$, $\jfam{{\Gamma}{C}}{R}$
and $\jterm{{\Gamma}{B}}{\subst{g}{\ctxwk{B}{R}}}{w}$. Then we have the
judgmental equalities
\begin{align*}
\jcomp{A}{f}{\tmext{\ctxwk{B}{C}}{\ctxwk{B}{R}}{g}{w}}
& \jdeq 
  \subst{f}{\ctxwk{A}{\tmext{\ctxwk{B}{C}}{\ctxwk{B}{R}}{g}{w}}}
  \\
& \jdeq 
  \subst
    {f}
    {\tmext{\ctxwk{A}{{B}{C}}}{\ctxwk{A}{{B}{R}}}{\ctxwk{A}{g}}{\ctxwk{A}{w}}}
  \\
& \jdeq 
  \tmext
    {\ctxwk{A}{C}}
    {\ctxwk{A}{R}}
    {\subst{f}{\ctxwk{A}{g}}}
    {\subst{f}{\ctxwk{A}{w}}}
  \\
& \jdeq 
  \tmext{\ctxwk{A}{C}}{\ctxwk{A}{R}}{\jcomp{A}{f}{g}}{\jcomp{A}{f}{w}}.
  \qedhere
\end{align*}
\end{proof}

There is a notion of morphism \emph{over} a morphism. We will develop this
notion because it will be needed in the theory of models later on.

\begin{defn}
Let $\jhom{\Gamma}{A}{B}{f}$ be a morphism from $A$ to $B$ in context $\Gamma$
and consider $\jfam{{\Gamma}{A}}{P}$ and $\jfam{{\Gamma}{B}}{Q}$. We define the
judgment\begin{equation*}
\jfhom{\Gamma}{A}{B}{f}{P}{Q}{F},
\end{equation*}
which is pronounced as `$F$ is a morphism from $P$ to $Q$ over $f$ in context
$\Gamma$', to be the judgment\begin{equation*}
\unfold{\jfhom{\Gamma}{A}{B}{f}{P}{Q}{F}}.
\end{equation*}
\end{defn}

\begin{rmk}
The judgment $\jfhom{\Gamma}{A}{B}{f}{P}{Q}{F}$ means the same thing as
\begin{equation*}
\jhom{{\Gamma}{A}}{P}{\jcomp{A}{f}{Q}}{F}.
\end{equation*}
Thus we see that a morphism from $P$ to $Q$ over the identity term $\idtm{A}$ in
context $\Gamma$ is the same thing as a morphism from $P$ to $Q$ in context
$\ctxext{\Gamma}{A}$, i.e.~the following inference rules are valid:
\begin{align*}
& \inference
  { \jfam{{\Gamma}{A}}{P}
    \jfam{{\Gamma}{A}}{Q}
    \jfhom{\Gamma}{A}{A}{\idtm{A}}{P}{Q}{f}
    }
  { \jhom{{\Gamma}{A}}{P}{Q}{f}
    }
  \\
& \inference
  { \jfam{{\Gamma}{A}}{P}
    \jfam{{\Gamma}{A}}{Q}
    \jhom{{\Gamma}{A}}{P}{Q}{f}
    }
  { \jfhom{\Gamma}{A}{A}{\idtm{A}}{P}{Q}{f}
    }
\end{align*}
To see this, we only have to note that
$\ctxwk{P}{\subst{\idtm{A}}{\ctxwk{A}{Q}}}\jdeq\ctxwk{P}{Q}$, which
holds by \autoref{idfunc-wk-defn}.
\end{rmk}

Suppose we have morphisms $\jhom{\Gamma}{A}{B}{f}$ and $\jhom{\Gamma}{B}{C}{g}$
and that we have the morphisms $\jfhom{\Gamma}{A}{B}{f}{P}{Q}{F}$ and
$\jfhom{\Gamma}{B}{C}{g}{Q}{R}{G}$ over them. Then we have\begin{equation*}
\jhom
  {{\Gamma}{A}}
  {\jcomp{A}{f}{Q}}
  {\jcomp{A}{f}{{B}{g}{R}}}
  {\jcomp{A}{f}{G}}
\end{equation*}
Because we also have $\jhom{{\Gamma}{A}}{P}{\jcomp{A}{f}{Q}}{F}$, we have the
composition\begin{equation*}
\jhom
  {{\Gamma}{A}}
  {P}
  {\jcomp{A}{f}{{B}{g}{R}}}
  {\jcomp{P}{F}{\jcomp{A}{f}{G}}}.
\end{equation*}
Because of 
the judgmental equality $\jcomp{A}{f}{{B}{g}{R}}\jdeq
\jcomp{A}{{A}{f}{g}}{R}$, it follows that 
$\jcomp{P}{F}{\jcomp{A}{f}{G}}$ is a morphism from $P$ to $R$ over
$\jcomp{A}{f}{g}$. This could be considered as the composition of $G$ with $F$.
In the following definition, we formulate this more generally.

\begin{defn}
\emph{Horizontal composition of morphisms over morphisms} is defined by
\begin{align*}
& \inference
  { \jfhom{\Gamma}{A}{B}{f}{P}{Q}{F}
    \jfam{{{\Gamma}{B}}{Q}}{R}
    }
  { \jfamdefn
      {{{\Gamma}{A}}{P}}
      {\jfcomp{A}{f}{P}{F}{R}}
      {\unfold{\jfcomp{A}{f}{P}{F}{R}}}
    }
  \\
& \inference
  { \jfhom{\Gamma}{A}{B}{f}{P}{Q}{F}
    \jfam{{{{\Gamma}{B}}{Q}}{R}}{S}
    }
  { \jfamdefn
      {{{{\Gamma}{A}}{P}}{\jfcomp{A}{f}{P}{F}{R}}}
      {\jfcomp{A}{f}{P}{F}{S}}
      {\unfold{\jfcomp{A}{f}{P}{F}{S}}}
    }
  \\
& \inference
  { \jfhom{\Gamma}{A}{B}{f}{P}{Q}{F}
    \jterm{{{{\Gamma}{B}}{Q}}{R}}{S}{k}
    }
  { \jtermdefn
      {{{{\Gamma}{A}}{P}}{\jfcomp{A}{f}{P}{F}{R}}}
      {\jfcomp{A}{f}{P}{F}{S}}
      {\jfcomp{A}{f}{P}{F}{k}}
      {\unfold{\jfcomp{A}{f}{P}{F}{S}}}
    }
\end{align*}
\end{defn}

Since horizontal composition is just consecutive composition, we will get the
compatibility with the empty context, extension, weakening, substitution and
the identity terms for free. We will state these compatibility properities in
\autoref{lem:jfcomp-emp,lem:jfcomp-ext,lem:jfcomp-wk,lem:jfcomp-subst,lem:jfcomp-idtm},
but the proofs are left to the reader.

\begin{lem}\label{lem:jfcomp-emp}
Horizontal composition is compatible with the empty context:
\begin{align*}
& \inference
  { \jfhom{\Gamma}{A}{B}{f}{P}{Q}{F}
    }
  { \jfameq
      {{{\Gamma}{A}}{P}}
      {\jfcomp{A}{f}{P}{F}{\emptyf}}
      {\emptyf}
    }
  \\
& \inference
  { \jfhom{\Gamma}{A}{B}{f}{P}{Q}{F}
    \jfam{{{\Gamma}{B}}{Q}}{R}
    }
  { \jfameq
      {{{{\Gamma}{A}}{P}}{\jfcomp{A}{f}{P}{F}{R}}}
      {\jfcomp{A}{f}{P}{F}{\emptyf}}
      {\emptyf}
    }
\end{align*}
\end{lem}

\begin{lem}\label{lem:jfcomp-ext}
Horizontal composition is compatible with extension:
\begin{align*}
& \inference
  { \jfhom{\Gamma}{A}{B}{f}{P}{Q}{F}
    \jfam{{{{\Gamma}{B}}{Q}}{R}}{S}
    }
  { \jfameq
      {{{\Gamma}{A}}{P}}
      {\jfcomp{A}{f}{P}{F}{\ctxext{R}{S}}}
      {\ctxext{\jfcomp{A}{f}{P}{F}{R}}{\jfcomp{A}{f}{P}{F}{S}}}
    }
  \\
& \inference
  { \jfhom{\Gamma}{A}{B}{f}{P}{Q}{F}
    \jfam{{{{{\Gamma}{B}}{Q}}{R}}{S}}{T}
    }
  { \jfameq
      {{{{\Gamma}{A}}{P}}{\jfcomp{A}{f}{P}{F}{R}}}
      {\jfcomp{A}{f}{P}{F}{\ctxext{S}{T}}}
      {\ctxext{\jfcomp{A}{f}{P}{F}{S}}{\jfcomp{A}{f}{P}{F}{T}}}
    }
  \\
& \inference
  { \jfhom{\Gamma}{A}{B}{f}{P}{Q}{F}
    \jterm{{{{\Gamma}{B}}{Q}}{R}}{{S}{T}}{\tmext{k}{l}}
    }
  { \jtermdefn
      {{{{\Gamma}{A}}{P}}{\jfcomp{A}{f}{P}{F}{R}}}
      {\ctxext{\jfcomp{A}{f}{P}{F}{S}}{\jfcomp{A}{f}{P}{F}{T}}}
      {\jfcomp{A}{f}{P}{F}{\tmext{k}{l}}}
      {\tmext{\jfcomp{A}{f}{P}{F}{k}}{\jfcomp{A}{f}{P}{F}{l}}}
    }
\end{align*}
\end{lem}

\begin{lem}\label{lem:jfcomp-wk}
Horizontal composition is compatible with weakening:
\begin{align*}
& \inference
  { \jfhom{\Gamma}{A}{B}{f}{P}{Q}{F}
    \jfam{{{{\Gamma}{B}}{Q}}{R}}{S}
    \jfam{{{{\Gamma}{B}}{Q}}{R}}{T}
    }
  { \jfameq
      {{{{{\Gamma}{A}}{P}}{\jfcomp{A}{f}{P}{F}{R}}}{\jfcomp{A}{f}{P}{F}{S}}}
      {\jfcomp{A}{f}{P}{F}{\ctxwk{S}{T}}}
      {\ctxwk{\jfcomp{A}{f}{P}{F}{S}}{\jfcomp{A}{f}{P}{F}{T}}}
    }
  \\
& \inference
  { \jfhom{\Gamma}{A}{B}{f}{P}{Q}{F}
    \jfam{{{{\Gamma}{B}}{Q}}{R}}{S}
    \jfam{{{{{\Gamma}{B}}{Q}}{R}}{T}}{U}
    }
  { \jfameq
      {{{{{{\Gamma}{A}}{P}}
        {\jfcomp{A}{f}{P}{F}{R}}}
        {\jfcomp{A}{f}{P}{F}{S}}}
        {\ctxwk{\jfcomp{A}{f}{P}{F}{S}}{\jfcomp{A}{f}{P}{F}{T}}}}
      {\jfcomp{A}{f}{P}{F}{\ctxwk{S}{U}}}
      {\ctxwk{\jfcomp{A}{f}{P}{F}{S}}{\jfcomp{A}{f}{P}{F}{U}}}
    }
  \\
& \inference
  { \jfhom{\Gamma}{A}{B}{f}{P}{Q}{F}
    \jfam{{{{\Gamma}{B}}{Q}}{R}}{S}
    \jterm{{{{{\Gamma}{B}}{Q}}{R}}{T}}{U}{m}
    }
  { \jtermeq
      {{{{{{\Gamma}{A}}{P}}
        {\jfcomp{A}{f}{P}{F}{R}}}
        {\jfcomp{A}{f}{P}{F}{S}}}
        {\ctxwk{\jfcomp{A}{f}{P}{F}{S}}{\jfcomp{A}{f}{P}{F}{T}}}}
      {\jfcomp{A}{f}{P}{F}{\ctxwk{S}{U}}}
      {\jfcomp{A}{f}{P}{F}{\ctxwk{S}{m}}}
      {\ctxwk{\jfcomp{A}{f}{P}{F}{S}}{\jfcomp{A}{f}{P}{F}{m}}}
    }
\end{align*}
\end{lem}

\begin{lem}\label{lem:jfcomp-subst}
Horizontal composition is compatible with substitution:
\begin{align*}
& \inference
  { \jfhom{\Gamma}{A}{B}{f}{P}{Q}{F}
    \jterm{{{{\Gamma}{B}}{Q}}{R}}{S}{k}
    \jfam{{{{{\Gamma}{B}}{Q}}{R}}{S}}{T}
    }
  { \jfameq
      {{{{\Gamma}{A}}{P}}{\jfcomp{A}{f}{P}{F}{R}}}
      {\jfcomp{A}{f}{P}{F}{\subst{k}{T}}}
      {\subst{\jfcomp{A}{f}{P}{F}{k}}{\jfcomp{A}{f}{P}{F}{T}}}
    }
  \\
& \inference
  { \jfhom{\Gamma}{A}{B}{f}{P}{Q}{F}
    \jterm{{{{\Gamma}{B}}{Q}}{R}}{S}{k}
    \jfam{{{{{{\Gamma}{B}}{Q}}{R}}{S}}{T}}{U}
    }
  { \jfameq
      {{{{{\Gamma}{A}}{P}}{\jfcomp{A}{f}{P}{F}{R}}}{\jfcomp{A}{f}{P}{F}{\subst{k}{T}}}}
      {\jfcomp{A}{f}{P}{F}{\subst{k}{U}}}
      {\subst{\jfcomp{A}{f}{P}{F}{k}}{\jfcomp{A}{f}{P}{F}{U}}}
    }
  \\
& \inference
  { \jfhom{\Gamma}{A}{B}{f}{P}{Q}{F}
    \jterm{{{{\Gamma}{B}}{Q}}{R}}{S}{k}
    \jterm{{{{{{\Gamma}{B}}{Q}}{R}}{S}}{T}}{U}{m}
    }
  { \jfameq
      {{{{{\Gamma}{A}}{P}}{\jfcomp{A}{f}{P}{F}{R}}}{\jfcomp{A}{f}{P}{F}{\subst{k}{T}}}}
      {\jfcomp{A}{f}{P}{F}{\subst{k}{U}}}
      {\jfcomp{A}{f}{P}{F}{\subst{k}{m}}}
      {\subst{\jfcomp{A}{f}{P}{F}{k}}{\jfcomp{A}{f}{P}{F}{m}}}
    }
\end{align*}
\end{lem}

\begin{lem}\label{lem:jfcomp-idtm}
Horizontal composition is compatible with identity terms:
\begin{equation*}
\inference
  { \jfhom{\Gamma}{A}{B}{f}{P}{Q}{F}
    \jfam{{{{\Gamma}{B}}{Q}}{R}}{S}
    }
  { \jtermeq
      {{{{{\Gamma}{A}}{P}}{\jfcomp{A}{f}{P}{F}{R}}}{\jfcomp{A}{f}{P}{F}{S}}}
      {\ctxwk{{\jfcomp{A}{f}{P}{F}{S}}}{\jfcomp{A}{f}{P}{F}{S}}}
      {\jfcomp{A}{f}{P}{F}{\idtm{S}}}
      {\idtm{\jfcomp{A}{f}{P}{F}{S}}}
    }
\end{equation*}
\end{lem}

There is also a notion of vertical composition, although this is not an
operation the way horizontal composition is. Nevertheless, to analyze the
properties of horizontal composition it is useful to
consider vertical composition.

\begin{defn}
Let $\jhom{\Gamma}{A}{B}{f}$ be a morphism from $A$ to $B$ in context $\Gamma$
and let $\jfhom{\Gamma}{A}{B}{f}{P}{Q}{F}$ be a morphism over $f$ in context 
$\Gamma$. We define
\begin{equation*}
\jhomdefn{\Gamma}{{A}{P}}{{B}{Q}}{\jvcomp{P}{f}{F}}{\unfold{\jvcomp{P}{f}{F}}}
\end{equation*}
\end{defn}

\begin{lem}\label{lem:composition-threesome}
We have the following composition threesome:
\begin{align*}
& \inference
  { \jfhom{\Gamma}{A}{B}{f}{P}{Q}{F}
    \jfam{{{\Gamma}{B}}{Q}}{R}
    }
  { \jfameq
      {{{\Gamma}{A}}{P}}
      {\jfcomp{A}{f}{P}{F}{R}}
      {\jcomp{{A}{P}}{\jvcomp{P}{f}{F}}{R}}
    }
  \\
& \inference
  { \jfhom{\Gamma}{A}{B}{f}{P}{Q}{F}
    \jfam{{{{\Gamma}{B}}{Q}}{R}}{S}
    }
  { \jfameq
      {{{{\Gamma}{A}}{P}}{\jfcomp{A}{f}{P}{F}{R}}}
      {\jfcomp{A}{f}{P}{F}{S}}
      {\jcomp{{A}{P}}{\jvcomp{P}{f}{F}}{S}}
    }
  \\
& \inference
  { \jfhom{\Gamma}{A}{B}{f}{P}{Q}{F}
    \jterm{{{{\Gamma}{B}}{Q}}{R}}{S}{k}
    }
  { \jtermeq
      {{{{\Gamma}{A}}{P}}{\jfcomp{A}{f}{P}{F}{R}}}
      {\jfcomp{A}{f}{P}{F}{S}}
      {\jfcomp{A}{f}{P}{F}{k}}
      {\jcomp{{A}{P}}{\jvcomp{P}{f}{F}}{k}}
    }
\end{align*}
\end{lem}

\begin{proof}
Let $F$ be a morphism from $P$ to $Q$ over $f$ in context $\Gamma$ and let
$R$ be a family over $\ctxext{{\Gamma}{B}}{Q}$. Then we have
\begin{align*}
\jfcomp{A}{f}{P}{F}{R}
& \jdeq
  \unfold{\jfcomp{A}{f}{P}{F}{R}}
  \tag{by definition}
  \\
& \jdeq
  \subst{F}{{\ctxwk{P}{f}}{\ctxwk{P}{{A}{R}}}}
  \tag{by \autoref{comp-ws-f}}
  \\
& \jdeq
  \subst{F}{{\ctxwk{P}{f}}{\ctxwk{\ctxext{A}{P}}{R}}}
  \tag{by \autoref{comp-ew-f}}
  \\
& \jdeq
  \subst{\tmext{\ctxwk{P}{f}}{F}}{\ctxwk{\ctxext{A}{P}}{R}}
  \tag{by \autoref{comp-es}}
  \\
& \jdeq
  \jcomp{{A}{P}}{\jvcomp{P}{f}{F}}{R}.
  \tag{by definition}
\end{align*}
\end{proof}

\begin{lem}\label{lem:composition-interchange}
We have the following interchange law for composition:
\begin{equation*}
\inference
  { \jfhom{\Gamma}{A}{B}{f}{P}{Q}{F}
    \jfhom{\Gamma}{B}{C}{g}{Q}{R}{G}
    }
  { \jhomeq
      {\Gamma}
      {{A}{P}}
      {{C}{R}}
      {\jcomp{{A}{P}}{\jvcomp{P}{f}{F}}{\jvcomp{Q}{g}{G}}}
      {\jvcomp{P}{\jcomp{A}{f}{g}}{\jfcomp{A}{f}{P}{F}{G}}}
    }
\end{equation*}
\end{lem}

\begin{proof}
Consider $\jfhom{\Gamma}{A}{B}{f}{P}{Q}{F}$ and 
$\jfhom{\Gamma}{B}{C}{g}{Q}{R}{G}$. Then we have
\begin{align*}
\jcomp{{A}{P}}{\jvcomp{P}{f}{F}}{\jvcomp{Q}{g}{G}}
& \jdeq
  \jcomp{P}{F}{\jcomp{A}{f}{\tmext{}{}{\ctxwk{Q}{g}}{G}}}
  \tag{by \autoref{lem:composition-threesome}}
  \\
& \jdeq
  \tmext{}{}
    {\jcomp{P}{F}{\jcomp{A}{f}{\ctxwk{Q}{g}}}}
    {\jcomp{P}{F}{\jcomp{A}{f}{G}}}
  \tag{by \autoref{lem:tmext-jcomp}}
  \\
& \jdeq
  \tmext{}{}
    {\jcomp{P}{F}{\jcomp{A}{f}{\ctxwk{Q}{g}}}}
    {\jfcomp{A}{f}{P}{F}{G}}
  \tag{by definition}
  \\
& \jdeq
  \tmext{}{}
    {\jcomp{P}{F}{\ctxwk{\jcomp{A}{f}{Q}}{\jcomp{A}{f}{g}}}}
    {\jfcomp{A}{f}{P}{F}{G}}
  \tag{by \autoref{lem:jcomp-wk}}
  \\
& \jdeq
  \tmext{}{}
    {\subst{F}{\ctxwk{P}{\ctxwk{\jcomp{A}{f}{Q}}{\jcomp{A}{f}{g}}}}}
    {\jfcomp{A}{f}{P}{F}{G}}
  \tag{by definition}
  \\
& \jdeq
  \tmext{}{}
    {\subst{F}{\ctxwk{{P}{\jcomp{A}{f}{Q}}}{{P}{\jcomp{A}{f}{g}}}}}
    {\jfcomp{A}{f}{P}{F}{G}}
  \tag{by \autoref{comp-ww-t}}
  \\
& \jdeq
  \tmext{}{}
    {\ctxwk{P}{\jcomp{A}{f}{g}}}
    {\jfcomp{A}{f}{P}{F}{G}}
  \tag{by \autoref{cancellation-ws-t}}
  \\
& \jdeq
  \jvcomp{P}{\jcomp{A}{f}{g}}{\jfcomp{A}{f}{P}{F}{G}}
  \tag{by definition}
\end{align*}
\end{proof}

\begin{lem}
Horizontal composition of morphisms over morphisms is associative, i.e.~the
following inference rules are valid:
\begin{align*}
& \inference
  { \jfhom{\Gamma}{A}{B}{f}{P}{Q}{F}
    \jfhom{\Gamma}{B}{C}{g}{Q}{R}{G}
    \jfam{{{\Gamma}{C}}{R}}{S}
    }
  { \jfameq
      {{{\Gamma}{A}}{P}}
      {\jfcomp{A}{f}{P}{F}{\jfcomp{B}{g}{Q}{G}{S}}}
      {\jfcomp{A}{\jcomp{A}{f}{g}}{P}{\jfcomp{A}{f}{P}{F}{G}}{S}}
    }
  \\
& \inference
  { \jfhom{\Gamma}{A}{B}{f}{P}{Q}{F}
    \jfhom{\Gamma}{B}{C}{g}{Q}{R}{G}
    \jfam{{{{\Gamma}{C}}{R}}{S}}{T}
    }
  { \jfameq
      {{{{\Gamma}{A}}{P}}{\jfcomp{A}{f}{P}{F}{\jfcomp{B}{g}{Q}{G}{S}}}}
      {\jfcomp{A}{f}{P}{F}{\jfcomp{B}{g}{Q}{G}{T}}}
      {\jfcomp{A}{\jcomp{A}{f}{g}}{P}{\jfcomp{A}{f}{P}{F}{G}}{T}}
    }
  \\
& \inference
  { \jfhom{\Gamma}{A}{B}{f}{P}{Q}{F}
    \jfhom{\Gamma}{B}{C}{g}{Q}{R}{G}
    \jterm{{{{\Gamma}{C}}{R}}{S}}{T}{l}
    }
  { \jtermeq
      {{{{\Gamma}{A}}{P}}{\jfcomp{A}{f}{P}{F}{\jfcomp{B}{g}{Q}{G}{S}}}}
      {\jfcomp{A}{f}{P}{F}{\jfcomp{B}{g}{Q}{G}{T}}}
      {\jfcomp{A}{f}{P}{F}{\jfcomp{B}{g}{Q}{G}{l}}}
      {\jfcomp{A}{\jcomp{A}{f}{g}}{P}{\jfcomp{A}{f}{P}{F}{G}}{l}}
    }
\end{align*}
\end{lem}

\begin{proof}
We only prove the first inference rule. Let
$\jfhom{\Gamma}{A}{B}{f}{P}{Q}{F}$, $\jfhom{\Gamma}{B}{C}{g}{Q}{R}{G}$ and
$\jfam{{{\Gamma}{C}}{R}}{S}$. Then we have the judgmental equalities
\begin{align*}
\jfcomp{A}{f}{P}{F}{\jfcomp{B}{g}{Q}{G}{S}}
& \jdeq
  \jcomp{{A}{P}}{\jvcomp{P}{f}{F}}{\jcomp{{B}{Q}}{\jvcomp{Q}{g}{G}}{S}}
  \tag{by \autoref{lem:composition-threesome}}
  \\
& \jdeq
  \jcomp{{A}{P}}{\jcomp{{A}{P}}{\jvcomp{P}{f}{F}}{\jvcomp{Q}{g}{G}}}{S}
  \tag{by \autoref{lem:jcomp-jcomp}}
  \\
& \jdeq
  \jcomp{{A}{P}}{\jvcomp{P}{\jcomp{A}{f}{g}}{\jfcomp{A}{f}{P}{F}{G}}}{S}
  \tag{by \autoref{lem:composition-interchange}}
  \\
& \jdeq
  \jfcomp{A}{\jcomp{A}{f}{g}}{P}{\jfcomp{A}{f}{P}{F}{G}}{S}.
  \tag{by \autoref{lem:composition-threesome}}
\end{align*}
\end{proof}

\begin{lem}
The following inference rules explain how vertically and horizontally composed 
morphisms are evaluated:
\begin{align*}
& \inference
  { \jfhom{\Gamma}{A}{B}{f}{P}{Q}{F}
    \jterm{\Gamma}{A}{x}
    \jterm{\Gamma}{\subst{x}{P}}{u}
    }
  { \jtermeq
      {\Gamma}
      {{B}{Q}}
      {\subst{\tmext{}{}{x}{u}}{\jvcomp{P}{f}{F}}}
      {\tmext{}{}{\subst{x}{f}}{\subst{u}{{x}{F}}}}
    }
\end{align*}
\end{lem}

\begin{proof}
Let $\jfhom{\Gamma}{A}{B}{f}{P}{Q}{F}$, $\jterm{\Gamma}{A}{x}$ and
$\jterm{\Gamma}{\subst{x}{P}}{u}$. Then we have the judgmental equalities
\begin{align*}
\subst{\tmext{}{}{x}{u}}{\jvcomp{P}{f}{F}}
& \jdeq
  \subst{\tmext{}{}{x}{u}}{\tmext{}{}{\ctxwk{P}{f}}{F}}
  \tag{by definition}
  \\
& \jdeq
  \tmext{\subst{\tmext{x}{u}}{\ctxwk{P}{f}}}{\subst{\tmext{x}{u}}{F}}
  \tag{by \autoref{lem:tmext-subst}}
  \\
& \jdeq
  \tmext{\subst{u}{{x}{\ctxwk{P}{f}}}}{\subst{u}{{x}{F}}}
  \tag{by \autoref{comp-es}}
  \\
& \jdeq
  \tmext{\subst{u}{\ctxwk{\subst{x}{P}}{\subst{x}{f}}}}{\subst{u}{{x}{F}}}
  \tag{by \autoref{comp-sw-t}}
  \\
& \jdeq
  \tmext{\subst{x}{f}}{\subst{u}{{x}{F}}}.
  \tag{by \autoref{cancellation-ws-t}}
\end{align*}
\end{proof}

\label{pullback}
Now that we have introduced the notions of morphisms and composition,
we can develop a diagramatic style of of displaying type dependencies
combined with morphisms. We give an informal, metatheoretical definition of
such diagrams by indicating what the various components mean. The definition
is informal because we will only use such diagrams occasionally to provide a
graphical indication of the situation in which we're working. In particular,
we will not shy away from using natural numbers and trust that the reader can
figure out what we mean.

\begin{defn}
A diagram is said to be a \emph{dependency diagram in context $\Gamma$}
if it is built up according to the following steps:
\begin{itemize}
\item The arrows appearing in a dependency diagram are either ordinary, like the
arrow%
$\begin{tikzcd}[ampersand replacement = \&]
X \ar{r} \& Y,
\end{tikzcd}$
or double-headed, like
$\begin{tikzcd}[ampersand replacement = \&]
X \ar[fib]{r} \& Y.
\end{tikzcd}$
\item An ordinary arrow 
\begin{equation*}
\begin{tikzcd}
A \ar{r}{f} & B
\end{tikzcd}
\end{equation*}
between two families $A$ and $B$ of contexts over $\Gamma$ indicates that
$f$ is a morphism from $A$ to $B$ in context $\Gamma$, i.e.~that we have the
judgment $\jhom{\Gamma}{A}{B}{f}$.
\item The set of double-headed arrows must form a forest and the root of
each maximal tree of double-headed arrows is a family of contexts over $\Gamma$.
In particular, if an object is not the domain of a double-headed arrow it must
be a family of contexts over $\Gamma$.
\item A sequence of double-headed 
arrows
\begin{equation*}
\begin{tikzcd}
P_{n} \ar[fib]{r} & \cdots \ar[fib]{r} & P_1 \ar[fib]{r} & A
\end{tikzcd}
\end{equation*}
indicates that $P_1$ is a family of contexts over $\ctxext{\Gamma}{A}$, that
$P_2$ is a family of contexts over $\ctxext{{\Gamma}{A}}{P_1}$, etcetera.
\item There can be two kinds of ladders of double-headed arrows:
\begin{equation*}
\begin{tikzcd}
P_{n} \ar{r}{F_{n}} \ar[fib]{d} & Q_{n} \ar[fib]{d}\\
\vdots \ar[fib]{d} & \vdots \ar[fib]{d}\\
P_1 \ar{r}{F_1} \ar[fib]{d} & Q_1 \ar[fib]{d}\\
A \ar{r}{f} & B
\end{tikzcd}
\qquad
\begin{tikzcd}[column sep = tiny]
P_{n+m} \ar{rr}{F_{n+m}} \ar[fib]{d} & & Q_{n+m} \ar[fib]{d}\\
\vdots \ar[fib]{d} & & \vdots \ar[fib]{d}\\
P_{n+1} \ar{rr}{F_{n+1}} \ar[fib]{dr} & & Q_{n+1} \ar[fib]{dl}\\
& P_n \ar[fib]{d}\\
& \vdots \ar[fib]{d}\\
& P_1 \ar[fib]{d}\\
& A
\end{tikzcd}
\end{equation*}
The ladder on the left 
indicates that $F_1$ is a morphism from $P_1$ to $Q_1$ \emph{over} $f$,
i.e.~that the judgment $\jfhom{\Gamma}{A}{B}{f}{P_1}{Q_1}{F_1}$ holds, that
$F_2$ is a morphism from $P_2$ to $Q_2$ over
the morphism $\tmext{\ctxwk{P_1}{f}}{F_1}$ from $\ctxext{A}{P_1}$ to
$\ctxext{B}{Q_1}$, etcetera.

The ladder on the right indicates that $F_{n+1}$ is a morphism from $P_{n+1}$ to
$Q_{n+1}$ in the appropriate context, that $F_{n+2}$ is a morphism from
$P_{n+2}$ to $Q_{n+2}$ over $F_{n+1}$, etcetera.
 
Note that the object(s) at the bottom of a ladder are always families of contexts
over $\Gamma$, so that the typing of the various ingredients makes sense.
\end{itemize}
Such a diagram is said to be commutative if the subdiagram consisting of only
the normal headed arrows is commutative in the usual sense (using judgmental
equality). Note that the ladders are inherently commutative.
\end{defn}

The most basic illustrative example of a commutative dependency diagram is
the diagram
\begin{equation*}
\begin{tikzcd}
P \ar[fib]{d} \ar{r}{F} & Q \ar[fib]{d} \\
A \ar{r}{f} & B
\end{tikzcd}
\end{equation*}
indicating a morphism $F$ from $P$ to $Q$ over the morphism $f$ from $A$ to
$B$ in a context $\Gamma$.

We can just copy the usual categorical definition of a pullback square to our
current situation, but we have to require that each arrow in the pullback square
is an ordinary arrow. When families (i.e. double-headed arrows) are involved
in the diagram, we make the following definition of a family pullback:

\begin{defn}
We say that a commutative dependency diagram of the form
\begin{equation*}
\begin{tikzcd}
P \ar[fib]{d} \ar{r}{F} & Q \ar[fib]{d} \\
A \ar{r}{f} & B
\end{tikzcd}
\end{equation*}
is a \emph{family pullback} if the following inference rules are valid:
\begin{align*}
& \inference
  { \jfam{{\Gamma}{A}}{P'}
    \jfhom{\Gamma}{A}{B}{f}{P'}{Q}{F'}
    }
  { \jhom{{\Gamma}{A}}{P'}{P}{u}
    }
  \\
& \inference
  { \jfam{{\Gamma}{A}}{P'}
    \jfhom{\Gamma}{A}{B}{f}{P'}{Q}{F'}
    }
  { \jfhomeq{\Gamma}{A}{B}{f}{P'}{Q}{\jcomp{}{u}{F}}{F'}
    }
  \\
& \inference
  { \jhom{{\Gamma}{A}}{P'}{P}{v}
    \jfhomeq{\Gamma}{A}{B}{f}{P'}{Q}{\jcomp{}{v}{F}}{F'}
    }
  { \jhomeq{{\Gamma}{A}}{P'}{P}{v}{u}
    }
\end{align*}
\end{defn}

The following lemma explains that when a square involving families is a
family pullback square whenever the corresponding square involving projections is a
pullback square. There is no proof in the the opposite direction.

\begin{lem}
A square
\begin{equation}\label{eq:fpb_to_pb_eqv_fpb}
\begin{tikzcd}
P \ar[fib]{d} \ar{r}{F} & Q \ar[fib]{d} \\
A \ar{r}{f} & B
\end{tikzcd}
\end{equation}
is a family pullback square whenever the square
\begin{equation}\label{eq:fpb_to_pb_eqv_pb}
\begin{tikzcd}[column sep = large]
\ctxext{A}{P} \ar{d}[swap]{\cprojfstf{A}{P}} \ar{r}{\tmext{\ctxwk{P}{f}}{F}} & \ctxext{B}{Q} \ar{d}{\cprojfstf{B}{Q}} \\
A \ar{r}{f} & B
\end{tikzcd}
\end{equation}
is a pullback square.
\end{lem}

The family pullback of a family along any morphism always exists. It is simply given
by the precomposition of the family with the morphism. Note that this fact does
not carry over to arbitrary pullbacks.

\begin{lem}
The diagram
\begin{equation*}
\begin{tikzcd}
\jcomp{}{f}{Q} \ar[fib]{d} \ar{r}{\idtm{\jcomp{}{f}{Q}}} & Q \ar[fib]{d} \\
A \ar{r}{f} & B
\end{tikzcd}
\end{equation*}
is a family pullback diagram.
\end{lem}

\begin{proof}
The proof is a triviality because $\jhom{{\Gamma}{A}}{P'}{\jcomp{}{f}{Q}}{F'}$
is the same judgment as $\jfhom{\Gamma}{A}{B}{f}{P'}{Q}{F'}$ and
$\jcomp{}{\idtm{\jcomp{}{f}{Q}}}{F'}\jdeq F'$.
\end{proof}

For arbitrary pullbacks we have the pasting lemma as usual, but for family
pullbacks we can only derive one of the parts of the pasting lemma.

\begin{lem}
Suppose we have the diagram
\begin{equation*}
\begin{tikzcd}
P \ar{r}{F} \ar[fib]{d} & Q \ar{r}{G} \ar[fib]{d} & R \ar[fib]{d}\\
A \ar{r}{f} & B \ar{r}{g} & C
\end{tikzcd}
\end{equation*}
where the square on the right and the outer rectangle are family pullback 
diagrams. Then the square on the left is a family pullback diagram.
\end{lem}

\begin{proof}
Let $\jfam{{\Gamma}{A}}{P'}$ be a family and let $\jfhom{\Gamma}{A}{B}{f}
{P'}{Q}{F}$ be a morphism over $f$.
\begin{itemize}
\item Then we compose $F'$ with $G$ to obtain a morphism over $\jcomp{}{f}{g}$.
\item Then we get $\jhom{{\Gamma}{A}}{P'}{P}{u}$ with a uniqueness property.
      The property that $\jcomp{}{u}{F}\jdeq F'$ follows from the assumption
      that the right square is a pullback.
\item Now assume that we have another such $v$. Compose it with $F$ and $G$.
      By the assumed properties this is the same as $u$ composed with $F$ and
      $G$. By the pullback condition we now get $u\jdeq v$. 
\end{itemize}
\end{proof}

\begin{lem}
For any $\jterm{\Gamma}{A}{x}$ and any $\jfam{{\Gamma}{A}}{P}$, the square
\begin{equation*}
\begin{tikzcd}
\subst{x}{P} \ar{d} \ar{r}{\finc{x}{P}} & \ctxext{A}{P} \ar{d}{\cprojfstf{A}{P}}\\
\emptyf \ar{r}{\ctxwk{\emptyf}{x}} & A
\end{tikzcd}
\end{equation*}
is a pullback square.
\end{lem}

In the following lemma we assert that pulling back a family $Q$ 
over $\ctxext{{\Gamma}{A}}{P}$ along a fiber
inclusion $\finc{x}{P}$ gives the family $\subst{x}{Q}$ over $\ctxext{\Gamma}{\subst{x}{P}}$. 

\begin{lem}
We have the following inference rule:
\begin{equation*}
\inference
  { \jfam{{{\Gamma}{A}}{P}}{Q}
    \jterm{\Gamma}{A}{x}
    }
  { \jfameq
      {{\Gamma}{\subst{x}{P}}}
      {\jcomp{}{\finc{x}{P}}{Q}}
      {\subst{x}{Q}}
    }
\end{equation*}
\end{lem}

\begin{proof}
We have the judgmental equalities:
\begin{align*}
\jcomp{\subst{x}{P}}{\finc{x}{P}}{Q}
& \jdeq
  \subst{\tmext{\ctxwk{\subst{x}{P}}{x}}{\idtm{\subst{x}{P}}}}{\ctxwk{\subst{x}{P}}{Q}}
  \\
& \jdeq
  \subst{\idtm{\subst{x}{P}}}{{\ctxwk{\subst{x}{P}}{x}}{\ctxwk{\subst{x}{P}}{Q}}}
  \\
& \jdeq
  \subst{\idtm{\subst{x}{P}}}{\ctxwk{\subst{x}{P}}{\subst{x}{Q}}}
  \\
& \jdeq
  \subst{x}{Q}.
\end{align*}
\end{proof}

\begin{lem}
The following inference rule is valid:
\begin{equation*}
\inference
  { \jfam{{\Gamma}{A}}{P}
    \jfam{{\Gamma}{A}}{Q}
    }
  { \jfameq
      {{\Gamma}{{A}{P}}}
      {\jcomp{{A}{P}}{\cprojfstf{A}{P}}{Q}}
      {\ctxwk{P}{Q}}
    }
\end{equation*}
\end{lem}

\begin{proof}
Let $\jfam{{\Gamma}{A}}{P}$ and $\jfam{{\Gamma}{A}}{Q}$ be
families. Then we have
\begin{align*}
\jcomp{{A}{P}}{\cprojfstf{A}{P}}{Q}
& \jdeq
  \unfoldall{\jcomp{{A}{P}}{\cprojfstf{A}{P}}{Q}}
  \tag{by definition}\\
& \jdeq 
  \subst{\ctxwk{P}{\idtm{A}}}{\ctxwk{P}{{A}{Q}}} 
  \tag{by \autoref{comp-ww-f}}\\
& \jdeq 
  \ctxwk{P}{\subst{\idtm{A}}{\ctxwk{A}{Q}}} 
  \tag{by \autoref{comp-ws-f}}\\
& \jdeq 
  \ctxwk{P}{Q} 
  \tag{by \autoref{idfunc-wk-defn}}
\end{align*}
\end{proof}


\subsection{Fiber inclusions}
We will use the insights of \autoref{extension-on-terms} to define and study
\emph{fiber inclusions}. The fiber inclusion of the \emph{fiber}
$\subst{x}{P}$ into the extension $\ctxext{A}{P}$ is a morphism
$\jhom{\Gamma}{\subst{x}{P}}{{A}{P}}{\finc{x}{P}}$, for any family
$\jfam{{\Gamma}{A}}{P}$ and any term $\jterm{\Gamma}{A}{x}$. Then we will determine
the ways in which it is compatible with the other operators. Note that in this
subsection we will focus on the compatibility properties; the fact that
the fiber inclusions also appear in a pullback diagram will be established in
\autoref{pullback}. 

\begin{defn}
Let $\jterm{\Gamma}{A}{x}$ be a term and let $\jfam{{\Gamma}{A}}{P}$ be a
family. Then we define the \emph{fiber inclusion} of $\subst{x}{P}$ into
$\ctxext{A}{P}$ in context $\Gamma$ to be the morphism
\begin{equation*}
\jhomdefn{\Gamma}{\subst{x}{P}}{{A}{P}}{\finc{x}{P}}{\unfoldnext{\finc{x}{P}}}.
\end{equation*}
\end{defn}

We can immediately show that composing with a fiber inclusion is substitution.
Thus, the following lemma asserts that we have a family pullback square
\begin{equation*}
\begin{tikzcd}
\subst{x}{Q} \ar[fib]{d} \ar{r} & Q \ar[fib]{d} \\
\subst{x}{P} \ar{r}[swap]{\finc{x}{P}} & \ctxext{A}{P}
\end{tikzcd}
\end{equation*}
in context $\Gamma$.
The upper arrow is just the identity morphism over $\finc{x}{P}$.

\begin{lem}\label{lem:finc-precomp}
Composition with $\finc{x}{P}$ is (the action on families of)
substitution with $x$:
\begin{align*}
& \inference
  { \jterm{\Gamma}{A}{x}
    \jfam{{{\Gamma}{A}}{P}}{Q}
    }
  { \jfameq
      {{\Gamma}{\subst{x}{P}}}
      {\jcomp{\subst{x}{P}}{\finc{x}{P}}{Q}}
      {\subst{x}{Q}}
    }
  \\
& \inference
  { \jterm{\Gamma}{A}{x}
    \jfam{{{{\Gamma}{A}}{P}}{Q}}{R}
    }
  { \jfameq
      {{{\Gamma}{\subst{x}{P}}}{\subst{x}{Q}}}
      {\jcomp{\subst{x}{P}}{\finc{x}{P}}{R}}
      {\subst{x}{R}}
    }
  \\
& \inference
  { \jterm{\Gamma}{A}{x}
    \jterm{{{{\Gamma}{A}}{P}}{Q}}{R}{h}
    }
  { \jtermeq
      {{{\Gamma}{\subst{x}{P}}}{\subst{x}{Q}}}
      {\subst{x}{R}}
      {\jcomp{\subst{x}{P}}{\finc{x}{P}}{h}}
      {\subst{x}{h}}
    }
\end{align*}
\end{lem}

\begin{proof}
We only prove the validity of the first inference rule. Let $\jterm{\Gamma}{A}{x}$
be a term and let $\jfam{{{\Gamma}{A}}{P}}{Q}$ be a family. Then we have the
judgmental equalities
\begin{align*}
\jcomp{\subst{x}{P}}{\finc{x}{P}}{Q}
& \jdeq
  \unfold{\jcomp{\subst{x}{P}}{\unfold{\finc{x}{P}}}{Q}}
  \tag{by definition}
  \\
& \jdeq
  \subst
    {\idtm{\subst{x}{P}}}
    {{\ctxwk{\subst{x}{P}}{x}}{\ctxwk{\subst{x}{P}}{Q}}}
  \tag{by \autoref{comp-es}}
  \\
& \jdeq
  \subst
    {\idtm{\subst{x}{P}}}
    {\ctxwk{\subst{x}{P}}{\subst{x}{Q}}}
  \tag{by \autoref{comp-ws-f}}
  \\
& \jdeq
  \subst{x}{Q}.
  \tag{by \autoref{idfunc-precomp}}
\end{align*}
\end{proof}

We can give a second characterization of fiber inclusions:

\begin{lem}\label{lem:finc-char2}
We have the following inference rule
\begin{equation*}
\inference
  { \jterm{\Gamma}{A}{x}
    \jfam{{\Gamma}{A}}{P}
    }
  { \jhomeq{\Gamma}{\subst{x}{P}}{{A}{P}}{\finc{x}{P}}{\subst{x}{\idtm{{A}{P}}}}
    } 
\end{equation*}
\end{lem}

\begin{proof}
The proof is the following calculation:
\begin{align*}
\finc{x}{P}
& \jdeq 
  \tmext{\ctxwk{\subst{x}{P}}{x}}{\idtm{\subst{x}{P}}}
  \tag{by definition}
  \\
& \jdeq
  \subst
    { \idtm{\subst{x}{P}}
      }
    { { \ctxwk{\subst{x}{P}}{x}
        }
      { \idtm{\ctxext{\ctxwk{\subst{x}{P}}{A}}{\ctxwk{\subst{x}{P}}{P}}}
        }
      }
  \tag{by definition}
  \\
& \jdeq
  \subst
    { \idtm{\subst{x}{P}}
      }
    { { \ctxwk{\subst{x}{P}}{x}
        }
      { \idtm{\ctxwk{\subst{x}{P}}{\ctxext{A}{P}}}
        }
      }
  \tag{by \autoref{comp-we-f}}
  \\
& \jdeq
  \subst
    { \idtm{\subst{x}{P}}
      }
    { { \ctxwk{\subst{x}{P}}{x}
        }
      { \ctxwk{\subst{x}{P}}{\idtm{\ctxext{A}{P}}}
        }
      }
  \tag{by \autoref{comp-wi-t}}
  \\
& \jdeq
  \subst
    { \idtm{\subst{x}{P}}
      }
    { \ctxwk
        { \subst{x}{P}
          }
        { \subst{x}{\idtm{\ctxext{A}{P}}}
          }
      }
  \tag{by \autoref{comp-ws-t}}
  \\
& \jdeq
  \subst{x}{\idtm{\ctxext{A}{P}}}.
  \tag{by \autoref{idfunc-precomp}}
\end{align*}
\end{proof}

\begin{cor}
The following inference rule is valid:
\begin{equation*}
\inference
  { \jterm{\Gamma}{A}{x}
    \jterm{\Gamma}{\subst{x}{P}}{u}
    }
  { \jtermeq{\Gamma}{{A}{P}}{\subst{u}{\finc{x}{P}}}{\tmext{x}{u}}
    }
\end{equation*}
\end{cor}

We have the following lemmas expressing the compatibility of the fiber
inclusions with the empty context, extension, weakening and substitution. 

\begin{lem}
The fiber inclusions are compatible with the empty families; i.e.~the following
inference rules are valid
\begin{align*}
& \inference
  { \jterm{\Gamma}{A}{x}
    }
  { \jtermeq
      {\Gamma}
      {A}
      {\finc{x}{\emptyf}}
      {x}
    }
  \\
& \inference
  { \jfam{\Gamma}{A}
    }
  { \jhomeq
      {\Gamma}
      {A}
      {A}
      {\finc{\emptytm}{A}}
      {\idtm{A}}
    }
\end{align*}
\end{lem}

\begin{proof}
Let $\jterm{\Gamma}{A}{x}$. We have the judgmental equalities
\begin{align*}
\finc{x}{\emptyf}
& \jdeq
  \unfold{\finc{x}{\emptyf}}
  \tag{by definition}
  \\
& \jdeq
  \tmext{\ctxwk{\emptyf}{x}}{\idtm{\emptyf}}
  \tag{by \autoref{comp-s0-c}}
  \\
& \jdeq
  \tmext{x}{\idtm{\emptyf}}
  \tag{by \autoref{comp-0w-t}}
  \\
& \jdeq
  \tmext{x}{\emptytm}
  \tag{by \autoref{comp-00-t}}
  \\
& \jdeq
  x
  \tag{by \autoref{lem:tmext-emp}}
\end{align*}
to prove the first judgmental equality. For the second, we have
\begin{align*}
\finc{\emptytm}{A}
& \jdeq
  \unfold{\finc{\emptytm}{A}}
  \tag{by definition}
  \\
& \jdeq
  \tmext{\emptytm}{\idtm{\subst{\emptytm}{A}}}
  \tag{by \autoref{comp-00-t}}
  \\
& \jdeq
  \idtm{\subst{\emptytm}{A}}
  \tag{by \autoref{lem:tmext-emp}}
  \\
& \jdeq
  \idtm{A}
  \tag{by \autoref{comp-0s-f}}.
\end{align*}
\end{proof}

\begin{lem}
The fiber inclusions are compatible with extension; i.e.~the following inference
rule is valid
\begin{equation*}
\inference
  { \jterm{\Gamma}{A}{x}
    \jterm{\Gamma}{\subst{x}{P}}{u}
    \jfam{{{\Gamma}{A}}{P}}{Q}
    }
  { \jhomeq
      {\Gamma}
      {\subst{\tmext{x}{u}}{Q}}
      {{{A}{P}}{Q}}
      {\finc{\tmext{x}{u}}{Q}}
      {\jcomp{}{\finc{u}{\subst{x}{Q}}}{\finc{x}{\ctxext{P}{Q}}}}
    }
\end{equation*}
\end{lem}

\begin{proof}
Let $\jterm{\Gamma}{A}{x}$, $\jterm{\Gamma}{\subst{x}{P}}{u}$ and
$\jfam{{{\Gamma}{A}}{P}}{Q}$. Then we have the judgmental equalities
\begin{align*}
\finc{\tmext{x}{u}}{Q}
& \jdeq
  \subst{\tmext{}{}{x}{u}}{\idtm{{{A}{P}}{Q}}}
  \tag{by \autoref{lem:finc-char2}}
  \\
& \jdeq
  \subst{u}{{x}{\idtm{{{A}{P}}{Q}}}}
  \tag{by \autoref{comp-es}}
  \\
& \jdeq
  \subst{u}{{x}{\idtm{{A}{{P}{Q}}}}}
  \tag{by \autoref{comp-ee-c}}
  \\
& \jdeq
  \subst{u}{\finc{x}{\ctxext{P}{Q}}}
  \tag{by \autoref{lem:finc-char2}}
  \\
& \jdeq
  \jcomp{}{\finc{u}{\subst{x}{Q}}}{\finc{x}{\ctxext{P}{Q}}}
  \tag{by \autoref{lem:finc-precomp}}
\end{align*}
\end{proof}

\begin{lem}
The fiber inclusions are compatible with weakening; i.e.~the following inference
rule is valid
\begin{equation*}
\inference
  { \jfam{\Gamma}{A}
    \jterm{\Gamma}{B}{y}
    \jfam{{\Gamma}{B}}{Q}
    }
  { \jhomeq
      {{\Gamma}{A}}
      {\subst{\ctxwk{A}{y}}{\ctxwk{A}{Q}}}
      {{\ctxwk{A}{B}}{\ctxwk{A}{Q}}}
      {\finc{\ctxwk{A}{y}}{\ctxwk{A}{Q}}}
      {\ctxwk{A}{\finc{y}{Q}}}
    }
\end{equation*}
\end{lem}

\begin{proof}
Let $\jterm{\Gamma}{B}{y}$ and $\jfam{{\Gamma}{B}}{Q}$. Then we have the
judgmental equalities
\begin{align*}
\finc{\ctxwk{A}{y}}{\ctxwk{A}{Q}}
& \jdeq
  \subst{\ctxwk{A}{y}}{\idtm{\ctxext{\ctxwk{A}{B}}{\ctxwk{A}{Q}}}}
  \tag{by \autoref{lem:finc-char2}}
  \\
& \jdeq
  \subst{\ctxwk{A}{y}}{\idtm{\ctxwk{A}{\ctxext{B}{Q}}}}
  \tag{by \autoref{comp-we-f}}
  \\
& \jdeq
  \subst{\ctxwk{A}{y}}{\ctxwk{A}{\idtm{\ctxext{B}{Q}}}}
  \tag{by \autoref{comp-wi-t}}
  \\
& \jdeq
  \ctxwk{A}{\subst{y}{\idtm{{B}{Q}}}}
  \tag{by \autoref{comp-ws-t}}
  \\
& \jdeq
  \ctxwk{A}{\finc{y}{Q}}.
  \tag{by \autoref{lem:finc-char2}}
\end{align*}
\end{proof}

\begin{lem}
The fiber inclusions are compatible with substitution; i.e.~the following
inference rule is valid
\begin{equation*}
\inference
  { \jterm{\Gamma}{A}{x}
    \jfam{{{\Gamma}{A}}{P}}{Q}
    \jterm{{\Gamma}{A}}{P}{f}
    }
  { \jhomeq
      {\Gamma}
      {\subst{{x}{f}}{{x}{Q}}}
      {{\subst{x}{P}}{\subst{x}{Q}}}
      {\finc{\subst{x}{f}}{\subst{x}{Q}}}
      {\subst{x}{\finc{f}{Q}}}
    }
\end{equation*}
\end{lem}

\begin{proof}
Let $\jterm{\Gamma}{A}{x}$, $\jterm{{\Gamma}{A}}{P}{f}$ and 
$\jfam{{{\Gamma}{A}}{P}}{Q}$. Then we have the judgmental equalities
\begin{align*}
\finc{\subst{x}{f}}{\subst{x}{Q}}
& \jdeq
  \subst{x}{{f}{\idtm{{P}{Q}}}}
  \tag{by \autoref{lem:finc-char2}}
  \\
& \jdeq
  \subst{{x}{f}}{{x}{\idtm{{P}{Q}}}}
  \tag{by \autoref{comp-ss-t}}
  \\
& \jdeq
  \subst{{x}{f}}{\idtm{\subst{x}{\ctxext{P}{Q}}}}
  \tag{by \autoref{comp-si-t}}
  \\
& \jdeq
  \subst{{x}{f}}{\idtm{\ctxext{\subst{x}{P}}{\subst{x}{Q}}}}
  \tag{by \autoref{comp-se-f}}
  \\
& \jdeq
  \finc{\subst{x}{f}}{\subst{x}{Q}}.
  \tag{by \autoref{lem:finc-char2}}
\end{align*}
\end{proof}

\begin{lem}
The fiber inclusions are compatible with identity terms; i.e.~the following
inference rule is valid
\begin{equation*}
\inference
  { \jfam{{\Gamma}{A}}{P}
    }
  { \jhomeq
      {{\Gamma}{A}}
      {P}
      {\ctxwk{A}{\ctxext{A}{P}}}
      {\finc{\idtm{A}}{\ctxwk{A}{P}}}
      {\idtm{{A}{P}}}
    }
\end{equation*}
\end{lem}

\begin{proof}
Let $\jfam{\Gamma}{A}$. Then we have the judgmental equalities
\begin{align*}
\finc{\idtm{A}}{\ctxwk{A}{P}}
& \jdeq
  \subst{\idtm{A}}{\idtm{\ctxext{\ctxwk{A}{A}}{\ctxwk{A}{P}}}}
  \tag{by \autoref{lem:finc-char2}}
  \\
& \jdeq
  \subst{\idtm{A}}{\idtm{\ctxwk{A}{\ctxext{A}{P}}}}
  \tag{by \autoref{comp-we-f}}
  \\
& \jdeq
  \subst{\idtm{A}}{\ctxwk{A}{\idtm{{A}{P}}}}
  \tag{by \autoref{comp-wi-t}}
  \\
& \jdeq
  \idtm{{A}{P}}
  \tag{by \autoref{idfunc-precomp}}
\end{align*}
\end{proof}

\begin{comment}
\subsection{Another special case of projections}
In this subsection we investigate the special case of a projection which
appears as a morphism from $\ctxext{{A}{P}}{\ctxwk{P}{Q}}$ to $\ctxext{A}{Q}$
in context $\Gamma$, where we assume to have the families 
$\jfam{{\Gamma}{A}}{P}$ and $\jfam{{\Gamma}{A}}{Q}$. 

Note that we have the judgmental
equalities
\begin{align*}
\ctxwk{\ctxext{{A}{P}}{\ctxwk{P}{\mfam{A}}}}{\ctxext{A}{\mfam{A}}}
& \jdeq 
  \ctxext
    {\ctxwk{\ctxext{{A}{P}}{\ctxwk{P}{\mfam{A}}}}{A}}
    {\ctxwk{\ctxext{{A}{P}}{\ctxwk{P}{\mfam{A}}}}{\mfam{A}}}
  \\
& \jdeq
  \ctxext
    {\ctxwk{{P}{\mfam{A}}}{{\ctxext{A}{P}}{A}}}
    {\ctxwk{\ctxext{{A}{P}}{\ctxwk{P}{\mfam{A}}}}{\mfam{A}}}
\end{align*}
Note that we have the term $\ctxwk{{P}{\mfam{A}}}{\cprojfstf{A}{P}}$ of the
family $\ctxwk{{P}{\mfam{A}}}{{\ctxext{A}{P}}{A}}$. Therefore, we need to
find a term of type $\subst{\ctxwk{{P}{\mfam{A}}}{\cprojfstf{A}{P}}}
{\ctxwk{\ctxext{{A}{P}}{\ctxwk{P}{\mfam{A}}}}{\mfam{A}}}$. Note that we have
the judgmental equalities:
\begin{align*}
\subst
  {\ctxwk{{P}{\mfam{A}}}{\cprojfstf{A}{P}}}
  {\ctxwk{\ctxext{{A}{P}}{\ctxwk{P}{\mfam{A}}}}{\mfam{A}}}
& \jdeq
  \subst
    {\ctxwk{{P}{\mfam{A}}}{\cprojfstf{A}{P}}}
    {\ctxwk{{P}{\mfam{A}}}{{\ctxext{A}{P}}{\mfam{A}}}}
  \\
& \jdeq
  \ctxwk
    { {P}{\mfam{A}}
      }
    { \subst
        {\cprojfstf{A}{P}}
        {\ctxwk{\ctxext{A}{P}}{\mfam{A}}}
      }
  \\
& \jdeq
  \ctxwk
    { {P}{\mfam{A}}
      }
    { {P}{\mfam{A}}
      }
  \\
& \jdeq
  \ctxwk{P}{{\mfam{A}}{\mfam{A}}}
\end{align*}
We find the term $\ctxwk{P}{\idtm{\mfam{A}}}$ here. Thus we can now define
$\bar{\typefont{pr}}$ by:
\begin{equation}\label{barproj}
\jhomdefn
  {\Gamma}
  {{{A}{P}}{\mfam{A}}}
  {{A}{\mfam{A}}}
  {\bar{\typefont{pr}}}
  {\tmext{\ctxwk{{P}{\mfam{A}}}{\cprojfstf{A}{P}}}{\ctxwk{P}{\idtm{\mfam{A}}}}}
\end{equation}

\begin{lem}
We have the judgmental equality
\begin{equation*}
\jfameq
  {{{{\Gamma}{A}}{P}}{\ctxwk{P}{\mfam{A}}}}
  {\jcomp{}{\bar{\typefont{pr}}}{Q}}
  {\ctxwk{P}{Q}}
\end{equation*}
for any family $Q$ of contexts over $\ctxext{{\Gamma}{A}}{\mfam{A}}$ 
\end{lem}
\end{comment}

\subsection{Extension algebras}
In this subsection our goal is to define the notion of extension algebras,
which are internal versions of the extension operation of the theory of
contexts, families and terms. In this article, their use will be mainly in
universes. The theory of extension algebras requires the full power (i.e.~all
of the ingredients) of the theory of contexts, families and terms in its
formulation and it is (perhaps surprisingly) quite involved to formulate it.

Let $P$ be a family over an extended context $\ctxext{\Gamma}{A}$. We could
mimic extension by requiring to have terms
\begin{align*}
\jhom*{\Gamma}{{A}{P}}{A}{\epsilon_0}\\
\jhom*{{\Gamma}{A}}{{P}{\jcomp{}{\epsilon_0}{P}}}{P}{\epsilon_1}.
\end{align*}

\begin{rmk}
Instead of looking at $\epsilon_1$ as a context morphism from $\ctxext{P}{\jcomp{}{\epsilon_0}{P}}$
to $P$, one could also look at $\epsilon_1$ as a morphism \emph{over $\cprojfstf{A}{P}$},
as indicated in the following diagram:
\begin{equation*}
\begin{tikzcd}
P
  \ar[fib]{d}
& \jcomp{}{\epsilon_0}{P}
  \ar[fib]{d}
  \ar{l}[swap]{\epsilon_1}
  \ar{r}
& P
  \ar[fib]{d}
  \\
A
& \ctxext{A}{P}
  \ar{l}{\cprojfstf{A}{P}}
  \ar{r}[swap]{\epsilon_0}
& A
\end{tikzcd}
\end{equation*}
This makes it clear that $\epsilon_1$ takes a family over an extended context as an
argument. The extended context consists of a `base part' and a `family part'. 
The output of $\epsilon_1$ is a new (extended) family over that base part. Forgetting 
the family part is what the projection takes care of.
\end{rmk}

Extension also satisfies the properties explained in \autoref{comp-ee}, so we
must find the two judgmental equalities for $\epsilon_0$ and $\epsilon_1$ mimicing those. 
The first of these judgmental equalities is easy to give: it says that the
following diagram commutes:
\begin{equation}\label{eq:ealg-eq1}
\begin{tikzcd}[column sep=huge]
\ctxext{A}{{P}{\jcomp{}{\epsilon_0}{P}}} 
  \ar{d}[swap]{\jvcomp{}{\epsilon_0}{\idtm{\jcomp{}{\epsilon_0}{P}}}
    } 
  \ar{r}{\jvcomp{}{\idtm{A}}{\epsilon_1}
    } 
  & \ctxext{A}{P} \ar{d}{\epsilon_0}\\
\ctxext{A}{P} \ar{r}[swap]{\epsilon_0} & A
\end{tikzcd}
\end{equation}
To get a feel for this judgmental equality we include the following lemma.

\begin{lem}
Let $A$, $P$, $\epsilon_0$ and $\epsilon_1$ be as above, satisfying \autoref{eq:ealg-eq1} and
let $x_0:A$,
$x_1:\subst{x_0}{P}$ and $x_2:\subst{x_1}{{x_0}{\jcomp{}{\epsilon_0}{P}}}$.
Then we have the judgmental equality
\begin{equation*}
\subst{{x_2}{{x_1}{\epsilon_1}}}{{x_0}{\epsilon_0}}
\jdeq
\subst{x_2}{{{x_1}{{x_0}{\epsilon_0}}}{\epsilon_0}}.
\end{equation*}
\end{lem}

\begin{proof}
The proof is a simple computation:
\begin{align*}
\subst{{x_2}{{x_1}{\epsilon_1}}}{{x_0}{\epsilon_0}}
& \jdeq
  \subst{\tmext{x_0}{\subst{x_2}{{x_1}{\epsilon_1}}}}{\epsilon_0}
  \\
& \jdeq 
  \subst{{x_2}{{x_1}{{x_0}{\jvcomp{}{\idtm{A}}{\epsilon_1}}}}}{\epsilon_0}
  \\
& \jdeq
  \subst{x_2}{{x_1}{{x_0}{\jcomp{}{\jvcomp{}{\idtm{A}}{\epsilon_1}}{\epsilon_0}}}}
  \\
& \jdeq
  \subst{x_2}{{x_1}{{x_0}{\jcomp{}{\jvcomp{}{\epsilon_0}{\idtm{\jcomp{}{\epsilon_0}{P}}}}{\epsilon_0}}}}
  \\
& \jdeq 
  \subst{{x_2}{{x_1}{{x_0}{\jvcomp{}{\epsilon_0}{\idtm{\jcomp{}{\epsilon_0}{P}}}}}}}{\epsilon_0}
  \\
& \jdeq
  \subst{\tmext{\subst{x_1}{{x_0}{\epsilon_0}}}{x_2}}{\epsilon_0}
  \\
& \jdeq
  \subst{x_2}{{{x_1}{{x_0}{\epsilon_0}}}{\epsilon_0}}.\qedhere
\end{align*}
\end{proof}

The second of the judgmental equalities is harder to describe, however. We need
to consider `higher' families, i.e.~families over families over families, and
thus we need to look at the family $\jcomp{}{\epsilon_0}{\jcomp{}{\epsilon_0}{P}}$ and find the
two dotted morphisms in the diagram
\begin{equation*}
\begin{tikzcd}
\ctxext{P}{{\jcomp{}{\epsilon_0}{P}}{\jcomp{}{\epsilon_0}{\jcomp{}{\epsilon_0}{P}}}}
  \ar[densely dotted]{d}
  \ar[densely dotted]{r}
& \ctxext{P}{\jcomp{}{\epsilon_0}{P}} \ar{d}{\epsilon_1}\\
\ctxext{P}{\jcomp{}{\epsilon_0}{P}} \ar{r}{\epsilon_1} & P
\end{tikzcd}
\end{equation*}
The first is easy to find. Note that we have the judgmental equality
\begin{equation*}
\ctxext{\jcomp{}{\epsilon_0}{P}}{\jcomp{}{\epsilon_0}{\jcomp{}{\epsilon_0}{P}}}
  \jdeq
  \jcomp{}{\epsilon_0}{\ctxext{P}{\jcomp{}{\epsilon_0}{P}}}
\end{equation*}
and therefore we may just take the morphism
\begin{equation*}
\jhom
  {{\Gamma}{A}}
  {{P}{{\jcomp{}{\epsilon_0}{P}}{\jcomp{}{\epsilon_0}{\jcomp{}{\epsilon_0}{P}}}}}
  {{P}{\jcomp{}{\epsilon_0}{P}}}
  {\jvcomp{}{\idtm{P}}{\jcomp{}{\epsilon_0}{\epsilon_1}}}.
\end{equation*}
For the other morphism we need to look at the family
$\ctxext{P}{{\jcomp{}{\epsilon_0}{P}}{\jcomp{}{\epsilon_0}{\jcomp{}{\epsilon_0}{P}}}}$ differently. We do
that in the following lemma.

\begin{lem}
Suppose we have $A$, $P$, $\epsilon_0$ and $\epsilon_1$ satisfying \autoref{eq:ealg-eq1}. Then
we have the judgmental equality
\begin{equation*}
\jcomp{}{\epsilon_0}{\jcomp{}{\epsilon_0}{P}}
  \jdeq
  \jcomp{}{\epsilon_1}{\jcomp{}{\epsilon_0}{P}}.
\end{equation*}
\end{lem}

\begin{proof}
The proof is a rather long computation. It was found by starting with the
application of \autoref{eq:ealg-eq1} and making the computation in both ways
from there. The judgmental equality below where \autoref{eq:ealg-eq1} is applied,
is marked with an $*$. The computation goes as follows:
\begin{align*}
\jcomp{}{\epsilon_0}{\jcomp{}{\epsilon_0}{P}}
& \jdeq
  \subst{\epsilon_0}
    {\ctxwk{\ctxext{A}{P}}{\jcomp{}{\epsilon_0}{P}}}
  \\
& \jdeq
  \subst
    {\idtm{\jcomp{}{\epsilon_0}{P}}}
    { \ctxwk
        {\jcomp{}{\epsilon_0}{P}}{\subst{\epsilon_0}
        {\ctxwk{\ctxext{A}{P}}{\jcomp{}{\epsilon_0}{P}}}}}
  \\
& \jdeq
  \subst
    {\idtm{\jcomp{}{\epsilon_0}{P}}}
    { {\ctxwk{\jcomp{}{\epsilon_0}{P}}{\epsilon_0}}
      {\ctxwk{\jcomp{}{\epsilon_0}{P}}{{\ctxext{A}{P}}{\jcomp{}{\epsilon_0}{P}}}}}
  \\
& \jdeq
  \subst
    {\idtm{\jcomp{}{\epsilon_0}{P}}}
    { {\ctxwk{\jcomp{}{\epsilon_0}{P}}{\epsilon_0}}
      {\ctxwk{\ctxext{{A}{P}}{\jcomp{}{\epsilon_0}{P}}}{\jcomp{}{\epsilon_0}{P}}}}
  \\
& \jdeq
  \subst
    {\idtm{\jcomp{}{\epsilon_0}{P}}}
    { {\ctxwk{\jcomp{}{\epsilon_0}{P}}{\epsilon_0}}
      {\ctxwk{\ctxext{A}{{P}{\jcomp{}{\epsilon_0}{P}}}}{\jcomp{}{\epsilon_0}{P}}}}
  \\
& \jdeq
  \subst
    {\unfold{\jvcomp{\jcomp{}{\epsilon_0}{P}}{\epsilon_0}{\idtm{\jcomp{}{\epsilon_0}{P}}}}}
    {\ctxwk{\ctxext{A}{{P}{\jcomp{}{\epsilon_0}{P}}}}{\jcomp{}{\epsilon_0}{P}}}
  \\
& \jdeq
  \subst
    {\jvcomp{}{\epsilon_0}{\idtm{\jcomp{}{\epsilon_0}{P}}}}
    {\ctxwk{\ctxext{A}{{P}{\jcomp{}{\epsilon_0}{P}}}}{\jcomp{}{\epsilon_0}{P}}}
  \\
& \jdeq
  \jcomp{}{\jvcomp{}{\epsilon_0}{\idtm{\jcomp{}{\epsilon_0}{P}}}}{\jcomp{}{\epsilon_0}{P}}
  \\
& \jdeq
  \jcomp{}{\jcomp{}{\jvcomp{}{\epsilon_0}{\idtm{\jcomp{}{\epsilon_0}{P}}}}{\epsilon_0}}{P}
  \\
& \stackrel{*}{\jdeq}
  \jcomp{}{
    \jcomp{}{\jvcomp{}{\idtm{A}}{\epsilon_1}}{\epsilon_0}}{P}
  \\
& \jdeq
  \jcomp{}{\jvcomp{}{\idtm{A}}{\epsilon_1}}{
    \jcomp{}{\epsilon_0}{P}}
  \\
& \jdeq
  \subst
    {\tmext{\ctxwk{\ctxext{P}{\jcomp{}{\epsilon_0}{P}}}{\idtm{A}}}{\epsilon_1}}
    {\ctxwk{\ctxext{A}{{P}{\jcomp{}{\epsilon_0}{P}}}}{\jcomp{}{\epsilon_0}{P}}}
  \\
& \jdeq
  \subst
    {\epsilon_1}
    { {\ctxwk{\ctxext{P}{\jcomp{}{\epsilon_0}{P}}}{\idtm{A}}}
      {\ctxwk{\ctxext{A}{{P}{\jcomp{}{\epsilon_0}{P}}}}{\jcomp{}{\epsilon_0}{P}}}
      }
  \\
& \jdeq
  \subst
    {\epsilon_1}
    { {\cprojfstf{A}{\ctxext{P}{\jcomp{}{\epsilon_0}{P}}}}
      {\ctxwk{\ctxext{A}{{P}{\jcomp{}{\epsilon_0}{P}}}}{\jcomp{}{\epsilon_0}{P}}}
      }
  \\
& \jdeq
  \subst
    {\epsilon_1}
    {\jcomp{}{\cprojfstf{A}{\ctxext{P}{\jcomp{}{\epsilon_0}{P}}}}{\jcomp{}{\epsilon_0}{P}}}
  \\
& \jdeq
  \subst
    {\epsilon_1}
    {\ctxwk{\ctxext{P}{\jcomp{}{\epsilon_0}{P}}}{\jcomp{}{\epsilon_0}{P}}}
  \\
& \jdeq
  \jcomp{}{\epsilon_1}{\jcomp{}{\epsilon_0}{P}}.
  \qedhere
\end{align*}
\end{proof}

Now we see that we can use the morphism
\begin{equation*}
\jhom
  {{\Gamma}{A}}
  {{P}{{\jcomp{}{\epsilon_0}{P}}{\jcomp{}{\epsilon_1}{{}{\epsilon_0}{P}}}}}
  {{P}{\jcomp{}{\epsilon_0}{P}}}
  {\jvcomp{}{\epsilon_1}{\idtm{\jcomp{}{\epsilon_1}{{}{\epsilon_0}{P}}}}}.
\end{equation*}
Thus, the second judgmental equality we will need is that the diagram
\begin{equation}\label{eq:ealg-eq2}
\begin{tikzcd}[column sep=huge]
\ctxext{P}{{\jcomp{}{\epsilon_0}{P}}{\jcomp{}{\epsilon_0}{{}{\epsilon_0}{P}}}} 
  \ar{r}{\jvcomp{}{\idtm{P}}{\jcomp{}{\epsilon_0}{\epsilon_1}}}
  \ar{d}[swap]{
    \jvcomp{}{\epsilon_1}{\idtm{\jcomp{}{\epsilon_1}{{}{\epsilon_0}{P}}}}
%    \tmext
%      {\ctxwk{\jcomp{}{\bar{e}}{{}{e}{P}}}{f}}
%      {\idtm{\jcomp{}{\bar{e}}{{}{e}{P}}}}
    }
& \ctxext{P}{\jcomp{}{\epsilon_0}{P}} \ar{d}{\epsilon_1}\\
\ctxext{P}{\jcomp{}{\epsilon_0}{P}} \ar{r}[swap]{\epsilon_1} & P
\end{tikzcd}
\end{equation}
commutes judgmentally. Now we can confidently formulate the definition of
extension algebras.

\begin{defn}
An \emph{extension algebra in context $\Gamma$} is a quadruple $(A,P,\epsilon_0,\epsilon_1)$
consisting of a family $A$ over context $\Gamma$, a family $P$ over the context
$\ctxext{\Gamma}{A}$, a morphism $\epsilon_0$ from $\ctxext{A}{P}$ to $A$ in context
$\Gamma$ and a morphism $\epsilon_1$ from $\ctxext{P}{\jcomp{}{\epsilon_0}{P}}$ to $P$ in context
$\ctxext{\Gamma}{A}$, satisfying the judgmental equalities of
\autoref{eq:ealg-eq1,eq:ealg-eq2}.
\end{defn}

We also give a bit of intuition to the requirement of \autoref{eq:ealg-eq2} by
means of the following lemma.

\begin{lem}
Let $(A,P,\epsilon_0,\epsilon_1)$ be an extension algebra in context $\Gamma$ and let
$y_0:P$, $y_1:\subst{y_0}{\jcomp{}{\epsilon_0}{P}}$ and 
$y_2:\subst{y_1}{{y_0}{\jcomp{}{\epsilon_0}{\jcomp{}{\epsilon_0}{P}}}}$. Then we have the
judgmental equality
\begin{equation*}
\subst{{y_2}{{y_1}{\jcomp{}{\epsilon_0}{\epsilon_1}}}}{{y_0}{\epsilon_1}}
  \jdeq
  \subst{y_2}{{{y_1}{{y_0}{\epsilon_1}}}{\epsilon_1}}
\end{equation*}
\end{lem}

\begin{proof}
The proof is a straightforward calculation:
\begin{align*}
\subst{{y_2}{{y_1}{\jcomp{}{\epsilon_0}{\epsilon_1}}}}{{y_0}{\epsilon_1}}
& \jdeq
  \subst{\tmext{y_0}{\subst{y_2}{{y_1}{\jcomp{}{\epsilon_0}{\epsilon_1}}}}}{\epsilon_1}
  \\
& \jdeq
  \subst{{y_2}{{y_1}{{y_0}{\jvcomp{}{\idtm{P}}{\jcomp{}{\epsilon_0}{\epsilon_1}}}}}}{\epsilon_1}
  \\
& \jdeq
  \subst
    {y_2}
    { {y_1}
      { {y_0}
        {\jcomp{}{\jvcomp{}{\idtm{P}}{\jcomp{}{\epsilon_0}{\epsilon_1}}}{\epsilon_1}}
        }
      }
  \\
& \jdeq
  \subst
    {y_2}
    { {y_1}
      { {y_0}
        {\jcomp{}{\jvcomp{}{\epsilon_1}{\idtm{\jcomp{}{\epsilon_1}{{}{\epsilon_0}{P}}}}}{\epsilon_1}}
        }
      }
  \\
& \jdeq
  \subst{{y_2}{{y_1}{{y_0}{\jvcomp{}{\epsilon_1}{\idtm{\jcomp{}{\epsilon_1}{{}{\epsilon_0}{P}}}}}}}}{\epsilon_1}
  \\
& \jdeq
  \subst{\tmext{\subst{y_1}{{y_0}{\epsilon_1}}}{y_2}}{\epsilon_1}
  \\
& \jdeq
  \subst{y_2}{{{y_1}{{y_0}{\epsilon_1}}}{\epsilon_1}}.\qedhere
\end{align*}
\end{proof}

\begin{comment}
Extension algebras don't come in isolation. There are also extension algebra
families and extension algebra terms. We now aim to define these and to
establish various constructions of new extension algebras out of old ones:
the empty extension algebra, and extensions, weakenings and substitutions
of extension algebras and of course the identity term as an extension algebra
term. We start with extension algebra families.

\begin{defn}
Consider an extension algebra $\mathcal{A}\defeq(A,P,e,f)$. 
An extension algebra family over $\mathcal{A}$ is likewise a quadruple
$\mathcal{B}\defeq(B,Q,g,h)$. Here we have a family $\jfam{{\Gamma}{A}}{B}$, a
family $\jfam{{{{\Gamma}{A}}{P}}{\ctxwk{P}{B}}}{Q}$ and
\begin{align*}
\jhom*{{{\Gamma}{A}}{P}}{\ctxext{\ctxwk{P}{B}}{Q}}{\jcomp{}{\epsilon_0}{B}}{g}\\
\jhom*{{{{\Gamma}{A}}{P}}{\ctxwk{P}{B}}}{\ctxext{Q}{\jcomp{}{g}{Q}}}{Q}{h}.
\end{align*}
The quadruple $(\jcomp{}{\epsilon_0}{B},Q,g,h)$ is required to be an extension algebra
in context $\ctxext{{\Gamma}{A}}{P}$.
\end{defn}

\begin{defn}
Suppose $\mathcal{A}$ is an extension algebra and $\mathcal{B}$ is an extension
algebra family over $\mathcal{A}$. A term of $\mathcal{B}$ is a pair $(x,y)$
consisting of
\begin{align*}
\jterm*{{\Gamma}{A}}{B}{x}\\
\jterm*{{{\Gamma}{A}}{P}}{\subst{\jcomp{}{\epsilon_0}{x}}{Q}}{y}
\end{align*}
such that the diagrams
\begin{equation*}
\begin{tikzcd}
\ctxext{\jcomp{}{\epsilon_0}{B}}{Q} 
  \ar{r}{g} 
  \ar[shift right=.7ex,fib]{d}
& B 
  \ar[shift right=.7ex,fib]{d} 
  \\
\ctxext{A}{P} 
  \ar[shift right=.7ex,dotted]{u}[swap]{\tmext{\jcomp{}{\epsilon_0}{x}}{y}}
  \ar{r}{e}
& A
  \ar[shift right=.7ex,dotted]{u}[swap]{x}
\end{tikzcd}
\end{equation*}
and
\begin{equation*}
\begin{tikzcd}
\jcomp{}{f}{\ctxext{Q}{\jcomp{}{g}{Q}}}
  \ar{r}{\jcomp{}{f}{h}}
  \ar[shift right=.7ex,fib]{d}
& Q
  \ar[shift right=.7ex,fib]{d}
  \\
\jcomp{}{f}{\jcomp{}{\epsilon_0}{B}}
  \ar{r}{\idtm{\jcomp{}{f}{\jcomp{}{\epsilon_0}{B}}}}
  \ar[shift right=.7ex,fib]{d}
  \ar[shift right=.7ex,dotted,mapsto]{u}[swap]{\jcomp{}{f}{y}}
& \jcomp{}{\epsilon_0}{B}
  \ar[shift right=.7ex,fib]{d}
  \ar[shift right=.7ex,dotted,mapsto]{u}[swap]{y}
  \\
\ctxext{P}{\jcomp{}{\epsilon_0}{P}}
  \ar{r}[swap]{f}
  \ar[shift right=.7ex,dotted]{u}[swap]{\jcomp{}{f}{\jcomp{}{\epsilon_0}{x}}}
& P
  \ar[shift right=.7ex,dotted]{u}[swap]{\jcomp{}{\epsilon_0}{x}}
\end{tikzcd}
\end{equation*}
commute.
\end{defn}

\begin{defn}
Suppose $\mathcal{A}$ and $\mathcal{B}$ are extension algebras in context
$\Gamma$. We define the extension algebra $\ctxwk{\mathcal{A}}{\mathcal{B}}$
to be the quadruple
\begin{equation*}
(\ctxwk{A}{B},\ctxwk{\ctxext{A}{P}}{Q},\ctxwk{\ctxext{A}{P}}{g},\ctxwk{\ctxext{A}{P}}{h}).
\end{equation*}
Note that $\ctxwk{\ctxext{A}{P}}{Q}$ is a family over $\ctxwk{\ctxext{A}{P}}{B}$,
whereas it should be a family over $\jcomp{}{\epsilon_0}{\ctxwk{A}{B}}$. These are the
same by \autoref{lem:prehom}.
\end{defn}

\begin{rmk}
Before we continue, let us explore what it means to be an extension algebra
term of the extension algebra $\ctxwk{\mathcal{A}}{\mathcal{B}}$. Such an
extension algebra term $(x,y)$ would consist of
\begin{align*}
\jterm*{{\Gamma}{A}}{\ctxwk{A}{B}}{x}\\
\jterm*{{{\Gamma}{A}}{P}}{\subst{\jcomp{}{\epsilon_0}{x}}{\ctxwk{\ctxext{A}{P}}{Q}}}{y}.
\end{align*}
Thus, $x$ is a context morphism from $A$ to $B$ and $y$ is nothing but a term
of $\jcomp{}{\jcomp{}{\epsilon_0}{x}}{Q}$. For $x$, we see that the diagram
\begin{equation*}
\begin{tikzcd}
\ctxext{B}{Q} 
  \ar{r}{g} 
& B 
  \\
\ctxext{A}{P} 
  \ar{u}{\jvcomp{}{x}{y}}
  \ar{r}{e}
& A
  \ar{u}[swap]{x}
\end{tikzcd}
\end{equation*}
commutes.
\end{rmk}
\end{comment}

\subsection{Extension-empty algebras}
\begin{defn}
Let $P$ be a family over the extended context $\ctxext{\Gamma}{A}$, let
$\jterm{\Gamma}{A}{\phi_0}$ and $\jterm{{\Gamma}{A}}{P}{\phi_1}$ be terms. Then the
quadruple $(A,P,\phi_0,\phi_1)$ is said to be an \emph{empty algebra in context $\Gamma$}
if the following judgmental equalities hold:
\begin{align}
\jfameq*{\Gamma}{\subst{\phi_0}{P}}{A}
  \label{empalg-eq1}
  \\
\jtermeq*{\Gamma}{A}{\subst{\phi_0}{\phi_1}}{\phi_0}.
  \label{empalg-eq2}
\end{align}
\end{defn}

Thus, the empty algebras are the kind of algebras that require that families
are compatible with contexts, just as our motivation in \autoref{empty}. We
now combine the notion of extension algebras and empty algebras.

An extension-empty algebra in context $\Gamma$ is going to be a sextuple
$(A,P,\epsilon_0,\epsilon_1,\phi_0,\phi_1)$ for which 
the quadruple $(A,P,\epsilon_0,\epsilon_1)$ is an extension algebra in context 
$\Gamma$, the quadruple $(A,P,\phi_0,\phi_1)$ is an empty algebra in context
$\Gamma$, satisfying additional judgmental equalities expressing the 
compatibility of $\epsilon_0$ and $\epsilon_1$ with $\phi_0$ and $\phi_1$.
There will be four such judgmental equalities.

We can immediately state the first two:
\begin{align}
\jtermeq*{{\Gamma}{A}}{\ctxwk{A}{A}}{\subst{\phi_0}{\epsilon_0}}{\idtm{A}}
  \label{extempalg-eq1}
  \\
\jtermeq*{{\Gamma}{A}}{\ctxwk{A}{A}}{\subst{\phi_1}{\epsilon_0}}{\idtm{A}}
  \label{extempalg-eq2}
\end{align}
To see what $\subst{\phi_0}{\epsilon_1}$ can be, we must know its type first.
It is a morphism from $\subst{\phi_0}{\ctxext{P}{\jcomp{}{\epsilon_0}{P}}}$ to
$\subst{\phi_0}{P}$. We already know that $\subst{\phi_0}{P}\jdeq A$ by
\autoref{empalg-eq1} and to compute $\subst{\phi_0}{\jcomp{}{\epsilon_0}{P}}$
we use the following lemma.

\begin{lem}\label{lem:empalg-mor}
Consider an empty algebra $(A,P,\phi_0,\phi_1)$ in context $\Gamma$
and a morphism $\jhom{\Gamma}{{A}{P}}{B}{f}$.
Then $\subst{\phi_i}{f}$ is a morphism from $A$ to $B$ in context $\Gamma$ and
the following inference rules are valid for $i$ being $0$ or $1$:
\begin{align*}
& \inference
  { \jfam{{\Gamma}{A}}{Q}
    }
  { \jfameq
      {{\Gamma}{A}}
      {\subst{\phi_i}{\jcomp{}{f}{Q}}}
      {\jcomp{}{\subst{\phi_i}{f}}{Q}}
    }
  \\
& \inference
  { \jfam{{{\Gamma}{A}}{Q}}{R}
    }
  { \jfameq
      {{{\Gamma}{A}}{\jcomp{}{\subst{\phi_i}{f}}{Q}}}
      {\subst{\phi_i}{\jcomp{}{f}{R}}}
      {\jcomp{}{\subst{\phi_i}{f}}{R}}
    }
  \\
& \inference
  { \jterm{{{\Gamma}{A}}{Q}}{R}{h}
    }
  { \jtermeq
      {{{\Gamma}{A}}{\jcomp{}{\subst{\phi_i}{f}}{Q}}}
      {\jcomp{}{\subst{\phi_i}{f}}{R}}
      {\subst{\phi_i}{\jcomp{}{f}{h}}}
      {\jcomp{}{\subst{\phi_i}{f}}{h}}
    }
\end{align*}
\end{lem}

\begin{proof}
We only prove the first inference rule in both cases.
Let $Q$ be a family over $\ctxext{\Gamma}{A}$. In the case $i=0$
 we have the judgmental equalities
\begin{align*}
\subst{\phi_0}{\jcomp{}{f}{Q}}
& \jdeq
  \subst{\phi_0}{{f}{\ctxwk{\ctxext{A}{P}}{Q}}}
  \tag{by definition}
  \\
& \jdeq
  \subst{{\phi_0}{f}}{{\phi_0}{\ctxwk{\ctxext{A}{P}}{Q}}}
  \tag{by \autoref{comp-ss-f}}
  \\
& \jdeq
  \subst{{\phi_0}{f}}{{\phi_0}{\ctxwk{P}{{A}{Q}}}}
  \tag{by \autoref{comp-ew-f}}
  \\
& \jdeq
  \subst{{\phi_0}{f}}{\ctxwk{\subst{\phi_0}{P}}{\subst{\phi_0}{\ctxwk{A}{Q}}}}
  \tag{by \autoref{comp-sw-f}}
  \\
& \jdeq
  \subst{{\phi_0}{f}}{\ctxwk{\subst{\phi_0}{P}}{Q}}
  \tag{by \autoref{cancellation-ws-f}}
  \\
& \jdeq
  \subst{{\phi_0}{f}}{\ctxwk{A}{Q}}
  \tag{by \autoref{empalg-eq1}}
  \\
& \jdeq
  \jcomp{}{\subst{\phi_0}{f}}{Q}.
  \tag{by definition}
\end{align*}
In the case $i=1$ we have the judgmental equalities
\begin{align*}
\subst{\phi_1}{\jcomp{}{f}{Q}}
& \jdeq
  \subst{\phi_1}{{f}{\ctxwk{\ctxext{A}{P}}{Q}}}
  \tag{by definition}
  \\
& \jdeq
  \subst{{\phi_1}{f}}{{\phi_1}{\ctxwk{\ctxext{A}{P}}{Q}}}
  \tag{by \autoref{comp-ss-f}}
  \\
& \jdeq
  \subst{{\phi_1}{f}}{{\phi_1}{\ctxwk{P}{{A}{Q}}}}
  \tag{by \autoref{comp-ew-f}}
  \\
& \jdeq
  \subst{{\phi_1}{f}}{\ctxwk{A}{Q}}
  \tag{by \autoref{cancellation-ws-f}}
  \\
& \jdeq
  \jcomp{}{\subst{\phi_1}{f}}{Q}.
  \tag{by definition}
\end{align*}
\end{proof}

As an immediate corollary, if we assume the judgmental equalities
\autoref{extempalg-eq1,extempalg-eq2} we get that 
\begin{equation}\label{cor:empalg-mor}
\jfameq{{\Gamma}{A}}{\subst{\phi_i}{\jcomp{}{\epsilon_0}{P}}}{P}
\end{equation}
and hence that $\subst{\phi_0}{\epsilon_1}$ is a 
morphism from $\ctxext{A}{P}$ to $A$. Thus, we can require
\begin{equation}\label{extempalg-eq3}
\jhomeq{\Gamma}{{A}{P}}{A}{\subst{\phi_0}{\epsilon_1}}{\epsilon_0}.
\end{equation}
For the final judgmental equality we need to explain the term
\begin{equation*}
\jterm
  {{{\Gamma}{A}}{\subst{\phi_1}{\jcomp{}{\epsilon_0}{P}}}}
  {\subst{\phi_1}{\ctxwk{\ctxext{P}{\jcomp{}{\epsilon_0}{P}}}{P}}}
  {\subst{\phi_1}{\epsilon_1}}.
\end{equation*}
We have already established that $\subst{\phi_1}{\jcomp{}{\epsilon_0}{P}}\jdeq
P$. We also see that 
\begin{align*}
\subst{\phi_1}{\ctxwk{\ctxext{P}{\jcomp{}{\epsilon_0}{P}}}{P}}
& \jdeq
  \subst{\phi_1}{\ctxwk{\jcomp{}{\epsilon_0}{P}}{{P}{P}}}
  \tag{by \autoref{comp-ew-f}}
  \\
& \jdeq
  \ctxwk{\subst{\phi_1}{\jcomp{}{\epsilon_0}{P}}}{\subst{\phi_1}{\ctxwk{P}{P}}}
  \tag{by \autoref{comp-sw-f}}
  \\
& \jdeq
  \ctxwk{P}{\subst{\phi_1}{\ctxwk{P}{P}}}
  \tag{by \autoref{cor:empalg-mor}}
  \\
& \jdeq
  \ctxwk{P}{P}
  \tag{by \autoref{cancellation-ws-f}}
\end{align*}
and we will therefore require that
\begin{equation}\label{extempalg-eq4}
\jtermeq{{{\Gamma}{A}}{P}}{\ctxwk{P}{P}}{\subst{\phi_1}{\epsilon_1}}{\idtm{P}}.
\end{equation}
We bring all this together in the definition of extension-empty algebras:

\begin{defn}
An \emph{extension-empty algebra in context $\Gamma$} 
is a sextuple $(A,P,\epsilon_0,\epsilon_1,\phi_0,\phi_1)$ for which 
the quadruple $(A,P,\epsilon_0,\epsilon_1)$ is an extension algebra in context 
$\Gamma$, the quadruple $(A,P,\phi_0,\phi_1)$ is an empty algebra in context
$\Gamma$, satisfying the judgmental equalities 
\autoref{extempalg-eq1,extempalg-eq2,extempalg-eq3,extempalg-eq4}.
\end{defn}

\subsection{Extension-weakening algebras}
The notion of (extension-)weakening algebra is dependent on the notion of extension algebra.
Although it is strictly speaking not dependent on the notion of empty-algebra,
we shall only formulate a weakening algebras in the setting of an
extension-empty algebras. When one wants to have a weakening operation which
also acts on the level of contexts in an extension algebra without empty
context, extra work has to be done to introduce these separately.

\begin{defn}
An extension-weakening algebra in context $\Gamma$ is an octuple
\begin{equation*}
(A,P,\epsilon_0,\epsilon_1,\phi_0,\phi_1,\omega_0,\omega_1)
\end{equation*}
where $(A,P,\epsilon_0,\epsilon_1,\phi_0,\phi_1)$ is an extension-empty algebra
in context $\Gamma$ and where
\begin{align*}
\jhom*
  {{{\Gamma}{A}}{P}}
  {\ctxwk{P}{P}}
  {\jcomp{}{\epsilon_0}{P}}
  {\omega_0}
  \\
\jfhom*
  {{{\Gamma}{A}}{P}}
  {\ctxwk{P}{P}}
  {\jcomp{}{\epsilon_0}{P}}
  {\omega_0}
  {\ctxwk{P}{\jcomp{}{\epsilon_0}{P}}}
  {\jcomp{}{\epsilon_0}{\jcomp{}{\epsilon_0}{P}}}
  {\omega_1}
\end{align*}
satisfying the judgmental equalities
\end{defn}

\subsection{Extension-substitution algebras}

\subsection{Pre-universes}
Pre-universes are internal versions of the theory of contexts, families and
terms. They interpret extension, the empty context, weakening, substitution
and identity terms all at once in a compatible way.
