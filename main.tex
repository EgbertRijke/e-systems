\documentclass{article}

%%%%%%%%%%%%%%%%%%%%%%%%%%%%%%%%%%%%%%%%%%%%%%%%%%%%%%%%%%%%%%%%%%%%%%%%%%%%%%%%
%%%% PACKAGES

%\usepackage[a4paper,margin=1cm,footskip=.5cm]{geometry}
\usepackage{fullpage}

\usepackage[utf8]{inputenc}
\usepackage[english]{babel}

%%%% Spicing up the document
\usepackage{mathpazo}
\usepackage[scaled=0.95]{helvet}
\usepackage{courier}
\linespread{1.05} % Palatino looks better with this
\usepackage{microtype}

\usepackage{fancyhdr} % To set headers and footers
\usepackage{enumitem,mathtools,xspace,xcolor}
\usepackage{comment}
\usepackage{ifthen}
\usepackage{pifont}
\newcommand{\cmark}{\ding{51}\xspace}
\newcommand{\xmark}{\ding{55}\xspace}

\usepackage{graphicx}
\usepackage{caption}
\usepackage{xcolor}

\DeclareCaptionFormat{myformat}{{\bf #1}#2#3{\color{black!20}\hrulefill}}
\captionsetup[figure]{format=myformat}

\usepackage{tikz-cd}
\usepackage{tikz}
\usetikzlibrary{decorations.pathmorphing}
\usepackage[inference]{semantic}
\usepackage{booktabs}

\usepackage[hyphens]{url} % This package has to be loaded *before* hyperref
\usepackage[pagebackref,colorlinks,citecolor=darkgreen,linkcolor=darkgreen,unicode]{hyperref}
\definecolor{darkgreen}{rgb}{0,0.45,0}

% For some reason the following can't be above hyperref...
\usepackage{amssymb,amsmath,amsthm,stmaryrd,mathrsfs,wasysym}
\usepackage{aliascnt}
\usepackage[capitalize]{cleveref}

% The braket macro shouldn't be necessary
\usepackage{braket} % used for \setof{ ... } macro

%%%%%%%%%%%%%%%%%%%%%%%%%%%%%%%%%%%%%%%%%%%%%%%%%%%%%%%%%%%%%%%%%%%%%%%%%%%%%%%%
%% To include references in TOC we should use this package rather than a hack.
\usepackage{tocbibind}
\usepackage{etoolbox}           % get \apptocmd
%\apptocmd{\thebibliography}{\addcontentsline{toc}{section}{References}}{}{} % tell bibliography to get itself into the table of contents


\begin{comment}
%%%% Header and footers
\pagestyle{fancyplain}
\setlength{\headheight}{15pt}
\renewcommand{\chaptermark}[1]{\markboth{\textsc{Chapter \thechapter. #1}}{}}
\renewcommand{\sectionmark}[1]{\markright{\textsc{\thesection\ #1}}}
\end{comment}

% TOC depth
\setcounter{tocdepth}{3}

\lhead[\fancyplain{}{{\thepage}}]%
      {\fancyplain{}{\nouppercase{\rightmark}}}
\rhead[\fancyplain{}{\nouppercase{\leftmark}}]%
      {\fancyplain{}{\thepage}}
\cfoot{\textsc{\footnotesize [Draft of \today]}}
\lfoot[]{}
\rfoot[]{}

%%%%%%%%%%%%%%%%%%%%%%%%%%%%%%%%%%%%%%%%%%%%%%%%%%%%%%%%%%%%%%%%%%%%%%%%%%%%%%%%
%%%% We mostly use the macros of the book, to keep notations
%%%% and conventions the same. Recall that when the macros file
%%%% is updated, we need to comment the lines containing the
%%%% string `[chapter]` since our article is not a book.
%%%%
%%%% Instructions for updating the macros.tex file:
%%%% - fetch the latest macros.tex file from the HoTT/book git repository.
%%%% - comment all lines containing "[chapter]" because this is not a book.
%%%% - comment the definition of pbcorner because the xypic package is not used.
%%%%
%%%% MACROS FOR NOTATION %%%%
% Use these for any notation where there are multiple options.

%%% Notes and exercise sections
\makeatletter
\newcommand{\sectionNotes}{\phantomsection\section*{Notes}\addcontentsline{toc}{section}{Notes}\markright{\textsc{\@chapapp{} \thechapter{} Notes}}}
\newcommand{\sectionExercises}[1]{\phantomsection\section*{Exercises}\addcontentsline{toc}{section}{Exercises}\markright{\textsc{\@chapapp{} \thechapter{} Exercises}}}
\makeatother

%%% Definitional equality (used infix) %%%
\newcommand{\jdeq}{\equiv}      % An equality judgment
\let\judgeq\jdeq
%\newcommand{\defeq}{\coloneqq}  % An equality currently being defined
\newcommand{\defeq}{\vcentcolon\equiv}  % A judgmental equality currently being defined

%%% Term being defined
\newcommand{\define}[1]{\textbf{#1}}

%%% Vec (for example)

\newcommand{\Vect}{\ensuremath{\mathsf{Vec}}}
\newcommand{\Fin}{\ensuremath{\mathsf{Fin}}}
\newcommand{\fmax}{\ensuremath{\mathsf{fmax}}}
\newcommand{\seq}[1]{\langle #1\rangle}

%%% Dependent products %%%
\def\prdsym{\textstyle\prod}
%% Call the macro like \prd{x,y:A}{p:x=y} with any number of
%% arguments.  Make sure that whatever comes *after* the call doesn't
%% begin with an open-brace, or it will be parsed as another argument.
\makeatletter
% Currently the macro is configured to produce
%     {\textstyle\prod}(x:A) \; {\textstyle\prod}(y:B),\ 
% in display-math mode, and
%     \prod_{(x:A)} \prod_{y:B}
% in text-math mode.
\def\prd#1{\@ifnextchar\bgroup{\prd@parens{#1}}{\@ifnextchar\sm{\prd@parens{#1}\@eatsm}{\prd@noparens{#1}}}}
\def\prd@parens#1{\@ifnextchar\bgroup%
  {\mathchoice{\@dprd{#1}}{\@tprd{#1}}{\@tprd{#1}}{\@tprd{#1}}\prd@parens}%
  {\@ifnextchar\sm%
    {\mathchoice{\@dprd{#1}}{\@tprd{#1}}{\@tprd{#1}}{\@tprd{#1}}\@eatsm}%
    {\mathchoice{\@dprd{#1}}{\@tprd{#1}}{\@tprd{#1}}{\@tprd{#1}}}}}
\def\@eatsm\sm{\sm@parens}
\def\prd@noparens#1{\mathchoice{\@dprd@noparens{#1}}{\@tprd{#1}}{\@tprd{#1}}{\@tprd{#1}}}
% Helper macros for three styles
\def\lprd#1{\@ifnextchar\bgroup{\@lprd{#1}\lprd}{\@@lprd{#1}}}
\def\@lprd#1{\mathchoice{{\textstyle\prod}}{\prod}{\prod}{\prod}({\textstyle #1})\;}
\def\@@lprd#1{\mathchoice{{\textstyle\prod}}{\prod}{\prod}{\prod}({\textstyle #1}),\ }
\def\tprd#1{\@tprd{#1}\@ifnextchar\bgroup{\tprd}{}}
\def\@tprd#1{\mathchoice{{\textstyle\prod_{(#1)}}}{\prod_{(#1)}}{\prod_{(#1)}}{\prod_{(#1)}}}
\def\dprd#1{\@dprd{#1}\@ifnextchar\bgroup{\dprd}{}}
\def\@dprd#1{\prod_{(#1)}\,}
\def\@dprd@noparens#1{\prod_{#1}\,}

%%% Lambda abstractions.
% Each variable being abstracted over is a separate argument.  If
% there is more than one such argument, they *must* be enclosed in
% braces.  Arguments can be untyped, as in \lam{x}{y}, or typed with a
% colon, as in \lam{x:A}{y:B}. In the latter case, the colons are
% automatically noticed and (with current implementation) the space
% around the colon is reduced.  You can even give more than one variable
% the same type, as in \lam{x,y:A}.
\def\lam#1{{\lambda}\@lamarg#1:\@endlamarg\@ifnextchar\bgroup{.\,\lam}{.\,}}
\def\@lamarg#1:#2\@endlamarg{\if\relax\detokenize{#2}\relax #1\else\@lamvar{\@lameatcolon#2},#1\@endlamvar\fi}
\def\@lamvar#1,#2\@endlamvar{(#2\,{:}\,#1)}
% \def\@lamvar#1,#2{{#2}^{#1}\@ifnextchar,{.\,{\lambda}\@lamvar{#1}}{\let\@endlamvar\relax}}
\def\@lameatcolon#1:{#1}
\let\lamt\lam
% This version silently eats any typing annotation.
\def\lamu#1{{\lambda}\@lamuarg#1:\@endlamuarg\@ifnextchar\bgroup{.\,\lamu}{.\,}}
\def\@lamuarg#1:#2\@endlamuarg{#1}

%%% Dependent products written with \forall, in the same style
\def\fall#1{\forall (#1)\@ifnextchar\bgroup{.\,\fall}{.\,}}

%%% Existential quantifier %%%
\def\exis#1{\exists (#1)\@ifnextchar\bgroup{.\,\exis}{.\,}}

%%% Dependent sums %%%
\def\smsym{\textstyle\sum}
% Use in the same way as \prd
\def\sm#1{\@ifnextchar\bgroup{\sm@parens{#1}}{\@ifnextchar\prd{\sm@parens{#1}\@eatprd}{\sm@noparens{#1}}}}
\def\sm@parens#1{\@ifnextchar\bgroup%
  {\mathchoice{\@dsm{#1}}{\@tsm{#1}}{\@tsm{#1}}{\@tsm{#1}}\sm@parens}%
  {\@ifnextchar\prd%
    {\mathchoice{\@dsm{#1}}{\@tsm{#1}}{\@tsm{#1}}{\@tsm{#1}}\@eatprd}%
    {\mathchoice{\@dsm{#1}}{\@tsm{#1}}{\@tsm{#1}}{\@tsm{#1}}}}}
\def\@eatprd\prd{\prd@parens}
\def\sm@noparens#1{\mathchoice{\@dsm@noparens{#1}}{\@tsm{#1}}{\@tsm{#1}}{\@tsm{#1}}}
\def\lsm#1{\@ifnextchar\bgroup{\@lsm{#1}\lsm}{\@@lsm{#1}}}
\def\@lsm#1{\mathchoice{{\textstyle\sum}}{\sum}{\sum}{\sum}({\textstyle #1})\;}
\def\@@lsm#1{\mathchoice{{\textstyle\sum}}{\sum}{\sum}{\sum}({\textstyle #1}),\ }
\def\tsm#1{\@tsm{#1}\@ifnextchar\bgroup{\tsm}{}}
\def\@tsm#1{\mathchoice{{\textstyle\sum_{(#1)}}}{\sum_{(#1)}}{\sum_{(#1)}}{\sum_{(#1)}}}
\def\dsm#1{\@dsm{#1}\@ifnextchar\bgroup{\dsm}{}}
\def\@dsm#1{\sum_{(#1)}\,}
\def\@dsm@noparens#1{\sum_{#1}\,}

%%% W-types
\def\wtypesym{{\mathsf{W}}}
\def\wtype#1{\@ifnextchar\bgroup%
  {\mathchoice{\@twtype{#1}}{\@twtype{#1}}{\@twtype{#1}}{\@twtype{#1}}\wtype}%
  {\mathchoice{\@twtype{#1}}{\@twtype{#1}}{\@twtype{#1}}{\@twtype{#1}}}}
\def\lwtype#1{\@ifnextchar\bgroup{\@lwtype{#1}\lwtype}{\@@lwtype{#1}}}
\def\@lwtype#1{\mathchoice{{\textstyle\mathsf{W}}}{\mathsf{W}}{\mathsf{W}}{\mathsf{W}}({\textstyle #1})\;}
\def\@@lwtype#1{\mathchoice{{\textstyle\mathsf{W}}}{\mathsf{W}}{\mathsf{W}}{\mathsf{W}}({\textstyle #1}),\ }
\def\twtype#1{\@twtype{#1}\@ifnextchar\bgroup{\twtype}{}}
\def\@twtype#1{\mathchoice{{\textstyle\mathsf{W}_{(#1)}}}{\mathsf{W}_{(#1)}}{\mathsf{W}_{(#1)}}{\mathsf{W}_{(#1)}}}
\def\dwtype#1{\@dwtype{#1}\@ifnextchar\bgroup{\dwtype}{}}
\def\@dwtype#1{\mathsf{W}_{(#1)}\,}

\newcommand{\suppsym}{{\mathsf{sup}}}
\newcommand{\supp}{\ensuremath\suppsym\xspace}

\def\wtypeh#1{\@ifnextchar\bgroup%
  {\mathchoice{\@lwtypeh{#1}}{\@twtypeh{#1}}{\@twtypeh{#1}}{\@twtypeh{#1}}\wtypeh}%
  {\mathchoice{\@@lwtypeh{#1}}{\@twtypeh{#1}}{\@twtypeh{#1}}{\@twtypeh{#1}}}}
\def\lwtypeh#1{\@ifnextchar\bgroup{\@lwtypeh{#1}\lwtypeh}{\@@lwtypeh{#1}}}
\def\@lwtypeh#1{\mathchoice{{\textstyle\mathsf{W}^h}}{\mathsf{W}^h}{\mathsf{W}^h}{\mathsf{W}^h}({\textstyle #1})\;}
\def\@@lwtypeh#1{\mathchoice{{\textstyle\mathsf{W}^h}}{\mathsf{W}^h}{\mathsf{W}^h}{\mathsf{W}^h}({\textstyle #1}),\ }
\def\twtypeh#1{\@twtypeh{#1}\@ifnextchar\bgroup{\twtypeh}{}}
\def\@twtypeh#1{\mathchoice{{\textstyle\mathsf{W}^h_{(#1)}}}{\mathsf{W}^h_{(#1)}}{\mathsf{W}^h_{(#1)}}{\mathsf{W}^h_{(#1)}}}
\def\dwtypeh#1{\@dwtypeh{#1}\@ifnextchar\bgroup{\dwtypeh}{}}
\def\@dwtypeh#1{\mathsf{W}^h_{(#1)}\,}


\makeatother

% Other notations related to dependent sums
\let\setof\Set    % from package 'braket', write \setof{ x:A | P(x) }.
\newcommand{\pair}{\ensuremath{\mathsf{pair}}\xspace}
\newcommand{\tup}[2]{(#1,#2)}
\newcommand{\proj}[1]{\ensuremath{\mathsf{pr}_{#1}}\xspace}
\newcommand{\fst}{\ensuremath{\proj1}\xspace}
\newcommand{\snd}{\ensuremath{\proj2}\xspace}
\newcommand{\ac}{\ensuremath{\mathsf{ac}}\xspace} % not needed in symbol index
\newcommand{\un}{\ensuremath{\mathsf{upun}}\xspace} % not needed in symbol index, uniqueness principle for unit type

%%% recursor and induction
\newcommand{\rec}[1]{\mathsf{rec}_{#1}}
\newcommand{\ind}[1]{\mathsf{ind}_{#1}}
\newcommand{\indid}[1]{\ind{=_{#1}}} % (Martin-Lof) path induction principle for identity types
\newcommand{\indidb}[1]{\ind{=_{#1}}'} % (Paulin-Mohring) based path induction principle for identity types 

%%% the uniqueness principle for product types, formerly called surjective pairing and named \spr:
\newcommand{\uppt}{\ensuremath{\mathsf{uppt}}\xspace}

% Paths in pairs
\newcommand{\pairpath}{\ensuremath{\mathsf{pair}^{\mathord{=}}}\xspace}
% \newcommand{\projpath}[1]{\proj{#1}^{\mathord{=}}}
\newcommand{\projpath}[1]{\ensuremath{\apfunc{\proj{#1}}}\xspace}

%%% For quotients %%%
%\newcommand{\pairr}[1]{{\langle #1\rangle}}
\newcommand{\pairr}[1]{{\mathopen{}(#1)\mathclose{}}}
\newcommand{\Pairr}[1]{{\mathopen{}\left(#1\right)\mathclose{}}}

% \newcommand{\type}{\ensuremath{\mathsf{Type}}} % this command is overridden below, so it's commented out
\newcommand{\im}{\ensuremath{\mathsf{im}}} % the image

%%% 2D path operations
\newcommand{\leftwhisker}{\mathbin{{\ct}_{\ell}}}
\newcommand{\rightwhisker}{\mathbin{{\ct}_{r}}}
\newcommand{\hct}{\star}

%%% modalities %%%
\newcommand{\modal}{\ensuremath{\ocircle}}
\let\reflect\modal
\newcommand{\modaltype}{\ensuremath{\type_\modal}}
% \newcommand{\ism}[1]{\ensuremath{\mathsf{is}_{#1}}}
% \newcommand{\ismodal}{\ism{\modal}}
% \newcommand{\existsmodal}{\ensuremath{{\exists}_{\modal}}}
% \newcommand{\existsmodalunique}{\ensuremath{{\exists!}_{\modal}}}
% \newcommand{\modalfunc}{\textsf{\modal-fun}}
% \newcommand{\Ecirc}{\ensuremath{\mathsf{E}_\modal}}
% \newcommand{\Mcirc}{\ensuremath{\mathsf{M}_\modal}}
\newcommand{\mreturn}{\ensuremath{\eta}}
\let\project\mreturn
%\newcommand{\mbind}[1]{\ensuremath{\hat{#1}}}
\newcommand{\ext}{\mathsf{ext}}
%\newcommand{\mmap}[1]{\ensuremath{\bar{#1}}}
%\newcommand{\mjoin}{\ensuremath{\mreturn^{-1}}}
% Subuniverse
\renewcommand{\P}{\ensuremath{\type_{P}}\xspace}

%%% Localizations
% \newcommand{\islocal}[1]{\ensuremath{\mathsf{islocal}_{#1}}\xspace}
% \newcommand{\loc}[1]{\ensuremath{\mathcal{L}_{#1}}\xspace}

%%% Identity types %%%
\newcommand{\idsym}{{=}}
\newcommand{\id}[3][]{\ensuremath{#2 =_{#1} #3}\xspace}
\newcommand{\idtype}[3][]{\ensuremath{\mathsf{Id}_{#1}(#2,#3)}\xspace}
\newcommand{\idtypevar}[1]{\ensuremath{\mathsf{Id}_{#1}}\xspace}
% A propositional equality currently being defined
\newcommand{\defid}{\coloneqq}

%%% Dependent paths
\newcommand{\dpath}[4]{#3 =^{#1}_{#2} #4}

%%% singleton
% \newcommand{\sgl}{\ensuremath{\mathsf{sgl}}\xspace}
% \newcommand{\sctr}{\ensuremath{\mathsf{sctr}}\xspace}

%%% Reflexivity terms %%%
% \newcommand{\reflsym}{{\mathsf{refl}}}
\newcommand{\refl}[1]{\ensuremath{\mathsf{refl}_{#1}}\xspace}

%%% Path concatenation (used infix, in diagrammatic order) %%%
\newcommand{\ct}{%
  \mathchoice{\mathbin{\raisebox{0.5ex}{$\displaystyle\centerdot$}}}%
             {\mathbin{\raisebox{0.5ex}{$\centerdot$}}}%
             {\mathbin{\raisebox{0.25ex}{$\scriptstyle\,\centerdot\,$}}}%
             {\mathbin{\raisebox{0.1ex}{$\scriptscriptstyle\,\centerdot\,$}}}
}

%%% Path reversal %%%
\newcommand{\opp}[1]{\mathord{{#1}^{-1}}}
\let\rev\opp

%%% Transport (covariant) %%%
\newcommand{\trans}[2]{\ensuremath{{#1}_{*}\mathopen{}\left({#2}\right)\mathclose{}}\xspace}
\let\Trans\trans
%\newcommand{\Trans}[2]{\ensuremath{{#1}_{*}\left({#2}\right)}\xspace}
\newcommand{\transf}[1]{\ensuremath{{#1}_{*}}\xspace} % Without argument
%\newcommand{\transport}[2]{\ensuremath{\mathsf{transport}_{*} \: {#2}\xspace}}
\newcommand{\transfib}[3]{\ensuremath{\mathsf{transport}^{#1}(#2,#3)\xspace}}
\newcommand{\Transfib}[3]{\ensuremath{\mathsf{transport}^{#1}\Big(#2,\, #3\Big)\xspace}}
\newcommand{\transfibf}[1]{\ensuremath{\mathsf{transport}^{#1}\xspace}}

%%% 2D transport
\newcommand{\transtwo}[2]{\ensuremath{\mathsf{transport}^2\mathopen{}\left({#1},{#2}\right)\mathclose{}}\xspace}

%%% Constant transport
\newcommand{\transconst}[3]{\ensuremath{\mathsf{transportconst}}^{#1}_{#2}(#3)\xspace}
\newcommand{\transconstf}{\ensuremath{\mathsf{transportconst}}\xspace}

%%% Map on paths %%%
\newcommand{\mapfunc}[1]{\ensuremath{\mathsf{ap}_{#1}}\xspace} % Without argument
\newcommand{\map}[2]{\ensuremath{{#1}\mathopen{}\left({#2}\right)\mathclose{}}\xspace}
\let\Ap\map
%\newcommand{\Ap}[2]{\ensuremath{{#1}\left({#2}\right)}\xspace}
\newcommand{\mapdepfunc}[1]{\ensuremath{\mathsf{apd}_{#1}}\xspace} % Without argument
% \newcommand{\mapdep}[2]{\ensuremath{{#1}\llparenthesis{#2}\rrparenthesis}\xspace}
\newcommand{\mapdep}[2]{\ensuremath{\mapdepfunc{#1}\mathopen{}\left(#2\right)\mathclose{}}\xspace}
\let\apfunc\mapfunc
\let\ap\map
\let\apdfunc\mapdepfunc
\let\apd\mapdep

%%% 2D map on paths
\newcommand{\aptwofunc}[1]{\ensuremath{\mathsf{ap}^2_{#1}}\xspace}
\newcommand{\aptwo}[2]{\ensuremath{\aptwofunc{#1}\mathopen{}\left({#2}\right)\mathclose{}}\xspace}
\newcommand{\apdtwofunc}[1]{\ensuremath{\mathsf{apd}^2_{#1}}\xspace}
\newcommand{\apdtwo}[2]{\ensuremath{\apdtwofunc{#1}\mathopen{}\left(#2\right)\mathclose{}}\xspace}

%%% Identity functions %%%
\newcommand{\idfunc}[1][]{\ensuremath{\mathsf{id}_{#1}}\xspace}

%%% Homotopies (written infix) %%%
\newcommand{\htpy}{\sim}

%%% Other meanings of \sim
\newcommand{\bisim}{\sim}       % bisimulation
\newcommand{\eqr}{\sim}         % an equivalence relation

%%% Equivalence types %%%
\newcommand{\eqv}[2]{\ensuremath{#1 \simeq #2}\xspace}
\newcommand{\eqvspaced}[2]{\ensuremath{#1 \;\simeq\; #2}\xspace}
\newcommand{\eqvsym}{\simeq}    % infix symbol
\newcommand{\texteqv}[2]{\ensuremath{\mathsf{Equiv}(#1,#2)}\xspace}
\newcommand{\isequiv}{\ensuremath{\mathsf{isequiv}}}
\newcommand{\qinv}{\ensuremath{\mathsf{qinv}}}
\newcommand{\ishae}{\ensuremath{\mathsf{ishae}}}
\newcommand{\linv}{\ensuremath{\mathsf{linv}}}
\newcommand{\rinv}{\ensuremath{\mathsf{rinv}}}
\newcommand{\biinv}{\ensuremath{\mathsf{biinv}}}
\newcommand{\lcoh}[3]{\mathsf{lcoh}_{#1}(#2,#3)}
\newcommand{\rcoh}[3]{\mathsf{rcoh}_{#1}(#2,#3)}
\newcommand{\hfib}[2]{{\mathsf{fib}}_{#1}(#2)}

%%% Map on total spaces %%%
\newcommand{\total}[1]{\ensuremath{\mathsf{total}(#1)}}

%%% Universe types %%%
%\newcommand{\type}{\ensuremath{\mathsf{Type}}\xspace}
\newcommand{\UU}{\ensuremath{\mathcal{U}}\xspace}
\let\bbU\UU
\let\type\UU
% Universes of truncated types
\newcommand{\typele}[1]{\ensuremath{{#1}\text-\mathsf{Type}}\xspace}
\newcommand{\typeleU}[1]{\ensuremath{{#1}\text-\mathsf{Type}_\UU}\xspace}
\newcommand{\typelep}[1]{\ensuremath{{(#1)}\text-\mathsf{Type}}\xspace}
\newcommand{\typelepU}[1]{\ensuremath{{(#1)}\text-\mathsf{Type}_\UU}\xspace}
\let\ntype\typele
\let\ntypeU\typeleU
\let\ntypep\typelep
\let\ntypepU\typelepU
\renewcommand{\set}{\ensuremath{\mathsf{Set}}\xspace}
\newcommand{\setU}{\ensuremath{\mathsf{Set}_\UU}\xspace}
\newcommand{\prop}{\ensuremath{\mathsf{Prop}}\xspace}
\newcommand{\propU}{\ensuremath{\mathsf{Prop}_\UU}\xspace}
%Pointed types
\newcommand{\pointed}[1]{\ensuremath{#1_\bullet}}

%%% Ordinals and cardinals
\newcommand{\card}{\ensuremath{\mathsf{Card}}\xspace}
\newcommand{\ord}{\ensuremath{\mathsf{Ord}}\xspace}
\newcommand{\ordsl}[2]{{#1}_{/#2}}

%%% Univalence
\newcommand{\ua}{\ensuremath{\mathsf{ua}}\xspace} % the inverse of idtoeqv
\newcommand{\idtoeqv}{\ensuremath{\mathsf{idtoeqv}}\xspace}
\newcommand{\univalence}{\ensuremath{\mathsf{univalence}}\xspace} % the full axiom

%%% Truncation levels
\newcommand{\iscontr}{\ensuremath{\mathsf{isContr}}}
\newcommand{\contr}{\ensuremath{\mathsf{contr}}} % The path to the center of contraction
\newcommand{\isset}{\ensuremath{\mathsf{isSet}}}
\newcommand{\isprop}{\ensuremath{\mathsf{isProp}}}
% h-propositions
% \newcommand{\anhprop}{a mere proposition\xspace}
% \newcommand{\hprops}{mere propositions\xspace}

%%% Homotopy fibers %%%
%\newcommand{\hfiber}[2]{\ensuremath{\mathsf{hFiber}(#1,#2)}\xspace}
\let\hfiber\hfib

%%% Bracket/squash/truncation types %%%
% \newcommand{\brck}[1]{\textsf{mere}(#1)}
% \newcommand{\Brck}[1]{\textsf{mere}\Big(#1\Big)}
% \newcommand{\trunc}[2]{\tau_{#1}(#2)}
% \newcommand{\Trunc}[2]{\tau_{#1}\Big(#2\Big)}
% \newcommand{\truncf}[1]{\tau_{#1}}
%\newcommand{\trunc}[2]{\Vert #2\Vert_{#1}}
\newcommand{\trunc}[2]{\mathopen{}\left\Vert #2\right\Vert_{#1}\mathclose{}}
\newcommand{\ttrunc}[2]{\bigl\Vert #2\bigr\Vert_{#1}}
\newcommand{\Trunc}[2]{\Bigl\Vert #2\Bigr\Vert_{#1}}
\newcommand{\truncf}[1]{\Vert \blank \Vert_{#1}}
\newcommand{\tproj}[3][]{\mathopen{}\left|#3\right|_{#2}^{#1}\mathclose{}}
\newcommand{\tprojf}[2][]{|\blank|_{#2}^{#1}}
\def\pizero{\trunc0}
%\newcommand{\brck}[1]{\trunc{-1}{#1}}
%\newcommand{\Brck}[1]{\Trunc{-1}{#1}}
%\newcommand{\bproj}[1]{\tproj{-1}{#1}}
%\newcommand{\bprojf}{\tprojf{-1}}

\newcommand{\brck}[1]{\trunc{}{#1}}
\newcommand{\bbrck}[1]{\ttrunc{}{#1}}
\newcommand{\Brck}[1]{\Trunc{}{#1}}
\newcommand{\bproj}[1]{\tproj{}{#1}}
\newcommand{\bprojf}{\tprojf{}}

% Big parentheses
\newcommand{\Parens}[1]{\Bigl(#1\Bigr)}

% Projection and extension for truncations
\let\extendsmb\ext
\newcommand{\extend}[1]{\extendsmb(#1)}

%
%%% The empty type
\newcommand{\emptyt}{\ensuremath{\mathbf{0}}\xspace}

%%% The unit type
\newcommand{\unit}{\ensuremath{\mathbf{1}}\xspace}
\newcommand{\ttt}{\ensuremath{\star}\xspace}

%%% The two-element type
\newcommand{\bool}{\ensuremath{\mathbf{2}}\xspace}
\newcommand{\btrue}{{1_{\bool}}}
\newcommand{\bfalse}{{0_{\bool}}}

%%% Injections into binary sums and pushouts
\newcommand{\inlsym}{{\mathsf{inl}}}
\newcommand{\inrsym}{{\mathsf{inr}}}
\newcommand{\inl}{\ensuremath\inlsym\xspace}
\newcommand{\inr}{\ensuremath\inrsym\xspace}

%%% The segment of the interval
\newcommand{\seg}{\ensuremath{\mathsf{seg}}\xspace}

%%% Free groups
\newcommand{\freegroup}[1]{F(#1)}
\newcommand{\freegroupx}[1]{F'(#1)} % the "other" free group

%%% Glue of a pushout
\newcommand{\glue}{\mathsf{glue}}

%%% Circles and spheres
\newcommand{\Sn}{\mathbb{S}}
\newcommand{\base}{\ensuremath{\mathsf{base}}\xspace}
\newcommand{\lloop}{\ensuremath{\mathsf{loop}}\xspace}
\newcommand{\surf}{\ensuremath{\mathsf{surf}}\xspace}

%%% Suspension
\newcommand{\susp}{\Sigma}
\newcommand{\north}{\mathsf{N}}
\newcommand{\south}{\mathsf{S}}
\newcommand{\merid}{\mathsf{merid}}

%%% Blanks (shorthand for lambda abstractions)
\newcommand{\blank}{\mathord{\hspace{1pt}\text{--}\hspace{1pt}}}

%%% Nameless objects
\newcommand{\nameless}{\mathord{\hspace{1pt}\underline{\hspace{1ex}}\hspace{1pt}}}

%%% Some decorations
%\newcommand{\bbU}{\ensuremath{\mathbb{U}}\xspace}
% \newcommand{\bbB}{\ensuremath{\mathbb{B}}\xspace}
\newcommand{\bbP}{\ensuremath{\mathbb{P}}\xspace}

%%% Some categories
\newcommand{\uset}{\ensuremath{\mathcal{S}et}\xspace}
\newcommand{\ucat}{\ensuremath{{\mathcal{C}at}}\xspace}
\newcommand{\urel}{\ensuremath{\mathcal{R}el}\xspace}
\newcommand{\uhilb}{\ensuremath{\mathcal{H}ilb}\xspace}
\newcommand{\utype}{\ensuremath{\mathcal{T}\!ype}\xspace}

% Pullback corner
%\newbox\pbbox
%\setbox\pbbox=\hbox{\xy \POS(65,0)\ar@{-} (0,0) \ar@{-} (65,65)\endxy}
%\def\pb{\save[]+<3.5mm,-3.5mm>*{\copy\pbbox} \restore}

% Macros for the categories chapter
\newcommand{\inv}[1]{{#1}^{-1}}
\newcommand{\idtoiso}{\ensuremath{\mathsf{idtoiso}}\xspace}
\newcommand{\isotoid}{\ensuremath{\mathsf{isotoid}}\xspace}
\newcommand{\op}{^{\mathrm{op}}}
\newcommand{\y}{\ensuremath{\mathbf{y}}\xspace}
\newcommand{\dgr}[1]{{#1}^{\dagger}}
\newcommand{\unitaryiso}{\mathrel{\cong^\dagger}}
\newcommand{\cteqv}[2]{\ensuremath{#1 \simeq #2}\xspace}
\newcommand{\cteqvsym}{\simeq}     % Symbol for equivalence of categories

%%% Natural numbers
\newcommand{\N}{\ensuremath{\mathbb{N}}\xspace}
%\newcommand{\N}{\textbf{N}}
\let\nat\N
\newcommand{\natp}{\ensuremath{\nat'}\xspace} % alternative nat in induction chapter

\newcommand{\zerop}{\ensuremath{0'}\xspace}   % alternative zero in induction chapter
\newcommand{\suc}{\mathsf{succ}}
\newcommand{\sucp}{\ensuremath{\suc'}\xspace} % alternative suc in induction chapter
\newcommand{\add}{\mathsf{add}}
\newcommand{\ack}{\mathsf{ack}}
\newcommand{\ite}{\mathsf{iter}}
\newcommand{\assoc}{\mathsf{assoc}}
\newcommand{\dbl}{\ensuremath{\mathsf{double}}}
\newcommand{\dblp}{\ensuremath{\dbl'}\xspace} % alternative double in induction chapter


%%% Lists
\newcommand{\lst}[1]{\mathsf{List}(#1)}
\newcommand{\nil}{\mathsf{nil}}
\newcommand{\cons}{\mathsf{cons}}

%%% Vectors of given length, used in induction chapter
\newcommand{\vect}[2]{\ensuremath{\mathsf{Vec}_{#1}(#2)}\xspace}

%%% Integers
\newcommand{\Z}{\ensuremath{\mathbb{Z}}\xspace}
\newcommand{\Zsuc}{\mathsf{succ}}
\newcommand{\Zpred}{\mathsf{pred}}

%%% Rationals
\newcommand{\Q}{\ensuremath{\mathbb{Q}}\xspace}

%%% Function extensionality
\newcommand{\funext}{\mathsf{funext}}
\newcommand{\happly}{\mathsf{happly}}

%%% A naturality lemma
\newcommand{\com}[3]{\mathsf{swap}_{#1,#2}(#3)}

%%% Code/encode/decode
\newcommand{\code}{\ensuremath{\mathsf{code}}\xspace}
\newcommand{\encode}{\ensuremath{\mathsf{encode}}\xspace}
\newcommand{\decode}{\ensuremath{\mathsf{decode}}\xspace}

% Function definition with domain and codomain
\newcommand{\function}[4]{\left\{\begin{array}{rcl}#1 &
      \longrightarrow & #2 \\ #3 & \longmapsto & #4 \end{array}\right.}

%%% Cones and cocones
\newcommand{\cone}[2]{\mathsf{cone}_{#1}(#2)}
\newcommand{\cocone}[2]{\mathsf{cocone}_{#1}(#2)}
% Apply a function to a cocone
\newcommand{\composecocone}[2]{#1\circ#2}
\newcommand{\composecone}[2]{#2\circ#1}
%%% Diagrams
\newcommand{\Ddiag}{\mathscr{D}}

%%% (pointed) mapping spaces
\newcommand{\Map}{\mathsf{Map}}

%%% The interval
\newcommand{\interval}{\ensuremath{I}\xspace}
\newcommand{\izero}{\ensuremath{0_{\interval}}\xspace}
\newcommand{\ione}{\ensuremath{1_{\interval}}\xspace}

%%% Arrows
\newcommand{\epi}{\ensuremath{\twoheadrightarrow}}
\newcommand{\mono}{\ensuremath{\rightarrowtail}}

%%% Sets
\newcommand{\bin}{\ensuremath{\mathrel{\widetilde{\in}}}}

%%% Semigroup structure
\newcommand{\semigroupstrsym}{\ensuremath{\mathsf{SemigroupStr}}}
\newcommand{\semigroupstr}[1]{\ensuremath{\mathsf{SemigroupStr}}(#1)}
\newcommand{\semigroup}[0]{\ensuremath{\mathsf{Semigroup}}}

%%% Macros for the formal type theory
\newcommand{\emptyctx}{\ensuremath{\cdot}}
\newcommand{\production}{\vcentcolon\vcentcolon=}
\newcommand{\conv}{\downarrow}
\newcommand{\wfctx}[1]{#1\ \ctx}
\newcommand{\oftp}[3]{#1 \vdash #2 : #3}
\newcommand{\jdeqtp}[4]{#1 \vdash #2 \jdeq #3 : #4}
\newcommand{\judg}[2]{#1 \vdash #2}
\newcommand{\tmtp}[2]{#1 \mathord{:} #2}

% rule names
\newcommand{\form}{\textsc{form}}
\newcommand{\intro}{\textsc{intro}}
\newcommand{\elim}{\textsc{elim}}
\newcommand{\comp}{\textsc{comp}}
\newcommand{\uniq}{\textsc{uniq}}
\newcommand{\Weak}{\mathsf{Wkg}}
\newcommand{\Vble}{\mathsf{Vble}}
\newcommand{\Exch}{\mathsf{Exch}}
\newcommand{\Subst}{\mathsf{Subst}}

%%% Macros for HITs
\newcommand{\cc}{\mathsf{c}}
\newcommand{\pp}{\mathsf{p}}
\newcommand{\cct}{\widetilde{\mathsf{c}}}
\newcommand{\ppt}{\widetilde{\mathsf{p}}}
\newcommand{\Wtil}{\ensuremath{\widetilde{W}}\xspace}

%%% Macros for n-types
\newcommand{\istype}[1]{\mathsf{is}\mbox{-}{#1}\mbox{-}\mathsf{type}}
\newcommand{\nplusone}{\ensuremath{(n+1)}}
\newcommand{\nminusone}{\ensuremath{(n-1)}}
\newcommand{\fact}{\mathsf{fact}}

%%% Macros for homotopy
\newcommand{\kbar}{\overline{k}} % Used in van Kampen's theorem

%%% Macros for induction
\newcommand{\natw}{\ensuremath{\mathbf{N^w}}\xspace}
\newcommand{\zerow}{\ensuremath{0^\mathbf{w}}\xspace}
\newcommand{\sucw}{\ensuremath{\mathbf{s^w}}\xspace}
\newcommand{\nalg}{\nat\mathsf{Alg}}
\newcommand{\nhom}{\nat\mathsf{Hom}}
\newcommand{\ishinitw}{\mathsf{isHinit}_{\mathsf{W}}}
\newcommand{\ishinitn}{\mathsf{isHinit}_\nat}
\newcommand{\w}{\mathsf{W}}
\newcommand{\walg}{\w\mathsf{Alg}}
\newcommand{\whom}{\w\mathsf{Hom}}

%%% Macros for real numbers
\newcommand{\RC}{\ensuremath{\mathbb{R}_\mathsf{c}}\xspace} % Cauchy
\newcommand{\RD}{\ensuremath{\mathbb{R}_\mathsf{d}}\xspace} % Dedekind
\newcommand{\R}{\ensuremath{\mathbb{R}}\xspace}           % Either 
\newcommand{\barRD}{\ensuremath{\bar{\mathbb{R}}_\mathsf{d}}\xspace} % Dedekind completion of Dedekind

\newcommand{\close}[1]{\sim_{#1}} % Relation of closeness
\newcommand{\closesym}{\mathord\sim}
\newcommand{\rclim}{\mathsf{lim}} % HIT constructor for Cauchy reals
\newcommand{\rcrat}{\mathsf{rat}} % Embedding of rationals into Cauchy reals
\newcommand{\rceq}{\mathsf{eq}_{\RC}} % HIT path constructor
\newcommand{\CAP}{\mathcal{C}}    % The type of Cauchy approximations
\newcommand{\Qp}{\Q_{+}}
\newcommand{\apart}{\mathrel{\#}}  % apartness
\newcommand{\dcut}{\mathsf{isCut}}  % Dedekind cut
\newcommand{\cover}{\triangleleft} % inductive cover
\newcommand{\intfam}[3]{(#2, \lam{#1} #3)} % family of rational intervals

% Macros for the Cauchy reals construction
\newcommand{\bsim}{\frown}
\newcommand{\bbsim}{\smile}

\newcommand{\hapx}{\diamondsuit\approx}
\newcommand{\hapname}{\diamondsuit}
\newcommand{\hapxb}{\heartsuit\approx}
\newcommand{\hapbname}{\heartsuit}
\newcommand{\tap}[1]{\bullet\approx_{#1}\triangle}
\newcommand{\tapname}{\triangle}
\newcommand{\tapb}[1]{\bullet\approx_{#1}\square}
\newcommand{\tapbname}{\square}

%%% Macros for surreals
\newcommand{\NO}{\ensuremath{\mathsf{No}}\xspace}
\newcommand{\surr}[2]{\{\,#1\,\big|\,#2\,\}}
\newcommand{\LL}{\mathcal{L}}
\newcommand{\RR}{\mathcal{R}}
\newcommand{\noeq}{\mathsf{eq}_{\NO}} % HIT path constructor

\newcommand{\ble}{\trianglelefteqslant}
\newcommand{\blt}{\vartriangleleft}
\newcommand{\bble}{\sqsubseteq}
\newcommand{\bblt}{\sqsubset}

\newcommand{\hle}{\diamondsuit\preceq}
\newcommand{\hlt}{\diamondsuit\prec}
\newcommand{\hlname}{\diamondsuit}
\newcommand{\hleb}{\heartsuit\preceq}
\newcommand{\hltb}{\heartsuit\prec}
\newcommand{\hlbname}{\heartsuit}
% \newcommand{\tle}{(\bullet\preceq\triangle)}
% \newcommand{\tlt}{(\bullet\prec\triangle)}
\newcommand{\tle}{\triangle\preceq}
\newcommand{\tlt}{\triangle\prec}
\newcommand{\tlname}{\triangle}
% \newcommand{\tleb}{(\bullet\preceq\square)}
% \newcommand{\tltb}{(\bullet\prec\square)}
\newcommand{\tleb}{\square\preceq}
\newcommand{\tltb}{\square\prec}
\newcommand{\tlbname}{\square}

%%% Macros for set theory
\newcommand{\vset}{\mathsf{set}}  % point constructor for cummulative hierarchy V
\def\cd{\tproj0}
\newcommand{\inj}{\ensuremath{\mathsf{inj}}} % type of injections
\newcommand{\acc}{\ensuremath{\mathsf{acc}}} % accessibility

\newcommand{\atMostOne}{\mathsf{atMostOne}}

\newcommand{\power}[1]{\mathcal{P}(#1)} % power set
\newcommand{\powerp}[1]{\mathcal{P}_+(#1)} % inhabited power set

%%%% THEOREM ENVIRONMENTS %%%%

% Hyperref includes the command \autoref{...} which is like \ref{...}
% except that it automatically inserts the type of the thing you're
% referring to, e.g. it produces "Theorem 3.8" instead of just "3.8"
% (and makes the whole thing a hyperlink).  This saves a slight amount
% of typing, but more importantly it means that if you decide later on
% that 3.8 should be a Lemma or a Definition instead of a Theorem, you
% don't have to change the name in all the places you referred to it.

% The following hack improves on this by using the same counter for
% all theorem-type environments, so that after Theorem 1.1 comes
% Corollary 1.2 rather than Corollary 1.1.  This makes it much easier
% for the reader to find a particular theorem when flipping through
% the document.
\makeatletter
\def\defthm#1#2#3{%
  %% Ensure all theorem types are numbered with the same counter
  \newaliascnt{#1}{thm}
  \newtheorem{#1}[#1]{#2}
  \aliascntresetthe{#1}
  %% This command tells cleveref's \cref what to call things
  \crefname{#1}{#2}{#3}}

% Now define a bunch of theorem-type environments.
\newtheorem{thm}{Theorem}[section]
\crefname{thm}{Theorem}{Theorems}
%\defthm{prop}{Proposition}   % Probably we shouldn't use "Proposition" in this way
\defthm{cor}{Corollary}{Corollaries}
\defthm{lem}{Lemma}{Lemmas}
\defthm{axiom}{Axiom}{Axioms}
% Since definitions and theorems in type theory are synonymous, should
% we actually use the same theoremstyle for them?
\theoremstyle{definition}
\defthm{defn}{Definition}{Definitions}
\theoremstyle{remark}
\defthm{rmk}{Remark}{Remarks}
\defthm{eg}{Example}{Examples}
\defthm{egs}{Examples}{Examples}
\defthm{notes}{Notes}{Notes}
% Number exercises within chapters, with their own counter.
%\newtheorem{ex}{Exercise}[chapter]
%\crefname{ex}{Exercise}{Exercises}

% Display format for sections
\crefformat{section}{\S#2#1#3}
\Crefformat{section}{Section~#2#1#3}
\crefrangeformat{section}{\S\S#3#1#4--#5#2#6}
\Crefrangeformat{section}{Sections~#3#1#4--#5#2#6}
\crefmultiformat{section}{\S\S#2#1#3}{ and~#2#1#3}{, #2#1#3}{ and~#2#1#3}
\Crefmultiformat{section}{Sections~#2#1#3}{ and~#2#1#3}{, #2#1#3}{ and~#2#1#3}
\crefrangemultiformat{section}{\S\S#3#1#4--#5#2#6}{ and~#3#1#4--#5#2#6}{, #3#1#4--#5#2#6}{ and~#3#1#4--#5#2#6}
\Crefrangemultiformat{section}{Sections~#3#1#4--#5#2#6}{ and~#3#1#4--#5#2#6}{, #3#1#4--#5#2#6}{ and~#3#1#4--#5#2#6}

% Display format for appendices
\crefformat{appendix}{Appendix~#2#1#3}
\Crefformat{appendix}{Appendix~#2#1#3}
\crefrangeformat{appendix}{Appendices~#3#1#4--#5#2#6}
\Crefrangeformat{appendix}{Appendices~#3#1#4--#5#2#6}
\crefmultiformat{appendix}{Appendices~#2#1#3}{ and~#2#1#3}{, #2#1#3}{ and~#2#1#3}
\Crefmultiformat{appendix}{Appendices~#2#1#3}{ and~#2#1#3}{, #2#1#3}{ and~#2#1#3}
\crefrangemultiformat{appendix}{Appendices~#3#1#4--#5#2#6}{ and~#3#1#4--#5#2#6}{, #3#1#4--#5#2#6}{ and~#3#1#4--#5#2#6}
\Crefrangemultiformat{appendix}{Appendices~#3#1#4--#5#2#6}{ and~#3#1#4--#5#2#6}{, #3#1#4--#5#2#6}{ and~#3#1#4--#5#2#6}

\crefname{part}{Part}{Parts}

\crefformat{paragraph}{\S#2#1#3}
\Crefformat{paragraph}{Paragraph~#2#1#3}
\crefrangeformat{paragraph}{\S\S#3#1#4--#5#2#6}
\Crefrangeformat{paragraph}{Paragraphs~#3#1#4--#5#2#6}
\crefmultiformat{paragraph}{\S\S#2#1#3}{ and~#2#1#3}{, #2#1#3}{ and~#2#1#3}
\Crefmultiformat{paragraph}{Paragraphs~#2#1#3}{ and~#2#1#3}{, #2#1#3}{ and~#2#1#3}
\crefrangemultiformat{paragraph}{\S\S#3#1#4--#5#2#6}{ and~#3#1#4--#5#2#6}{, #3#1#4--#5#2#6}{ and~#3#1#4--#5#2#6}
\Crefrangemultiformat{paragraph}{Paragraphs~#3#1#4--#5#2#6}{ and~#3#1#4--#5#2#6}{, #3#1#4--#5#2#6}{ and~#3#1#4--#5#2#6}

% Number subsubsections
\setcounter{secnumdepth}{5}

% Display format for figures
\crefname{figure}{Figure}{Figures}

% Use cleveref instead of hyperref's \autoref
\let\autoref\cref


%%%% EQUATION NUMBERING %%%%

% The following hack uses the single theorem counter to number
% equations as well, so that we don't have both Theorem 1.1 and
% equation (1.1).
\let\c@equation\c@thm
\numberwithin{equation}{section}


%%%% ENUMERATE NUMBERING %%%%

% Number the first level of enumerates as (i), (ii), ...
\renewcommand{\theenumi}{(\roman{enumi})}
\renewcommand{\labelenumi}{\theenumi}


%%%% MARGINS %%%%

% This is a matter of personal preference, but I think the left
% margins on enumerates and itemizes are too wide.
\setitemize[1]{leftmargin=2em}
\setenumerate[1]{leftmargin=*}

% Likewise that they are too spaced out.
\setitemize[1]{itemsep=-0.2em}
\setenumerate[1]{itemsep=-0.2em}

%%% Notes %%%
\def\noteson{%
\gdef\note##1{\mbox{}\marginpar{\color{blue}\textasteriskcentered\ ##1}}}
\gdef\notesoff{\gdef\note##1{\null}}
\noteson

\newcommand{\Coq}{\textsc{Coq}\xspace}
\newcommand{\Agda}{\textsc{Agda}\xspace}
\newcommand{\NuPRL}{\textsc{NuPRL}\xspace}

%%%% CITATIONS %%%%

% \let \cite \citep

%%%% INDEX %%%%

\newcommand{\footstyle}[1]{{\hyperpage{#1}}n} % If you index something that is in a footnote
\newcommand{\defstyle}[1]{\textbf{\hyperpage{#1}}}  % Style for pageref to a definition

\newcommand{\indexdef}[1]{\index{#1|defstyle}}   % Index a definition
\newcommand{\indexfoot}[1]{\index{#1|footstyle}} % Index a term in a footnote
\newcommand{\indexsee}[2]{\index{#1|see{#2}}}    % Index "see also"


%%%% Standard phrasing or spelling of common phrases %%%%

\newcommand{\ZF}{Zermelo--Fraenkel}
\newcommand{\CZF}{Constructive \ZF{} Set Theory}

\newcommand{\LEM}[1]{\ensuremath{\mathsf{LEM}_{#1}}\xspace}
\newcommand{\choice}[1]{\ensuremath{\mathsf{AC}_{#1}}\xspace}

%%%% MISC %%%%

\newcommand{\mentalpause}{\medskip} % Use for "mental" pause, instead of \smallskip or \medskip

%% Use \symlabel instead of \label to mark a pageref that you need in the index of symbols
\newcounter{symindex}
\newcommand{\symlabel}[1]{\refstepcounter{symindex}\label{#1}}

% Local Variables:
% mode: latex
% TeX-master: "hott-online"
% End:


\newcommand{\idsymbin}{=}

%%%%%%%%%%%%%%%%%%%%%%%%%%%%%%%%%%%%%%%%%%%%%%%%%%%%%%%%%%%%%%%%%%%%%%%%%%%%%%%%
%%%% Our commands which are not part of the macros.tex file.
%%%% We should keep these commands separate, because we will
%%%% update the macros.tex following the updates of the book.

%%%% First we redefine the \id, \eqv and \ct commands so that they accept an
%%%% arbitrary number of arguments. This is useful when writing longer strings
%%%% of equalities or equivalences.

\makeatletter

\renewcommand{\id}[3][]{
  \@ifnextchar\bgroup
    {#2 \mathbin{\idsym_{#1}} \id[#1]{#3}}
    {#2 \mathbin{\idsym_{#1}} #3}
  }

\renewcommand{\eqv}[2]{
  \@ifnextchar\bgroup
    {#1 \eqvsym \eqv{#2}}
    {#1 \eqvsym #2}
  }

\newcommand{\ctsym}{%
  \mathchoice{\mathbin{\raisebox{0.5ex}{$\displaystyle\centerdot$}}}%
             {\mathbin{\raisebox{0.5ex}{$\centerdot$}}}%
             {\mathbin{\raisebox{0.25ex}{$\scriptstyle\,\centerdot\,$}}}%
             {\mathbin{\raisebox{0.1ex}{$\scriptscriptstyle\,\centerdot\,$}}}
  }

\renewcommand{\ct}[3][]{
  \@ifnextchar\bgroup
    {#2 \mathbin{\ctsym_{#1}} \ct[#1]{#3}}
    {#2 \mathbin{\ctsym_{#1}} #3}
  }

\makeatother

%%%% We always use textstyle products and sums...
%\renewcommand{\prd}{\tprd}
%\renewcommand{\sm}{\tsm}
\makeatletter
\renewcommand{\@dprd}{\@tprd}
\renewcommand{\@dsm}{\@tsm}
\renewcommand{\@dprd@noparens}{\@tprd}
\renewcommand{\@dsm@noparens}{\@tsm}

%%%% ...with a bit more spacing
\renewcommand{\@tprd}[1]{\mathchoice{{\textstyle\prod_{(#1)}\,}}{\prod_{(#1)}\,}{\prod_{(#1)}\,}{\prod_{(#1)}\,}}
\renewcommand{\@tsm}[1]{\mathchoice{{\textstyle\sum_{(#1)}\,}}{\sum_{(#1)}\,}{\sum_{(#1)}\,}{\sum_{(#1)}\,}}

%%%%%%%%%%%%%%%%%%%%%%%%%%%%%%%%%%%%%%%%%%%%%%%%%%%%%%%%%%%%%%%%%%%%%%%%%%%%%%%%
%%%% We adjust the \prd command so that implicit arguments become possible.
%%%%
%%%% First, we have the following switch. Set it to true if implicit arguments
%%%% are desired, or to false if not. Note turning off implicit arguments
%%%% might render some parts of the text harder to comprehend, since in the
%%%% text might appear $f(x)$ where we would have $f(i,x)$ without implicit
%%%% arguments.

\newcommand{\implicitargumentson}{\boolean{true}}

%%%% If one wants to use implicit arguments in the notation for product types,
%%%% a * has to be put before the argument that has to be implicit.
%%%% For example: in $\prd{x:A}*{y:B}{u:P(y)}Q(x,y,u)$, the argument y is
%%%% implicit. Any of the arguments can be made implicit this way.

%%%% First of all, we should make the command \prd search not only for a
%%%% brace, but also for a star. We introduce an auxiliary command that
%%%% determines whether the next character is a star or brace.
\newcommand{\@ifnextchar@starorbrace}[2]
%  {\@ifnextcharamong{#1}{#2}{*}{\bgroup};}
  {\@ifnextchar*{#1}{\@ifnextchar\bgroup{#1}{#2}}}
  
%%%% When encountering the \prd command, latex should determine whether it
%%%% should print implicit argument brackets or not. So the first branching
%%%% happens right here.
\renewcommand{\prd}{\@ifnextchar*{\@iprd}{\@prd}}

\newcommand{\@prd}[1]
  {\@ifnextchar@starorbrace
    {\prd@parens{#1}}
    {\@ifnextchar\sm{\prd@parens{#1}\@eatsm}{\prd@noparens{#1}}}}
\newcommand{\@prd@parens}{\@ifnextchar*{\@iprd}{\prd@parens}}
\renewcommand{\prd@parens}[1]
  {\@ifnextchar@starorbrace
    {\@theprd{#1}\@prd@parens}
    {\@ifnextchar\sm{\@theprd{#1}\@eatsm}{\@theprd{#1}}}}
\newcommand{\@theprd}[1]
  {\mathchoice{\@dprd{#1}}{\@tprd{#1}}{\@tprd{#1}}{\@tprd{#1}}}
\renewcommand{\dprd}[1]{\@dprd{#1}\@ifnextchar@starorbrace{\dprd}{}}
\renewcommand{\tprd}[1]{\@tprd{#1}\@ifnextchar@starorbrace{\tprd}{}}

%%%% Here we tell the actual symbols to be printed.
\newcommand{\@theiprd}[1]{\mathchoice{\@diprd{#1}}{\@tiprd{#1}}{\@tiprd{#1}}{\@tiprd{#1}}}
\newcommand{\@iprd}[2]{\@ifnextchar@starorbrace%
  {\@theiprd{#2}\@prd@parens}%
  {\@ifnextchar\sm%
    {\@theiprd{#2}\@eatsm}%
    {\@theiprd{#2}}}}
\def\@tiprd#1{
  \ifthenelse{\implicitargumentson}
    {\@@tiprd{#1}\@ifnextchar\bgroup{\@tiprd}{}}
    {\@tprd{#1}}}
\def\@@tiprd#1{\mathchoice{{\textstyle\prod_{\{#1\}}\,}}{\prod_{\{#1\}}\,}{\prod_{\{#1\}}\,}{\prod_{\{#1\}}\,}}
\def\@diprd{
  \ifthenelse{\implicitargumentson}
    {\@tiprd}
    {\@tprd}}
    

%%%% And finally we need to redefine \@eatprd so that implicit arguments also
%%%% works in the scope of a dependent sum.    
\def\@eatprd\prd{\@prd@parens}

\makeatother

%%%%%%%%%%%%%%%%%%%%%%%%%%%%%%%%%%%%%%%%%%%%%%%%%%%%%%%%%%%%%%%%%%%%%%%%%%%%%%%%
%%%% Redefining the quantifiers, so that some of the longer 
%%%% formulas appear one a single line without problems

%%% Dependent products written with \forall, in the same style
\makeatletter
\def\tfall#1{\forall_{(#1)}\@ifnextchar\bgroup{\,\tfall}{\,}}
\renewcommand{\fall}{\tfall}

%%% Existential quantifier %%%
\def\texis#1{\exists_{(#1)}\@ifnextchar\bgroup{\,\texis}{\,}}
\renewcommand{\exis}{\texis}

%%% Unique existence %%%
\def\uexis#1{\exists!_{(#1)}\@ifnextchar\bgroup{\,\uexis}{\,}}
\makeatother

%%%%%%%%%%%%%%%%%%%%%%%%%%%%%%%%%%%%%%%%%%%%%%%%%%%%%%%%%%%%%%%%%%%%%%%%%%%%%%%%
%%%%

\newcommand{\famsym}{\mathcal{F}}
\newcommand{\tmsym}{\mathcal{T}}

%%%%%%%%%%%%%%%%%%%%%%%%%%%%%%%%%%%%%%%%%%%%%%%%%%%%%%%%%%%%%%%%%%%%%%%%%%%%%%%%
%%%% 

%%%%%%%%%%%%%%%%%%%%%%%%%%%%%%%%%%%%%%%%%%%%%%%%%%%%%%%%%%%%%%%%%%%%%%%%%%%%%%%%
%%%% UNFOLD
%%%%
%%%% For each definition in the type theory we make two versions of the macro:
%%%% the macro introducing the new notation and an @unfold version of the macro
%%%% which outputs the meaning of that new notation. Thus, we can use the
%%%% following construction to write our text. When we introduce \macro, we can
%%%% write \unfold{\macro} and the output will be the result of \macro@unfold.

\makeatletter
\newcommand{\unfold}{%
  \unfoldnext}
\newcommand{\unfoldall}[1]{%
  \begingroup%
  \renewcommand{\jhom}{\jhom@unfold}%
  \renewcommand{\jhomeq}{\jhomeq@unfold}%
  \renewcommand{\jhomdefn}{\jhomdefn@unfold}%
  \renewcommand{\jfhom}{\jfhom@unfold}%
  \renewcommand{\jcomp}{\jcomp@unfold}%
  \renewcommand{\@jcomp@nested}{\@jcomp@unfold@nested}%
  \renewcommand{\@jcomp@parens}{\@jcomp@unfold@parens}%
  \renewcommand{\tmext}{\tmext@unfold}%
  \renewcommand{\@tmext@nested}{\@tmext@unfold@nested}%
  \renewcommand{\@tmext@parens}{\@tmext@unfold@parens}%
  \renewcommand{\cprojfstf}{\cprojfstf@unfold}%
  \renewcommand{\cprojfst}{\cprojfst@unfold}%
  \renewcommand{\cprojsndf}{\cprojsndf@unfold}%
  \renewcommand{\cprojsnd}{\cprojsnd@unfold}%
  \renewcommand{\jfcomp}{\jfcomp@unfold}%
%  \renewcommand{\@jfcomp@nested}{\@jfcomp@unfold@nested}%
%  \renewcommand{\@jfcomp@parens}{\@jfcomp@unfold@parens}%
  \renewcommand{\finc}{\finc@unfold}%
  \renewcommand{\jvcomp}{\jvcomp@unfold}%
  \renewcommand{\jfvcomp}{\jfvcomp@unfold}
  \renewcommand{\subst@type@unfold}[1]{
    \@ifnextchar\cprojfstf{\@eatdo{\cprojfstf@parens}}{%
      ##1}
    }
  #1%
  \endgroup%
  }

%%%% The following command is useful when you have checked with '\@ifnextchar'
%%%% that the next character is a macro '\firstmacro' and you want to replace
%%%% it by '\secondmacro'. To establish this, simply call for
%%%% '\@ifnextchar\firstmacro{\@eatdo{\secondmacro}}{}' with the second 
%%%% argument of \@eatdo left unspecified.
\newcommand{\@eatdo}[2]{#1}

%%%% The intention of '\unfoldnext' is to unfold only the definition of the
%%%% next character, provided that it is in the list of unfoldable macros.
\newcommand{\unfoldnext}[1]{%
  \unfold@#1%
}
\newcommand*{\unfold@}[1]{%
  \@ifundefined{unfold@@\detokenize{#1}}{#1}{%
    \csname unfold@@\detokenize{#1}\endcsname
  }%
}
\newcommand*{\unfold@def}[1]{%
  \expandafter\def\csname unfold@@\detokenize{#1}\expandafter\endcsname
  \expandafter{%
    \csname\expandafter\@gobble\string#1@unfold\endcsname
  }%
}

%%%% Unfolding the \jhom judgment and relatives
\newcommand{\jhom@unfold}[4]{%
  \jterm%
    {{#1}{#2}}%
    {\ctxwk{\default@ctxext #2}{\default@ctxext@parens #3}}%
    {#4}%
  }
\unfold@def\jhom

\newcommand{\jhomeq@unfold}[5]{%
  \jtermeq%
    {{#1}{#2}}%
    {\ctxwk{\default@ctxext #2}{\default@ctxext@parens #3}}%
    {#4}%
    {#5}%
  }
\unfold@def\jhomeq

\newcommand{\jhomdefn@unfold}[5]{%
  \jtermdefn%
    {{#1}{#2}}%
    {\ctxwk{\default@ctxext #2}{\default@ctxext@parens #3}}%
    {#4}%
    {#5}%
  }
\unfold@def\jhomdefn

\newcommand{\jfhom@unfold}[7]{%
  \jterm
    {{{#1}{#2}}{#5}}
    {\ctxwk{\default@ctxext #5}{\jcomp{#2}{#4}{#6}}}
    {#7}}
\unfold@def\jfhom

%%%% Unfolding the \jcomp judgment and relatives
\newcommand{\jcomp@unfold}[3]{%
  \subst
    {\jcomp@unfold@test@preside #2}
    {\ctxwk{\default@ctxext #1}{\jcomp@unfold@test@postside #3}}}
\newcommand{\jcomp@unfold@test@preside}{%
  \@ifnextchar\bgroup{\@jcomp@unfold@parens}{}}
\newcommand{\jcomp@unfold@test@postside}{%
  \@ifnextchar\bgroup{\@jcomp@unfold@parens}{%
  \@ctxwk@pass
  }}
\unfold@def\jcomp

\newcommand{\@jcomp@unfold@nested}[4]{%
  \@jcomp@unfold@parens{#2}{#3}{#4}}
\unfold@def\@jcomp@nested

\newcommand{\@jcomp@unfold@parens}[3]{%
  (\jcomp@unfold{#1}{#2}{#3})}
\unfold@def\@jcomp@parens

%%%% The unfolded term extension commands
\newcommand{\tmext@unfold}[2]{%
  \@ifnextchar\bgroup{\tmext@unfold@full{#1}{#2}}{\tmext@short{#1}{#2}}}
\newcommand{\tmext@unfold@full}[4]{%  
  \subst{#4}{{#3}{\idtm{\ctxext{#1}{#2}}}}}
\unfold@def\tmext

\newcommand{\@tmext@unfold@nested}[1]{%
  \@tmext@unfold@parens}
\unfold@def\@tmext@nested

\newcommand{\@tmext@unfold@parens}[4]{%
  (\tmext@unfold{#1}{#2}{#3}{#4})}
\unfold@def\@tmext@parens

%%%% Unfolding the \jvcomp judgment and relatives
\newcommand{\jvcomp@unfold}[3]{%
  \tmext{}{}{\ctxwk{#1}{#2}}{#3}
  }
\newcommand{\jvcomp@unfold@test@preside}{%
  \@ifnextchar\bgroup{\@jvcomp@unfold@parens}{}}
\newcommand{\jvcomp@unfold@test@postside}{%
  \@ifnextchar\bgroup{\@jvcomp@unfold@parens}{}}
\newcommand{\@jvcomp@unfold@nested}[4]{%
  \@jvcomp@unfold@parens{#2}{#3}{#4}}
\newcommand{\@jvcomp@unfold@parens}[3]{%
  (\jvcomp@unfold{#1}{#2}{#3})}
\unfold@def\jvcomp

%%%% Unfolding \jfcomp
\newcommand{\jfcomp@unfold}[5]{%
  \jcomp{#3}{#4}{{#1}{#2}{#5}}}
\unfold@def\jfcomp

%%%% Unfolding the \jfvcomp judgment and relatives
\newcommand{\jfvcomp@unfold}[3]{%
  \tmext{}{}{\ctxwk{#1}{#2}}{#3}
  }
\newcommand{\jfvcomp@unfold@test@preside}{%
  \@ifnextchar\bgroup{\@jfvcomp@unfold@parens}{}}
\newcommand{\jfvcomp@unfold@test@postside}{%
  \@ifnextchar\bgroup{\@jfvcomp@unfold@parens}{}}
\newcommand{\@jfvcomp@unfold@nested}[4]{%
  \@jfvcomp@unfold@parens{#2}{#3}{#4}}
\newcommand{\@jfvcomp@unfold@parens}[3]{%
  (\jfvcomp@unfold{#1}{#2}{#3})}
\unfold@def\jfvcomp

%%%% Unfolding the projections
\newcommand{\cprojfstf@unfold}[2]{%
  \ctxwk{\default@ctxext #2}\idtm{\default@ctxext #1}}
\unfold@def\cprojfstf

\newcommand{\cprojfst@unfold}[3]{%
  \subst{#3}{(\cprojfstf@unfold{#1}{#2})}}
\unfold@def\cprojfst

\newcommand{\cprojsndf@unfold}[2]{%
  \idtm{\default@ctxext #2}}
\unfold@def\cprojsndf

\newcommand{\cprojsnd@unfold}[3]{%
  \subst{#3}{\cprojsnd@unfold{#1}{#2}}}
\unfold@def\cprojsnd

%%%% Unfolding the fiber inclusion
\newcommand{\finc@unfold}[3][]{%
  \tmext%
    {\ctxwk{\subst{#2}{#3}}{#1}}%
    {\ctxwk{\subst{#2}{#3}}{#3}}%
    {\ctxwk{\subst{#2}{#3}}{#2}}%
    {\idtm{\subst{#2}{#3}}}%
  }
\unfold@def\finc

%%%% Unfolding jtermashom
\newcommand{\jtermashom@unfold}[1]{\tmext{\idtm{A}}{#1}}
\unfold@def\jtermashom

%%%% Unfolding algebras
\newcommand{\cftalg@unfold}[1]{%
  ( \cftalgc{\cftalg{#1}},
    \cftalgf{\cftalg{#1}},
    \cftalgt{\cftalg{#1}},
    \cftctxext[\cftalg{#1}],
    \cftfamext[\cftalg{#1}],
    \cftempc{\cftalg{#1}},
    \cftempf{\cftalg{#1}})
  }
\unfold@def\cftalg

\newcommand{\extalg@unfold}[1]{%
  (\cftalgc{\algfont{#1}},
    \cftalgf{\algfont{#1}},
    \cftalgt{\algfont{#1}},
    \cftctxext[\algfont{#1}],
    \cftfamext[\algfont{#1}])
  }
\unfold@def\extalg

\newcommand{\cfthom@unfold}[1]{%
  ( \cfthomc{\cfthom{#1}},
    \cfthomf{\cfthom{#1}},
    \cfthomt{\cfthom{#1}})
  }
\unfold@def\cfthom

\newcommand{\cftidhom@unfold}[1]{
  ( \idtm{\cftalgc{#1}},
    \idtm{\cftalgf{#1}},
    \idtm{\cftalgt{#1}})
  }
\unfold@def\cftidhom

\newcommand{\cfthomcomp@unfold}[2]{%
  ( \jcomp{}{\cfthomc{#1}}{\cfthomc{#2}},
    \jfcomp{}{}{}{\cfthomf{#1}}{\cfthomf{#2}},
    \jfcomp{}{}{}{\cfthomf{#1}}{\cfthomt{#2}})
  }
\unfold@def\cfthomcomp

\makeatother


%%%%%%%%%%%%%%%%%%%%%%%%%%%%%%%%%%%%%%%%%%%%%%%%%%%%%%%%%%%%%%%%%%%%%%%%%%%%%%%%
%%%% THE APPLY MACRO
%%%%
%%%% The apply macro is used in front of a command which are applicable. Here,
%%%% applicable means that there is a version \macro@apply of \macro. The
%%%% apply macro is mainly used in the macros for article-models-ttinternal.tex
%%%% where there appear abstracted versions of the type theoretical operations.
%%%% Those are operations with possibly many arguments, which are rendered
%%%% invisible for the internalization. When one does want to apply the 
%%%% arguments, the apply macro provides a clean syntax compared to using
%%%% \subst directly.

\makeatletter
\newcommand{\apply}[1]{%
\apply@#1%
}

\newcommand*{\apply@}[1]{%
  \@ifundefined{apply@@\detokenize{#1}}{#1}{%
    \csname apply@@\detokenize{#1}\endcsname
  }%
}

\newcommand*{\apply@def}[1]{%
  \expandafter\def\csname apply@@\detokenize{#1}\expandafter\endcsname
  \expandafter{%
    \csname\expandafter\@gobble\string#1@apply\endcsname
  }%
}

%%%% Now we make @apply versions of various macros

\newcommand{\cftctxext@apply}[3][]{\subst{#3}{{#2}{\cftctxext[#1]}}}
\apply@def\cftctxext

\newcommand{\cftfamext@apply}[4][]{\subst{#4}{{#3}{{#2}{\cftfamext[#1]}}}}
\apply@def\cftfamext

\newcommand{\cftwk@apply}[3]{\subst{#3}{{#2}{\cftwk{#1}}}}
\apply@def\cftwk

\newcommand{\cftwkc@apply}[4]{\subst{#4}{{#3}{{#2}{\cftwkc{#1}}}}}
\apply@def\cftwkc

\newcommand{\cftwkf@apply}[5]{\subst{#5}{{#4}{{#3}{{#2}{\cftwkf{#1}}}}}}
\apply@def\cftwkf

\newcommand{\cftwkt@apply}[6]{\subst{#6}{{#5}{{#4}{{#3}{{#2}{\cftwkt{#1}}}}}}}
\apply@def\cftwkt

\newcommand{\cftsubst@apply}[4]{\subst{#4}{{#3}{{#2}{\cftsubst{#1}}}}}
\apply@def\cftsubst

\newcommand{\cftsubstc@apply}[5]{\subst{#5}{{#4}{{#3}{{#2}{\cftsubstc{#1}}}}}}
\apply@def\cftsubstc

\newcommand{\cftsubstf@apply}[6]{\subst{#6}{{#5}{{#4}{{#3}{{#2}{\cftsubstf{#1}}}}}}}
\apply@def\cftsubstf

\newcommand{\cftsubstt@apply}[7]{\subst{#7}{{#6}{{#5}{{#4}{{#3}{{#2}{\cftsubstt{#1}}}}}}}}
\apply@def\cftsubstt
\makeatother


%%%%%%%%%%%%%%%%%%%%%%%%%%%%%%%%%%%%%%%%%%%%%%%%%%%%%%%%%%%%%%%%%%%%%%%%%%%%%%%%
%%%% A PRETTY PRINTER
%%%%
%%%% We write a \pretty command that pretty prints judgments or types by
%%%% diplaying variables and omitting explicit notation for weakening.
%%%%
%%%% This command should work similar to the \unfold command
%%%%
%%%% -- UNDER CONSTRUCTION

\makeatletter
\newcommand{\vardis}[2]{\@vardis@type #2{}(\@vardis@term #1)}
\newcommand{\@vardis}{\@ifnextchar\bgroup{\@@vardis}{}}
\newcommand{\@@vardis}[1]{\@ifnextchar\bgroup{\vardis{#1}}{#1}}
\newcommand{\@vardis@term}{\@vardis}
\newcommand{\@vardis@type}{\@ifnextchar\ctxext{\@ctxext@nested}{\@ifnextchar\ctxwk{\@ctxwk@nested}{\@vardis}}}
\newcommand{\@vardis@nested}[3]{\@vardis@parens{#2}{#3}}
\newcommand{\@vardis@parens}[2]{(\vardis{#1}{#2})}
\makeatother

\makeatletter
\newcommand{\jvctx}{\jctx}
\newcommand{\jvctxeq}{\jctxeq}

\newcommand{\cctxextcombi}[2]{\@ifnextchar\bgroup{\@cctxextcombi #1}{#1:}#2}
\newcommand{\@cctxextcombi}[4]{\cctxext{{\cctxextcombi{#1}{#3}}{\@@cctxextcombi{#1}{#2}{#4}}}}
\newcommand{\@@cctxextcombi}[3]{\@ifnextchar\bgroup{\@@@ctxextcombi #2}{#2(#1):}#3(\cctxext{#1})}
\newcommand{\@@@ctxextcombi}[8] % the 5th argument is (, the 6th is \cctxext and the 8th is ).
  {\@@ctxextcombi{#7}{#1}{#3},\@@ctxextcombi{{#7}{#3}}{#2}{#4}}
\newcommand{\cctxext}[1]{\@ifnextchar\bgroup{\@cctxext}{}#1}
\newcommand{\@cctxext}[2]{\cctxext{#1},\cctxext{#2}}

\newcommand{\jvfamcombi}[3]{
  \cctxextcombi{#1}{#2} \vdash \vardis{\cctxext{#1}}{#3}
}

\newcommand{\jvfam}{\@ifnextchar*{\@jvfamAlignTrue}{\@jvfamAlignFalse}}
\newcommand{\@jvfamAlignTrue}[4]{\jfam*{#2:#3}{\vardis{#2}{#4}}}
\newcommand{\@jvfamAlignFalse}[3]{\jfam{#1:#2}{\vardis{#1}{#3}}\quad@test}

\newcommand{\jvfameq}{\@ifnextchar*{\@jvfameqAlignTrue}{\@jvfameqAlignFalse}}
\newcommand{\@jvfameqAlignTrue}[5]{\jfameq*{#2:#3}{\vardis{#2}{#4}}{\vardis{#2}{#5}}}
\newcommand{\@jvfameqAlignFalse}[4]{\jfameq{#1:#2}{\vardis{#1}{#3}}{\vardis{#1}{#4}}\quad@test}

\newcommand{\jvtype}{\@ifnextchar*{\@jvtypeAlignTrue}{\@jvtypeAlignFalse}}
\newcommand{\@jvtypeAlignTrue}[4]{\jtype*{#2:#3}{\vardis{#2}{#4}}}
\newcommand{\@jvtypeAlignFalse}[3]{\jtype{#1:#2}{\vardis{#1}{#3}}\quad@test}

\newcommand{\jvtypeeq}{\@ifnextchar*{\@jvtypeeqAlignTrue}{\@jvtypeeqAlignFalse}}
\newcommand{\@jvtypeeqAlignTrue}[5]{\jtypeeq*{#2:#3}{\vardis{#2}{#4}}{\vardis{#2}{#5}}}
\newcommand{\@jvtypeeqAlignFalse}[4]{\jtypeeq{#1:#2}{\vardis{#1}{#3}}{\vardis{#1}{#4}}\quad@test}

\newcommand{\jvterm}{\@ifnextchar*{\@jvtermAlignTrue}{\@jvtermAlignFalse}}
\newcommand{\@jvtermAlignTrue}[5]{\jterm*{#2:#3}{\vardis{#2}{#4}}{\vardis{#2}{#5}}}
\newcommand{\@jvtermAlignFalse}[4]{\jterm{#1:#2}{\vardis{#1}{#3}}{\vardis{#1}{#4}}\quad@test}

\newcommand{\jvtermeq}{\@ifnextchar*{\@jvtermeqAlignTrue}{\@jvtermeqAlignFalse}}
\newcommand{\@jvtermeqAlignTrue}[6]{\jtermeq*{#2:#3}{\vardis{#2}{#4}}{\vardis{#2}{#5}}{\vardis{#2}{#6}}}
\newcommand{\@jvtermeqAlignFalse}[5]{\jtermeq{#1:#2}{\vardis{#1}{#3}}{\vardis{#1}{#4}}{\vardis{#1}{#5}}\quad@test}
\makeatother


%%%%%%%%%%%%%%%%%%%%%%%%%%%%%%%%%%%%%%%%%%%%%%%%%%%%%%%%%%%%%%%%%%%%%%%%%%%%%%%%
%%%% JUDGMENTS
%%%%
%%%% Below we define several commands for the judgments of type theory. There
%%%% are commands
%%%% * \jctx for the judgment that something is a context.
%%%% * \jctxeq for the judgment that two contexts are the same
%%%% * \jtype for the judgment that something is a type in a context
%%%% * \jtypeeq for the judgment that two types in the same context are the same
%%%% * \jterm for the judgment that something is a term of a type in a context
%%%% * \jtermeq for the judgment that two terms of the same type are the same

\makeatletter
% We first make a generic judgment command
\newcommand{\judgment}[7]{%
  \@judgment@ctx{#1} \vdash \@judgment@rel{#2}{#3}{#4}{#5}{#6} #7\quad@test}
\newcommand{\@judgment@ctx}{\@judgment@ext}
\newcommand{\@judgment@rel}[5]{
  { \default@ctxext #1
    }
  #2 
  { \default@ctxext #3
    }
  #4
  { \default@ctxext #5
    }}
\newcommand{\@judgment@kind}[1]{~~\textit{#1}}
\newcommand{\@judgment@ext}[1]{\default@ctxext #1}

\newcommand{\quad@test}{%
  \@ifnextchar\jctx{\quad}{%
  \@ifnextchar\jctxeq{\quad}{%
  \@ifnextchar\jvctx{\quad}{%
  \@ifnextchar\jvctxeq{\quad}{%
  \@ifnextchar\jfam{\quad}{%
  \@ifnextchar\jfameq{\quad}{%
  \@ifnextchar\jvfam{\quad}{%
  \@ifnextchar\jvfameq{\quad}{%
  \@ifnextchar\jtype{\quad}{%
  \@ifnextchar\jtypeeq{\quad}{%
  \@ifnextchar\jvtype{\quad}{%
  \@ifnextchar\jvtypeeq{\quad}{%
  \@ifnextchar\jterm{\quad}{%
  \@ifnextchar\jtermeq{\quad}{%
  \@ifnextchar\jvterm{\quad}{%
  \@ifnextchar\jvtermeq{\quad}{%
  \@ifnextchar\jhom{\quad}{%
  \@ifnextchar\jhomeq{\quad}{%
  \@ifnextchar\jfhom{\quad}{%
  \@ifnextchar\jfhomeq{\quad}{%
  }}}}}}}}}}}}}}}}}}}}}

%%%% Judgments about contexts
\newcommand{\jctx@sym}{%
  \@judgment@kind{ctx}}
\newcommand{\jctx}[1]{%
  \judgment{}{}{}{}{}{#1}{\jctx@sym}}
\newcommand{\jctxeq}[2]{%
  \judgment{}{#1}{\jdeq}{#2}{}{}{\jctx@sym}}
\newcommand{\jctxdefn}[2]{%
  \judgment{}{#1}{\defeq}{#2}{}{}{\jctx@sym}}

%%%% Judgments about families
\newcommand{\jfam@sym}{%
  \@judgment@kind{fam}}
\newcommand{\jfam}[2]{%
  \judgment{#1}{}{}{}{}{#2}{\jfam@sym}}
\newcommand{\jfameq}[3]{%
  \judgment{#1}{#2}{\jdeq}{#3}{}{}{\jfam@sym}}
\newcommand{\jfamdefn}[3]{%
  \judgment{#1}{#2}{\defeq}{#3}{}{}{\jfam@sym}}

%%%% Judgments about types
\newcommand{\jtype@sym}{%
  \@judgment@kind{type}}
\newcommand{\jtype}[2]{%
  \judgment{#1}{}{}{}{}{#2}{\jtype@sym}}
\newcommand{\jtypeeq}[3]{%
  \judgment{#1}{#2}{\jdeq}{#3}{}{}{\jtype@sym}}
\newcommand{\jtypedefn}[3]{%
  \judgment{#1}{#2}{\defeq}{#3}{}{}{\jtype@sym}}
  
%%%% Judgments about terms
\newcommand{\jterm}[3]{%
  \judgment{#1}{}{}{#3}{:}{#2}{}}
\newcommand{\jtermeq}[4]{%
  \judgment{#1}{#3}{\jdeq}{#4}{:}{#2}{}}
\newcommand{\jtermdefn}[4]{%
  \judgment{#1}{#3}{\defeq}{#4}{:}{#2}{}}

%%%% The judgment that f is a morphism from A to B in context \Gamma.
\newcommand{\jhomsym}[3][]{%
  ~~\textit{hom}_{#1}(\default@ctxext #2,\default@ctxext #3)}
\newcommand{\jhom}[4]{%
  \judgment{#1}{}{}{#4}{}{}{\jhomsym{#2}{#3}}}
\newcommand{\jhomeq}[5]{%
  \judgment{#1}{#4}{\jdeq}{#5}{}{}{\jhomsym{#2}{#3}}}
\newcommand{\jhomdefn}[5]{%
  \judgment{#1}{#4}{\defeq}{#5}{}{}{\jhomsym{#2}{#3}}}
\makeatother


%%%%%%%%%%%%%%%%%%%%%%%%%%%%%%%%%%%%%%%%%%%%%%%%%%%%%%%%%%%%%%%%%%%%%%%%%%%%%%%%
%%%% THE EMPTY CONTEXT

\newcommand{\emptysym}{[\;]}
\newcommand{\emptyc}{{\emptysym}}
\newcommand{\emptyf}[1][]{{\emptysym}_{#1}}
\newcommand{\emptytm}[1][]{\typefont{\#}_{#1}}

%%%%%%%%%%%%%%%%%%%%%%%%%%%%%%%%%%%%%%%%%%%%%%%%%%%%%%%%%%%%%%%%%%%%%%%%%%%%%%%%
%%%% CONTEXT EXTENSION 
%%%%
%%%% The context extension command.
%%%%
%%%% To get a feeling of how the command works, here are a few examples.
%%%% \ctxext{A}{B} will print A.B
%%%% \ctxext{{A}{B}}{C} will print (A.B).C
%%%% \ctxext{{{A}{B}}{C}}{{D}{E}} will print ((A.B).C).(D.E)

\makeatletter
\newcommand{\ctxext}[2]{\@ctxext@ctx #1.\@ctxext@type #2}
\newcommand{\@ctxext}{\@ifnextchar\bgroup{\@@ctxext}{}}
\newcommand{\@ctxext@ctx}{%
  \@ifnextchar\ctxext{\@ctxext@nested}{%
  \@ifnextchar\ctxwk{\@ctxwk@nested}{%
  \@ifnextchar\jcomp{\@jcomp@nested}{%
  \@ifnextchar\jvcomp{\@jvcomp@nested}{%
  \@ifnextchar\jfcomp{\@jfcomp@nested}{%
  \@ifnextchar\jfvcomp{\@jfvcomp@nested}{%
  \@ctxext
  }}}}}}}
\newcommand{\@ctxext@type}{%
  \@ifnextchar\ctxext{\@ctxext@nested}{%
  \@ifnextchar\subst{\@subst@nested}{%
  \@ifnextchar\jcomp{\@jcomp@nested}{%
  \@ifnextchar\jvcomp{\@jvcomp@nested}{%
  \@ifnextchar\jfcomp{\@jfcomp@nested}{%
  \@ifnextchar\jfvcomp{\@jfvcomp@nested}{%
  \@ctxext
  }}}}}}}
\newcommand{\@@ctxext}[1]{\@ifnextchar\bgroup{\@ctxext@parens{#1}}{#1}}
\newcommand{\@ctxext@parens}[2]{(\ctxext{#1}{#2})}
\newcommand{\@ctxext@nested}[3]{\@ctxext@parens{#2}{#3}}

%%%% We want that some commands accept binary trees as arguments that default
%%%% into extensions. We make the following command to realize this
\newcommand{\default@ctxext}{\@ifnextchar\bgroup{\ctxext}{}}
\newcommand{\default@ctxext@parens}{\@ifnextchar\bgroup{\@ctxext@parens}{}}
\makeatother

%%%%%%%%%%%%%%%%%%%%%%%%%%%%%%%%%%%%%%%%%%%%%%%%%%%%%%%%%%%%%%%%%%%%%%%%%%%%%%%%
%%%% SUBSTITUTION

%%%% The substitution command will act the following way
%%%%
%%%% \subst{x}{P} will print P[x]
%%%% \subst{x}{{f}{Q}} will print Q[f][x]
%%%% \subst{{x}{f}}{{x}{Q}} will print Q[x][f[x]]

\makeatletter
\newcommand{\subst}[3][]{%
  \@subst@type #3{}[\@subst@term #2]^{#1}}
\newcommand{\@subst}{%
  \@ifnextchar\bgroup{\@@subst}{}}
\newcommand{\@@subst}[1]{%
  \@ifnextchar\bgroup{\subst{#1}}{#1}}
\newcommand{\@subst@term}{%
  \@subst}
\newcommand{\@subst@type}{%
  \@ifnextchar\ctxext{\@ctxext@nested}{%
  \@ifnextchar\ctxwk{\@ctxwk@nested}{%
  \@ifnextchar\jcomp{\@jcomp@nested}{%
  \@ifnextchar\tmext{\@tmext@nested}{%
  \@ifnextchar\jvcomp{\@jvcomp@nested}{%
  \@ifnextchar\jfcomp{\@jfcomp@nested}{%
  \@ifnextchar\jfvcomp{\@jfvcomp@nested}{%
%  \@ifnextchar\mfam{\@mfam@nested}{%
%  \@ifnextchar\mtm{\@mtm@nested}}
\newcommand{\subst@type@unfold}[1]{#1}
\newcommand{\@subst@nested}[3]{%
  \@subst@parens{#2}{#3}}
\newcommand{\@subst@parens}[2]{%
  (\subst{#1}{#2})}
\makeatother

%%%%%%%%%%%%%%%%%%%%%%%%%%%%%%%%%%%%%%%%%%%%%%%%%%%%%%%%%%%%%%%%%%%%%%%%%%%%%%%%
%%%% WEAKENING

%%%% The weakening command is very much like the substitution command.

\makeatletter
\newcommand{\ctxwk}[3][]{%
  \langle\@ctxwk@act #2\rangle^{#1} \@ctxwk@pass #3}
\newcommand{\@ctxwk}{%
  \@ifnextchar\bgroup{\@@ctxwk}{}}
\newcommand{\@@ctxwk}[1]{%
  \@ifnextchar\bgroup{\ctxwk{#1}}{#1}}
\newcommand{\@ctxwk@act}{%
  \@ctxwk}
\newcommand{\@ctxwk@pass}{%
  \@ifnextchar\ctxext{\@ctxext@nested}{%
  \@ifnextchar\subst{\@subst@nested}{%
  \@ifnextchar\jcomp{\@jcomp@nested}{%
  \@ifnextchar\tmext{\@tmext@nested}{%
  \@ifnextchar\jvcomp{\@jvcomp@nested}{%
  \@ifnextchar\jfcomp{\@jfcomp@nested}{%
  \@ifnextchar\jfvcomp{\@jfvcomp@nested}{%
%  \@ifnextchar\mfam{\@mfam@nested}{%
%  \@ifnextchar\mtm{\@mtm@nested}}
\newcommand{\@ctxwk@parens}[2]{%
  (\ctxwk{#1}{#2})}
\newcommand{\@ctxwk@nested}[3]{%
  \@ctxwk@parens{#2}{#3}}
\makeatother

%%%% Not sure if we're gonna need the following.
\newcommand{\ctxwkop}[2]{%
  \ctxwk{#2}{#1}}
  
%%%%%%%%%%%%%%%%%%%%%%%%%%%%%%%%%%%%%%%%%%%%%%%%%%%%%%%%%%%%%%%%%%%%%%%%%%%%%%%%
%%%% IDENTITY TERMS

\makeatletter
\newcommand{\idtm}[1]{\typefont{id}_{\default@ctxext #1}}
\makeatother

%%%%%%%%%%%%%%%%%%%%%%%%%%%%%%%%%%%%%%%%%%%%%%%%%%%%%%%%%%%%%%%%%%%%%%%%%%%%%%%%
%%%% TERM EXTENSION
%%%%
%%%% The term extension command \tmext is slightly complicated because 
%%%% \tmext@unfold should do different things depending on whether it has two
%%%% or four arguments. Thus \tmext has a full form and a short form, where
%%%% the short form has two arguments and the full form has four. 

\makeatletter

%%%% The basic term extension commands
\newcommand{\default@tmext}{\@ifnextchar\bgroup{\tmext}{}}
\newcommand{\tmext}[2]{%
  \@ifnextchar\bgroup{\tmext@full{#1}{#2}}{\tmext@short{#1}{#2}}}
\newcommand{\tmext@full}[4]{%
  \ctxext{\tmext@testleft #3}{\tmext@testright #4}}
\newcommand{\tmext@short}[2]{%
  \ctxext{\tmext@testleft #1}{\tmext@testright #2}}
\newcommand{\tmext@testleft}{%
  \@ifnextchar\bgroup{\@tmext@parens}{%
  \@ifnextchar\tmext{\@tmext@nested}{%
  \@ifnextchar\ctxwk{\@ctxwk@nested}{%
  \@ifnextchar\jcomp{\@jcomp@nested}{%
  \@ifnextchar\jvcomp{\@jvcomp@nested}{%
  \@ifnextchar\jfcomp{\@jfcomp@nested}{%
  \@ifnextchar\jfvcomp{\@jfvcomp@nested}{%
%  \default@tmext
  }}}}}}}}
\newcommand{\tmext@testright}{%
  \@ifnextchar\bgroup{\@tmext@parens}{%
  \@ifnextchar\tmext{\@tmext@nested}{%
  \@ifnextchar\subst{\@subst@nested}{%
  \@ifnextchar\jcomp{\@jcomp@nested}{%
  \@ifnextchar\jvcomp{\@jvcomp@nested}{%
  \@ifnextchar\jfcomp{\@jfcomp@nested}{%
  \@ifnextchar\jfvcomp{\@jfvcomp@nested}{%
  \@ifnextchar\cprojfst{\cprojfst@nested}{%
  \@ifnextchar\cprojsnd{\cprojsnd@nested}{%
%  \default@tmext
  }}}}}}}}}}
\newcommand{\@tmext@nested}[1]{%
  \@tmext@parens}
\newcommand{\@tmext@parens}[2]{%
  \@ifnextchar\bgroup
    {\tmext@full@parens{#1}{#2}}
    {(\tmext@short{#1}{#2})}}
\newcommand{\tmext@full@parens}[4]{%
  (\tmext@full{#1}{#2}{#3}{#4})}
\makeatother

%%%%%%%%%%%%%%%%%%%%%%%%%%%%%%%%%%%%%%%%%%%%%%%%%%%%%%%%%%%%%%%%%%%%%%%%%%%%%%%%
%%%% JUDGMENTAL MORPHISMS

\makeatletter

%%%% Composition of morphisms
\newcommand{\comp@testtonest}[1]{%
  \@ifnextchar\jcomp{\@jcomp@nested}{%
  \@ifnextchar\jvcomp{\@jvcomp@nested}{%
  \@ifnextchar\jfcomp{\@jfcomp@nested}{%
  \@ifnextchar\jfvcomp{\@jfvcomp@nested}{%
  #1
  }}}}}

\newcommand{\jcomp}[3]{%
  \jcomp@testleft #3 \circ \jcomp@testright #2}
\newcommand{\jcomp@testleft}{%
  \@ifnextchar\jcomp{\@jcomp@nested}{%
  \@ifnextchar\ctxwk{\@ctxwk@nested}{%
  \@ifnextchar\ctxext{\@ctxext@nested}{%
  \@ifnextchar\bgroup{\@jcomp@parens}{%
  \@ifnextchar\tmext{\@tmext@nested}{%
  \@ifnextchar\jvcomp{\@jvcomp@nested}{%
  \@ifnextchar\jfcomp{\@jfcomp@nested}{%
  \@ifnextchar\jfvcomp{\@jfvcomp@nested}{%
  }}}}}}}}}
\newcommand{\jcomp@testright}{%
  \@ifnextchar\jcomp{\@jcomp@nested}{%
  \@ifnextchar\subst{\@subst@nested}{%
  \@ifnextchar\ctxext{\@ctxext@nested}{%
  \@ifnextchar\bgroup{\@jcomp@parens}{%
  \@ifnextchar\tmext{\@tmext@nested}{%
  \@ifnextchar\jvcomp{\@jvcomp@nested}{%
  \@ifnextchar\jfcomp{\@jfcomp@nested}{%
  \@ifnextchar\jfvcomp{\@jfvcomp@nested}{%
  }}}}}}}}}
\newcommand{\@jcomp@nested}[4]{%
  \@jcomp@parens{#2}{#3}{#4}}
\newcommand{\@jcomp@parens}[3]{%
  (\jcomp{#1}{#2}{#3})}

%%%% Vertical composition of morphisms.
\newcommand{\jvcomp}[3]{%
  \jvcomp@testleft #2 * \jvcomp@testright #3}
\newcommand{\jvcomp@testleft}{%
  \@ifnextchar\jvcomp{\@jvcomp@nested}{%
  \@ifnextchar\jfvcomp{\@jfvcomp@nested}{%
  \@ifnextchar\ctxwk{\@ctxwk@nested}{%
  \@ifnextchar\ctxext{\@ctxext@nested}{%
  \@ifnextchar\bgroup{\@jvcomp@parens}{%
  \@ifnextchar\tmext{\@tmext@nested}{%
  \@ifnextchar\jcomp{\@jcomp@nested}{%
  \@ifnextchar\jfcomp{\@jfcomp@nested}{%
  }}}}}}}}}
\newcommand{\jvcomp@testright}{%
  \@ifnextchar\jvcomp{\@jvcomp@nested}{%
  \@ifnextchar\jfvcomp{\@jfvcomp@nested}{%
  \@ifnextchar\subst{\@subst@nested}{%
  \@ifnextchar\ctxext{\@ctxext@nested}{%
  \@ifnextchar\bgroup{\@jvcomp@parens}{%
  \@ifnextchar\tmext{\@tmext@nested}{%
  \@ifnextchar\jcomp{\@jcomp@nested}{%
  \@ifnextchar\jfcomp{\@jfcomp@nested}{%
  }}}}}}}}}
\newcommand{\@jvcomp@nested}[4]{%
  \@jvcomp@parens{#2}{#3}{#4}}
\newcommand{\@jvcomp@parens}[3]{%
  (\jvcomp{#1}{#2}{#3})}

%%%% Vertical family composition
\newcommand{\jfvcomp}[3]{%
  \jfvcomp@testleft #2 \circledast \jfvcomp@testright #3}
\newcommand{\jfvcomp@testleft}{%
  \jvcomp@testleft}
\newcommand{\jfvcomp@testright}{%
  \jvcomp@testright}
\newcommand{\@jfvcomp@nested}[4]{%
  \@jfvcomp@parens{#2}{#3}{#4}}
\newcommand{\@jfvcomp@parens}[3]{%
  (\jfvcomp{#1}{#2}{#3})}

%%%% The judgment that F is a morphism from P to Q over f in context \Gamma.
\newcommand{\jfhomsym}[3]{\jhomsym[{#1}]{#2}{#3}}
\newcommand{\jfhom}[7]{
  \judgment{#1}{}{}{#7}{}{}{\jfhomsym{#4}{#5}{#6}}}
\newcommand{\jfhomeq}[8]{%
  \judgment{#1}{#7}{\jdeq}{#8}{}{}{\jhomsym[{#4}]{#5}{#6}}}
\newcommand{\jfhomdefn}[8]{%
  \judgment{#1}{#7}{\defeq}{#8}{}{}{\jhomsym[{#4}]{#5}{#6}}}
    
\newcommand{\jfcomp}[5]{%
%  \jfcomp@testleft #5 \circledcirc \jfcomp@testright #4}
  \jfcomp@testleft #5 \bullet \jfcomp@testright #4}
\newcommand{\jfcomp@testleft}{%
  \@ifnextchar\bgroup{\@jfcomp@parens}{%
  \@ifnextchar\jfcomp{\@jfcomp@nested}{%
  \@ifnextchar\jcomp{\@jcomp@nested}{%
  \@ifnextchar\ctxwk{\@ctxwk@nested}{%
  \@ifnextchar\tmext{\@tmext@nested}{%
  \@ifnextchar\jvcomp{\@jvcomp@nested}{%
  \@ifnextchar\jfvcomp{\@jfvcomp@nested}{%
  }}}}}}}}
\newcommand{\jfcomp@testright}{%
  \@ifnextchar\bgroup{\@jfcomp@parens}{%
  \@ifnextchar\jfcomp{\@jfcomp@nested}{%
  \@ifnextchar\jcomp{\@jcomp@nested}{%
  \@ifnextchar\subst{\@subst@nested}{%
  \@ifnextchar\tmext{\@tmext@nested}{%
  \@ifnextchar\jvcomp{\@jvcomp@nested}{%
  \@ifnextchar\jfvcomp{\@jfvcomp@nested}{%
  }}}}}}}}
\newcommand{\@jfcomp@nested}[1]{%
  \@jfcomp@parens}
\newcommand{\@jfcomp@parens}[5]{%
  (\jfcomp{#1}{#2}{#3}{#4}{#5})}
  
\makeatother

%%%%%%%%%%%%%%%%%%%%%%%%%%%%%%%%%%%%%%%%%%%%%%%%%%%%%%%%%%%%%%%%%%%%%%%%%%%%%%%%
%%%% TERMS AS MORPHISMS

\makeatletter
\newcommand{\jtermashom}[1]{\hat{#1}}
\makeatother

%%%%%%%%%%%%%%%%%%%%%%%%%%%%%%%%%%%%%%%%%%%%%%%%%%%%%%%%%%%%%%%%%%%%%%%%%%%%%%%%
%%%% JUDGMENTAL TRIVIAL COFIBRATIONS

\newcommand{\jtcext}{\tilde}

%%%%%%%%%%%%%%%%%%%%%%%%%%%%%%%%%%%%%%%%%%%%%%%%%%%%%%%%%%%%%%%%%%%%%%%%%%%%%%%%
%%%% CONTEXT PROJECTIONS

\makeatletter
\newcommand{\cprojgenf}[3]{%
  \typefont{pr}^{%
    \@ifnextchar\bgroup{\@ctxext@parens}{%
    \@ifnextchar\ctxext{\@ctxext@nested}{%
    }}
    #2,
    \@ifnextchar\bgroup{\@ctxext@parens}{%
    \@ifnextchar\ctxext{\@ctxext@nested}{%
    }}
    #3
    }_{#1}}
\newcommand{\cprojgen}[4]{%
  \subst{#4}{\cprojgenf{#1}{#2}{#3}}}
\newcommand{\cprojgenf@nested}[1]{%
  \cprojgenf@parens}
\newcommand{\cprojgenf@parens}[3]{%
  (\cprojgenf{#1}{#2}{#3})}
\newcommand{\cprojgen@nested}[1]{%
  \cprojgen@parens}
\newcommand{\cprojgen@parens}[4]{%
  (\cprojgen{#1}{#2}{#3}{#4})}

\newcommand{\cprojfstf}[2]{%
  \cprojgenf{0}{#1}{#2}}
\newcommand{\cprojfstf@nested}[1]{%
  \cprojfstf@parens}
\newcommand{\cprojfstf@parens}[2]{%
  (\cprojfstf{#1}{#2})}

\newcommand{\cprojfst}[3]{%
  \cprojgen{0}{#1}{#2}{#3}}
\newcommand{\cprojfst@nested}[1]{%
  \cprojfst@parens}
\newcommand{\cprojfst@parens}[3]{%
  (\cprojfst{#1}{#2}{#3})}

\newcommand{\cprojsndf}[2]{%
  \cprojgenf{1}{#1}{#2}}
\newcommand{\cprojsndf@nested}[1]{%
  \cprojsndf@parens}
\newcommand{\cprojsndf@parens}[2]{%
  (\cprojsndf{#1}{#2})}

\newcommand{\cprojsnd}[3]{%
  \cprojgen{1}{#1}{#2}{#3}}
\newcommand{\cprojsnd@nested}[1]{%
  \cprojsnd@parens}
\newcommand{\cprojsnd@parens}[3]{%
  (\cprojsnd{#1}{#2}{#3})}
  
\makeatother

%%%%%%%%%%%%%%%%%%%%%%%%%%%%%%%%%%%%%%%%%%%%%%%%%%%%%%%%%%%%%%%%%%%%%%%%%%%%%%%%
%%%% FIBER INCLUSIONS

\makeatletter
\newcommand{\finc}[3][]{\typefont{in}^{#3}_{#2}}
\makeatother

%%%%%%%%%%%%%%%%%%%%%%%%%%%%%%%%%%%%%%%%%%%%%%%%%%%%%%%%%%%%%%%%%%%%%%%%%%%%%%%%
%%%% DEPENDENT FUNCTION TYPES

\makeatletter
\newcommand{\sprd}[2]{\Pi(\default@ctxext #1,\default@ctxext #2)}
\begin{comment}
\newcommand{\@sprd@test@cod}[2]{%
  \@ifnextchar\bgroup{\@sprd@do@cod{#1}}{%
  \Pi(\@sprd@test@dom{#1}{#2} #1,
  }}
\newcommand{\@sprd@do@cod}[4]{%
  \ctxext{\@sprd{#1}{#2}}{\@sprd{#1}{#3}}
  }
\newcommand{\@sprd}[2]{
  \@ifnextchar\bgroup{\@@sprd}{%
    \Pi(}
    #1,{#2})
  }
\newcommand{\@@sprd}[5]{%
  \sprd{#1}{\sprd{#2}{#4}}
  }
\end{comment}

\newcommand{\slam}[3]{%
  \lambda^{{\default@ctxext@parens #1},{\default@ctxext@parens #2}}
  (\default@ctxext #3)
  }
\newcommand{\sev}[3]{\tfev^{{\default@ctxext@parens #1},{\default@ctxext@parens #2}}(#3)}

\newcommand{\evtm}[2]{\mathsf{ev}^{#1,#2}}

\newcommand{\ctxev}[4][]{\varepsilon^{#1}_{#2,#3}(#4)}

\makeatother

%%%%%%%%%%%%%%%%%%%%%%%%%%%%%%%%%%%%%%%%%%%%%%%%%%%%%%%%%%%%%%%%%%%%%%%%%%%%%%%%
%%%% NON-DEPENDENT FUNCTION TYPES

\newcommand{\jfun}[2]{#1\to#2}

%%%%%%%%%%%%%%%%%%%%%%%%%%%%%%%%%%%%%%%%%%%%%%%%%%%%%%%%%%%%%%%%%%%%%%%%%%%%%%%%
%%%% THE UNIT TYPE

\makeatletter
\newcommand{\unitc}[1]{%
  \unit^0_{\default@ctxext #1}}
\newcommand{\unitct}[1]{%
  \ttt^0_{\default@ctxext #1}}
\newcommand{\unitf}[2]{%
  \unit^1_{\default@ctxext #1,\default@ctxext #2}}
\newcommand{\unitft}[2]{%
  \ttt^1_{\default@ctxext #1,\default@ctxext #2}}
\makeatother


%%%%%%%%%%%%%%%%%%%%%%%%%%%%%%%%%%%%%%%%%%%%%%%%%%%%%%%%%%%%%%%%%%%%%%%%%%%%%%%%
%%%% ALIGNING JUDGMENTS
%%%%
%%%% Sometimes judgments need to be aligned and we wish to do so without
%%%% having to break the usual macros for judgments. Therefore we provide the
%%%% \jalign macro. Alignment will always be at the turnstyle symbol.

\makeatletter
\newcommand{\jalign}[1]{%
\jalign@#1%
}

\newcommand*{\jalign@}[1]{%
  \@ifundefined{jalign@@\detokenize{#1}}{#1}{%
    \csname jalign@@\detokenize{#1}\endcsname
  }%
}

\newcommand*{\jalign@def}[1]{%
  \expandafter\def\csname jalign@@\detokenize{#1}\expandafter\endcsname
  \expandafter{%
    \csname\expandafter\@gobble\string#1@jalign\endcsname
  }%
}

%%%%%%%%%%%%%%%%%%%%%%%%%%%%%%%%%%%%%%%%%%%%%%%%%%%%%%%%%%%%%%%%%%%%%%%%%%%%%%%%

\newcommand{\judgment@jalign}[7]{%
  \@judgment@ctx{#1} & \vdash \@judgment@rel{#2}{#3}{#4}{#5}{#6} #7}
\jalign@def\judgment

\newcommand{\jctx@jalign}[1]{%
  \jalign\judgment{}{}{}{}{}{#1}{\jctx@sym}}
\jalign@def\jctx

\newcommand{\jctxeq@jalign}[2]{%
  \jalign\judgment{}{#1}{\jdeq}{#2}{}{}{\jctx@sym}}
\jalign@def\jctxeq

\newcommand{\jctxdefn@jalign}[2]{%
  \jalign\judgment{}{#1}{\defeq}{#2}{}{}{\jctx@sym}}
\jalign@def\jctxdefn

\newcommand{\jfam@jalign}[2]{%
  \jalign\judgment{#1}{}{}{}{}{#2}{\jfam@sym}}
\jalign@def\jfam

\newcommand{\jfameq@jalign}[3]{%
  \jalign\judgment{#1}{#2}{\jdeq}{#3}{}{}{\jfam@sym}}
\jalign@def\jfameq

\newcommand{\jfamdefn@jalign}[3]{%
  \jalign\judgment{#1}{#2}{\defeq}{#3}{}{}{\jfam@sym}}
\jalign@def\jfamdefn

\newcommand{\jtype@jalign}[2]{%
  \jalign\judgment{#1}{}{}{}{}{#2}{\jtype@sym}}
\jalign@def\jtype

\newcommand{\jtypeeq@jalign}[3]{%
  \jalign\judgment{#1}{#2}{\jdeq}{#3}{}{}{\jtype@sym}}
\jalign@def\jtypeeq

\newcommand{\jtypedefn@jalign}[3]{%
  \jalign\judgment{#1}{#2}{\defeq}{#3}{}{}{\jtype@sym}}
\jalign@def\jtypedefn

\newcommand{\jterm@jalign}[3]{%
  \jalign\judgment{#1}{}{}{#3}{:}{#2}{}}
\jalign@def\jterm

\newcommand{\jtermeq@jalign}[4]{%
  \jalign\judgment{#1}{#3}{\jdeq}{#4}{:}{#2}{}}
\jalign@def\jtermeq

\newcommand{\jtermdefn@jalign}[4]{%
  \jalign\judgment{#1}{#3}{\defeq}{#4}{:}{#2}{}}
\jalign@def\jtermdefn

\newcommand{\jhom@jalign}[4]{%
  \jalign\judgment{#1}{}{}{#4}{}{}{\jhomsym{#2}{#3}}}
\jalign@def\jhom

\newcommand{\jhomeq@jalign}[5]{%
  \jalign\judgment{#1}{#4}{\jdeq}{#5}{}{}{\jhomsym{#2}{#3}}}
\jalign@def\jhomeq

\newcommand{\jhomdefn@jalign}[5]{%
  \jalign\judgment{#1}{#4}{\defeq}{#5}{}{}{\jhomsym{#2}{#3}}}
\jalign@def\jhomdefn

\newcommand{\jfhom@jalign}[7]{
  \jalign\judgment{#1}{}{}{#7}{}{}{\jfhomsym{#4}{#5}{#6}}}
\jalign@def\jfhom

\newcommand{\jfhomeq@jalign}[8]{%
  \jalign\judgment{#1}{#7}{\jdeq}{#8}{}{}{\jhomsym[{#4}]{#5}{#6}}}
\jalign@def\jfhomeq

\newcommand{\jfhomdefn@jalign}[8]{%
  \jalign\judgment{#1}{#7}{\defeq}{#8}{}{}{\jhomsym[{#4}]{#5}{#6}}}
\jalign@def\jfhomdefn

\newcommand{\jextalg@jalign}[2]{%
  \jalign\judgment{#1}{}{}{}{}{\extalg{#2}}{\extalg@sym}}
\jalign@def\jextalg

\newcommand{\jextalgctx@jalign}[2]{%
  \jalign\jfam{#1}{\cftalgc{\cftalg{#2}}}}
\jalign@def\jextalgctx

\newcommand{\jextalgfam@jalign}[2]{%
  \jalign\jfam{#1}{\cftalgf{\cftalg{#2}}}}
\jalign@def\jextalgfam

\newcommand{\jextalgtm@jalign}[2]{%
  \jalign\jfam{{#1}{\cftalgf{\cftalg{#2}}}}{\cftalgt{\cftalg{#2}}}}
\jalign@def\jextalgtm

\newcommand{\jextalgctxext@jalign}[2]{%
  \jalign\jhom%
    {#1}%
    {{\cftalgc{\cftalg{#2}}}{\cftalgf{\cftalg{#2}}}}%
    {\cftalgc{\cftalg{#2}}}%
    {\cftctxext[\cftalg{#2}]}}
\jalign@def\jextalgctxext

\newcommand{\jextalgfamext@jalign}[2]{%
  \jalign\jhom
    { {#1}{\cftalgc{\cftalg{#2}}}}
    { {\cftalgf{\cftalg{#2}}}
      {\jcomp{}{\cftctxext{\cftalg{#2}}}{\cftalgf{\cftalg{#2}}}}}
    { \cftalgf{\cftalg{#2}}}
    { \cftfamext[\cftalg{#2}]}}
\jalign@def\jextalgfamext



\makeatother


%%%%%%%%%%%%%%%%%%%%%%%%%%%%%%%%%%%%%%%%%%%%%%%%%%%%%%%%%%%%%%%%%%%%%%%%%%%%%%%%
%%%% ALGEBRAS FOR THE THEORY OF CONTEXTS, FAMILIES AND TERMS

\makeatletter
\newcommand{\algfont}{\mathbf}

%%%% EXTENSION ALGEBRAS
\newcommand{\extalg}{\algfont}
\newcommand{\extalg@sym}{\@judgment@kind{ExtAlg}}
\newcommand{\jextalg}[2]{\judgment{#1}{}{}{}{}{\extalg{#2}}{\extalg@sym}}

\newcommand{\jextalgctx}[2]{%
  \jfam{#1}{\cftalgc{\cftalg{#2}}}}

\newcommand{\jextalgfam}[2]{%
  \jfam{{#1}{\cftalgc{\cftalg{#2}}}}{\cftalgf{\cftalg{#2}}}}

\newcommand{\jextalgtm}[2]{%
  \jfam{{{#1}{\cftalgc{\cftalg{#2}}}}{\cftalgf{\cftalg{#2}}}}{\cftalgt{\cftalg{#2}}}}

\newcommand{\jextalgctxext}[2]{%
  \jhom{#1}{{\cftalgc{\cftalg{#2}}}{\cftalgf{\cftalg{#2}}}}{\cftalgc{\cftalg{#2}}}{\cftctxext[\cftalg{#2}]}}

\newcommand{\jextalgfamext}[2]{%
  \jhom{{#1}{\cftalgc{\cftalg{#2}}}}{{\cftalgf{\cftalg{#2}}}{\jcomp{}{\cftctxext[\cftalg{#2}]}{\cftalgf{\cftalg{#2}}}}}{\cftalgf{\cftalg{#2}}}{\cftfamext[\cftalg{#2}]}}
  
\newcommand{\extalgc}[1]{{\renewcommand\extalg[1]{##1}#1}}
\newcommand{\extalgf}[1]{\mathcal{F}_{#1}}
\newcommand{\extalgt}[1]{\mathcal{T}_{#1}}

%%%% EXTENSION TERMS
\newcommand{\cftext}[2][]{\epsilon_{#1}^{#2}}
\newcommand{\cftctxext}[1][]{\cftext[0]{#1}}
\newcommand{\cftfamext}[1][]{\cftext[1]{#1}}

%%%% THE FAMILY EXTENSION ALGEBRA
\newcommand{\extfamalg}[1]{\algfont{F}_{#1}}

%%%% EXTENSION HOMOMORPHISMS
\newcommand{\exthom}[1]{\algfont{#1}}
\newcommand{\exthomc}[1]{{\renewcommand\exthom[1]{##1}#1}}
\newcommand{\exthomf}[1]{\mathcal{F}_{#1}}
\newcommand{\exthomt}[1]{\mathcal{T}_{#1}}
\newcommand{\extfamhom}[1]{\algfont{F}_{#1}}

%%%% CFT-ALGEBRAS
\newcommand{\cftalg}{\algfont}

%%%% The intended use of `\cftctx` is `\cftctx{\cftalg{#1}}`.
\newcommand{\cftalgc}[1]{\@gobble #1}

%%%% The intended use of `\cftfam` is `\cftfam{\cftalg{#1}}`.
\newcommand{\cftalgf}[1]{\mathcal{F}_{#1}}
\newcommand{\cftalgt}[1]{\mathcal{T}_{#1}}

\newcommand{\cftemp}[2][]{\phi_{#1}^{#2}}
\newcommand{\cftempc}{\cftemp[0]}
\newcommand{\cftempf}{\cftemp[1]}

\newcommand{\cftfamalg}[1]{\algfont{F}_{#1}}

%%%% CFT-HOMOMORPHISMS
\newcommand{\cfthom}{\algfont}

\newcommand{\cfthomc}[1]{%
  { \renewcommand{\cfthom}[1]{##1}
    #1
    }
  }

\newcommand{\cfthomf}[1]{\mathcal{F}_{#1}}

\newcommand{\cfthomcomp}[2]{#2\circ #1}

\newcommand{\cfthomt}[1]{\mathcal{T}_{#1}}

\newcommand{\cftfamhom}[1]{\algfont{F}_{#1}}

\newcommand{\cftidhom}[1]{\algfont{id}_{#1}}

%%%% THE WEAKENING TERM
\newcommand{\cftwk}[1]{\boldsymbol\omega^{#1}}
\newcommand{\cftwkc}[1]{\omega^{#1}}
\newcommand{\cftwkf}[1]{\mathcal{F}_{\boldsymbol\omega^{#1}}}
\newcommand{\cftwkt}[1]{\mathcal{T}_{\boldsymbol\omega^{#1}}}

%%%% THE SUBSTITUTION TERM
\newcommand{\cftsubst}[1]{\boldsymbol\sigma^{#1}}
\newcommand{\cftsubstc}[1]{\sigma^{#1}}
\newcommand{\cftsubstf}[1]{\mathcal{F}_{\cftsubst{#1}}}
\newcommand{\cftsubstt}[1]{\mathcal{T}_{\cftsubst{#1}}}

%%%% THE IDENTITY TERM
\newcommand{\cftidtm}[1]{\boldsymbol\iota^{#1}}

\newcommand{\genealg}[3]{\algfont{E}(#1,#2,#3)}
\newcommand{\genealgc}[3]{E(#1,#2,#3)}
\newcommand{\genealgf}[3]{\cftalgf{\genealg{#1}{#2}{#3}}}
\newcommand{\genealgt}[3]{\cftalgt{\genealg{#1}{#2}{#3}}}
\newcommand{\genealgincl}[3]{\boldsymbol\eta_{#1,#2,#3}}

\makeatother


\newcommand{\cat}{\mathbb}

\makeatletter
\newcommand{\pullback@test}{%
  \@ifnextchar\pullback{\@eatdo{\pullback@parens}}{%
  \@ifnextchar\stesysff{\@eatdo{(\stesysff)}}{}}}
\newcommand{\pullback@parens}[4]{(\pullback{#1}{#2}{#3}{#4})}
\newcommand{\pullback}[4]{
  \pullback@test #1\times_{\pullback@test #3,\pullback@test #4} \pullback@test #2}
\newcommand{\pullbackpr}[3]{\pi_{#1}(#2,#3)}

\newcommand{\eft}[1][]{\mathrm{ft}_{#1}}
\newcommand{\ebd}[1][]{\partial_{#1}}
\newcommand{\esys}[3]{\mathrm{#1#2#3}}
\newcommand{\stesysc}{C}
\newcommand{\stesysf}{F}
\newcommand{\stesyst}{T}
\newcommand{\stesys}{\esys{\stesysc}{\stesysf}{\stesyst}}
\newcommand{\stesysff}{\pullback{\stesysf}{\stesysf}{\ectxext}{\eft}}
\newcommand{\stesysft}{\pullback{\stesysf}{\stesysf}{\ectxext}{\eft\circ\utmb}\to \stesyst}
\newcommand{\esysc}[1]{\@ifnextchar\esys\esys@c{\@ifnextchar\stesys{\@eatdo{\stesysc}}{}} #1}
\newcommand{\esys@c}[4]{#2}
\newcommand{\esysf}[1]{\@ifnextchar\esys\esys@f{\@ifnextchar\stesys{\@eatdo{\stesysf}}{}} #1}
\newcommand{\esys@f}[4]{#3}
\newcommand{\esyst}[1]{\@ifnextchar\esys\esys@t{\@ifnextchar\stesys{\@eatdo{\stesyst}}{}} #1}
\newcommand{\esys@t}[4]{#4}
\newcommand{\famesys}[1]{\mathbf{F}_{#1}}
\newcommand{\cobesys}{\pullback} %Change of base

\newcommand{\ehom}{\mathbf}
\newcommand{\famehom}[1]{\mathbf{F}_{#1}}
\newcommand{\ehomc}[1]{\@gobble #1_0}
\newcommand{\ehomf}[1]{\@gobble #1_1}
\newcommand{\ehomt}[1]{\@gobble #1_2}
\newcommand{\eext}[1]{\mathrm{e}_{#1}}
\newcommand{\ectxext}{\eext{0}}
\newcommand{\efamext}{\eext{1}}

\makeatother


%%%%%%%%%%%%%%%%%%%%%%%%%%%%%%%%%%%%%%%%%%%%%%%%%%%%%%%%%%%%%%%%%%%%%%%%%%%%%%%%
%%%% THE CONSTRUCTORS OF THE TYPE THEORY OF MODELS

\makeatletter
%%%% The initial model
\newcommand{\mctx}{%
  \mathcal{C}}

%%%% The family constructor
\newcommand{\mfam}[2][]{%
  \mathcal{F}_{\default@ctxext #2}^{#1}}
\newcommand{\@mfam@nested}[1]{\@mfam@parens}
\newcommand{\@mfam@parens}[2][]{(\mfam[#1]{#2})}

%%%% The terms constructor
\newcommand{\mtm}[2][]{%
  \mathcal{T}_{\default@ctxext #2}^{#1}}
\newcommand{\@mtm@nested}[1]{\@mtm@parens}
\newcommand{\@mtm@parens}[2][]{(\mtm[#1]{#2})}

%%%% The empty type constructor
\newcommand{\tfemp}[1]{%
  \typefont{emp}_{\default@ctxext #1}}
\newcommand{\tft}[1]{%
  \typefont{t}_{\default@ctxext #1}}

%%%% The extension constructor
\newcommand{\tfext}[1]{%
  \typefont{ext}_{\default@ctxext #1}}

%%%% The substitution constructor
\newcommand{\tfsubst}[1]{%
  \typefont{subst}_{\default@ctxext #1}}
  
%%%% The weakening constructor
\newcommand{\tfwk}[1]{%
  \typefont{wk}_{\default@ctxext #1}}

%%%% The identity function constructor
\newcommand{\tfid}[1]{%
  \typefont{idtm}_{\default@ctxext #1}}
\makeatother

%%%%%%%%%%%%%%%%%%%%%%%%%%%%%%%%%%%%%%%%%%%%%%%%%%%%%%%%%%%%%%%%%%%%%%%%%%%%%%%%

%%%% Introducing logical usage of fonts.
\newcommand{\modelfont}{\mathit} % use 'mf' in command to indicate model font
\newcommand{\typefont}{\mathsf} % use 'tf' in command to indicate type font
\newcommand{\catfont}{\mathrm} % use 'cf' in command to indicate cat font

%%%%%%%%%%%%%%%%%%%%%%%%%%%%%%%%%%%%%%%%%%%%%%%%%%%%%%%%%%%%%%%%%%%%%%%%%%%%%%%%
%%%% Some macros of the book are redefined.

\renewcommand{\UU}{\typefont{U}}
\renewcommand{\isequiv}{\typefont{isEquiv}}
\renewcommand{\happly}{\typefont{hApply}}
\renewcommand{\pairr}[1]{{\mathopen{}\langle #1\rangle\mathclose{}}}
\renewcommand{\type}{\typefont{Type}}
\renewcommand{\op}[1]{{{#1}^\typefont{op}}}
\renewcommand{\susp}{\typefont{\Sigma}}

%%%%%%%%%%%%%%%%%%%%%%%%%%%%%%%%%%%%%%%%%%%%%%%%%%%%%%%%%%%%%%%%%%%%%%%%%%%%%%%%
%%%% The following is a big unorganized list of new macros that we use in the
%%%% notes. 

\newcommand{\tfW}{\typefont{W}}
\newcommand{\tfM}{\typefont{M}}
\newcommand{\mfM}{\modelfont{M}}
\newcommand{\mfN}{\modelfont{N}}
\newcommand{\tfctx}{\typefont{ctx}}
\newcommand{\mftypfunc}[1]{{\modelfont{typ}^{#1}}}
\newcommand{\mftyp}[2]{{\mftypfunc{#1}(#2)}}
\newcommand{\tftypfunc}[1]{{\typefont{typ}^{#1}}}
\newcommand{\tftyp}[2]{{\tftypfunc{#1}(#2)}}
\newcommand{\hfibfunc}[1]{\typefont{fib}_{#1}}
\newcommand{\mappingcone}[1]{\mathcal{C}_{#1}}
\newcommand{\equifib}{\typefont{equiFib}}
\newcommand{\tfcolim}{\typefont{colim}}
\newcommand{\tflim}{\typefont{lim}}
\newcommand{\tfdiag}{\typefont{diag}}
\newcommand{\tfGraph}{\typefont{Graph}}
\newcommand{\mfGraph}{\modelfont{Graph}}
\newcommand{\unitGraph}{\unit^\mfGraph}
\newcommand{\UUGraph}{\UU^\mfGraph}
\newcommand{\tfrGraph}{\typefont{rGraph}}
\newcommand{\mfrGraph}{\modelfont{rGraph}}
\newcommand{\isfunction}{\typefont{isFunction}}
\newcommand{\tfconst}{\typefont{const}}
\newcommand{\conemap}{\typefont{coneMap}}
\newcommand{\coconemap}{\typefont{coconeMap}}
\newcommand{\tflimits}{\typefont{limits}}
\newcommand{\tfcolimits}{\typefont{colimits}}
\newcommand{\islimiting}{\typefont{isLimiting}}
\newcommand{\iscolimiting}{\typefont{isColimiting}}
\newcommand{\islimit}{\typefont{isLimit}}
\newcommand{\iscolimit}{\typefont{iscolimit}}
\newcommand{\pbcone}{\typefont{cone_{pb}}}
\newcommand{\tfinj}{\typefont{inj}}
\newcommand{\tfsurj}{\typefont{surj}}
\newcommand{\tfepi}{\typefont{epi}}
\newcommand{\tftop}{\typefont{top}}
\newcommand{\sbrck}[1]{\Vert #1\Vert}
\newcommand{\strunc}[2]{\Vert #2\Vert_{#1}}
\newcommand{\gobjclass}{{\typefont{U}^\mfGraph}}
\newcommand{\gcharmap}{\typefont{fib}}
\newcommand{\diagclass}{\typefont{T}}
\newcommand{\opdiagclass}{\op{\diagclass}}
\newcommand{\equifibclass}{\diagclass^{\eqv{}{}}}
\newcommand{\universe}{\typefont{U}}
\newcommand{\catid}[1]{{\catfont{id}_{#1}}}
\newcommand{\isleftfib}{\typefont{isLeftFib}}
\newcommand{\isrightfib}{\typefont{isRightFib}}
\newcommand{\leftLiftings}{\typefont{leftLiftings}}
\newcommand{\rightLiftings}{\typefont{rightLiftings}}
\newcommand{\psh}{\typefont{Psh}}
\newcommand{\rgclass}{\typefont{\Omega^{RG}}}
\newcommand{\terms}[2][]{\lfloor #2 \rfloor^{#1}}
\newcommand{\grconstr}[2]
             {\mathchoice % max size is textstyle size.
             {{\textstyle \int_{#1}}#2}% 
             {\int_{#1}#2}%
             {\int_{#1}#2}%
             {\int_{#1}#2}}
\newcommand{\ctxhom}[3][]{\typefont{hom}_{#1}(#2,#3)}
\newcommand{\graphcharmapfunc}[1]{\gcharmap_{#1}}
\newcommand{\graphcharmap}[2][]{\graphcharmapfunc{#1}(#2)}
\newcommand{\tfexp}[1]{\typefont{exp}_{#1}}
\newcommand{\tffamfunc}{\typefont{fam}}
\newcommand{\tffam}[1]{\tffamfunc(#1)}
\newcommand{\tfev}{\typefont{ev}}
\newcommand{\tfcomp}{\typefont{comp}}
\newcommand{\isDec}[1]{\typefont{isDecidable}(#1)}
\newcommand{\smal}{\mathcal{S}}
\renewcommand{\modal}{{\ensuremath{\ocircle}}}
\newcommand{\eqrel}{\typefont{EqRel}}
\newcommand{\piw}{\ensuremath{\Pi\typefont{W}}} %% to be used in conjunction with -pretopos.
\renewcommand{\sslash}{/\!\!/}
\newcommand{\mprd}[3][]{\Pi^{#1}(#2,#3)}
\newcommand{\msm}[2]{\Sigma(#1,#2)}
\newcommand{\midt}[1]{\idvartype_#1}
\newcommand{\reflf}[1]{\typefont{refl}^{#1}}
\newcommand{\tfJ}{\typefont{J}}
\newcommand{\tftrans}{\typefont{trans}}

\newcommand{\tfT}{\typefont{T}}
\newcommand{\reflsym}{{\mathsf{refl}}}
\newcommand{\strans}[2]{\ensuremath{{#1}_{*}({#2})}}
\newcommand{\eqtype}[1]{\typefont{Eq}_{#1}}
\newcommand{\eqtoid}[1]{\typefont{eqtoid}(#1)}
\newcommand{\greek}{\mathrm}
\newcommand{\product}[2]{{#1}\times{#2}}
\newcommand{\pairp}[1]{(#1)}
\newcommand{\jequalizer}[3]{\{#1|#2\jdeq #3\}}
\newcommand{\jequalizerin}[2]{\iota_{#1,#2}}
\newcommand{\tounit}[1]{{!_{#1}}}
\newcommand{\trwk}{\typefont{trwk}}
\newcommand{\trext}{\typefont{trext}}
\newcommand{\tfindf}[2][]{\typefont{ind}_{#2}^{#1}}
\newcommand{\thom}[2]{\mathrm{thom}(#1,#2)}
\newcommand{\thomd}[3]{\mathrm{thom}_{#1}(#2,#3)}

\makeatletter
\newcommand{\tfind}[3][]{\tfindf[#1]{#2}(\default@ctxext #3)}
\makeatother
%%%%%%%%%%%%%%%%%%%%%%%%%%%%%%%%%%%%%%%%%%%%%%%%%%%%%%%%%%%%%%%%%%%%%%%%%%%%%%%%
%%%% When investigation pointed structures we use the \pt macro.

\makeatletter
\newcommand{\pt}[1][]{*_{
  \@ifnextchar\undergraph{\@undergraph@nested}
    {\@ifnextchar\underovergraph{\@underovergraph@nested}{}}#1}}
\makeatother

%%%%%%%%%%%%%%%%%%%%%%%%%%%%%%%%%%%%%%%%%%%%%%%%%%%%%%%%%%%%%%%%%%%%%%%%%%%%%%%%
%%%% OPERATIONS ON GRAPHS
%%%%
%%%% First of all, each graph has a type of vertices and a type of edges. The
%%%% type of vertices of a graph $\Gamma$ is denoted by $\pts{\Gamma}$;
%%%% and likewise for the type of edges.

\makeatletter
\newcommand{\pts}[1]{{\@graphop@nested{#1}}_{0}}
\newcommand{\edg}[1]{{\@graphop@nested{#1}}_{1}}
\newcommand{\@graphop@nested}[1]
  {\@ifnextchar\ctxext{\@ctxext@nested}
      {\@ifnextchar\undergraph{\@undergraph@nested}
         {\@ifnextchar\underovergraph{\@underovergraph@nested}{}}}
    #1}
\makeatother

%%%% The following operations of \undergraph and \underovergraph are used to
%%%% define the free category and the free groupoid of a graph, respectively

\makeatletter
\newcommand{\@undergraphtest}[2]{\@ifnextchar({#1}{#2}}
\newcommand{\undergraph}[2]{\@undergraphtest{\@undergraph@parens{#1}{#2}}{\@undergraph{#1}{#2}}}
\newcommand{\@undergraph}[2]{{#2/#1}}
\newcommand{\@undergraph@nested}[3]{\@undergraph@parens{#2}{#3}}
\newcommand{\@undergraph@parens}[2]{(\@undergraph{#1}{#2})}
\makeatother

\makeatletter
\newcommand{\underovergraph}[2]{\@underovergraphtest{\@underovergraph@parens{#1}{#2}}{\@underovergraph{#1}{#2}}}
\newcommand{\@underovergraph}[2]{{#2}\,{\parallel}\,{#1}}
\newcommand{\@underovergraphtest}{\@undergraphtest}
\newcommand{\@underovergraph@parens}[2]{(\@underovergraph{#1}{#2})}
\newcommand{\@underovergraph@nested}[3]{\@underovergraph@parens{#2}{#3}}
\makeatother

\newcommand{\graphid}[1]{\mathrm{id}_{#1}}
\newcommand{\freecat}[1]{\mathcal{C}(#1)}
\newcommand{\freegrpd}[1]{\mathcal{G}(#1)}


%%%%%%%%%%%%%%%%%%%%%%%%%%%%%%%%%%%%%%%%%%%%%%%%%%%%%%%%%%%%%%%%%%%%%%%%%%%%%%%%
%% Some tikz macros to typeset diagrams uniformly.

\tikzset{patharrow/.style={double,double equal sign distance,-,font=\scriptsize}}
\tikzset{description/.style={fill=white,inner sep=2pt}}
\tikzset{fib/.style={->>,font=\scriptsize}}

%% Used for extra wide diagrams, e.g. when the label is too large otherwise.
\tikzset{commutative diagrams/column sep/Huge/.initial=18ex}

%%%%%%%%%%%%%%%%%%%%%%%%%%%%%%%%%%%%%%%%%%%%%%%%%%%%%%%%%%%%%%%%%%%%%%%%%%%%%%%%
%%%% New environment for constructions.

%\expandafter\let\expandafter\oldproof\csname\string\proof\endcsname
%\let\oldendproof\endproof
\newenvironment{constr}{%
  \begin{proof}[Construction]%
}{\end{proof}}

%%%%%%%%%%%%%%%%%%%%%%%%%%%%%%%%%%%%%%%%%%%%%%%%%%%%%%%%%%%%%%%%%%%%%%%%%%%%%%%%
%%%% New theorem environment for conjectures.

\defthm{conj}{Conjecture}{Conjectures}

%%%%%%%%%%%%%%%%%%%%%%%%%%%%%%%%%%%%%%%%%%%%%%%%%%%%%%%%%%%%%%%%%%%%%%%%%%%%%%%%
%%%% The following environment for desiderata should not be there. It is better
%%%% to use the issue tracker for desiderata.

\newenvironment{desiderata}{\begingroup\color{blue}\textbf{Desiderata.}}
{\endgroup}

%%%%%%%%%%%%%%%%%%%%%%%%%%%%%%%%%%%%%%%%%%%%%%%%%%%%%%%%%%%%%%%%%%%%%%%%%%%%%%%%
%%%% The following piece of code from tex.stackexchange:
%%%%
%%%% http://tex.stackexchange.com/a/55180/14653
%%%%
%%%% We include it so that inference rules in align environments have enough
%%%% vertical space.

\newlength\minalignvsep

\makeatletter
\def\align@preamble{%
   &\hfil
    \setboxz@h{\@lign$\m@th\displaystyle{##}$}%
    \ifnum\row@>\@ne
    \ifdim\ht\z@>\ht\strutbox@
    \dimen@\ht\z@
    \advance\dimen@\minalignvsep
    \ht\strutbox\dimen@
    \fi\fi
    \strut@
    \ifmeasuring@\savefieldlength@\fi
    \set@field
    \tabskip\z@skip
   &\setboxz@h{\@lign$\m@th\displaystyle{{}##}$}%
    \ifnum\row@>\@ne
    \ifdim\ht\z@>\ht\strutbox@
    \dimen@\ht\z@
    \advance\dimen@\minalignvsep
    \ht\strutbox@\dimen@
    \fi\fi
    \strut@
    \ifmeasuring@\savefieldlength@\fi
    \set@field
    \hfil
    \tabskip\alignsep@
}
\makeatother

\minalignvsep.2em

\allowdisplaybreaks

%%%%%%%%%%%%%%%%%%%%%%%%%%%%%%%%%%%%%%%%%%%%%%%%%%%%%%%%%%%%%%%%%%%%%%%%%%%%%%%%

\setdescription[1]{itemsep=-0.2em}


%\includeonly{Algebras/article-models-cat}

%%%%%%%%%%%%%%%%%%%%%%%%%%%%%%%%%%%%%%%%%%%%%%%%%%%%%%%%%%%%%%%%%%%%%%%%%%%%%%%%
\title{An essentially algebraic formulation of dependent type theory}
\author{Egbert Rijke}
\date{\today}

\begin{document}

\maketitle

%\tableofcontents

%\section{Introduction}

As of yet, there is no theory of internal models of type theory of sufficient
generality. The problem of internalizing type theory has two facets. On the one
hand one may choose to model the definitional equalities of type theory by
identifications. While this approach would have the benefint of being completely
internal (allowing one to use the internal logic of type theory to prove things
about internal models), complications in this approach do arise when the 
underlying types are not sets. This immediately leaves univalent universes beyond
the scope of that program. On the other hand, one may want to interpret the
judgmental equalities of type theory with new judgmental equalities. The problem
here is that the theory of judgmental equality in many variations of dependent
type theory is not developed systematically, leaving anyone essentially in the
dark on this matter.


%\part{Type theory}

\begin{comment}
In this part we develop type theory from the ground up. We start with a type
theory without any of the basic constructors. This is the theory of contexts
families and terms which has the basic operations of extension, weakening,
substitution and identity terms. Type theory before type constructors has not
been studied very much. Dependent product types or even universes tend to make
an early appearance in just about any presentation of type theory.
A noteable exception is the theory of categories with families of Dybjer
in \cite{Dybjer1996}, which has been elaborated on further by Dybjer and
Clairambault in \cite{DybjerClairambault2011} and in unpublished work by
Awodey \cite{Awodey2013} on natural models of type theory,
which makes the connection between categories with families and representable
transformations of presheaves. In the way type theory is presented by the
Univalent Foundations Project in \cite{TheBook}, which seems to have won the race of introducing universes
as early as possible hands down, it seems entirely unfeasible to study type
theory without type constructors. Also, the 
proof-assistant \Coq\ {\color{red}(and \Agda too?)} has universes and dependent product and pair types 
built-in, making it impossible to study type theory without type constructors in 
that environment.

Nevertheless, type theory without type constructors has received some attention
contributors to the Univalent Foundations Program recently. We name two
further investigations on this topic, other than the mentioned work by Awodey.
In \cite{Garner2014}, Garner describes the combinatorial structure
of the type operations of the weakening, substitution and projection monads
(their projections are our identity terms) and suggests lots of further research
that can be done on type theory without constructors. Also, Joyal has presented
his theory of tribes, which is a categorical explanation of type theory
without type constructors.

After we have described the E-system, which is the flavor of type theory without
type constructors presented in \autoref{tt}, we will demonstrate in \autoref{ttderived}
that the theory gives rise to a rich categorical structure. The notions and
properties we derive here will be essential for the further work on internal
models for E-systems, presented in \autoref{part:models}.
\end{comment}

\section{The fundamental structure of dependent type theory}

\subsection{Judgments and valid inference rules}
\label{judgments}

The theory we describe here is a theory of contexts, families of
contexts and terms thereof. The families of contexts are by some authors called
dependent contexts, but they are handled a bit differently here because they
become the primary object of study. Dependent contexts can be types; they could
be seen as atomic or indecomposable dependent contexts.

Thus we make eight kinds of judgments: ``$\Gamma$ is a context'',
``$A$ is a family of contexts over $\Gamma$'', ``$A$ is a type in context $\Gamma$''
and ``$x$ is a term of the family $A$ of contexts over $\Gamma$''. The other four
judgments are for judgmental equality.
\begin{align*}
\jalign\jctx{\Gamma} 
& \jalign\jctxeq{\Gamma}{\Gamma'}
  \\
\jalign\jfam{\Gamma}{A} 
& \jalign\jfameq{\Gamma}{A}{B}
  \\
\jalign\jterm{\Gamma}{A}{x} 
& \jalign\jtermeq{\Gamma}{A}{x}{y}.
\end{align*}

Strictly speaking, we have three different judgmental equalities in play and one
could request for a notational difference to signify that fact. For instance,
we could denote the judgmental equalities of contexts, families and terms by
$\jdeq_c$, $\jdeq_f$ and $\jdeq_t$ respectively. It will, however, always be
clear which of the three kinds of judgmental equality is meant when we assert
a judgmental equality and therefore we shall not bother to make this notational
distinction.

We note that what we call families over contexts
here could also have been named dependent contexts or telescopes, see
\cite{deBruijn1991,hofmann1995extensional}. The term family is in agreement
with the terminology scheme of \cite{TheBook}, though the reader should be
warned that the notion of familie means something slightly different there than
it does here.

The theory of contexts, families and terms shall be described by means of a
declaration of valid inference rules. An inference rule is an expression of the
form
\begin{equation*}
\inference{\mathcal{H}_1\quad\cdots\quad\mathcal{H}_n}{\mathcal{J}}
\end{equation*}
where each of $\mathcal{H}_1$, ..., $\mathcal{H}_n$ and $\mathcal{J}$ is 
judgment. The judgments $\mathcal{H}_1$, ..., $\mathcal{H}_n$ are called the
hypotheses and the judgment $\mathcal{J}$ is the conclusion of the inference
rule. When a judgment is the conclusion of a valid inference rule, we also say
that the judgment is valid or well-formed. In fact, the only way to construct
new judgments is via the valid inference rules we describe in this section.

The class of valid inference rules is closed under composition and under
rearranging the hypotheses. That
means that if
\begin{equation*}
\inference{\mathcal{I}_{1k}\quad\cdots\quad\mathcal{I}_{m_kk}}{\mathcal{H}_k}
\end{equation*}
and
\begin{equation*}
\inference{\mathcal{H}_1\quad\cdots\quad\mathcal{H}_n}{\mathcal{J}}
\end{equation*}
are valid inference rules for $1\leq k\leq n$, then
\begin{equation*}
\inference{\mathcal{I}_{11}\quad\cdots\quad\mathcal{I}_{m_11}\quad\cdots\quad\mathcal{I}_{1n}\quad\cdots\quad\mathcal{I}_{m_nn}}{\mathcal{J}}
\end{equation*}
is also a valid inference rule. Also, in a valid inference rule 
duplicated hypotheses may always be contracted and hypotheses may always be 
reordered; the result is still considered a valid inference rule.

The rules for judgmental equality establish that it is an equivalence relation
in all three cases (contexts, families and terms). Thus, the following inference
rules shall be required to be valid:
\bgroup\small
\begin{align*}
& \inference
  { \jctx{\Gamma}
    }
  { \jctxeq{\Gamma}{\Gamma}
    } 
& & \inference
    { \jctxeq{\Gamma}{\Delta}
      }
    { \jctxeq{\Delta}{\Gamma}
      } 
& & \inference
    { \jctxeq{\Gamma}{\Delta}
      \jctxeq{\Delta}{\greek{E}}
      }
    { \jctxeq{\Gamma}{\greek{E}}
      }
    \\
& \inference
  { \jfam{\Gamma}{A}
    }
  { \jfameq{\Gamma}{A}{A}
    } 
& & \inference
    { \jfameq{\Gamma}{A}{B}
      }
    { \jfameq{\Gamma}{B}{A}
      }
& & \inference
    { \jfameq{\Gamma}{A}{B}
      \jfameq{\Gamma}{B}{C}
      }
    { \jfameq{\Gamma}{A}{C}
      }
    \\
& \inference
  { \jterm{\Gamma}{A}{x}
    }
  { \jtermeq{\Gamma}{A}{x}{x}
    }
& & \inference
    { \jtermeq{\Gamma}{A}{x}{y}
      }
    { \jtermeq{\Gamma}{A}{y}{x}
      }
& & \inference
    { \jtermeq{\Gamma}{A}{x}{y}
      \jtermeq{\Gamma}{A}{y}{z}
      }
    { \jtermeq{\Gamma}{A}{x}{z}
      }
\end{align*}
\egroup

The following convertibility rules are responsible for the strictness
of judgmental equality, which sets it apart from equivalences or identifications:
\begin{align*}
& \inference
  { \jctxeq{\Gamma}{\Delta}
    \jfam{\Gamma}{A}
    }
  { \jfam{\Delta}{A}
    }
& & \inference
    { \jctxeq{\Gamma}{\Delta}
      \jfameq{\Gamma}{A}{B}
      }
    { \jfameq{\Delta}{A}{B}
      }
    \\
& \inference
  { \jctxeq{\Gamma}{\Delta}
    \jterm{\Gamma}{A}{x}
    }
  { \jterm{\Delta}{A}{x}
    }
& & \inference
    { \jctxeq{\Gamma}{\Delta}
      \jtermeq{\Gamma}{A}{x}{y}
      }
    { \jtermeq{\Delta}{A}{x}{y}
      }
    \\
& \inference
  { \jfameq{\Gamma}{A}{B}
    \jterm{\Gamma}{A}{x}
    }
  { \jterm{\Gamma}{B}{x}
    }
& & \inference
    { \jfameq{\Gamma}{A}{B}
      \jtermeq{\Gamma}{A}{x}{y}
      }
    { \jtermeq{\Gamma}{B}{x}{y}
      }
\end{align*}


\subsection{Semantics of the fundamental structure of dependent type theory}
For the semantics of the theory of E-systems, we will assume that $\cat{C}$ is a 
category with finite limits and
whenever we write a pullback, we assume that it is chosen. Recall that for
any morphism $f:A\to B$ in a category $\cat{C}$ with chosen pullbacks, there
is a functor
\begin{equation*}
f^\ast : \cat{C}/B\to\cat{C}/A.
\end{equation*}
As usual, when $g:X\to B$ is a morphism, we will write $f^\ast(X)$ for the
domain of $f^\ast(g)$. When there is more than one morphism $X\to B$ involved,
as will be the case below, we will write $\pullback{A}{X}{f}{g}$. The projections
will be written as $\pullbackpr{1}{f}{g}$ and $\pullbackpr{2}{f}{g}$. So in this notation, a
typical pullback diagram has the following form:
\begin{equation*}
\begin{tikzcd}[column sep=large]
\pullback{A}{X}{f}{g}
  \ar{r}{\pullbackpr{1}{f}{g}}
  \ar{d}[swap]{\pullbackpr{2}{f}{g}}
  &
A \ar{d}{f}
  \\
X \ar{r}[swap]{g}
  &
B
\end{tikzcd}
\end{equation*}
Also, when we have a commutative diagram of the form
\begin{equation*}
\begin{tikzcd}
A \ar{r}{f}
  \ar{d}{a}
  &
X \ar{d}
  & 
B \ar{l}[swap]{g}
  \ar{d}{b}
  \\
A'
  \ar{r}[swap]{f'}
  &
X'
  &
B'
  \ar{l}{g'}
\end{tikzcd}
\end{equation*}
we will denote the unique map from $\pullback{A}{B}{f}{g}$ to $\pullback{A'}{B'}{f'}{g'}$
such that the diagram
\begin{equation*}
\begin{tikzcd}
  {}
  & 
\pullback{A'}{B'}{f'}{g'}
  \ar{dd}
  \ar{rr}
  &
  &
B'
  \ar{dd}{g'}
  \\
\pullback{A}{B}{f}{g}
  \ar{dd}
  \ar[crossing over]{rr}
  \ar[dotted]{ur}{\pullback{a}{b}{f'}{g'}}
  &
  &
B \ar{ur}{b}
  \\
  {}
  &
A'
  \ar{rr}
  &
  &
X'
  \\
A \ar{rr}[swap]{f}
  \ar{ur}{a}
  &
  &
X \ar[crossing over,leftarrow]{uu}[near end,swap]{g}
  \ar{ur}
\end{tikzcd}
\end{equation*}
commutes, by $\pullback{a}{b}{f'}{g'}$. In the current work, we shall
write $A\times B$ for the pullback of $A\rightarrow 1\leftarrow B$, and
$\pi_1$ and $\pi_2$ for its projections (thus, no separate choice of
cartesian products is made).

\begin{defn}
A \emph{fundamental structure} $\stesys$ in $\cat{C}$ consists of a diagram of the form
\begin{equation*}
\begin{tikzcd}
\stesyst
  \ar{d}[swap]{\ebd}
  \\
\stesysf
  \ar{d}[swap]{\eft}
  \\
\stesysc
\end{tikzcd}
\end{equation*}
in $\cat{C}$. In this diagram, $C$ is the object of \emph{contexts}, $F$ is
the object of \emph{families}, and $T$ is the object of \emph{terms}.
\end{defn}



\subsection{The semantics of extension}
\begin{defn}
A \emph{pre-extension algebra $\stesys$ in $\cat{C}$} consists of a fundamental structure
$\stesys$
in $\cat{C}$ together with the \emph{context extension} and \emph{family extension} operations
\begin{align*}
\ectxext &:\stesysf\to \stesysc\\
\efamext & :\stesysff\to \stesysf,
\end{align*}
respectively, such that the diagram
\begin{equation*}
\begin{tikzcd}
\stesysf_2 
  \ar{r}{\efamext} 
  \ar{d}[swap]{\eft[2]} 
  & 
\stesysf 
  \ar{d}{\eft}
  \\
\stesysf
  \ar{r}[swap]{\eft} 
  & 
\stesysc
\end{tikzcd}
\end{equation*}
commutes.
\end{defn}

\begin{defn}
We introduce the following notation:
\begin{align*}
\stesysf_2 
  & := \stesysff
  \\
\eft[2] 
  & := \pullbackpr{1}{\ectxext}{\eft} : \stesysf_2\to\stesysf
  \\
\stesysf_3 & := \pullback{\stesysf_2}{\stesysf_2}{\efamext}{\eft[2]}
  \\
\eft[3]
  & := \pullbackpr{1}{\efamext}{\eft[2]} : \stesysf_3\to\stesysf_2.
\end{align*}
Then it follows that the outer square in the diagram
\begin{equation*}
\begin{tikzcd}[column sep=large]
\stesysf_3
  \ar[dotted]{dr}{\eext{2}}
  \ar{rr}{\pullback{\pullbackpr{2}{\ectxext}{\eft}}{\pullbackpr{2}{\ectxext}{\eft}}{\ectxext}{\eft}}
  \ar{dd}[swap]{\eft[3]}
  & 
  &
\stesysf_2
  \ar{d}{\efamext}
  \\
  &
\stesysf_2
  \ar{d}[swap]{\eft[2]}
  \ar{r}{\pullbackpr{2}{\ectxext}{\eft}}
  &
\stesysf
  \ar{d}{\eft}
  \\
\stesysf_2
  \ar{r}[swap]{\eft[2]}
  &
\stesysf
  \ar{r}[swap]{\ectxext}
  &
\stesysc
\end{tikzcd}
\end{equation*}
commutes. We define $\eext{2}$ to be the unique morphism rendering the above diagram
commutative. Now we may continue to define
\begin{align*}
\stesysf_4 
  & := 
\pullback{\stesysf_3}{\stesysf_3}{\eext{2}}{\eft[3]}
  \\
\eft[4] 
  & := 
\pullbackpr{1}{\eext{2}}{\eft[3]}.
\end{align*}
Then we see that the outer square of the diagram
\begin{equation*}
\begin{tikzcd}[column sep=large]
\stesysf_4
  \ar[dotted]{dr}{\eext{3}}
  \ar{rr}{\pullback{\pullbackpr{2}{\efamext}{\eft[2]}}{\pullbackpr{2}{\efamext}{\eft[2]}}{\efamext}{\eft[2]}}
  \ar{dd}[swap]{\eft[4]}
  & 
  &
\stesysf_3
  \ar{d}{\eext{2}}
  \\
  &
\stesysf_3
  \ar{d}[swap]{\eft[3]}
  \ar{r}{\pullbackpr{2}{\efamext}{\eft[2]}}
  &
\stesysf_2
  \ar{d}{\eft[2]}
  \\
\stesysf_3
  \ar{r}[swap]{\eft[3]}
  &
\stesysf_2
  \ar{r}[swap]{\efamext}
  &
\stesysf
\end{tikzcd}
\end{equation*}
commutes,
so we may define $\eext{3}$ to be the unique map which renders the diagram
commutative. It
is straightforward to continue this process by induction, but we shall need not
go any further in this article.
\end{defn}

\begin{defn} 
An extension algebra is a pre-extension algebra $\stesys$ for which 
the diagrams
\begin{equation*}
\begin{tikzcd}
\stesysf_2 
  \ar{d}[swap]{\pullbackpr{2}{\ectxext}{\eft}} 
  \ar{r}{\efamext} 
  & 
\stesysf 
  \ar{d}{\ectxext}
  \\
\stesysf 
  \ar{r}[swap]{\ectxext} 
  & 
\stesysc
\end{tikzcd}
\qquad
\begin{tikzcd}
\stesysf_3
  \ar{d}[swap]{\pullbackpr{2}{\efamext}{\eft[2]}}
  \ar{r}{\eext{2}}
  & 
\stesysf_2 
  \ar{d}{\efamext} 
  \\
\stesysf_2 
  \ar{r}[swap]{\efamext} 
  &
\stesysf
\end{tikzcd}
\end{equation*}
commute.
\end{defn}

\begin{comment}
\begin{lem}
There exists an isomorphism $\alpha$ such that the triangle
\begin{equation*}
\begin{tikzcd}[column sep=tiny]
\pullback{\stesysf}{\stesysf_2}{\ectxext}{\eft\circ\eft[2]}
  \ar[dotted]{rr}{\alpha}
  \ar{dr}[swap]{\pullback{\catid{\stesysf}}{\efamext}{\ectxext}{\eft}}
  &
  &
\stesysf_3
  \ar{dl}{\eext{2}}
  \\
& \stesysf_2
\end{tikzcd}
\end{equation*}
commutes
\end{lem}

\begin{proof}
There is a unique morphism $\alpha:
\pullback{\stesysf}{\stesysf_2}{\ectxext}{\eft\circ\eft[2]}\to\stesysf_3$
rendering the diagram
\begin{equation*}
\begin{tikzcd}[column sep=large]
\pullback{\stesysf}{\stesysf_2}{\ectxext}{\eft\circ\eft[2]}
  \ar[bend left=10,yshift=.5ex]{drrr}{\pullbackpr{2}{\ectxext}{\eft}\circ\pullbackpr{2}{\ectxext}{\eft\circ\eft[2]}}
  \ar[bend right=10]{ddr}[swap]{\pullback{\catid{\stesysf}}{\eft[2]}{\ectxext}{\eft}}
  \ar[dotted]{dr}{\alpha}
  \\
& \stesysf_3
  \ar{r}{\pullbackpr{2}{\efamext}{\eft[2]}}
  \ar{d}{\eft[3]}
  &
\stesysf_2
  \ar{d}[swap]{\eft[2]}
  \ar{r}[swap]{\pullbackpr{2}{\ectxext}{\eft}}
  &
\stesysf
  \ar{d}{\eft}
  \\
{} & \stesysf_2
  \ar{r}[swap]{\efamext}
  &
\stesysf
  \ar{r}[swap]{\ectxext}
  &
\stesysc
\end{tikzcd}
\end{equation*}
\end{proof}
\end{comment}

\begin{defn}
Suppose $\stesys$ is a pre-extension algebra of $\cat{C}$. Then we define the pre-extension algebra
$\famesys{\stesys}$ to consist of the fundamental structure
\begin{equation*}
\begin{tikzcd}
\stesyst_2
  \ar{d}{\ebd[2]}
  \\
\stesysf_2
  \ar{d}{\eft[2]}
  \\
\stesysf
\end{tikzcd}
\end{equation*}
where
\begin{align*}
\stesyst_2 
  & := \pullback{\stesysf}{\stesyst}{\ectxext}{\eft\circ\ebd}
  \\
\ebd[2]
  & := \ectxext^\ast(\ebd),
\end{align*}
with the extension operations
\begin{align*}
\efamext 
  & 
  : \stesysf_2\to\stesysf\\
\eext{2} & : \stesysf_3\to\stesysf_2.
\end{align*}
\end{defn}

In \autoref{famextobj} below we shall show that $\famesys{\stesys}$ is an
extension algebra whenever $\stesys$ is an extension algebra. We shall need
a handful of lemmas to give the proof.

\begin{defn}
We define
\begin{align*}
\beta_1 
  & := 
\pullbackpr{2}{\ectxext}{\eft}
  & &
  : \stesysf_2\to\stesysf
  \\
\beta_2
  & :=
\pullback{\beta_1}{\beta_1}{\ectxext}{\eft}
  & &
  : \stesysf_3\to\stesysf_2
  \\
\beta_3
  & :=
\pullback{\beta_2}{\beta_2}{\efamext}{\eft[2]}
  & &
  : \stesysf_4\to\stesysf_3.
\end{align*}
\end{defn}

\begin{lem}
Let $\stesys$ be a pre-extension algebra. Then the square
\begin{equation*}
\begin{tikzcd}[column sep=10em]
\stesysf_4
  \ar{r}{\pullback{\pullbackpr{2}{\efamext}{\eft[2]}}{\pullbackpr{2}{\efamext}{\eft[2]}}{\efamext}{\eft[2]}}
  \ar{d}[swap]{\beta_3}
  &
\stesysf_3
  \ar{d}{\beta_2}
  \\
\stesysf_3
  \ar{r}[swap]{\beta_2}
  &
\stesysf_2
\end{tikzcd}
\end{equation*}
commutes.
\end{lem}

\begin{proof}
Left to the reader.
\end{proof}

\begin{comment}
\begin{proof}
It is straightforward to verify the equalities
\begin{align*}
\pullbackpr{1}{\ectxext}{\eft}\circ\beta_2\circ
  (\pullback{\pullbackpr{2}{\efamext}{\eft[2]}}{\pullbackpr{2}{\efamext}{\eft[2]}}{\efamext}{\eft[2]})
  & =
\pullbackpr{1}{\ectxext}{\eft}\circ\beta_2\circ\beta_3
  \\
\pullbackpr{2}{\ectxext}{\eft}\circ\beta_2\circ
  (\pullback{\pullbackpr{2}{\efamext}{\eft[2]}}{\pullbackpr{2}{\efamext}{\eft[2]}}{\efamext}{\eft[2]})
  & =
\pullbackpr{2}{\ectxext}{\eft}\circ\beta_2\circ\beta_3.\qedhere
\end{align*}
\end{proof}
\end{comment}

Note that the square
\begin{equation*}
\begin{tikzcd}
\stesysf_3
  \ar{r}{\beta_2}
  \ar{d}[swap]{\eext{2}}
  &
\stesysf_2
  \ar{d}{\efamext}
  \\
\stesysf_2
  \ar{r}[swap]{\beta_1}
  &
\stesysf
\end{tikzcd}
\end{equation*}
commutes by definition. We have a similar result relating $\eext{3}$ and
$\eext{2}$.

\begin{lem}
Let $\stesys$ be a pre-extension algebra. Then the square
\begin{equation*}
\begin{tikzcd}
\stesysf_4
  \ar{r}{\beta_3}
  \ar{d}[swap]{\eext{3}}
  &
\stesysf_3
  \ar{d}{\eext{2}}
  \\
\stesysf_3
  \ar{r}[swap]{\beta_2}
  &
\stesysf_2
\end{tikzcd}
\end{equation*}
commutes.
\end{lem}

\begin{proof}
Left to the reader.
\end{proof}
\begin{comment}
\begin{proof}
It is straightforward to verify the equalities
\begin{align*}
\pullbackpr{1}{\ectxext}{\eft}\circ\beta_2\circ\eext{3}
  & = \beta_1\circ\eft[3]\circ\eft[4]
  \\
\pullbackpr{1}{\ectxext}{\eft}\circ\eext{2}\circ\beta_3
  & = \beta_1\circ\eft[3]\circ\eft[4].
\end{align*}
Thus, it remains to verify that
\begin{equation*}
\pullbackpr{2}{\ectxext}{\eft}\circ\beta_2\circ\eext{3}
  = \pullbackpr{2}{\ectxext}{\eft}\circ\eext{2}\circ\beta_3.
\end{equation*}
It is straightforward to see that the diagram
\begin{equation*}
\begin{tikzcd}[column sep=large]
\stesysf_4
  \ar{dd}[swap]{\pullback{\pullbackpr{2}{\efamext}{\eft[2]}}{\pullbackpr{2}{\efamext}{\eft[2]}}{\efamext}{\eft[2]}}
  \ar{r}{\eext{3}}
  &
\stesysf_3
  \ar{r}{\beta_2}
  \ar{d}[swap]{\pullbackpr{2}{\efamext}{\eft[2]}}
  &
\stesysf_2
  \ar{dd}{\pullbackpr{2}{\ectxext}{\eft}}
  \\
  {} &
\stesysf_2
  \ar{dr}{\beta_1}
  \\
\stesysf_3
  \ar{ur}{\eext{2}}
  \ar{r}[swap]{\beta_2}
  &
\stesysf_2
  \ar{r}[swap]{\efamext}
  &
\stesysf
\end{tikzcd}
\end{equation*}
commutes. It is likewise straightforward to see that the diagram
\begin{equation*}
\begin{tikzcd}
\stesysf_4
  \ar{r}{\beta_3}
  \ar{d}[swap]{\pullback{\pullbackpr{2}{\efamext}{\eft[2]}}{\pullbackpr{2}{\efamext}{\eft[2]}}{\efamext}{\eft[2]}}
  &
\stesysf_3
  \ar{r}{\eext{2}}
  \ar{d}[swap]{\beta_2}
  &
\stesysf_2
  \ar{d}{\pullbackpr{2}{\ectxext}{\eft}}
  \\
\stesysf_3
  \ar{r}[swap]{\beta_2}
  &
\stesysf_2
  \ar{r}[swap]{\efamext}
  &
\stesysf
\end{tikzcd}
\end{equation*}
commutes, completing our goal.
\end{proof}
\end{comment}

\begin{thm}[Local extension structure]\label{famextobj}
If $\stesys$ is an extension algebra, then so is $\famesys{\stesys}$.
\end{thm}

\begin{proof}
Note that the diagram
\begin{equation*}
\begin{tikzcd}
\stesysf_3
  \ar{d}[swap]{\pullbackpr{2}{\efamext}{\eft[2]}}
  \ar{r}{\eext{2}}
  & 
\stesysf_2 
  \ar{d}{\efamext} 
  \\
\stesysf_2 
  \ar{r}[swap]{\efamext} 
  &
\stesysf
\end{tikzcd}
\end{equation*}
commutes by assumption. For the second condition, we have to show that the
diagram
\begin{equation*}
\begin{tikzcd}
\stesysf_4
  \ar{d}[swap]{\pullbackpr{2}{\eext{2}}{\eft[3]}}
  \ar{r}{\eext{3}}
  & 
\stesysf_3
  \ar{d}{\eext{2}} 
  \\
\stesysf_3
  \ar{r}[swap]{\eext{2}} 
  &
\stesysf_2
\end{tikzcd}
\end{equation*}
Since this is a question about two maps into a pullback, it suffices to verify
that
\begin{align*}
\pullbackpr{1}{\ectxext}{\eft}\circ\eext{2}\circ\eext{3}
  & =
\pullbackpr{1}{\ectxext}{\eft}\circ\eext{2}\circ\pullbackpr{2}{\eext{2}}{\eft[3]}
  \\
\pullbackpr{2}{\ectxext}{\eft}\circ\eext{2}\circ\eext{3}
  & =
\pullbackpr{2}{\ectxext}{\eft}\circ\eext{2}\circ\pullbackpr{2}{\eext{2}}{\eft[3]}.
\end{align*}
For the first equality, it is fairly straightforward to show that both the
equalities
\begin{equation*}
\pullbackpr{1}{\ectxext}{\eft}\circ\eext{2}\circ\eext{3}
  =
\eft[2]\circ\eft[3]\circ\eft[4]
\end{equation*}
and
\begin{equation*}
\pullbackpr{1}{\ectxext}{\eft}\circ\eext{2}\circ\pullbackpr{2}{\eext{2}}{\eft[3]}
  =
\eft[2]\circ\eft[3]\circ\eft[4].
\end{equation*}
hold. For the second subgoal (which is more tricky). Notice first that the
diagram
\begin{equation*}
\begin{tikzcd}
\stesysf_4
  \ar{r}{\eext{3}}
  \ar{d}[swap]{\beta_3}
  &
\stesysf_3
  \ar{r}{\eext{2}}
  \ar{d}[swap]{\beta_2}
  &
\stesysf_2
  \ar{d}{\pullbackpr{2}{\ectxext}{\eft}}
  \\
\stesysf_3
  \ar{r}[swap]{\eext{2}}
  &
\stesysf_2
  \ar{r}[swap]{\efamext}
  &
\stesysf
\end{tikzcd}
\end{equation*}
commutes. We also have the commutative diagram
\begin{equation*}
\begin{tikzcd}[column sep=large]
\stesysf_4
  \ar{r}{\pullbackpr{2}{\eext{2}}{\eft[3]}}
  \ar{d}[swap]{\beta_3}
  &
\stesysf_3
  \ar{r}{\eext{2}}
  \ar{d}[swap]{\beta_2}
  &
\stesysf_2
  \ar{d}{\pullbackpr{2}{\ectxext}{\eft}}
  \\
\stesysf_3
  \ar{r}{\pullbackpr{2}{\efamext}{\eft[2]}}
  \ar{dr}[swap]{\eext{2}}
  &
\stesysf_2
  \ar{r}{\efamext}
  &
\stesysf
  \\
  {} &
\stesysf_2
  \ar{ur}[swap]{\efamext}
\end{tikzcd}
\end{equation*}
completing the proof.
\end{proof}

\subsection{(Pre-)extension homomorphisms}\label{subsection:e_extension_homomorphisms}
In this subsection we start with the study of pre-extension homomorphisms, which
will include the extension homomorphisms since they will be the pre-extension
homomorphisms of which both the domain and codomain are extension algebras.
Our main examples of extension homomorphisms will be the operations of weakening
and substitution. There are some basic examples of pre-extension homomorphisms
that will be useful too, which get introduced in the this section and in
\autoref{subsection:change_of_base}. In this section, we will mainly be
interested in pre-extension homomorphisms between local pre-extension algebras.
We will end this section by proving that a retract of an extension algebra is
always an extension algebra.

\begin{defn}
Let $\stesys$ and $\stesys'$ be pre-extension algebras. A \emph{pre-extension 
homomorphism $f$ from $\stesys'$ to $\stesys$} is a triple $(f_0,f_1,f^t)$ 
consisting of morphisms
\begin{equation*}
\begin{tikzcd}
\stesyst' 
  \ar{r}{f^t}
  \ar{d}[swap]{\ebd'}
  &
\stesyst
  \ar{d}{\ebd}
  \\
\stesysf'
  \ar{r}{f_1}
  \ar{d}[swap]{\eft'}
  &
\stesysf
  \ar{d}{\eft}
  \\
\stesysc' 
  \ar{r}[swap]{f_0}
  &
\stesysc
\end{tikzcd}
\end{equation*}
such that the indicated squares commute, for which furthermore the squares
\begin{equation*}
\begin{tikzcd}
\stesysf' 
  \ar{r}{f_1}
  \ar{d}[swap]{\ectxext'}
  &
\stesysf
  \ar{d}{\ectxext}
  \\
\stesysc'
  \ar{r}[swap]{f_0}
  &
\stesysc
\end{tikzcd}
\end{equation*}
and
\begin{equation*}
\begin{tikzcd}[column sep=huge]
\stesysf'\times_{\ectxext',\eft'} \stesysf'
  \ar{r}{f_1\times_{\ectxext,\eft} f_1}
  \ar{d}[swap]{\efamext'}
  &
\stesysf\times_{\ectxext,\eft} \stesysf
  \ar{d}{\efamext}
  \\
\stesysf'
  \ar{r}[swap]{f_1}
  &
\stesysf
\end{tikzcd}
\end{equation*}
Composition and the identity homomorphism are defined in the expected way. We
define furthermore
\begin{align*}
f_2 & := \pullback{f_1}{f_1}{\ectxext}{\eft}
  \\
f_3 & := \pullback{f_2}{f_2}{\efamext}{\eft[2]}.
\end{align*}
\end{defn}

\begin{defn}
A pre-extension homomorphism between extension algebras is called an extension
homomorphism.
\end{defn}

\begin{lem}
Let $\stesys$ be an extension algebra. Then
\begin{equation*}
\begin{tikzcd}[column sep=huge]
\stesyst_2
  \ar{r}{\pullbackpr{2}{\ectxext}{\eft\circ\ebd}}
  \ar{d}[swap]{\ebd[2]}
  &
\stesyst
  \ar{d}{\ebd}
  \\
\stesysf_2
  \ar{r}{\pullbackpr{2}{\ectxext}{\eft}}
  \ar{d}[swap]{\eft[2]}
  &
\stesysf
  \ar{d}{\eft}
  \\
\stesysf
  \ar{r}[swap]{\ectxext}
  &
\stesysc
\end{tikzcd}
\end{equation*}
assembles an extension homomorphism $\mathbf{e}_0:\famesys{\stesys}\to\stesys$.
Likewise, we have an extension homomorphism
$\mathbf{e}_1:\famesys{\famesys{\stesys}}\to\famesys{\stesys}$. Thus, a
pre-extension algebra is an extension algebra if and only if $\mathbf{e}_0$
and $\mathbf{e}_1$ are pre-extension homomorphisms.
\end{lem}

\begin{proof}
Immediate from the conditions of being an extension algebra.
\end{proof}

\begin{defn}
Let $\stesys$ be a pre-extension algebra. Then
\begin{equation*}
\begin{tikzcd}
\stesyst_3
  \ar{r}{\beta^t}
  \ar{d}[swap]{\ebd[3]}
  &
\stesyst_2
  \ar{d}{\ebd[2]}
  \\
\stesysf_3
  \ar{r}{\beta_2}
  \ar{d}[swap]{\eft[3]}
  &
\stesysf_2
  \ar{d}{\eft[2]}
  \\
\stesysf_2
  \ar{r}[swap]{\beta_1}
  &
\stesysf
\end{tikzcd}
\qquad
\text{and}
\qquad
\begin{tikzcd}
\stesyst_4
  \ar{r}{\beta^t_2}
  \ar{d}[swap]{\ebd[4]}
  &
\stesyst_3
  \ar{d}{\ebd[3]}
  \\
\stesysf_4
  \ar{r}{\beta_3}
  \ar{d}[swap]{\eft[4]}
  &
\stesysf_3
  \ar{d}{\eft[3]}
  \\
\stesysf_3
  \ar{r}[swap]{\beta_2}
  &
\stesysf_2
\end{tikzcd}
\end{equation*}
assemble pre-extension homomorphisms 
\(
\boldsymbol{\beta}
  :
\famesys{\famesys{\stesys}}
  \to
\famesys{\stesys}
\) 
and
\(
\boldsymbol{\beta}_\mathbf{2}
  :
\famesys{\famesys{\famesys{\stesys}}}
  \to
\famesys{\famesys{\stesys}}
\).
\end{defn}

\begin{defn}\label{famehom}
Suppose that $f:\stesys'\to\stesys$ is a pre-extension homomorphism. Then we
define $\famehom{f}:\famesys{\stesys'}\to\famesys{\stesys}$ to consist of
\begin{equation*}
\begin{tikzcd}
\stesyst_2'
  \ar{r}{f^t_2}
  \ar{d}[swap]{\ebd[2]'}
  &
\stesyst_2
  \ar{d}{\ebd[2]}
  \\
\stesysf_2'
  \ar{r}{f_2}
  \ar{d}[swap]{\eft[2]'}
  &
\stesysf_2
  \ar{d}{\eft[2]}
  \\
\stesysf'
  \ar{r}[swap]{f_1}
  &
\stesysf
\end{tikzcd}
\end{equation*}
where we define
\begin{equation*}
f^t_2 := \pullback{f_1}{f^t}{\ectxext}{\eft\circ\ebd}.
\end{equation*}
\end{defn}

\begin{lem}
The triple $\famehom{f}$ defined in \autoref{famehom} is a pre-extension homomorphism.
\end{lem}

\begin{proof}
Note that the square
\begin{equation*}
\begin{tikzcd}
\stesysf_2'
  \ar{r}{f_2}
  \ar{d}[swap]{\efamext'}
  &
\stesysf_2
  \ar{d}{\efamext}
  \\
\stesysf'
  \ar{r}[swap]{f_1}
  &
\stesysf
\end{tikzcd}
\end{equation*}
commutes by assumption. Thus, it remains to show that the square
\begin{equation*}
\begin{tikzcd}
\stesysf_3'
  \ar{r}{f_3}
  \ar{d}[swap]{\eext{2}'}
  &
\stesysf_3
  \ar{d}{\eext{2}}
  \\
\stesysf_2'
  \ar{r}[swap]{f_2}
  &
\stesysf_2
\end{tikzcd}
\end{equation*}
commutes. It is equivalent to show that the equalities
\begin{align*}
\pullbackpr{1}{\ectxext}{\eft}\circ f_2\circ\eext{2}'
  & =
\pullbackpr{1}{\ectxext}{\eft}\circ \eext{2}\circ f_3
  \\
\pullbackpr{2}{\ectxext}{\eft}\circ f_2\circ\eext{2}'
  & =
\pullbackpr{2}{\ectxext}{\eft}\circ \eext{2}\circ f_3
\end{align*}
both hold. For the first, it is straightforward to verify that the diagram
\begin{equation*}
\begin{tikzcd}[column sep=large]
{} &
\stesysf_2'
  \ar{r}{f_2}
  \ar{dr}[near end]{\pullbackpr{1}{\ectxext'}{\eft'}}
  &
\stesysf_2
  \ar{dr}{\pullbackpr{1}{\ectxext}{\eft}}
  \\
\stesysf_3'
  \ar{ur}{\eext{2}'}
  \ar{r}[swap]{\beta_2'}
  \ar{ddr}[swap]{f_3}
  &
\stesysf_2'
  \ar{r}{\efamext'}
  \ar{dr}[swap]{f_2}
  &
\stesysf'
  \ar{r}{f_1}
  &
\stesysf
  \\
{} & {} &
\stesysf_2
  \ar{ur}[near start]{\efamext}
  \\
{} &
\stesysf_3
  \ar{r}[swap]{\eext{2}}
  \ar{ur}{\beta_2}
  &
\stesysf_2
  \ar{uur}[swap]{\pullbackpr{1}{\ectxext}{\eft}}
\end{tikzcd}
\end{equation*}
commutes. For the second, note that the diagram
\begin{equation*}
\begin{tikzcd}[column sep=large]
{} &
\stesysf_2'
  \ar{r}{f_2}
  \ar{dr}{\beta_1'}
  &
\stesysf_2
  \ar{ddr}{\pullbackpr{2}{\ectxext}{\eft}}
  \\
{} & {} &
\stesysf'
  \ar{dr}[swap,near start]{f_1}
  \\
\stesysf_3'
  \ar{uur}{\eext{2}'}
  \ar{r}{\beta_2'}
  \ar{dr}[swap]{f_3}
  &
\stesysf_2'
  \ar{r}{f_2}
  \ar{ur}{\efamext'}
  &
\stesysf_2
  \ar{r}[swap]{\efamext}
  &
\stesysf
  \\
{} &
\stesysf_3
  \ar{r}[swap]{\eext{2}}
  \ar{ur}[near start]{\beta_2}
  &
\stesysf_2
  \ar{ur}[swap]{\pullbackpr{2}{\ectxext}{\eft}}
\end{tikzcd}
\end{equation*}
commutes.
\end{proof}

\begin{lem}[Stability under retracts]\label{esys-retract}
Suppose $f:\stesys\to\stesys'$ is a pre-extension homomorphism between
pre-extension algebras. If there is a pre-extension homomorphism $g:\stesys'\to
\stesys$ such that $g\circ f=\catid{\stesys}$ and $\stesys'$ is an extension
algebra, then $\stesys$ is an extension algebra.
\end{lem}

Before we start with the proof, note that we have the equalities
$g_2\circ f_2=\catid{\stesysf_2}$ and $g_3\circ f_3=\catid{\stesysf_3}$
under the hypotheses of the lemma.

\begin{proof}
Our first subgoal is to show that the square
\begin{equation*}
\begin{tikzcd}
\stesysf_2 
  \ar{r}{\efamext} 
  \ar{d}[swap]{\pullbackpr{2}{\ectxext}{\eft}} 
  & 
\stesysf 
  \ar{d}{\ectxext}
  \\
\stesysf
  \ar{r}[swap]{\ectxext} 
  & 
\stesysc
\end{tikzcd}
\end{equation*}
commutes. Note that in the diagram
\begin{equation*}
\begin{tikzcd}
  {}
  & 
\stesysf
  \ar{dd}[near start]{\ectxext}
  \ar{rr}{f_1}
  &
  &
\stesysf'
  \ar{dd}[near start]{\ectxext'}
  \ar{rr}{g_1}
  &
  &
\stesysf
  \ar{dd}{\ectxext}
  \\
\stesysf_2
  \ar{dd}[swap]{\pullbackpr{2}{\ectxext}{\eft}}
  \ar[crossing over]{rr}[swap,near start]{f_2}
  \ar{ur}{\efamext}
  &
  &
\stesysf_2'
  \ar{ur}[near start]{\efamext'}
  \ar[crossing over]{rr}[swap,near start]{g_2}
  &
  &
\stesysf_2
  \ar{ur}[swap,near start]{\efamext}
  \\
  {}
  &
\stesysc
  \ar{rr}[near start]{f_0}
  &
  &
\stesysc'
  \ar{rr}[near start]{g_0}
  &
  &
\stesysc
  \\
\stesysf 
  \ar{rr}[swap]{f_1}
  \ar{ur}{\ectxext}
  &
  &
\stesysf' 
  \ar[crossing over,leftarrow]{uu}[near end,swap]{\pullbackpr{2}{\ectxext'}{\eft'}}
  \ar{ur}[swap,near end]{\ectxext'}
  \ar{rr}[swap]{g_1}
  &
  &
\stesysf
  \ar[crossing over,leftarrow]{uu}[near end,swap]{\pullbackpr{2}{\ectxext}{\eft}}
  \ar{ur}[swap]{\ectxext}
\end{tikzcd}
\end{equation*}
all the faces minus the far left and far right face commute. Using that $g$
is a section of $f$, we can read off that also the far left face commutes,
completing our first subgoal.
 
For the second subgoal, note that also $\famehom{g}\circ\famehom{f}=
\catid{\famesys{\stesys}}$ and that $\famesys{\stesys'}$ is an extension algebra.
Thus we can apply what we have proven so far to conclude that the square
\begin{equation*}
\begin{tikzcd}
\stesysf_3 
  \ar{r}{\eext{2}} 
  \ar{d}[swap]{\pullbackpr{2}{\efamext}{\eft[2]}} 
  & 
\stesysf_2 
  \ar{d}{\efamext}
  \\
\stesysf_2
  \ar{r}[swap]{\efamext} 
  & 
\stesysf
\end{tikzcd}
\end{equation*}
commutes.
\end{proof}

\subsection{The change of base of (pre-)extension algebras}
\label{subsection:change_of_base}
An important construction of (pre-)extension algebras is the change of base. It
allows us to consider `parametrized homomorphisms', such as weakening and
substitution.

\begin{defn}
Suppose $f:\stesys\to\stesys'$ is a pre-extension homomorphism. We say that
a diagram
\begin{equation*}
\begin{tikzcd}
\stesys
  \ar{r}{f}
  \ar{d}[swap]{p}
  &
\stesys'
  \ar{d}{p'}
  \\
X \ar{r}[swap]{g}
  &
Y
\end{tikzcd}
\end{equation*}
commutes if the diagram
\begin{equation*}
\begin{tikzcd}
\stesysc
  \ar{r}{f_0}
  \ar{d}[swap]{p}
  &
\stesysc'
  \ar{d}{p'}
  \\
X \ar{r}[swap]{g}
  &
Y
\end{tikzcd}
\end{equation*}
commutes.
\end{defn}

The first goal in this subsection is to define for every (pre-)extension algebra 
$\stesys$ and every $p:\stesysc\rightarrow X\leftarrow Y:g$, a (pre-)extension
algebra $\cobesys{Y}{\stesys}{g}{p}$ with a homomorphism $\pullbackpr{2}{g}{p}:
\cobesys{Y}{\stesys}{g}{p}\to\stesys$ and a morphism $\pullbackpr{1}{g}{p}:
\pullback{Y}{\stesysc}{g}{p}\to Y$ such that for every diagram
\begin{equation*}
\begin{tikzcd}[column sep=large]
\stesys'
  \ar[bend right=10]{ddr}[swap]{p'}
  \ar[bend left=10]{rrd}{f}
  \ar[dotted]{dr}[near end]{[p',f]}
  \\
  {}&
\cobesys{Y}{\stesys}{g}{p}
  \ar{d}{\pullbackpr{1}{g}{p}}
  \ar{r}[swap]{\pullbackpr{2}{g}{p}}
  &
\stesys
  \ar{d}{p}
  \\
  {}&
Y \ar{r}[swap]{g}
  &
X
\end{tikzcd}
\end{equation*}
of which the outer square commutes, the (pre-)extension homomorphism $[p',f]$ exists
and is unique with the property that it renders the diagram commutative. We will
give the definition of $\cobesys{Y}{\stesys}{g}{p}$ in \autoref{cobesys}. After
proving that the change of base of a pre-extension algebra is indeed a
pre-extension algebra (\autoref{cobesys-preext}) and that the change of base
of an extension algebra is an extension algebra (\autoref{cobesys-ext}), we
will demonstrate the above unique existence in \autoref{cobesys-existence,%
cobesys-pullback}.

The second goal in this subsection is to follow the same procedure for
$\famesys{\famesys{\stesys}}$ to show that it is equivalent to
$\cobesys{\stesysf}{\famesys{\stesys}}{\ectxext}{\eft}$. We will do this by
verifying directly that it has the universal property of the change of base
described above, because we will use the ingredients in our definition of
weakening and substitution algebras.

\begin{defn}[Change of base]\label{cobesys}
Suppose $\stesys$ is a pre-extension algebra in $\cat{C}$ and that 
$p:\stesysc\rightarrow X\leftarrow Y:g$.
Then we define the pre-extension algebra $\cobesys{Y}{\stesys}{g}{p}$ to consist of
\begin{equation*}
\begin{tikzcd}
\cobesys{Y}{\stesyst}{g}{p\circ\eft\circ\ebd}
  \ar{r}
  \ar{d}[swap]{g^\ast(\ebd)}
  &
\stesyst
  \ar{d}{\ebd}
  \\
\cobesys{Y}{\stesysf}{g}{p\circ\eft}
  \ar{r}
  \ar{d}[swap]{g^\ast(\eft)}
  &
\stesysf
  \ar{d}{\eft}
  \\
\cobesys{Y}{\stesysc}{g}{p}
  \ar{r}
  \ar{d}[swap]{\pullbackpr{1}{g}{p}}
  &
\stesysc
  \ar{d}{p}
  \\
Y \ar{r}[swap]{g}
  &
X
\end{tikzcd}
\end{equation*} 
and the operations
\begin{align*}
\cobesys{Y}{\ectxext}{g}{p} 
  & : \pullback{Y}{\stesysf}{g}{p\circ\eft}\to \pullback{Y}{\stesysc}{g}{p}\\
\cobesys{Y}{\efamext}{g}{p} 
  & : \pullback
    {\pullback{Y}{\stesysf}{g}{p\circ\eft}}
    {\pullback{Y}{\stesysf}{g}{p\circ\eft}}
    {\cobesys{Y}{\ectxext}{g}{p}}
    {g^\ast(\eft)}
  \to 
  \pullback{Y}{\stesysf}{g}{p\circ\eft}.
\end{align*}
defined by
\begin{equation*}
\cobesys{Y}{\ectxext}{g}{p} := \pullback{\catid{Y}}{\ectxext}{g}{p}
\end{equation*}
and where $\cobesys{Y}{\efamext}{g}{p}$ is defined by rendering the diagram
\begin{equation*}
\begin{tikzcd}[column sep=large]
(\cobesys{Y}{\stesysf}{g}{p\circ\eft})_2
  \ar{rr}{\pullback{\pullbackpr{2}{g}{p\circ\eft}}{\pullbackpr{2}{g}{p\circ\eft}}{\ectxext}{\eft}}
  \ar{dd}[swap]{\pullbackpr{1}{\cobesys{Y}{\ectxext}{g}{p}}{g^\ast(\eft)}}
  \ar[dotted]{dr}[swap]{\cobesys{Y}{\efamext}{g}{p}}
  &
  &
\stesysf_2
  \ar{d}{\efamext}
  \\
  {}&
\cobesys{Y}{\stesysf}{g}{p\circ\eft}
  \ar{r}{\pullbackpr{2}{g}{p\circ\eft}}
  \ar{d}[swap]{\pullbackpr{1}{g}{p\circ\eft}}
  &
\stesysf
  \ar{d}{p\circ\eft}
  \\
\cobesys{Y}{\stesysf}{g}{p\circ\eft}
  \ar{r}[swap]{\pullbackpr{1}{g}{p\circ\eft}}
  &
Y \ar{r}[swap]{g}
  &
X
\end{tikzcd}
\end{equation*} 
commutative. 
The process of obtaining the pre-extension algebra $\cobesys{Y}{\stesys}{g}{p}$ out of $\stesys$
and $g:Y\to X$ is also called the \emph{change of base}.
\end{defn}

\begin{lem}\label{cobesys-preext}
Any change of base of a pre-extension algebra is a pre-extension algebra.
\end{lem}

\begin{proof}
Let $\stesys$ be an extension algebra and consider $p:\stesysc\rightarrow X\leftarrow Y:g$.
We need to verify that the square
\begin{equation*}
\begin{tikzcd}[column sep=large]
(\pullback{Y}{\stesysf}{g}{p\circ\eft})_2
  \ar{r}{\cobesys{Y}{\efamext}{g}{p}} 
  \ar{d}[swap]{\pullbackpr{1}{\cobesys{Y}{\ectxext}{g}{p}}{g^\ast(\eft)}} 
  & 
\pullback{Y}{\stesysf}{g}{p\circ\eft}
  \ar{d}{g^\ast(\eft)}
  \\
\pullback{Y}{\stesysf}{g}{p\circ\eft}
  \ar{r}[swap]{g^\ast(\eft)} 
  & 
\pullback{Y}{\stesysc}{g}{p}
\end{tikzcd}
\end{equation*}
commutes. It is fairly obvious that
\begin{equation*}
\pullbackpr{1}{g}{p}\circ g^\ast(\eft)\circ (\cobesys{Y}{\efamext}{g}{p})
  =
\pullbackpr{1}{g}{p\circ\eft}\circ \pullbackpr{1}{\cobesys{Y}{\ectxext}{g}{p}}{g^\ast(\eft)}
\end{equation*}
and that the diagram
\begin{equation*}
\begin{tikzcd}
  {}&
  {}&
\pullback{Y}{\stesysf}{g}{p\circ\eft}
  \ar{rr}{g^\ast(\eft)}
  \ar{dr}[swap]{\pullbackpr{2}{g}{p\circ\eft}}
  &
  {}&
\pullback{Y}{\stesysc}{g}{p}
  \ar{ddr}{\pullbackpr{2}{g}{p}}
  \\
  {}&
  {}&
  {}&
\stesysf
  \ar{drr}[swap]{\eft}
  \\
(\pullback{y}{\stesysf}{g}{p\circ\eft})_2
  \ar{uurr}{\cobesys{Y}{\efamext}{g}{p}}
  \ar{rr}[swap,yshift=-.5ex]{\pullback{\pullbackpr{2}{g}{p\circ\eft}}{\pullbackpr{2}{g}{p\circ\eft}}{\ectxext}{\eft}}
  \ar{ddrr}[swap]{\pullbackpr{1}{\cobesys{Y}{\ectxext}{g}{p}}{g^\ast(\eft)}}
  &
  {}&
\stesysf_2
  \ar{ur}{\efamext}
  \ar{dr}[swap]{\eft[2]}
  &
  {}&
  {}&
\stesysc
  \\
  {}&
  {}&
  {}&
\stesysf
  \ar{urr}{\eft}
  \\
  {}&
  {}&
\pullback{Y}{\stesysf}{g}{p\circ\eft}
  \ar{rr}[swap]{g^\ast(\eft)}
  \ar{ur}{\pullbackpr{2}{g}{p\circ\eft}}
  &
  {}&
\pullback{Y}{\stesysc}{g}{p}
  \ar{uur}[swap]{\pullbackpr{2}{g}{p}}
\end{tikzcd}
\end{equation*}
commutes.
\end{proof}

\begin{thm}\label{cobesys-ext}
The change of base of an extension algebra is an extension algebra.
\end{thm}

\begin{proof}
Our first subgoal is to verify that the square
\begin{equation*}
\begin{tikzcd}[column sep=large]
(\pullback{Y}{\stesysf}{g}{p\circ\eft})_2
  \ar{r}{\cobesys{Y}{\efamext}{g}{p}} 
  \ar{d}[swap]{\pullbackpr{2}{\cobesys{Y}{\ectxext}{g}{p}}{g^\ast(\eft)}} 
  & 
\pullback{Y}{\stesysf}{g}{p\circ\eft}
  \ar{d}{\cobesys{Y}{\ectxext}{g}{p}}
  \\
\pullback{Y}{\stesysf}{g}{p\circ\eft}
  \ar{r}[swap]{\cobesys{Y}{\ectxext}{g}{p}} 
  & 
\pullback{Y}{\stesysc}{g}{p}
\end{tikzcd}
\end{equation*}
\end{proof}

The following construction is useful for defining extension homomorphisms into
`higher' extension algebras

\begin{defn}\label{cobesys-existence}
Consider a commutative diagram
\begin{equation*}
\begin{tikzcd}
\stesys'
  \ar{r}{f}
  \ar{d}[swap]{p'}
  &
\stesys
  \ar{d}{p}
  \\
Y \ar{r}[swap]{g}
  &
X
\end{tikzcd}
\end{equation*}
Then we construct $[p,f]:\stesys'\to\cobesys{Y}{\stesys}{g}{p}$
\begin{itemize}
\item by defining $[p,f]_0:\stesysc'\to\pullback{Y}{\stesysc}{g}{p}$ be the uniqe
morphism rendering the diagram
\begin{equation*}
\begin{tikzcd}[column sep=large]
\stesysc'
  \ar[bend right=10]{ddr}[swap]{p'}
  \ar[bend left=10]{rrd}{f_0}
  \ar{dr}[near end]{[p,f]_0}
  \\
  {}&
\pullback{Y}{\stesysc}{g}{p}
  \ar{r}[swap]{\pullbackpr{2}{g}{p}}
  \ar{d}{\pullbackpr{1}{g}{p}}
  &
\stesysc
  \ar{d}{p}
  \\
  {}&
Y \ar{r}[swap]{g}
  &
X
\end{tikzcd}
\end{equation*}
commutative.
\item by defining $[p,f]_1:\stesysf'\to\pullback{Y}{\stesysf}{g}{p\circ\eft}$ be the uniqe
morphism rendering the diagram
\begin{equation*}
\begin{tikzcd}[column sep=huge]
\stesysf'
  \ar[bend right=10]{ddr}[swap]{p'\circ\eft'}
  \ar[bend left=10]{rrd}{f_1}
  \ar{dr}[near end]{[p,f]_1}
  \\
  {}&
\pullback{Y}{\stesysf}{g}{p\circ\eft}
  \ar{r}[swap]{\pullbackpr{2}{g}{p\circ\eft}}
  \ar{d}{\pullbackpr{1}{g}{p\circ\eft}}
  &
\stesysc
  \ar{d}{p\circ\eft}
  \\
  {}&
Y \ar{r}[swap]{g}
  &
X
\end{tikzcd}
\end{equation*}
commutative.
\item by defining $[p,f]^t:\stesyst'\to\pullback{Y}{\stesyst}{g}{p\circ\eft\circ\ebd}$ be the uniqe
morphism rendering the diagram
\begin{equation*}
\begin{tikzcd}[column sep=huge]
\stesyst'
  \ar[bend right=10]{ddr}[swap]{p'\circ\eft'\circ\ebd'}
  \ar[bend left=10]{rrd}{f^t}
  \ar{dr}[near end]{[p,f]^t}
  \\
  {}&
\pullback{Y}{\stesysf}{g}{p\circ\eft\circ\ebd}
  \ar{r}[swap]{\pullbackpr{2}{g}{p\circ\eft\circ\ebd}}
  \ar{d}{\pullbackpr{1}{g}{p\circ\eft\circ\ebd}}
  &
\stesysc
  \ar{d}{p\circ\eft\circ\ebd}
  \\
  {}&
Y \ar{r}[swap]{g}
  &
X
\end{tikzcd}
\end{equation*}
commutative.
\end{itemize}
\end{defn}

\begin{thm}\label{cobesys-pullback}
For every diagram
\begin{equation*}
\begin{tikzcd}[column sep=large]
\stesys'
  \ar[bend right=10]{ddr}[swap]{p'}
  \ar[bend left=10]{rrd}{f}
  \ar[dotted]{dr}[near end]{[p',f]}
  \\
  {}&
\cobesys{Y}{\stesys}{g}{p}
  \ar{d}{\pullbackpr{1}{g}{p}}
  \ar{r}[swap]{\pullbackpr{2}{g}{p}}
  &
\stesys
  \ar{d}{p}
  \\
  {}&
Y \ar{r}[swap]{g}
  &
X
\end{tikzcd}
\end{equation*}
of which the outer square commutes, the pre-extension homomorphism $[p',f]$
is unique with the property that it renders the whole diagram commutative.
\end{thm}

\begin{defn}\label{famfamstesys_into}
Consider a commutative square
\begin{equation*}
\begin{tikzcd}
\stesys'
  \ar{r}{f}
  \ar{d}[swap]{p}
  &
\famesys\stesys
  \ar{d}{\eft}
  \\
\stesysf \ar{r}[swap]{\ectxext}
  &
\stesysc
\end{tikzcd}
\end{equation*}
Then we construct
\begin{equation*}
[p,f]:\stesys'\to\famesys{\famesys{\stesys}}
\end{equation*}
as follows:
\begin{itemize}
\item let $[p,f]_0:\stesysc'\to\stesysf_2$ be the unique morphism rendering
the diagram
\begin{equation*}
\begin{tikzcd}[column sep=large]
\stesysc' 
  \ar[bend left=10]{rrd}{f_0}
  \ar[swap,bend right=10]{ddr}{p}
  \ar[dotted]{dr}[near end]{[p,f]_0}
  \\
  {}&
\stesysf_2
  \ar{r}[swap]{\pullbackpr{2}{\ectxext}{\eft}}
  \ar{d}{\eft[2]}
  &
\stesysf
  \ar{d}{\eft}
  \\
  {}&
\stesysf
  \ar{r}[swap]{\ectxext}
  &
\stesysc
\end{tikzcd}
\end{equation*}
commutative.
\item Let $[p,f]_1:\stesysf'\to\stesysf_3$ be the unique morphism rendering
the diagram
\begin{equation*}
\begin{tikzcd}[column sep=large]
\stesysf'
  \ar[bend left=10]{drr}{f_1}
  \ar[swap]{dd}{\eft'}
  \ar[dotted]{dr}[near end]{[p,f]_1}
  \\
  {}&
\stesysf_3
  \ar{r}[swap]{\pullbackpr{2}{\efamext}{\eft[2]}}
  \ar{d}{\eft[2]}
  &
\stesysf_2
  \ar{d}{\eft[2]}
  \\
\stesysc'
  \ar{r}[swap]{[p,f]_0}
  &
\stesysf_2
  \ar{r}[swap]{\efamext}
  &
\stesysf
\end{tikzcd}
\end{equation*}
commutative.
\item Let $[p,f]^t:\stesyst'\to\stesyst_3$ be the unique morphism rendering
the diagram
\begin{equation*}
\begin{tikzcd}[column sep=huge]
\stesyst'
  \ar[bend left=10]{drr}{f^t}
  \ar[swap]{dd}{\eft'\circ\ebd'}
  \ar[dotted]{dr}[near end]{[p,f]^t}
  \\
  {}&
\stesyst_3
  \ar{r}[swap]{\pullbackpr{2}{\efamext}{\eft[2]\circ\ebd[2]}}
  \ar{d}[swap]{\pullbackpr{1}{\efamext}{\eft[2]\circ\ebd[2]}}
  &
\stesyst_2
  \ar{d}{\eft[2]\circ\ebd[2]}
  \\
\stesysc'
  \ar{r}[swap]{[p,f]_0}
  &
\stesysf_2
  \ar{r}[swap]{\efamext}
  &
\stesysf
\end{tikzcd}
\end{equation*}
commutative.
\end{itemize}
\end{defn}

\begin{lem}
Under the hypotheses of \autoref{famfamstesys_into}, $[p,f]$ is a pre-extension
homomorphism. Moreover, it is the unique pre-extension homomorphism for which
the diagram
\begin{equation*}
\begin{tikzcd}[column sep=large]
\stesys' 
  \ar[bend left=10]{rrd}{f}
  \ar[swap,bend right=10]{ddr}{p}
  \ar[dotted]{dr}[near end]{[p,f]}
  \\
  {}&
\famesys{\famesys{\stesys}}
  \ar{r}[swap]{\pullbackpr{2}{\ectxext}{\eft}}
  \ar{d}{\eft[2]}
  &
\famesys{\stesys}
  \ar{d}{\eft}
  \\
  {}&
\stesysf
  \ar{r}[swap]{\ectxext}
  &
\stesysc
\end{tikzcd}
\end{equation*}
commutes.
\end{lem}

\begin{lem}
Suppose $f:\stesys\to \stesys'$ is a pre-extension homomorphism and consider a morphism
$p:\stesys'\to X$ and $g:Y\to X$. Then the change of base 
$g^\ast(f):\cobesys{Y}{\stesys}{g}{p\circ f_0}\to
\cobesys{Y}{\stesys'}{g}{p}$ is a pre-extension morphism.
\end{lem}

\begin{lem}
Let $\stesys$ be a pre-extension algebra and consider $p:\stesysc\rightarrow X\leftarrow Y:g$.
Then there is an isomorphism
\begin{equation*}
\varphi:\famesys{\cobesys{Y}{\stesys}{g}{p}}
  \simeq
\cobesys{Y}{\famesys{\stesys}}{g}{p\circ\eft}
\end{equation*}
uniquely determined by
\end{lem}

\begin{proof}
This follows from the pasting lemma for pullbacks.
\end{proof}


\section{The theory and semantics of weakening}

\subsection{The theory of weakening}
\label{weakening}

When $A$ is a family in context $\Gamma$, the operation of weakening by $A$
takes a family $B$ in context $\Gamma$ and provides a family $\ctxwk{A}{B}$
in context $\ctxext{\Gamma}{A}$. The context family $\ctxwk{A}{B}$ can be seen
as the constant family over $\ctxext{\Gamma}{A}$ with value $B$. This idea will
be axiomatized in the cancellation property of weakening and substitution in
\autoref{cancellation-ws}. In \autoref{morphisms} we will take the terms 
$\unfold{\jhom{\Gamma}{A}{B}{f}}$ to be the morphisms of families from $A$ to 
$B$. These will be at the heart of the categorical structure of the theory.

The weakening operation acts on three levels: on contexts, on families and
on terms. The `action on contexts' of weakening is the action we described
above: it takes a family $B$ over $\Gamma$ to a family $\ctxwk{A}{B}$ over
$\ctxext{\Gamma}{A}$; the `action on families' of weakening takes a family
$Q$ over $\ctxext{\Gamma}{B}$ to a family $\ctxwk[\famsym]{A}{Q}$ over
$\ctxext{{\Gamma}{A}}{\ctxwk{A}{B}}$; the `action on terms' of weakening takes
a term $g$ of $Q$ to a term $\ctxwk[\tmsym]{A}{g}$ of $\ctxwk[\famsym]{A}{Q}$.
\begin{align}
& \inference
  { \jfam{\Gamma}{A}
    \jfam{\Gamma}{B}
    }
  { \jfam{{\Gamma}{A}}{\ctxwk{A}{B}}
    }
& & \inference
    { \jfameq{\Gamma}{A}{A'}
      \jfameq{\Gamma}{B}{B'}
      }
    { \jfameq{{\Gamma}{A}}{\ctxwk{A}{B}}{\ctxwk{A'}{B'}}
      }
    \\
& \inference
  { \jfam{\Gamma}{A}
    \jfam{{\Gamma}{B}}{Q}
    }
  { \jfam{{{\Gamma}{A}}{\ctxwk{A}{B}}}{\ctxwk[\famsym]{A}{Q}}
    }
& & \inference
    { \jfameq{\Gamma}{A}{A'}
      \jfameq{{\Gamma}{B}}{Q}{Q'}
      }
    { \jfameq
        {{{\Gamma}{A}}{\ctxwk{A}{B}}}
        {\ctxwk[\famsym]{A}{Q}}
        {\ctxwk[\famsym]{A'}{Q'}}
      }
    \\
& \inference
  { \jfam{\Gamma}{A}
    \jterm{{\Gamma}{B}}{Q}{g}
    }
  { \jterm{{{\Gamma}{A}}{\ctxwk{A}{B}}}{\ctxwk[\famsym]{A}{Q}}{\ctxwk[\tmsym]{A}{g}}
    }
& & \inference
    { \jfameq{\Gamma}{A}{A'}
      \jtermeq{{\Gamma}{B}}{Q}{g}{g'}
      }
    { \jtermeq
        {{{\Gamma}{A}}{\ctxwk{A}{B}}}
        {\ctxwk[\famsym]{A}{Q}}
        {\ctxwk[\tmsym]{A}{g}}
        {\ctxwk[\tmsym]{A'}{g'}}
      }
\end{align}

\subsubsection{Weakenings of extensions}
\label{comp-we}
The following rules assert that when an extended family is weakened, the
weakening distributes over the extension factors.
\begin{align}
& \inference
  { \jfam{\Gamma}{A}
    \jfam{{\Gamma}{B}}{Q}
    }
  { \jfameq
      {\ctxext{\Gamma}{A}}
      {\ctxwk{A}{\ctxext{B}{Q}}}
      {\ctxext{\ctxwk{A}{B}}{\ctxwk{A}{Q}}}
    }
  \label{comp-we-c}
  \\
& \inference
  { \jfam{\Gamma}{A}
    \jfam{{{\Gamma}{B}}{Q}}{R}
    }
  { \jfameq
      {\ctxext{{\Gamma}{A}}{\ctxwk{A}{B}}}
      {\ctxwk{A}{\ctxext{Q}{R}}}
      {\ctxext{\ctxwk{A}{Q}}{\ctxwk{A}{R}}}
    }
  \label{comp-we-f}
\end{align}
When thinking of terms of $\ctxwk{A}{B}$ as morphisms of families from $A$ to
$B$, this looks already like form of type theoretic choice. It is weaker in that
it is not stated with function types, yet it is stronger in that it states a
judgmental equality between two families.

There is also a version of this property where an extended term is weakened.
This variant is stated and proved in \autoref{comp-we-t}.

\subsubsection{Weakening of weakenings}
Weakening by a family $A$ in context $\Gamma$ brings things in context $\Gamma$
to things in context $\ctxext{\Gamma}{A}$. Since we have all the ingredients of
the theory of contexts, families and terms in the context $\Gamma$ as well it
has in particular it's own weakening by families. Suppose we have a family
$\jfam{{\Gamma}{B}}{Q}$ over $B$ in context $\Gamma$. Weakening by $Q$ brings
things from context $\ctxext{\Gamma}{B}$ to $\ctxext{{\Gamma}{B}}{Q}$. Thus
we can provide rules asserting what will happen when we first weaken by $Q$ and
then (via the action on families) by $A$. We will require the following
inference rules to be valid:
\label{comp-ww}
\begin{align}
& \inference
  { \jfam{\Gamma}{A}
    \jfam{{\Gamma}{B}}{Q}
    \jfam{{\Gamma}{B}}{R}
    }
  { \jfameq
      {{{{\Gamma}{A}}{\ctxwk{A}{B}}}{\ctxwk{A}{Q}}}
      {\ctxwk{A}{{Q}{R}}}
      {\ctxwk{{A}{Q}}{{A}{R}}}
    }
  \label{comp-ww-c}\\
& \inference
  { \jfam{\Gamma}{A}
    \jfam{{\Gamma}{B}}{Q}
    \jfam{{{\Gamma}{B}}{R}}{S}
    }
  { \jfameq
      {{{{{\Gamma}{A}}{\ctxwk{A}{B}}}{\ctxwk{A}{Q}}}{\ctxwk{A}{{Q}{R}}}}
      {\ctxwk{A}{{Q}{S}}}
      {\ctxwk{{A}{Q}}{{A}{S}}}
    }
  \label{comp-ww-f}\\
& \inference
  { \jfam{\Gamma}{A}
    \jfam{{\Gamma}{B}}{Q}
    \jterm{{{\Gamma}{B}}{R}}{S}{k}
    }
  { \jtermeq
      {{{{{\Gamma}{A}}{\ctxwk{A}{B}}}{\ctxwk{A}{Q}}}{\ctxwk{A}{{Q}{R}}}}
      {\ctxwk{A}{{Q}{S}}}
      {\ctxwk{A}{{Q}{k}}}
      {\ctxwk{{A}{Q}}{{A}{k}}}
    }
  \label{comp-ww-t}
\end{align}

\begin{rmk}
As an important special case of these inference rules we have the following
valid inference rules:
\begin{align*}
& \inference
  { \jfam{\Gamma}{A}
    \jfam{\Gamma}{B}
    \jfam{\Gamma}{C}
    }
  { \jfameq
      {{{\Gamma}{A}}{\ctxwk{A}{B}}}
      {\ctxwk{A}{{B}{C}}}
      {\ctxwk{{A}{B}}{{A}{C}}}
    }
  \\
& \inference
  { \jfam{\Gamma}{A}
    \jfam{\Gamma}{B}
    \jfam{{\Gamma}{C}}{R}
    }
  { \jfameq
      {{{{\Gamma}{A}}{\ctxwk{A}{B}}}{\ctxwk{A}{{B}{C}}}}
      {\ctxwk{A}{{B}{R}}}
      {\ctxwk{{A}{B}}{{A}{R}}}
    }
  \\
& \inference
  { \jfam{\Gamma}{A}
    \jfam{\Gamma}{B}
    \jterm{{\Gamma}{C}}{R}{h}
    }
  { \jtermeq
      {{{{\Gamma}{A}}{\ctxwk{A}{B}}}{\ctxwk{A}{{B}{C}}}}
      {\ctxwk{A}{{B}{R}}}
      {\ctxwk{A}{{B}{h}}}
      {\ctxwk{{A}{B}}{{A}{h}}}
    }
\end{align*}
Moreover, we have
\begin{equation*}
\inference
  { \jfam{\Gamma}{A}
    \jfam{\Gamma}{B}
    \jterm{\Gamma}{C}{z}
    }
  { \jtermeq
      {{{\Gamma}{A}}{\ctxwk{A}{B}}}
      {\ctxwk{A}{{B}{C}}}
      {\ctxwk{A}{{B}{z}}}
      {\ctxwk{{A}{B}}{{A}{z}}}
    }
\end{equation*}
\end{rmk}

\subsubsection{Currying for weakening}
\label{comp-ew}
The rules expressing that extension is compatible with weakening assert that
weakening by an extension is the same thing as weakening twice in the
appropriate way.
\begin{align}
& \inference
  { \jfam{\Gamma}{A}
    \jfam{{\Gamma}{A}}{P}
    \jfam{\Gamma}{B}
    }
  { \jfameq
      {{{\Gamma}{A}}{P}}
      {\ctxwk{\ctxext{A}{P}}{B}}
      {\ctxwk{P}{{A}{B}}}
    }
  \label{comp-ew-c}\\
& \inference
  { \jfam{\Gamma}{A}
    \jfam{{\Gamma}{A}}{P}
    \jfam{{\Gamma}{B}}{Q}
    }
  { \jfameq
      {{{{\Gamma}{A}}{P}}{\ctxwk{P}{{A}{B}}}}
      {\ctxwk{\ctxext{A}{P}}{Q}}
      {\ctxwk{P}{{A}{Q}}}
    }
  \label{comp-ew-f}\\
& \inference
  { \jfam{\Gamma}{A}
    \jfam{{\Gamma}{A}}{P}
    \jterm{{\Gamma}{B}}{Q}{g}
    }
  { \jtermeq
      {{{{\Gamma}{A}}{P}}{\ctxwk{P}{{A}{B}}}}
      {\ctxwk{P}{{A}{Q}}}
      {\ctxwk{\ctxext{A}{P}}{g}}
      {\ctxwk{P}{{A}{g}}}
    } 
  \label{comp-ew-t}
\end{align}


\subsection{Pre-weakening algebras}
\begin{defn}
Let $\stesys$ be an extension algebra in $\cat{C}$. A pre-weakening operation
on $\stesys$ is an extension homomorphism 
$ \mathbf{w}(\stesys)
    :
  \cobesys{\stesysf}{\famesys{\stesys}}{\eft}{\eft}
    \to
  \famesys{\famesys{\stesys}}$
for which the diagram
\begin{equation*}
\begin{tikzcd}[column sep=large]
\cobesys{\stesysf}{\famesys{\stesys}}{\eft}{\eft}
  \ar{r}{\mathbf{w}(\stesys)}
  \ar{dr}[swap]{\pullbackpr{1}{\eft}{\eft}}
  &
\famesys{\famesys{\stesys}}
  \ar{d}{\eft[2]}
  \\
& \stesysf
\end{tikzcd}
\end{equation*}
commutes.
\end{defn}

\begin{defn}
Let $\stesys$ be an extension algebra with pre-weakening operation
$\mathbf{w}(\stesys)$. Then $\famesys{\stesys}$ has the pre-weakening operation
$\mathbf{w}(\famesys{\stesys})$ which is uniquely determined by rendering the
diagram
\begin{equation*}
\begin{tikzcd}[column sep=large]
\cobesys{\stesysf_2}{\famesys{\famesys{\stesys}}}{\eft[2]}{\eft[2]}
  \ar{rr}{%
      \pullback{\beta_1}{\boldsymbol{\beta}}{\eft}{\eft}
    }
  \ar[bend right]{ddr}[swap]{\pullbackpr{1}{\eft[2]}{\eft[2]}}
  \ar[dotted]{dr}{\mathbf{w}(\famesys{\stesys})}
  &
  {}&
\cobesys{\stesysf}{\famesys{\stesys}}{\eft}{\eft}
  \ar{r}{\mathbf{w}(\stesys)}
  &
\famesys{\famesys{\stesys}}
  \ar{d}{\boldsymbol{\beta}}
  \\
  {}&
\famesys{\famesys{\famesys{\stesys}}}
  \ar{r}{\boldsymbol{\beta}_\mathbf{2}}
  \ar{d}[swap]{\eft[3]}
  &
\famesys{\famesys{\stesys}}
  \ar{d}{\eft[2]}
  \ar{r}{\boldsymbol{\beta}}
  &
\famesys{\stesys}
  \ar{d}{\eft}
  \\
  {}&
\stesysf_2
  \ar{r}[swap]{\efamext}
  &
\stesysf
  \ar{r}[swap]{\ectxext}
  &
\stesysc
\end{tikzcd}
\end{equation*}
commutative.
\end{defn}

\begin{defn}
A pre-weakening algebra $\stesys$ in $\cat{C}$ is an extension algebra $\stesys$ 
with a pre-weakening operation 
$ \mathbf{w}(\stesys)
    :
  \cobesys{\stesysf}{\famesys{\stesys}}{\eft}{\eft}
    \to
  \famesys{\famesys{\stesys}}$
for which the diagram
\begin{equation*}
\begin{tikzcd}[column sep=15em]
\cobesys{\stesysf_2}{\famesys{\stesys}}{\eft\circ\eft[2]}{\eft}
  \ar[bend right=10]{dr}[swap]%
    { [ \pullbackpr{1}{\eft\circ\eft[2]}{\eft},%
        \mathbf{w}(\stesys)%
          \circ%
        (\pullback{\efamext}{\catid{\famesys{\stesys}}}{\eft}{\eft})%
        ]%
      }
  \ar{r}{
    [ \pullbackpr{1}{\eft\circ\eft[2]}{\eft},%
      \mathbf{w}(\stesys)%
        \circ%
      (\pullback{\eft[2]}{\catid{\famesys{\stesys}}}{\eft}{\eft})%
      ]}%
  &
\cobesys{\stesysf_2}{\famesys{\famesys{\stesys}}}{\eft[2]}{\eft[2]}
  \ar{d}{\mathbf{w}(\famesys{\stesys})}
  \\
  {}&
\famesys{\famesys{\famesys{\stesys}}}
\end{tikzcd}
\end{equation*}
commutes. This condition is called \emph{Currying for weakening}.
\end{defn}

\begin{lem}
If $\stesys$ is a pre-weakening algebra, then so is $\famesys{\stesys}$. 
\end{lem}

\begin{proof}
\end{proof}

\begin{defn}
A pre-weakening morphism between preweakening algebras $\stesys$ and $\stesys'$ is an
extension homomorphism $f:\stesys\to \stesys'$ such that additionally the diagram
\begin{equation*}
\begin{tikzcd}[column sep=large]
\cobesys{\stesysf}{\famesys{\stesys}}{\eft}{\eft}
  \ar{d}[swap]{\mathbf{w}(\stesys)}
  \ar{r}{\pullback{f_1}{\famehom{f}}{\eft'}{\eft'}}
  &
\cobesys{\stesysf'}{\famesys{\stesys'}}{\eft'}{\eft'}
  \ar{d}{\mathbf{w}(\stesys')}
  \\
\famesys{\famesys{\stesys}}
  \ar{r}[swap]{\famehom{\famehom{f}}}
  &
\famesys{\famesys{\stesys'}}
\end{tikzcd}
\end{equation*}
commutes.
\end{defn}

\begin{defn}
Let $\stesys$ be a pre-weakening algebra and consider $p:\stesysc\rightarrow X\leftarrow Y:p$.
Then we define
\begin{equation*}
\mathbf{w}(\cobesys{Y}{\stesys}{g}{p})
  :
\cobesys
  { (\pullback{Y}{\stesysf}{g}{p\circ\eft})}
  { \famesys{\cobesys{Y}{\stesys}{g}{p}}}
  { g^\ast(\eft)}
  { g^\ast(\eft)}
  \to
\famesys{\famesys{\cobesys{Y}{\stesys}{g}{p}}}
\end{equation*}
to be the unique extension homomorphism rendering the diagram
\begin{equation*}
\begin{tikzcd}
\cobesys
  { \pullback{Y}{\stesysf}{g}{p\circ\eft}}
  { \famesys{\cobesys{Y}{\stesys}{g}{p}}}
  { g^\ast(\eft)}
  { g^\ast(\eft)}
  \ar[bend right]{ddr}[swap]{\pullbackpr{1}{g^\ast(\eft)}{g^\ast(\eft)}}
  \ar{rr}{%
    \pullback
      { \pullbackpr{2}{g}{p\circ\eft}}
      { \boldsymbol{\pi}_\mathbf{2}(g,p\circ\eft)}
      { \eft}
      { \eft}
    }
  \ar[dotted]{dr}{\mathbf{w}(\cobesys{Y}{\stesys}{g}{p})}
  &
  {}&
\cobesys{\stesysf}{\famesys{\stesys}}{\eft}{\eft}
  \ar{r}{\mathbf{w}(\stesys)}
  &
\famesys{\famesys{\stesys}}
  \ar{d}{\boldsymbol{\beta}}
  \\
  {}&
\famesys{\famesys{\cobesys{Y}{\stesys}{g}{p}}}
  \ar{r}{\beta}
  \ar{d}{g^\ast(\eft)_1}
  &
\famesys{\cobesys{Y}{\stesys}{g}{p}}
  \ar{r}{\boldsymbol{\pi}_2(g,p\circ\eft)}
  \ar{d}{g^\ast(\eft)}
  &
\famesys{\stesys}
  \ar{d}{\eft}
  \\
  {}&
\pullback{Y}{\stesysf}{g}{p\circ\eft}
  \ar{r}[swap]{\cobesys{Y}{\ectxext}{g}{p}}
  &
\pullback{Y}{\stesysc}{g}{p}
  \ar{r}[swap]{\pullbackpr{2}{g}{p}}
  &
\stesysc
\end{tikzcd}
\end{equation*}
commutative.
\end{defn}

\subsection{Weakening algebras}
Since we have shown that the property of being a pre-weakening algebra is closed
under the relevant operations, we can make the following definition:

\begin{defn}
A weakening algebra is a pre-weakening algebra $\stesys$ with the property that
$\mathbf{w}(\stesys)$ is a pre-weakening morphism.
\end{defn}

\begin{defn}
A weakening homomorphism is a pre-weakening homomorphism such that the domain
and codomain are weakening algebras.
\end{defn}

\begin{thm}
Suppose $\stesys$ is a weakening algebra, then so is $\famesys{\stesys}$
\end{thm}

\begin{thm}
The change of base of any weakening algebra is again a weakening algebra.
\end{thm}



\subsection{Projection algebras}
\begin{defn}
A pre-projection algebra is a weakening algebra $\stesys$ for which there is a term
$\mathbf{i}:\stesysf\to \stesyst_2$ such that the diagram
\begin{equation*}
\begin{tikzcd}[column sep=large]
\stesysf \ar{r}{\mathbf{i}} \ar{d}[swap]{\Delta_{\eft}} & \stesyst_2 \ar{d}{\ebd[2]}\\
\pullback{\stesysf}{\stesysf}{\eft}{\eft} \ar{r}[swap]{w(\stesys)_0} & \stesysf_2
\end{tikzcd}
\end{equation*}
commutes. In this diagram $\Delta_{\eft}:\stesysf\to \pullback{\stesysf}{\stesysf}{\eft}{\eft}$ is the diagonal.
\end{defn}

\begin{defn}
A pre-projection homomorphism from $\stesys$ to $\stesys'$ is a weakening homomorphism
$f:\stesys\to \stesys'$ such that the square
\begin{equation*}
\begin{tikzcd}[column sep=large]
\stesyst_2
  \ar{r}{{f_t}_1}
  &
\stesyst_2'
  \\
\stesysf \ar{r}[swap]{f_1}
  \ar{u}{\mathbf{i}}
  &
\stesysf'
  \ar{u}[swap]{\mathbf{i}'}
\end{tikzcd}
\end{equation*}
commutes.
\end{defn}

\begin{lem}
The change of base of a pre-projection algebra is again a pre-projection algebra.
\end{lem}

\begin{lem}
If $CFT$ is a pre-projection algebra, then so is $\mathbf{F}_{CFT}$, where
$\mathbf{F}_{\mathbf{i}}$ is defined to be $F\times_{e_0,c}\mathbf{i}$ is
a pre-projection algebra.
\end{lem}

\begin{defn}
A projection algebra is a pre-projection algebra for which weakening is a
pre-projection homomorphism.
\end{defn}

\begin{cor}
The change of base of a projection algebra is again a projection algebra.
\end{cor}

\begin{cor}
If $CFT$ is a projection algebra, then so is $\mathbf{F}_{CFT}$, where
$\mathbf{F}_{\mathbf{i}}$ is defined to be $F\times_{e_0,c}\mathbf{i}$ is
a projection algebra.
\end{cor}


\section{The theory and semantics of substitution}

\subsection{The theory of substitution}
\label{substitution}

The theory of substitution can be formulated separately from the theory of
weakening. Therefore, we will only require the theory of extension and the
empty context and families. In \autoref{sec:esystem-equalities} we will
provide the additional judgmental equality rules that will be required when
combining the theories of substitution and of projections.

Given a family $P$ over $A$ and a term $x$ of $A$, substitution gives a way to
consider the \emph{fiber $\subst{x}{P}$ of $P$ at $x$}. As was the case with
weakening, the substitution operation comes in three parts: the `action on
contexts' of substitution is the operation just described; the `action on
families' of substitution takes a family $Q$ over $\ctxext{{\Gamma}{A}}{P}$
to a family $\subst[\famsym]{x}{Q}$ over $\ctxext{\Gamma}{\subst{x}{P}}$; the
`action on terms' of substitution takes a term $g$ of $Q$ over
$\ctxext{{\Gamma}{A}}{P}$ to a term $\subst[\tmsym]{x}{g}$ of 
$\subst[\famsym]{x}{Q}$.
\begin{align}
& \inference
  { \jterm{\Gamma}{A}{x}
    \jany{{\Gamma}{A}}{e}
    }
  { \jany{\Gamma}{\subst{x}{e}}
    }
\end{align}

\subsubsection{Fibers of extensions}
\label{comp-se}
The following inference rules assert that if we take the fiber of an extended
family at a term $x$ of $A$ in context $\Gamma$, the substitution by $x$
distributes over the factors of the extension.
\begin{align}
& \inference
  { \jterm{\Gamma}{A}{x}
    \jfam{{{\Gamma}{A}}{P}}{Q}
    }
  { \jfameq
      {\Gamma}
      {\subst{x}{\ctxext{P}{Q}}}
      {\ctxext{\subst{x}{P}}{\subst{x}{Q}}}
    }
  \label{comp-se-c}
  \\
& \inference
  { \jterm{\Gamma}{A}{x}
    \jfam{{{{\Gamma}{A}}{P}}{Q}}{R}
    }
  { \jfameq
      {{\Gamma}{\subst{x}{P}}}
      {\subst{x}{\ctxext{Q}{R}}}
      {\ctxext{\subst{x}{Q}}{\subst{x}{R}}}
    }
  \label{comp-se-f}
\end{align}
There is also a version of this statement in which extended terms are considered.
This variant is stated and proved in \autoref{comp-se-t}.

\subsubsection{Fibers of fibers}
\label{comp-ss}

The following rules explain what happens when we first substitute by a term
$g$ of $Q$ over $P$ in context $\ctxext{\Gamma}{A}$ and then by a term $x$ of
$A$:
\begin{align}
& \inference
  { \jterm{\Gamma}{A}{x}
    \jterm{{{\Gamma}{A}}{P}}{Q}{g}
    \jany{{{{\Gamma}{A}}{P}}{Q}}{e}
    }
  { \janyeq
      {{{\Gamma}{\subst{x}{P}}}{\subst{x}{Q}}}
      {\subst{x}{{g}{e}}}
      {\subst{{x}{g}}{{x}{e}}}
    }
  \label{comp-ss-any}
\end{align}

\begin{rmk}
As an important special case of these inference rules we have the following
valid inference rules:
\begin{align*}
& \inference
  { \jterm{\Gamma}{A}{x}
    \jterm{{\Gamma}{A}}{P}{f}
    \jany{{{\Gamma}{A}}{P}}{e}
    }
  { \janyeq
      {\Gamma}
      {\subst{x}{{f}{e}}}
      {\subst{{x}{f}}{{x}{e}}}
    }
\end{align*}
Moreover, in the presence of empty families we get
\begin{equation*}
\inference
  { \jterm{\Gamma}{A}{x}
    \jterm{{\Gamma}{A}}{P}{f}
    \jterm{{{\Gamma}{A}}{P}}{Q}{g}
    }
  { \jtermeq
      {\Gamma}
      {\subst{x}{{f}{Q}}}
      {\subst{x}{{f}{g}}}
      {\subst{{x}{f}}{{x}{g}}}
    }
\end{equation*}
\end{rmk}


\subsection{Substitution algebras}

\begin{defn}
A \emph{pre-substitution} for an extension algebra $\stesys$ is an
extension homomorphism
\begin{equation*}
\mathbf{s}(\stesys):\cobesys{\stesyst}{\famesys{\famesys{\stesys}}}{\ebd}{\eft[2]}\to \famesys{\stesys}
\end{equation*}
for which the square
\begin{equation*}
\begin{tikzcd}[column sep=large]
\pullback{\stesyst}{\stesysf_2}{\ebd}{\eft[2]}
  \ar{r}{s(\stesys)_0}
  \ar{d}[swap]{\ebd\circ\pullbackpr{1}{\ebd}{\eft[2]}}
  &
\stesysf 
  \ar{d}{\eft}
  \\
\stesysf 
  \ar{r}[swap]{\eft}
  &
\stesysc
\end{tikzcd}
\end{equation*}
commutes. A \emph{pre-substitution algebra} is an extension algebra
together with a pre-substitution.
\end{defn}

\begin{defn}
A \emph{pre-substitution homomorphism} is an extension homomorphism $f:\stesys'\to \stesys$
for which the square
\begin{equation*}
\begin{tikzcd}[column sep=huge]
\cobesys{\stesyst'}{\famesys{\famesys{\stesys'}}}{\ebd'}{\eft[2]'}
  \ar{r}{\pullback{f^t}{\famehom{\famehom{f}}}{\ebd}{\eft[2]}}
  \ar{d}[swap]{\mathbf{s}'(\stesys')}
  &
\cobesys{\stesyst}{\famesys{\famesys{\stesys}}}{\ebd}{\eft[2]}
  \ar{d}{\mathbf{s}(\stesys)}
  \\
\famesys{\stesys'}
  \ar{r}[swap]{\famehom{f}}
  &
\famesys{\stesys}
\end{tikzcd}
\end{equation*}
commutes.
\end{defn}

\begin{lem}
The change of base of a pre-substitution algebra is again a pre-substitution algebra.
\end{lem}

\begin{lem}
If $\stesys$ is a pre-substitution algebra, then so is $\famesys{\stesys}$ with
$\mathbf{s}(\famesys{\stesys})$ defined to be the unique extension homomorphism
rendering the diagram
\begin{equation*}
\begin{tikzcd}
\cobesys{\stesyst_2}{\famesys{\famesys{\famesys{\stesys}}}}{\ebd[2]}{\eft[3]}
  \ar[dotted]{dr}{\mathbf{s}(\famesys{\stesys})}
  \ar{rr}{\pullback{\pullbackpr{2}{\ectxext}{\eft\circ\ebd}}{\boldsymbol{\beta}_\mathbf{2}}{\ebd}{\eft[2]}}
  \ar{dd}[swap]{\ebd[2]\circ\pullbackpr{1}{\ebd[2]}{\eft[3]}}
  &
  {}&
\cobesys{\stesyst}{\famesys{\famesys{\stesys}}}{\ebd}{\eft[2]}
  \ar{d}{\mathbf{s}(\stesys)}
  \\
  {}&
\famesys{\famesys{\stesys}}
  \ar{r}{\boldsymbol{\beta}}
  \ar{d}{\eft[2]}
  &
\famesys{\stesys}
  \ar{d}{\eft}
  \\
\stesysf_2
  \ar{r}[swap]{\eft[2]}
  &
\stesysf
  \ar{r}[swap]{\ectxext}
  &
\stesysc
\end{tikzcd}
\end{equation*}
commutative.
\end{lem}

\begin{proof}
The requirement on pre-substitutions holds by construction.
\end{proof}

It makes sense now to consider the possibility that the pre-substitution
itself is a pre-substitution homomorphism.

\begin{defn}
A \emph{substitution algebra} is a pre-substitution algebra for which substitution is
a pre-substitution homomorphism.
\end{defn}

\begin{cor}
The change of base of a substitution algebra is again a substitution algebra.
\end{cor}

\begin{cor}
If $\stesys$ is a substitution algebra, then so is $\famesys{\stesys}$.
\end{cor}



\section{The theory of the empty context and families}

\subsection{The inference rules of the empty context and the empty families}
\label{empty}
We introduce an empty context and an family over $\Gamma$ for every context $\Gamma$. 
It has been suggested by some to only include empty families and not an
empty context because an empty context is not necessary, but we do have several
reasons to include them. Having an empty context requires also rules asserting 
that a context is the same thing as a family over the empty context and this
gives the categorical structure on contexts for free once one has it for 
families. The main ingredients that will be missing from the theory once an
empty context is avoided are weakening by a context and identity morphisms from
a context to itself. Including these by hand also requires to formulate all the
compatibility rules involving weakening once more for the cases of weakening by
a context and identity terms at contexts. We prefer to state these rules once
and only once and including an empty context helps in this respect.

We also prefer to have our scheme of compatibility rules as symmetrical as
possible. The structure of type dependency should look exactly the same in the
default case as in any context. In that respect, having an empty family but not
an empty context seems a bit odd. Also, some sets of compatibility rules 
(like the rules stating that extension is compatible with the empty families)
will become assymetrical as a result of not including an empty context. Moreover,
we would eventually like to include a stratification of the theory by means of
a type judgment (asserting that a family in a context $\Gamma$ is a type) and
study closed types (i.e.~types in the empty context). It would be possible to
provide a notion of closed types without having an empty context, but this would
have to be formulated separately and we would have to restate all the rules for
types (if any) for closed types all over again.

One of the main uses of the empty context and the empty families is that we
get the property that the `action on contexts' of an operation is compatible
with its `action on families'.

The empty family over a context $\Gamma$ is introduced by the following rule
inference rule:
\begin{align}
& \inference
  { }
  { \jctx{\emptyc}
    }
  \\
& \inference
  { \jctx{\Gamma}
    }
  { \jfam{\Gamma}{\emptyf[\Gamma]}
    }
  \\
& \inference
  { \jctxeq{\Gamma}{\Gamma'}
    }
  { \jfameq{\Gamma}{\emptyf[\Gamma]}{\emptyf[\Gamma']}
    }
\end{align}

By regarding contexts as families of contexts over the empty context, we
enable ourselves also to speak of terms of contexts. A term of a context
$\Gamma$ is a term of the family $\Gamma$ over the empty context. These ideas
are captured in the following convertibility rules:
\begin{align}
& \inference
  { \jctx{\Gamma}
    }
  { \jfam{\emptyc}{i(\Gamma)}
    } 
  &
& \inference
  { \jctxeq{\Gamma}{\Delta}
    }
  { \jfameq{\emptyc}{i(\Gamma)}{i(\Delta)}
    }
\end{align}

The reader may wonder whether the empty family $\emptyf$ always has a
term. This shall follow from the rules stating the compatibility of extension
with the empty families in \autoref{comp-0e} below and from
identity terms (\autoref{identityterms}).

\subsubsection{Compatibility of extension with the empty context and families}
In the following set of inference rules we state that the empty context and
the empty family are neutral objects for both context extension (the first two
rules) and family extension (the last two rules).
\label{comp-e0}\label{comp-0e}
\begin{align}
& \inference
  { \jctx{\Gamma}
    }
  { \jctxeq{\ctxext{\emptyc}{i(\Gamma)}}{\Gamma}
    }
  \label{comp-0e-c}
  \\
& \inference
  { \jctx{\Gamma}
    }
  { \jctxeq{\ctxext{\Gamma}{\emptyf}}{\Gamma}
    }
  \label{comp-e0-c}\\
& \inference
  { \jfam{\Gamma}{A}
    }
  { \jfameq{\Gamma}{\ctxext{\emptyf}{A}}{A}
    }
  \label{comp-0e-f}
  \\
& \inference
  { \jfam{\Gamma}{A}
    }
  { \jfameq{\Gamma}{{A}{\emptyf}}{A}
    }
  \label{comp-e0-f}
  \\
& \inference
  { \jfam{\emptyc}{A}
    }
  { \jfameq{\emptyc}{i(\ctxext{\emptyc}{A})}{A}
    }
\end{align}

\subsubsection{Family extension restricted to the empty context}
Strictly speaking we should have used a different notation for context extension
as for family extension, because the following rule asserting that family extension
in the empty context is the same thing as context extension would look tautological
without a difference. So let us denote, only for the moment, context extension
of $\Gamma$ by $A$ by $(\ctxext{\Gamma}{A})^c$ and family extension of $A$ by
$P$ in context $\Gamma$ by $(\ctxext{A}{P})^\famsym$. 

Note that we may consider a context $\Gamma$ as a family over $\emptyc$ and
a family $\jfam{\Gamma}{A}$ as a family $\jfam{{\emptyc}{\Gamma}}{A}$. 
Therefore we will require the following rule:
\begin{align}
& \inference
  { \jfam{\Gamma}{A}
    }
  { \jctxeq{\ctxext{\Gamma}{A}}{\ctxext{i(\Gamma)}{A}}
    }
\end{align}
Note that this rule actually justifies that we have not utilized two different
notations for context extension and family extension.


\subsection{Extension algebras with empty context and families}

\begin{defn}
An extension algebra $\stesys$ is said to have \emph{empty families} if there
is a section
\begin{equation*}
\phi_1(\stesys):\stesysc\to\stesysf
\end{equation*}
of $\eft$, satisfying the following additional properties:
\begin{enumerate}
\item $\phi_1(\stesys)$ is also a section of $\ectxext$.
\item The unique morphism $\phi_2$ rendering the diagram
\begin{equation*}
\begin{tikzcd}[column sep=huge]
\stesysf
  \ar[bend left=20]{drr}{\phi_1(\stesys)\circ\ectxext}
  \ar[equals,bend right=20]{ddr}
  \ar[dotted]{dr}{\phi_2}
  \\
  {}&
\stesysf_2
  \ar{d}{\eft[2]}
  \ar{r}{\pullbackpr{2}{\ectxext}{\eft}}
  &
\stesysf
  \ar{d}{\eft}
  \\
  {}&
\stesysf
  \ar{r}[swap]{\ectxext}
  &
\stesysc
\end{tikzcd}
\end{equation*}
commutative, is a section of $\efamext$.
\item The unique morphism $\iota_1$ rendering the diagram
\begin{equation*}
\begin{tikzcd}[column sep=huge]
\stesysf
  \ar[equals,bend left=20]{drr}
  \ar[bend right=20]{ddr}[swap]{\phi_1(\stesys)\circ\eft}
  \ar[dotted]{dr}{\iota_1}
  \\
  {}&
\stesysf_2
  \ar{d}{\eft[2]}
  \ar{r}[swap]{\pullbackpr{2}{\ectxext}{\eft}}
  &
\stesysf
  \ar{d}{\eft}
  \\
  {}&
\stesysf
  \ar{r}[swap]{\ectxext}
  &
\stesysc
\end{tikzcd}
\end{equation*}
commutative, is a section of $\efamext$.
\end{enumerate}
\end{defn}

\begin{defn}
A homomorphism of extension algebras with empty families is an extension
homomorphism $f:\stesys'\to\stesys$ for which the diagram
\begin{equation*}
\begin{tikzcd}
\stesysf'
  \ar{r}{f_1}
  &
\stesysf
  \\
\stesysc'
  \ar{u}{\phi_1(\stesys')}
  \ar{r}[swap]{f_0}
  &
\stesysc
  \ar{u}[swap]{\phi_1(\stesys)}
\end{tikzcd}
\end{equation*}
commutes.
\end{defn}

\begin{lem}
Suppose $\stesys$ is an extension algebra with empty families. Then
$\famesys{\stesys}$ is an extension algebra with empty families, with
$\phi_1(\famesys{\stesys}):=\phi_2$. 
\end{lem}

\begin{lem}
Suppose $\stesys$ is an extension algebra with empty families and consider
$p:\stesysc\rightarrow X\leftarrow Y:p$. Then
$\cobesys{Y}{\stesys}{g}{p}$ is an extension algebra with empty families
with $\phi_1(\cobesys{Y}{\stesys}{g}{p}):=g^\ast(\phi_1)$.
\end{lem}



\section{Joining the theories into dependent type theory}

\subsection{Joining the theory of projections with the theory of substitution}
\label{sec:esystem-equalities}

\subsubsection{Weakening of an empty family}
The following inference rules express that when the empty family is
weakened, the result is the empty family.
\label{comp-w0}
\begin{align}
& \inference
  { \jfam{\Gamma}{A}
    }
  { \jfameq{{\Gamma}{A}}{\ctxwk{A}{\emptyf}}{\emptyf}
    }
  \label{comp-w0-c}\\
& \inference
  { \jfam{\Gamma}{A}
    \jfam{\Gamma}{B}
    }
  { \jfameq
    {{{\Gamma}{A}}{\ctxwk{A}{B}}}
    {\ctxwk[\famsym]{A}{\emptyf}}
    {\emptyf}
    }
  \label{comp-w0-f}
\end{align}

Because a family over $\Gamma$ is the same as a family over 
$\ctxext{\Gamma}{\emptyf}$ we can apply both the action on contexts and the
action on families of weakening to a family $B$ over $\Gamma$. When we apply
the action on families, we obtain a family $\ctxwk[\famsym]{A}{B}$ over the
context $\ctxext{{\Gamma}{A}}{\ctxwk{A}{\emptyf}}$. However, since we have
postulated the judgmental equalities $\ctxwk{A}{\emptyf}\jdeq\emptyf$ and
$\ctxext{{\Gamma}{A}}{\emptyf}\jdeq\ctxext{\Gamma}{A}$, we see that we can
compare $\ctxwk[\famsym]{A}{B}$ with $\ctxwk{A}{B}$. The following inference
rule postulates that these two are judgmentally equal:
\begin{equation}
\inference
{ \jfam{\Gamma}{A}
  \jfam{\Gamma}{B}
  }
{ \jfameq{{\Gamma}{A}}{\ctxwk[\famsym]{A}{B}}{\ctxwk{A}{B}}
  }
\end{equation}
Due to this rule, the action on contexts and the action on families of weakening
are compatible with each other and consequently there can be no possible
confusion when we omit the annotations $\famsym$ and $\tmsym$ alltogether. In
the future, the weakening of a family $Q$ over $\ctxext{\Gamma}{B}$ shall
be denoted just by $\ctxwk{A}{Q}$ and likewise the weakening of a term $g$ of
$Q$ shall be denoted by $\ctxwk{A}{g}$.

Because a family $B$ over $\Gamma$ can be treated as a family by the operation
of weakening, weakening also acts on the terms of $B$. The weakening of a term
$y$ of $B$ by $A$ can be seen as the constant term (or function) of the
family $\ctxwk{A}{B}$.

\subsubsection{Weakening by the empty family}
Note that we can also weaken by the empty family over $\Gamma$.
Weakening by the empty family $\emptyf$ over a context $\Gamma$ leaves families, 
their terms, families over those families and terms of those unchanged:
\label{comp-0w}
\begin{align}
& \inference
  { \jany{\Gamma}{e}
    }
  { \janyeq{\Gamma}{\ctxwk{\emptyf}{e}}{e}
    }
  \label{comp-0w-any}
\end{align}

\subsubsection{Fibers of an empty family}
The following inference rules establish that the fibers of the empty family are 
the empty families:
\label{comp-s0}
\begin{align}
& \inference
  { \jterm{\Gamma}{A}{x}
    }
  { \jfameq{\Gamma}{\subst{x}{\emptyf}}{\emptyf}
    }
  \label{comp-s0-c}
  \\
& \inference
  { \jterm{\Gamma}{A}{x}
    \jfam{{\Gamma}{A}}{P}
    }
  { \jfameq
      {{\Gamma}{\subst{x}{P}}}
      {\subst{x}{\emptyf}}
      {\emptyf}
    }
  \label{comp-s0-f}
\end{align}

We use the above rule to state the compatibility of the action on families of
substitution with the action on contexts of substitution. Note that a family
$P$ over $\ctxext{\Gamma}{A}$ may be regarded as a family over
$\ctxext{{\Gamma}{A}}{\emptyf}$. Thus, we may consider the family
$\subst[\famsym]{x}{P}$ over $\ctxext{\Gamma}{\subst{x}{\emptyf}}$. Since
$\subst{x}{\emptyf}$ is judgmentally equal to the empty family, we may compare
$\subst[\famsym]{x}{P}$ with $\subst{x}{P}$:
\begin{equation}
\inference
{ \jfam{{\Gamma}{A}}{P}
  }
{ \jfameq{\Gamma}{\subst[\famsym]{x}{P}}{\subst{x}{P}}
  }
\end{equation}
Due to this rule we need not make the annotations $\famsym$ and $\tmsym$ in
the notation for substitution anymore and thus we shall omit them from now on.
Note that because a family $P$ over $\ctxext{\Gamma}{A}$ is eligible for
application of the action on families of substitution, we may also substitute
terms of $P$. Thus, given terms $\jterm{\Gamma}{A}{x}$ and $\jterm{{\Gamma}{A}}{P}{f}$,
we get a term $\jterm{\Gamma}{\subst{x}{P}}{\subst{x}{f}}$, the \emph{value of
$f$ at $x$}.

In \autoref{morphisms} we will use a combination of weakening and substitution
to define composition of morphisms of families. However, we have to rely
on the cancellation rule stated in \autoref{cancellation-ws} before we can
meaningfully state the definition of composition.

\subsubsection{Fibers of weakenings}\label{comp-sw}
The following rules assert what happens when we first weaken by a family
$Q$ over $P$ in context $\ctxext{\Gamma}{A}$ and then substitute by a term
$x$ of $A$:
\begin{align}
& \inference
  { \jterm{\Gamma}{A}{x}
    \jfam{{{\Gamma}{A}}{P}}{Q}
    \jany{{{\Gamma}{A}}{P}}{e}
    }
  { \janyeq
      {{{\Gamma}{\subst{x}{P}}}{\subst{x}{Q}}}
      {\subst{x}{\ctxwk{Q}{e}}}
      {\ctxwk{\subst{x}{Q}}{\subst{x}{e}}}
    }
  \label{comp-sw-any}
\end{align}

\begin{rmk}
As an important special case of these inference rules we have the following
valid inference rules:
\begin{align*}
& \inference
  { \jterm{\Gamma}{A}{x}
    \jfam{{\Gamma}{A}}{P}
    \jany{{\Gamma}{A}}{e}
    }
  { \janyeq
      {{\Gamma}{\subst{x}{P}}}
      {\subst{x}{\ctxwk{P}{e}}}
      {\ctxwk{\subst{x}{P}}{\subst{x}{e}}}
    }
\end{align*}
Moreover, we get
\begin{equation*}
\inference
  { \jterm{\Gamma}{A}{x}
    \jfam{{\Gamma}{A}}{P}
    \jterm{{\Gamma}{A}}{Q}{g}
    }
  { \jtermeq
      {{\Gamma}{\subst{x}{P}}}
      {\subst{x}{\ctxwk{P}{Q}}}
      {\subst{x}{\ctxwk{P}{g}}}
      {\ctxwk{\subst{x}{P}}{\subst{x}{g}}}
    }
\end{equation*}
\end{rmk}

\subsubsection{Weakenings of fibers}
\label{comp-ws}
The following inference rules explain what happens when we first weaken by a
term $g$ of $Q$ in context $\ctxext{\Gamma}{B}$ and then weaken by a family
$A$ over $\Gamma$.
\begin{align}
& \inference
  { \jfam{\Gamma}{A}
    \jterm{{\Gamma}{B}}{Q}{g}
    \jany{{{\Gamma}{B}}{Q}}{e}
    }
  { \janyeq
      {{{\Gamma}{A}}{\ctxwk{A}{B}}}
      {\ctxwk{A}{\subst{g}{e}}}
      {\subst{\ctxwk{A}{g}}{\ctxwk{A}{e}}}
    }
  \label{comp-ws-any}
\end{align}

\begin{rmk}
As an important special case of these inference rules we have the following 
valid inference rules:
\begin{align*}
& \inference
  { \jfam{\Gamma}{A}
    \jterm{\Gamma}{B}{y}
    \jany{{\Gamma}{B}}{e}
    }
  { \janyeq
      {{\Gamma}{A}}
      {\ctxwk{A}{\subst{y}{e}}}
      {\subst{\ctxwk{A}{y}}{\ctxwk{A}{e}}}
    }
\end{align*}
Moreover, we get
\begin{equation*}
\inference
  { \jfam{\Gamma}{A}
    \jterm{\Gamma}{B}{y}
    \jterm{{\Gamma}{B}}{Q}{g}
    }
  { \jtermeq
      {{\Gamma}{A}}
      {\ctxwk{A}{\subst{y}{Q}}}
      {\ctxwk{A}{\subst{y}{g}}}
      {\subst{\ctxwk{A}{y}}{\ctxwk{A}{g}}}
    }
\end{equation*}
\end{rmk}

\subsubsection{The cancellation property of weakening and substitution}
\label{cancellation-ws}
The judgmental equalities we're about to describe assert that substituting a term
in the weakening a thing gives you the thing back. In the case of contexts we get that each fiber
$\subst{x}{\ctxwk{A}{B}}$ is just $B$ and in the case of terms we get 
that $\ctxwk{A}{y}$ is the constant function
mapping everything to $y:B$. Thus, these rules actually establish the weakening
as the weakening. After stating the rules we will describe what it means to
compose context morphisms (terms of weakened contexts).
\begin{align}
& \inference
  { \jterm{\Gamma}{A}{x}
    \jany{\Gamma}{e}
    }
  { \janyeq
      {\Gamma}
      {\subst{x}{\ctxwk{A}{e}}}
      {e}
    }
  \label{cancellation-ws-e}
\end{align}

\subsubsection{The identity term of a substituted family}
\label{comp-si}
The identity term of a substituted family is the substitution of the identity term
\begin{equation}
\inference
  { \jterm{\Gamma}{A}{x}
    \jfam{{{\Gamma}{A}}{P}}{Q}
    }
  { \unfoldall{\jhomeq
      {{\Gamma}{\subst{x}{P}}}
      {\subst{x}{Q}}
      {\subst{x}{Q}}
      {\subst{x}{\idtm{Q}}}
      {\idtm{\subst{x}{Q}}}
    }}
  \label{comp-si-t}
\end{equation}

\begin{rmk}
An important special case is the judgmental equality
\begin{equation*}
\unfoldall{\jhomeq
      {\Gamma}
      {\subst{x}{P}}
      {\subst{x}{P}}
      {\subst{x}{\idtm{P}}}
      {\idtm{\subst{x}{P}}}
    }
\end{equation*}
for a family $\jfam{{\Gamma}{A}}{P}$.
\end{rmk}

\subsubsection{The cancellation property of identity terms}
\label{cancellation-i}
Identity terms are determined by their behavior with respect to substitution combined with
weakening. The identity terms will also be subject to compatibility rules.
\begin{align}
& \inference
  { \jterm{\Gamma}{A}{x}
    }
  { \jtermeq{\Gamma}{A}{\subst{x}{\idtm{A}}}{x}
    }
  \label{cancellation-si}\\
& \inference
  { \jany{{\Gamma}{A}}{e}
    }
  { \janyeq{{\Gamma}{A}}{\subst{\idtm{A}}{\ctxwk{A}{e}}}{e}
    }
  \label{precomp-idtm-any}
\end{align}


\subsection{E-systems}
\begin{defn}
An \emph{E-system} is an extension algebra with the structure of a projection algebra,
the structure of a substitution algebra and which has an empty context and families,
such that additionally:
\begin{enumerate}
\item substitution is a projection homomorphism
\item weakening is a substitution homomorphism
\item both weakening and substitution are empty-CF homomorphisms.
\item 
\end{enumerate}
\end{defn}



%\section{Derived notions of the theory of contexts, families and terms}
\label{ttderived}

In this section we use the framework we have developed in \autoref{tt}
to derive new notions and their properties. In particular, we will
develop the notion of \emph{extension on terms} together with the projection
maps from an extension to the `base' context of family, the 
\emph{family pullback} which is a version of pullbacks for families and thirdly
the \emph{inductive morphisms} which are morphisms of type theory that allow
to find terms of families over the codomain context by pulling them back to
the domain context and finding a term there.
\emph{This section contains no new assumptions.}

In \autoref{extension-on-terms} on the extension operation on terms we will
derive all the compatibility rules that one would expect to hold for extension
on terms. The key to most of these results is the currying operation, which
could be seen as the missing feature in the table above. The extension on terms
operation depends in an essential way on the substitution operation, on the
identity terms and therefore indirectly also on the weakening operation. Thus,
we will see here all of the features of the theory we develop in
\autoref{tt} come to the 
surface.

Next, we introduce the inclusion of the fibers $\subst{x}{P}$ into the extension
$\ctxext{A}{P}$ as a morphism in context $\Gamma$. As was the case with
extension on terms and with projections, there will be a ton of compatibility
properties which we will prove about these inclusions. 

It should be kept in mind though that in the current formulation there is no
sealed deal establishing a relationship between families over a context
and any kind of morphisms -- neither with morphisms having the `base' of the
family as its codomain nor with families into a universe (universes will be
introduced in \autoref{universes}). The only thing we know here is
that a family $P$ over $\ctxext{\Gamma}{A}$ determines a context morphism
from $\ctxext{A}{P}$ to $A$ in context $\Gamma$, the projection. 
We do not see this as
a shortcoming of the theory of contexts families and terms. Rather, such a
correspondence is a feature of a theory which does incorporate universes. The
fact that we're lacking a clear connection between families and (a specified
class of) morphisms, however, does show up in our treatment of the notion we
called familie pullbacks. For instance, we can't show that a square of families
is a pullback precisely when the corresponding square of projections is a
pullback: only the backwards direction holds. 
The discrepancies continue: ordinary pullbacks do not always exist
whereas family pullbacks do but the composition of two family pullback squares
need not be a family pullback square again whereas the pasting lemma of
ordinary pullbacks holds as usually.
We feel that pointing out what we can and can't do in the current setting is
an important aspect of developing an intuition with the system and therefore
we include this subsection even though the theory of family pullbacks
might feel a bit different than the usual theory of pullbacks.

%%%%%%%%%%%%%%%%%%%%%%%%%%%%%%%%%%%%%%%%%%%%%%%%%%%%%%%%%%%%%%%%%%%%%%%%%%%%%%%%
\subsection{Morphisms}\label{morphisms}
Using the rules of the compatibility of substitution with weakening and of the
compatibility of weakening with itself, we see that we can show

\begin{lem}\label{lem:prehom}
The inference rule\begin{equation*}
\inference
  { \jfam{\Gamma}{A}
    \jfam{\Gamma}{B}
    \jfam{\Gamma}{C}
    \jhom{\Gamma}{A}{B}{f}
    }
  { \jfameq
    {{\Gamma}{A}}
    {\subst{f}{\ctxwk{A}{{B}{C}}}}
    {\ctxwk{A}{C}}
    }
\end{equation*}
is valid.
\end{lem}

\begin{proof}
Let $\jfam{\Gamma}{A}$, $\jfam{\Gamma}{B}$, $\jfam{\Gamma}{C}$ and $\jhom{\Gamma}{A}{B}{f}$.
Then we have the judgmental equalities
\begin{align*}
\subst{f}{\ctxwk{A}{{B}{C}}}
& \jdeq 
  \subst{f}{\ctxwk{{A}{B}}{{A}{C}}}
  \tag{by \autoref{comp-ww-f}}\\
& \jdeq 
  \ctxwk{A}{C}.
  \tag{by \autoref{cancellation-ws-t}}
\end{align*}
\end{proof}

It follows that for $\jterm{{\Gamma}{B}}{\ctxwk{B}{C}}{g}$ we can compose $f$
with $g$ to obtain a term of $\ctxwk{A}{C}$ in context $\ctxext{\Gamma}{A}$.
In the following definition, we work with in a slightly greater generality.

\begin{defn}
We define the judgment
\begin{equation*}
\jhom{\Gamma}{A}{B}{f},
\end{equation*}
which is pronounced as `$f$ is a morphism from $A$ to $B$ in context $\Gamma$',
to be the judgment
\begin{equation*}
\unfold{\jhom{\Gamma}{A}{B}{f}}.
\end{equation*}
Likewise, we define the judgment
\begin{equation*}
\jhomeq{\Gamma}{A}{B}{f}{f'}
\end{equation*}
to be the judgment
\begin{equation*}
\unfold{\jhomeq{\Gamma}{A}{B}{f}{f'}}.
\end{equation*}
As usual, when $f$ is a morphism from $A$ to $B$ in context $\Gamma$ we say that
$A$ is the \emph{domain of $f$} and that $B$ is the \emph{codomain of $f$}. 
\end{defn}

In two special cases we have alternative explanations of what a morphism is.
These special cases are when the domain is either empty or an extension.

\begin{lem}
We have the following valid inference rules expressing that a morphism from
the empty family in context $\Gamma$ is the same thing as a term:
\begin{align*}
& \inference
  { \jhom{\Gamma}{\emptyf}{A}{x}
    }
  { \jterm{\Gamma}{A}{x}
    }
  \\
& \inference
  { \jterm{\Gamma}{A}{x}
    }
  { \jhom{\Gamma}{\emptyf}{A}{x}
    }
\end{align*}
\end{lem}

\begin{lem}\label{lem:jhomdomext-jhomcodwk}
We have the following valid inference rules expressing that a morphism from
an extension $\ctxext{A}{P}$ to $B$ is the same thing as a morphism from $P$ into
$\ctxwk{A}{B}$:
\begin{align*}
& \inference
  { \jhom{\Gamma}{{A}{P}}{B}{f}
    }
  { \jhom{{\Gamma}{A}}{P}{\ctxwk{A}{B}}{f}
    }
  \\
& \inference
  { \jhom{{\Gamma}{A}}{P}{\ctxwk{A}{B}}{f}
    }
  { \jhom{\Gamma}{{A}{P}}{B}{f}
    }
\end{align*}
\end{lem}

We devote the rest of this subsection to precomposition by a morphism. We treat
precomposition by a morphism $f$ from $A$ to $B$ in context $\Gamma$ as an
\emph{operation} from things in context $\ctxext{\Gamma}{B}$ to things in
context $\ctxext{\Gamma}{A}$ rather than as something which only acts on
morphisms with domain $B$. Since precomposition by a morphism is seen as an
operation, we will give it an `action on contexts', an `action on families' and
and `action on terms' and we will follow the usual scheme in verifying that
it is compatible with the empty context and families, with context and family
extension, weakening, substitution and identity terms. Moreover, we will show
that precomposition is compatible with itself; this is a slight generalization
of associativity of composition. We will be able to retrieve the usual
composition of morphisms via the action on terms of the precomposition operation.

\begin{defn}
Let $\jhom{\Gamma}{A}{B}{f}$ and consider a family $\jfam{{\Gamma}{B}}{Q}$,
a family $\jfam{{{\Gamma}{B}}{Q}}{R}$ and a term $\jterm{{{\Gamma}{B}}{Q}}{R}{h}$.
We define
\begin{align*}
\jalign\jfamdefn
  {{\Gamma}{A}}
  {\jcomp{A}{f}{Q}}
  {\unfold{\jcomp{A}{f}{Q}}}
  \\
\jalign\jfamdefn
  {{{\Gamma}{A}}{\jcomp{A}{f}{Q}}}
  {\jcomp{A}{f}{R}}
  {\unfold{\jcomp{A}{f}{R}}}
  \\
\jalign\jtermdefn
  {{{\Gamma}{A}}{\jcomp{A}{f}{Q}}}
  {\jcomp{A}{f}{R}}
  {\jcomp{A}{f}{h}}
  {\unfold{\jcomp{A}{f}{h}}}.
\end{align*}
\end{defn}

\begin{lem}\label{lem:jcomp-emp}
Let $\jhom{\Gamma}{A}{B}{f}$. Then the inference rules
\begin{align*}
& \inference
  { %
    }
  { \jfameq{{\Gamma}{A}}{\jcomp{A}{f}{\emptyf}}{\emptyf}
    }
  \\
& \inference
  { \jfam{{\Gamma}{B}}{Q}
    }
  { \jfameq{{{\Gamma}{B}}{\jcomp{A}{f}{Q}}}{\jcomp{A}{f}{\emptyf}}{\emptyf}
    }
\end{align*}
are valid.
\end{lem}

\begin{proof}
For the first, note that we have the judgmental equalities
\begin{align*}
\jcomp{A}{f}{\emptyf}
& \jdeq
  \unfold{\jcomp{A}{f}{\emptyf}}
  \tag{by definition}
  \\
& \jdeq
  \subst{f}{\emptyf}
  \tag{by \autoref{comp-w0-c}}
  \\
& \jdeq
  \emptyf.
  \tag{by \autoref{comp-s0-c}}
\end{align*}
The second judgmental equality is proven similarly.
\end{proof}

\begin{rmk}
Recall that we can treat a family $\jfam{{\Gamma}{B}}{Q}$ as a family
$\jfam{{{\Gamma}{B}}{\emptyf}}{Q}$ and that $\jcomp{A}{f}{\emptyf}\jdeq
\emptyf$. Thus we can apply composition with $f$ to terms 
$\jterm{{\Gamma}{B}}{Q}{g}$. We get
\begin{equation*}
\jtermeq
  {{\Gamma}{B}}
  {\jcomp{A}{f}{Q}}
  {\jcomp{A}{f}{g}}
  {\unfold{\jcomp{A}{f}{g}}}.
\end{equation*}
In the particular situation where we take $Q$ to be a weakened family
$\ctxwk{B}{C}$, we see that we can apply composition with $f$ to morphisms
from $B$ to $C$ and we can use \autoref{lem:prehom} to see that we get
\begin{equation*}
\jhomeq{\Gamma}{A}{C}{\jcomp{A}{f}{g}}{\unfold{\jcomp{A}{f}{g}}}
\end{equation*}
for $\jhom{\Gamma}{B}{C}{g}$. 

One might argue that the notation for composition should be reserved to only
this special case, to not confuse with common intuition of composition. It is
however very convenient to see composition as an operation of the theory of
contexts, families and terms. This allows us to follow the scheme of
compatibility rules which are provable for this form of composition. 
\end{rmk}

\begin{lem}\label{lem:jcomp-ext}
We have the following inference rule expressing that composition with $f$ is
compatible with extension:
\begin{equation*}
\inference
  { \jfam{{{{\Gamma}{B}}{Q}}{R}}{S}
    }
  { \jfameq
      {{{\Gamma}{A}}{\jcomp{A}{f}{Q}}}
      {\jcomp{A}{f}{\ctxext{R}{S}}}
      {\ctxext{\jcomp{A}{f}{R}}{\jcomp{A}{f}{S}}}
    }
\end{equation*}
\end{lem}

\begin{proof}
Let $\jfam{{{{\Gamma}{B}}{Q}}{R}}{S}$. Then we have the judgmental equalities
\begin{align*}
\jcomp{A}{f}{\ctxext{R}{S}}
& \jdeq
  \unfold{\jcomp{A}{f}{\ctxext{R}{S}}}
  \tag{by definition}
  \\
& \jdeq
  \subst{f}{\ctxext{\ctxwk{A}{R}}{\ctxwk{A}{S}}}
  \tag{by \autoref{comp-we-f}}
  \\
& \jdeq
  \unfoldall{\ctxext{\jcomp{A}{f}{R}}{\jcomp{A}{f}{S}}}
  \tag{by \autoref{comp-se-f}}
  \\
& \jdeq
  \ctxext{\jcomp{A}{f}{R}}{\jcomp{A}{f}{S}}.
  \tag{by definition}
\end{align*}
\end{proof}

\begin{lem}\label{lem:jcomp-wk}
We have the following inference rules expressing that composition with $f$ is
compatible with weakening:
\begin{align*}
& \inference
  { \jfam{{{\Gamma}{B}}{Q}}{R}
    \jfam{{{\Gamma}{B}}{Q}}{S}
    }
  { \jfameq
      {{{{\Gamma}{A}}{\jcomp{A}{f}{Q}}}{\jcomp{A}{f}{R}}}
      {\jcomp{A}{f}{\ctxwk{R}{S}}}
      {\ctxwk{\jcomp{A}{f}{R}}{\jcomp{A}{f}{S}}}
    }
  \\
& \inference
  { \jfam{{{\Gamma}{B}}{Q}}{R}
    \jterm{{{\Gamma}{B}}{Q}}{S}{k}
    }
  { \jtermeq
      {{{{\Gamma}{A}}{\jcomp{A}{f}{Q}}}{\jcomp{A}{f}{R}}}
      {\jcomp{A}{f}{\ctxwk{R}{S}}}
      {\jcomp{A}{f}{\ctxwk{R}{k}}}
      {\ctxwk{\jcomp{A}{f}{R}}{\jcomp{A}{f}{k}}}
    }
\end{align*}
\end{lem}

\begin{proof}
Consider the families $\jfam{{{\Gamma}{B}}{Q}}{R}$ and 
$\jfam{{{\Gamma}{B}}{Q}}{S}$. Then we have the judgmental equalities
\begin{align*}
\jcomp{A}{f}{\ctxwk{R}{S}}
& \jdeq
  \unfold{\jcomp{A}{f}{\ctxwk{R}{S}}}
  \tag{by definition}
  \\
& \jdeq
  \subst{f}{\ctxwk{{A}{R}}{{A}{S}}}
  \tag{by \autoref{comp-ww-f}}
  \\
& \jdeq
  \unfoldall{\ctxwk{\jcomp{A}{f}{R}}{\jcomp{A}{f}{S}}}
  \tag{by \autoref{comp-sw-f}}
  \\
& \jdeq
  \ctxwk{\jcomp{A}{f}{R}}{\jcomp{A}{f}{S}}.
  \tag{by definition}
\end{align*}
The proof of the second property is similar.
\end{proof}

\begin{lem}\label{lem:jcomp-subst}
We have the following inference rules expressing that composition with $f$ is
compatible with substitution:
\begin{align*}
& \inference
  { \jterm{{{\Gamma}{B}}{Q}}{R}{h}
    \jfam{{{{\Gamma}{B}}{Q}}{R}}{S}
    }
  { \jfameq
      {{{\Gamma}{A}}{\jcomp{A}{f}{Q}}}
      {\jcomp{A}{f}{\subst{h}{S}}}
      {\subst{\jcomp{A}{f}{h}}{\jcomp{A}{f}{S}}}
    }
  \\
& \inference
  { \jterm{{{\Gamma}{B}}{Q}}{R}{h}
    \jterm{{{{\Gamma}{B}}{Q}}{R}}{S}{k}
    }
  { \jtermeq
      {{{\Gamma}{A}}{\jcomp{A}{f}{Q}}}
      {\jcomp{A}{f}{\subst{h}{S}}}
      {\jcomp{A}{f}{\subst{h}{k}}}
      {\subst{\jcomp{A}{f}{h}}{\jcomp{A}{f}{k}}}
    }
\end{align*}
\end{lem}

\begin{proof}
Let $\jterm{{{\Gamma}{B}}{Q}}{R}{h}$ and $\jfam{{{{\Gamma}{B}}{Q}}{R}}{S}$.
Then we have the judgmental equalities
\begin{align*}
\jcomp{A}{f}{\subst{h}{S}}
& \jdeq
  \unfold{\jcomp{A}{f}{\subst{h}{S}}}
  \tag{by definition}
  \\
& \jdeq
  \subst{f}{{\ctxwk{A}{h}}{\ctxwk{A}{S}}}
  \tag{by \autoref{comp-ws-f}}
  \\
& \jdeq
  \unfoldall{\subst{\jcomp{A}{f}{h}}{\jcomp{A}{f}{S}}}
  \tag{by \autoref{comp-ss-f}}
  \\
& \jdeq
  \subst{\jcomp{A}{f}{h}}{\jcomp{A}{f}{S}}.
  \tag{by definition}
\end{align*}
The proof of the second inference rule is similar.
\end{proof}

\begin{lem}\label{lem:jcomp-idtm}
We have the following inference rule expressing that composition with $f$ is
compatible with identity terms:
\begin{equation*}
\inference
  { \jfam{{{\Gamma}{B}}{Q}}{R}
    }
  { \jtermeq
      {{{{\Gamma}{A}}{\jcomp{A}{f}{Q}}}{\jcomp{A}{f}{R}}}
      {\ctxwk{\jcomp{A}{f}{R}}{\jcomp{A}{f}{R}}}
      {\jcomp{A}{f}{\idtm{R}}}
      {\idtm{\jcomp{A}{f}{R}}}
    }
\end{equation*}
\end{lem}

\begin{proof}
Let $\jfam{{{\Gamma}{B}}{Q}}{R}$. We have the following judgmental equalities:
\begin{align*}
\jcomp{A}{f}{\idtm{R}}
& \jdeq
  \unfold{\jcomp{A}{f}{\idtm{R}}}
  \tag{by definition}
  \\
& \jdeq
  \subst{f}{\idtm{\ctxwk{A}{R}}}
  \tag{by \autoref{comp-wi-t}}
  \\
& \jdeq
  \unfoldall{\idtm{\jcomp{A}{f}{R}}}
  \tag{by \autoref{comp-si-t}}
  \\
& \jdeq
  \idtm{\jcomp{A}{f}{R}}.
  \tag{by definition}
\end{align*}
\end{proof}

\begin{lem}\label{lem:jcomp-jcomp}
We have the following inference rules about the situation where something is
substituted by a composition:\begin{align*}
& \inference
  { \jhom{\Gamma}{A}{B}{f}
    \jhom{\Gamma}{B}{C}{g}
    \jfam{{{\Gamma}{A}}{\ctxwk{A}{C}}}{R}
    }
  { \jfameq
      {{\Gamma}{A}}
      {\subst{\jcomp{A}{f}{g}}{R}}
      {\subst{f}{{\ctxwk{A}{g}}{\ctxwk{{A}{B}}{R}}}}
    }
  \\
& \inference
  { \jhom{\Gamma}{A}{B}{f}
    \jhom{\Gamma}{B}{C}{g}
    \jterm{{{\Gamma}{A}}{\ctxwk{A}{C}}}{R}{h}
    }
  { \jfameq
    {{\Gamma}{A}}
    {\subst{\jcomp{A}{f}{g}}{h}}
    {\subst{f}{{\ctxwk{A}{g}}{\ctxwk{{A}{B}}{h}}}}
    }
\end{align*}
We also have the following related inference rules, asserting that composition
is strictly associative:\begin{align*}
& \inference
  { \jhom{\Gamma}{A}{B}{f}
    \jhom{\Gamma}{B}{C}{g}
    \jfam{{\Gamma}{C}}{R}
    }
  { \jfameq
      {{\Gamma}{A}}
      {\jcomp{A}{{A}{f}{g}}{R}}
      {\jcomp{A}{f}{{B}{g}{R}}}
    }
  \\
& \inference
    { \jhom{\Gamma}{A}{B}{f}
      \jhom{\Gamma}{B}{C}{g}
      \jterm{{\Gamma}{C}}{R}{h}
      }
    { \jtermeq
        {{\Gamma}{A}}
        {\jcomp{A}{{A}{f}{g}}{R}}
        {\jcomp{A}{{A}{f}{g}}{h}}
        {\jcomp{A}{f}{{B}{g}{h}}}
      }
\end{align*}
\end{lem}

\begin{proof}
Consider family morphisms $\jhom{\Gamma}{A}{B}{f}$ and $\jhom{\Gamma}{B}{C}{g}$
and a family $\jfam{{{\Gamma}{A}}{\ctxwk{A}{C}}}{R}$. Then we have the judgmental
equalities\begin{align*}
\subst{\jcomp{A}{f}{g}}{R} 
& \jdeq 
  \subst{{f}{\ctxwk{A}{g}}}{R}
  \tag{by definition}
  \\
& \jdeq 
  \subst{{f}{\ctxwk{A}{g}}}{\subst{f}{\ctxwk{{A}{B}}{R}}}
  \tag{by \autoref{cancellation-ws-f}}
  \\
& \jdeq 
  \subst{f}{{\ctxwk{A}{g}}{\ctxwk{{A}{B}}{R}}}.
  \tag{by \autoref{comp-ss-f}}
\end{align*}
The proof that 
$\subst{\jcomp{A}{f}{g}}{h}\jdeq\subst{f}{{\ctxwk{A}{g}}{\ctxwk{{A}{B}}{h}}}$
is similar.

Now suppose that $\jfam{{\Gamma}{C}}{R}$ instead. Then we have\begin{align*}
\jcomp{A}{{A}{f}{g}}{R} 
& \jdeq 
  \subst{\jcomp{A}{f}{g}}{\ctxwk{A}{R}}
  \tag{by definition}
  \\
& \jdeq 
  \subst{{f}{\ctxwk{A}{g}}}{\ctxwk{A}{R}}
  \tag{by definition}
  \\
& \jdeq
  \subst{{f}{\ctxwk{A}{g}}}{{f}{\ctxwk{{A}{B}}{{A}{R}}}}
  \tag{by \autoref{cancellation-ws-t}}
  \\
& \jdeq 
  \subst{f}{{\ctxwk{A}{g}}{\ctxwk{{A}{B}}{{A}{R}}}}
  \tag{by \autoref{comp-ss-f}}
  \\
& \jdeq 
  \subst{f}{{\ctxwk{A}{g}}{\ctxwk{A}{{B}{R}}}}
  \tag{by \autoref{comp-ww-f}}
  \\
& \jdeq 
  \subst{f}{\ctxwk{A}{\subst{g}{\ctxwk{B}{R}}}}
  \tag{by \autoref{comp-ws-f}}
  \\
& \jdeq 
  \subst{f}{\ctxwk{A}{\jcomp{B}{g}{R}}}
  \tag{by definition}
  \\
& \jdeq 
  \jcomp{A}{f}{{B}{g}{R}}.
  \tag{by definition}
\end{align*}
Again, the proof is similar for terms $h$ of $R$ in context $\ctxext{\Gamma}{C}$.
\end{proof}

We established already in \autoref{lem:jcomp-wk} that composition is compatible
with weakening. However, in that lemma we did not consider the possibility
of weakening a composition and neither did we consider the possibility of
composition by or with a constant morphism. We do this in the following lemma.

\begin{lem}\label{lem:jcomp-const}
The inference rule
\begin{align*}
& \inference
  { \jhom{\Gamma}{A}{B}{f}
    \jhom{\Gamma}{B}{C}{g}
    \jfam{{\Gamma}{A}}{P}
    }
  { \jhomeq
      {\Gamma}
      {{A}{P}}
      {C}
      {\ctxwk{P}{\jcomp{A}{f}{g}}}
      {\jcomp{{A}{P}}{\ctxwk{P}{f}}{g}}
    }
\end{align*}
tells what happens when we weaken a composition, which is a term of 
$\ctxwk{A}{C}$, by a family $P$ over $\ctxext{\Gamma}{P}$. We also have the following
inference rules expressing that compositions by or with constant morphisms are
again constant morphisms.
\begin{align*}
& \inference
  { \jterm{\Gamma}{B}{y}
    \jhom{\Gamma}{B}{C}{g}
    }
  { \jhomeq
      {\Gamma}
      {A}
      {C}
      {\jcomp{A}{\ctxwk{A}{y}}{g}}
      {\ctxwk{A}{\subst{y}{g}}}
    }
  \\
& \inference
  { \jhom{\Gamma}{A}{B}{f}
    \jterm{\Gamma}{C}{z}
    }
  { \jhomeq
      {\Gamma}
      {A}
      {C}
      {\jcomp{A}{f}{\ctxwk{B}{z}}}
      {\ctxwk{A}{z}}
    }
\end{align*}
\end{lem}

\begin{proof}
Let $\jhom{\Gamma}{A}{B}{f}$, $\jhom{\Gamma}{B}{C}{g}$ and $\jfam{{\Gamma}{A}}{P}$.
Then we have the judgmental equalities\begin{align*}
\ctxwk{P}{\jcomp{A}{f}{g}} 
& \jdeq 
  \ctxwk{P}{\subst{f}{\ctxwk{A}{g}}}
  \tag{by definition}
  \\
& \jdeq 
  \subst{\ctxwk{P}{f}}{\ctxwk{P}{{A}{g}}}
  \tag{by \autoref{comp-ws-t}}
  \\
& \jdeq 
  \subst{\ctxwk{P}{f}}{\ctxwk{\ctxext{A}{P}}{g}}
  \tag{by \autoref{comp-ew-t}}
  \\
& \jdeq 
  \jcomp{{A}{P}}{\ctxwk{P}{f}}{g}.
  \tag{by definition}
\end{align*}
Now let $\jterm{\Gamma}{B}{y}$ and $\jhom{\Gamma}{B}{C}{g}$. Then we have the
judgmental equalities\begin{align*}
\jcomp{A}{\ctxwk{A}{y}}{g}
& \jdeq 
  \subst{\ctxwk{A}{y}}{\ctxwk{A}{g}}
  \tag{by definition}
  \\
& \jdeq 
  \ctxwk{A}{\subst{y}{g}}.
  \tag{by \autoref{comp-ws-t}}
\end{align*}
For the third assertion, let $\jhom{\Gamma}{A}{B}{f}$ and $\jterm{\Gamma}{C}{z}$.
Then we have the judgmental equalities\begin{align*}
\jcomp{A}{f}{\ctxwk{B}{z}} 
& \jdeq 
  \subst{f}{\ctxwk{A}{{B}{z}}}
  \tag{by definition}
  \\
& \jdeq 
  \subst{f}{\ctxwk{{A}{B}}{{A}{z}}}
  \tag{by \autoref{comp-ww-t}}
  \\
& \jdeq 
  \ctxwk{A}{z}.
  \tag{by \autoref{cancellation-ws-t}}
\end{align*}
\end{proof}

Likewise, we can substitute a term $\jterm{\Gamma}{A}{x}$ in a composed morphism
$\jhom{\Gamma}{A}{C}{\jcomp{A}{f}{g}}$. What we get is the \emph{value}
$\subst{{x}{f}}{g}$. This is the content of one part of the following lemma.

There is a second, related case where we can substitute: when
we consider morphisms $\jhom{{\Gamma}{A}}{P}{Q}{f}$ and 
$\jhom{{\Gamma}{A}}{Q}{R}{g}$ we can restrict them to the fibers, obtaining
the inference rule
\begin{equation*}
\inference
  { \jhom{{\Gamma}{A}}{P}{Q}{f}
    \jhom{{\Gamma}{A}}{Q}{R}{g}
    \jterm{\Gamma}{A}{x}
    }
  { \jhomeq
      {\Gamma}
      {\subst{x}{P}}
      {\subst{x}{R}}
      {\subst{x}{\jcomp{P}{f}{g}}}
      {\jcomp{\subst{x}{P}}{\subst{x}{f}}{\subst{x}{g}}}
    }
\end{equation*}
In the following lemma we prove the validity of a more general version of this
inference rule.

\begin{lem}\label{lem:jcomp-fiber}
The following inference rules are valid:
\begin{align*}
& \inference
  { \jhom{{\Gamma}{A}}{P}{Q}{f}
    \jfam{{{\Gamma}{A}}{Q}}{R}
    \jterm{\Gamma}{A}{x}
    }
  { \jfameq
      {{\Gamma}{\subst{x}{P}}}
      {\subst{x}{\jcomp{P}{f}{R}}}
      {\jcomp{\subst{x}{P}}{\subst{x}{f}}{\subst{x}{R}}}
    }
  \\
& \inference
  { \jhom{{\Gamma}{A}}{P}{Q}{f}
    \jfam{{{{\Gamma}{A}}{Q}}{R}}{S}
    \jterm{\Gamma}{A}{x}
    }
  { \jfameq
      {{{\Gamma}{\subst{x}{P}}}{\subst{x}{\jcomp{P}{f}{R}}}}
      {\subst{x}{\jcomp{P}{f}{S}}}
      {\jcomp{\subst{x}{P}}{\subst{x}{f}}{\subst{x}{S}}}
    }
  \\
& \inference
  { \jhom{{\Gamma}{A}}{P}{Q}{f}
    \jterm{{{{\Gamma}{A}}{Q}}{R}}{S}{k}
    \jterm{\Gamma}{A}{x}
    }
  { \jtermeq
      {{{\Gamma}{\subst{x}{P}}}{\subst{x}{\jcomp{P}{f}{R}}}}
      {\subst{x}{\jcomp{P}{f}{S}}}
      {\subst{x}{\jcomp{P}{f}{k}}}
      {\jcomp{\subst{x}{P}}{\subst{x}{f}}{\subst{x}{k}}}
    }
\end{align*}
Also, the following inference rule computing the value at $x$ of a composed
morphism is valid:
\begin{align*}
& \inference
  { \jhom{\Gamma}{A}{B}{f}
    \jhom{\Gamma}{B}{C}{g}
    \jterm{\Gamma}{A}{x}
    }
  { \jtermeq
      {\Gamma}
      {C}
      {\subst{x}{\jcomp{A}{f}{g}}}
      {\subst{{x}{f}}{g}}
    }
\end{align*}
\end{lem}

\begin{proof}
Of the first three inference rules, we only prove the first.
Let $\jhom{{\Gamma}{A}}{P}{Q}{f}$, $\jfam{{{\Gamma}{A}}{Q}}{R}$ and 
$\jterm{\Gamma}{A}{x}$.
Then we have the judgmental equalities\begin{align*}
\subst{x}{\jcomp{A}{f}{R}}
& \jdeq 
  \subst{x}{{f}{\ctxwk{P}{R}}}
  \tag{by definition}
  \\
& \jdeq 
  \subst{{x}{f}}{{x}{\ctxwk{P}{R}}}
  \tag{by \autoref{comp-ss-f}}
  \\
& \jdeq 
  \subst{{x}{f}}{\ctxwk{\subst{x}{P}}{\subst{x}{R}}}
  \tag{by \autoref{comp-sw-f}}
  \\
& \jdeq 
  \jcomp{\subst{x}{P}}{\subst{x}{f}}{\subst{x}{R}}.
  \tag{by definition}
\end{align*}
Now let $\jhom{\Gamma}{A}{B}{f}$, $\jhom{\Gamma}{B}{C}{g}$ and $\jterm{\Gamma}{A}{x}$.
Then we have the judgmental equalities\begin{align*}
\subst{x}{\jcomp{A}{f}{g}}
& \jdeq 
  \subst{x}{{f}{\ctxwk{A}{g}}}
  \tag{by definition}
  \\
& \jdeq 
  \subst{{x}{f}}{{x}{\ctxwk{A}{g}}}
  \tag{by \autoref{comp-ss-t}}
  \\
& \jdeq 
  \subst{{x}{f}}{g}.
  \tag{by \autoref{cancellation-ws-t}}
\end{align*}
\end{proof}

We end this subsection with a simple application of the above lemmas that is
illustrative for the more general formulation of composition we have
introduced. It is a special case of a consecutive composition where we consider
a morphism $\jhom{\Gamma}{A}{B}{f}$ and a morphism
$\jhom{{{\Gamma}{B}}{Q}}{R}{S}{g}$.

\begin{lem}\label{lem:jcomp-higherjcomp}
Let $\jhom{\Gamma}{A}{B}{f}$ and $\jhom{{{\Gamma}{B}}{Q}}{R}{S}{g}$ be context
morphisms. Then the following inference rules are valid:
\begin{align*}
& \inference
  { \jfam{{{{\Gamma}{B}}{Q}}{S}}{T}
    }
  { \jfameq
      {{{{\Gamma}{A}}{\jcomp{A}{f}{Q}}}{\jcomp{A}{f}{R}}}
      {\jcomp{A}{f}{\jcomp{R}{g}{T}}}
      {\jcomp{\jcomp{A}{f}{R}}{\jcomp{A}{f}{g}}{\jcomp{A}{f}{T}}}
    }  
  \\
& \inference
  { \jfam{{{{{\Gamma}{B}}{Q}}{S}}{T}}{U}
    }
  { \jfameq
      {{{{{\Gamma}{A}}{\jcomp{A}{f}{Q}}}{\jcomp{A}{f}{R}}}{\jcomp{A}{f}{\jcomp{R}{g}{T}}}}
      {\jcomp{A}{f}{\jcomp{R}{g}{U}}}
      {\jcomp{\jcomp{A}{f}{R}}{\jcomp{A}{f}{g}}{\jcomp{A}{f}{U}}}
    }  
  \\
& \inference
  { \jterm{{{{{\Gamma}{B}}{Q}}{S}}{T}}{U}{m}
    }
  { \jtermeq
      {{{{{\Gamma}{A}}{\jcomp{A}{f}{Q}}}{\jcomp{A}{f}{R}}}{\jcomp{A}{f}{\jcomp{R}{g}{T}}}}
      {\jcomp{A}{f}{\jcomp{R}{g}{U}}}
      {\jcomp{A}{f}{\jcomp{R}{g}{m}}}
      {\jcomp{\jcomp{A}{f}{R}}{\jcomp{A}{f}{g}}{\jcomp{A}{f}{m}}}
    }  
\end{align*}
\end{lem}

\begin{proof}
We only prove the first inference rule. Let $\jfam{{{{\Gamma}{B}}{Q}}{S}}{T}$ be
a family. Then we have the judgmental equalities
\begin{align*}
\jcomp{A}{f}{\jcomp{R}{g}{T}}
& \jdeq
  \jcomp{A}{f}{\subst{g}{\ctxwk{R}{T}}}
  \tag{by definition}
  \\
& \jdeq
  \subst{\jcomp{A}{f}{g}}{\jcomp{A}{f}{\ctxwk{R}{T}}}
  \tag{by \autoref{lem:jcomp-subst}}
  \\
& \jdeq
  \subst{\jcomp{A}{f}{g}}{\ctxwk{\jcomp{A}{f}{R}}{\jcomp{A}{f}{T}}}
  \tag{by \autoref{lem:jcomp-wk}}
  \\
& \jdeq
  \jcomp{\jcomp{A}{f}{R}}{\jcomp{A}{f}{g}}{\jcomp{A}{f}{T}}.
  \tag{by definition}
\end{align*}
\end{proof}

\begin{lem}
The following inference rule is valid
\begin{equation*}
\inference
  { \jhom{\Gamma}{A}{B}{f}
    }
  { \jhomeq{\Gamma}{A}{B}{\jcomp{A}{f}{\idtm{B}}}{f}
    }
\end{equation*}
\end{lem}

\begin{proof}
We have the judgmental equalities
\begin{align*}
\jcomp{A}{f}{\idtm{B}}
& \jdeq
  \subst{f}{\ctxwk{A}{\idtm{B}}}
  \tag{by definition}
  \\
& \jdeq
  \subst{f}{\idtm{\ctxwk{A}{B}}}
  \tag{by \autoref{comp-wi-t}}
  \\
& \jdeq
  f.
  \tag{by \autoref{cancellation-si}}
\end{align*}
\end{proof}

Because composition is proven to be associative and because identity morphisms
are neutral for both pre- and postcomposition, we can draw diagrams involving
families and morphisms between them in a context $\Gamma$, just as is usual in
category theory and the order in which the morphisms are composed has no
relevance. We can also speak about pullback diagrams, with universality in a
judgmental sense. In \autoref{sec:fhom} we will introduce a more general kind
of diagrams which may also include families and `morphisms over morphisms'.

%%%%%%%%%%%%%%%%%%%%%%%%%%%%%%%%%%%%%%%%%%%%%%%%%%%%%%%%%%%%%%%%%%%%%%%%%%%%%%%%
\subsection{Projections and extension on terms}\label{extension-on-terms}
In this subsection we consider the notion of extension on terms, which has now
become definable inside our theory. Moreover, every compatibility rule one may
dream of is provable as well, using the compatibility rules we have introduced
earlier.

\begin{defn}
When $\jterm{\Gamma}{A}{x}$ and $\jterm{\Gamma}{\subst{x}{P}}{u}$ are terms,
we define 
\begin{equation*}
\jtermdefn
  {\Gamma}
  {\ctxext{A}{P}}
  {\tmext{A}{P}{x}{u}}
  {\unfold{\tmext{A}{P}{x}{u}}}.
\end{equation*} 
\end{defn}

Thus, the term $\tmext{A}{P}{x}{u}$ is the pairing of $x$ and $u$. Note that because
we have the judgmental equality 
$\ctxwk{P}{{A}{\ctxext{A}{P}}}\jdeq\ctxwk{\ctxext{A}{P}}{\ctxext{A}{P}}$ in the
context $\ctxext{{\Gamma}{A}}{P}$, the
pairing function could just be defined as $\idtm{\ctxext{A}{P}}$. 

When we substitute by an extended term we get an equal result as when we
substitute two consecutive times, like the way currying works.

\begin{lem}\label{comp-es}
The following inference rules are valid:
\begin{align*}
& \inference
  { \jterm{\Gamma}{A}{x}
    \jterm{\Gamma}{\subst{x}{P}}{u}
    \jfam{{{\Gamma}{A}}{P}}{Q}
    }
  { \jfameq
      {\Gamma}
      {\subst{\tmext{A}{P}{x}{u}}{Q}}
      {\subst{u}{{x}{Q}}}
    }
  \\
& \inference
  { \jterm{\Gamma}{A}{x}
    \jterm{\Gamma}{\subst{x}{P}}{u}
    \jfam{{{{\Gamma}{A}}{P}}{Q}}{R}
    }
  { \jfameq
      {{\Gamma}{\subst{u}{{x}{Q}}}}
      {\subst{\tmext{A}{P}{x}{u}}{R}}
      {\subst{u}{{x}{R}}}
    }
  \\
& \inference
  { \jterm{\Gamma}{A}{x}
    \jterm{\Gamma}{\subst{x}{P}}{u}
    \jterm{{{{\Gamma}{A}}{P}}{Q}}{R}{t}
    }
  { \jtermeq
      {{\Gamma}{\subst{u}{{x}{Q}}}}
      {\subst{u}{{x}{R}}}
      {\subst{\tmext{A}{P}{x}{u}}{h}}
      {\subst{u}{{x}{h}}}
    }
\end{align*}
\end{lem}

\begin{proof}
We prove only the first judgmental equality. All the others are similar.
Let $\jterm{\Gamma}{A}{x}$ and $\jterm{\Gamma}{\subst{x}{P}}{u}$
be terms and let $\jfam{{{\Gamma}{A}}{P}}{Q}$ be a family. Then we have
\begin{align*}
\subst
  {\tmext{A}{P}{x}{u}}
  {Q} 
& \jdeq 
  \subst
    {{u}{{x}{\idtm{\ctxext{A}{P}}}}}
    {Q}
  \tag{by definition}\\
& \jdeq 
  \subst
    {{u}{{x}{\idtm{\ctxext{A}{P}}}}}
    {{x}{\ctxwk{A}{Q}}}
  \tag{by \autoref{cancellation-ws-f}}\\
& \jdeq 
  \subst
    {{u}{{x}{\idtm{\ctxext{A}{P}}}}}
    {{u}{\ctxwk{\subst{x}{P}}{\subst{x}{\ctxwk{A}{Q}}}}}
  \tag{by \autoref{cancellation-ws-f}}\\
& \jdeq 
  \subst
    {{u}{{x}{\idtm{\ctxext{A}{P}}}}}
    {{u}{{x}{\ctxwk{P}{{A}{Q}}}}}
  \tag{by \autoref{comp-sw-f}}\\
& \jdeq 
  \subst
    {u}
    {{{x}{\idtm{\ctxext{A}{P}}}}{{x}{\ctxwk{{P}{{A}{Q}}}}}}
  \tag{by \autoref{comp-ss-f}}\\
& \jdeq 
  \subst
    {u}
    {{x}{{\idtm{\ctxext{A}{P}}}{\ctxwk{P}{{A}{Q}}}}}
  \tag{by \autoref{comp-ss-f}}\\
& \jdeq 
  \subst
    {u}
    {{x}{{\idtm{\ctxext{A}{P}}}{\ctxwk{\ctxext{A}{P}}{Q}}}}
  \tag{by \autoref{comp-ew-f}}\\
& \jdeq 
  \subst
    {u}
    {{x}{Q}}
  \tag{by \autoref{precomp-idtm-c}}
\end{align*}
\end{proof}

We have seen above that the pairing function into $\ctxext{A}{P}$ is just the identity term on
$\ctxext{A}{P}$. To analyze the pairing functin a little further, we will also
need the projection maps from $\ctxext{A}{P}$ to $A$ and from $\ctxext{A}{P}$
to $P$. We will now define these and see that the identity term of an
extended family is the extension (or pairing) of the identity
functions on the components in the apropriate way.

To find out what the
apropriate way is, note that
\begin{align*}
\ctxwk{\ctxext{A}{P}}{\ctxext{A}{P}}
& \jdeq
  \ctxext{\ctxwk{\ctxext{A}{P}}{A}}{\ctxwk{\ctxext{A}{P}}{P}}
  \tag{by \autoref{comp-we-f}}
  \\
& \jdeq
  \ctxext{\ctxwk{P}{{A}{A}}}{\ctxwk{P}{{A}{P}}}
  \tag{by \autoref{comp-ew-f}}
\end{align*}
We have the term $\jterm{{\Gamma}{{A}{P}}}{\ctxwk{P}{A}}{\ctxwk{P}{\idtm{A}}}$.
Thus we need to find out what $\subst{\ctxwk{P}{\idtm{A}}}{\ctxwk{P}{{A}{P}}}$ is:
\begin{align*}
\subst{\ctxwk{P}{\idtm{A}}}{\ctxwk{P}{{A}{P}}} 
& \jdeq 
  \ctxwk{P}{\subst{\idtm{A}}{\ctxwk{A}{P}}}
  \tag{by \autoref{comp-ws-f}}
  \\
& \jdeq 
  \ctxwk{P}{P},
  \tag{by \autoref{precomp-idtm-c}}
\end{align*}
where we find the term $\idtm{P}$. Therefore we define:

\begin{defn}
Let $\jfam{\Gamma}{A}$ and $\jfam{{\Gamma}{A}}{P}$ be families. We define
\begin{align*}
\jalign\jhomdefn
  {\Gamma}
  {{A}{P}}
  {A}
  {\cprojfstf{A}{P}}
  {\unfold{\cprojfstf{A}{P}}}
  \\
\jalign\jtermdefn
  {\ctxext{\Gamma}{{A}{P}}}
  {\ctxwk{P}{P}}
  {\cprojsndf{A}{P}}
  {\unfold{\cprojsndf{A}{P}}}
\end{align*}
\end{defn}

The constructions of the terms $\tmext{A}{P}{x}{u}$ and $\cprojfst{A}{P}{w}$ and
$\cprojsnd{A}{P}{w}$ are subject to various rules, with all of them being
consequences of earlier introduced inference rules.

\begin{lem}\label{lem:tmext-basic}
The following inference rules expressing that pairing is a strict
inverse to the combination of decompositions, are valid:
\begin{align*}
& \inference
  { \jterm{\Gamma}{\ctxext{A}{P}}{w}
    }
  { \jtermeq
      {\Gamma}
      {\ctxext{A}{P}}
      {\tmext{A}{P}{\cprojfst{A}{P}{w}}{\cprojsnd{A}{P}{w}}}
      {w}
    }
  \\
& \inference
  { \jterm{\Gamma}{A}{x}
    \jterm{\Gamma}{\subst{x}{P}}{u}
    }
  { \jtermeq
      {\Gamma}
      {A}
      {\cprojfst{A}{P}{\tmext{A}{P}{x}{u}}}
      {x}
    }
  \\
& \inference
  { \jterm{\Gamma}{A}{x}
    \jterm{\Gamma}{\subst{x}{P}}{u}
    }
  { \jtermeq
      {\Gamma}
      {\subst{x}{P}}
      {\cprojsnd{A}{P}{\tmext{A}{P}{x}{u}}}
      {u}
    }
\end{align*}
\end{lem}

\begin{proof}
To prove the first judgmental equality, note that
\begin{align*}
w 
& \jdeq 
  \subst{w}{\idtm{\ctxext{A}{P}}} 
  \tag{by \autoref{cancellation-si}}\\
& \jdeq 
  \subst
    { w}
    { { \idtm{P}}
      { { \ctxwk{P}{\idtm{A}}}
        { \idtm{\ctxwk{\ctxext{A}{P}}{\ctxext{A}{P}}}}
        }
      }
  \tag{by \autoref{comp-ie}}\\
& \jdeq 
  \subst
    { {w}
      {\idtm{P}}
      }
    { {w}
      { { \ctxwk{P}{\idtm{A}}
          }
        { \idtm{\ctxwk{\ctxext{A}{P}}{\ctxext{A}{P}}}
          }
        }
      }
  \tag{by \autoref{comp-ss-t}}\\
& \jdeq 
  \subst
    { {w}
      {\idtm{P}}
      }
    { { {w}
        {\ctxwk{P}{\idtm{A}}}
        }
      { {w}
        {\idtm{\ctxwk{\ctxext{A}{P}}{\ctxext{A}{P}}}}
        }
      }
  \tag{by \autoref{comp-ss-t}}\\
& \jdeq 
  \subst
    { {w}
      {\idtm{P}}
      }
    { { {w}
        {\ctxwk{P}{\idtm{A}}}
        }
      { {w}
        {\ctxwk{\ctxext{A}{P}}{\idtm{\ctxext{A}{P}}}}
        }
      }
  \tag{by \autoref{comp-wi-t}}\\
& \jdeq 
  \subst
    { {w}
      {\idtm{P}}
      }
    { { {w}
        {\ctxwk{P}{\idtm{A}}}
        }
      { \idtm{\ctxext{A}{P}}
        }
      }
  \tag{by \autoref{cancellation-ws-t}}\\
& \jdeq 
  \tmext{A}{P}{\cprojfst{A}{P}{w}}{\cprojsnd{A}{P}{w}}
  \tag{by definition}
\end{align*}
To prove the second judgmental equality, let $\jterm{\Gamma}{A}{x}$ and
$\jterm{\Gamma}{\subst{x}{P}}{u}$. Then we have
\begin{align*}
\cprojfst{A}{P}{\tmext{A}{P}{x}{u}}
& \jdeq 
  \subst{\tmext{A}{P}{x}{u}}{\ctxwk{P}{\idtm{A}}}
  \tag{by definition}
  \\
& \jdeq 
  \subst{u}{{x}{\ctxwk{P}{\idtm{A}}}} 
  \tag{by \autoref{comp-es}}
  \\
& \jdeq 
  \subst{u}{\ctxwk{\subst{x}{P}}{\subst{x}{\idtm{A}}}}
  \tag{by \autoref{comp-sw-t}}
  \\
& \jdeq 
  \subst{x}{\idtm{A}}
  \tag{by \autoref{cancellation-ws-t}}
  \\
& \jdeq 
  x.
  \tag{by \autoref{cancellation-si}}
\end{align*}
To prove the third judgmental equality, note that
\begin{align*}
\cprojsnd{A}{P}{\tmext{A}{P}{x}{u}}
& \jdeq 
  \subst{\tmext{A}{P}{x}{u}}{\idtm{P}}
  \tag{by definition}
  \\
& \jdeq 
  \subst{u}{{x}{\idtm{P}}}
  \tag{by \autoref{comp-es}}
  \\
& \jdeq 
  \subst{u}{\idtm{\subst{x}{P}}}
  \tag{by \autoref{comp-si-t}}
  \\
& \jdeq 
  u.
  \tag{by \autoref{cancellation-si}}
\end{align*}
\end{proof}

In \autoref{lem:tmext-ext,lem:tmext-wk,lem:tmext-subst,comp-ie,lem:tmext-jcomp}
we show that term extension and the projections are
compatible with extension (\autoref{lem:tmext-tmext}),
weakening (\autoref{lem:tmext-wk}), substitution (\autoref{lem:tmext-subst}), 
the identity terms (\autoref{comp-ie}) and with precomposition (\autoref{lem:tmext-jcomp}). 
The property that term extension and the projections are compatible with identity terms
is in fact a generalization of the above \autoref{lem:tmext-basic}. In
\autoref{lem:tmext-emp} we will also show that term extension and the projections
are compatible with the empty object.

\begin{lem}\label{lem:tmext-ext}\label{lem:tmext-tmext}
The following compatibility rules for two consecutive term extensions are valid:
\begin{align*}
& \inference
  { \jterm{\Gamma}{A}{x}
    \jterm{\Gamma}{\subst{x}{P}}{u}
    \jterm{\Gamma}{\subst{\tmext{A}{P}{x}{u}}{Q}}{v}
    }
  { \jtermeq
      {\Gamma}
      {\ctxext{{A}{P}}{Q}}
      {\tmext{A}{{P}{Q}}{x}{{\subst{x}{P}}{\subst{x}{Q}}{u}{v}}}
      {\tmext{{A}{P}}{Q}{{A}{P}{x}{u}}{v}}
    }
  \\
& \inference
  { \jfam{{{\Gamma}{A}}{P}}{Q}
    }
  { \jhomeq
      {\Gamma}
      {\ctxext{{A}{P}}{Q}}
      {A}
      {\jcomp{}{\cprojfstf{{A}{P}}{Q}}{\cprojfstf{A}{P}}}
      {\cprojfstf{A}{{P}{Q}}}
    }
  \\
& \inference
  { \jfam{{{\Gamma}{A}}{P}}{Q}
    }
  { \jhomeq
      {{\Gamma}{A}}
      {{P}{Q}}
      {P}
      {\jcomp{}{\cprojfstf{{A}{P}}{Q}}{\cprojsndf{A}{P}}}
      {\jcomp{}{\cprojsndf{A}{{P}{Q}}}{\cprojfstf{P}{Q}}}
    }
  \\
& \inference
    { \jfam{{{\Gamma}{A}}{P}}{Q}
      }
    { \jhomeq
        {{{\Gamma}{A}}{P}}
        {Q}
        {Q}
        {\cprojsndf{{A}{P}}{Q}}
        {\jcomp{}{\cprojsndf{A}{{P}{Q}}}{\cprojsndf{P}{Q}}}
      }
\end{align*}
\end{lem}

\begin{proof}
Consider terms $\jterm{\Gamma}{A}{x}$, $\jterm{\Gamma}{\subst{x}{P}}{u}$ and
$\jterm{\Gamma}{\subst{u}{{x}{Q}}}{v}$. Then we have
\begin{align*}
\tmext{x}{{u}{v}}
& \jdeq 
  \subst
    {\tmext{u}{v}}{{x}{\idtm{\ctxext{A}{{P}{Q}}}}}
  \tag{by \autoref{comp-es}}
  \\
& \jdeq 
  \subst{v}{{u}{{x}{\idtm{\ctxext{A}{{P}{Q}}}}}}
  \tag{by \autoref{comp-es}}
  \\
& \jdeq 
  \subst{v}{{u}{{x}{\idtm{\ctxext{{A}{P}}{Q}}}}}
  \tag{by \autoref{comp-ee-c}}
  \\
& \jdeq 
  \subst{v}{{\tmext{x}{u}}{\idtm{\ctxext{{A}{P}}{Q}}}}
  \tag{by \autoref{comp-es}}
  \\
& \jdeq 
  \tmext{{x}{u}}{v}.
  \tag{by \autoref{comp-es}}
\end{align*}
To prove the judmental equality
\begin{equation*}
\jhomeq
  {\Gamma}
  {\ctxext{{A}{P}}{Q}}
  {A}
  {\jcomp{}{\cprojfstf{{A}{P}}{Q}}{\cprojfstf{A}{P}}}
  {\cprojfstf{A}{{P}{Q}}}
\end{equation*}
note that we have the judgmental equalities
\begin{align*}
\jcomp{{{A}{P}}{Q}}{\cprojfstf{{A}{P}}{Q}}{\cprojfstf{A}{P}}
& \jdeq 
  \unfoldall{\jcomp{{{A}{P}}{Q}}{\cprojfstf{{A}{P}}{Q}}{\cprojfstf{A}{P}}}
  \tag{by definition}
  \\
& \jdeq 
  \subst
    {\ctxwk{Q}{\idtm{{A}{P}}}}
    {\ctxwk{Q}{{\ctxext{A}{P}}{{P}{\idtm{A}}}}}
  \tag{by \autoref{comp-ew-t}}
  \\
& \jdeq
  \ctxwk{Q}{\subst{\idtm{{A}{P}}}{\ctxwk{\ctxext{A}{P}}{{P}{\idtm{A}}}}}
  \tag{by \autoref{comp-ws-t}}
  \\
& \jdeq
  \ctxwk{Q}{{P}{\idtm{A}}}
  \tag{by \autoref{precomp-idtm-t}}
  \\
& \jdeq
  \ctxwk{\ctxext{P}{Q}}{\idtm{A}}
  \tag{by \autoref{comp-ew-t}}
  \\
& \jdeq
  \cprojfstf{A}{{P}{Q}}
  \tag{by definition}
\end{align*}
To prove the judgmental equality
\begin{equation*}
\jhomeq
  {{\Gamma}{A}}
  {{P}{Q}}
  {P}
  {\jcomp{}{\cprojfstf{{A}{P}}{Q}}{\cprojsndf{A}{P}}}
  {\jcomp{}{\cprojsndf{A}{{P}{Q}}}{\cprojfstf{P}{Q}}}
\end{equation*}
note that we have the judgmental equalities
\begin{align*}
\jcomp{{{A}{P}}{Q}}{\cprojfstf{{A}{P}}{Q}}{\cprojsndf{A}{P}}
& \jdeq 
  \unfoldall{\jcomp{{{A}{P}}{Q}}{\cprojfstf{{A}{P}}{Q}}{\cprojsndf{A}{P}}}
  \tag{by definition}
  \\
& \jdeq
  \subst{\ctxwk{Q}{\idtm{{A}{P}}}}{\ctxwk{Q}{{\ctxext{A}{P}}{\idtm{P}}}}
  \tag{by \autoref{comp-ew-t}}
  \\
& \jdeq
  \ctxwk{Q}{\subst{\idtm{{A}{P}}}{\ctxwk{\ctxext{A}{P}}{\idtm{P}}}}
  \tag{by \autoref{comp-ws-t}}
  \\
& \jdeq
  \ctxwk{Q}{\idtm{P}}
  \tag{by \autoref{precomp-idtm-t}}
  \\
& \jdeq
  \unfoldall{\jcomp{{P}{Q}}{\cprojsndf{A}{{P}{Q}}}{\cprojfstf{P}{Q}}}
  \tag{by \autoref{precomp-idtm-t}}
  \\
& \jdeq
  \jcomp{{P}{Q}}{\cprojsndf{A}{{P}{Q}}}{\cprojfstf{P}{Q}}
  \tag{by definition}
\end{align*}
To prove the judgmental equality
\begin{equation*}
\jhomeq
  {{{\Gamma}{A}}{P}}
  {Q}
  {Q}
  {\cprojsndf{{A}{P}}{Q}}
  {\jcomp{}{\cprojsndf{A}{{P}{Q}}}{\cprojsndf{P}{Q}}}
\end{equation*}
note that we have the judgmental equalities
\begin{align*}
\cprojsndf{{A}{P}}{Q}
& \jdeq
  \unfoldall{\cprojsndf{{A}{P}}{Q}}
  \tag{by definition}
  \\
& \jdeq 
  \unfoldall{\jcomp{{P}{Q}}{\cprojsndf{A}{{P}{Q}}}{\cprojsndf{P}{Q}}}
  \tag{by \autoref{precomp-idtm-t}}
  \\
& \jdeq
  \jcomp{}{\cprojsndf{A}{{P}{Q}}}{\cprojsndf{P}{Q}}
  \tag{by definition}
\end{align*}
\end{proof}

\begin{lem}\label{lem:tmext-wk}\label{comp-we-t}
When we weaken a term $\tmext{B}{Q}{y}{v}$ of $\ctxext{B}{Q}$ in context 
$\Gamma$ by a family $A$, the term that we get is 
$\tmext{\ctxwk{A}{B}}{\ctxwk{A}{Q}}{\ctxwk{A}{y}}{\ctxwk{A}{v}}$. More
precisely, the following inference rules are valid:
\begin{align*}
& \inference
  { \jterm{{\Gamma}{B}}{Q}{g}
    \jterm{{\Gamma}{B}}{\subst{g}{R}}{t}
    }
  { \jtermeq
      {{{\Gamma}{A}}{\ctxwk{A}{B}}}
      {\ctxwk{A}{\ctxext{Q}{R}}}
      {\ctxwk{A}{\tmext{Q}{R}{g}{t}}}
      {\tmext{\ctxwk{A}{Q}}{\ctxwk{A}{R}}{\ctxwk{A}{g}}{\ctxwk{A}{t}}}
    }
  \\
& \inference
  { \jfam{{{\Gamma}{B}}{Q}}{R}
    }
  { \jhomeq
      {{{\Gamma}{A}}{\ctxwk{A}{B}}}
      {{\ctxwk{A}{Q}}{\ctxwk{A}{R}}}
      {\ctxwk{A}{Q}}
      {\cprojfstf{\ctxwk{A}{Q}}{\ctxwk{A}{R}}}
      {\ctxwk{A}{\cprojfstf{Q}{R}}}
    }
  \\
& \inference
  { \jfam{{{\Gamma}{B}}{Q}}{R}
    }
  { \jhomeq
      {{{{\Gamma}{A}}{\ctxwk{A}{B}}}{\ctxwk{A}{Q}}}
      {\ctxwk{A}{R}}
      {\ctxwk{A}{R}}
      {\cprojsndf{\ctxwk{A}{Q}}{\ctxwk{A}{R}}}
      {\ctxwk{A}{\cprojsndf{Q}{R}}}
    }
\end{align*}
\end{lem}

\begin{proof}
Consider $\jterm{{\Gamma}{B}}{Q}{g}$ and $\jterm{{\Gamma}{B}}{\subst{g}{R}}{t}$.
Then we have the judgmental equalities
\begin{align*}
\ctxwk{A}{\tmext{Q}{R}{g}{t}}
& \jdeq 
  \ctxwk{A}{\subst{t}{{g}{\idtm{\ctxext{Q}{R}}}}}
  \tag{by definition}
  \\
& \jdeq 
  \subst{\ctxwk{A}{t}}{\ctxwk{A}{\subst{g}{\idtm{\ctxext{Q}{R}}}}}
  \tag{by \autoref{comp-ws-t}}
  \\
& \jdeq 
  \subst{\ctxwk{A}{t}}{{\ctxwk{A}{g}}{\ctxwk{A}{\idtm{\ctxext{Q}{R}}}}}
  \tag{by \autoref{comp-ws-t}}
  \\
& \jdeq 
  \subst{\ctxwk{A}{t}}{{\ctxwk{A}{g}}{\idtm{\ctxwk{A}{\ctxext{Q}{R}}}}}
  \tag{by \autoref{comp-wi-t}}
  \\
& \jdeq 
  \subst{\ctxwk{A}{t}}{{\ctxwk{A}{g}}{\idtm{\ctxext{\ctxwk{A}{Q}}{\ctxwk{A}{R}}}}}
  \tag{by \autoref{comp-we-f}}
  \\
& \jdeq 
  \tmext{\ctxwk{A}{Q}}{\ctxwk{A}{R}}{\ctxwk{A}{g}}{\ctxwk{A}{t}}
  \tag{by definition}
\end{align*}
Next, we want to prove the judgmental equality
\begin{equation*}
\jhomeq
  {{{\Gamma}{A}}{\ctxwk{A}{B}}}
  {{\ctxwk{A}{Q}}{\ctxwk{A}{R}}}
  {\ctxwk{A}{Q}}
  {\cprojfstf{\ctxwk{A}{Q}}{\ctxwk{A}{R}}}
  {\ctxwk{A}{\cprojfstf{Q}{R}}}
\end{equation*}
Note that we have the judgmental equalities
\begin{align*}
\cprojfstf{\ctxwk{A}{Q}}{\ctxwk{A}{R}}
& \jdeq
  \unfoldall{\cprojfstf{\ctxwk{A}{Q}}{\ctxwk{A}{R}}}
  \tag{by definition}
  \\
& \jdeq
  \ctxwk{{A}{R}}{{A}{\idtm{Q}}}
  \tag{by \autoref{comp-wi-t}}
  \\
& \jdeq
  \unfoldall{\ctxwk{A}{\cprojfstf{Q}{R}}}
  \tag{by \autoref{comp-ww-t}}
  \\
& \jdeq
  \ctxwk{A}{\cprojfstf{Q}{R}}.
  \tag{by definition}
\end{align*}
Finally, we want to prove the judgmental equality
\begin{equation*}
\jhomeq
  {{{{\Gamma}{A}}{\ctxwk{A}{B}}}{\ctxwk{A}{Q}}}
  {\ctxwk{A}{R}}
  {\ctxwk{A}{R}}
  {\cprojsndf{\ctxwk{A}{Q}}{\ctxwk{A}{R}}}
  {\ctxwk{A}{\cprojsndf{Q}{R}}}
\end{equation*}
Note that we have the judgmental equalities
\begin{align*}
\cprojsndf{\ctxwk{A}{Q}}{\ctxwk{A}{R}}
& \jdeq
  \unfoldall{\cprojsndf{\ctxwk{A}{Q}}{\ctxwk{A}{R}}}
  \tag{by definition}
  \\
& \jdeq
  \unfoldall{\ctxwk{A}{\cprojsndf{Q}{R}}}
  \tag{by \autoref{comp-wi-t}}
  \\
& \jdeq
  \ctxwk{A}{\cprojsndf{Q}{R}}.
  \tag{by definition}
\end{align*}
\end{proof}

\begin{lem}\label{lem:tmext-subst}\label{comp-se-t}
When we substitute an extended term $\tmext{P}{Q}{f}{g}$ of $\ctxext{P}{Q}$ by a term
$x$ of $A$, the term that we get is $\tmext{\subst{x}{P}}{\subst{x}{Q}}{\subst{x}{f}}{\subst{x}{g}}$.
More precisely, the following inference rules are valid:
\begin{align*}
& \inference
  { \jterm{\Gamma}{A}{x}
    \jterm{{{\Gamma}{A}}{P}}{Q}{g}
    \jterm{{{\Gamma}{A}}{P}}{\subst{g}{R}}{t}
    }
  { \jtermeq
      {{\Gamma}{\subst{x}{P}}}
      {\ctxext{\subst{x}{Q}}{\subst{x}{R}}}
      {\subst{x}{\tmext{Q}{R}{g}{t}}}
      {\tmext{\subst{x}{Q}}{\subst{x}{R}}{\subst{x}{g}}{\subst{x}{t}}}
    }
  \\
& \inference
  { \jterm{\Gamma}{A}{x}
    \jfam{{{{\Gamma}{A}}{P}}{Q}}{R}
    }
  { \jhomeq
      {{\Gamma}{\subst{x}{P}}}
      {{\subst{x}{Q}}{\subst{x}{R}}}
      {\subst{x}{Q}}
      {\cprojfstf{\subst{x}{Q}}{\subst{x}{R}}}
      {\subst{x}{\cprojfstf{Q}{R}}}
    }
  \\
& \inference
  { \jterm{\Gamma}{A}{x}
    \jfam{{{{\Gamma}{A}}{P}}{Q}}{R}
    }
  { \jhomeq
      {{{\Gamma}{\subst{x}{P}}}{\subst{x}{Q}}}
      {\subst{x}{R}}
      {\subst{x}{R}}
      {\cprojsndf{\subst{x}{Q}}{\subst{x}{R}}}
      {\subst{x}{\cprojsndf{Q}{R}}}
    }
\end{align*}
\end{lem}

\begin{proof}
Consider $\jterm{{\Gamma}{B}}{Q}{g}$ and $\jterm{{\Gamma}{B}}{\subst{g}{R}}{t}$.
Then we have the judgmental equalities
\begin{align*}
\subst{x}{\tmext{Q}{R}{g}{t}}
& \jdeq 
  \subst{x}{{t}{{g}{\idtm{\ctxext{Q}{R}}}}}
  \tag{by definition}
  \\
& \jdeq 
  \subst{{x}{t}}{{x}{{g}{\idtm{\ctxext{Q}{R}}}}}
  \tag{by \autoref{comp-ss-t}}
  \\
& \jdeq 
  \subst{{x}{t}}{{{x}{g}}{{x}{\idtm{\ctxext{Q}{R}}}}}
  \tag{by \autoref{comp-ss-t}}
  \\
& \jdeq 
  \subst{{x}{t}}{{{x}{g}}{\idtm{\subst{x}{\ctxext{Q}{R}}}}}
  \tag{by \autoref{comp-si-t}}
  \\
& \jdeq 
  \subst{{x}{t}}{{{x}{g}}{\idtm{\ctxext{\subst{x}{Q}}{\subst{x}{R}}}}}
  \tag{by \autoref{comp-se-f}}
  \\
& \jdeq 
  \tmext{\subst{x}{Q}}{\subst{x}{R}}{\subst{x}{g}}{\subst{x}{t}}.
  \tag{by definition}
\end{align*}
Next, we want to prove the judgmental equality
\begin{equation*}
\jhomeq
  {{\Gamma}{\subst{x}{P}}}
  {{\subst{x}{Q}}{\subst{x}{R}}}
  {\subst{x}{Q}}
  {\cprojfstf{\subst{x}{Q}}{\subst{x}{R}}}
  {\subst{x}{\cprojfstf{Q}{R}}}
\end{equation*}
Note that we have the judgmental equalities
\begin{align*}
\cprojfstf{\subst{x}{Q}}{\subst{x}{R}}
& \jdeq
  \unfoldall{\cprojfstf{\subst{x}{Q}}{\subst{x}{R}}}
  \tag{by definition}
  \\
& \jdeq
  \ctxwk{\subst{x}{R}}{\subst{x}{\idtm{Q}}}
  \tag{by \autoref{comp-si-t}}
  \\
& \jdeq
  \unfoldall{\subst{x}{\cprojfstf{Q}{R}}}
  \tag{by \autoref{comp-ss-t}}
  \\
& \jdeq
  \subst{x}{\cprojfstf{Q}{R}}.
  \tag{by definition}
\end{align*}
And finally we want to prove the judgmental equality
\begin{equation*}
\jhomeq
  {{{\Gamma}{\subst{x}{P}}}{\subst{x}{Q}}}
  {\subst{x}{R}}
  {\subst{x}{R}}
  {\cprojsndf{\subst{x}{Q}}{\subst{x}{R}}}
  {\subst{x}{\cprojsndf{Q}{R}}}
\end{equation*}
Note that we have the judgmental equalities
\begin{align*}
\cprojsndf{\subst{x}{Q}}{\subst{x}{R}}
& \jdeq
  \unfoldall{\cprojsndf{\subst{x}{Q}}{\subst{x}{R}}}
  \tag{by definition}
  \\
& \jdeq
  \unfoldall{\subst{x}{\cprojsndf{Q}{R}}}
  \tag{by \autoref{comp-si-t}}
  \\
& \jdeq
  \subst{x}{\cprojsndf{Q}{R}}.
  \tag{by definition}
\end{align*}
\end{proof}

We find the following inference rule, which expresses that the identity term
is compatible with extension:

\begin{lem}\label{comp-ie}
For any $\jfam{\Gamma}{A}$ and $\jfam{{\Gamma}{A}}{P}$ we have
\begin{equation*}
\inference
  { \jfam{\Gamma}{A}
    \jfam{{\Gamma}{A}}{P}
    }
  { \jhomeq
      {\Gamma}
      {{A}{P}}{{A}{P}}
      {\idtm{\ctxext{A}{P}}}
      { \tmext
          {\ctxwk{\ctxext{A}{P}}{A}}
          {\ctxwk{\ctxext{A}{P}}{P}}
          {\cprojfstf{A}{P}}
          {\cprojsndf{A}{P}}
        }
    }
\end{equation*}
\end{lem}

\begin{proof}
Consider the families $\jfam{\Gamma}{A}$ and $\jfam{{\Gamma}{A}}{P}$. Then
we have the judgmental equalities
\begin{align*}
\tmext
  {\ctxwk{\ctxext{A}{P}}{A}}
  {\ctxwk{\ctxext{A}{P}}{P}}
  {\cprojfstf{A}{P}}
  {\cprojsndf{A}{P}}
& \jdeq 
  \unfold
  { \tmext
      {\ctxwk{\ctxext{A}{P}}{A}}
      {\ctxwk{\ctxext{A}{P}}{P}}
      {\cprojfstf{A}{P}}
      {\cprojsndf{A}{P}}
    }
  \tag{by definition}
  \\
& \jdeq 
  \subst
    { \idtm{P}
      }
    { {\ctxwk{P}{\idtm{A}}}
      {\idtm{\ctxwk{\ctxext{A}{P}}{\ctxext{A}{P}}}}
      }
  \tag{by definition}
  \\
& \jdeq 
  \subst
    { \idtm{P}
      }
    { {\ctxwk{P}{\idtm{A}}}
      {\ctxwk{\ctxext{A}{P}}{\idtm{\ctxext{A}{P}}}}
      }
  \tag{by \autoref{comp-wi-t}}
  \\
& \jdeq 
  \subst
    { \idtm{P}
      }
    { {\ctxwk{P}{\idtm{A}}}
      {\ctxwk{P}{{A}{\idtm{\ctxext{A}{P}}}}}
      }
  \tag{by \autoref{comp-ew-t}}
  \\
& \jdeq 
  \subst
    { \idtm{P}
      }
    {\ctxwk
      {P}
      { \subst
        {\idtm{A}}
        {\ctxwk{A}{\idtm{\ctxext{A}{P}}}}
        }
      }
  \tag{by \autoref{comp-ws-t}}
  \\
& \jdeq 
  \subst
    { \idtm{A}
      }
    { \ctxwk{A}{\idtm{\ctxext{A}{P}}}
      }
  \tag{by \autoref{precomp-idtm-t}}
  \\
& \jdeq 
  \idtm{\ctxext{A}{P}}.
  \tag{by \autoref{precomp-idtm-t}}
\end{align*}
\end{proof}

We also have the following lemma about the compatibility of pairing and composition:

\begin{lem}\label{lem:tmext-jcomp}
The following inference rules are valid
\begin{align*}
& \inference
  { \jhom{\Gamma}{A}{B}{f}
    \jterm{{{\Gamma}{B}}{Q}}{R}{h}
    \jterm{{{\Gamma}{B}}{Q}}{\subst{h}{S}}{k}
    }
  { \jtermeq
      {{{\Gamma}{A}}{\jcomp{A}{f}{Q}}}
      {\jcomp{A}{f}{\ctxext{R}{S}}}
      {\jcomp{A}{f}{\tmext{h}{k}}}
      {\tmext{\jcomp{A}{f}{h}}{\jcomp{A}{f}{k}}}
    }
  \\
& \inference
  { \jhom{\Gamma}{A}{B}{f}
    \jfam{{{{\Gamma}{B}}{Q}}{R}}{S}
    }
  { \jhomeq
      {{{\Gamma}{A}}{\jcomp{A}{f}{Q}}}
      {\jcomp{A}{f}{\ctxext{R}{S}}}
      {\jcomp{A}{f}{R}}
      {\jcomp{A}{f}{\cprojfstf{R}{S}}}
      {\cprojfstf{\jcomp{A}{f}{R}}{\jcomp{A}{f}{S}}}
    }
  \\
& \inference
  { \jhom{\Gamma}{A}{B}{f}
    \jfam{{{{\Gamma}{B}}{Q}}{R}}{S}
    }
  { \jhomeq
      {{{{\Gamma}{A}}{\jcomp{A}{f}{Q}}}{\jcomp{A}{f}{S}}}
      {\jcomp{A}{f}{S}}
      {\jcomp{A}{f}{S}}
      {\jcomp{A}{f}{\cprojsndf{R}{S}}}
      {\cprojsndf{\jcomp{A}{f}{R}}{\jcomp{A}{f}{S}}}
    }
\end{align*}
\end{lem}

\begin{proof}
Let $\jhom{\Gamma}{A}{B}{f}$, $\jterm{{{\Gamma}{B}}{Q}}{R}{h}$
and $\jterm{{{\Gamma}{B}}{Q}}{\subst{h}{S}}{k}$. Then we have the
judgmental equalities
\begin{align*}
\jcomp{A}{f}{\tmext{\ctxwk{B}{C}}{\ctxwk{B}{R}}{g}{h}}
& \jdeq 
  \subst{f}{\ctxwk{A}{\tmext{\ctxwk{B}{C}}{\ctxwk{B}{R}}{g}{h}}}
  \tag{by definition}
  \\
& \jdeq 
  \subst
    {f}
    {\tmext{\ctxwk{A}{{B}{C}}}{\ctxwk{A}{{B}{R}}}{\ctxwk{A}{g}}{\ctxwk{A}{h}}}
  \tag{by \autoref{lem:tmext-wk}}
  \\
& \jdeq 
  \tmext
    {\ctxwk{A}{C}}
    {\ctxwk{A}{R}}
    {\subst{f}{\ctxwk{A}{g}}}
    {\subst{f}{\ctxwk{A}{h}}}
  \tag{by \autoref{lem:tmext-subst}}
  \\
& \jdeq 
  \tmext{\ctxwk{A}{C}}{\ctxwk{A}{R}}{\jcomp{A}{f}{g}}{\jcomp{A}{f}{h}}
  \tag{by definition}
\end{align*}
validating the first inference rule. For the second and the third, note that
we have the judgmental equalities
\begin{align*}
\jcomp{A}{f}{\cprojfstf{R}{S}}
& \jdeq
  \unfold\jcomp{A}{f}{\cprojfstf{R}{S}}
  \tag{by definition}
  \\
& \jdeq
  \subst{f}{\cprojfstf{\ctxwk{A}{R}}{\ctxwk{A}{S}}}
  \tag{by \autoref{lem:tmext-wk}}
  \\
& \jdeq
  \cprojfstf{\subst{f}{\ctxwk{A}{R}}}{\subst{f}{\ctxwk{A}{R}}}
  \tag{by \autoref{lem:tmext-subst}}
  \\
& \jdeq
  \cprojfstf{\jcomp{A}{f}{R}}{\jcomp{A}{f}{S}}
  \tag{by definition}
\end{align*}
and
\begin{align*}
\jcomp{A}{f}{\cprojsndf{R}{S}}
& \jdeq
  \unfold\jcomp{A}{f}{\cprojsndf{R}{S}}
  \tag{by definition}
  \\
& \jdeq
  \subst{f}{\cprojsndf{\ctxwk{A}{R}}{\ctxwk{A}{S}}}
  \tag{by \autoref{lem:tmext-wk}}
  \\
& \jdeq
  \cprojsndf{\subst{f}{\ctxwk{A}{R}}}{\subst{f}{\ctxwk{A}{R}}}
  \tag{by \autoref{lem:tmext-subst}}
  \\
& \jdeq
  \cprojsndf{\jcomp{A}{f}{R}}{\jcomp{A}{f}{S}}
  \tag{by definition}
\end{align*}
\end{proof}

In \autoref{lem:mor-tmext-cproj} below we will explain what happens in the
case where the codomain of the function $f$ of the previous lemma is itself
extended.

Recall that the judgment $\jhom{\Gamma}{A}{{B}{Q}}{f}$
unfolds as
\begin{equation*}
\unfold{\jhom{\Gamma}{A}{{B}{Q}}{f}}
\end{equation*}
and that we have the judgmental equality 
$ \jfameq
    {{\Gamma}{A}}
    {\ctxwk{A}{\ctxext{B}{Q}}}
    {\ctxext{\ctxwk{A}{B}}{\ctxwk{A}{Q}}}.
  $
Therefore, each morphism into an extended family can itself be described as
an extended term. The following lemma explains how this goes.

\begin{lem}\label{lem:mor-tmext-cproj}
Let $\jhom{\Gamma}{A}{{B}{Q}}{f}$ be a morphism from $A$ to $\ctxext{B}{Q}$
in a context $\Gamma$. Then we have
\begin{equation*}
\jhomeq
  {\Gamma}
  {A}
  {{B}{Q}}
  {f}
  {\tmext{\jcomp{}{f}{\cprojfstf{B}{Q}}}{\jcomp{}{f}{\cprojsndf{B}{Q}}}}.
\end{equation*}
Alternatively, when $\jhom{\Gamma}{A}{B}{f_0}$ and 
$\jterm{{\Gamma}{A}}{\jcomp{}{f_0}{Q}}{f_1}$ we obtain a morphism
$\jhom{\Gamma}{A}{{B}{Q}}{\tmext{f_0}{f_1}}$ with the property that
\begin{align*}
\jalign\jhomeq{\Gamma}{A}{B}{\jcomp{}{\tmext{f_0}{f_1}}{\cprojfstf{B}{Q}}}{f_0}\\
\jalign\jtermeq{{\Gamma}{A}}{\jcomp{}{f_0}{Q}}{\jcomp{}{\tmext{f_0}{f_1}}{\cprojsndf{B}{Q}}}{f_1}.
\end{align*}
\end{lem}

\begin{proof}
Let $\jhom{\Gamma}{A}{{B}{Q}}{f}$ be a morphism in context $\Gamma$. Then we
have the judgmental equalities
\begin{align*}
\cprojfst{\ctxwk{A}{B}}{\ctxwk{A}{Q}}{f}
& \jdeq
  \subst{f}{\ctxwk{A}{\cprojfstf{B}{Q}}}
  \tag{by \autoref{lem:tmext-wk}}
  \\
& \jdeq
  \jcomp{}{f}{\cprojfstf{B}{Q}}
  \tag{by definition}
  \\
\cprojsnd{\ctxwk{A}{B}}{\ctxwk{A}{Q}}{f}
& \jdeq
  \subst{f}{\ctxwk{A}{\cprojsndf{B}{Q}}}
  \tag{by \autoref{lem:tmext-wk}}
  \\
& \jdeq
  \jcomp{}{f}{\cprojsndf{B}{Q}}
  \tag{by definition}
\end{align*}
The alternative formulation of the statement is a direct corollary.
\end{proof}

Precomposing by the first projection $\cprojfstf{A}{P}:\ctxext{A}{P}\to A$
is judgmentally equal to weakening by $P$:

\begin{lem}\label{lem:jcomp-cprojfstf}
The following inference rule is valid:
\begin{align*}
& \inference
  { \jfam{{\Gamma}{A}}{P}
    \jfam{{\Gamma}{A}}{Q}
    }
  { \jfameq
      {{\Gamma}{{A}{P}}}
      {\jcomp{{A}{P}}{\cprojfstf{A}{P}}{Q}}
      {\ctxwk{P}{Q}}
    }
  \\
& \inference
  { \jfam{{\Gamma}{A}}{P}
    \jfam{{{\Gamma}{A}}{Q}}{R}
    }
  { \jfameq
      {{{\Gamma}{{A}{P}}}{\ctxwk{A}{Q}}}
      {\jcomp{{A}{P}}{\cprojfstf{A}{P}}{R}}
      {\ctxwk{P}{R}}
    }
  \\
& \inference
  { \jfam{{\Gamma}{A}}{P}
    \jterm{{{\Gamma}{A}}{Q}}{R}{h}
    }
  { \jtermeq
      {{{\Gamma}{{A}{P}}}{\ctxwk{A}{Q}}}
      {\ctxwk{A}{R}}
      {\jcomp{{A}{P}}{\cprojfstf{A}{P}}{h}}
      {\ctxwk{P}{h}}
    }
\end{align*}
\end{lem}

\begin{proof}
Let $\jfam{{\Gamma}{A}}{P}$ and $\jfam{{\Gamma}{A}}{Q}$ be
families. Then we have
\begin{align*}
\jcomp{{A}{P}}{\cprojfstf{A}{P}}{Q}
& \jdeq
  \unfoldall{\jcomp{{A}{P}}{\cprojfstf{A}{P}}{Q}}
  \tag{by definition}\\
& \jdeq 
  \subst{\ctxwk{P}{\idtm{A}}}{\ctxwk{P}{{A}{Q}}} 
  \tag{by \autoref{comp-ww-f}}\\
& \jdeq 
  \ctxwk{P}{\subst{\idtm{A}}{\ctxwk{A}{Q}}} 
  \tag{by \autoref{comp-ws-f}}\\
& \jdeq 
  \ctxwk{P}{Q} 
  \tag{by \autoref{precomp-idtm-c}}
\end{align*}
The other proofs are similar.
\end{proof}

As an application of these lemmas we prove that every term $\jterm{{\Gamma}{A}}{P}{f}$
induces a section $\jtermashom{f}$ of $\cprojfstf{A}{P}$.
Note that we can directly view any term $x$ of $A$ in context $\Gamma$ as a
morphism from the empty family $\emptyf$ to $A$ in context $\Gamma$. 

Now let $\jterm{{\Gamma}{A}}{P}{f}$ be a term of $P$. Then we can view $f$ as a
morphism from $A$ to $\ctxext{A}{P}$ in the following way:

\begin{defn}\label{defn:terms-as-sections}
Let $\jterm{{\Gamma}{A}}{P}{f}$ be a term. Then we define
\begin{equation*}
\jhomdefn{\Gamma}{A}{{A}{P}}{\jtermashom{f}}{\unfold{\jtermashom{f}}}
\end{equation*}
\end{defn}

In the following lemma we will see that $\jtermashom{f}$ is a section for
$\cprojfstf{A}{P}$.

\begin{cor}
Consider a family $\jfam{{\Gamma}{A}}{P}$ and a term $\jterm{\Gamma}{A}{x}$.
Then we have the judgmental equality
\begin{equation*}
\jhomeq{\Gamma}{A}{A}{\jcomp{A}{\jtermashom{f}}{\cprojfstf{A}{P}}}{\idtm{A}}.
\end{equation*}
\end{cor}

\begin{proof}
The asserted judgmental equality follows immediately from \autoref{lem:mor-tmext-cproj}.
\end{proof}


%%%%%%%%%%%%%%%%%%%%%%%%%%%%%%%%%%%%%%%%%%%%%%%%%%%%%%%%%%%%%%%%%%%%%%%%%%%%%%%%
\subsection{Family morphisms}\label{sec:fhom}
In this section we are going to develop dependent morphisms and analyze various
sorts of composition. The notion of family morphism is the notion of `a 
morphism over a context morphism'. When we have established all the associativity
and interchange laws for all the compositions that come with the theory of
family morphisms, we will also be able to consider \emph{family diagrams}. Those
are graphical displays of situations in the theory of contexts, families and
terms, but unlike diagrams in category theory they can contain families (which
are strictly speaking not morphisms) and terms (which can be seen as morphisms,
but not in the way we're about to display them).

\begin{defn}
Let $\jhom{\Gamma}{A}{B}{f}$ be a morphism from $A$ to $B$ in context $\Gamma$
and consider $\jfam{{\Gamma}{A}}{P}$ and $\jfam{{\Gamma}{B}}{Q}$. We define the
judgment\begin{equation*}
\jfhom{\Gamma}{A}{B}{f}{P}{Q}{F},
\end{equation*}
which is pronounced as `$F$ is a morphism from $P$ to $Q$ over $f$ in context
$\Gamma$', to be the judgment\begin{equation*}
\unfold{\jfhom{\Gamma}{A}{B}{f}{P}{Q}{F}}.
\end{equation*}
\end{defn}

\begin{rmk}\label{rmk:jhom-as-jfhom}
The judgment $\jfhom{\Gamma}{A}{B}{f}{P}{Q}{F}$ means the same thing as
\begin{equation*}
\jhom{{\Gamma}{A}}{P}{\jcomp{A}{f}{Q}}{F}.
\end{equation*}
Thus we see that a morphism from $P$ to $Q$ over the identity term $\idtm{A}$ in
context $\Gamma$ is the same thing as a morphism from $P$ to $Q$ in context
$\ctxext{\Gamma}{A}$, i.e.~the following inference rules are valid:
\begin{align*}
& \inference
  { \jfam{{\Gamma}{A}}{P}
    \jfam{{\Gamma}{A}}{Q}
    \jfhom{\Gamma}{A}{A}{\idtm{A}}{P}{Q}{f}
    }
  { \jhom{{\Gamma}{A}}{P}{Q}{f}
    }
  \\
& \inference
  { \jfam{{\Gamma}{A}}{P}
    \jfam{{\Gamma}{A}}{Q}
    \jhom{{\Gamma}{A}}{P}{Q}{f}
    }
  { \jfhom{\Gamma}{A}{A}{\idtm{A}}{P}{Q}{f}
    }
\end{align*}
To see this, we only have to note that
$\ctxwk{P}{\subst{\idtm{A}}{\ctxwk{A}{Q}}}\jdeq\ctxwk{P}{Q}$, which
holds by \autoref{precomp-idtm-c}.
\end{rmk}

\subsubsection{Horizontal and vertical composition}

Suppose we have morphisms $\jhom{\Gamma}{A}{B}{f}$ and $\jhom{\Gamma}{B}{C}{g}$
and that we have the morphisms $\jfhom{\Gamma}{A}{B}{f}{P}{Q}{F}$ and
$\jfhom{\Gamma}{B}{C}{g}{Q}{R}{G}$ over them. Then we have\begin{equation*}
\jhom
  {{\Gamma}{A}}
  {\jcomp{A}{f}{Q}}
  {\jcomp{A}{f}{{B}{g}{R}}}
  {\jcomp{A}{f}{G}}
\end{equation*}
Because we also have $\jhom{{\Gamma}{A}}{P}{\jcomp{A}{f}{Q}}{F}$, we have the
composition\begin{equation*}
\jhom
  {{\Gamma}{A}}
  {P}
  {\jcomp{A}{f}{{B}{g}{R}}}
  {\jcomp{P}{F}{\jcomp{A}{f}{G}}}.
\end{equation*}
Because of 
the judgmental equality $\jcomp{A}{f}{{B}{g}{R}}\jdeq
\jcomp{A}{{A}{f}{g}}{R}$, it follows that 
$\jcomp{P}{F}{\jcomp{A}{f}{G}}$ is a morphism from $P$ to $R$ over
$\jcomp{A}{f}{g}$. This could be considered as the composition of $G$ with $F$.
In the following definition, we formulate this more generally.

\begin{defn}
\emph{Horizontal composition of morphisms over morphisms} is defined by
\begin{align*}
& \inference
  { \jfhom{\Gamma}{A}{B}{f}{P}{Q}{F}
    \jfam{{{\Gamma}{B}}{Q}}{R}
    }
  { \jfamdefn
      {{{\Gamma}{A}}{P}}
      {\jfcomp{A}{f}{P}{F}{R}}
      {\unfold{\jfcomp{A}{f}{P}{F}{R}}}
    }
  \\
& \inference
  { \jfhom{\Gamma}{A}{B}{f}{P}{Q}{F}
    \jfam{{{{\Gamma}{B}}{Q}}{R}}{S}
    }
  { \jfamdefn
      {{{{\Gamma}{A}}{P}}{\jfcomp{A}{f}{P}{F}{R}}}
      {\jfcomp{A}{f}{P}{F}{S}}
      {\unfold{\jfcomp{A}{f}{P}{F}{S}}}
    }
  \\
& \inference
  { \jfhom{\Gamma}{A}{B}{f}{P}{Q}{F}
    \jterm{{{{\Gamma}{B}}{Q}}{R}}{S}{k}
    }
  { \jtermdefn
      {{{{\Gamma}{A}}{P}}{\jfcomp{A}{f}{P}{F}{R}}}
      {\jfcomp{A}{f}{P}{F}{S}}
      {\jfcomp{A}{f}{P}{F}{k}}
      {\unfold{\jfcomp{A}{f}{P}{F}{S}}}
    }
\end{align*}
\end{defn}

Since horizontal composition is just consecutive composition, we will get the
compatibility with the empty context, extension, weakening, substitution and
the identity terms for free. We will state these compatibility properities in
\autoref{lem:jfcomp-emp,lem:jfcomp-ext,lem:jfcomp-wk,lem:jfcomp-subst,lem:jfcomp-idtm},
but the proofs are left to the reader.

\begin{lem}\label{lem:jfcomp-emp}
Horizontal composition is compatible with the empty context:
\begin{align*}
& \inference
  { \jfhom{\Gamma}{A}{B}{f}{P}{Q}{F}
    }
  { \jfameq
      {{{\Gamma}{A}}{P}}
      {\jfcomp{A}{f}{P}{F}{\emptyf}}
      {\emptyf}
    }
  \\
& \inference
  { \jfhom{\Gamma}{A}{B}{f}{P}{Q}{F}
    \jfam{{{\Gamma}{B}}{Q}}{R}
    }
  { \jfameq
      {{{{\Gamma}{A}}{P}}{\jfcomp{A}{f}{P}{F}{R}}}
      {\jfcomp{A}{f}{P}{F}{\emptyf}}
      {\emptyf}
    }
\end{align*}
\end{lem}

\begin{lem}\label{lem:jfcomp-ext}
Horizontal composition is compatible with extension:
\begin{align*}
& \inference
  { \jfhom{\Gamma}{A}{B}{f}{P}{Q}{F}
    \jfam{{{{\Gamma}{B}}{Q}}{R}}{S}
    }
  { \jfameq
      {{{\Gamma}{A}}{P}}
      {\jfcomp{A}{f}{P}{F}{\ctxext{R}{S}}}
      {\ctxext{\jfcomp{A}{f}{P}{F}{R}}{\jfcomp{A}{f}{P}{F}{S}}}
    }
  \\
& \inference
  { \jfhom{\Gamma}{A}{B}{f}{P}{Q}{F}
    \jfam{{{{{\Gamma}{B}}{Q}}{R}}{S}}{T}
    }
  { \jfameq
      {{{{\Gamma}{A}}{P}}{\jfcomp{A}{f}{P}{F}{R}}}
      {\jfcomp{A}{f}{P}{F}{\ctxext{S}{T}}}
      {\ctxext{\jfcomp{A}{f}{P}{F}{S}}{\jfcomp{A}{f}{P}{F}{T}}}
    }
  \\
& \inference
  { \jfhom{\Gamma}{A}{B}{f}{P}{Q}{F}
    \jterm{{{{\Gamma}{B}}{Q}}{R}}{{S}{T}}{\tmext{k}{l}}
    }
  { \jtermdefn
      {{{{\Gamma}{A}}{P}}{\jfcomp{A}{f}{P}{F}{R}}}
      {\ctxext{\jfcomp{A}{f}{P}{F}{S}}{\jfcomp{A}{f}{P}{F}{T}}}
      {\jfcomp{A}{f}{P}{F}{\tmext{k}{l}}}
      {\tmext{\jfcomp{A}{f}{P}{F}{k}}{\jfcomp{A}{f}{P}{F}{l}}}
    }
\end{align*}
\end{lem}

\begin{lem}\label{lem:jfcomp-wk}
Horizontal composition is compatible with weakening:
\begin{align*}
& \inference
  { \jfhom{\Gamma}{A}{B}{f}{P}{Q}{F}
    \jfam{{{{\Gamma}{B}}{Q}}{R}}{S}
    \jfam{{{{\Gamma}{B}}{Q}}{R}}{T}
    }
  { \jfameq
      {{{{{\Gamma}{A}}{P}}{\jfcomp{A}{f}{P}{F}{R}}}{\jfcomp{A}{f}{P}{F}{S}}}
      {\jfcomp{A}{f}{P}{F}{\ctxwk{S}{T}}}
      {\ctxwk{\jfcomp{A}{f}{P}{F}{S}}{\jfcomp{A}{f}{P}{F}{T}}}
    }
  \\
& \inference
  { \jfhom{\Gamma}{A}{B}{f}{P}{Q}{F}
    \jfam{{{{\Gamma}{B}}{Q}}{R}}{S}
    \jfam{{{{{\Gamma}{B}}{Q}}{R}}{T}}{U}
    }
  { \jfameq
      {{{{{{\Gamma}{A}}{P}}
        {\jfcomp{A}{f}{P}{F}{R}}}
        {\jfcomp{A}{f}{P}{F}{S}}}
        {\ctxwk{\jfcomp{A}{f}{P}{F}{S}}{\jfcomp{A}{f}{P}{F}{T}}}}
      {\jfcomp{A}{f}{P}{F}{\ctxwk{S}{U}}}
      {\ctxwk{\jfcomp{A}{f}{P}{F}{S}}{\jfcomp{A}{f}{P}{F}{U}}}
    }
  \\
& \inference
  { \jfhom{\Gamma}{A}{B}{f}{P}{Q}{F}
    \jfam{{{{\Gamma}{B}}{Q}}{R}}{S}
    \jterm{{{{{\Gamma}{B}}{Q}}{R}}{T}}{U}{m}
    }
  { \jtermeq
      {{{{{{\Gamma}{A}}{P}}
        {\jfcomp{A}{f}{P}{F}{R}}}
        {\jfcomp{A}{f}{P}{F}{S}}}
        {\ctxwk{\jfcomp{A}{f}{P}{F}{S}}{\jfcomp{A}{f}{P}{F}{T}}}}
      {\jfcomp{A}{f}{P}{F}{\ctxwk{S}{U}}}
      {\jfcomp{A}{f}{P}{F}{\ctxwk{S}{m}}}
      {\ctxwk{\jfcomp{A}{f}{P}{F}{S}}{\jfcomp{A}{f}{P}{F}{m}}}
    }
\end{align*}
\end{lem}

\begin{lem}\label{lem:jfcomp-subst}
Horizontal composition is compatible with substitution:
\begin{align*}
& \inference
  { \jfhom{\Gamma}{A}{B}{f}{P}{Q}{F}
    \jterm{{{{\Gamma}{B}}{Q}}{R}}{S}{k}
    \jfam{{{{{\Gamma}{B}}{Q}}{R}}{S}}{T}
    }
  { \jfameq
      {{{{\Gamma}{A}}{P}}{\jfcomp{A}{f}{P}{F}{R}}}
      {\jfcomp{A}{f}{P}{F}{\subst{k}{T}}}
      {\subst{\jfcomp{A}{f}{P}{F}{k}}{\jfcomp{A}{f}{P}{F}{T}}}
    }
  \\
& \inference
  { \jfhom{\Gamma}{A}{B}{f}{P}{Q}{F}
    \jterm{{{{\Gamma}{B}}{Q}}{R}}{S}{k}
    \jfam{{{{{{\Gamma}{B}}{Q}}{R}}{S}}{T}}{U}
    }
  { \jfameq
      {{{{{\Gamma}{A}}{P}}{\jfcomp{A}{f}{P}{F}{R}}}{\jfcomp{A}{f}{P}{F}{\subst{k}{T}}}}
      {\jfcomp{A}{f}{P}{F}{\subst{k}{U}}}
      {\subst{\jfcomp{A}{f}{P}{F}{k}}{\jfcomp{A}{f}{P}{F}{U}}}
    }
  \\
& \inference
  { \jfhom{\Gamma}{A}{B}{f}{P}{Q}{F}
    \jterm{{{{\Gamma}{B}}{Q}}{R}}{S}{k}
    \jterm{{{{{{\Gamma}{B}}{Q}}{R}}{S}}{T}}{U}{m}
    }
  { \jfameq
      {{{{{\Gamma}{A}}{P}}{\jfcomp{A}{f}{P}{F}{R}}}{\jfcomp{A}{f}{P}{F}{\subst{k}{T}}}}
      {\jfcomp{A}{f}{P}{F}{\subst{k}{U}}}
      {\jfcomp{A}{f}{P}{F}{\subst{k}{m}}}
      {\subst{\jfcomp{A}{f}{P}{F}{k}}{\jfcomp{A}{f}{P}{F}{m}}}
    }
\end{align*}
\end{lem}

\begin{lem}\label{lem:jfcomp-idtm}
Horizontal composition is compatible with identity terms:
\begin{equation*}
\inference
  { \jfhom{\Gamma}{A}{B}{f}{P}{Q}{F}
    \jfam{{{{\Gamma}{B}}{Q}}{R}}{S}
    }
  { \jtermeq
      {{{{{\Gamma}{A}}{P}}{\jfcomp{A}{f}{P}{F}{R}}}{\jfcomp{A}{f}{P}{F}{S}}}
      {\ctxwk{{\jfcomp{A}{f}{P}{F}{S}}}{\jfcomp{A}{f}{P}{F}{S}}}
      {\jfcomp{A}{f}{P}{F}{\idtm{S}}}
      {\idtm{\jfcomp{A}{f}{P}{F}{S}}}
    }
\end{equation*}
\end{lem}

There is also a notion of vertical composition, although this is not an
operation the way horizontal composition is. Nevertheless, to analyze the
properties of horizontal composition it is useful to
consider vertical composition.

\begin{defn}
Let $\jhom{\Gamma}{A}{B}{f}$ be a morphism from $A$ to $B$ in context $\Gamma$
and let $\jfhom{\Gamma}{A}{B}{f}{P}{Q}{F}$ be a morphism over $f$ in context 
$\Gamma$. We define
\begin{equation*}
\jhomdefn{\Gamma}{{A}{P}}{{B}{Q}}{\jvcomp{P}{f}{F}}{\unfold{\jvcomp{P}{f}{F}}}
\end{equation*}
\end{defn}

\begin{lem}
The following inference rule explains how vertically and horizontally composed 
morphisms are evaluated:
\begin{align*}
& \inference
  { \jfhom{\Gamma}{A}{B}{f}{P}{Q}{F}
    \jterm{\Gamma}{A}{x}
    \jterm{\Gamma}{\subst{x}{P}}{u}
    }
  { \jtermeq
      {\Gamma}
      {{B}{Q}}
      {\subst{\tmext{}{}{x}{u}}{\jvcomp{P}{f}{F}}}
      {\tmext{}{}{\subst{x}{f}}{\subst{u}{{x}{F}}}}
    }
\end{align*}
\end{lem}

\begin{proof}
Let $\jfhom{\Gamma}{A}{B}{f}{P}{Q}{F}$, $\jterm{\Gamma}{A}{x}$ and
$\jterm{\Gamma}{\subst{x}{P}}{u}$. Then we have the judgmental equalities
\begin{align*}
\subst{\tmext{}{}{x}{u}}{\jvcomp{P}{f}{F}}
& \jdeq
  \subst{\tmext{}{}{x}{u}}{\tmext{}{}{\ctxwk{P}{f}}{F}}
  \tag{by definition}
  \\
& \jdeq
  \tmext{\subst{\tmext{x}{u}}{\ctxwk{P}{f}}}{\subst{\tmext{x}{u}}{F}}
  \tag{by \autoref{lem:tmext-subst}}
  \\
& \jdeq
  \tmext{\subst{u}{{x}{\ctxwk{P}{f}}}}{\subst{u}{{x}{F}}}
  \tag{by \autoref{comp-es}}
  \\
& \jdeq
  \tmext{\subst{u}{\ctxwk{\subst{x}{P}}{\subst{x}{f}}}}{\subst{u}{{x}{F}}}
  \tag{by \autoref{comp-sw-t}}
  \\
& \jdeq
  \tmext{\subst{x}{f}}{\subst{u}{{x}{F}}}.
  \tag{by \autoref{cancellation-ws-t}}
\end{align*}
\end{proof}

\begin{lem}\label{lem:jvcomp-cprojfstf}
Consider $\jhom{\Gamma}{A}{B}{f}$ and $\jfhom{\Gamma}{A}{B}{f}{P}{Q}{F}$. Then
the diagram
\begin{equation*}
\begin{tikzcd}
\ctxext{A}{P}
  \ar{r}{\jvcomp{P}{f}{F}}
  \ar{d}[swap]{\cprojfstf{A}{P}}
& \ctxext{B}{Q}
  \ar{d}{\cprojfstf{B}{Q}}
  \\
A \ar{r}[swap]{f}
& B
\end{tikzcd}
\end{equation*}
commutes judgmentally.
\end{lem}

\begin{proof}
We have the judgmental equalities
\begin{align*}
\jcomp{{A}{P}}{\jvcomp{P}{f}{F}}{\cprojfstf{B}{Q}}
& \jdeq
  \jcomp{{A}{P}}{\tmext{\ctxwk{P}{f}}{F}}{\cprojfstf{B}{Q}}
  \tag{by definition}
  \\
& \jdeq
  \ctxwk{P}{f}
  \tag{by \autoref{lem:mor-tmext-cproj}}
  \\
& \jdeq
  \jcomp{{A}{P}}{\cprojfstf{A}{P}}{f}
  \tag{by \autoref{lem:jcomp-cprojfstf}}
\end{align*}
\end{proof}

\subsubsection{Interactions of horizontal and vertical composition}
Our current goal is to show that the three sorts of composition are all
compatbile with each other. This includes showing associativity and interchange
properties. The following lemma of the composition threesome stands at the
basis of the further compatibility results.

\begin{lem}\label{lem:composition-threesome}
The following inference rules are valid:
\begin{align*}
& \inference
  { \jfhom{\Gamma}{A}{B}{f}{P}{Q}{F}
    \jfam{{{\Gamma}{B}}{Q}}{R}
    }
  { \jfameq
      {{{\Gamma}{A}}{P}}
      {\jfcomp{A}{f}{P}{F}{R}}
      {\jcomp{{A}{P}}{\jvcomp{P}{f}{F}}{R}}
    }
  \\
& \inference
  { \jfhom{\Gamma}{A}{B}{f}{P}{Q}{F}
    \jfam{{{{\Gamma}{B}}{Q}}{R}}{S}
    }
  { \jfameq
      {{{{\Gamma}{A}}{P}}{\jfcomp{A}{f}{P}{F}{R}}}
      {\jfcomp{A}{f}{P}{F}{S}}
      {\jcomp{{A}{P}}{\jvcomp{P}{f}{F}}{S}}
    }
  \\
& \inference
  { \jfhom{\Gamma}{A}{B}{f}{P}{Q}{F}
    \jterm{{{{\Gamma}{B}}{Q}}{R}}{S}{k}
    }
  { \jtermeq
      {{{{\Gamma}{A}}{P}}{\jfcomp{A}{f}{P}{F}{R}}}
      {\jfcomp{A}{f}{P}{F}{S}}
      {\jfcomp{A}{f}{P}{F}{k}}
      {\jcomp{{A}{P}}{\jvcomp{P}{f}{F}}{k}}
    }
\end{align*}
\end{lem}

\begin{proof}
Let $F$ be a morphism from $P$ to $Q$ over $f$ in context $\Gamma$ and let
$R$ be a family over $\ctxext{{\Gamma}{B}}{Q}$. Then we have
\begin{align*}
\jfcomp{A}{f}{P}{F}{R}
& \jdeq
  \unfold{\jfcomp{A}{f}{P}{F}{R}}
  \tag{by definition}
  \\
& \jdeq
  \subst{F}{{\ctxwk{P}{f}}{\ctxwk{P}{{A}{R}}}}
  \tag{by \autoref{comp-ws-f}}
  \\
& \jdeq
  \subst{F}{{\ctxwk{P}{f}}{\ctxwk{\ctxext{A}{P}}{R}}}
  \tag{by \autoref{comp-ew-f}}
  \\
& \jdeq
  \subst{\tmext{\ctxwk{P}{f}}{F}}{\ctxwk{\ctxext{A}{P}}{R}}
  \tag{by \autoref{comp-es}}
  \\
& \jdeq
  \jcomp{{A}{P}}{\jvcomp{P}{f}{F}}{R}.
  \tag{by definition}
\end{align*}
\end{proof}

\begin{lem}\label{lem:composition-interchange}
We have the following interchange law for composition:
\begin{equation*}
\inference
  { \jfhom{\Gamma}{A}{B}{f}{P}{Q}{F}
    \jfhom{\Gamma}{B}{C}{g}{Q}{R}{G}
    }
  { \jhomeq
      {\Gamma}
      {{A}{P}}
      {{C}{R}}
      {\jcomp{{A}{P}}{\jvcomp{P}{f}{F}}{\jvcomp{Q}{g}{G}}}
      {\jvcomp{P}{\jcomp{A}{f}{g}}{\jfcomp{A}{f}{P}{F}{G}}}
    }
\end{equation*}
\end{lem}

\begin{proof}
Consider $\jfhom{\Gamma}{A}{B}{f}{P}{Q}{F}$ and 
$\jfhom{\Gamma}{B}{C}{g}{Q}{R}{G}$. Then we have
\begin{align*}
\jcomp{{A}{P}}{\jvcomp{P}{f}{F}}{\jvcomp{Q}{g}{G}}
& \jdeq
  \jcomp{P}{F}{\jcomp{A}{f}{\tmext{}{}{\ctxwk{Q}{g}}{G}}}
  \tag{by \autoref{lem:composition-threesome}}
  \\
& \jdeq
  \tmext{}{}
    {\jcomp{P}{F}{\jcomp{A}{f}{\ctxwk{Q}{g}}}}
    {\jcomp{P}{F}{\jcomp{A}{f}{G}}}
  \tag{by \autoref{lem:tmext-jcomp}}
  \\
& \jdeq
  \tmext{}{}
    {\jcomp{P}{F}{\jcomp{A}{f}{\ctxwk{Q}{g}}}}
    {\jfcomp{A}{f}{P}{F}{G}}
  \tag{by definition}
  \\
& \jdeq
  \tmext{}{}
    {\jcomp{P}{F}{\ctxwk{\jcomp{A}{f}{Q}}{\jcomp{A}{f}{g}}}}
    {\jfcomp{A}{f}{P}{F}{G}}
  \tag{by \autoref{lem:jcomp-wk}}
  \\
& \jdeq
  \tmext{}{}
    {\subst{F}{\ctxwk{P}{\ctxwk{\jcomp{A}{f}{Q}}{\jcomp{A}{f}{g}}}}}
    {\jfcomp{A}{f}{P}{F}{G}}
  \tag{by definition}
  \\
& \jdeq
  \tmext{}{}
    {\subst{F}{\ctxwk{{P}{\jcomp{A}{f}{Q}}}{{P}{\jcomp{A}{f}{g}}}}}
    {\jfcomp{A}{f}{P}{F}{G}}
  \tag{by \autoref{comp-ww-t}}
  \\
& \jdeq
  \tmext{}{}
    {\ctxwk{P}{\jcomp{A}{f}{g}}}
    {\jfcomp{A}{f}{P}{F}{G}}
  \tag{by \autoref{cancellation-ws-t}}
  \\
& \jdeq
  \jvcomp{P}{\jcomp{A}{f}{g}}{\jfcomp{A}{f}{P}{F}{G}}
  \tag{by definition}
\end{align*}
\end{proof}

\begin{lem}\label{lem:jfcomp-jfcomp}
Horizontal composition of morphisms over morphisms is associative, i.e.~the
following inference rules are valid:
\begin{align*}
& \inference
  { \jfhom{\Gamma}{A}{B}{f}{P}{Q}{F}
    \jfhom{\Gamma}{B}{C}{g}{Q}{R}{G}
    \jfam{{{\Gamma}{C}}{R}}{S}
    }
  { \jfameq
      {{{\Gamma}{A}}{P}}
      {\jfcomp{A}{f}{P}{F}{\jfcomp{B}{g}{Q}{G}{S}}}
      {\jfcomp{A}{\jcomp{A}{f}{g}}{P}{\jfcomp{A}{f}{P}{F}{G}}{S}}
    }
  \\
& \inference
  { \jfhom{\Gamma}{A}{B}{f}{P}{Q}{F}
    \jfhom{\Gamma}{B}{C}{g}{Q}{R}{G}
    \jfam{{{{\Gamma}{C}}{R}}{S}}{T}
    }
  { \jfameq
      {{{{\Gamma}{A}}{P}}{\jfcomp{A}{f}{P}{F}{\jfcomp{B}{g}{Q}{G}{S}}}}
      {\jfcomp{A}{f}{P}{F}{\jfcomp{B}{g}{Q}{G}{T}}}
      {\jfcomp{A}{\jcomp{A}{f}{g}}{P}{\jfcomp{A}{f}{P}{F}{G}}{T}}
    }
  \\
& \inference
  { \jfhom{\Gamma}{A}{B}{f}{P}{Q}{F}
    \jfhom{\Gamma}{B}{C}{g}{Q}{R}{G}
    \jterm{{{{\Gamma}{C}}{R}}{S}}{T}{l}
    }
  { \jtermeq
      {{{{\Gamma}{A}}{P}}{\jfcomp{A}{f}{P}{F}{\jfcomp{B}{g}{Q}{G}{S}}}}
      {\jfcomp{A}{f}{P}{F}{\jfcomp{B}{g}{Q}{G}{T}}}
      {\jfcomp{A}{f}{P}{F}{\jfcomp{B}{g}{Q}{G}{l}}}
      {\jfcomp{A}{\jcomp{A}{f}{g}}{P}{\jfcomp{A}{f}{P}{F}{G}}{l}}
    }
\end{align*}
\end{lem}

\begin{proof}
We only prove the first inference rule. Let
$\jfhom{\Gamma}{A}{B}{f}{P}{Q}{F}$, $\jfhom{\Gamma}{B}{C}{g}{Q}{R}{G}$ and
$\jfam{{{\Gamma}{C}}{R}}{S}$. Then we have the judgmental equalities
\begin{align*}
\jfcomp{A}{f}{P}{F}{\jfcomp{B}{g}{Q}{G}{S}}
& \jdeq
  \jcomp{{A}{P}}{\jvcomp{P}{f}{F}}{\jcomp{{B}{Q}}{\jvcomp{Q}{g}{G}}{S}}
  \tag{by \autoref{lem:composition-threesome}}
  \\
& \jdeq
  \jcomp{{A}{P}}{\jcomp{{A}{P}}{\jvcomp{P}{f}{F}}{\jvcomp{Q}{g}{G}}}{S}
  \tag{by \autoref{lem:jcomp-jcomp}}
  \\
& \jdeq
  \jcomp{{A}{P}}{\jvcomp{P}{\jcomp{A}{f}{g}}{\jfcomp{A}{f}{P}{F}{G}}}{S}
  \tag{by \autoref{lem:composition-interchange}}
  \\
& \jdeq
  \jfcomp{A}{\jcomp{A}{f}{g}}{P}{\jfcomp{A}{f}{P}{F}{G}}{S}.
  \tag{by \autoref{lem:composition-threesome}}
\end{align*}
\end{proof}

Likewise, vertical composition is associative. For the following lemma, keep in
mind that the inference rules
\begin{align*}
& \inference
  { \jfhom{\Gamma}{{A}{P}}{{B}{Q}}{\jvcomp{P}{f_0}{f_1}}{R}{S}{f_2}
    }
  { \jfhom{{\Gamma}{A}}{P}{\jcomp{A}{f_0}{Q}}{f_1}{R}{\jcomp{A}{f_0}{S}}{f_2}
    }
  \\
& \inference
  { \jfhom{{\Gamma}{A}}{P}{\jcomp{A}{f_0}{Q}}{f_1}{R}{\jcomp{A}{f_0}{S}}{f_2}
    }
  { \jfhom{\Gamma}{{A}{P}}{{B}{Q}}{\jvcomp{P}{f_0}{f_1}}{R}{S}{f_2}
    }
\end{align*}
are valid. Thus, a morphism over the vertical composition of $f_1$ over $f_0$
is the same thing as a morphism over $f_1$ in the appropriate sense. Thus, we
have vertical composition also available to compose $f_2$ with $f_1$ vertically
and we don't need to introduce a seperate `vertical composition' for this.

\begin{lem}\label{lem:jvcomp-jvcomp}
Vertical composition is associative: the inverence rule
\begin{equation*}
\inference
  { \jhom{\Gamma}{A}{B}{f_0}
    \jfhom{\Gamma}{A}{B}{f_0}{P}{Q}{f_1}
    \jfhom{\Gamma}{{A}{P}}{{B}{Q}}{\jvcomp{P}{f_0}{f_1}}{R}{S}{f_2}
    }
  { \jhomeq
      {\Gamma}
      {{{A}{P}}{R}}
      {{{B}{Q}}{S}}
      {\jvcomp{R}{\jvcomp{P}{f_0}{f_1}}{f_2}}
      {\jvcomp{{P}{R}}{f_0}{\jvcomp{R}{f_1}{f_2}}}
    }
\end{equation*}
\end{lem}

\begin{proof}
Consider the morphisms $\jhom{\Gamma}{A}{B}{f_0}$, 
$\jfhom{\Gamma}{A}{B}{f_0}{P}{Q}{f_1}$ and 
$\jfhom{\Gamma}{{A}{P}}{{B}{Q}}{\jvcomp{P}{f_0}{f_1}}{R}{S}{f_2}$. Then we have
the judgmental equalities
\begin{align*}
\jvcomp{R}{\jvcomp{P}{f_0}{f_1}}{f_2}
& \jdeq 
  \tmext{\ctxwk{R}{\tmext{\ctxwk{P}{f_0}}{f_1}}}{f_2}
  \tag{by definition}
  \\
& \jdeq
  \tmext{{\ctxwk{R}{{P}{f_0}}}{\ctxwk{R}{f_1}}}{f_2}
  \tag{by \autoref{lem:tmext-wk}}
  \\
& \jdeq
  \tmext{\ctxwk{R}{{P}{f_0}}}{{\ctxwk{R}{f_1}}{f_2}}
  \tag{by \autoref{lem:tmext-tmext}}
  \\
& \jdeq
  \tmext{\ctxwk{R}{{P}{f_0}}}{\jvcomp{R}{f_1}{f_2}}
  \tag{by definition}
  \\
& \jdeq
  \tmext{\ctxwk{\ctxext{P}{R}}{f_0}}{\jvcomp{R}{f_1}{f_2}}
  \tag{by \autoref{comp-ew-t}}
  \\
& \jdeq
  \jvcomp{{P}{R}}{f_0}{\jvcomp{R}{f_1}{f_2}}.
  \tag{by definition}
\end{align*}
\end{proof}

\subsubsection{Diagrams in the theory of contexts, families and terms}
\label{pullback}
Now that we have introduced the notions of morphisms and composition,
we can develop a diagramatic style of of displaying type dependencies
combined with morphisms. We have proved all the possible associativity and
interchange properties concerning the various kinds of composition, which
ensures that when a diagram is asserted to commute, the order in which
morphisms are combined has no relevance.

We give an informal, metatheoretical definition of
such diagrams by indicating what the various components mean. The definition
is informal because we will only use such diagrams occasionally to provide a
graphical indication of the situation in which we're working. In particular,
we will not shy away from using natural numbers and trust that the reader can
figure out what we mean.

\begin{defn}
A diagram is said to be a \emph{dependency diagram in context $\Gamma$}
if it is built up according to the following steps:
\begin{itemize}
\item The arrows appearing in a dependency diagram are either ordinary, like the
arrow%
$\begin{tikzcd}[ampersand replacement = \&]
X \ar{r} \& Y,
\end{tikzcd}$
or double-headed, like
$\begin{tikzcd}[ampersand replacement = \&]
X \ar[fib]{r} \& Y.
\end{tikzcd}$
\item An ordinary arrow 
\begin{equation*}
\begin{tikzcd}
A \ar{r}{f} & B
\end{tikzcd}
\end{equation*}
between two families $A$ and $B$ of contexts over $\Gamma$ indicates that
$f$ is a morphism from $A$ to $B$ in context $\Gamma$, i.e.~that we have the
judgment $\jhom{\Gamma}{A}{B}{f}$.
\item The set of double-headed arrows must form a forest and the root of
each maximal tree of double-headed arrows is a family of contexts over $\Gamma$.
In particular, if an object is not the domain of a double-headed arrow it must
be a family of contexts over $\Gamma$.
\item A sequence of double-headed 
arrows
\begin{equation*}
\begin{tikzcd}
P_{n} \ar[fib]{r} & \cdots \ar[fib]{r} & P_1 \ar[fib]{r} & A
\end{tikzcd}
\end{equation*}
indicates that $P_1$ is a family of contexts over $\ctxext{\Gamma}{A}$, that
$P_2$ is a family of contexts over $\ctxext{{\Gamma}{A}}{P_1}$, etcetera.
\item There can be two kinds of ladders of double-headed arrows:
\begin{equation*}
\begin{tikzcd}
P_{n} \ar{r}{F_{n}} \ar[fib]{d} & Q_{n} \ar[fib]{d}\\
\vdots \ar[fib]{d} & \vdots \ar[fib]{d}\\
P_1 \ar{r}{F_1} \ar[fib]{d} & Q_1 \ar[fib]{d}\\
A \ar{r}{f} & B
\end{tikzcd}
\qquad
\begin{tikzcd}[column sep = tiny]
P_{n+m} \ar{rr}{F_{n+m}} \ar[fib]{d} & & Q_{n+m} \ar[fib]{d}\\
\vdots \ar[fib]{d} & & \vdots \ar[fib]{d}\\
P_{n+1} \ar{rr}{F_{n+1}} \ar[fib]{dr} & & Q_{n+1} \ar[fib]{dl}\\
& P_n \ar[fib]{d}\\
& \vdots \ar[fib]{d}\\
& P_1 \ar[fib]{d}\\
& A
\end{tikzcd}
\end{equation*}
The ladder on the left 
indicates that $F_1$ is a morphism from $P_1$ to $Q_1$ \emph{over} $f$,
i.e.~that the judgment $\jfhom{\Gamma}{A}{B}{f}{P_1}{Q_1}{F_1}$ holds, that
$F_2$ is a morphism from $P_2$ to $Q_2$ over
the morphism $\tmext{\ctxwk{P_1}{f}}{F_1}$ from $\ctxext{A}{P_1}$ to
$\ctxext{B}{Q_1}$, etcetera.

The ladder on the right indicates that $F_{n+1}$ is a morphism from $P_{n+1}$ to
$Q_{n+1}$ in the appropriate context, that $F_{n+2}$ is a morphism from
$P_{n+2}$ to $Q_{n+2}$ over $F_{n+1}$, etcetera.
 
Note that the object(s) at the bottom of a ladder are always families of contexts
over $\Gamma$, so that the typing of the various ingredients makes sense.
\end{itemize}
Such a diagram is said to be commutative if the subdiagram consisting of only
the normal headed arrows is commutative in the usual sense (using judgmental
equality). Note that the ladders are inherently commutative.
\end{defn}

The most basic illustrative example of a commutative dependency diagram is
the diagram
\begin{equation*}
\begin{tikzcd}
P \ar[fib]{d} \ar{r}{F} & Q \ar[fib]{d} \\
A \ar{r}{f} & B
\end{tikzcd}
\end{equation*}
indicating a morphism $F$ from $P$ to $Q$ over the morphism $f$ from $A$ to
$B$ in a context $\Gamma$.

We can just copy the usual categorical definition of a pullback square to our
current situation, but we have to require that each arrow in the pullback square
is an ordinary arrow. When families (i.e. double-headed arrows) are involved
in the diagram, we make the following definition of a family pullback:

\begin{defn}
We say that a commutative dependency diagram of the form
\begin{equation*}
\begin{tikzcd}
P \ar[fib]{d} \ar{r}{F} & Q \ar[fib]{d} \\
A \ar{r}{f} & B
\end{tikzcd}
\end{equation*}
is a \emph{family pullback} if the following inference rules are valid:
\begin{align*}
& \inference
  { \jfam{{\Gamma}{A}}{P'}
    \jfhom{\Gamma}{A}{B}{f}{P'}{Q}{F'}
    }
  { \jhom{{\Gamma}{A}}{P'}{P}{u}
    }
  \\
& \inference
  { \jfam{{\Gamma}{A}}{P'}
    \jfhom{\Gamma}{A}{B}{f}{P'}{Q}{F'}
    }
  { \jfhomeq{\Gamma}{A}{B}{f}{P'}{Q}{\jcomp{}{u}{F}}{F'}
    }
  \\
& \inference
  { \jhom{{\Gamma}{A}}{P'}{P}{v}
    \jfhomeq{\Gamma}{A}{B}{f}{P'}{Q}{\jcomp{}{v}{F}}{F'}
    }
  { \jhomeq{{\Gamma}{A}}{P'}{P}{v}{u}
    }
\end{align*}
\end{defn}

The following lemma explains that when a square involving families is a
family pullback square whenever the corresponding square involving projections is a
pullback square. There is no proof in the the opposite direction.

\begin{lem}
A square
\begin{equation}\label{eq:fpb_to_pb_eqv_fpb}
\begin{tikzcd}
P \ar[fib]{d} \ar{r}{F} & Q \ar[fib]{d} \\
A \ar{r}{f} & B
\end{tikzcd}
\end{equation}
is a family pullback square whenever the square
\begin{equation}\label{eq:fpb_to_pb_eqv_pb}
\begin{tikzcd}[column sep = large]
\ctxext{A}{P} \ar{d}[swap]{\cprojfstf{A}{P}} \ar{r}{\tmext{\ctxwk{P}{f}}{F}} & \ctxext{B}{Q} \ar{d}{\cprojfstf{B}{Q}} \\
A \ar{r}{f} & B
\end{tikzcd}
\end{equation}
is a pullback square.
\end{lem}

The family pullback of a family along any morphism always exists. It is simply given
by the precomposition of the family with the morphism. Note that this fact does
not carry over to arbitrary pullbacks.

\begin{lem}
The diagram
\begin{equation*}
\begin{tikzcd}
\jcomp{}{f}{Q} \ar[fib]{d} \ar{r}{\idtm{\jcomp{}{f}{Q}}} & Q \ar[fib]{d} \\
A \ar{r}{f} & B
\end{tikzcd}
\end{equation*}
is a family pullback diagram.
\end{lem}

\begin{proof}
The proof is a triviality because $\jhom{{\Gamma}{A}}{P'}{\jcomp{}{f}{Q}}{F'}$
is the same judgment as $\jfhom{\Gamma}{A}{B}{f}{P'}{Q}{F'}$ and
$\jcomp{}{\idtm{\jcomp{}{f}{Q}}}{F'}\jdeq F'$.
\end{proof}

For arbitrary pullbacks we have the pasting lemma as usual, but for family
pullbacks we can only derive one of the parts of the pasting lemma.

\begin{lem}
Suppose we have the diagram
\begin{equation*}
\begin{tikzcd}
P \ar{r}{F} \ar[fib]{d} & Q \ar{r}{G} \ar[fib]{d} & R \ar[fib]{d}\\
A \ar{r}{f} & B \ar{r}{g} & C
\end{tikzcd}
\end{equation*}
where the square on the right and the outer rectangle are family pullback 
diagrams. Then the square on the left is a family pullback diagram.
\end{lem}

\begin{proof}
Let $\jfam{{\Gamma}{A}}{P'}$ be a family and let $\jfhom{\Gamma}{A}{B}{f}
{P'}{Q}{F}$ be a morphism over $f$.
\begin{itemize}
\item Then we compose $F'$ with $G$ to obtain a morphism over $\jcomp{}{f}{g}$.
\item Then we get $\jhom{{\Gamma}{A}}{P'}{P}{u}$ with a uniqueness property.
      The property that $\jcomp{}{u}{F}\jdeq F'$ follows from the assumption
      that the right square is a pullback.
\item Now assume that we have another such $v$. Compose it with $F$ and $G$.
      By the assumed properties this is the same as $u$ composed with $F$ and
      $G$. By the pullback condition we now get $u\jdeq v$. 
\end{itemize}
\end{proof}

%%%%%%%%%%%%%%%%%%%%%%%%%%%%%%%%%%%%%%%%%%%%%%%%%%%%%%%%%%%%%%%%%%%%%%%%%%%%%%%%
\subsection{Fiber inclusions}
We will use the insights of \autoref{extension-on-terms} to define and study
\emph{fiber inclusions}. The fiber inclusion of the \emph{fiber}
$\subst{x}{P}$ into the extension $\ctxext{A}{P}$ is a morphism
$\jhom{\Gamma}{\subst{x}{P}}{{A}{P}}{\finc{x}{P}}$, for any family
$\jfam{{\Gamma}{A}}{P}$ and any term $\jterm{\Gamma}{A}{x}$. Then we will determine
the ways in which it is compatible with the other operators. Note that in this
subsection we will focus on the compatibility properties; the fact that
the fiber inclusions also appear in a pullback diagram will be established in
\autoref{pullback}. 

\begin{defn}
Let $\jterm{\Gamma}{A}{x}$ be a term and let $\jfam{{\Gamma}{A}}{P}$ be a
family. Then we define the \emph{fiber inclusion} of $\subst{x}{P}$ into
$\ctxext{A}{P}$ in context $\Gamma$ to be the morphism
\begin{equation*}
\jhomdefn{\Gamma}{\subst{x}{P}}{{A}{P}}{\finc{x}{P}}{\unfoldnext{\finc{x}{P}}}.
\end{equation*}
\end{defn}

We can immediately show that composing with a fiber inclusion is substitution.
Thus, the following lemma asserts that we have a family pullback square
\begin{equation*}
\begin{tikzcd}
\subst{x}{Q} \ar[fib]{d} \ar{r} & Q \ar[fib]{d} \\
\subst{x}{P} \ar{r}[swap]{\finc{x}{P}} & \ctxext{A}{P}
\end{tikzcd}
\end{equation*}
in context $\Gamma$.
The upper arrow is just the identity morphism over $\finc{x}{P}$.

\begin{lem}\label{lem:finc-precomp}
Composition with $\finc{x}{P}$ is (the action on families of)
substitution with $x$:
\begin{align*}
& \inference
  { \jterm{\Gamma}{A}{x}
    \jfam{{{\Gamma}{A}}{P}}{Q}
    }
  { \jfameq
      {{\Gamma}{\subst{x}{P}}}
      {\jcomp{\subst{x}{P}}{\finc{x}{P}}{Q}}
      {\subst{x}{Q}}
    }
  \\
& \inference
  { \jterm{\Gamma}{A}{x}
    \jfam{{{{\Gamma}{A}}{P}}{Q}}{R}
    }
  { \jfameq
      {{{\Gamma}{\subst{x}{P}}}{\subst{x}{Q}}}
      {\jcomp{\subst{x}{P}}{\finc{x}{P}}{R}}
      {\subst{x}{R}}
    }
  \\
& \inference
  { \jterm{\Gamma}{A}{x}
    \jterm{{{{\Gamma}{A}}{P}}{Q}}{R}{h}
    }
  { \jtermeq
      {{{\Gamma}{\subst{x}{P}}}{\subst{x}{Q}}}
      {\subst{x}{R}}
      {\jcomp{\subst{x}{P}}{\finc{x}{P}}{h}}
      {\subst{x}{h}}
    }
\end{align*}
\end{lem}

\begin{proof}
We only prove the validity of the first inference rule. Let $\jterm{\Gamma}{A}{x}$
be a term and let $\jfam{{{\Gamma}{A}}{P}}{Q}$ be a family. Then we have the
judgmental equalities
\begin{align*}
\jcomp{\subst{x}{P}}{\finc{x}{P}}{Q}
& \jdeq
  \unfold{\jcomp{\subst{x}{P}}{\unfold{\finc{x}{P}}}{Q}}
  \tag{by definition}
  \\
& \jdeq
  \subst
    {\idtm{\subst{x}{P}}}
    {{\ctxwk{\subst{x}{P}}{x}}{\ctxwk{\subst{x}{P}}{Q}}}
  \tag{by \autoref{comp-es}}
  \\
& \jdeq
  \subst
    {\idtm{\subst{x}{P}}}
    {\ctxwk{\subst{x}{P}}{\subst{x}{Q}}}
  \tag{by \autoref{comp-ws-f}}
  \\
& \jdeq
  \subst{x}{Q}.
  \tag{by \autoref{precomp-idtm-t}}
\end{align*}
\end{proof}

\begin{lem}
For any $\jterm{\Gamma}{A}{x}$ and any $\jfam{{\Gamma}{A}}{P}$, the square
\begin{equation*}
\begin{tikzcd}
\subst{x}{P} \ar{d} \ar{r}{\finc{x}{P}} & \ctxext{A}{P} \ar{d}{\cprojfstf{A}{P}}\\
\emptyf \ar{r}{\ctxwk{\emptyf}{x}} & A
\end{tikzcd}
\end{equation*}
is a pullback square.
\end{lem}

We can give a second characterization of fiber inclusions:

\begin{lem}\label{lem:finc-char2}
We have the following inference rule
\begin{equation*}
\inference
  { \jterm{\Gamma}{A}{x}
    \jfam{{\Gamma}{A}}{P}
    }
  { \jhomeq{\Gamma}{\subst{x}{P}}{{A}{P}}{\finc{x}{P}}{\subst{x}{\idtm{{A}{P}}}}
    } 
\end{equation*}
\end{lem}

\begin{proof}
The proof is the following calculation:
\begin{align*}
\finc{x}{P}
& \jdeq 
  \tmext{\ctxwk{\subst{x}{P}}{x}}{\idtm{\subst{x}{P}}}
  \tag{by definition}
  \\
& \jdeq
  \subst
    { \idtm{\subst{x}{P}}
      }
    { { \ctxwk{\subst{x}{P}}{x}
        }
      { \idtm{\ctxext{\ctxwk{\subst{x}{P}}{A}}{\ctxwk{\subst{x}{P}}{P}}}
        }
      }
  \tag{by definition}
  \\
& \jdeq
  \subst
    { \idtm{\subst{x}{P}}
      }
    { { \ctxwk{\subst{x}{P}}{x}
        }
      { \idtm{\ctxwk{\subst{x}{P}}{\ctxext{A}{P}}}
        }
      }
  \tag{by \autoref{comp-we-f}}
  \\
& \jdeq
  \subst
    { \idtm{\subst{x}{P}}
      }
    { { \ctxwk{\subst{x}{P}}{x}
        }
      { \ctxwk{\subst{x}{P}}{\idtm{\ctxext{A}{P}}}
        }
      }
  \tag{by \autoref{comp-wi-t}}
  \\
& \jdeq
  \subst
    { \idtm{\subst{x}{P}}
      }
    { \ctxwk
        { \subst{x}{P}
          }
        { \subst{x}{\idtm{\ctxext{A}{P}}}
          }
      }
  \tag{by \autoref{comp-ws-t}}
  \\
& \jdeq
  \subst{x}{\idtm{\ctxext{A}{P}}}.
  \tag{by \autoref{precomp-idtm-t}}
\end{align*}
\end{proof}

\begin{cor}
The following inference rule is valid:
\begin{equation*}
\inference
  { \jterm{\Gamma}{A}{x}
    \jterm{\Gamma}{\subst{x}{P}}{u}
    }
  { \jtermeq{\Gamma}{{A}{P}}{\subst{u}{\finc{x}{P}}}{\tmext{x}{u}}
    }
\end{equation*}
\end{cor}

We have the following lemmas expressing the compatibility of the fiber
inclusions with extension, weakening and substitution. 

\begin{lem}
The fiber inclusions are compatible with extension; i.e.~the following inference
rule is valid
\begin{equation*}
\inference
  { \jterm{\Gamma}{A}{x}
    \jterm{\Gamma}{\subst{x}{P}}{u}
    \jfam{{{\Gamma}{A}}{P}}{Q}
    }
  { \jhomeq
      {\Gamma}
      {\subst{\tmext{x}{u}}{Q}}
      {{{A}{P}}{Q}}
      {\finc{\tmext{x}{u}}{Q}}
      {\jcomp{}{\finc{u}{\subst{x}{Q}}}{\finc{x}{\ctxext{P}{Q}}}}
    }
\end{equation*}
\end{lem}

\begin{proof}
Let $\jterm{\Gamma}{A}{x}$, $\jterm{\Gamma}{\subst{x}{P}}{u}$ and
$\jfam{{{\Gamma}{A}}{P}}{Q}$. Then we have the judgmental equalities
\begin{align*}
\finc{\tmext{x}{u}}{Q}
& \jdeq
  \subst{\tmext{}{}{x}{u}}{\idtm{{{A}{P}}{Q}}}
  \tag{by \autoref{lem:finc-char2}}
  \\
& \jdeq
  \subst{u}{{x}{\idtm{{{A}{P}}{Q}}}}
  \tag{by \autoref{comp-es}}
  \\
& \jdeq
  \subst{u}{{x}{\idtm{{A}{{P}{Q}}}}}
  \tag{by \autoref{comp-ee-c}}
  \\
& \jdeq
  \subst{u}{\finc{x}{\ctxext{P}{Q}}}
  \tag{by \autoref{lem:finc-char2}}
  \\
& \jdeq
  \jcomp{}{\finc{u}{\subst{x}{Q}}}{\finc{x}{\ctxext{P}{Q}}}
  \tag{by \autoref{lem:finc-precomp}}
\end{align*}
\end{proof}

\begin{lem}
The fiber inclusions are compatible with weakening; i.e.~the following inference
rule is valid
\begin{equation*}
\inference
  { \jfam{\Gamma}{A}
    \jterm{\Gamma}{B}{y}
    \jfam{{\Gamma}{B}}{Q}
    }
  { \jhomeq
      {{\Gamma}{A}}
      {\subst{\ctxwk{A}{y}}{\ctxwk{A}{Q}}}
      {{\ctxwk{A}{B}}{\ctxwk{A}{Q}}}
      {\finc{\ctxwk{A}{y}}{\ctxwk{A}{Q}}}
      {\ctxwk{A}{\finc{y}{Q}}}
    }
\end{equation*}
\end{lem}

\begin{proof}
Let $\jterm{\Gamma}{B}{y}$ and $\jfam{{\Gamma}{B}}{Q}$. Then we have the
judgmental equalities
\begin{align*}
\finc{\ctxwk{A}{y}}{\ctxwk{A}{Q}}
& \jdeq
  \subst{\ctxwk{A}{y}}{\idtm{\ctxext{\ctxwk{A}{B}}{\ctxwk{A}{Q}}}}
  \tag{by \autoref{lem:finc-char2}}
  \\
& \jdeq
  \subst{\ctxwk{A}{y}}{\idtm{\ctxwk{A}{\ctxext{B}{Q}}}}
  \tag{by \autoref{comp-we-f}}
  \\
& \jdeq
  \subst{\ctxwk{A}{y}}{\ctxwk{A}{\idtm{\ctxext{B}{Q}}}}
  \tag{by \autoref{comp-wi-t}}
  \\
& \jdeq
  \ctxwk{A}{\subst{y}{\idtm{{B}{Q}}}}
  \tag{by \autoref{comp-ws-t}}
  \\
& \jdeq
  \ctxwk{A}{\finc{y}{Q}}.
  \tag{by \autoref{lem:finc-char2}}
\end{align*}
\end{proof}

\begin{lem}
The fiber inclusions are compatible with substitution; i.e.~the following
inference rule is valid
\begin{equation*}
\inference
  { \jterm{\Gamma}{A}{x}
    \jfam{{{\Gamma}{A}}{P}}{Q}
    \jterm{{\Gamma}{A}}{P}{f}
    }
  { \jhomeq
      {\Gamma}
      {\subst{{x}{f}}{{x}{Q}}}
      {{\subst{x}{P}}{\subst{x}{Q}}}
      {\finc{\subst{x}{f}}{\subst{x}{Q}}}
      {\subst{x}{\finc{f}{Q}}}
    }
\end{equation*}
\end{lem}

\begin{proof}
Let $\jterm{\Gamma}{A}{x}$, $\jterm{{\Gamma}{A}}{P}{f}$ and 
$\jfam{{{\Gamma}{A}}{P}}{Q}$. Then we have the judgmental equalities
\begin{align*}
\finc{\subst{x}{f}}{\subst{x}{Q}}
& \jdeq
  \subst{x}{{f}{\idtm{{P}{Q}}}}
  \tag{by \autoref{lem:finc-char2}}
  \\
& \jdeq
  \subst{{x}{f}}{{x}{\idtm{{P}{Q}}}}
  \tag{by \autoref{comp-ss-t}}
  \\
& \jdeq
  \subst{{x}{f}}{\idtm{\subst{x}{\ctxext{P}{Q}}}}
  \tag{by \autoref{comp-si-t}}
  \\
& \jdeq
  \subst{{x}{f}}{\idtm{\ctxext{\subst{x}{P}}{\subst{x}{Q}}}}
  \tag{by \autoref{comp-se-f}}
  \\
& \jdeq
  \finc{\subst{x}{f}}{\subst{x}{Q}}.
  \tag{by \autoref{lem:finc-char2}}
\end{align*}
\end{proof}

\begin{lem}
The fiber inclusions are compatible with identity terms; i.e.~the following
inference rule is valid
\begin{equation*}
\inference
  { \jfam{{\Gamma}{A}}{P}
    }
  { \jhomeq
      {{\Gamma}{A}}
      {P}
      {\ctxwk{A}{\ctxext{A}{P}}}
      {\finc{\idtm{A}}{\ctxwk{A}{P}}}
      {\idtm{{A}{P}}}
    }
\end{equation*}
\end{lem}

\begin{proof}
Let $\jfam{\Gamma}{A}$. Then we have the judgmental equalities
\begin{align*}
\finc{\idtm{A}}{\ctxwk{A}{P}}
& \jdeq
  \subst{\idtm{A}}{\idtm{\ctxext{\ctxwk{A}{A}}{\ctxwk{A}{P}}}}
  \tag{by \autoref{lem:finc-char2}}
  \\
& \jdeq
  \subst{\idtm{A}}{\idtm{\ctxwk{A}{\ctxext{A}{P}}}}
  \tag{by \autoref{comp-we-f}}
  \\
& \jdeq
  \subst{\idtm{A}}{\ctxwk{A}{\idtm{{A}{P}}}}
  \tag{by \autoref{comp-wi-t}}
  \\
& \jdeq
  \idtm{{A}{P}}
  \tag{by \autoref{precomp-idtm-t}}
\end{align*}
\end{proof}

\begin{comment}
\subsection{Another special case of projections}
In this subsection we investigate the special case of a projection which
appears as a morphism from $\ctxext{{A}{P}}{\ctxwk{P}{Q}}$ to $\ctxext{A}{Q}$
in context $\Gamma$, where we assume to have the families 
$\jfam{{\Gamma}{A}}{P}$ and $\jfam{{\Gamma}{A}}{Q}$. 

Note that we have the judgmental
equalities
\begin{align*}
\ctxwk{\ctxext{{A}{P}}{\ctxwk{P}{\mfam{A}}}}{\ctxext{A}{\mfam{A}}}
& \jdeq 
  \ctxext
    {\ctxwk{\ctxext{{A}{P}}{\ctxwk{P}{\mfam{A}}}}{A}}
    {\ctxwk{\ctxext{{A}{P}}{\ctxwk{P}{\mfam{A}}}}{\mfam{A}}}
  \\
& \jdeq
  \ctxext
    {\ctxwk{{P}{\mfam{A}}}{{\ctxext{A}{P}}{A}}}
    {\ctxwk{\ctxext{{A}{P}}{\ctxwk{P}{\mfam{A}}}}{\mfam{A}}}
\end{align*}
Note that we have the term $\ctxwk{{P}{\mfam{A}}}{\cprojfstf{A}{P}}$ of the
family $\ctxwk{{P}{\mfam{A}}}{{\ctxext{A}{P}}{A}}$. Therefore, we need to
find a term of type $\subst{\ctxwk{{P}{\mfam{A}}}{\cprojfstf{A}{P}}}
{\ctxwk{\ctxext{{A}{P}}{\ctxwk{P}{\mfam{A}}}}{\mfam{A}}}$. Note that we have
the judgmental equalities:
\begin{align*}
\subst
  {\ctxwk{{P}{\mfam{A}}}{\cprojfstf{A}{P}}}
  {\ctxwk{\ctxext{{A}{P}}{\ctxwk{P}{\mfam{A}}}}{\mfam{A}}}
& \jdeq
  \subst
    {\ctxwk{{P}{\mfam{A}}}{\cprojfstf{A}{P}}}
    {\ctxwk{{P}{\mfam{A}}}{{\ctxext{A}{P}}{\mfam{A}}}}
  \\
& \jdeq
  \ctxwk
    { {P}{\mfam{A}}
      }
    { \subst
        {\cprojfstf{A}{P}}
        {\ctxwk{\ctxext{A}{P}}{\mfam{A}}}
      }
  \\
& \jdeq
  \ctxwk
    { {P}{\mfam{A}}
      }
    { {P}{\mfam{A}}
      }
  \\
& \jdeq
  \ctxwk{P}{{\mfam{A}}{\mfam{A}}}
\end{align*}
We find the term $\ctxwk{P}{\idtm{\mfam{A}}}$ here. Thus we can now define
$\bar{\typefont{pr}}$ by:
\begin{equation}\label{barproj}
\jhomdefn
  {\Gamma}
  {{{A}{P}}{\mfam{A}}}
  {{A}{\mfam{A}}}
  {\bar{\typefont{pr}}}
  {\tmext{\ctxwk{{P}{\mfam{A}}}{\cprojfstf{A}{P}}}{\ctxwk{P}{\idtm{\mfam{A}}}}}
\end{equation}

\begin{lem}
We have the judgmental equality
\begin{equation*}
\jfameq
  {{{{\Gamma}{A}}{P}}{\ctxwk{P}{\mfam{A}}}}
  {\jcomp{}{\bar{\typefont{pr}}}{Q}}
  {\ctxwk{P}{Q}}
\end{equation*}
for any family $Q$ of contexts over $\ctxext{{\Gamma}{A}}{\mfam{A}}$ 
\end{lem}
\end{comment}

%%%%%%%%%%%%%%%%%%%%%%%%%%%%%%%%%%%%%%%%%%%%%%%%%%%%%%%%%%%%%%%%%%%%%%%%%%%%%%%%
\subsection{The terms of the empty families}
In this subsection we will define and derive properties of the term of the empty
family in a context. We will leave most proofs to the reader.

For the following definition, note that we have the judgmental equalities
$\jctxeq{\Gamma}{{\Gamma}{\emptyf}}$ and 
$\jfameq{\Gamma}{\ctxwk{\emptyf}{\emptyf}}{\emptyf}$.

\begin{defn}
Let $\Gamma$ be a context. We define
\begin{equation*}
\jtermdefn{\Gamma}{\emptyf}{\emptytm}{\idtm{\emptyf}}
\end{equation*}
\end{defn}

\begin{rmk}
There seems to be no way to derive that this term is
judgmentally unique. Also, there seems to be no reason we need that. This has
been a suggestion of Voevodsky and Garner.
\end{rmk}

We will first show that the terms of the empty families are compatible with
weakening and substitution before we show that they are compatible with
extension on terms.

\begin{lem}\label{lem:emptytm-wk}
The following inference rule expressing that the terms of the empty families are
compatible with weakening is valid:
\begin{equation*}
\inference
  { \jfam{\Gamma}{A}
    }
  { \jtermeq
      {{\Gamma}{A}}
      {\emptyf}
      {\ctxwk{A}{\emptytm}}
      {\emptytm}
    }
\end{equation*}
\end{lem}

\begin{lem}\label{lem:emptytm-subst}
The following inference rule expressing that the terms of the empty families are
compatible with substitutuion is valid:
\begin{equation*}
\inference
  { \jterm{\Gamma}{A}{x}
    }
  { \jtermeq
      {\Gamma}
      {\emptyf}
      {\subst{x}{\emptytm}}
      {\emptytm}
    }
\end{equation*}
\end{lem}

\begin{lem}\label{comp-0s}
The following rules asserting that substituting by the term $\jterm{\Gamma}{\emptyf}{\emptytm}$
leaves everything unchanged are valid:
\begin{align}
& \inference
  { \jfam{\Gamma}{A}
    }
  { \jfameq{\Gamma}{\subst{\emptytm}{A}}{A}
    }
  \\
%& \inference
%  { \jterm{\Gamma}{A}{x}
%    }
%  { \jtermeq{\Gamma}{A}{\subst{\emptytm}{x}}{x}
%    }
%  \\
& \inference
  { \jfam{{\Gamma}{A}}{P}
    }
  { \jfameq{{\Gamma}{A}}{\subst{\emptytm}{P}}{P}
    }
  \\
& \inference
  { \jterm{{\Gamma}{A}}{P}{f}
    }
  { \jtermeq{{\Gamma}{A}}{P}{\subst{\emptytm}{f}}{f}
    }.
\end{align}
\end{lem}

\begin{proof}
This follows directly from the properties in \autoref{comp-0w,cancellation-i}.
\end{proof}

\begin{lem}\label{lem:tmext-emp}
The following compatibility rules for extensions of the term of the empty family
are valid:
\begin{align*}
& \inference
  { \jterm{\Gamma}{A}{x}
    }
  { \jtermeq{\Gamma}{A}{\tmext{\emptytm}{x}}{x}
    }
& & \inference
  { \jterm{\Gamma}{A}{x}
    }
  { \jtermeq{\Gamma}{A}{\tmext{x}{\emptytm}}{x}
    }
  \\
& \inference
  { \jfam{\Gamma}{A}
    }
  { \jtermeq
      {{\Gamma}{A}}
      {\emptyf}
      {\cprojfstf{\emptyf}{A}}
      {\emptytm}
    }
& & \inference
  { \jfam{\Gamma}{A}
    }
  { \jhomeq
      {\Gamma}
      {A}
      {A}
      {\cprojfstf{A}{\emptyf}}
      {\idtm{A}}
    }
  \\
& \inference
  { \jfam{\Gamma}{A}
    }
  { \jhomeq
      {\Gamma}
      {A}
      {A}
      {\cprojsndf{\emptyf}{A}}
      {\idtm{A}}
    }
& & \inference
  { \jfam{\Gamma}{A}
    }
  { \jtermeq
      {{\Gamma}{A}}
      {\emptyf}
      {\cprojsndf{A}{\emptyf}}
      {\emptytm}
    }
\end{align*}
\end{lem}

\begin{proof}
We only display a proof of the first:
\begin{equation*}
\tmext{\emptytm}{x}
\jdeq \unfold{\tmext{\emptyf}{A}{\emptytm}{x}}
\jdeq \subst{x}{\idtm{{\emptyf}{A}}}
\jdeq \subst{x}{\idtm{A}}
\jdeq x.\qedhere
\end{equation*}
\end{proof}

\begin{lem}\label{lem:finc-emp}
The fiber inclusions are compatible with the empty families; i.e.~the following
inference rules are valid
\begin{align*}
& \inference
  { \jterm{\Gamma}{A}{x}
    }
  { \jtermeq
      {\Gamma}
      {A}
      {\finc{x}{\emptyf}}
      {x}
    }
  \\
& \inference
  { \jfam{\Gamma}{A}
    }
  { \jhomeq
      {\Gamma}
      {A}
      {A}
      {\finc{\emptytm}{A}}
      {\idtm{A}}
    }
\end{align*}
\end{lem}

\begin{proof}
Let $\jterm{\Gamma}{A}{x}$. We have the judgmental equalities
\begin{align*}
\finc{x}{\emptyf}
& \jdeq
  \unfold{\finc{x}{\emptyf}}
  \tag{by definition}
  \\
& \jdeq
  \tmext{\ctxwk{\emptyf}{x}}{\idtm{\emptyf}}
  \tag{by \autoref{comp-s0-c}}
  \\
& \jdeq
  \tmext{x}{\idtm{\emptyf}}
  \tag{by \autoref{comp-0w-t}}
  \\
& \jdeq
  \tmext{x}{\emptytm}
  \tag{by definition}
  \\
& \jdeq
  x
  \tag{by \autoref{lem:tmext-emp}}
\end{align*}
to prove the first judgmental equality. For the second, we have
\begin{align*}
\finc{\emptytm}{A}
& \jdeq
  \unfold{\finc{\emptytm}{A}}
  \tag{by definition}
  \\
& \jdeq
  \tmext{\emptytm}{\idtm{\subst{\emptytm}{A}}}
  \tag{by \autoref{lem:emptytm-wk}}
  \\
& \jdeq
  \idtm{\subst{\emptytm}{A}}
  \tag{by \autoref{lem:tmext-emp}}
  \\
& \jdeq
  \idtm{A}
  \tag{by \autoref{comp-0s}}.
\end{align*}
\end{proof}

\subsection{Morphisms with extended domains}\label{sec:ehom}

We cover some properties of morphisms with extended domains because we will
need them later on.

\begin{rmk}
Suppose $\jhom{\Gamma}{{A}{P}}{B}{f}$ is a morphism with an extended domain.
Then the inference rules
\begin{align*}
& \inference
  { \jterm{\Gamma}{A}{x}
    }
  { \jhom{\Gamma}{\subst{x}{P}}{B}{\subst{x}{f}}
    }
  \\
& \inference
  { \jterm{{\Gamma}{A}}{P}{y}
    }
  { \jhom{\Gamma}{A}{B}{\subst{y}{f}}
    }
\end{align*}
are valid.
This follows from the observation in \autoref{lem:jhomdomext-jhomcodwk} 
that a morphism from
$\ctxext{A}{P}$ to $B$ in context $\Gamma$ is the same thing as a morphism
from $P$ to $\ctxwk{A}{B}$ in context $\ctxext{\Gamma}{A}$.
\end{rmk}

\begin{lem}\label{lem:ehom-subst}
Let $\jhom{\Gamma}{{A}{P}}{B}{f}$ be a morphism and let $\jterm{\Gamma}{A}{x}$
be a term. The following three inference rules are valid
\begin{align*}
& \inference
  { \jfam{{\Gamma}{B}}{Q}
    }
  { \jfameq
      {{\Gamma}{\subst{x}{P}}}
      {\subst{x}{\jcomp{}{f}{Q}}}
      {\jcomp{}{\subst{x}{f}}{Q}}
    }
  \\
& \inference
  { \jfam{{{\Gamma}{B}}{Q}}{R}
    }
  { \jfameq
      {{{\Gamma}{\subst{x}{P}}}{\jcomp{}{\subst{x}{f}}{Q}}}
      {\subst{x}{\jcomp{}{f}{R}}}
      {\jcomp{}{\subst{x}{f}}{R}}
    }
  \\
& \inference
  { \jterm{{{\Gamma}{B}}{Q}}{R}{h}
    }
  { \jtermeq
      {{{\Gamma}{\subst{x}{P}}}{\jcomp{}{\subst{x}{f}}{Q}}}
      {\jcomp{}{\subst{x}{f}}{R}}
      {\subst{x}{\jcomp{}{f}{h}}}
      {\jcomp{}{\subst{x}{f}}{h}}
    }
\end{align*}
\end{lem}

\begin{proof}
Let $Q$ be a family over $\ctxext{\Gamma}{A}$.
We have the judgmental equalities
\begin{align*}
\subst{x}{\jcomp{}{f}{Q}}
& \jdeq
  \subst
    {x}
    {{f}{\ctxwk{\ctxext{A}{P}}{Q}}}
  \tag{by definition}
  \\
& \jdeq
  \subst
    { {x}{f}
      }
    { {x}
      {\ctxwk{\ctxext{A}{P}}{Q}}
      }
  \tag{by \autoref{comp-ss-f}}
  \\
& \jdeq
  \subst
    { {x}{f}
      }
    { {x}
      {\ctxwk{P}{{A}{Q}}}
      }
  \tag{by \autoref{comp-ew-f}}
  \\
& \jdeq
  \subst
    { {x}{f}
      }
    { \ctxwk
        {\subst{x}{P}}
        {\subst{x}{\ctxwk{A}{Q}}}
      }
  \tag{by \autoref{comp-sw-f}}
  \\
& \jdeq
  \subst
    { {x}{f}
      }
    { \ctxwk
        {\subst{x}{P}}
        {Q}
      }
  \tag{by \autoref{cancellation-ws-f}}
  \\
& \jdeq
  \jcomp{}{\subst{x}{f}}{Q}.
  \tag{by definition}
\end{align*}
\begin{comment}
In the case where $\xi$ is $y$ we have the judgmental equalities
\begin{align*}
\subst{y}{\jcomp{}{f}{Q}}
& \jdeq
  \subst
    { y
      }
    { {f}
      {\ctxwk{\ctxext{\cftalgc{\cftalg{A}}}{\cftalgf{\cftalg{A}}}}{Q}}
      }
  \tag{by definition}
  \\
& \jdeq
  \subst
    { {y}{f}
      }
    { {y}
      {\ctxwk{\ctxext{\cftalgc{\cftalg{A}}}{\cftalgf{\cftalg{A}}}}{Q}}
      }
  \tag{by \autoref{comp-ss-f}}
  \\
& \jdeq
  \subst
    { {y}{f}
      }
    { {y}
      {\ctxwk{\cftalgf{\cftalg{A}}}{{\cftalgc{\cftalg{A}}}{Q}}}
      }
  \tag{by \autoref{comp-ew-f}}
  \\
& \jdeq
  \subst
    { {y}{f}
      }
    { \ctxwk{\cftalgc{\cftalg{A}}}{Q}
      }
  \tag{by \autoref{cancellation-ws-f}}
  \\
& \jdeq
  \jcomp{}{\subst{y}{f}}{Q}.
  \tag{by definition}
\end{align*}
\end{comment}
\end{proof}


\begin{lem}\label{lem:edom-edom-comp}
Consider a morphism $\jhom{{\Gamma}{A}}{{P}{Q}}{R}{f}$ and a morphism
$\jhom{\Gamma}{{A}{R}}{B}{g}$. Then the diagram
\begin{equation*}
\begin{tikzcd}
\ctxext{A}{{P}{Q}}
  \ar{r}{\jvcomp{}{\idtm{A}}{f}}
  \ar{dr}[swap]{\jcomp{}{f}{g}}
  &
\ctxext{A}{R}
  \ar{d}{g}
  \\
  &
B
\end{tikzcd}
\end{equation*}
in context $\Gamma$ commutes judgmentally.
\end{lem}

\begin{proof}
It is just a triviality, as we have the judgmental equalities
\begin{align*}
\jcomp{}{f}{g}
& \jdeq
  \jcomp{}{f}{\jcomp{}{\idtm{A}}{g}}
  \tag{by \autoref{precomp-idtm-t}}
  \\
& \jdeq
  \jcomp{}{\jvcomp{}{\idtm{A}}{f}}{g}.
  \tag{by \autoref{lem:composition-threesome}}
\end{align*}
Note that we may apply \autoref{lem:composition-threesome} because by
\autoref{rmk:jhom-as-jfhom} every ordinary morphism is a morphism over the
identity morphism.
\end{proof}


%\part{Internal Models}\label{part:models}

%\section{Internalizing the theory of contexts families and terms}
One of the guiding ideas behind the design of the theory of contexts, families
and terms was that it would have to be possible to consider internal versions
of the theory. In this section we aim for this internalization. We stress that
we shall not make any further assumptions in this section, and thus that
it is \emph{by default} possible to consider internal models of the theory
of contexts, families and terms in itself. In particular, we do not assume that
there are universes; this is the subject of a later section in this part.

It would be interesting to write out a weak version of pre-universes, internal
to Martin-L\"of type theory with the function extensionality principle. 
To do this, the empty context needs to
be replaced by a contractible type, extension by dependent pair types,
judgmental equalities of terms by identifications and judgmental equalities
of families by equivalences of types. We conjecture that it is possible to
carry this out (in particular to make sure that all the constructions 
type-check). This could serve as a starting point for investigating internal
models without truncatedness assumptions and for investigating internal higher
categories. 
Moreover, one could then extend the notion of `weak' pre-universes
with the requirement that every internal morphism is weakly anodyne. This
could give an internal theory of weak higher groupoids.

\subsection{Extension algebras}\label{sec:extension-algebras}
In this subsection our goal is to define the notion of extension algebras,
which are internal versions of the extension operation of the theory of
contexts, families and terms. In this article, their use will be mainly in
universes. The theory of extension algebras requires the full power (i.e.~all
of the ingredients) of the theory of contexts, families and terms in its
formulation and it is (perhaps surprisingly) quite involved to formulate it.
The definition of extension algebras is given in \autoref{defn:extension-algebras}.

Let $P$ be a family over an extended context $\ctxext{\Gamma}{A}$. We could
mimic extension by requiring to have terms
\begin{align*}
\jalign\jhom{\Gamma}{{A}{P}}{A}{e_0}\\
\jalign\jhom{{\Gamma}{A}}{{P}{\jcomp{}{e_0}{P}}}{P}{e_1}.
\end{align*}

\begin{rmk}
Instead of looking at $e_1$ as a context morphism from $\ctxext{P}{\jcomp{}{e_0}{P}}$
to $P$ in context $\ctxext{\Gamma}{A}$, one could also look at $e_1$ as a 
morphism \emph{over $\cprojfstf{A}{P}$} in context $\Gamma$, as indicated in the
following diagram:
\begin{equation*}
\begin{tikzcd}
P
  \ar[fib]{d}
& \jcomp{}{e_0}{P}
  \ar[fib]{d}
  \ar{l}[swap]{e_1}
  \ar{r}
& P
  \ar[fib]{d}
  \\
A
& \ctxext{A}{P}
  \ar{l}{\cprojfstf{A}{P}}
  \ar{r}[swap]{e_0}
& A
\end{tikzcd}
\end{equation*}
This makes it clear that $e_1$ takes a family over an extended context as an
argument. The extended context consists of a `base part' and a `family part'. 
The output of $e_1$ is a new (extended) family over that base part. Forgetting 
the family part is what the projection takes care of.
\end{rmk}

Extension also satisfies the properties explained in \autoref{comp-ee}, so we
must find the two judgmental equalities for $e_0$ and $e_1$ interpreting those. 
The first of these judgmental equalities is easy to give: it says that the
following diagram in context $\Gamma$ commutes judgmentally:
\begin{equation}\label{eq:extalg-eq1}
\begin{tikzcd}[column sep=huge]
\ctxext{A}{{P}{\jcomp{}{e_0}{P}}} 
  \ar{d}[swap]{\jvcomp{}{e_0}{\idtm{\jcomp{}{e_0}{P}}}
    } 
  \ar{r}{\jvcomp{}{\idtm{A}}{e_1}
    } 
  & \ctxext{A}{P} \ar{d}{e_0}\\
\ctxext{A}{P} \ar{r}[swap]{e_0} & A
\end{tikzcd}
\end{equation}

In the following lemma we find an equivalent reformulation of the condition in 
\autoref{eq:extalg-eq1}.

\begin{lem}\label{lem:extalg-twins}
Suppose we have $\jfam{\Gamma}{A}$, $\jfam{{\Gamma}{A}}{P}$, 
$\jhom{\Gamma}{{A}{P}}{A}{e_0}$ and 
$\jhom{{\Gamma}{A}}{{P}{\jcomp{}{P}{e_0}}}{P}{e_1}$. 
Then the inference rules
\begin{align*}
& \inference
  { }
  { \jhomeq
      {\Gamma}
      {\ctxext{A}{{P}{\jcomp{}{e_0}{P}}}}
      {A}
      {\jcomp{}{\jvcomp{}{e_0}{\idtm{\jcomp{}{e_0}{P}}}}{e_0}}
      {\jcomp{}{e_0}{e_0}}
    }
\intertext{and}
& \inference
  { }
  { \jhomeq
      {\Gamma}
      {\ctxext{A}{{P}{\jcomp{}{e_0}{P}}}}
      {A}
      {\jcomp{}{\jvcomp{}{\idtm{A}}{e_1}}{e_0}}
      {\jcomp{}{e_1}{e_0}}
    }
\end{align*}
As a consequence, the inference rules
\begin{align*}
& \inference
  { \jhomeq
      {\Gamma}
      {\ctxext{A}{{P}{\jcomp{}{e_0}{P}}}}
      {A}
      {\jcomp{}{\jvcomp{}{e_0}{\idtm{\jcomp{}{e_0}{P}}}}{e_0}}
      {\jcomp{}{\jvcomp{}{\idtm{A}}{e_1}}{e_0}}
    }
  { \jhomeq
      {\Gamma}
      {\ctxext{A}{{P}{\jcomp{}{e_0}{P}}}}
      {A}
      {\jcomp{}{e_0}{e_0}}
      {\jcomp{}{e_1}{e_0}}
    }
\intertext{and}
& \inference
  { \jhomeq
      {\Gamma}
      {\ctxext{A}{{P}{\jcomp{}{e_0}{P}}}}
      {A}
      {\jcomp{}{e_0}{e_0}}
      {\jcomp{}{e_1}{e_0}}
    }
  { \jhomeq
      {\Gamma}
      {\ctxext{A}{{P}{\jcomp{}{e_0}{P}}}}
      {A}
      {\jcomp{}{\jvcomp{}{e_0}{\idtm{\jcomp{}{e_0}{P}}}}{e_0}}
      {\jcomp{}{\jvcomp{}{\idtm{A}}{e_1}}{e_0}}
    }
\end{align*}
are valid.
\end{lem}

\begin{proof}
To prove the first inference rule, note that we have the judgmental equalities
\begin{align*}
\jcomp{}{\jvcomp{}{e_0}{\idtm{\jcomp{}{e_0}{P}}}}{e_0}
& \jdeq
  \jcomp{}{\idtm{\jcomp{}{e_0}{P}}}{\jcomp{}{e_0}{e_0}}
  \tag{by \autoref{lem:composition-threesome}}
  \\
& \jdeq
  \jcomp{}{e_0}{e_0}
  \tag{by \autoref{precomp-idtm-t}}
\end{align*}
The second inference rule follows immediately from \autoref{lem:edom-edom-comp}.
\end{proof}

The second condition we will put on context and family extension essentially 
asserts the quadruple $(P,\jcomp{}{e_0}{P},e_1,\jcomp{}{e_0}{e_1})$ also satisfies 
\autoref{eq:extalg-eq1}. Here we consider `higher' families, i.e.~families over
families over families, which are described by the family 
$\jcomp{}{e_0}{P}$ over $\ctxext{{\Gamma}{A}}{P}$. The family
extension $e_1$ becomes context extension when we consider the quadruple
$(P,\jcomp{}{e_0}{P},e_1,\jcomp{}{e_0}{e_1})$ and likewise, the morphism
$\jcomp{}{e_0}{e_1}$ becomes family extension. Thus, the second requirement is 
that the diagram
\begin{equation}\label{eq:extalg-eq2}
\begin{tikzcd}[column sep=huge]
\ctxext{P}{{\jcomp{}{e_0}{P}}{\jcomp{}{e_1}{\jcomp{}{e_0}{P}}}} 
  \ar{d}[swap]{\jvcomp{}{e_1}{\idtm{\jcomp{}{e_1}{\jcomp{}{e_0}{P}}}}
    } 
  \ar{r}{\jvcomp{}{\idtm{P}}{\jcomp{}{e_0}{e_1}}
    } 
  & \ctxext{P}{\jcomp{}{e_0}{P}} \ar{d}{e_1}\\
\ctxext{P}{\jcomp{}{e_0}{P}} \ar{r}[swap]{e_1} & P
\end{tikzcd}
\end{equation}

\begin{defn}\label{defn:extension-algebras}
An \emph{extension algebra $\extalg{A}$ in context $\Gamma$} is a quintuple
\begin{equation*}
\unfold{\extalg{A}}
\end{equation*}
consisting of a family $\jextalgctx{\Gamma}{A}$ of \emph{$\extalg{A}$-contexts}, 
a family $\jextalgfam{\Gamma}{A}$ of \emph{$\extalg{A}$-families}, a family 
$\jextalgtm{\Gamma}{A}$ of \emph{$\extalg{A}$-terms}, a term
$\jextalgctxext{\Gamma}{A}$ called the \emph{$\extalg{A}$-context extension}
and a term $\jextalgfamext{\Gamma}{A}$ called the \emph{$\extalg{A}$-family
extension}, satisfying the judgmental equalities of
\autoref{eq:extalg-eq1,eq:extalg-eq2}.
\begin{comment}
We define the judgment
\begin{equation*}
\jextalg{\Gamma}{A}
\end{equation*}
asserting that $\extalg{A}$ is an \emph{extension algebra in context $\Gamma$}
to be the conjunction of the following seven judgments:
\begin{align*}
\jalign\jextalgctx{\Gamma}{A}
  \\
\jalign\jextalgfam{\Gamma}{A}
  \\
\jalign\jextalgtm{\Gamma}{A}
  \\
\jalign\jextalgctxext{\Gamma}{A}
  \\
\jalign\jextalgfamext{\Gamma}{A}
  \\
\jalign\jhomeq
  { \Gamma}
  { {\cftalgc{\cftalg{A}}}
    { {\cftalgf{\cftalg{A}}}
      {\jcomp{}{\cftctxext[\cftalg{A}]}{\cftalgf{\cftalg{A}}}}
      }
    }
  { \cftalgc{\cftalg{A}}}
  { \jcomp{}
      { \jvcomp{}
          {\cftctxext[\cftalg{A}]}
          {\idtm{\jcomp{}{\cftctxext[\cftalg{A}]}{\cftalgf{\cftalg{A}}}}}
        }
      { \cftctxext[\cftalg{A}]}
    }
  { \jcomp{}
      { \jvcomp{}
          {\idtm{\cftalgc{\cftalg{A}}}}
          {\cftfamext[\cftalg{A}]}
        }
      { \cftctxext[\cftalg{A}]}
    }
  \\
\jalign\jhomeq
  { {\Gamma}{\cftalgc{\cftalg{A}}}}
  { { \cftalgf{\cftalg{A}}}
    { { \jcomp{}{\cftctxext[\cftalg{A}]}{\cftalgf{\cftalg{A}}}}
      { \jcomp{}
          {\cftctxext[\cftalg{A}]}
          {{}{\cftctxext[\cftalg{A}]}{\cftalgf{\cftalg{A}}}}
        }
      }
    }
  { \cftalgf{\cftalg{A}}}
  { \jcomp{}
      { \cftfamext[\cftalg{A}]}
      { \jvcomp{}
          {\cftfamext[\cftalg{A}]}
          {\idtm{\jcomp{}{\cftfamext[\cftalg{A}]}{{}{\cftctxext[\cftalg{A}]}{\cftalgf{\cftalg{A}}}}}}}
    }
  { \jcomp{}
      { \cftfamext[\cftalg{A}]}
      { \jvcomp{}
          {\idtm{\cftalgf{\cftalg{A}}}}
          {\jcomp{}{\cftctxext[\cftalg{A}]}{\cftfamext[\cftalg{A}]}}
        }
    }
\end{align*}
In other words, an extension algebra $\extalg{A}$ in context $\Gamma$
is a quintuple $\unfold{\extalg{A}}$ 
satisfying the judgmental equalities displayed in the diagrams in
\autoref{eq:extalg-eq1,eq:extalg-eq2}.
\end{comment}
Extension algebras are judgmentally
equal if they are component-wise judgmentally equal.

When $\extalg{A}$ is an extension algebra in context $\Gamma$, we also refer to
$\cftctxext[\cftalg{A}]$ as the \emph{context extension of $\extalg{A}$} and to
$\cftfamext[\cftalg{A}]$ as the \emph{family extension of $\extalg{A}$}.
\end{defn}

The following lemma both provides intuition behind the judgmental equalities
we have required for context and family extension and it proves that context 
and family extension in fact satisfy the compatibility rules stated in 
\autoref{comp-ee}. 

\begin{lem}
For $\gamma:\cftalgc{\cftalg{A}}$, 
$a:\subst{\gamma}{\cftalgf{\cftalg{A}}}$ and 
$p:\subst{\apply\cftctxext[\extalg{A}]{\gamma}{a}}{\cftalgf{\cftalg{A}}}$ 
we have the judgmental equality
\begin{align*}
\apply\cftctxext[\extalg{A}]{\gamma}{\apply\cftfamext[\extalg{A}]{\gamma}{a}{p}}
& \jdeq
  \apply\cftctxext[\extalg{A}]{\apply\cftctxext[\extalg{A}]{\gamma}{a}}{p}.
  \intertext{%
and when we also have 
$q: \subst
      {\apply\cftctxext[\extalg{A}]{\apply\cftctxext[\extalg{A}]{\gamma}{a}}{p}}
      {\cftalgf{\extalg{A}}}$, %
we have the judgmental equality}
\apply\cftfamext[\extalg{A}]{\gamma}{a}{%
  \apply\cftfamext[\extalg{A}]{\apply\cftctxext[\extalg{A}]{\gamma}{a}}{p}{q}}
& \jdeq
  \apply\cftfamext[\extalg{A}]{\gamma}{%
    \apply\cftfamext[\extalg{A}]{\gamma}{a}{p}}{q}.
\end{align*}
\end{lem}

\begin{proof}
Both proofs are simple calculations. For the first judgmental equality we have
\begin{align*}
\apply\cftctxext[\extalg{A}]{\gamma}{\apply\cftfamext[\extalg{A}]{\gamma}{a}{p}}
& \jdeq
  \subst
    {\tmext{\gamma}{\apply\cftfamext[\extalg{A}]{\gamma}{a}{p}}}
    {\cftctxext[\cftalg{A}]}
  \tag{by \autoref{comp-es}}
  \\
& \jdeq 
  \subst
    { {p}
      { {a}
        { {\gamma}
          {\jvcomp{}{\idtm{\cftalgc{\cftalg{A}}}}{\cftfamext[\cftalg{A}]}}
          }
        }
      }
    { \cftctxext[\cftalg{A}]}
  \\
& \jdeq
  \subst{p}{{a}{{\gamma}{\jcomp{}{\jvcomp{}{\idtm{\cftalgc{\cftalg{A}}}}{\cftfamext[\cftalg{A}]}}{\cftctxext[\cftalg{A}]}}}}
  \\
& \jdeq
  \subst{p}{{a}{{\gamma}{\jcomp{}{\jvcomp{}{\cftctxext[\cftalg{A}]}{\idtm{\jcomp{}{\cftctxext[\cftalg{A}]}{\cftalgf{\cftalg{A}}}}}}{\cftctxext[\cftalg{A}]}}}}
  \\
& \jdeq 
  \subst{{p}{{a}{{\gamma}{\jvcomp{}{\cftctxext[\cftalg{A}]}{\idtm{\jcomp{}{\cftctxext[\cftalg{A}]}{\cftalgf{\cftalg{A}}}}}}}}}{\cftctxext[\cftalg{A}]}
  \\
& \jdeq
  \subst
    {\tmext{\apply\cftctxext[\extalg{A}]{\gamma}{a}}{p}}
    {\cftctxext[\cftalg{A}]}
  \\
& \jdeq
  \apply\cftctxext[\extalg{A}]{\apply\cftctxext[\extalg{A}]{\gamma}{a}}{p}.
\end{align*}
and to prove the second judgmental equality we calculate
\begin{align*}
\apply\cftfamext[\extalg{A}]{\gamma}{a}{%
  \apply\cftfamext[\extalg{A}]{\apply\cftctxext[\extalg{A}]{\gamma}{a}}{p}{q}}
& \jdeq
  \apply\cftfamext[\extalg{A}]{\gamma}{%
    \apply\cftfamext[\extalg{A}]{\gamma}{a}{p}}{q}.\qedhere
\end{align*}
\end{proof}

There is a trivial class of examples of extension algebras we can give right
away. More examples will be introduced by universes, later on.

\begin{eg}
Let $A$ be a family in context $\Gamma$. Then the quadruple
\begin{equation*}
(A,\emptyf,\emptyf,\idtm{A},\emptytm)
\end{equation*}
is an extension algebra in context $\Gamma$, as is the quadruple
\begin{equation*}
(\emptyf,A,\emptyf,\emptytm,\idtm{A}).
\end{equation*}
Also, the quadruple
\begin{equation*}
(A,\ctxwk{A}{A},\emptyf,\cprojfstf{A}{\ctxwk{A}{A}},\cprojfstf{\ctxwk{A}}{\ctxwk{A}{{A}{A}}})
\end{equation*}
is an extension algebra in context $\Gamma$.
\end{eg}

As a consequence of the following theorem, every extension algebra gives rise
to infinitely many extension algebras by constructing the extension algebra
of families of $\cftalg{A}$ for each extension algebra $\cftalg{A}$. Notice
that this also explains that the family $\jcomp{}{\cftctxext}{\cftalgf{\cftalg{A}}}$
over $\ctxext{{\Gamma}{\cftalgc{\cftalg{A}}}}{\cftalgf{\cftalg{A}}}$ is the family
of families over families. Likewise, higher families are obtained by pulling
back more times along $\cftctxext[\cftalg{A}]$.

\begin{thm}\label{thm:extalg-fam}
Suppose that $\extalg{A}$ is an extension algebra in context
$\Gamma$. Then 
\begin{equation*}
\cftfamalg{\cftalg{A}}
\defeq
( \cftalgf{\cftalg{A}},
  \jcomp{}{\cftctxext[\cftalg{A}]}{\cftalgf{\cftalg{A}}},
  \jcomp{}{\cftctxext[\cftalg{A}]}{\cftalgt{\cftalg{A}}},
  \cftfamext,
  \jcomp{}{\cftctxext[\cftalg{A}]}{\cftfamext})
\end{equation*}
is an extension algebra in context $\ctxext{\Gamma}{\cftalgc{\cftalg{A}}}$.
\end{thm}

\begin{proof}
In this proof, we shall use the short-hand notations $\cftctxext$ and $\cftfamext$
only to refer to $\cftctxext[\cftalg{A}]$ and $\cftfamext[\cftalg{A}]$, respectively,
and not to $\cftctxext[\cftfamalg{\cftalg{A}}]$ or 
$\cftfamext[\cftfamalg{\cftalg{A}}]$.

We first need to verify that the domain of the morphism 
$\jcomp{}{\cftctxext}{\cftfamext}$ is indeed
$\ctxext{\jcomp{}{\cftctxext}{\cftalgf{\cftalg{A}}}}{\jcomp{}{\cftfamext}{{}{\cftctxext}{\cftalgf{\cftalg{A}}}}}$. 
This follows from the judgmental equality
$\jcomp{}{\cftfamext}{{}{\cftctxext}{\cftalgf{\cftalg{A}}}}\jdeq
\jcomp{}{\cftctxext}{{}{\cftctxext}{\cftalgf{\cftalg{A}}}}$, which we have proved in
\autoref{lem:extalg-twins}. Notice how the diagram in \autoref{eq:extalg-eq2} is
of exactly the right sort, so the quadruple
$(\cftalgf{\cftalg{A}},\jcomp{}{\cftctxext}{\cftalgf{\cftalg{A}}},\cftfamext,\jcomp{}{\cftctxext}{\cftfamext})$
satisfies its version of \autoref{eq:extalg-eq1}. It is left to verify that the diagram
\begin{small}
\begin{equation*}
\begin{tikzcd}[column sep=huge]
\ctxext
  { \jcomp{}{\cftctxext}{\cftalgf{\cftalg{A}}}
    }
  { { \jcomp{}{\cftfamext}{%
        \jcomp{}{\cftctxext}{\cftalgf{\cftalg{A}}}
        }
      }
    { \jcomp{}{\cftfamext}{%
        \jcomp{}{\cftfamext}{%
          \jcomp{}{\cftctxext}{\cftalgf{\cftalg{A}}}
          }
        }
      }
    } 
  \ar{r}
    { \jvcomp{}{\idtm{\jcomp{}{\cftctxext}{\cftalgf{\cftalg{A}}}}}{%
        \jcomp{}{\cftfamext}{%
          \jcomp{}{\cftctxext}{\cftfamext}}}}
  \ar{d}[swap]{
    \jvcomp{}{\jcomp{}{\cftctxext}{\cftfamext}}{%
      \idtm{
        \jcomp{}{\jcomp{}{\cftctxext}{\cftfamext}}{%
          \jcomp{}{\cftfamext}{%
            \jcomp{}{\cftctxext}{\cftalgf{\cftalg{A}}}
            }
          }
        }
      }
    }
& \ctxext
    {\jcomp{}{\cftctxext}{\cftalgf{\cftalg{A}}}}
    {\jcomp{}{\cftfamext}{\jcomp{}{\cftctxext}{\cftalgf{\cftalg{A}}}}} 
  \ar{d}{\jcomp{}{\cftctxext}{\cftfamext}}
  \\
\ctxext
  {\jcomp{}{\cftctxext}{\cftalgf{\cftalg{A}}}}
  {\jcomp{}{\cftfamext}{\jcomp{}{\cftctxext}{\cftalgf{\cftalg{A}}}}} 
  \ar{r}[swap]{\jcomp{}{\cftctxext}{\cftfamext}} 
& \jcomp{}{\cftctxext}{\cftalgf{\cftalg{A}}}
\end{tikzcd}
\end{equation*}
\end{small}%
commutes judgmentally; this diagram is the version of \autoref{eq:extalg-eq2}
for the quadruple
$(\cftalgf{\cftalg{A}},\jcomp{}{\cftctxext}{\cftalgf{\cftalg{A}}},\cftfamext,\jcomp{}{\cftctxext}{\cftfamext})$. Note
that this follows from \autoref{eq:extalg-eq2} provided that we can show that
\begin{align}
\jvcomp{}{\idtm{\jcomp{}{\cftctxext}{\cftalgf{\cftalg{A}}}}}{%
  \jcomp{}{\cftfamext}{%
    \jcomp{}{\cftctxext}{\cftfamext}}}
& \jdeq
  \jcomp{}{\cftctxext}{%
    \jvcomp{}{\idtm{\cftalgf{\cftalg{A}}}}{\jcomp{}{\cftctxext}{\cftfamext}}
    }
  \label{eq:extalg-infty1}
  \\
\jvcomp{}{\jcomp{}{\cftctxext}{\cftfamext}}{%
  \idtm{
    \jcomp{}{\jcomp{}{\cftctxext}{\cftfamext}}{%
      \jcomp{}{\cftfamext}{%
        \jcomp{}{\cftctxext}{\cftalgf{\cftalg{A}}}
        }
      }
    }
  }
& \jdeq
\jcomp{}{\cftctxext}{%
  \jvcomp{}{\cftfamext}{%
    \idtm{
      \jcomp{}{\cftfamext}{%
        \jcomp{}{\cftctxext}{\cftalgf{\cftalg{A}}}
        }
      }
    }
  }
  \label{eq:extalg-infty2}
\end{align}
Note that \autoref{eq:extalg-infty1} follows if we can show that
\begin{equation*}
\jcomp{}{\cftfamext}{\jcomp{}{\cftctxext}{\cftfamext}}
  \jdeq
  \jcomp{}{\cftctxext}{\jcomp{}{\cftctxext}{\cftfamext}}.
\end{equation*}
This is a special case of \autoref{lem:extalg-twins}. The second judgmental
equality, \autoref{eq:extalg-infty2}, is trivial.
\end{proof}

\begin{thm}\label{thm:extalg-wk}
Let $\extalg{Q}$ be an extension algebra in context $\ctxext{\Gamma}{B}$ and let
$\jfam{\Gamma}{A}$ be a family of contexts. Then the quintuple
\begin{equation*}
\ctxwk{A}{\extalg{Q}}
  \defeq
  ( \ctxwk{A}{\cftalgc{\cftalg{Q}}},
    \ctxwk{A}{\cftalgf{\cftalg{Q}}},
    \ctxwk{A}{\cftalgt{\cftalg{Q}}},
    \ctxwk{A}{\cftctxext[\cftalg{Q}]},
    \ctxwk{A}{\cftfamext[\cftalg{Q}]})
\end{equation*}
is an extension algebra in context $\ctxext{{\Gamma}{A}}{\ctxwk{A}{B}}$.
\end{thm}

\begin{proof}
The proof follows from the fact that weakening by $A$ is compatible with all
the involved operations.
\end{proof}

\begin{thm}\label{thm:extalg-subst}
Let $\extalg{Q}$ be an extension algebra in context $\ctxext{{\Gamma}{A}}{P}$
and let $\jterm{\Gamma}{A}{x}$ be a term. Then the quintuple
\begin{equation*}
\subst{x}{\cftalg{Q}}
  \defeq
  ( \subst{x}{\cftalgc{\cftalg{Q}}},
    \subst{x}{\cftalgf{\cftalg{Q}}},
    \subst{x}{\cftalgt{\cftalg{Q}}},
    \subst{x}{\cftctxext[\cftalg{Q}]},
    \subst{x}{\cftfamext[\cftalg{Q}]})
\end{equation*}
is an extension algebra in context $\ctxext{\Gamma}{\subst{x}{P}}$.
\end{thm}

\begin{proof}
The proof follows from the fact that substitution with $x$ is compatible with
all the involved operations.
\end{proof}

\begin{cor}
Let $\extalg{Q}$ be an extension algebra in context $\ctxext{\Gamma}{B}$
and let $\jhom{\Gamma}{A}{B}{f}$. Then the quintuple
\begin{equation*}
\jcomp{A}{f}{\extalg{Q}}
  \defeq
  ( \jcomp{A}{f}{\cftalgc{\cftalg{Q}}},
    \jcomp{A}{f}{\cftalgf{\cftalg{Q}}},
    \jcomp{A}{f}{\cftalgt{\cftalg{Q}}},
    \jcomp{A}{f}{\cftctxext[\cftalg{Q}]},
    \jcomp{A}{f}{\cftfamext[\cftalg{Q}]})
\end{equation*}
is an extension algebra in context $\ctxext{\Gamma}{A}$.
\end{cor}

\begin{cor}
Let $\extalg{A}$ be an extension algebra in context $\Gamma$. Then
$\extfamalg{\extfamalg{\extalg{A}}}$ and $\jcomp{}{\cftctxext[\extalg{A}]}{\extfamalg{\extalg{A}}}$ are
judgmentally equal extension algebras in context $\ctxext{{\Gamma}{\cftalgc{\cftalg{A}}}}{\cftalgf{\cftalg{A}}}$.
\end{cor}

\begin{lem}
Let $\extalg{Q}$ be an extension algebra in context $\ctxext{\Gamma}{B}$ and let
$\jfam{\Gamma}{A}$. Then we have the judgmental equality
\begin{equation*}
\ctxwk{A}{\extfamalg{\extalg{Q}}}\jdeq\extfamalg{\ctxwk{A}{\extalg{Q}}}
\end{equation*}
of extension algebras in context $\ctxext{{\Gamma}{A}}{\ctxwk{A}{B}}$.
\end{lem}

\begin{lem}
Let $\extalg{Q}$ be an extension algebra in context $\ctxext{{\Gamma}{A}}{P}$ 
and let $\jterm{\Gamma}{A}{x}$. Then we have the judgmental equality
\begin{equation*}
\subst{x}{\extfamalg{\extalg{Q}}}\jdeq\extfamalg{\subst{x}{\extalg{Q}}}
\end{equation*}
of extension algebras in context $\ctxext{\Gamma}{\subst{x}{P}}$.
\end{lem}

%%%%%%%%%%%%%%%%%%%%%%%%%%%%%%%%%%%%%%%%%%%%%%%%%%%%%%%%%%%%%%%%%%%%%%%%%%%%%%%%
\subsection{Extension homomorphisms}

\begin{defn}
An \emph{extension homomorphism $\cfthom{f}$ from $\extalg{A}$ to
$\extalg{B}$ in context $\Gamma$} is a triple $\unfold{\cfthom{f}}$ consisting of
\begin{align*}
\jalign\jhom
  {\Gamma}
  {\cftalgc{\cftalg{A}}}
  {\cftalgc{\cftalg{B}}}
  {\cfthomc{\cfthom{f}}}
  \\
\jalign\jfhom
  {\Gamma}
  {\cftalgc{\cftalg{A}}}
  {\cftalgc{\cftalg{B}}}
  {\cfthomc{\cfthom{f}}}
  {\cftalgf{\cftalg{A}}}
  {\cftalgf{\cftalg{B}}}
  {\cfthomf{\cfthom{f}}}
  \\
\jalign\jfhom
  {\Gamma}
  {{\cftalgc{\cftalg{A}}}{\cftalgf{\cftalg{A}}}}
  {{\cftalgc{\cftalg{B}}}{\cftalgf{\cftalg{B}}}}
  {\jvcomp{}{\cfthomc{\cfthom{f}}}{\cfthomf{\cfthom{f}}}}
  {\cftalgt{\cftalg{A}}}
  {\cftalgt{\cftalg{B}}}
  {\cfthomt{\cfthom{f}}}
\end{align*}
for which the diagrams
\begin{equation}\label{eq:exthom1}
\begin{tikzcd}
\ctxext{\cftalgc{\cftalg{A}}}{\cftalgf{\cftalg{A}}}
  \ar{r}{\jvcomp{}{\cfthomc{\cfthom{f}}}{\cfthomf{\cfthom{f}}}}
  \ar{d}[swap]{\cftctxext[\cftalg{A}]}
& \ctxext{\cftalgc{\cftalg{B}}}{\cftalgf{\cftalg{B}}}
  \ar{d}{\cftctxext[\cftalg{B}]}
  \\
\cftalgc{\cftalg{A}}
  \ar{r}[swap]{\cfthomc{\cfthom{f}}}
& \cftalgc{\cftalg{B}}
\end{tikzcd}
\end{equation}
and
\begin{equation}\label{eq:exthom2}
\begin{tikzcd}[column sep=huge]
\ctxext{\cftalgf{\cftalg{A}}}{\jcomp{}{\cftctxext[\cftalg{A}]}{\cftalgf{\cftalg{A}}}}
  \ar{r}{\jvcomp{}{\cfthomf{\cfthom{f}}}{\jcomp{}{\cftctxext[\cftalg{A}]}{\cfthomf{\cfthom{f}}}}}
  \ar{d}[swap]{\cftfamext[\cftalg{A}]}
& \jcomp{}{\cfthomc{\cfthom{f}}}{\ctxext{\cftalgf{\cftalg{B}}}{\jcomp{}{\cftctxext[\cftalg{B}]}{\cftalgf{\cftalg{B}}}}}
  \ar{d}{\jcomp{}{\cfthomc{\cfthom{f}}}{\cftfamext[\cftalg{B}]}}
  \\
\cftalgf{\cftalg{A}}
  \ar{r}[swap]{\cfthomf{\cfthom{f}}}
& \jcomp{}{\cfthomc{\cfthom{f}}}{\cftalgf{\cftalg{B}}}
\end{tikzcd}
\end{equation}
commute judgmentally.
\end{defn}

\begin{rmk}
To see that the upper morphism in the diagram of \autoref{eq:exthom2} has
indeed the indicated codomain provided that the diagram of \autoref{eq:exthom1}
commutes judgmentally, note that we have the judgmental equalities
\begin{align*}
\jcomp{}{\cfthomf{\cfthom{f}}}{\jcomp{}{\cfthomc{\cfthom{f}}}{\jcomp{}{\cftctxext[\cftalg{B}]}{\cftalgf{\cftalg{B}}}}}
& \jdeq 
  \jcomp{}{\jvcomp{}{\cfthomc{\cfthom{f}}}{\cfthomf{\cfthom{f}}}}{\jcomp{}{\cftctxext[\cftalg{B}]}{\cftalgf{\cftalg{B}}}}
  \tag{by \autoref{lem:composition-threesome}}
  \\
& \jdeq
  \jcomp{}{\jcomp{}{\jvcomp{}{\cfthomc{\cfthom{f}}}{\cfthomf{\cfthom{f}}}}{\cftctxext[\cftalg{B}]}}{\cftalgf{\cftalg{B}}}
  \tag{by \autoref{lem:jcomp-jcomp}}
  \\
& \jdeq
  \jcomp{}{\jcomp{}{\cftctxext[\cftalg{A}]}{\cfthomc{\cfthom{f}}}}{\cftalgf{\cftalg{B}}}
  \tag{by \autoref{eq:exthom1}}
  \\
& \jdeq
  \jcomp{}{\cftctxext[\cftalg{A}]}{\jcomp{}{\cfthomc{\cfthom{f}}}{\cftalgf{\cftalg{B}}}}.
  \tag{by \autoref{lem:jcomp-jcomp}}
\end{align*}
and we indeed have the morphism $\jcomp{}{\cftctxext[\cftalg{A}]}{\cfthomf{\cfthom{f}}}$ from 
$\jcomp{}{\cftctxext[\cftalg{A}]}{\cftalgf{\cftalg{A}}}$ to $\jcomp{}{\cftctxext[\cftalg{A}]}{\jcomp{}{\cfthomc{\cfthom{f}}}{\cftalgf{\cftalg{B}}}}$.
\end{rmk}

\begin{thm}
Let $\extalg{A}$ be an extension algebra in context $\Gamma$. Then the triple
\begin{equation*}
\cftidhom{\extalg{A}}\defeq\unfold{\cftidhom{\extalg{A}}}
\end{equation*}
is an extension homomorphism from $\cftalg{A}$ to $\cftalg{A}$ in context
$\Gamma$.
\end{thm}

\begin{thm}
Let $\cfthom{f}$ be an extension homomorphism from $\cftalg{A}$ to $\cftalg{B}$
in context $\Gamma$. Then
\begin{equation*}
\cftfamhom{\cfthom{f}}
  \defeq
  ( \cfthomf{\cfthom{f}},
    \jcomp{}{\cftctxext[\cftalg{A}]}{\cfthomf{\cfthom{f}}},
    \jcomp{}{\cftctxext[\cftalg{A}]}{\cfthomt{\cfthom{f}}})
\end{equation*}
is an extension homomorphism from $\cftfamalg{\cftalg{A}}$ to 
$\jcomp{}{\cfthomc{\cfthom{f}}}{\cftfamalg{\cftalg{B}}}$
in context $\ctxext{\Gamma}{\cftalgc{\cftalg{A}}}$. 
\end{thm}

\begin{proof}
We have to verify that the diagrams in \autoref{eq:exthom1,eq:exthom2}
commute for $\cftfamhom{\cfthom{f}}$. Unfolding the ingredients of \autoref{eq:exthom1}
for $\cftfamhom{\cfthom{f}}$ gives us quite directly \autoref{eq:exthom2}. The
diagram in \autoref{eq:exthom2} for $\cftfamhom{\cfthom{f}}$ is judgmentally equal
to the pullback of everything in the diagram in \autoref{eq:exthom2} for
$\cfthom{f}$ by $\epsilon_0$, and therefore it commutes too.
\end{proof}

\begin{lem}
Let $\cfthom{f}$ be an extension homomorphism from $\cftalg{A}$ to $\cftalg{B}$
in context $\Gamma$. Then we have the judgmental equality
\begin{equation*}
\cftfamhom{\cftfamhom{\cfthom{f}}}
  \jdeq
  \jcomp{}{\cftctxext[\cftalg{A}]}{\cftfamhom{\cfthom{f}}}
\end{equation*}
of extension homomorphisms from $\cftfamalg{\cftfamalg{\cftalg{A}}}$ to
$\jcomp{}{\cfthomf{\cfthom{f}}}
  {\jcomp{}{\cfthomc{\cfthom{f}}}{\cftfamalg{\cftfamalg{\cftalg{B}}}}}$ in
context $\ctxext{{\Gamma}{A}}{\cftalgf{\cftalg{A}}}$.
\end{lem}

\begin{defn}
Let $\cfthom{f}$ and $\cfthom{g}$ be extension homomorphisms from
$\extalg{A}$ to $\extalg{B}$ and from $\extalg{B}$ to $\extalg{C}$, respectively.
We define the composition
\begin{equation*}
\cfthomcomp{\cfthom{f}}{\cfthom{g}}
  \defeq
  \unfold{\cfthomcomp{\cfthom{f}}{\cfthom{g}}}.
\end{equation*}
In other words, extension homomorphisms are composed by taking the horizontal 
rectangles in the diagram
\begin{equation*}
\begin{tikzcd}
\extalgt{\extalg{A}} 
  \ar[fib]{d}
  \ar{r}{\cfthomt{\cfthom{f}}}
& \extalgt{\extalg{B}} 
  \ar[fib]{d}
  \ar{r}{\cfthomt{\cfthom{g}}}
& \extalgt{\extalg{C}}
  \ar[fib]{d}
  \\
\extalgf{\extalg{A}} 
  \ar[fib]{d}
  \ar{r}{\cfthomf{\cfthom{f}}}
& \extalgf{\extalg{B}} 
  \ar[fib]{d}
  \ar{r}{\cfthomf{\cfthom{g}}}
& \extalgf{\extalg{C}}
  \ar[fib]{d}
  \\
\extalgc{\extalg{A}}
  \ar{r}{\cfthomc{\cfthom{f}}}
& \extalgc{\extalg{B}}
  \ar{r}{\cfthomc{\cfthom{g}}}
& \extalgc{\extalg{C}}
\end{tikzcd}
\end{equation*}
\end{defn}

\begin{rmk}
It follows from \autoref{lem:jcomp-jcomp,lem:jfcomp-jfcomp} that composition
of extension homomorphisms is associative.
\end{rmk}

\begin{thm}
Let $\cfthom{f}$ and $\cfthom{g}$ be extension homomorphisms from
$\extalg{A}$ to $\extalg{B}$ and from $\extalg{B}$ to $\extalg{C}$, respectively.
Then $\cfthomcomp{\cfthom{f}}{\cfthom{g}}$ is an extension homomorphism from
$\extalg{A}$ to $\extalg{C}$. 
\end{thm}

\begin{proof}
Both judgmental equalities are applications of the interchange law for composition,
\autoref{lem:composition-interchange}. 
\end{proof}

\begin{lem}
Let $\exthom{g}$ be an extension homomorphism from $\extalg{Q}$ to $\extalg{R}$ in
context $\ctxext{\Gamma}{B}$ and let $\jfam{\Gamma}{A}$. Then the triple
\begin{equation*}
\ctxwk{A}{\cfthom{g}}
  \defeq
  ( \ctxwk{A}{\cfthomc{\cfthom{g}}},
    \ctxwk{A}{\cfthomf{\cfthom{g}}},
    \ctxwk{A}{\cfthomt{\cfthom{g}}})
\end{equation*}
is an extension homomorphism from $\ctxwk{A}{\cftalg{Q}}$ to $\ctxwk{A}{\cftalg{R}}$
in context $\ctxext{{\Gamma}{A}}{\ctxwk{A}{B}}$.
\end{lem}

\begin{lem}
Let $\cfthom{g}$ be an extension homomorphism from $\cftalg{Q}$ to $\cftalg{R}$ in
context $\ctxext{{\Gamma}{A}}{P}$ and let $\jterm{\Gamma}{A}{x}$. Then the
triple
\begin{equation*}
\subst{x}{\cfthom{g}}
  \defeq
  ( \subst{x}{\cfthomc{\cfthom{g}}},
    \subst{x}{\cfthomf{\cfthom{g}}},
    \subst{x}{\cfthomt{\cfthom{g}}})
\end{equation*}
is an extension homomorphism from $\subst{x}{\cftalg{Q}}$ to $\subst{x}{\cftalg{R}}$
in context $\ctxext{\Gamma}{\subst{x}{P}}$.
\end{lem}

The following theorem explains how context extension can be seen as an extension
homomorphism. Note that in combination with \autoref{thm:extalg-fam}, this also
explains how family extension can be seen as an extension homomorphism.

\begin{thm}
Let $\extalg{A}$ be an extension algebra in context $\Gamma$. Then
\begin{align*}
& ( \ctxext{\extalgc{\extalg{A}}}{\extalgf{\extalg{A}}},
    \jcomp{}{\cftctxext}{\extalgf{\extalg{A}}},
    \jvcomp{}{\ctxwk{\extalgf{\extalg{A}}}{\idtm{\extalgc{\extalg{A}}}}}{\cftfamext},
    \jcomp{}{\cftctxext}{\cftfamext}
    )
\intertext{and}
& ( \ctxext{\extalgc{\extalg{A}}}{\extalgf{\extalg{A}}},
    \jcomp{}{\cftctxext}{\extalgf{\extalg{A}}},
    \jvcomp{}{\cftctxext}{\idtm{\jcomp{}{\cftctxext}{\extalgf{\extalg{A}}}}},
    \jcomp{}{\cftctxext}{\cftfamext}
    )
\end{align*}
are extension algebras in context $\Gamma$ and
\begin{equation*}
\boldsymbol{\cftctxext}\defeq ( \cftctxext,
  \idtm{\jcomp{}{\cftctxext}{\extalgf{\extalg{A}}}}
  )
\end{equation*}
is an extension homomorphism from both of them to $\extalg{A}$. 
\end{thm}

As with ordinary morphisms, extension homomorphism can be considered \emph{over} a
morphism.

\begin{defn}
Let $\jhom{\Gamma}{A}{B}{f}$ be a morphism, let $\extalg{P}$ be an extension
algebra in context $\ctxext{\Gamma}{A}$ and let $\extalg{Q}$ be an extension
algebra in context $\ctxext{\Gamma}{B}$. An \emph{extension homomorphism from
$\cftalg{P}$ to $\cftalg{Q}$ over $f$ in context $\Gamma$} is defined to be
an extension homomorphism from $\extalg{P}$ to $\jcomp{A}{f}{\extalg{Q}}$ in
context $\ctxext{\Gamma}{A}$. 
\end{defn}

\begin{eg}
Let $\cfthom{f}$ be an extension homomorphism from $\extalg{A}$ to $\extalg{B}$
in context $\Gamma$. Then $\cftfamhom{\cfthom{f}}$ is an
extension homomorphism from $\extfamalg{\extalg{A}}$ to $\extfamalg{\extalg{B}}$ over
$\cfthomc{\cfthom{f}}$ in context $\Gamma$.
\end{eg}

%%%%%%%%%%%%%%%%%%%%%%%%%%%%%%%%%%%%%%%%%%%%%%%%%%%%%%%%%%%%%%%%%%%%%%%%%%%%%%%%
\subsection{Empty family algebras}
The notion of empty family algebras that we will
study in this subsection will be extension algebras which also have an empty 
context $\cftempc{\cftalg{A}}$ and an empty family $\cftempf{\cftalg{A}}$ 
satisfying (among other conditions) that families over the empty context in
$\cftalg{A}$ are just the contexts of $\cftalg{A}$.

\begin{defn}
An \emph{empty family algebra $\cftalg{A}$ in context $\Gamma$} is a septuple
\begin{equation*}
\unfold{\cftalg{A}}
\end{equation*}
where the quintuple $\unfold{\extalg{A}}$ is an extension
algebra in context $\Gamma$, and where
\begin{align*}
\jalign\jterm{\Gamma}{\cftalgc{\cftalg{A}}}{\cftempc{\cftalg{A}}}
  \\
\jalign\jterm{{\Gamma}{\cftalgc{\cftalg{A}}}}{\cftalgf{\cftalg{A}}}{\cftempf{\cftalg{A}}}
\end{align*}
which satisfy the following judgmental equalities
\begin{enumerate}
\item Families over the empty context in $\cftalg{A}$ are contexts of $\cftalg{A}$:
\begin{align}
\jalign\jfameq
  {\Gamma}
  {\subst{\cftempc{\cftalg{A}}}{\cftalgf{\cftalg{A}}}}
  {\cftalgc{\cftalg{A}}}
  \label{empalg-eq1}
\end{align}
\item The empty family over the empty context of $\cftalg{A}$ is the empty context of $\cftalg{A}$:
\begin{align}
\jalign\jtermeq
  {\Gamma}
  {\cftalgc{\cftalg{A}}}
  {\subst{\cftempc{\cftalg{A}}}{\cftempf{\cftalg{A}}}}
  {\cftempc{\cftalg{A}}}.
  \label{empalg-eq2}
\end{align}
\item Context extension of $\cftalg{A}$ is compatible with the empty context and
family of $\cftalg{A}$:
\begin{align}
\jalign\jtermeq
  {{\Gamma}{\cftalgc{\cftalg{A}}}}
  {\ctxwk{\cftalgc{\cftalg{A}}}{\cftalgc{\cftalg{A}}}}
  {\subst{\cftempc{\cftalg{A}}}{\cftctxext[\cftalg{A}]}}
  {\idtm{\cftalgc{\cftalg{A}}}}
  \label{cftalg-eq1}
  \\
\jalign\jtermeq
  {{\Gamma}{\cftalgc{\cftalg{A}}}}
  {\ctxwk{\cftalgc{\cftalg{A}}}{\cftalgc{\cftalg{A}}}}
  {\subst{\cftempf{\cftalg{A}}}{\cftctxext[\cftalg{A}]}}
  {\idtm{\cftalgc{\cftalg{A}}}}
  \label{cftalg-eq2}
\end{align}
\item Family extension of $\cftalg{A}$ is compatible with the empty family and the empty family
over families of $\cftalg{A}$:
\begin{align}
\jalign\jtermeq
  {{{\Gamma}{\cftalgc{\cftalg{A}}}}{\cftalgf{\cftalg{A}}}}
  {\ctxwk{\cftalgf{\cftalg{A}}}{\cftalgf{\cftalg{A}}}}
  {\subst{\cftempf{\cftalg{A}}}{\cftfamext}}
  {\idtm{\cftalgf{\cftalg{A}}}}
  \label{cftalg-eq3}
  \\
\jalign\jtermeq
  {{{\Gamma}{\cftalgc{\cftalg{A}}}}{\cftalgf{\cftalg{A}}}}
  {\ctxwk{\cftalgf{\cftalg{A}}}{\cftalgf{\cftalg{A}}}}
  {\subst{\jcomp{}{\cftctxext}{\cftempf{\cftalg{A}}}}{\cftfamext}}
  {\idtm{\cftalgf{\cftalg{A}}}}
  \label{cftalg-eq4}
\end{align}
\item Family extension over the empty context of $\cftalg{A}$ is context extension
of $\cftalg{A}$:
\begin{align}
\jalign\jhomeq
  {\Gamma}
  {{\cftalgc{\cftalg{A}}}{\cftalgf{\cftalg{A}}}}
  {\cftalgc{\cftalg{A}}}
  {\subst{\cftempc{\cftalg{A}}}{\cftfamext}}
  {\cftctxext}
  \label{cftalg-eq5}
\end{align}
\end{enumerate}
\end{defn}

\begin{rmk}\label{rmk:cftalg-defn}
We need to verify that the judgmental equalities
\autoref{cftalg-eq1,cftalg-eq2,cftalg-eq3,cftalg-eq4,cftalg-eq5}
are indeed well-typed (i.e.~compare two terms of the same type).
\begin{enumerate}
\item \autoref{lem:ehom-subst} implies that the morphisms 
$\subst{\cftempc{\cftalg{A}}}{\cftctxext}$ and
$\subst{\cftempf{\cftalg{A}}}{\cftctxext}$ both go from $\cftalgc{\cftalg{A}}$ 
to $\cftalgc{\cftalg{A}}$.
\item \label{rmkenum:famfamempf-jdeq-fam}
Now we have the judgmental equalities
\begin{align*}
\subst{\cftempf{\cftalg{A}}}{\jcomp{}{\cftctxext}{\cftalgf{\cftalg{A}}}}
& \jdeq
  \jcomp{}{\subst{\cftempf{\cftalg{A}}}{\cftctxext}}{\cftalgf{\cftalg{A}}}
  \tag{by \autoref{lem:ehom-subst}}
  \\
& \jdeq
  \jcomp{}{\idtm{\cftalgc{\cftalg{A}}}}{\cftalgf{\cftalg{A}}}
  \tag{by \autoref{cftalg-eq1}}
  \\
& \jdeq
  \cftalgf{\cftalg{A}}
  \tag{by \autoref{precomp-idtm-c}}
\end{align*}
Therefore we can apply \autoref{lem:ehom-subst} with the triple
$(\cftalgf{\cftalg{A}},\jcomp{}{\cftctxext}{\cftalgf{\cftalg{A}}},
\cftempf{\cftalg{A}})$ and the morphism $\cftfamext$ to see that both 
$\subst{\cftempf{\cftalg{A}}}{\cftfamext}$ and
$\subst{\jcomp{}{\cftctxext}{\cftempf{\cftalg{A}}}}{\cftfamext}$
are morphisms from $\cftalgf{\cftalg{A}}$ to $\cftalgf{\cftalg{A}}$.
\item The morphism $\subst{\cftempc{\cftalg{A}}}{\cftfamext}$ goes from
$ \subst
    {\cftempc{\cftalg{A}}}
    {\ctxext{\cftalgf{\cftalg{A}}}{\jcomp{}{\cftctxext}{\cftalgf{\cftalg{A}}}}}
  $
to $\subst{\cftempc{\cftalg{A}}}{\cftalgf{\cftalg{A}}}$. For the domain we
have the judgmental equalities
\begin{align*}
  \subst
    {\cftempc{\cftalg{A}}}
    {\ctxext{\cftalgf{\cftalg{A}}}{\jcomp{}{\cftctxext}{\cftalgf{\cftalg{A}}}}}
& \jdeq
  \ctxext
    {\subst{\cftempc{\cftalg{A}}}{\cftalgf{\cftalg{A}}}}
    {\subst{\cftempc{\cftalg{A}}}{\jcomp{}{\cftctxext}{\cftalgf{\cftalg{A}}}}}
  \tag{by \autoref{comp-se-c}}
  \\
& \jdeq
  \ctxext
    {\cftalgc{\cftalg{A}}}
    {\subst{\cftempc{\cftalg{A}}}{\jcomp{}{\cftctxext}{\cftalgf{\cftalg{A}}}}}
  \tag{by \autoref{empalg-eq1}}
  \\
& \jdeq
  \ctxext
    {\cftalgc{\cftalg{A}}}
    {\cftalgf{\cftalg{A}}}
  \tag{by \autoref{rmkenum:famfamempf-jdeq-fam} above}
\end{align*}
The codomain of $\subst{\cftempc{\cftalg{A}}}{\cftfamext}$
is judgmentally equal to $\cftalgc{\cftalg{A}}$ by
\autoref{empalg-eq1}.
\end{enumerate}
\end{rmk}

We can extend the results of 
\autoref{thm:extalg-fam,thm:extalg-wk,thm:extalg-subst} to empty family algebras.

\begin{thm}\label{thm:cftalg-fam}
Let $\cftalg{A}$ be an empty family algebra
in context $\Gamma$. Then the septuple
\begin{equation*}
\cftfamalg{\cftalg{A}}
  \defeq
  ( \cftalgf{\cftalg{A}},
    \jcomp{}{\cftctxext}{\cftalgf{\cftalg{A}}},
    \jcomp{}{\cftctxext}{\cftalgt{\cftalg{A}}},
    \cftfamext,
    \jcomp{}{\cftctxext}{\cftfamext},
    \cftempf{\cftalg{A}},
    \jcomp{}{\cftctxext}{\cftempf{\cftalg{A}}})
\end{equation*}
is an empty family algebra in context $\ctxext{\Gamma}{\cftalgc{\cftalg{A}}}$. 
\end{thm}

\begin{proof}
\begin{enumerate}
\item The judgmental equality
\begin{equation*}
\jfameq{{\Gamma}{\cftalgc{\cftalg{A}}}}{\subst{\cftempf{\cftalg{A}}}{\jcomp{}{\cftctxext}{\cftalgf{\cftalg{A}}}}}{\cftalgf{\cftalg{A}}}
\end{equation*}
was verified in \autoref{rmkenum:famfamempf-jdeq-fam} of \autoref{rmk:cftalg-defn}.
\item Now we verify the judgmental equality
\begin{equation*}
\jtermeq
  {{\Gamma}{\cftalgc{\cftalg{A}}}}
  {\cftalgf{\cftalg{A}}}
  {\subst{\cftempf{\cftalg{A}}}{\jcomp{}{\cftctxext}{\cftempf{\cftalg{A}}}}}
  {\cftempf{\cftalg{A}}}.
\end{equation*}
Note that we can apply \autoref{lem:ehom-subst}, so we get the judgmental
equalities
\begin{align*}
\subst{\cftempf{\cftalg{A}}}{\jcomp{}{\cftctxext}{\cftempf{\cftalg{A}}}}
& \jdeq
  \jcomp{}{\subst{\cftempf{\cftalg{A}}}{\cftctxext}}{\cftempf{\cftalg{A}}}
  \tag{by \autoref{lem:ehom-subst}}
  \\
& \jdeq
  \jcomp{}{\idtm{\cftalgf{\cftalg{A}}}}{\cftempf{\cftalg{A}}}
  \tag{by \autoref{cftalg-eq3}}
  \\
& \jdeq
  \cftempf{\cftalg{A}}.
  \tag{by \autoref{precomp-idtm-t}}
\end{align*}
\item The judgmental equalities
\begin{align*}
\subst{\cftempf{\cftalg{A}}}{\cftfamext}
& \jdeq 
  \idtm{\cftalgf{\cftalg{A}}}
  \\
\subst{\jcomp{}{\cftctxext}{\cftempf{\cftalg{A}}}}{\cftfamext}
& \jdeq 
  \idtm{\cftalgf{\cftalg{A}}}
\end{align*}
are given by assumption.
\item The judgmental equalities
\begin{align*}
\subst{\jcomp{}{\cftctxext}{\cftempf{\cftalg{A}}}}{\jcomp{}{\cftctxext}{\cftfamext}}
& \jdeq
  \idtm{\jcomp{}{\cftctxext}{\cftalgf{\cftalg{A}}}}
  \\
\subst{\jcomp{}{\cftfamext}{\jcomp{}{\cftctxext}{\cftempf{\cftalg{A}}}}}{\jcomp{}{\cftctxext}{\cftfamext}}
& \jdeq
  \idtm{\jcomp{}{\cftctxext}{\cftalgf{\cftalg{A}}}}
\end{align*}
follow from the judgmental equalities
\begin{align*}
\subst{\jcomp{}{\cftctxext}{\cftempf{\cftalg{A}}}}{\jcomp{}{\cftctxext}{\cftfamext}}
& \jdeq
  \jcomp{}{\cftctxext}{\subst{\cftempf{\cftalg{A}}}{\cftfamext}}
  \tag{by \autoref{lem:jcomp-subst}}
  \\
& \jdeq
  \jcomp{}{\cftctxext}{\idtm{\cftalgf{\cftalg{A}}}}
  \tag{by \autoref{cftalg-eq3}}
  \\
& \jdeq
  \idtm{\jcomp{}{\cftctxext}{\cftalgf{\cftalg{A}}}}
  \tag{by \autoref{comp-wi-t}}
\end{align*}
and
\begin{align*}
\subst
  {\jcomp{}{\cftfamext}{\jcomp{}{\cftctxext}{\cftempf{\cftalg{A}}}}}
  {\jcomp{}{\cftctxext}{\cftfamext}}
& \jdeq
  \subst
    {\jcomp{}{\cftctxext}{\jcomp{}{\cftctxext}{\cftempf{\cftalg{A}}}}}
    {\jcomp{}{\cftctxext}{\cftfamext}}
  \tag{by \autoref{lem:extalg-twins}}
  \\
& \jdeq
  \jcomp
    {}
    {\cftctxext}
    {\subst{\jcomp{}{\cftctxext}{\cftempf{\cftalg{A}}}}{\cftfamext}}
  \tag{by \autoref{comp-sw-t}}
  \\
& \jdeq
  \jcomp
    {}
    {\cftctxext}
    {\idtm{\cftalgf{\cftalg{A}}}}
  \tag{by \autoref{cftalg-eq4}}
  \\
& \jdeq
  \idtm{\jcomp{}{\cftctxext}{\cftalgf{\cftalg{A}}}}.
  \tag{by \autoref{comp-wi-t}}
\end{align*}
\item Finally, the judgmental equality
\begin{equation*}
\subst{\cftempf{\cftalg{A}}}{\jcomp{}{\cftctxext}{\cftfamext}}
  \jdeq
  \cftfamext
\end{equation*}
is verified as follows:
\begin{align*}
\subst{\cftempf{\cftalg{A}}}{\jcomp{}{\cftctxext}{\cftfamext}}
& \jdeq
  \jcomp{}{\subst{\cftempf{\cftalg{A}}}{\cftctxext}}{\cftfamext}
  \tag{by \autoref{lem:ehom-subst}}
  \\
& \jdeq
  \jcomp{}{\idtm{\cftalgc{\cftalg{A}}}}{\cftfamext}
  \tag{by \autoref{cftalg-eq1}}
  \\
& \jdeq
  \cftfamext.
  \tag{by \autoref{precomp-idtm-t}}
\end{align*}
\end{enumerate}
\end{proof}

\begin{thm}\label{thm:cftalg-wk}
Let $\cftalg{Q}$ be an empty family algebra in context $\ctxext{\Gamma}{B}$ and let
$\jfam{\Gamma}{A}$ be a family of contexts. Then the septuple
\begin{equation*}
\ctxwk{A}{\cftalg{Q}}
  \defeq
  ( \ctxwk{A}{\cftalgc{\cftalg{Q}}},
    \ctxwk{A}{\cftalgf{\cftalg{Q}}},
    \ctxwk{A}{\cftalgt{\cftalg{Q}}},
    \ctxwk{A}{\cftctxext[\cftalg{Q}]},
    \ctxwk{A}{\cftfamext[\cftalg{Q}]},
    \ctxwk{A}{\cftempc{\cftalg{Q}}},
    \ctxwk{A}{\cftempf{\cftalg{Q}}})
\end{equation*}
is an empty family algebra in context $\ctxext{{\Gamma}{A}}{\ctxwk{A}{B}}$.
\end{thm}

\begin{proof}
The proof follows from the fact that weakening by $A$ is compatible with all
the involved operations.
\end{proof}

\begin{thm}\label{thm:cftalg-subst}
Let $\extalg{Q}$ be an empty family algebra in context $\ctxext{{\Gamma}{A}}{P}$
and let $\jterm{\Gamma}{A}{x}$ be a term. Then the septuple
\begin{equation*}
\subst{x}{\cftalg{Q}}
  \defeq
  ( \subst{x}{\cftalgc{\cftalg{Q}}},
    \subst{x}{\cftalgf{\cftalg{Q}}},
    \subst{x}{\cftalgt{\cftalg{Q}}},
    \subst{x}{\cftctxext[\cftalg{Q}]},
    \subst{x}{\cftfamext[\cftalg{Q}]},
    \subst{x}{\cftempc{\cftalg{Q}}},
    \subst{x}{\cftempf{\cftalg{Q}}})
\end{equation*}
is an empty family algebra in context $\ctxext{\Gamma}{\subst{x}{P}}$.
\end{thm}

\begin{proof}
The proof follows from the fact that substitution with $x$ is compatible with
all the involved operations.
\end{proof}

\begin{cor}
Let $\cftalg{Q}$ be an empty family algebra in context $\ctxext{\Gamma}{B}$
and let $\jhom{\Gamma}{A}{B}{f}$. Then the septuple
\begin{equation*}
\jcomp{A}{f}{\cftalg{Q}}
  \defeq
  ( \jcomp{A}{f}{\cftalgc{\cftalg{Q}}},
    \jcomp{A}{f}{\cftalgf{\cftalg{Q}}},
    \jcomp{A}{f}{\cftalgt{\cftalg{Q}}},
    \jcomp{A}{f}{\cftctxext[\cftalg{Q}]},
    \jcomp{A}{f}{\cftfamext[\cftalg{Q}]},
    \jcomp{A}{f}{\cftempc{\cftalg{Q}}},
    \jcomp{A}{f}{\cftempf{\cftalg{Q}}})
\end{equation*}
is an empty family algebra in context $\ctxext{\Gamma}{A}$.
\end{cor}

\begin{cor}
Let $\cftalg{A}$ be an empty family algebra in context $\Gamma$. Then
$\cftfamalg{\cftfamalg{\cftalg{A}}}$ and 
$\jcomp{}{\cftctxext[\cftalg{A}]}{\cftfamalg{\cftalg{A}}}$ are
judgmentally equal empty family algebras in context 
$\ctxext{{\Gamma}{\cftalgc{\cftalg{A}}}}{\cftalgf{\cftalg{A}}}$.
\end{cor}

\begin{lem}
Let $\cftalg{Q}$ be an empty family algebra in context $\ctxext{\Gamma}{B}$ and let
$\jfam{\Gamma}{A}$. Then we have the judgmental equality
\begin{equation*}
\ctxwk{A}{\cftfamalg{\cftalg{Q}}}\jdeq\cftalg{F}_{\ctxwk{A}{\cftalg{Q}}}
\end{equation*}
of empty family algebras in context $\ctxext{{\Gamma}{A}}{\ctxwk{A}{B}}$.
\end{lem}

\begin{lem}
Let $\cftalg{Q}$ be an empty family algebra in context $\ctxext{{\Gamma}{A}}{P}$ 
and let $\jterm{\Gamma}{A}{x}$. Then we have the judgmental equality
\begin{equation*}
\subst{x}{\cftfamalg{\cftalg{Q}}}\jdeq\cftalg{F}_{\subst{x}{\cftalg{Q}}}
\end{equation*}
of empty family algebras in context $\ctxext{\Gamma}{\subst{x}{P}}$.
\end{lem}

%%%%%%%%%%%%%%%%%%%%%%%%%%%%%%%%%%%%%%%%%%%%%%%%%%%%%%%%%%%%%%%%%%%%%%%%%%%%%%%%
\subsection{Empty family homomorphisms}
\begin{defn}
An empty family homomorphism $\cfthom{f}$ from $\cftalg{A}$ to $\cftalg{B}$ in context
$\Gamma$ is an extension homomorphism from $\unfold{\extalg{A}}$ to
$\unfold{\extalg{B}}$ in context $\Gamma$ for which we have the judgmental
equalities
\begin{align*}
\subst{\cftempc{\cftalg{A}}}{\cfthomc{\cfthom{f}}}
& \jdeq
  \cftempc{\cftalg{B}}
  \\
\subst{\cftempc{\cftalg{A}}}{\cfthomf{\cfthom{f}}}
& \jdeq
  \cfthomc{\cfthom{f}}
  \\
\subst{\cftempf{\cftalg{A}}}{\cfthomf{\cfthom{f}}}
& \jdeq
  \jcomp{}{\cfthomc{\cfthom{f}}}{\cftempf{\cftalg{B}}}
\end{align*}

Judgmental equality of empty family homomorphisms from $\cftalg{A}$ to $\cftalg{B}$ in
context $\Gamma$ is taken to be judgmental equality of extension homomorphisms
from $\extalg{A}$ to $\extalg{B}$ in context $\Gamma$.
\end{defn}

\begin{lem}
For any empty family algebra $\cftalg{A}$, the extension homomorphism $\cftidhom{\cftalg{A}}$
is an empty family homomorphism.
\end{lem}

\begin{lem}
Suppose that $\cfthom{f}$ is an empty family homomorphism from $\cftalg{A}$ to $\cftalg{B}$
in context $\Gamma$ and that $\cfthom{g}$ is an empty family homomorphism from $\cftalg{B}$
to $\cftalg{C}$ in context $\Gamma$. Then $\jcomp{}{\cfthom{f}}{\cfthom{g}}$
is an empty family homomorphism from $\cftalg{A}$ to $\cftalg{C}$ in context $\Gamma$.
\end{lem}

\begin{lem}
Let $\cfthom{f}$ be an empty family homomorphism from $\cftalg{A}$ to $\cftalg{B}$ in
context $\Gamma$. Then $\cftfamhom{\exthom{f}}$ is an empty family homomorphism from
$\cftfamalg{\cftalg{A}}$ to $\jcomp{}{\cfthomc{\cfthom{f}}}{\cftfamalg{\cftalg{B}}}$
in context $\ctxext{\Gamma}{\cftalgc{\cftalg{A}}}$. 
\end{lem}

\begin{lem}
Let $\cfthom{g}$ be an empty family homomorphism from $\cftalg{Q}$ to $\cftalg{R}$ in
context $\ctxext{\Gamma}{B}$ and let $\jfam{\Gamma}{A}$. Then 
$\ctxwk{A}{\cfthom{g}}$
is an empty family homomorphism from $\ctxwk{A}{\cftalg{Q}}$ to $\ctxwk{A}{\cftalg{R}}$
in context $\ctxext{{\Gamma}{A}}{\ctxwk{A}{B}}$.
\end{lem}

\begin{lem}
Let $\cfthom{g}$ be an empty family homomorphism from $\cftalg{Q}$ to $\cftalg{R}$ in
context $\ctxext{{\Gamma}{A}}{P}$ and let $\jterm{\Gamma}{A}{x}$. Then 
$\subst{x}{\cfthom{g}}$
is an empty family homomorphism from $\subst{x}{\cftalg{Q}}$ to $\subst{x}{\cftalg{R}}$
in context $\ctxext{\Gamma}{\subst{x}{P}}$.
\end{lem}

%%%%%%%%%%%%%%%%%%%%%%%%%%%%%%%%%%%%%%%%%%%%%%%%%%%%%%%%%%%%%%%%%%%%%%%%%%%%%%%%
\subsection{Weakening algebras}
Weakening algebras will be extension algebras with certain added structure.
We shall first formulate the notion of a pre-weakening algebra, which is an
extension algebra together with an extension homomorphism describing weakening,
but without the additional judgmental equalities which will have to hold for
weakening algebras. The advantage of this approach is that this allows us to
consider pre-weakening homomorphisms and require that the weakening of the
pre-weakening algebra is a pre-weakening homomorphism. This is the condition
of \autoref{comp-ww} that weakening is compatible with itself. We could also
formulate this condition directly, but it is complicated to state and it would
not be very insightful if we did that.

\begin{defn}
A pre-weakening algebra $\cftalg{A}$ in context $\Gamma$ is an extension algebra
$\cftalg{A}$ in context $\Gamma$ together with an extension homomorphism
\begin{equation}\label{cftwk}
\jhom
  {{{\Gamma}{\cftalgc{\cftalg{A}}}}{\cftalgf{\cftalg{A}}}}
  {\ctxwk{\cftalgf{\cftalg{A}}}{\cftfamalg{\cftalg{A}}}}
  {\cftfamalg{\cftfamalg{\cftalg{A}}}}
  {\cftwk{\cftalg{A}}}
\end{equation}
called the \emph{weakening of $\cftalg{A}$}.
\end{defn}

Weakening algebras will be pre-weakening algebras for which the weakening
satisfies judgmental equalities which express abstracted analogues of
the rules in \autoref{comp-ww,comp-ew,comp-0w}. The
definition of weakening algebras can be found in \autoref{sec:cftwkalg-defn}.

We begin with some trivial remarks to familiarize ourselves with the situation
of pre-weakening algebras.
The extension homomorphism $\cftwk{\cftalg{A}}$ of \autoref{cftwk} is a triple
$(\cftwkc{\cftalg{A}},\cftwkf{\cftalg{A}},\cftwkt{\cftalg{A}})$ consisting of
the morphisms displayed in the following diagram:
\begin{equation*}
\begin{tikzcd}
\ctxwk{\cftalgf{\cftalg{A}}}{\jcomp{}{\cftctxext}{\cftalgt{\cftalg{A}}}}
  \ar[fib]{d}
  \ar{r}{\cftwkt{\cftalg{A}}}
& \jcomp{}{\cftctxext}{\jcomp{}{\cftctxext}{\cftalgt{\cftalg{A}}}}
  \ar[fib]{d}
  \\
\ctxwk{\cftalgf{\cftalg{A}}}{\jcomp{}{\cftctxext}{\cftalgf{\cftalg{A}}}}
  \ar[fib]{d}
  \ar{r}{\cftwkf{\cftalg{A}}}
& \jcomp{}{\cftctxext}{\jcomp{}{\cftctxext}{\cftalgf{\cftalg{A}}}}
  \ar[fib]{d}
  \\
\ctxwk{\cftalgf{\cftalg{A}}}{\cftalgf{\cftalg{A}}}
  \ar{r}[swap]{\cftwkc{\cftalg{A}}}
& \jcomp{}{\cftctxext}{\cftalgf{\cftalg{A}}}
\end{tikzcd}
\end{equation*}
In particular, we see that $\cftwk{\cftalg{A}}$ is an extension homomorphism from
$\ctxwk{\cftalgf{\cftalg{A}}}{\cftfamalg{\cftalg{A}}}$ to $\cftfamalg{\cftalg{A}}$ over
$\cftctxext$ in context $\Gamma$. 

\begin{rmk}
Let $\cftalg{A}$ be an extension algebra and let $\cftwk{\cftalg{A}}$ be as in
\autoref{cftwk}. Then $\apply\cftwk{\cftalg{A}}{\gamma}{a}$ is an extension homomorphism
from $\subst{\gamma}{\cftfamalg{\cftalg{A}}}$ to 
$\subst{a}{{\gamma}{\cftfamalg{\cftfamalg{\cftalg{A}}}}}$. Furthermore, we have
\begin{equation*}
\jterm
  {\Gamma}
  {\subst{\apply\cftctxext{\gamma}{a}}{\cftalgf{\cftalg{A}}}}
  {\apply\cftwkc{\cftalg{A}}{\gamma}{a}{b}}
\end{equation*}
for $\gamma:A$ and $a,b:\subst{\gamma}{\cftalgf{\cftalg{A}}}$; we have
\begin{equation*}
\jterm
  { \Gamma}
  { \subst
      { \apply\cftctxext
          {\apply\cftctxext{\gamma}{a}}
          {\apply\cftwkc{\cftalg{A}}{\gamma}{a}{b}}
        }
      { \cftalgf{\cftalg{A}}}}
  { \apply\cftwkf{\cftalg{A}}{\gamma}{a}{b}{q}}
\end{equation*}
for $\gamma:A$, $a,b:\subst{\gamma}{\cftalgf{\cftalg{A}}}$ and
$q:\subst{\apply\cftctxext{\gamma}{b}}{\cftalgf{\cftalg{A}}}$; and finally, we have
\begin{equation*}
\jterm
  { \Gamma}
  { \subst
      { \apply\cftwkf{\cftalg{A}}{\gamma}{a}{b}{q}}
      { { \apply\cftctxext
            {\apply\cftctxext{\gamma}{a}}
            {\apply\cftwkc{\cftalg{A}}{\gamma}{a}{b}}
          }
        {\cftalgt{\cftalg{A}}}
        }
    }
  {\subst{g}{{q}{{b}{{a}{{\gamma}{\cftwkt{\cftalg{A}}}}}}}}
\end{equation*}
for $\gamma:A$, $a,b:\subst{\gamma}{\cftalgf{\cftalg{A}}}$,
$q:\subst{\apply\cftctxext{\gamma}{b}}{\cftalgf{\cftalg{A}}}$ and
$g:\subst{q}{{\apply\cftctxext{\gamma}{b}}{\cftalgt{\cftalg{A}}}}$.
These observations show that the weakening morphism $\cftwk{\cftalg{A}}$ is
indeed an abstraction of the introduction rules for the weakening operation.
\end{rmk}

\begin{defn}
Suppose $\cftalg{A}$ is a pre-weakening algebra in context $\Gamma$. Then we
define the pre-weakening algebra $\cftfamalg{\cftalg{A}}$ in context 
$\ctxext{\Gamma}{\cftalgc{\cftalg{A}}}$ by taking
\begin{equation*}
\cftwk{\cftfamalg{\cftalg{A}}}
  \defeq \jcomp{}{\cftctxext}{\cftwk{\cftalg{A}}}
\end{equation*}
\end{defn}

\begin{defn}
Suppose $\cftalg{Q}$ is a pre-weakening algebra in context $\ctxext{\Gamma}{B}$ and
let $\jfam{\Gamma}{A}$. Then we define the pre-weakening algebra 
$\ctxwk{A}{\cftalg{Q}}$ in context $\ctxext{{\Gamma}{A}}{\ctxwk{A}{B}}$ by 
taking
\begin{equation*}
\cftwk{\ctxwk{A}{\cftalg{Q}}}\defeq \ctxwk{A}\cftwk{\cftalg{Q}}.
\end{equation*}
\end{defn}

\begin{defn}
Suppose $\cftalg{Q}$ is a pre-weakening algebra in context $\ctxext{{\Gamma}{A}}{P}$
and let $\jterm{\Gamma}{A}{x}$. Then we define the pre-weakening algebra 
$\subst{x}{\cftalg{Q}}$ in context $\ctxext{\Gamma}{\subst{x}{P}}$ by taking
\begin{equation*}
\cftwk{\subst{x}{\cftalg{Q}}} \defeq \subst{x}{\cftwk{\cftalg{Q}}}.
\end{equation*}
\end{defn}

\begin{lem}
Suppose $\cftalg{A}$ is a pre-weakening algebra in context $\Gamma$. Then
$\jcomp{}{\cftctxext}{\cftfamalg{\cftalg{A}}}$ and $\cftfamalg{\cftfamalg{\cftalg{A}}}$
are judgmentally equal pre-weakening algebras.
\end{lem}

\begin{proof}
This follows immediately from the facts that
$\jcomp{}{\cftctxext}{\cftfamalg{\cftalg{A}}}$ and $\cftfamalg{\cftfamalg{\cftalg{A}}}$
are judgmentally equal extension algebras in context $\Gamma$ and that
\begin{equation*}
\jcomp{}{\cftctxext}{\cftwk{\cftfamalg{\cftalg{A}}}}
  \jdeq
\jcomp{}{\cftctxext}{\jcomp{}{\cftctxext}{\cftwk{\cftalg{A}}}}
  \jdeq
\jcomp{}{\cftfamext}{\jcomp{}{\cftctxext}{\cftwk{\cftalg{A}}}}
  \jdeq
\cftwk{\cftfamalg{\cftfamalg{\cftalg{A}}}}.
  \qedhere
\end{equation*}
\end{proof}

\begin{lem}
Suppose $\cftalg{Q}$ is a pre-weakening algebra in context $\ctxext{\Gamma}{B}$
and that $\jfam{\Gamma}{A}$. Then
\end{lem}

\subsubsection{Pre-weakening homomorphisms}
It is useful to start with the definition of pre-weakening homomorphisms. Those
are extension homomorphisms which also preserve weakening. Before we give the
definition, recall from \autoref{lem:jvcomp-cprojfstf} that we have a
commutative square
\begin{equation*}
\begin{tikzcd}
\ctxext{\cftalgc{\cftalg{A}}}{\cftalgf{\cftalg{A}}}
  \ar{r}{\jvcomp{}{\cfthomc{\cfthom{f}}}{\cfthomf{\cfthom{f}}}}
  \ar{d}[swap]{\cprojfstf{\cftalgc{\cftalg{A}}}{\cftalgf{\cftalg{A}}}}
  &
\ctxext{\cftalgc{\cftalg{A}}}{\cftalgf{\cftalg{B}}}
  \ar{d}{\cprojfstf{\cftalgc{\cftalg{B}}}{\cftalgf{\cftalg{B}}}}
  \\
\cftalgc{\cftalg{A}}
  \ar{r}[swap]{\cfthomc{\cfthom{f}}}
  &
\cftalgc{\cftalg{B}}
\end{tikzcd}
\end{equation*}
of morphisms in context $\Gamma$ for every extension homomorphism $\cfthom{f}$ from
$\cftalg{A}$ to $\cftalg{B}$ in context $\Gamma$. Because weakening is the
same thing as pulling back along projection, we therefore see that the
judgmental equality
\begin{equation*}
\jfameq
  {{{\Gamma}{\cftalgc{\cftalg{A}}}}{\cftalgf{\cftalg{A}}}}
  { \ctxwk
      {\cftalgf{\cftalg{A}}}
      {\jcomp{}{\cfthomc{\cfthom{f}}}{\cftfamalg{\cftalg{B}}}}
    }
  { \jcomp{}
      { \cfthomf{\cfthom{f}}}
      { \jcomp{}
        {\cfthomc{\cfthom{f}}}
        {\ctxwk{\cftalgf{\cftalg{B}}}{\cftfamalg{\cftalg{B}}}}
        }
    }
\end{equation*}
is valid for every extension homomorphism $\cfthom{f}$ from $\cftalg{A}$ to $\cftalg{B}$
in context $\Gamma$. This judgmental equality helps us seeing that the condition
on pre-weakening homomorphisms is well-typed.

\begin{defn}
Let $\cftalg{A}$ and $\cftalg{B}$ be pre-weakening algebras in context $\Gamma$.
A \emph{pre-weakening homomorphism from $\cftalg{A}$ to $\cftalg{B}$ in context
$\Gamma$} is an extension homomorphism $\cfthom{f}$ from $\cftalg{A}$ to $\cftalg{B}$
in context $\Gamma$ for which the diagram
\begin{equation*}
\begin{tikzcd}[column sep=huge]
\ctxwk{\cftalgf{\cftalg{A}}}{\cftfamalg{\cftalg{A}}}
  \ar{r}{\ctxwk{\cftalgf{\cftalg{A}}}{\cftfamhom{\cfthom{f}}}}
  \ar{d}[swap]{\cftwk{\cftalg{A}}}
  &
\jcomp{}
  { \cfthomf{\cfthom{f}}}
  { \jcomp{}
    {\cfthomc{\cfthom{f}}}
    {\ctxwk{\cftalgf{\cftalg{B}}}{\cftfamalg{\cftalg{B}}}}
    }
  \ar{d}
    { \jcomp{}
        {\cfthomf{\cfthom{f}}}
        {\jcomp{}{\cfthomc{\cfthom{f}}}{\cftwk{\cftalg{B}}}}
      }
  \\
\cftfamalg{\cftfamalg{\cftalg{A}}}
  \ar{r}[swap]{\cftfamhom{\cftfamhom{\cfthom{f}}}}
  &
\jcomp{}
  {\cfthomf{\cfthom{f}}}
  {\jcomp{}{\cfthomc{\cfthom{f}}}{\cftfamalg{\cftfamalg{\cftalg{B}}}}}
\end{tikzcd}
\end{equation*}
of extension algebras in context 
$\ctxext{{\Gamma}{\cftalgc{\cftalg{A}}}}{\cftalgf{\cftalg{A}}}$ commutes
judgmentally.
\end{defn}

\begin{rmk}\label{rmk:cftwk-prewk}
In this remark we explain what it means that weakening is a pre-weakening
homomorphism. Let $\cftalg{A}$ be a pre-weakening algebra. Then its weakening
$\cftwk{\cftalg{A}}$ is an extension homomorphism from
$\ctxwk{\cftalgf{\cftalg{A}}}{\cftfamalg{\cftalg{A}}}$ to
$\cftfamalg{\cftfamalg{\cftalg{A}}}$. Therefore, $\cftwk{\cftalg{A}}$ is a
pre-weakening homomorphism if the diagram
\begin{equation*}\label{diag:cftwk-cftwk}
\begin{tikzcd}[column sep=8em]
\ctxwk
  {\cftalgf{\ctxwk{\cftalgf{\cftalg{A}}}{\cftfamalg{\cftalg{A}}}}}
  {\cftfamalg{\ctxwk{\cftalgf{\cftalg{A}}}{\cftfamalg{\cftalg{A}}}}}
  \ar{r}
    { \ctxwk
        {\cftalgf{\ctxwk{\cftalgf{\cftalg{A}}}{\cftfamalg{\cftalg{A}}}}}
        {\cftfamhom{\cftwk{\cftalg{A}}}}
      }
  \ar{d}[swap]{\cftwk{\ctxwk{\cftalgf{\cftalg{A}}}{\cftfamalg{\cftalg{A}}}}}
  &
\jcomp{}
  { \cftwkf{\cftalg{A}}}
  { \jcomp{}
    {\cftwkc{\cftalg{A}}}
    {\ctxwk{\cftalgf{\cftfamalg{\cftfamalg{\cftalg{A}}}}}{\cftfamalg{\cftfamalg{\cftfamalg{\cftalg{A}}}}}}
    }
  \ar{d}
    { \jcomp{}
        { \cftwkf{\cftalg{A}}}
        { \jcomp{}
            { \cftwkc{\cftalg{A}}}
            { \cftwk{\cftfamalg{\cftfamalg{\cftalg{A}}}}}}
      }
  \\
\cftfamalg{\cftfamalg{\ctxwk{\cftalgf{\cftalg{A}}}{\cftfamalg{\cftalg{A}}}}}
  \ar{r}[swap]
    { \cftfamhom
        { \cftfamhom
            { \cftwk{\cftalg{A}}}
          }
      }
  &
\jcomp{}
  { \cftwkf{\cftalg{A}}}
  { \jcomp{}
      {\cftwkc{\cftalg{A}}}
      {\cftfamalg{\cftfamalg{\cftfamalg{\cftfamalg{\cftalg{A}}}}}}
    }
\end{tikzcd}
\end{equation*}
of pre-weakening algebras in context
$ \ctxext
    { { { { \Gamma}
          { \cftalgc{\cftalg{A}}}
          }
        { \cftalgf{\cftalg{A}}}
        }
      { \ctxwk{\cftalgf{\cftalg{A}}}{\cftalgf{\cftalg{A}}}}
      }
    { \cftalgf{\ctxwk{\cftalgf{\cftalg{A}}}{\cftfamalg{\cftalg{A}}}}
      }$
commutes judgmentally.
\end{rmk}

\begin{comment}
\subsubsection{Currying for weakening}
Let $\cftalg{A}$ be an extension algebra in context $\Gamma$ and let
\begin{equation*}
\jhom
  {{{\Gamma}{\cftalgc{\cftalg{A}}}}{\cftalgf{\cftalg{A}}}}
  {\ctxwk{\cftalgf{\cftalg{A}}}{\cftfamalg{\cftalg{A}}}}
  {\cftfamalg{\cftfamalg{\cftalg{A}}}}
  {\cftwk{\cftalg{A}}}
\end{equation*}
be an extension homomorphism. The abstraction of the rules in \autoref{comp-ew} is the
condition that the diagram
\begin{equation}\label{diag:cftwk-cftextcurry}
\begin{tikzcd}[column sep=large]
\ctxwk
  {\ctxext{\cftalgf{\cftalg{A}}}{\cftalgf{\cftfamalg{\cftalg{A}}}}}
  {\cftfamalg{\cftalg{A}}}
  \ar{r}{\ctxwk{\cftalgf{\cftfamalg{\cftalg{A}}}}{\cftwk{\cftalg{A}}}}
  \ar{dr}[swap]{\jcomp{}{\cftfamext}{\cftwk{\cftalg{A}}}}
& \ctxwk{\cftalgf{\cftfamalg{\cftalg{A}}}}{\cftfamalg{\cftfamalg{\cftalg{A}}}}
  \ar{d}{\jcomp{}{\cftctxext}{\cftwk{\cftalg{A}}}}
  \\
& \cftfamalg{\cftfamalg{\cftfamalg{\cftalg{A}}}}
\end{tikzcd}
\end{equation}
of extension algebras in context 
$\ctxext
    {{{\Gamma}{\cftalgc{\cftalg{A}}}}{\cftalgf{\cftalg{A}}}}
    {\cftalgf{\cftfamalg{\cftalg{A}}}}
$ %
commutes.

\begin{lem}
Let $\cftalg{A}$ be an extension algebra in context gamma with an extension homomorphism 
$ \jhom
  {{{\Gamma}{\cftalgc{\cftalg{A}}}}{\cftalgf{\cftalg{A}}}}
  {\ctxwk{\cftalgf{\cftalg{A}}}{\cftfamalg{\cftalg{A}}}}
  {\jcomp{}{\cftctxext}{\cftfamalg{\cftalg{A}}}}
  {\cftwk{\cftalg{A}}}
$
satisfying the judgmental equality of \autoref{diag:cftwk-cftextcurry}. Then we
have the judgmental equality
\begin{equation*}
\apply\cftwk{\cftalg{A}}{\gamma}{\apply\cftfamext{\gamma}{a}{p}}
\jdeq
\jcomp{}
    {\apply\cftwk{\cftalg{A}}{\gamma}{a}}
    {\apply\cftwk{\cftalg{A}}{\apply\cftctxext{\gamma}{a}}{p}}
\end{equation*}
of extension homomorphisms from $\subst{\gamma}{\cftfamalg{\cftalg{A}}}$ to 
$\subst{p}{{a}{{\gamma}{\cftalg{F_{F_{F_A}}}}}}$ in context $\Gamma$,
for every $\gamma:A$, $a:\subst{\gamma}{\cftalgf{\cftalg{A}}}$ and
$p:\subst{a}{{\gamma}{\cftalgf{\cftfamalg{\cftalg{A}}}}}$. 
\end{lem}

\subsubsection{Weakening by the empty family}
\begin{equation}\label{eq:cftwk-cftempcurry}
\jhomeq
  { {\Gamma}{\cftalgc{\cftalg{A}}}
    }
  { \cftfamalg{\cftalg{A}}
    }
  { \cftfamalg{\cftalg{A}}
    }
  { \subst{\cftempf{\cftalg{A}}}{\cftwk{\cftalg{A}}}
    }
  { \cftidhom{\cftfamalg{\cftalg{A}}}
    }
\end{equation}

\subsubsection{Weakening preserves itself}
Let $\cftalg{A}$ be an extension algebra in context $\Gamma$ and let
\begin{equation*}
\jhom
  {{{\Gamma}{\cftalgc{\cftalg{A}}}}{\cftalgf{\cftalg{A}}}}
  {\ctxwk{\cftalgf{\cftalg{A}}}{\cftfamalg{\cftalg{A}}}}
  {\cftfamalg{\cftfamalg{\cftalg{A}}}}
  {\cftwk{\cftalg{A}}}
\end{equation*}
be an extension homomorphism. The abstraction of the rules in \autoref{comp-ww} is the
condition that the diagram
\begin{equation}\label{diag:cftwk-cftwk}
\begin{tikzcd}[column sep=10em]
\ctxwk{\cftalgf{\cftalg{A}}}{{\cftalgf{\cftfamalg{\cftalg{A}}}}{\cftfamalg{\cftfamalg{\cftalg{A}}}}}
  \ar{d}
    [swap]{ \ctxwk
        {{\cftalgf{\cftalg{A}}}{\cftalgf{\cftfamalg{\cftalg{A}}}}}
        {\cftfamhom{\cftwk{\cftalg{A}}}}
      }
  \ar{r}{\ctxwk{\cftalgf{\cftalg{A}}}{\jcomp{}{\cftctxext}{\cftwk{\cftalg{A}}}}}
& \ctxwk{\cftalgf{\cftalg{A}}}{\cftfamalg{\cftfamalg{\cftfamalg{\cftalg{A}}}}}
  \ar{d}{\jcomp{}{\ctxwk{\cftalgf{\cftalg{A}}}{\cftfamext}}{\cftfamhom{\cftwk{\cftalg{A}}}}}
  \\
\ctxwk{{\cftalgf{\cftalg{A}}}{\cftalgf{\cftfamalg{\cftalg{A}}}}}{\jcomp{}{\cftwkc{\cftalg{A}}}{\cftfamalg{\cftfamalg{\cftfamalg{\cftalg{A}}}}}}
  \ar{r}[swap]{%
    \jcomp{}
      { \cftwkf{\cftalg{A}}}
      { \jcomp{}
          {\cftwkc{\cftalg{A}}}
          { \jcomp{}
              {\cftctxext}
              { \jcomp{}
                  {\cftctxext}
                  {\cftwk{\cftalg{A}}}
                }
            }
        }
    }
& \jcomp{}
    {\ctxwk{\cftalgf{\cftalg{A}}}{\cftfamext}}
    {\jcomp{}{\cftwkc{\cftalg{A}}}{\cftfamalg{\cftfamalg{\cftfamalg{\cftalg{A}}}}}}
\end{tikzcd}
\end{equation}
of extension algebras in context 
$ \ctxext
    { { { {\Gamma}
          {\cftalgc{\cftalg{A}}}
          }
        { \cftalgf{\cftalg{A}}}
        }
      { \ctxwk
          {\cftalgf{\cftalg{A}}}
          {\cftalgf{\cftalg{A}}}
        }
      }
    { \ctxwk
        {\cftalgf{\cftalg{A}}}
        {\cftalgf{\cftfamalg{\cftalg{A}}}}
      }$ %
commutes. In this diagram, it is not immediately obvious that\ldots
\begin{enumerate}
\item \ldots the domain of the extension homomorphism
$\jcomp{}{\ctxwk{\cftalgf{\cftalg{A}}}{\cftfamext}}{\cftfamhom{\cftwk{\cftalg{A}}}}$
is the indicated domain. Note that we have the judgmental equalities
\begin{align*}
\jcomp{}
  {\ctxwk{\cftalgf{\cftalg{A}}}{\cftfamext}}
  {\ctxwk{\cftalgf{\cftalg{A}}}{\cftalgf{\cftfamalg{\cftalg{A}}}}}
& \jdeq
\ctxwk
  {\cftalgf{\cftalg{A}}}
  {\jcomp{}{\cftfamext}{\cftfamalg{\cftfamalg{\cftalg{A}}}}}
  \\
& \jdeq
\ctxwk
  {\cftalgf{\cftalg{A}}}
  {\jcomp{}{\jcomp{}{\cftctxext}{\cftfamext}}{\cftfamalg{\cftfamalg{\cftalg{A}}}}}
  \\
& \jdeq
\ctxwk
  {\cftalgf{\cftalg{A}}}
  {\cftfamalg{\cftfamalg{\cftfamalg{\cftalg{A}}}}}.
\end{align*}
\item \ldots the domain and codomain of the extension homomorphism
\begin{equation*}
\jcomp{}
      { \cftwkf{\cftalg{A}}}
      { \jcomp{}
          {\cftwkc{\cftalg{A}}}
          { \jcomp{}
              {\cftctxext}
              { \jcomp{}
                  {\cftctxext}
                  {\cftwk{\cftalg{A}}}
                }
            }
        }
\end{equation*}
from
$ \jcomp{}
      { \cftwkf{\cftalg{A}}}
      { \jcomp{}
          {\cftwkc{\cftalg{A}}}
          { \jcomp{}
              {\cftctxext}
              { \jcomp{}
                  {\cftctxext}
                  {\ctxwk{\cftalgf{\cftalg{A}}}{\cftfamalg{\cftalg{A}}}}
                }
            }
        }
$ to
$ \jcomp{}
      { \cftwkf{\cftalg{A}}}
      { \jcomp{}
          {\cftwkc{\cftalg{A}}}
          { \jcomp{}
              {\cftctxext}
              { \jcomp{}
                  {\cftctxext}
                  {\cftfamalg{\cftfamalg{\cftalg{A}}}}
                }
            }
        }
$ are indeed judgmentally equal to the indicated domain and codomain. 

To see
that the two domains are equal, note that $\cftwkf{\cftalg{A}}$ may be seen as
a morphism from $\ctxwk{\cftalgf{\cftalg{A}}}{\cftalgf{\cftfamalg{\cftalg{A}}}}$
to $\cftalgf{\cftalg{A}}$ over 
$\jcomp{}{\cftwkc{\cftalg{A}}}{\jcomp{}{\cftctxext}{\cftctxext}}$ in context
$\Gamma$, i.e. we have the following commuting diagram in context $\Gamma$:
\begin{equation*}
\begin{tikzcd}[column sep=huge]
\ctxwk{\cftalgf{\cftalg{A}}}{\cftalgf{\cftfamalg{\cftalg{A}}}}
  \ar{r}
    {\cftwkf{\cftalg{A}}}
  \ar[fib]{d}
  &
\cftalgf{\cftalg{A}}
  \ar[fib]{d}
  \\
\ctxext{ {\cftalgc{\cftalg{A}}}
         {\cftalgf{\cftalg{A}}}
         }
       { \ctxwk{\cftalgf{\cftalg{A}}}{\cftalgf{\cftalg{A}}}
         }
  \ar{r}[swap]
    {\jcomp{}{\cftwkc{\cftalg{A}}}{\jcomp{}{\cftctxext}{\cftctxext}}}
  &
\cftalgc{\cftalg{A}}
\end{tikzcd}
\end{equation*}
Using the associativity of composition, we see that we have the judgmental 
equality
\begin{equation*}
\jcomp{}
  { \cftwkf{\cftalg{A}}}
  { \jcomp{}
      {\cftwkc{\cftalg{A}}}
      { \jcomp{}
          {\cftctxext}
          { \jcomp{}
              {\cftctxext}
              {\ctxwk{\cftalgf{\cftalg{A}}}{\cftfamalg{\cftalg{A}}}}
            }
        }
    }
\jdeq
\jcomp{}
  { \jvcomp{}
      { \jcomp{}
          { \cftwkc{\cftalg{A}}}
          { \jcomp{}
              {\cftctxext}
              {\cftctxext}
            }
        }
      { \cftwkf{\cftalg{A}}
        }
    }
  { \ctxwk{\cftalgf{\cftalg{A}}}{\cftfamalg{\cftalg{A}}}}  
\end{equation*}
Recall that weakening is the same as pulling
back along a projection. Therefore, we see that
it suffices to show that the diagram
\begin{equation*}
\begin{tikzcd}[column sep=10em]
\ctxext{ { { \cftalgc{\cftalg{A}}}
           { \cftalgf{\cftalg{A}}}
           }
         { \ctxwk{\cftalgf{\cftalg{A}}}{\cftalgf{\cftalg{A}}}}
         }
       { \ctxwk{\cftalgf{\cftalg{A}}}{\cftalgf{\cftfamalg{\cftalg{A}}}}
         }
  \ar{r}
    { \jvcomp{}
        { \jcomp{}
            {\cftwkc{\cftalg{A}}}
            { \jcomp{}
                {\cftctxext}
                {\cftctxext}
              }
          }
        { \cftwkf{\cftalg{A}}}
      }
  \ar{d}[swap]
    { \cprojfstf
        { { { \cftalgc{\cftalg{A}}}
            { \cftalgf{\cftalg{A}}}
            }
          { \ctxwk{\cftalgf{\cftalg{A}}}{\cftalgf{\cftalg{A}}}}
          }
        { \ctxwk{\cftalgf{\cftalg{A}}}{\cftalgf{\cftfamalg{\cftalg{A}}}}
          }
      }
  &
\ctxext{\cftalgc{\cftalg{A}}}{\cftalgf{\cftalg{A}}}
  \ar{d}
    {\cprojfstf{\cftalgc{\cftalg{A}}}{\cftalgf{\cftalg{A}}}}
  \\
\ctxext{ {\cftalgc{\cftalg{A}}}
         {\cftalgf{\cftalg{A}}}
         }
       { \ctxwk{\cftalgf{\cftalg{A}}}{\cftalgf{\cftalg{A}}}
         }
  \ar{r}[swap]
    {\jcomp{}{\cftwkc{\cftalg{A}}}{\jcomp{}{\cftctxext}{\cftctxext}}}
  &
\cftalgc{\cftalg{A}}
\end{tikzcd}
\end{equation*}
commutes judgmentally. This follows from \autoref{lem:jvcomp-cprojfstf}.

To see that the codomains are equal, use the fact that $\cftwk{\cftalg{A}}$ is
an extension homomorphism.
\end{enumerate}

\begin{lem}
Let $\cftalg{A}$ be an extension algebra in context $\Gamma$ and let
\begin{equation*}
\jhom
  {{{\Gamma}{\cftalgc{\cftalg{A}}}}{\cftalgf{\cftalg{A}}}}
  {\ctxwk{\cftalgf{\cftalg{A}}}{\cftfamalg{\cftalg{A}}}}
  {\cftfamalg{\cftfamalg{\cftalg{A}}}}
  {\cftwk{\cftalg{A}}}
\end{equation*}
be a homomorphism of extension algebras for which the diagram in
\autoref{diag:cftwk-cftwk} commutes judgmentally. 
Then the diagram
\begin{equation*}
\begin{tikzcd}[column sep=large]
\subst
  {\apply\cftctxext{\gamma}{b}}
  {\cftfamalg{\cftalg{A}}}
  \ar{r}{\apply\cftwk{\cftalg{A}}{\apply\cftctxext{\gamma}{b}}{q}}
  \ar{d}[swap]{\subst{b}{{a}{{\gamma}{\cftfamhom{\cftwk{\cftalg{A}}}}}}}
& \subst
    { q}
    { {\apply\cftctxext{\gamma}{b}}
      {\cftfamalg{\cftfamalg{\cftalg{A}}}}
      }
  \ar{d}
    { \subst
        {\apply\cftfamext{\gamma}{b}{q}}
        {{a}{{\gamma}{\cftfamhom{\cftwk{\cftalg{A}}}}}}
      }
  \\
\subst
  { \apply\cftwkc{\cftalg{A}}{\gamma}{a}{b}
    }
  { { \apply\cftctxext{\gamma}{a}}
    { \cftfamalg{\cftfamalg{\cftalg{A}}}}
    }
  \ar{r}[swap,yshift=-1em]
    { \subst
        { \apply\cftwkf{\cftalg{A}}{\gamma}{a}{b}{q}}
        { {\apply\cftctxext{\apply\cftctxext{\gamma}{a}}{\apply\cftwkc{\cftalg{A}}{\gamma}{a}{b}}}
          {\cftwk{\cftalg{A}}}
          }
      }
& \subst
    { \apply\cftwkf{\cftalg{A}}{\gamma}{a}{b}{q}
      }
    { {\apply\cftwkc{\cftalg{A}}{\gamma}{a}{b}}
      {{a}{{\gamma}{\cftfamalg{\cftfamalg{\cftfamalg{\cftfamalg{\cftalg{A}}}}}}}}
      }
\end{tikzcd}
\end{equation*}
of extension algebras in context $\Gamma$ commutes judgmentally for
$\gamma:A$, $a,b:\subst{\gamma}{\cftalgf{\cftalg{A}}}$ and
$q:\subst{b}{{\gamma}{\cftalgf{\cftfamalg{\cftalg{A}}}}}$.
\end{lem}

\begin{proof}
All one has to do to prove this is to substitute all of the ingredients of
the diagram in \autoref{diag:cftwk-cftwk} consecutively by
$\gamma$, by $a$, by $b$ and finally by $q$.
\end{proof}
\end{comment}

\subsubsection{The definition of weakening algebras}\label{sec:cftwkalg-defn}

\begin{defn}
A weakening algebra in context $\Gamma$ is a pre-weakening algebra $\cftalg{A}$
in context $\Gamma$ for which the weakening $\cftwk{\cftalg{A}}$ is a
pre-weakening homomorphism as explained in \autoref{rmk:cftwk-prewk},
satisfying the two additional judgmental equalities
expressing that: (1) the diagram
\begin{equation}\label{diag:cftwk-cftextcurry}
\begin{tikzcd}[column sep=large]
\ctxwk
  {\ctxext{\cftalgf{\cftalg{A}}}{\cftalgf{\cftfamalg{\cftalg{A}}}}}
  {\cftfamalg{\cftalg{A}}}
  \ar{r}{\ctxwk{\cftalgf{\cftfamalg{\cftalg{A}}}}{\cftwk{\cftalg{A}}}}
  \ar{dr}[swap]{\jcomp{}{\cftfamext}{\cftwk{\cftalg{A}}}}
& \ctxwk{\cftalgf{\cftfamalg{\cftalg{A}}}}{\cftfamalg{\cftfamalg{\cftalg{A}}}}
  \ar{d}{\jcomp{}{\cftctxext}{\cftwk{\cftalg{A}}}}
  \\
& \cftfamalg{\cftfamalg{\cftfamalg{\cftalg{A}}}}
\end{tikzcd}
\end{equation}
of pre-weakening algebras in context $\ctxext{{{\Gamma}{\cftalgc{\cftalg{A}}}}{\cftalgf{\cftalg{A}}}}{\cftalgf{\cftfamalg{\cftalg{A}}}}$ commutes judgmentally and (2) that
\begin{equation}\label{eq:cftwk-cftempcurry}
\jhomeq
  { {\Gamma}{\cftalgc{\cftalg{A}}}
    }
  { \cftfamalg{\cftalg{A}}
    }
  { \cftfamalg{\cftalg{A}}
    }
  { \subst{\cftempf{\cftalg{A}}}{\cftwk{\cftalg{A}}}
    }
  { \cftidhom{\cftfamalg{\cftalg{A}}}
    }
\end{equation}
\end{defn}

\begin{rmk}
The condition in \autoref{diag:cftwk-cftextcurry} implements the rules of
\autoref{comp-ew} and the condition in \autoref{eq:cftwk-cftempcurry} implements
the rules of \autoref{comp-0w}.
\end{rmk}

\begin{defn}
A \emph{weakening homomorphism} is a pre-weakening homomorphism between
weakening algebras.
\end{defn}

\subsubsection{Derivable properties of weakening algebras}

\begin{thm}
Suppose $\cftalg{A}$ is a weakening algebra in context $\Gamma$. Then
$\cftfamalg{\cftalg{A}}$ together with the extension homomorphism
\begin{equation*}
\cftwk{\cftfamalg{\cftalg{A}}}
  \defeq \jcomp{}{\cftctxext}{\cftwk{\cftalg{A}}}
\end{equation*}
forms a weakening algebra in context $\ctxext{\Gamma}{\cftalgc{\cftalg{A}}}$.
\end{thm}

\begin{proof}
To show that $\cftwk{\cftfamalg{\cftalg{A}}}$ satisfies the Currying rule, we
must verify that the diagram
\begin{equation*}
\begin{tikzcd}[column sep=large]
\ctxwk
  {\ctxext{\cftalgf{\cftfamalg{\cftalg{A}}}}{\cftalgf{\cftfamalg{\cftfamalg{\cftalg{A}}}}}}
  {\cftfamalg{\cftfamalg{\cftalg{A}}}}
  \ar{r}{\ctxwk{\cftalgf{\cftfamalg{\cftfamalg{\cftalg{A}}}}}{\cftwk{\cftfamalg{\cftalg{A}}}}}
  \ar{dr}[swap]{\jcomp{}{\jcomp{}{\cftctxext}{\cftfamext}}{\cftwk{\cftfamalg{\cftalg{A}}}}}
& \ctxwk{\cftalgf{\cftfamalg{\cftfamalg{\cftalg{A}}}}}{\cftfamalg{\cftfamalg{\cftfamalg{\cftalg{A}}}}}
  \ar{d}{\jcomp{}{\cftfamext}{\cftwk{\cftfamalg{\cftalg{A}}}}}
  \\
& \cftfamalg{\cftfamalg{\cftfamalg{\cftfamalg{\cftalg{A}}}}}
\end{tikzcd}
\end{equation*}
commutes judgmentally. We do this by showing that each of the ingredients of this
diagram is judgmentally equal to the corresponding ingredient of the diagram in
\autoref{diag:cftwk-cftextcurry} pulled back along $\cftctxext$. Thus, we are going
to show that
\begin{align*}
\ctxwk{\cftalgf{\cftfamalg{\cftfamalg{\cftalg{A}}}}}{\cftwk{\cftfamalg{\cftalg{A}}}}
& \jdeq
\jcomp{}{\cftctxext}{\ctxwk{\cftalgf{\cftfamalg{\cftalg{A}}}}{\cftwk{\cftalg{A}}}}
  \tag{1}
  \\
\jcomp{}{\cftfamext}{\cftwk{\cftfamalg{\cftalg{A}}}}
& \jdeq
\jcomp{}{\cftctxext}{\jcomp{}{\cftctxext}{\cftwk{\cftalg{A}}}}
  \tag{2}
  \\
\jcomp{}{\jcomp{}{\cftctxext}{\cftfamext}}{\cftwk{\cftfamalg{\cftalg{A}}}}
& \jdeq
\jcomp{}{\cftctxext}{\jcomp{}{\cftfamext}{\cftwk{\cftalg{A}}}}
  \tag{3}
\end{align*}
For the first, we have the judgmental equalities
\begin{align*}
\ctxwk{\cftalgf{\cftfamalg{\cftfamalg{\cftalg{A}}}}}{\cftwk{\cftfamalg{\cftalg{A}}}}
& \jdeq
\ctxwk
  {\jcomp{}{\cftctxext}{\cftalgf{\cftfamalg{\cftalg{A}}}}}
  {\jcomp{}{\cftctxext}{\cftwk{\cftalg{A}}}}
  \\
& \jdeq
\jcomp{}{\cftctxext}{\ctxwk{\cftalgf{\cftfamalg{\cftalg{A}}}}{\cftwk{\cftalg{A}}}}
\end{align*}
For the second, we have the judgmental equalities
\begin{align*}
\jcomp{}{\cftfamext}{\cftwk{\cftfamalg{\cftalg{A}}}}
& \jdeq
  \jcomp{}{\cftfamext}{\jcomp{}{\cftctxext}{\cftwk{\cftalg{A}}}}
  \\
& \jdeq
  \jcomp{}{\cftctxext}{\jcomp{}{\cftctxext}{\cftwk{\cftalg{A}}}}
\end{align*}
For the third, note that we have the judgmental equalities
\begin{align*}
\jcomp{}{\jcomp{}{\cftctxext}{\cftfamext}}{\cftwk{\cftfamalg{\cftalg{A}}}}
& \jdeq
  \jcomp{}{\jcomp{}{\cftctxext}{\cftfamext}}{\jcomp{}{\cftctxext}{\cftwk{\cftalg{A}}}}
  \\
& \jdeq
\jcomp{}{\cftctxext}{\jcomp{}{\cftfamext}{\cftwk{\cftalg{A}}}}
\end{align*}
Thus we may conclude that the diagram above a pullback of the diagram in 
\autoref{diag:cftwk-cftextcurry} by $\cftctxext$ and therefore it commutes as desired.

To show that $\cftwk{\cftfamalg{\cftalg{A}}}$ preserves itself, we must verify
that the diagram
\begin{equation*}
\begin{tikzcd}[column sep=huge]
\ctxwk{\cftalgf{\cftfamalg{\cftalg{A}}}}{{\cftalgf{\cftfamalg{\cftfamalg{\cftalg{A}}}}}{\cftfamalg{\cftfamalg{\cftfamalg{\cftalg{A}}}}}}
  \ar{d}
    [swap]{ \ctxwk
        {{\cftalgf{\cftfamalg{\cftalg{A}}}}{\cftalgf{\cftfamalg{\cftfamalg{\cftalg{A}}}}}}
        {\cftfamhom{\cftwk{\cftfamalg{\cftalg{A}}}}}
      }
  \ar{r}{\ctxwk{\cftalgf{\cftfamalg{\cftalg{A}}}}{\jcomp{}{\cftfamext}{\cftwk{\cftfamalg{\cftalg{A}}}}}}
& \ctxwk{\cftalgf{\cftfamalg{\cftalg{A}}}}{\cftfamalg{\cftfamalg{\cftfamalg{\cftfamalg{\cftalg{A}}}}}}
  \ar{d}{\jcomp{}{\ctxwk{\cftalgf{\cftfamalg{\cftalg{A}}}}{\jcomp{}{\cftctxext}{\cftfamext}}}{\cftfamhom{\cftwk{\cftfamalg{\cftalg{A}}}}}}
  \\
\ctxwk{{\cftalgf{\cftfamalg{\cftalg{A}}}}{\cftalgf{\cftfamalg{\cftfamalg{\cftalg{A}}}}}}{\jcomp{}{\cftwkc{\cftfamalg{\cftalg{A}}}}{\cftfamalg{\cftfamalg{\cftfamalg{\cftfamalg{\cftalg{A}}}}}}}
  \ar{r}[swap,yshift=-1em]{%
    \jcomp{}
      { \cftwkf{\cftfamalg{\cftalg{A}}}}
      { \jcomp{}
          {\cftwkc{\cftfamalg{\cftalg{A}}}}
          { \jcomp{}
              {\cftfamext}
              { \jcomp{}
                  {\cftfamext}
                  {\cftwk{\cftfamalg{\cftalg{A}}}}
                }
            }
        }
    }
& \jcomp{}
    {\ctxwk{\cftalgf{\cftfamalg{\cftalg{A}}}}{\jcomp{}{\cftctxext}{\cftfamext}}}
    {\jcomp{}{\cftwkc{\cftfamalg{\cftalg{A}}}}{\cftfamalg{\cftfamalg{\cftfamalg{\cftfamalg{\cftalg{A}}}}}}}
\end{tikzcd}
\end{equation*}
commutes judgmentally. We do this by showing that each of the ingredients of this
diagram is judgmentally equal to the corresponding ingredient of the diagram in
\autoref{diag:cftwk-cftwk} pulled back along $\cftctxext$. Thus, we are going
to show that
\begin{align*}
\ctxwk
  {\cftalgf{\cftfamalg{\cftalg{A}}}}
  {\jcomp{}{\cftfamext}{\cftwk{\cftfamalg{\cftalg{A}}}}}
& \jdeq
  \jcomp{}
    { \cftctxext}
    { \ctxwk{\cftalgf{\cftalg{A}}}{\jcomp{}{\cftctxext}{\cftwk{\cftalg{A}}}}}
  \tag{1}
  \\
\jcomp{}
  {\ctxwk{\cftalgf{\cftfamalg{\cftalg{A}}}}{\jcomp{}{\cftctxext}{\cftfamext}}}
  {\cftfamhom{\cftwk{\cftfamalg{\cftalg{A}}}}}
& \jdeq
  \jcomp{}
    { \cftctxext}
    { \jcomp{}
        {\ctxwk{\cftalgf{\cftalg{A}}}{\cftfamext}}
        {\cftfamhom{\cftwk{\cftalg{A}}}}
      }
  \tag{2}
  \\
\ctxwk
  {{\cftalgf{\cftfamalg{\cftalg{A}}}}{\cftalgf{\cftfamalg{\cftfamalg{\cftalg{A}}}}}}
  {\cftfamhom{\cftwk{\cftfamalg{\cftalg{A}}}}}
& \jdeq
  \jcomp{}
    { \cftctxext}
    { \ctxwk
        {{\cftalgf{\cftalg{A}}}{\cftalgf{\cftfamalg{\cftalg{A}}}}}
        {\cftfamhom{\cftwk{\cftalg{A}}}}
      }
  \tag{3}
  \\
\jcomp{}
      { \cftwkf{\cftfamalg{\cftalg{A}}}}
      { \jcomp{}
          {\cftwkc{\cftfamalg{\cftalg{A}}}}
          { \jcomp{}
              {\cftfamext}
              { \jcomp{}
                  {\cftfamext}
                  {\cftwk{\cftfamalg{\cftalg{A}}}}
                }
            }
        }
& \jdeq
  \jcomp{}
    { \cftctxext}
    { \jcomp{}
        { \cftwkf{\cftalg{A}}}
        { \jcomp{}
            {\cftwkc{\cftalg{A}}}
            { \jcomp{}
                {\cftctxext}
                { \jcomp{}
                    {\cftctxext}
                    {\cftwk{\cftalg{A}}}
                  }
              }
          }
      }
  \tag{4}
\end{align*}
For the first, note that we have the judgmental equalities
\begin{align*}
\ctxwk
  {\cftalgf{\cftfamalg{\cftalg{A}}}}
  {\jcomp{}{\cftfamext}{\cftwk{\cftfamalg{\cftalg{A}}}}}
& \jdeq 
  \ctxwk
    {\jcomp{}{\cftctxext}{\cftalgf{\cftalg{A}}}}
    {\jcomp{}{\cftfamext}{\jcomp{}{\cftctxext}{\cftwk{\cftalg{A}}}}}
  \\
& \jdeq
  \ctxwk
    {\jcomp{}{\cftctxext}{\cftalgf{\cftalg{A}}}}
    {\jcomp{}{\cftctxext}{\jcomp{}{\cftctxext}{\cftwk{\cftalg{A}}}}}
  \\
& \jdeq
  \jcomp{}
    { \cftctxext}
    { \ctxwk{\cftalgf{\cftalg{A}}}{\jcomp{}{\cftctxext}{\cftwk{\cftalg{A}}}}}
\end{align*}
For the second, note that we have the judgmental equalities
\begin{align*}
\jcomp{}
  {\ctxwk{\cftalgf{\cftfamalg{\cftalg{A}}}}{\jcomp{}{\cftctxext}{\cftfamext}}}
  {\cftfamhom{\cftwk{\cftfamalg{\cftalg{A}}}}}
& \jdeq
  \jcomp{}
    {\ctxwk{\jcomp{}{\cftctxext}{\cftalgf{\cftalg{A}}}}{\jcomp{}{\cftctxext}{\cftfamext}}}
    {\cftfamhom{\jcomp{}{\cftctxext}{\cftwk{\cftalg{A}}}}}
  \\
& \jdeq
  \jcomp{}
    {\jcomp{}{\cftctxext}{\ctxwk{\cftalgf{\cftalg{A}}}{\cftfamext}}}
    {\cftfamhom{\jcomp{}{\cftctxext}{\cftwk{\cftalg{A}}}}}
  \\
& \jdeq
  \jcomp{}
    {\jcomp{}{\cftctxext}{\ctxwk{\cftalgf{\cftalg{A}}}{\cftfamext}}}
    {\jcomp{}{\cftctxext}{\cftfamhom{\cftwk{\cftalg{A}}}}}
  \\
& \jdeq
  \jcomp{}
    { \cftctxext}
    { \jcomp{}
        {\ctxwk{\cftalgf{\cftalg{A}}}{\cftfamext}}
        {\cftfamhom{\cftwk{\cftalg{A}}}}
      }
\end{align*}
For the third, note that we have the judgmental equalities
\begin{align*}
\ctxwk
  {{\cftalgf{\cftfamalg{\cftalg{A}}}}{\cftalgf{\cftfamalg{\cftfamalg{\cftalg{A}}}}}}
  {\cftfamhom{\cftwk{\cftfamalg{\cftalg{A}}}}}
& \jdeq
  \ctxwk
    {{\jcomp{}{\cftctxext}{\cftalgf{\cftalg{A}}}}{\jcomp{}{\cftctxext}{\cftalgf{\cftfamalg{\cftalg{A}}}}}}
    {\cftfamhom{\cftwk{\cftfamalg{\cftalg{A}}}}}
  \\
& \jdeq
  \ctxwk
    {\jcomp{}{\cftctxext}{\ctxwk{\cftalgf{\cftalg{A}}}{\cftalgf{\cftfamalg{\cftalg{A}}}}}}
    {\cftfamhom{\cftwk{\cftfamalg{\cftalg{A}}}}}
  \\
& \jdeq
  \ctxwk
    {\jcomp{}{\cftctxext}{\ctxwk{\cftalgf{\cftalg{A}}}{\cftalgf{\cftfamalg{\cftalg{A}}}}}}
    {\cftfamhom{\jcomp{}{\cftctxext}{\cftwk{\cftalg{A}}}}}
  \\
& \jdeq
  \ctxwk
    {\jcomp{}{\cftctxext}{\ctxwk{\cftalgf{\cftalg{A}}}{\cftalgf{\cftfamalg{\cftalg{A}}}}}}
    {\jcomp{}{\cftctxext}{\cftfamhom{\cftwk{\cftalg{A}}}}}
  \\
& \jdeq
  \jcomp{}
    { \cftctxext}
    { \ctxwk
        {{\cftalgf{\cftalg{A}}}{\cftalgf{\cftfamalg{\cftalg{A}}}}}
        {\cftfamhom{\cftwk{\cftalg{A}}}}
      }
\end{align*}
For the fourth, note that we have the judgmental equalities
\begin{align*}
\jcomp{}
      { \cftwkf{\cftfamalg{\cftalg{A}}}}
      { \jcomp{}
          {\cftwkc{\cftfamalg{\cftalg{A}}}}
          { \jcomp{}
              {\cftfamext}
              { \jcomp{}
                  {\cftfamext}
                  {\cftwk{\cftfamalg{\cftalg{A}}}}
                }
            }
        }
& \jdeq
  \jcomp{}
      { \cftwkf{\cftfamalg{\cftalg{A}}}}
      { \jcomp{}
          { \jcomp{}{\cftctxext}{\cftwkc{\cftalg{A}}}}
          { \jcomp{}
              {\cftfamext}
              { \jcomp{}
                  {\cftfamext}
                  {\jcomp{}{\cftctxext}{\cftwk{\cftalg{A}}}}
                }
            }
        }
  \\
& \jdeq
  \jcomp{}
      { \cftwkf{\cftfamalg{\cftalg{A}}}}
      { \jcomp{}
          { \jcomp{}{\cftctxext}{\cftwkc{\cftalg{A}}}}
          { \jcomp{}
              {\cftctxext}
              { \jcomp{}
                  {\cftctxext}
                  {\jcomp{}{\cftctxext}{\cftwk{\cftalg{A}}}}
                }
            }
        }
  \\
& \jdeq
  \jcomp{}
      { \cftwkf{\cftfamalg{\cftalg{A}}}}
      { \jcomp{}
          { \cftctxext}
          { \jcomp{}
              { \cftwkc{\cftalg{A}}}
              { \jcomp{}
                  {\cftctxext}
                  {\jcomp{}{\cftctxext}{\cftwk{\cftalg{A}}}}
                }
            }
        }
  \\
& \jdeq
  \jcomp{}
      { \jcomp{}{\cftctxext}{\cftwkf{\cftalg{A}}}}
      { \jcomp{}
          { \cftctxext}
          { \jcomp{}
              { \cftwkc{\cftalg{A}}}
              { \jcomp{}
                  {\cftctxext}
                  {\jcomp{}{\cftctxext}{\cftwk{\cftalg{A}}}}
                }
            }
        }
  \\
& \jdeq
  \jcomp{}
    { \cftctxext}
    { \jcomp{}
        { \cftwkf{\cftalg{A}}}
        { \jcomp{}
            {\cftwkc{\cftalg{A}}}
            { \jcomp{}
                {\cftctxext}
                { \jcomp{}
                    {\cftctxext}
                    {\cftwk{\cftalg{A}}}
                  }
              }
          }
      }
\end{align*}
Thus we may conclude that the diagram above is a pullback of the diagram in \autoref{diag:cftwk-cftwk}
by $\cftctxext$ and therefore it commutes as desired.
\end{proof}

\begin{thm}
Suppose $\cftalg{Q}$ is a weakening algebra in context $\ctxext{\Gamma}{B}$ and
let $\jfam{\Gamma}{A}$. Then $\ctxwk{A}{\cftalg{Q}}$ together with the
extension homomorphism
\begin{equation*}
\cftwk{\ctxwk{A}{\cftalg{Q}}}\defeq \ctxwk{A}\cftwk{\cftalg{Q}}
\end{equation*}
forms a weakening algebra.
\end{thm}

\begin{thm}
Suppose $\cftalg{Q}$ is a weakening algebra in context $\ctxext{{\Gamma}{A}}{P}$
and let $\jterm{\Gamma}{A}{x}$. Then $\subst{x}{\cftalg{Q}}$ together with the
extension homomorphism
\begin{equation*}
\cftwk{\subst{x}{\cftalg{Q}}} \defeq \subst{x}{\cftwk{\cftalg{Q}}}
\end{equation*}
forms a weakening algebra.
\end{thm}

%%%%%%%%%%%%%%%%%%%%%%%%%%%%%%%%%%%%%%%%%%%%%%%%%%%%%%%%%%%%%%%%%%%%%%%%%%%%%%%%
\subsection{Projection algebras}
Projection algebras will be weakening algebras with additional terms implementing
the identity terms. We call these algebras projection algebras because the
identity terms can only be formulated in weakening algebras and together with
weakening, the identity terms provide for all the projections. 

\begin{defn}
A \emph{pre-projection algebra $\cftalg{A}$ in context $\Gamma$} consists of a 
weakening algebra $\cftalg{A}$ in context $\Gamma$ together with an 
\emph{internal identity term}
\begin{equation*}
\jterm
  { { {\Gamma}
      {\cftalgc{\cftalg{A}}}
      }
    { \cftalgf{\cftalg{A}}
      }
    }
  {\subst{{\idtm{\cftalgf{\cftalg{A}}}}{\cftwkc{\cftalg{A}}}}{\cftalgt{\cftfamalg{\cftalg{A}}}}}
  {\cftidtm{\cftalg{A}}}.
\end{equation*}
\end{defn}

\begin{defn}
Suppose $\cftalg{A}$ is a pre-projection algebra in context $\Gamma$. Then we
define the pre-projection algebra $\cftfamalg{\cftalg{A}}$ in context 
$\ctxext{\Gamma}{\cftalgc{\cftalg{A}}}$ by taking
\begin{equation*}
\cftidtm{\cftfamalg{\cftalg{A}}}
  \defeq \jcomp{}{\cftctxext}{\cftidtm{\cftalg{A}}}.
\end{equation*}
\end{defn}

\begin{defn}
Suppose $\cftalg{Q}$ is a pre-projection algebra in context $\ctxext{\Gamma}{B}$ and
let $\jfam{\Gamma}{A}$. Then we define the pre-projection algebra 
$\ctxwk{A}{\cftalg{Q}}$ in context $\ctxext{{\Gamma}{A}}{\ctxwk{A}{B}}$ by 
taking
\begin{equation*}
\cftidtm{\ctxwk{A}{\cftalg{Q}}}\defeq \ctxwk{A}\cftidtm{\cftalg{Q}}.
\end{equation*}
\end{defn}

\begin{defn}
Suppose $\cftalg{Q}$ is a pre-projection algebra in context $\ctxext{{\Gamma}{A}}{P}$
and let $\jterm{\Gamma}{A}{x}$. Then we define the pre-projection algebra 
$\subst{x}{\cftalg{Q}}$ in context $\ctxext{\Gamma}{\subst{x}{P}}$ by taking
\begin{equation*}
\cftidtm{\subst{x}{\cftalg{Q}}} \defeq \subst{x}{\cftidtm{\cftalg{Q}}}.
\end{equation*}
\end{defn}

\begin{defn}
A pre-projection homomorphism from $\cftalg{A}$ to $\cftalg{B}$ in context
$\Gamma$ is a weakening homomorphism $\cfthom{f}$ from $\cftalg{A}$ to
$\cftalg{B}$ in context $\Gamma$ for which the diagram
\begin{equation*}
\begin{tikzcd}[column sep=10em]
\subst
  {{\idtm{\cftalgf{\cftalg{A}}}}{\cftwkc{\cftalg{A}}}}
  {\cftalgt{\cftfamalg{\cftalg{A}}}}
  \ar{r}{\subst{{\idtm{\cftalgf{\cftalg{A}}}}{\cftwkc{\cftalg{A}}}}{\cfthomt{\cftfamhom{\cfthom{f}}}}}
  \ar[fib,xshift=-.7ex]{d}
  &
\subst
  {{\idtm{\cftalgf{\cftalg{B}}}}{\cftwkc{\cftalg{B}}}}
  {\cftalgt{\cftfamalg{\cftalg{B}}}}
  \ar[fib,xshift=-.7ex]{d}
  \\
\cftalgf{\cftalg{A}}
  \ar{r}[swap]{\cfthomf{\cfthom{f}}}
  \ar[fib]{d}
  \ar[densely dotted,xshift=.7ex]{u}[swap]{\cftidtm{\cftalg{A}}}
  &
\cftalgf{\cftalg{B}}
  \ar[fib]{d}
  \ar[densely dotted,xshift=.7ex]{u}[swap]{\cftidtm{\cftalg{B}}}
  \\
\cftalgc{\cftalg{A}}
  \ar{r}[swap]{\cfthomc{\cfthom{f}}}
  &
\cftalgc{\cftalg{B}}
\end{tikzcd}
\end{equation*}
\end{defn}

\begin{defn}
A projection algebra in context $\Gamma$ is a pre-projection algebra in context
$\Gamma$ for which weakening is a pre-projection homomorphism.
\end{defn}

\begin{defn}
A projection homomorphism is a pre-projection homomorphism between projection
algebras.
\end{defn}

\begin{thm}
Suppose $\cftalg{A}$ is a projection algebra in context $\Gamma$. Then the 
pre-projection algebra $\cftfamalg{\cftalg{A}}$ is a projection algebra in context 
$\ctxext{\Gamma}{\cftalgc{\cftalg{A}}}$.
\end{thm}

\begin{thm}
Suppose $\cftalg{Q}$ is a projection algebra in context $\ctxext{\Gamma}{B}$ and
let $\jfam{\Gamma}{A}$. Then the pre-projection algebra 
$\ctxwk{A}{\cftalg{Q}}$ is a
projection algebra in context $\ctxext{{\Gamma}{A}}{\ctxwk{A}{B}}$.
\end{thm}

\begin{thm}
Suppose $\cftalg{Q}$ is a projection algebra in context $\ctxext{{\Gamma}{A}}{P}$
and let $\jterm{\Gamma}{A}{x}$. Then the pre-projection algebra 
$\subst{x}{\cftalg{Q}}$ is a
projection algebra in context $\ctxext{\Gamma}{\subst{x}{P}}$.
\end{thm}

\begin{cor}
In a projection algebra $\cftalg{A}$ in context $\Gamma$, weakening is a
projection homomorphism.
\end{cor}

%%%%%%%%%%%%%%%%%%%%%%%%%%%%%%%%%%%%%%%%%%%%%%%%%%%%%%%%%%%%%%%%%%%%%%%%%%%%%%%%
\subsection{Substitution algebras}
\begin{defn}
A \emph{pre-substitution algebra $\cftalg{A}$ in context $\Gamma$} consists of an extension
algebra $\cftalg{A}$ in context $\Gamma$ and an extension homomorphism
\begin{equation*}
\jhom
  {{{{\Gamma}{\cftalgc{\cftalg{A}}}}{\cftalgf{\cftalg{A}}}}{\cftalgt{\cftalg{A}}}}
  {\ctxwk{\cftalgt{\cftalg{A}}}{\cftalg{F_{F_A}}}}
  {\ctxwk{\cftalgt{\cftalg{A}}}{{\cftalgf{\cftalg{A}}}{\cftfamalg{\cftalg{A}}}}}
  {\cftsubst{\cftalg{A}}}.
\end{equation*}
\end{defn}

\begin{defn}
Suppose $\cftalg{A}$ is a pre-substitution algebra in context $\Gamma$. Then we
define the pre-substitution algebra $\cftfamalg{\cftalg{A}}$ in context 
$\ctxext{\Gamma}{\cftalgc{\cftalg{A}}}$ by taking
\begin{equation*}
\cftsubst{\cftfamalg{\cftalg{A}}}
  \defeq \jcomp{}{\cftctxext}{\cftsubst{\cftalg{A}}}.
\end{equation*}
\end{defn}

\begin{defn}
Suppose $\cftalg{Q}$ is a pre-substitution algebra in context $\ctxext{\Gamma}{B}$ and
let $\jfam{\Gamma}{A}$. Then we define the pre-substitution algebra 
$\ctxwk{A}{\cftalg{Q}}$ in context $\ctxext{{\Gamma}{A}}{\ctxwk{A}{B}}$ by 
taking
\begin{equation*}
\cftsubst{\ctxwk{A}{\cftalg{Q}}}\defeq \ctxwk{A}\cftsubst{\cftalg{Q}}.
\end{equation*}
\end{defn}

\begin{defn}
Suppose $\cftalg{Q}$ is a pre-substitution algebra in context $\ctxext{{\Gamma}{A}}{P}$
and let $\jterm{\Gamma}{A}{x}$. Then we define the pre-substitution algebra 
$\subst{x}{\cftalg{Q}}$ in context $\ctxext{\Gamma}{\subst{x}{P}}$ by taking
\begin{equation*}
\cftsubst{\subst{x}{\cftalg{Q}}} \defeq \subst{x}{\cftsubst{\cftalg{Q}}}.
\end{equation*}
\end{defn}

\begin{defn}
A \emph{pre-substitution homomorphism $\cfthom{f}$ from $\cftalg{A}$ to $\cftalg{B}$ in
context $\Gamma$} is an extension homomorphism $\cfthom{f}$ for which the diagram
\begin{equation*}
\begin{tikzcd}[column sep=8em]
\ctxwk{\cftalgt{\cftalg{A}}}{\cftfamalg{\cftfamalg{\cftalg{A}}}}
  \ar{r}{\ctxwk{\cftalgt{\cftalg{A}}}{\cftfamhom{\cftfamhom{\cfthom{f}}}}}
  \ar{d}[swap]{\cftsubst{\cftalg{A}}}
  &
\jcomp{}
  { \cfthomt{\cfthom{f}}}
  { \jcomp{}
      { \cfthomf{\cfthom{f}}}
      { \jcomp{}
          { \cfthomc{\cfthom{f}}}
          { \ctxwk{\cftalgt{\cftalg{B}}}{\cftfamalg{\cftfamalg{\cftalg{B}}}}}
        }
    }
  \ar{d}
    { \jcomp{}
        { \cfthomt{\cfthom{f}}}
        { \jcomp{}
            { \cfthomf{\cfthom{f}}}
            { \jcomp{}
                { \cfthomc{\cfthom{f}}}
                {\cftsubst{\cftalg{B}}}
              }
          }
      }   
  \\
\ctxwk{\cftalgt{\cftalg{A}}}{{\cftalgf{\cftalg{A}}}{\cftfamalg{\cftalg{A}}}}
  \ar{r}[swap]
    {\ctxwk{\cftalgt{\cftalg{A}}}{{\cftalgf{\cftalg{A}}}{\cftfamhom{\cfthom{f}}}}}
  &
\jcomp{}
  { \cfthomt{\cfthom{f}}}
  { \jcomp{}
      { \cfthomf{\cfthom{f}}}
      { \jcomp{}
          { \cfthomc{\cfthom{f}}}
          { \ctxwk
              {\cftalgt{\cftalg{B}}}
              {{\cftalgf{\cftalg{B}}}{\cftfamalg{\cftalg{B}}}}
            }
        }
    }
\end{tikzcd}
\end{equation*}
of pre-substitution algebras in context
$\ctxext{{{\Gamma}{\cftalgc{\cftalg{A}}}}{\cftalgf{\cftalg{A}}}}{\cftalgt{\cftalg{A}}}$
commutes judgmentally.
\end{defn}

\begin{defn}
A \emph{substitution algebra in context $\Gamma$} is a pre-substitution algebra
in context $\Gamma$ for which substitution is a pre-substitution homomorphism.
\end{defn}

\begin{rmk}
The condition that the substitution of a pre-substitution algebra is a
pre-substitution homomorphism assert that the diagram
\begin{small}
\begin{equation*}
\begin{tikzcd}[column sep=large]
\ctxwk
  {\cftalgt{\ctxwk{\cftalgt{\cftalg{A}}}{\cftalg{F_{F_A}}}}}
  {\cftfamalg{\cftfamalg{\ctxwk{\cftalgt{\cftalg{A}}}{\cftalg{F_{F_A}}}}}}
  \ar{r}
    { \ctxwk
        {\cftalgt{\ctxwk{\cftalgt{\cftalg{A}}}{\cftalg{F_{F_A}}}}}
        {\cftfamhom{\cftfamhom{\cftsubst{\cftalg{A}}}}}
      }
  \ar{d}[swap]
    {\cftsubst{\ctxwk{\cftalgt{\cftalg{A}}}{\cftalg{F_{F_A}}}}}
  &
\jcomp{}
  { \cftsubstt{\cftalg{A}}}
  { \jcomp{}
      { \cftsubstf{\cftalg{A}}}
      { \jcomp{}
          { \cftsubstc{\cftalg{A}}}
          { \ctxwk
              { \cftalgt
                  { \ctxwk
                      { \cftalgt{\cftalg{A}}}
                      { {\cftalgf{\cftalg{A}}}{\cftfamalg{\cftalg{A}}}}
                    }
                }
              { \cftfamalg
                  { \cftfamalg
                      { \ctxwk
                          { \cftalgt{\cftalg{A}}}
                          { {\cftalgf{\cftalg{A}}}{\cftfamalg{\cftalg{A}}}}
                        }
                    }
                }
            }
        }
    }
  \ar{d}
    { \jcomp{}
        { \cftsubstt{\cftalg{A}}}
        { \jcomp{}
            { \cftsubstf{\cftalg{A}}}
            { \jcomp{}
                { \cftsubstc{\cftalg{A}}}
                { \cftsubst
                    { \ctxwk
                        {\cftalgt{\cftalg{A}}}
                        {{\cftalgf{\cftalg{A}}}{\cftfamalg{\cftalg{A}}}}
                      }
                  }
              }
          }
      }
  \\
\ctxwk
  { \cftalgt{\ctxwk{\cftalgt{\cftalg{A}}}{\cftalg{F_{F_A}}}}}
  { {\cftalgf{\ctxwk{\cftalgt{\cftalg{A}}}{\cftalg{F_{F_A}}}}}
    {\cftfamalg{\ctxwk{\cftalgt{\cftalg{A}}}{\cftalg{F_{F_A}}}}}
    }
  \ar{r}[swap,yshift=-1em]
    { \ctxwk
        { \cftalgt{\ctxwk{\cftalgt{\cftalg{A}}}{\cftalg{F_{F_A}}}}}
        { {\cftalgf{\ctxwk{\cftalgt{\cftalg{A}}}{\cftalg{F_{F_A}}}}}
          {\cftfamhom{\cftsubst{\cftalg{A}}}}}
          }
  &
\jcomp{}
  { \cftsubstt{\cftalg{A}}}
  { \jcomp{}
      { \cftsubstf{\cftalg{A}}}
      { \jcomp{}
          { \cftsubstc{\cftalg{A}}}
          { \ctxwk
              { \cftalgt
                  { \ctxwk
                      {\cftalgt{\cftalg{A}}}
                      {{\cftalgf{\cftalg{A}}}{\cftfamalg{\cftalg{A}}}}
                    }
                }
              { { \cftalgf
                    { \ctxwk
                        { \cftalgt{\cftalg{A}}}
                        {{\cftalgf{\cftalg{A}}}{\cftfamalg{\cftalg{A}}}}
                      }
                  }
                { \cftfamalg
                    { \ctxwk
                        {\cftalgt{\cftalg{A}}}
                        {{\cftalgf{\cftalg{A}}}{\cftfamalg{\cftalg{A}}}}
                      }
                  }
                }
            }
        }
    }
\end{tikzcd}
\end{equation*}
\end{small}
\end{rmk}

\begin{defn}
A substitution homomorphism is a pre-substitution homomorphism between
substitution algebras.
\end{defn}

\begin{thm}
Suppose $\cftalg{A}$ is a substitution algebra in context $\Gamma$. Then
the pre-substitution algebra $\cftfamalg{\cftalg{A}}$ is a substitution algebra
in context 
$\ctxext{\Gamma}{\cftalgc{\cftalg{A}}}$.
\end{thm}

\begin{thm}
Suppose $\cftalg{Q}$ is a substitution algebra in context $\ctxext{\Gamma}{B}$ 
and let $\jfam{\Gamma}{A}$. Then the pre-substitution algebra 
$\ctxwk{A}{\cftalg{Q}}$ is a substitution algebra in context 
$\ctxext{{\Gamma}{A}}{\ctxwk{A}{B}}$.
\end{thm}

\begin{thm}
Suppose $\cftalg{Q}$ is a substitution algebra in context $\ctxext{{\Gamma}{A}}{P}$
and let $\jterm{\Gamma}{A}{x}$. Then the pre-substitution algebra 
$\subst{x}{\cftalg{Q}}$ is a substitution algebra in context 
$\ctxext{\Gamma}{\subst{x}{P}}$.
\end{thm}

\begin{cor}
In a substitution algebra $\cftalg{A}$ in context $\Gamma$, substitution is a
substitution homomorphism.
\end{cor}

%%%%%%%%%%%%%%%%%%%%%%%%%%%%%%%%%%%%%%%%%%%%%%%%%%%%%%%%%%%%%%%%%%%%%%%%%%%%%%%%
\subsection{E-algebras}
\begin{defn}
An extension algebra $\cftalg{A}$ in context $\Gamma$ with internal empty context,
empty family, weakening, identity terms and
substitution
\begin{align*}
\jalign\jhom
  {{{\Gamma}{\cftalgc{\cftalg{A}}}}{\cftalgf{\cftalg{A}}}}
  {\ctxwk{\cftalgf{\cftalg{A}}}{\cftfamalg{\cftalg{A}}}}
  {\cftfamalg{\cftfamalg{\cftalg{A}}}}
  {\cftwk{\cftalg{A}}}
  \\
\jalign\jterm
  { { {\Gamma}
      {\cftalgc{\cftalg{A}}}
      }
    { \cftalgf{\cftalg{A}}
      }
    }
  {\subst{{\idtm{\cftalgf{\cftalg{A}}}}{\cftwkc{\cftalg{A}}}}{\cftalgt{\cftfamalg{\cftalg{A}}}}}
  {\cftidtm{\cftalg{A}}}
  \\
\jalign\jhom
  {{{{\Gamma}{\cftalgc{\cftalg{A}}}}{\cftalgf{\cftalg{A}}}}{\cftalgt{\cftalg{A}}}}
  {\ctxwk{\cftalgt{\cftalg{A}}}{\cftalg{F_{F_A}}}}
  {\ctxwk{\cftalgt{\cftalg{A}}}{{\cftalgf{\cftalg{A}}}{\cftfamalg{\cftalg{A}}}}}
  {\cftsubst{\cftalg{A}}}
\end{align*}
is said to be an \emph{E-algebra in context $\Gamma$} if both weakening and substitution are both
projection and substitution homomorphisms and if (1) the diagram
\begin{equation*}
\begin{tikzcd}[column sep=large]
\ctxwk
  { \cftalgt{\cftalg{A}}}
  { { \cftalgf{\cftalg{A}}}
    { \cftfamalg{\cftalg{A}}}
    }
  \ar{r}{\ctxwk{\cftalgt{\cftalg{A}}}{\cftwk{\cftalg{A}}}}
  \ar{dr}[swap]{\cftidhom{}}
  &
\ctxwk
  { \cftalgt{\cftalg{A}}}
  { \cftfamalg{\cftfamalg{\cftalg{A}}}}
  \ar{d}{\cftsubst{\cftalg{A}}}
  \\
  &
\ctxwk
  { \cftalgt{\cftalg{A}}}
  { { \cftalgf{\cftalg{A}}}
    { \cftfamalg{\cftalg{A}}}
    }
\end{tikzcd}
\end{equation*}
of extension algebras in context 
$\ctxext{{{\Gamma}{\cftalgc{\cftalg{A}}}}{\cftalgf{\cftalg{A}}}}{\cftalgt{\cftalg{A}}}$
commutes judgmentally
and (2) the diagram
\begin{equation*}
\begin{tikzcd}[column sep=large]
\cftfamalg{\cftfamalg{\cftalg{A}}}
  \ar{r}{\subst{\idtm{\cftalgf{\cftalg{A}}}}{\cftfamhom{\cftwk{\cftalg{A}}}}}
  \ar{dr}[swap]{\cftidhom{}}
  &
\subst
  { \idtm{\cftalgf{\cftalg{A}}}}
  { \jcomp{}
      { \cftwk{\cftalg{A}}}
      { \cftfamalg{\cftfamalg{\cftfamalg{\cftalg{A}}}}}
    }
  \ar{d}
    { \subst
        { \cftidtm{\cftalg{A}}}
        { { { \idtm{\cftalgf{\cftalg{A}}}}
            { \cftwkc{\cftalg{A}}}
            }
          { \cftsubst{\cftalgf{\cftalg{A}}}}
          }
      }
  \\
  &
    { \subst
        { \cftidtm{\cftalg{A}}}
        { { { \idtm{\cftalgf{\cftalg{A}}}}
            { \cftwkc{\cftalg{A}}}
            }
          { \ctxwk
              { \cftalgt{\cftfamalg{\cftalg{A}}}}
              { { \cftalgf{\cftfamalg{\cftalg{A}}}}
                { \cftfamalg{\cftfamalg{\cftalg{A}}}}
                }
            }
          }
      }
\end{tikzcd}
\end{equation*}
of extension algebras in context 
$\ctxext{{\Gamma}{\cftalgc{\cftalg{A}}}}{\cftalgf{\cftalg{A}}}$ commutes
judgmentally.

E-algebras are considered judgmentally equal if their underlying extension algebras
and their internal weakening, identity terms and substitution are judgmentally
equal.
\end{defn}

\begin{thm}
Suppose $\cftalg{A}$ is an E-algebra in context $\Gamma$. Then
$\cftfamalg{\cftalg{A}}$ is an E-algebra in context $\ctxext{\Gamma}{\cftalgc{\cftalg{A}}}$.
\end{thm}

\begin{thm}
Suppose $\cftalg{Q}$ is an E-algebra in context $\ctxext{\Gamma}{B}$ and that
$\jfam{\Gamma}{A}$. Then $\ctxwk{A}{\cftalg{Q}}$ is an E-algebra in context
$\ctxext{{\Gamma}{A}}{\ctxwk{A}{B}}$. 
\end{thm}

\begin{thm}
Suppose $\cftalg{Q}$ is an E-algebra in context $\ctxext{{\Gamma}{A}}{P}$ and
that $\jterm{\Gamma}{A}{x}$. Then $\subst{x}{\cftalg{Q}}$ is an E-algebra in
context $\ctxext{\Gamma}{\subst{x}{P}}$.
\end{thm}

\begin{thm}
Suppose $\cftalg{Q}$ is an E-algebra in context $\ctxext{\Gamma}{B}$ and that
$\jfam{\Gamma}{A}$. Then we have the
judgmental equality $\cftfamalg{\ctxwk{A}{\cftalg{Q}}}\jdeq\ctxwk{A}{\cftfamalg{\cftalg{Q}}}$
of E-algebras in context $\ctxext{{{\Gamma}{A}}{\ctxwk{A}{B}}}{\ctxwk{A}{\cftalgc{\cftalg{Q}}}}$.
\end{thm}

\begin{thm}
Suppose $\cftalg{Q}$ is an E-algebra in context $\ctxext{{\Gamma}{A}}{P}$ and
that $\jterm{\Gamma}{A}{x}$. Then we have the judgmental equality
$\cftfamalg{\subst{x}{\cftalg{Q}}}\jdeq\subst{x}{\cftfamalg{\cftalg{Q}}}$
of E-algebras in context $\ctxext{{\Gamma}{\subst{x}{P}}}{\subst{x}{\cftalgc{\cftalg{Q}}}}$.
\end{thm}

\subsection{E-homomorphisms}
\begin{defn}
An \emph{E-homomorphism from $\cftalg{A}$ to $\cftalg{B}$ in context $\Gamma$}
is a extension homomorphism which is both a projection homomorphism and a
substitution homomorphism.
\end{defn}


%\part{Type constructors}

%\section{The dependent function constructor}
As we have done with most everything so far, we will take the point of view that
the dependent function constructor is an operation with an action on contexts,
on families and on terms. When $A$ is a family of contexts over $\Gamma$,
$\mprd{A}{\blank}$ takes things in context $\ctxext{\Gamma}{A}$ to the context
$\Gamma$. It's action on terms restricted to the empty context is usually 
denoted by $\lambda$, so we shall denote the entire action by $\lambda$. As
usual, $\lambda$ is reversed by evaluation. 

The dependent function type constructor not only acts on families $P$ of
contexts over $\ctxext{\Gamma}{A}$, but it also sends families $Q$ of contexts
over $\ctxext{{\Gamma}{A}}{P}$ to families $\mprd[\famsym]{A}{Q}$ over $\ctxext{\Gamma}
{\mprd{A}{P}}$. Moreover, we will also have a version of $\lambda$-abstraction
for terms of such families $Q$.

\begin{align*}
& \inference
    { \jfam{{\Gamma}{A}}{P}
      }
    { \jfam{\Gamma}{\mprd{A}{P}}
      }
& & \inference
      { \jfameq{\Gamma}{A}{A'}
        \jfameq{{\Gamma}{A}}{P}{P'}
        }
      { \jfameq{\Gamma}{\mprd{A}{P}}{\mprd{A'}{P'}}
        }
  \\
& \inference
    { \jfam{{{\Gamma}{A}}{P}}{Q}
      }
    { \jfam{{\Gamma}{\mprd{A}{P}}}{\mprd[\famsym]{A}{Q}}
      }
& & \inference
      { \jfameq{\Gamma}{A}{A'}
        \jfameq{{{\Gamma}{A}}{P}}{Q}{Q'}
        }
      { \jfameq{{\Gamma}{\mprd{A}{P}}}{\mprd[\famsym]{A}{Q}}{\mprd[\famsym]{A'}{Q'}}
        }
  \\
& \inference
    { \jterm{{{\Gamma}{A}}{P}}{Q}{g}
      }
    { \jterm{{\Gamma}{\mprd{A}{P}}}{\mprd[\famsym]{A}{Q}}{\slam{A}{Q}{g}}
      }
& & \inference
      { \jfameq{\Gamma}{A}{A'}
        \jtermeq{{{\Gamma}{A}}{P}}{Q}{g}{g'}
        }
      { \jtermeq{{\Gamma}{\mprd{A}{P}}}{\mprd[\famsym]{A}{Q}}{\slam{A}{Q}{g}}{\slam{A'}{Q}{g'}}
        }
\end{align*}

\begin{defn}
Let $A$ and $B$ be families of contexts over $\Gamma$. Then we define
\begin{equation*}
\jfamdefn{\Gamma}{\jfun{A}{B}}{\sprd{A}{\ctxwk{A}{B}}}.
\end{equation*}
\end{defn}

\subsection{The compatibility rules for the dependent product constructor}
In this subsection we lay out the compatibility rules which we will require
for the dependent product constructor.

\subsubsection{Dependent products are compatible with the empty context}
The empty context can appear in the domain and in the codomain of the dependent
function type constructor. We have the following inference rules explaining
what happens when the empty context appears in the domain:
\begin{align}
& \inference
    { \jfam{\Gamma}{P}
      }
    { \jfameq{\Gamma}{\mprd{\emptyf}{P}}{P}
      }
  \\
& \inference
    { \jfam{{\Gamma}{P}}{Q}
      }
    { \jfameq{{\Gamma}{P}}{\mprd[\famsym]{\emptyf}{Q}}{Q}
      }
  \\
& \inference
    { \jterm{{\Gamma}{P}}{Q}{g}
      }
    { \jtermeq{{\Gamma}{P}}{Q}{\slam{\emptyf}{Q}{g}}{g}
      }
\end{align}
We have the following infernece rules explaining what happens when the empty
context appears in the codomain:
\begin{align}
& \inference
    { \jfam{\Gamma}{A}
      }
    { \jfameq{\Gamma}{\mprd{A}{\emptyf}}{\emptyf}
      }
  \\
& \inference
    { \jfam{{\Gamma}{A}}{P}
      }
    { \jfameq{{\Gamma}{\mprd{A}{P}}}{\mprd[\famsym]{A}{\emptyf}}{\emptyf}
      }
\end{align}
Now we see that we can compare $\mprd{A}{P}$ with $\mprd[\famsym]{A}{P}$ by
seeing $P$ as a family over $\ctxext{{\Gamma}{A}}{\emptyf}$. We will require
the following inference rule to be valid:
\begin{equation}
\inference
  { \jfam{{\Gamma}{A}}{P}
    }
  { \jfameq{\Gamma}{\mprd{A}{P}}{\mprd[\famsym]{A}{P}}
    }
\end{equation}
This rule will allow us to omit the annotation indicating the action on families,
which we will do from now on. We also note that because the empty family over
$\ctxext{\Gamma}{A}$ is mapped to the empty family over $\Gamma$ by the
dependent function constructor, we obtain the important special cases of
lambda abstraction and evaluation that
\begin{align*}
& \inference
  { \jterm{{\Gamma}{A}}{P}{f}
    }
  { \jterm{\Gamma}{\mprd{A}{P}}{\slam{A}{P}{f}}
    }
\end{align*}
Thus, we retrieve what is originally meant by lambda abstraction.

\subsubsection{Dependent products are compatible with extension}
The following rules describe what happens when a dependent function constructor
is taken over an extension

\begin{align*}
& \inference
  { \jfam{{\Gamma}{A}}{P}
    \jfam{{{\Gamma}{A}}{P}}{Q}
    }
  { \jfameq
      {\Gamma}
      {\mprd{\ctxext{A}{P}}{Q}}
      {\mprd{A}{\mprd{P}{Q}}}
    }
  \\
& \inference
  { \jfam{{\Gamma}{A}}{P}
    \jfam{{{{\Gamma}{A}}{P}}{Q}}{R}
    }
  { \jfameq
      {{\Gamma}{\mprd{\ctxext{A}{P}}{Q}}}
      {\mprd{\ctxext{A}{P}}{R}}
      {\mprd{A}{\mprd{P}{R}}}
    }
  \\
& \inference
  { \jfam{{\Gamma}{A}}{P}
    \jterm{{{{\Gamma}{A}}{P}}{Q}}{R}{h}
    }
  { \jtermeq
      {{\Gamma}{\mprd{\ctxext{A}{P}}{Q}}}
      {\mprd{\ctxext{A}{P}}{R}}
      {\slam{\ctxext{A}{P}}{R}{h}}
      {\slam{A}{\mprd{P}{R}}{\slam{P}{R}{h}}}
    }
\end{align*}

When the dependent product constructor is applied to an extension, we get the
following:

\begin{align*}
& \inference
  { \jfam{{{\Gamma}{A}}{P}}{Q}
    }
  { \jfameq
      {\Gamma}
      {\mprd{A}{\ctxext{P}{Q}}}
      {\ctxext{\mprd{A}{P}}{\mprd{A}{Q}}}
    }
  \\
& \inference
  { \jfam{{{{\Gamma}{A}}{P}}{Q}}{R}
    }
  { \jfameq
      {{\Gamma}{\mprd{A}{P}}}
      {\mprd{A}{\ctxext{Q}{R}}}
      {\ctxext{\mprd{A}{Q}}{\mprd{A}{R}}}
    }
\end{align*}
We show in \autoref{lem:slam-ext} that the remaining two properties
involving lambda abstraction and evaluation are derivable.

\subsubsection{Weakening is compatible with the dependent product constructor}

\begin{align*}
& \inference
  { \jfam{\Gamma}{A}
    \jfam{{{\Gamma}{B}}{Q}}{R}
    }
  { \jfameq
      {{{\Gamma}{A}}{\ctxwk{A}{B}}}
      {\ctxwk{A}{\mprd{Q}{R}}}
      {\mprd{\ctxwk{A}{Q}}{\ctxwk{A}{R}}}
    }
  \\
& \inference
  { \jfam{\Gamma}{A}
    \jfam{{{{\Gamma}{B}}{Q}}{R}}{S}
    }
  { \jfameq
      {{{{\Gamma}{A}}{\ctxwk{A}{B}}}{\ctxwk{A}{\mprd{Q}{R}}}}
      {\ctxwk{A}{\mprd{Q}{S}}}
      {\mprd{\ctxwk{A}{Q}}{\ctxwk{A}{S}}}
    }
  \\
& \inference
  { \jfam{\Gamma}{A}
    \jterm{{{{\Gamma}{B}}{Q}}{R}}{S}{k}
    }
  { \jtermeq
      {{{{\Gamma}{A}}{\ctxwk{A}{B}}}{\ctxwk{A}{\mprd{Q}{R}}}}
      {\ctxwk{A}{\mprd{Q}{S}}}
      {\ctxwk{A}{\slam{Q}{S}{k}}}
      {\slam{\ctxwk{A}{Q}}{\ctxwk{A}{S}}{\ctxwk{A}{k}}}
    }
\end{align*}

\subsubsection{Dependent products are taken fiberwise}
The rules explaining that substitution is compatible with the dependent product
constructor assert that dependent products are taken fiberwise (as is usual).

\begin{align*}
& \inference
  { \jterm{\Gamma}{A}{x}
    \jfam{{{{\Gamma}{A}}{P}}{Q}}{R}
    }
  { \jfameq
      {{\Gamma}{\subst{x}{P}}}
      {\subst{x}{\mprd{Q}{R}}}
      {\mprd{\subst{x}{Q}}{\subst{x}{R}}}
    }
  \\
& \inference
  { \jterm{\Gamma}{A}{x}
    \jfam{{{{{\Gamma}{A}}{P}}{Q}}{R}}{S}
    }
  { \jfameq
      {{{\Gamma}{\subst{x}{P}}}{\mprd{\subst{x}{Q}}{\subst{x}{R}}}}
      {\subst{x}{\mprd{Q}{S}}}
      {\mprd{\subst{x}{Q}}{\subst{x}{S}}}
    }
  \\
& \inference
  { \jterm{\Gamma}{A}{x}
    \jterm{{{{{\Gamma}{A}}{P}}{Q}}{R}}{S}{k}
    }
  { \jtermeq
      {{{\Gamma}{\subst{x}{P}}}{\mprd{\subst{x}{Q}}{\subst{x}{R}}}}
      {\subst{x}{\mprd{Q}{S}}}
      {\subst{x}{\slam{Q}{S}{k}}}
      {\slam{\subst{x}{Q}}{\subst{x}{S}}{\subst{x}{k}}}
    }
\end{align*}

\subsubsection{The dependent function constructor is compatible with weakening}
\begin{align*}
& \inference
  { \jfam{{{\Gamma}{A}}{P}}{Q}
    \jfam{{{\Gamma}{Q}}{P}}{R}
    }
  { \jfameq
      {{{\Gamma}{\mprd{A}{P}}}{\mprd{A}{Q}}}
      {\mprd{A}{\ctxwk{Q}{R}}}
      {\ctxwk{\mprd{A}{Q}}{\mprd{A}{R}}}
    }
  \\
& \inference
  { \jfam{{{\Gamma}{A}}{P}}{Q}
    \jfam{{{{\Gamma}{P}}{Q}}{R}}{S}
    }
  { \jfameq
      {{{{\Gamma}{\mprd{A}{P}}}{\mprd{A}{Q}}}{\mprd{A}{\ctxwk{Q}{R}}}}
      {\mprd{A}{\ctxwk{Q}{S}}}
      {\ctxwk{\mprd{A}{Q}}{\mprd{A}{S}}}
    }
  \\
& \inference
  { \jfam{{{\Gamma}{A}}{P}}{Q}
    \jterm{{{{\Gamma}{P}}{Q}}{R}}{S}{k}
    }
  { \jtermeq
      {{{{\Gamma}{\mprd{A}{P}}}{\mprd{A}{Q}}}{\mprd{A}{\ctxwk{Q}{R}}}}
      {\mprd{A}{\ctxwk{Q}{S}}}
      {\slam{A}{\ctxwk{Q}{S}}{\ctxwk{Q}{k}}}
      {\ctxwk{\mprd{A}{Q}}{\slam{A}{S}{k}}}
    }
\end{align*}

\subsubsection{The dependent function constructor is compatible with substitution}
\begin{align*}
& \inference
  { \jterm{{{\Gamma}{A}}{P}}{Q}{g}
    \jfam{{{{\Gamma}{A}}{P}}{Q}}{R}
    }
  { \jfameq
      {{\Gamma}{\mprd{A}{P}}}
      {\mprd{A}{\subst{g}{R}}}
      {\subst{\slam{A}{Q}{g}}{\mprd{A}{R}}}
    }
  \\
& \inference
  { \jterm{{{\Gamma}{A}}{P}}{Q}{g}
    \jfam{{{{{\Gamma}{A}}{P}}{Q}}{R}}{S}
    }
  { \jfameq
      {{{\Gamma}{\mprd{A}{P}}}{\mprd{A}{\subst{g}{R}}}}
      {\mprd{A}{\subst{g}{S}}}
      {\subst{\slam{A}{Q}{g}}{\mprd{A}{S}}}
    }
  \\
& \inference
  { \jterm{{{\Gamma}{A}}{P}}{Q}{g}
    \jterm{{{{{\Gamma}{A}}{P}}{Q}}{R}}{S}{k}
    }
  { \jtermeq
      {{{\Gamma}{\mprd{A}{P}}}{\mprd{A}{\subst{g}{R}}}}
      {\mprd{A}{\subst{g}{S}}}
      {\slam{A}{\subst{g}{S}}{\subst{g}{k}}}
      {\subst{\slam{A}{Q}{g}}{\slam{A}{S}{k}}}
    }
  \\
& \inference
  { \jterm{{{\Gamma}{A}}{P}}{Q}{g}
    \jterm{{\Gamma}{\mprd{A}{\ctxext{{P}{Q}}{R}}}}{\mprd{A}{S}}{k}
    }
  { \jtermeq
      {{{{\Gamma}{A}}{P}}{\subst{g}{R}}}
      {\subst{g}{S}}
      {\sev{A}{\subst{g}{S}}{\subst{\slam{A}{Q}{g}}{k}}}
      {\subst{g}{\sev{A}{S}{k}}}
    }
\end{align*}

\subsubsection{The dependent function constructor is compatible with the identity terms}
\begin{align*}
& \inference
  { \jfam{{{\Gamma}{A}}{P}}{Q}
    }
  { \jtermeq
      {{{\Gamma}{\mprd{A}{P}}}{\mprd{A}{Q}}}
      {\ctxwk{\mprd{A}{Q}}{\mprd{A}{Q}}}
      {\slam{A}{\ctxwk{Q}{Q}}{\idtm{Q}}}
      {\idtm{\mprd{A}{Q}}}
    }
  \\
& \inference
  { \jfam{{{\Gamma}{A}}{P}}{Q}
    }
  { \jtermeq
      {{{{\Gamma}{A}}{P}}{Q}}
      {\ctxwk{Q}{Q}}
      {\sev{A}{\ctxwk{Q}{Q}}{\idtm{\mprd{A}{Q}}}}
      {\idtm{Q}}
    }
\end{align*}

\subsection{Consequences of the rules for the dependent function constructors}

\begin{lem}\label{lem:slam-ext}
The inference rules asserting that lambda abstraction is compatible with term
extension
\begin{align*}
& \inference
  { \jterm{{{\Gamma}{A}}{P}}{{Q}{R}}{h}
    }
  { \jtermeq
      {{\Gamma}{\mprd{A}{P}}}
      {\mprd{A}{\ctxext{Q}{R}}}
      {\cprojfst{\mprd{A}{Q}}{\mprd{A}{R}}{\slam{A}{\ctxext{Q}{R}}{h}}}
      {\slam{A}{Q}{\cprojfst{Q}{R}{h}}}
    }
  \\
& \inference
  { \jterm{{{\Gamma}{A}}{P}}{{Q}{R}}{h}
    }
  { \jtermeq
      {{\Gamma}{\mprd{A}{P}}}
      {\mprd{A}{\ctxext{Q}{R}}}
      {\cprojfst{\mprd{A}{Q}}{\mprd{A}{R}}{\slam{A}{\ctxext{Q}{R}}{h}}}
      {\slam{A}{R}{\cprojsnd{Q}{R}{h}}}
    }
\intertext{and the inference rules asserting that evaluation is compatible with
term extension}
& \inference
  { \jterm{{\Gamma}{\mprd{A}{P}}}{\mprd{A}{\ctxext{Q}{R}}}{h}
    }
  { \jtermeq
      {{{\Gamma}{A}}{P}}
      {{Q}{R}}
      {\cprojfst{Q}{R}{\sev{A}{\ctxext{Q}{R}}{h}}}
      {\sev{A}{Q}{\cprojfst{\mprd{A}{Q}}{\mprd{A}{R}}{h}}}
    }
  \\
& \inference
  { \jterm{{\Gamma}{\mprd{A}{P}}}{\mprd{A}{\ctxext{Q}{R}}}{h}
    }
  { \jtermeq
      {{{\Gamma}{A}}{P}}
      {{Q}{R}}
      {\cprojsnd{Q}{R}{\sev{A}{\ctxext{Q}{R}}{h}}}
      {\sev{A}{R}{\cprojsnd{\mprd{A}{Q}}{\mprd{A}{R}}{h}}}
    }
\end{align*}
are valid.
\end{lem}

\begin{proof}
We have the judgmental equalities
\begin{align*}
\cprojfst{\mprd{A}{Q}}{\mprd{A}{R}}{\slam{A}{\ctxext{Q}{R}}{h}}
& \jdeq
  \subst{\slam{A}{\ctxext{Q}{R}}{h}}{\ctxwk{\mprd{A}{R}}{\idtm{\mprd{A}{Q}}}}
  \\
& \jdeq 
  \subst{\slam{A}{\ctxext{Q}{R}}{h}}{\ctxwk{\mprd{A}{R}}{\slam{A}{\ctxwk{Q}{Q}}{\idtm{Q}}}}
  \\
& \jdeq
  \subst{\slam{A}{\ctxext{Q}{R}}{h}}{\slam{A}{\ctxwk{R}{{Q}{Q}}}{\ctxwk{R}{\idtm{Q}}}}
  \\
& \jdeq
  \subst{\slam{A}{\ctxext{Q}{R}}{h}}{\slam{A}{\ctxwk{\ctxext{Q}{R}}{Q}}{\ctxwk{R}{\idtm{Q}}}}
  \\
& \jdeq
  \slam{A}{Q}{\subst{h}{\ctxwk{R}{\idtm{Q}}}}
  \\
& \jdeq
  \slam{A}{Q}{\cprojfst{Q}{R}{h}}
\end{align*}
\end{proof}

\begin{rmk}
It follows from these rules that on the projections (i.e.~on the fibrations),
lambda abstraction and evaluation are each other's inverses even though we
have not even postulated that evaluation is a retraction of lambda 
abstraction. Thus, in combination with univalence this should give that
they are homotopically inverse to each other.
\end{rmk}

\subsection{The evaluation term}
Until now we have just introduced the dependent function constructor, but we 
have not treated one of its main characteristic, the evaluation map. Usually
in type theory, evaluation is introduces by a rule like
\begin{equation*}
\inference
    { \jterm{\Gamma}{\mprd{A}{P}}{f}
      }
    { \jterm{{\Gamma}{A}}{P}{\mathsf{ev}(f)}
      }
\end{equation*}
or even a rule like
\begin{equation*}
\inference
    { \jterm{\Gamma}{\mprd{A}{P}}{f}
      \jterm{\Gamma}{A}{x}
      }
    { \jterm{\Gamma}{\subst{x}{P}}{\mathsf{ev}(f,x)}
      }
\end{equation*}
involving a term of $A$. We immediately discard the latter as a viable option.
In the current setting however, there is an even better option than the first. 
We need a better option because given a family 
$\jfam{{{\Gamma}{A}}{P}}{Q}$, we will need to have a family over
$\ctxext{{\Gamma}{\mprd{A}{P}}}{\ctxwk{\mprd{A}{P}}{A}}$ with fibers
$\subst{{x}{\mathsf{ev}(f)}}{{x}{Q}}$ at $f:\mprd{A}{P}$ and $x:A$. In a 
name-free setting, such a family is impossible to obtain when using evaluation
as introduced above. The solution is to introduce evaluation via the rule
\begin{equation}\label{eq:evtm}
\inference
  { \jfam{{\Gamma}{A}}{P}
    }
  { \jterm
      {{{\Gamma}{\mprd{A}{P}}}{\ctxwk{\mprd{A}{P}}{A}}}
      {\ctxwk{\mprd{A}{P}}{P}}
      {\evtm{A}{P}}
    }
\end{equation}
The advantages are as follows:
\begin{enumerate}
\item It matches the categorical notion of locally cartesian closedness, where
the unit and the counit consist of morphisms in the category.
\item We now have the family 
\begin{equation*}
\jfam
  {{{\Gamma}{\mprd{A}{P}}}{\ctxwk{\mprd{A}{P}}{A}}}
  {\subst{\evtm{A}{P}}{\ctxwk{\mprd{A}{P}}{Q}}}
\end{equation*}
Using this family, we hope to find the judgmental equality
\begin{equation*}
\jfameq
  {{\Gamma}{\mprd{A}{P}}}
  {\mprd{A}{Q}}
  {\mprd{\ctxwk{\mprd{A}{P}}{A}}{\subst{\evtm{A}{P}}{\ctxwk{\mprd{A}{P}}{Q}}}}
\end{equation*}
expressing the action on families of the dependent function constructor in terms
of its action on contexts. This is in agreement with the usual approach of
\cite{TheBook}, where no action on families of the dependent function type
constructor is needed. We will also be able to use the family
\begin{equation*}
\jfam
  {{\Gamma}{\mprd{A}{P}}}
  {{\ctxwk{\mprd{A}{P}}{A}}{\subst{\evtm{A}{P}}{\ctxwk{\mprd{A}{P}}{Q}}}}
\end{equation*}
to formulate the compatibility of the dependent function constructor with itself.
\end{enumerate}
Of course, we will also need the following convertibility rule before we 
continue:

\begin{rmk}
It is common practice, for example in \cite{TheBook}, to omit explicit notation
for evaluation. In our setting that would mean that we would denote a term 
$\subst{f}{\evtm{A}{P}}$ of $P$ in context $\ctxext{\Gamma}{A}$ by $f$.
However, since the evaluation term is set to be an actual term of the theory,
it seems to be harmful to the exposition of the theory to copy this practice.
\end{rmk}

\begin{comment}
\subsubsection{Introducing the evaluation operator}
The operation $\ctxev{A}{P}{\blank}$ brings things in context $\ctxext{{\Gamma}{A}}{P}$ to
the context $\ctxext{{\Gamma}{\mprd{A}{P}}}{\ctxwk{\mprd{A}{P}}{A}}$.
\begin{align*}
& \inference
  { \jfam{{{\Gamma}{A}}{P}}{Q}
    }
  { \jfamdefn
      {{{\Gamma}{\mprd{A}{P}}}{\ctxwk{\mprd{A}{P}}{A}}}
      {\ctxev{A}{P}{Q}}
      {\subst{\evtm{A}{P}}{\ctxwk{\mprd{A}{P}}{Q}}}
    }
  \\
& \inference
  { \jfam{{{{\Gamma}{A}}{P}}{Q}}{R}
    }
  { \jfamdefn
      {{{{\Gamma}{\mprd{A}{P}}}{\ctxwk{\mprd{A}{P}}{A}}}{\ctxev{A}{P}{Q}}}
      {\ctxev[\famsym]{A}{P}{R}}
      {\subst{\evtm{A}{P}}{\ctxwk{\mprd{A}{P}}{R}}}
    }
  \\
& \inference
  { \jterm{{{{\Gamma}{A}}{P}}{Q}}{R}{h}
    }
  { \jtermdefn
      {{{{\Gamma}{\mprd{A}{P}}}{\ctxwk{\mprd{A}{P}}{A}}}{\ctxev{A}{P}{Q}}}
      {\ctxev[\famsym]{A}{P}{R}}
      {\ctxev[\tmsym]{A}{P}{h}}
      {\subst{\evtm{A}{P}}{\ctxwk{\mprd{A}{P}}{h}}}
    }
\end{align*}
\end{comment}

\subsubsection{The compatibility rules of the evaluation terms}

\paragraph{Compatibility with the empty context}
The following rule explains that the evaluation term is judgmentally equal to
the identity term when the domain of the dependent function constructor is the
empty family over $\Gamma$.
\begin{align*}
& \inference
    { \jfam{\Gamma}{P}
      }
    { \jtermeq
        {{\Gamma}{P}}
        {\ctxwk{P}{P}}
        {\evtm{\emptyf}{P}}
        {\idtm{P}}
      }
\end{align*}

\paragraph{Currying for the evaluation term}
We have asserted that there is a judgmental equality
\begin{equation*}
\jfameq
      {\Gamma}
      {\mprd{\ctxext{A}{P}}{Q}}
      {\mprd{A}{\mprd{P}{Q}}}
\end{equation*}
for every family $Q$ over $\ctxext{{\Gamma}{A}}{P}$. It follows that we have
the judgmental equalities
\begin{align*}
& \ctxwk{\mprd{A}{\mprd{P}{Q}}}{Q}
  \\
& \jdeq
  \subst
    { \evtm{A}{\mprd{P}{Q}}}
    { \ctxwk
        {{\mprd{A}{\mprd{P}{Q}}}{\mprd{P}{Q}}}
        {{\mprd{A}{\mprd{P}{Q}}}{Q}}
      }
  \\
& \jdeq
  \subst
    { \evtm{A}{\mprd{P}{Q}}}
    { \ctxwk
        {\mprd{A}{\mprd{P}{Q}}}
        {{\mprd{P}{Q}}{Q}}
      }
\end{align*}
Since $\evtm{P}{Q}$ is a term of $\ctxwk{\mprd{P}{Q}}{Q}$ in 
context $\ctxext{{{\Gamma}{A}}{\mprd{P}{Q}}}{\ctxwk{\mprd{P}{Q}}{P}}$ we may
require the following compatibility rule:
\begin{align}
& \inference
  { \jfam{{{\Gamma}{A}}{P}}{Q}
    }
  { \begin{array}{l}
    \ctxext
      {{\Gamma}{\mprd{\ctxext{A}{P}}{Q}}}{\ctxwk{\mprd{\ctxext{A}{P}}{Q}}{\ctxext{A}{P}}}
      \\
    \jtermeq
      {\qquad}
      {\ctxwk{\mprd{\ctxext{A}{P}}{Q}}{Q}}
      {\evtm{\ctxext{A}{P}}{Q}}
      {\subst{\evtm{A}{\mprd{P}{Q}}}{\ctxwk{\mprd{A}{\mprd{P}{Q}}}{\evtm{P}{Q}}}}
    \end{array}
    }
\end{align}
To get a feel for what it says we substitute terms in both expressions involved
in the following lemma.

\begin{lem}
Consider $f:\mprd{\ctxext{A}{P}}{Q}$, $x:A$ and $u:\subst{x}{P}$ be terms in
context $\Gamma$. Then we have the judgmental equality
\begin{equation*}
\jtermeq
  {\Gamma}
  {\subst{u}{{x}{Q}}}
  {\subst{\tmext{x}{u}}{{f}{\evtm{\ctxext{A}{P}}{Q}}}}
  {\subst{u}{{{x}{{f}{\evtm{A}{\mprd{P}{Q}}}}}{\evtm{\subst{x}{P}}{\subst{x}{Q}}}}}
\end{equation*}
is valid.
\end{lem}

\paragraph{Weakening is compatible with the evaluation term}
\begin{align*}
& \inference
  { \jfam{\Gamma}{A}
    \jfam{{{\Gamma}{B}}{Q}}{R}
    }
  { \begin{array}{l}
    \ctxext
        { { { {\Gamma}
              {A}
              }
            { \ctxwk{A}{B}
              }
            }
          { \mprd{\ctxwk{A}{Q}}{\ctxwk{A}{R}}
            }
          }
        { \ctxwk{\mprd{\ctxwk{A}{Q}}{\ctxwk{A}{R}}}{\ctxwk{A}{Q}}
          }
      \\
    \jtermeq
      {\qquad}
      {\ctxwk{\mprd{\ctxwk{A}{Q}}{\ctxwk{A}{R}}}{\ctxwk{A}{R}}}
      {\evtm{\ctxwk{A}{Q}}{\ctxwk{A}{R}}}
      {\ctxwk{A}{\evtm{Q}{R}}}  
    \end{array}
    }
\end{align*}

\paragraph{Substitution is compatible with the evaluation term}
\begin{align*}
& \inference
  { \jterm{\Gamma}{A}{x}
    \jfam{{{{\Gamma}{A}}{P}}{Q}}{R}
    }
  { \begin{array}{l}
    \ctxext
      {{{\Gamma}{\subst{x}{P}}}{\mprd{\subst{x}{Q}}{\subst{x}{R}}}}
      {\ctxwk{\mprd{\subst{x}{Q}}{\subst{x}{R}}}{\subst{x}{Q}}}
      \\
    \jtermeq
      {\qquad}
      {\ctxwk{\mprd{\subst{x}{Q}}{\subst{x}{R}}}{\subst{x}{R}}}
      {\subst{x}{\evtm{Q}{R}}}
      {\evtm{\subst{x}{Q}}{\subst{x}{R}}}
    \end{array}
    }  
\end{align*}

\paragraph{The evaluation term is compatible with weakening}
\begin{align*}
& \inference
  { \jfam{{{\Gamma}{A}}{P}}{Q}
    \jfam{{{{\Gamma}{P}}{Q}}{R}}{S}
    \jterm{{{\Gamma}{\mprd{A}{P}}}{\mprd{A}{R}}}{\mprd{A}{S}}{k}
    }
  { \jtermeq
      {{{{{\Gamma}{A}}{P}}{Q}}{\ctxwk{Q}{R}}}
      {\ctxwk{Q}{S}}
      {\sev{A}{\ctxwk{Q}{S}}{\ctxwk{\mprd{A}{Q}}{k}}}
      {\ctxwk{Q}{\sev{A}{S}{k}}}
    }
\end{align*}

\subsection{Composition of dependent functions}

\subsection{\texorpdfstring{$\Pi$}{Π}E-algebras}

\begin{defn}
An E-algebra $\cftalg{A}$ is said to be a $\Pi$E-algebra if there is 
an $E$-homomorphism
\begin{equation*}
\check\Pi:\cftfamalg{\cftfamalg{\cftalg{A}}}\to\ctxwk{\cftalgf{\cftalg{A}}}{\cftfamalg{\cftalg{A}}}
\end{equation*} 
in context $\ctxext{{\Gamma}{\cftalgc{\cftalg{A}}}}{\cftalgf{\cftalg{A}}}$, and a term
\begin{equation*}
\check{\mathsf{ev}}:
\end{equation*}
\end{defn}


%\section{Universes}\label{sec:universes}

\subsection{Generating E-algebras}

\begin{defn}
Consider a formal triple $A,P,T$ in context $\Gamma$ consisting of
$\jfam{\Gamma}{A}$, $\jfam{{\Gamma}{A}}{P}$ and $\jfam{{{\Gamma}{A}}{P}}{T}$.
Then we form the \emph{E-algebra $\genealg{A}{P}{T}$
in context $\Gamma$ generated by $A,T,P$} to be an E-algebra in context
$\Gamma$. We introduce 
\begin{equation*}
\begin{tikzcd}
T 
  \ar{r}{\cfthomt{\genealgincl{A}{P}{T}}}
  \ar[fib]{d}
& \genealgt{A}{P}{T}
  \ar[fib]{d}
  \\
P 
  \ar{r}{\cfthomf{\genealgincl{A}{P}{T}}}
  \ar[fib]{d}
& \genealgf{A}{P}{T}
  \ar[fib]{d}
  \\
A 
  \ar{r}{\cfthomc{\genealgincl{A}{P}{T}}}
& \genealgc{A}{P}{T}
\end{tikzcd}
\end{equation*}
\end{defn}

\subsection{The universe operator}\label{universes}

\begin{defn}
Let $\jfam{\Gamma}{A}$ and $\jfam{{\Gamma}{A}}{P}$. Then we define $\mathbf{U}(A,P)$ to consist of $\jfam{\Gamma}{\UU(A,P)}$ and $\jfam{{\Gamma}{\UU(A,P)}}{\tilde\UU(A,P)}$ for which there is an inclusion
\begin{equation*}
\begin{tikzcd}
P 
  \ar[fib]{d}
  \ar{r}
& \tilde\UU(A,P)
  \ar[fib]{d}
  \\
A \ar{r}
& \UU(A,P)
\end{tikzcd}
\end{equation*}
such that
\begin{equation*}
\begin{tikzcd}
\mprd{\tilde\UU(A,P)}{\ctxwk{\tilde\UU(A,P)}{{\UU(A,P)}{\tilde\UU(A,P)}}}
  \ar[fib]{d}
  \\
\mprd{\tilde\UU(A,P)}{\ctxwk{\tilde\UU(A,P)}{{\UU(A,P)}{\UU(A,P)}}}
  \ar[fib]{d}
  \\
\UU(A,P)
\end{tikzcd}
\end{equation*}
is an E-algebra
\end{defn}


%\section{Inductive constructions}

\subsection{Strong inductive morphisms}
In this subsection we investigate the notion of inductive morphisms. An 
inductive morphism $f$ from $A$ to $B$ in context $\Gamma$ is a morphism which
induces an operation which is judgmentally the inverse of composition
with $f$. We explore a notion of inductive morphism which is much stronger
than the usual notion: it pushes families over $A$ to families over $B$,
families over families over $A$ to families over families over $B$ and terms
thereof to terms of the output families over families in a manner compatible
with the empty family, extension, weakening, substitution and identity terms.
As a result, inductive morphisms will be stable under extension, weakening,
substitution and the identity term is an inductive morphism.

Inductive morphisms are useful to for inductive types which are defined using
only one (ordinary morphism as) constructor, like the unit type and dependent 
pair types.
They can't be used to define the empty type, disjoint sums,
the natural numbers or identity types. 

{\color{red} Maybe
we can solve this by writing down a type theory of inductive constructions.}

\begin{defn}
Let $\jhom{\gamma}{A}{B}{f}$ be a context morphism. We say that $f$ is an
\emph{inductive morphism} if there is an operation $\tfindf{f}$ given by
\begin{align*}
& \inference
  { \jfam{{\Gamma}{A}}{P}
    }
  { \jfam{{\Gamma}{B}}{\tfind{f}{P}}
    }
  \\
& \inference
  { \jfam{{{\Gamma}{A}}{P}}{Q}
    }
  { \jfam{{{\Gamma}{B}}{\tfind{f}{P}}}{\tfind[\famsym]{f}{Q}}
    }
  \\
& \inference
  { \jterm{{{\Gamma}{A}}{P}}{Q}{g}
    }
  { \jterm{{{\Gamma}{B}}{\tfind{f}{P}}}{\tfind[\famsym]{f}{Q}}{\tfind[\tmsym]{f}{g}}
    }
\end{align*}
for which the inference rules in the following list are valid:
\begin{enumerate}
\item The operation $\tfindf{f}$ is compatible with the empty context:
\begin{align*}
& \inference
  { 
    }
  { \jfameq{{\Gamma}{B}}{\tfind{f}{\emptyf}}{\emptyf}
    }
  \\
& \inference
  { \jfam{{\Gamma}{A}}{P}
    }
  { \jfameq{{{\Gamma}{B}}{\tfind{f}{P}}}{\tfind[\famsym]{f}{\emptyf}}{\emptyf}
    }
\end{align*}
\item The action on families $\tfindf[\famsym]{f}$ of $\tfindf{f}$ is compatible
with the action on contexts:
\begin{equation*}
\inference
  { \jfam{{\Gamma}{A}}{P}
    }
  { \jfameq{{\Gamma}{B}}{\tfind[\famsym]{f}{P}}{\tfind{f}{P}}
    }
\end{equation*}
Because of this inference rule we shall henceforth omit the annotations
$\famsym$ and $\tmsym$ from the operation $\tfindf{f}$ as usual.
\item The operation $\tfindf{f}$ is compatible with extension:
\begin{align*}
& \inference
  { \jfam{{{\Gamma}{A}}{P}}{Q}
    }
  { \jfameq
      {{\Gamma}{B}}
      {\tfind{f}{\ctxext{P}{Q}}}
      {\ctxext{\tfind{f}{P}}{\tfind{f}{Q}}}
    }
  \\
& \inference
  { \jfam{{{{\Gamma}{A}}{P}}{Q}}{R}
    }
  { \jfameq
      {{{\Gamma}{B}}{\tfind{f}{P}}}
      {\tfind{f}{\ctxext{Q}{R}}}
      {\ctxext{\tfind{f}{Q}}{\tfind{f}{R}}}
    }
\end{align*}
\item The operation $\tfindf{f}$ is compatible with weakening:
\begin{align*}
& \inference
  { \jfam{{\Gamma}{A}}{P}
    \jfam{{\Gamma}{A}}{Q}
    }
  { \jfameq
      {{{\Gamma}{B}}{\tfind{f}{P}}}
      {\tfind{f}{\ctxwk{P}{Q}}}
      {\ctxwk{\tfind{f}{P}}{\tfind{f}{Q}}}
    }
  \\
& \inference
  { \jfam{{\Gamma}{A}}{P}
    \jfam{{{\Gamma}{A}}{Q}}{R}
    }
  { \jfameq
      {{{{\Gamma}{B}}{\tfind{f}{P}}}{\ctxwk{\tfind{f}{P}}{\tfind{f}{Q}}}}
      {\tfind{f}{\ctxwk{P}{R}}}
      {\ctxwk{\tfind{f}{P}}{\tfind{f}{R}}}
    }
  \\
& \inference
  { \jfam{{\Gamma}{A}}{P}
    \jterm{{{\Gamma}{A}}{Q}}{R}{h}
    }
  { \jtermeq
      {{{{\Gamma}{B}}{\tfind{f}{P}}}{\ctxwk{\tfind{f}{P}}{\tfind{f}{Q}}}}
      {\ctxwk{\tfind{f}{P}}{\tfind{f}{R}}}
      {\tfind{f}{\ctxwk{P}{h}}}
      {\ctxwk{\tfind{f}{P}}{\tfind{f}{h}}}
    }
\end{align*}
\item We will also require that the operation $\tfindf{f}$ is compatible with
weakening by $A$:
\begin{align*}
& \inference
  { \jfam{\Gamma}{X}
    }
  { \jfameq
      {{\Gamma}{B}}
      {\tfind{f}{\ctxwk{A}{X}}}
      {\ctxwk{B}{X}}
    }
  \\
& \inference
  { \jfam{{\Gamma}{X}}{Y}
    }
  { \jfameq
      {{{\Gamma}{B}}{\ctxwk{B}{X}}}
      {\tfind{f}{\ctxwk{A}{Y}}}
      {\ctxwk{B}{Y}}
    }
  \\
& \inference
  { \jterm{{\Gamma}{X}}{Y}{y}
    }
  { \jtermeq
      {{{\Gamma}{B}}{\ctxwk{B}{X}}}
      {\ctxwk{B}{Y}}
      {\tfind{f}{\ctxwk{A}{y}}}
      {\ctxwk{B}{y}}
    }
\end{align*}
These rules assert that constant families and terms are mapped to constant
families and terms.
\item The operation $\tfindf{f}$ is compatible with substitution:
\begin{align*}
& \inference
  { \jterm{{{\Gamma}{A}}{P}}{Q}{g}
    \jfam{{{{\Gamma}{A}}{P}}{Q}}{R}
    }
  { \jfameq
      {{{\Gamma}{B}}{\tfind{f}{P}}}
      {\tfind{f}{\subst{g}{R}}}
      {\subst{\tfind{f}{g}}{\tfind{f}{R}}}
    }
  \\
& \inference
  { \jterm{{{\Gamma}{A}}{P}}{Q}{g}
    \jfam{{{{{\Gamma}{A}}{P}}{Q}}{R}}{S}
    }
  { \jfameq
      {{{{\Gamma}{B}}{\tfind{f}{P}}}{\subst{\tfind{f}{g}}{\tfind{f}{R}}}}
      {\tfind{f}{\subst{g}{S}}}
      {\subst{\tfind{f}{g}}{\tfind{f}{S}}}
    }
  \\
& \inference
  { \jterm{{{\Gamma}{A}}{P}}{Q}{g}
    \jterm{{{{{\Gamma}{A}}{P}}{Q}}{R}}{S}{k}
    }
  { \jtermeq
      {{{{\Gamma}{B}}{\tfind{f}{P}}}{\subst{\tfind{f}{g}}{\tfind{f}{R}}}}
      {\subst{\tfind{f}{g}}{\tfind{f}{S}}}
      {\tfind{f}{\subst{g}{k}}}
      {\subst{\tfind{f}{g}}{\tfind{f}{k}}}
    }
\end{align*}
\item The operation $\tfindf{f}$ is compatible with the identity terms:
\begin{equation*}
\inference
  { \jfam{{{\Gamma}{A}}{P}}{Q}
    }
  { \jtermeq
      {{{{\Gamma}{B}}{\tfind{f}{P}}}{\tfind{f}{Q}}}
      {\ctxwk{\tfind{f}{Q}}{\tfind{f}{Q}}}
      {\tfind{f}{\idtm{Q}}}
      {\idtm{\tfind{f}{Q}}}
    }
\end{equation*}
\item We will also require that $\tfindf{f}$ is compatible with $f$ itself:
\begin{equation*}
\inference
  {
    }
  { \jtermeq
      {{\Gamma}{B}}
      {\ctxwk{B}{B}}
      {\tfind{f}{f}}
      {\idtm{B}}
    }
\end{equation*}
\item Finally, we require that $\tfindf{f}$ is the right inverse of composition
with $f$:
\begin{align*}
& \inference
  { \jfam{{\Gamma}{A}}{P}
    }
  { \jfameq
      {{\Gamma}{A}}
      {\jcomp{}{f}{\tfind{f}{P}}}
      {P}
    }
  \\
& \inference
  { \jfam{{{\Gamma}{A}}{P}}{Q}
    }
  { \jfameq
      {{{\Gamma}{A}}{P}}
      {\jcomp{}{f}{\tfind{f}{Q}}}
      {Q}
    }
  \\
& \inference
  { \jterm{{{\Gamma}{A}}{P}}{Q}{g}
    }
  { \jtermeq
      {{{\Gamma}{A}}{P}}
      {Q}
      {\jcomp{}{f}{\tfind{f}{g}}}
      {g}
    }
\end{align*}
\end{enumerate}
\end{defn}

\begin{rmk}
The rules expressing that $\tfindf{f}$ is a right inverse to composition with
$f$ are usually called the `computation rules' of the induction principle.

Recall that we had announced that $\tfindf{f}$ would be an actual inverse of
composition with $f$, while we only have stated explicitly that $\tfindf{f}$
is a right inverse. We get the fact that it is also a left inverse from the
other compatibility rules. For example: given $\jfam{{\Gamma}{B}}{Q}$ we get
\begin{equation*}
\tfind{f}{\jcomp{}{f}{Q}}
  \jdeq
  \tfind{f}{\subst{f}{\ctxwk{A}{Q}}}
  \jdeq
  \subst{\tfind{f}{f}}{\tfind{f}{\ctxwk{A}{Q}}}
  \jdeq
  \subst{\idtm{B}}{\ctxwk{B}{Q}}
  \jdeq
  Q.
\end{equation*}
We thus recover the usual sort of induction principle. When $Q$ is a family
over $\ctxext{\Gamma}{B}$, all we have to do to find a term of $Q$ is to find
a term $g$ of $\jcomp{}{f}{Q}$. The result of applying $\tfindf{f}$ to $g$
will be a term of $Q$.
\end{rmk}

\begin{rmk}
These stronger rules also seem to imply that not every equivalence is going
to be an inductive morphism (when we add all the type constructors to the
theory). For instance, the interval is equivalent to the unit type. If the
unit type is defined via an inductive morphism $\emptyc\to\unit$ we get that
every family over $\unit$ is definitionally a constant type because
every family over $\emptyc$ is a weakening by the empty family. If the
equivalence from $\unit$ to the interval were inductive, this would in turn
imply that every type family over the interval is constant. However, this is
not the case because we have the family which has the unit type as a fiber
above one endpoint and the interval above the other.
\end{rmk}

\begin{lem}
The identity term
$\jhom{\Gamma}{A}{A}{\idtm{A}}$ is an inductive morphism
for each family $A$ of contexts over $\Gamma$
\end{lem}

\begin{proof}
Composition with the identity morphism is an identity operation.
\end{proof}

\begin{itemize}
\item Extensions of inductive morphisms are inductive
\item Weakenings of inductive morphisms are inductive
\item Substitutions of inductive morphisms are inductive
\end{itemize}


%\part{Categorical semantics}

%\section{E-objects in categories with finite limits}
In this section we assume that $\cat{C}$ is a finitely complete category and
whenever we write a pullback, we assume that it is chosen. Recall that for
any morphism $f:A\to B$ in a category $\cat{C}$ with chosen pullbacks, there
is a functor
\begin{equation*}
f^\ast : \cat{C}/B\to\cat{C}/A.
\end{equation*}
As usual, when $g:X\to B$ is a morphism, we will write $f^\ast(X)$ for the
domain of $f^\ast(g)$. When there is more than one morphism $X\to B$ involved,
as will be the case below, we will write $\pullback{A}{X}{f}{g}$. The projections
will be written as $\pullbackpr{1}{f}{g}$ and $\pullbackpr{2}{f}{g}$. So in this notation, a
typical pullback diagram has the following form:
\begin{equation*}
\begin{tikzcd}[column sep=large]
\pullback{A}{X}{f}{g}
  \ar{r}{\pullbackpr{1}{f}{g}}
  \ar{d}[swap]{\pullbackpr{2}{f}{g}}
  &
A \ar{d}{f}
  \\
X \ar{r}[swap]{g}
  &
B
\end{tikzcd}
\end{equation*}
Also, when we have a commutative diagram of the form
\begin{equation*}
\begin{tikzcd}
A \ar{r}{f}
  \ar{d}{a}
  &
X \ar{d}
  & 
B \ar{l}[swap]{g}
  \ar{d}{b}
  \\
A'
  \ar{r}[swap]{f'}
  &
X'
  &
B'
  \ar{l}{g'}
\end{tikzcd}
\end{equation*}
we will denote the unique map from $\pullback{A}{B}{f}{g}$ to $\pullback{A'}{B'}{f'}{g'}$
such that the diagram
\begin{equation*}
\begin{tikzcd}
  {}
  & 
\pullback{A'}{B'}{f'}{g'}
  \ar{dd}
  \ar{rr}
  &
  &
B'
  \ar{dd}{g'}
  \\
\pullback{A}{B}{f}{g}
  \ar{dd}
  \ar[crossing over]{rr}
  \ar[dotted]{ur}{\pullback{a}{b}{f'}{g'}}
  &
  &
B \ar{ur}{b}
  \\
  {}
  &
A'
  \ar{rr}
  &
  &
X'
  \\
A \ar{rr}[swap]{f}
  \ar{ur}{a}
  &
  &
X \ar[crossing over,leftarrow]{uu}[near end,swap]{g}
  \ar{ur}
\end{tikzcd}
\end{equation*}
commutes, by $\pullback{a}{b}{f'}{g'}$. In the current work, we shall
write $A\times B$ for the pullback of $A\rightarrow 1\leftarrow B$, and
$\pi_1$ and $\pi_2$ for its projections (thus, no separate choice of
cartesian products is made).

\subsection{Extension objects}
\begin{defn}
A \emph{pre-extension object $\stesys$ in $\cat{C}$} consists of a \emph{fundamental structure}, which is a diagram of the form
\begin{equation*}
\begin{tikzcd}
\stesyst
  \ar{d}[swap]{\ebd}
  \\
\stesysf
  \ar{d}[swap]{\eft}
  \\
\stesysc
\end{tikzcd}
\end{equation*}
in $\cat{C}$ together with the \emph{context extension} and \emph{family extension} operations
\begin{align*}
\ectxext &:\stesysf\to \stesysc\\
\efamext & :\stesysff\to \stesysf,
\end{align*}
respectively, such that the diagram
\begin{equation*}
\begin{tikzcd}
\stesysf_2 
  \ar{r}{\efamext} 
  \ar{d}[swap]{\eft[2]} 
  & 
\stesysf 
  \ar{d}{\eft}
  \\
\stesysf
  \ar{r}[swap]{\eft} 
  & 
\stesysc
\end{tikzcd}
\end{equation*}
commutes.
\end{defn}

\begin{defn}
We introduce the following notation:
\begin{align*}
\stesysf_2 
  & := \stesysff
  \\
\eft[2] 
  & := \pullbackpr{1}{\ectxext}{\eft} : \stesysf_2\to\stesysf
  \\
\stesysf_3 & := \pullback{\stesysf_2}{\stesysf_2}{\efamext}{\eft[2]}
  \\
\eft[3]
  & := \pullbackpr{1}{\efamext}{\eft[2]} : \stesysf_3\to\stesysf_2.
\end{align*}
Then it follows that the outer square in the diagram
\begin{equation*}
\begin{tikzcd}[column sep=large]
\stesysf_3
  \ar[dotted]{dr}{\eext{2}}
  \ar{rr}{\pullback{\pullbackpr{2}{\ectxext}{\eft}}{\pullbackpr{2}{\ectxext}{\eft}}{\ectxext}{\eft}}
  \ar{dd}[swap]{\eft[3]}
  & 
  &
\stesysf_2
  \ar{d}{\efamext}
  \\
  &
\stesysf_2
  \ar{d}[swap]{\eft[2]}
  \ar{r}{\pullbackpr{2}{\ectxext}{\eft}}
  &
\stesysf
  \ar{d}{\eft}
  \\
\stesysf_2
  \ar{r}[swap]{\eft[2]}
  &
\stesysf
  \ar{r}[swap]{\ectxext}
  &
\stesysc
\end{tikzcd}
\end{equation*}
commutes. We define $\eext{2}$ to be the unique morphism rendering the above diagram
commutative. Now we may continue to define
\begin{align*}
\stesysf_4 
  & := 
\pullback{\stesysf_3}{\stesysf_3}{\eext{2}}{\eft[3]}
  \\
\eft[4] 
  & := 
\pullbackpr{1}{\eext{2}}{\eft[3]}.
\end{align*}
Then we see that the outer square of the diagram
\begin{equation*}
\begin{tikzcd}[column sep=large]
\stesysf_4
  \ar[dotted]{dr}{\eext{3}}
  \ar{rr}{\pullback{\pullbackpr{2}{\efamext}{\eft[2]}}{\pullbackpr{2}{\efamext}{\eft[2]}}{\efamext}{\eft[2]}}
  \ar{dd}[swap]{\eft[4]}
  & 
  &
\stesysf_3
  \ar{d}{\eext{2}}
  \\
  &
\stesysf_3
  \ar{d}[swap]{\eft[3]}
  \ar{r}{\pullbackpr{2}{\efamext}{\eft[2]}}
  &
\stesysf_2
  \ar{d}{\eft[2]}
  \\
\stesysf_3
  \ar{r}[swap]{\eft[3]}
  &
\stesysf_2
  \ar{r}[swap]{\efamext}
  &
\stesysf
\end{tikzcd}
\end{equation*}
commutes,
so we may define $\eext{3}$ to be the unique map which renders the diagram
commutative. It
is straightforward to continue this process by induction, but we shall need not
go any further in this article.
\end{defn}

\begin{defn} 
An extension object is a pre-extension object $\stesys$ for which 
the diagrams
\begin{equation*}
\begin{tikzcd}
\stesysf_2 
  \ar{d}[swap]{\pullbackpr{2}{\ectxext}{\eft}} 
  \ar{r}{\efamext} 
  & 
\stesysf 
  \ar{d}{\ectxext}
  \\
\stesysf 
  \ar{r}[swap]{\ectxext} 
  & 
\stesysc
\end{tikzcd}
\qquad
\begin{tikzcd}
\stesysf_3
  \ar{d}[swap]{\pullbackpr{2}{\efamext}{\eft[2]}}
  \ar{r}{\eext{2}}
  & 
\stesysf_2 
  \ar{d}{\efamext} 
  \\
\stesysf_2 
  \ar{r}[swap]{\efamext} 
  &
\stesysf
\end{tikzcd}
\end{equation*}
commute.
\end{defn}

\begin{comment}
\begin{lem}
There exists an isomorphism $\alpha$ such that the triangle
\begin{equation*}
\begin{tikzcd}[column sep=tiny]
\pullback{\stesysf}{\stesysf_2}{\ectxext}{\eft\circ\eft[2]}
  \ar[dotted]{rr}{\alpha}
  \ar{dr}[swap]{\pullback{\catid{\stesysf}}{\efamext}{\ectxext}{\eft}}
  &
  &
\stesysf_3
  \ar{dl}{\eext{2}}
  \\
& \stesysf_2
\end{tikzcd}
\end{equation*}
commutes
\end{lem}

\begin{proof}
There is a unique morphism $\alpha:
\pullback{\stesysf}{\stesysf_2}{\ectxext}{\eft\circ\eft[2]}\to\stesysf_3$
rendering the diagram
\begin{equation*}
\begin{tikzcd}[column sep=large]
\pullback{\stesysf}{\stesysf_2}{\ectxext}{\eft\circ\eft[2]}
  \ar[bend left=10,yshift=.5ex]{drrr}{\pullbackpr{2}{\ectxext}{\eft}\circ\pullbackpr{2}{\ectxext}{\eft\circ\eft[2]}}
  \ar[bend right=10]{ddr}[swap]{\pullback{\catid{\stesysf}}{\eft[2]}{\ectxext}{\eft}}
  \ar[dotted]{dr}{\alpha}
  \\
& \stesysf_3
  \ar{r}{\pullbackpr{2}{\efamext}{\eft[2]}}
  \ar{d}{\eft[3]}
  &
\stesysf_2
  \ar{d}[swap]{\eft[2]}
  \ar{r}[swap]{\pullbackpr{2}{\ectxext}{\eft}}
  &
\stesysf
  \ar{d}{\eft}
  \\
{} & \stesysf_2
  \ar{r}[swap]{\efamext}
  &
\stesysf
  \ar{r}[swap]{\ectxext}
  &
\stesysc
\end{tikzcd}
\end{equation*}
\end{proof}
\end{comment}

\begin{defn}
Suppose $\stesys$ is a pre-extension object of $\cat{C}$. Then we define the pre-extension object
$\famesys{\stesys}$ to consist of the fundamental structure
\begin{equation*}
\begin{tikzcd}
\stesyst_2
  \ar{d}{\ebd[2]}
  \\
\stesysf_2
  \ar{d}{\eft[2]}
  \\
\stesysf
\end{tikzcd}
\end{equation*}
where
\begin{align*}
\stesyst_2 
  & := \pullback{\stesysf}{\stesyst}{\ectxext}{\eft\circ\ebd}
  \\
\ebd[2]
  & := \ectxext^\ast(\ebd),
\end{align*}
with the extension operations
\begin{align*}
\efamext 
  & 
  : \stesysf_2\to\stesysf\\
\eext{2} & : \stesysf_3\to\stesysf_2.
\end{align*}
\end{defn}

In \autoref{famextobj} below we shall show that $\famesys{\stesys}$ is an
extension algebra whenever $\stesys$ is an extension algebra. We shall need
a handful of lemmas to give the proof.

\begin{defn}
We define
\begin{align*}
\beta_1 
  & := 
\pullbackpr{2}{\ectxext}{\eft}
  & &
  : \stesysf_2\to\stesysf
  \\
\beta_2
  & :=
\pullback{\beta_1}{\beta_1}{\ectxext}{\eft}
  & &
  : \stesysf_3\to\stesysf_2
  \\
\beta_3
  & :=
\pullback{\beta_2}{\beta_2}{\efamext}{\eft[2]}
  & &
  : \stesysf_4\to\stesysf_3.
\end{align*}
\end{defn}

\begin{lem}
Let $\stesys$ be a pre-extension object. Then the square
\begin{equation*}
\begin{tikzcd}[column sep=10em]
\stesysf_4
  \ar{r}{\pullback{\pullbackpr{2}{\efamext}{\eft[2]}}{\pullbackpr{2}{\efamext}{\eft[2]}}{\efamext}{\eft[2]}}
  \ar{d}[swap]{\beta_3}
  &
\stesysf_3
  \ar{d}{\beta_2}
  \\
\stesysf_3
  \ar{r}[swap]{\beta_2}
  &
\stesysf_2
\end{tikzcd}
\end{equation*}
commutes.
\end{lem}

\begin{proof}
Left to the reader.
\end{proof}

\begin{comment}
\begin{proof}
It is straightforward to verify the equalities
\begin{align*}
\pullbackpr{1}{\ectxext}{\eft}\circ\beta_2\circ
  (\pullback{\pullbackpr{2}{\efamext}{\eft[2]}}{\pullbackpr{2}{\efamext}{\eft[2]}}{\efamext}{\eft[2]})
  & =
\pullbackpr{1}{\ectxext}{\eft}\circ\beta_2\circ\beta_3
  \\
\pullbackpr{2}{\ectxext}{\eft}\circ\beta_2\circ
  (\pullback{\pullbackpr{2}{\efamext}{\eft[2]}}{\pullbackpr{2}{\efamext}{\eft[2]}}{\efamext}{\eft[2]})
  & =
\pullbackpr{2}{\ectxext}{\eft}\circ\beta_2\circ\beta_3.\qedhere
\end{align*}
\end{proof}
\end{comment}

Note that the square
\begin{equation*}
\begin{tikzcd}
\stesysf_3
  \ar{r}{\beta_2}
  \ar{d}[swap]{\eext{2}}
  &
\stesysf_2
  \ar{d}{\efamext}
  \\
\stesysf_2
  \ar{r}[swap]{\beta_1}
  &
\stesysf
\end{tikzcd}
\end{equation*}
commutes by definition. We have a similar result relating $\eext{3}$ and
$\eext{2}$.

\begin{lem}
Let $\stesys$ be a pre-extension object. Then the square
\begin{equation*}
\begin{tikzcd}
\stesysf_4
  \ar{r}{\beta_3}
  \ar{d}[swap]{\eext{3}}
  &
\stesysf_3
  \ar{d}{\eext{2}}
  \\
\stesysf_3
  \ar{r}[swap]{\beta_2}
  &
\stesysf_2
\end{tikzcd}
\end{equation*}
commutes.
\end{lem}

\begin{proof}
Left to the reader.
\end{proof}
\begin{comment}
\begin{proof}
It is straightforward to verify the equalities
\begin{align*}
\pullbackpr{1}{\ectxext}{\eft}\circ\beta_2\circ\eext{3}
  & = \beta_1\circ\eft[3]\circ\eft[4]
  \\
\pullbackpr{1}{\ectxext}{\eft}\circ\eext{2}\circ\beta_3
  & = \beta_1\circ\eft[3]\circ\eft[4].
\end{align*}
Thus, it remains to verify that
\begin{equation*}
\pullbackpr{2}{\ectxext}{\eft}\circ\beta_2\circ\eext{3}
  = \pullbackpr{2}{\ectxext}{\eft}\circ\eext{2}\circ\beta_3.
\end{equation*}
It is straightforward to see that the diagram
\begin{equation*}
\begin{tikzcd}[column sep=large]
\stesysf_4
  \ar{dd}[swap]{\pullback{\pullbackpr{2}{\efamext}{\eft[2]}}{\pullbackpr{2}{\efamext}{\eft[2]}}{\efamext}{\eft[2]}}
  \ar{r}{\eext{3}}
  &
\stesysf_3
  \ar{r}{\beta_2}
  \ar{d}[swap]{\pullbackpr{2}{\efamext}{\eft[2]}}
  &
\stesysf_2
  \ar{dd}{\pullbackpr{2}{\ectxext}{\eft}}
  \\
  {} &
\stesysf_2
  \ar{dr}{\beta_1}
  \\
\stesysf_3
  \ar{ur}{\eext{2}}
  \ar{r}[swap]{\beta_2}
  &
\stesysf_2
  \ar{r}[swap]{\efamext}
  &
\stesysf
\end{tikzcd}
\end{equation*}
commutes. It is likewise straightforward to see that the diagram
\begin{equation*}
\begin{tikzcd}
\stesysf_4
  \ar{r}{\beta_3}
  \ar{d}[swap]{\pullback{\pullbackpr{2}{\efamext}{\eft[2]}}{\pullbackpr{2}{\efamext}{\eft[2]}}{\efamext}{\eft[2]}}
  &
\stesysf_3
  \ar{r}{\eext{2}}
  \ar{d}[swap]{\beta_2}
  &
\stesysf_2
  \ar{d}{\pullbackpr{2}{\ectxext}{\eft}}
  \\
\stesysf_3
  \ar{r}[swap]{\beta_2}
  &
\stesysf_2
  \ar{r}[swap]{\efamext}
  &
\stesysf
\end{tikzcd}
\end{equation*}
commutes, completing our goal.
\end{proof}
\end{comment}

\begin{thm}[Local extension structure]\label{famextobj}
If $\stesys$ is an extension object, then so is $\famesys{\stesys}$.
\end{thm}

\begin{proof}
Note that the diagram
\begin{equation*}
\begin{tikzcd}
\stesysf_3
  \ar{d}[swap]{\pullbackpr{2}{\efamext}{\eft[2]}}
  \ar{r}{\eext{2}}
  & 
\stesysf_2 
  \ar{d}{\efamext} 
  \\
\stesysf_2 
  \ar{r}[swap]{\efamext} 
  &
\stesysf
\end{tikzcd}
\end{equation*}
commutes by assumption. For the second condition, we have to show that the
diagram
\begin{equation*}
\begin{tikzcd}
\stesysf_4
  \ar{d}[swap]{\pullbackpr{2}{\eext{2}}{\eft[3]}}
  \ar{r}{\eext{3}}
  & 
\stesysf_3
  \ar{d}{\eext{2}} 
  \\
\stesysf_3
  \ar{r}[swap]{\eext{2}} 
  &
\stesysf_2
\end{tikzcd}
\end{equation*}
Since this is a question about two maps into a pullback, it suffices to verify
that
\begin{align*}
\pullbackpr{1}{\ectxext}{\eft}\circ\eext{2}\circ\eext{3}
  & =
\pullbackpr{1}{\ectxext}{\eft}\circ\eext{2}\circ\pullbackpr{2}{\eext{2}}{\eft[3]}
  \\
\pullbackpr{2}{\ectxext}{\eft}\circ\eext{2}\circ\eext{3}
  & =
\pullbackpr{2}{\ectxext}{\eft}\circ\eext{2}\circ\pullbackpr{2}{\eext{2}}{\eft[3]}.
\end{align*}
For the first equality, it is fairly straightforward to show that both the
equalities
\begin{equation*}
\pullbackpr{1}{\ectxext}{\eft}\circ\eext{2}\circ\eext{3}
  =
\eft[2]\circ\eft[3]\circ\eft[4]
\end{equation*}
and
\begin{equation*}
\pullbackpr{1}{\ectxext}{\eft}\circ\eext{2}\circ\pullbackpr{2}{\eext{2}}{\eft[3]}
  =
\eft[2]\circ\eft[3]\circ\eft[4].
\end{equation*}
hold. For the second subgoal (which is more tricky). Notice first that the
diagram
\begin{equation*}
\begin{tikzcd}
\stesysf_4
  \ar{r}{\eext{3}}
  \ar{d}[swap]{\beta_3}
  &
\stesysf_3
  \ar{r}{\eext{2}}
  \ar{d}[swap]{\beta_2}
  &
\stesysf_2
  \ar{d}{\pullbackpr{2}{\ectxext}{\eft}}
  \\
\stesysf_3
  \ar{r}[swap]{\eext{2}}
  &
\stesysf_2
  \ar{r}[swap]{\efamext}
  &
\stesysf
\end{tikzcd}
\end{equation*}
commutes. We also have the commutative diagram
\begin{equation*}
\begin{tikzcd}[column sep=large]
\stesysf_4
  \ar{r}{\pullbackpr{2}{\eext{2}}{\eft[3]}}
  \ar{d}[swap]{\beta_3}
  &
\stesysf_3
  \ar{r}{\eext{2}}
  \ar{d}[swap]{\beta_2}
  &
\stesysf_2
  \ar{d}{\pullbackpr{2}{\ectxext}{\eft}}
  \\
\stesysf_3
  \ar{r}{\pullbackpr{2}{\efamext}{\eft[2]}}
  \ar{dr}[swap]{\eext{2}}
  &
\stesysf_2
  \ar{r}{\efamext}
  &
\stesysf
  \\
  {} &
\stesysf_2
  \ar{ur}[swap]{\efamext}
\end{tikzcd}
\end{equation*}
completing the proof.
\end{proof}

\subsection{(Pre-)extension homomorphisms}\label{subsection:e_extension_homomorphisms}
In this subsection we start with the study of pre-extension homomorphisms, which
will include the extension homomorphisms since they will be the pre-extension
homomorphisms of which both the domain and codomain are extension objects.
Our main examples of extension homomorphisms will be the operations of weakening
and substitution. There are some basic examples of pre-extension homomorphisms
that will be useful too, which get introduced in the this section and in
\autoref{subsection:change_of_base}. In this section, we will mainly be
interested in pre-extension homomorphisms between local pre-extension objects.
We will end this section by proving that a retract of an extension object is
always an extension object.

\begin{defn}
Let $\stesys$ and $\stesys'$ be pre-extension algebras. A \emph{pre-extension 
homomorphism $f$ from $\stesys'$ to $\stesys$} is a triple $(f_0,f_1,f^t)$ 
consisting of morphisms
\begin{equation*}
\begin{tikzcd}
\stesyst' 
  \ar{r}{f^t}
  \ar{d}[swap]{\ebd'}
  &
\stesyst
  \ar{d}{\ebd}
  \\
\stesysf'
  \ar{r}{f_1}
  \ar{d}[swap]{\eft'}
  &
\stesysf
  \ar{d}{\eft}
  \\
\stesysc' 
  \ar{r}[swap]{f_0}
  &
\stesysc
\end{tikzcd}
\end{equation*}
such that the indicated squares commute, for which furthermore the squares
\begin{equation*}
\begin{tikzcd}
\stesysf' 
  \ar{r}{f_1}
  \ar{d}[swap]{\ectxext'}
  &
\stesysf
  \ar{d}{\ectxext}
  \\
\stesysc'
  \ar{r}[swap]{f_0}
  &
\stesysc
\end{tikzcd}
\end{equation*}
and
\begin{equation*}
\begin{tikzcd}[column sep=huge]
\stesysf'\times_{\ectxext',\eft'} \stesysf'
  \ar{r}{f_1\times_{\ectxext,\eft} f_1}
  \ar{d}[swap]{\efamext'}
  &
\stesysf\times_{\ectxext,\eft} \stesysf
  \ar{d}{\efamext}
  \\
\stesysf'
  \ar{r}[swap]{f_1}
  &
\stesysf
\end{tikzcd}
\end{equation*}
Composition and the identity homomorphism are defined in the expected way. We
define furthermore
\begin{align*}
f_2 & := \pullback{f_1}{f_1}{\ectxext}{\eft}
  \\
f_3 & := \pullback{f_2}{f_2}{\efamext}{\eft[2]}.
\end{align*}
\end{defn}

\begin{defn}
A pre-extension homomorphism between extension objects is called an extension
homomorphism.
\end{defn}

\begin{lem}
Let $\stesys$ be an extension object. Then
\begin{equation*}
\begin{tikzcd}[column sep=huge]
\stesyst_2
  \ar{r}{\pullbackpr{2}{\ectxext}{\eft\circ\ebd}}
  \ar{d}[swap]{\ebd[2]}
  &
\stesyst
  \ar{d}{\ebd}
  \\
\stesysf_2
  \ar{r}{\pullbackpr{2}{\ectxext}{\eft}}
  \ar{d}[swap]{\eft[2]}
  &
\stesysf
  \ar{d}{\eft}
  \\
\stesysf
  \ar{r}[swap]{\ectxext}
  &
\stesysc
\end{tikzcd}
\end{equation*}
assembles an extension homomorphism $\mathbf{e}_0:\famesys{\stesys}\to\stesys$.
Likewise, we have an extension homomorphism
$\mathbf{e}_1:\famesys{\famesys{\stesys}}\to\famesys{\stesys}$. Thus, a
pre-extension object is an extension object if and only if $\mathbf{e}_0$
and $\mathbf{e}_1$ are pre-extension homomorphisms.
\end{lem}

\begin{proof}
Immediate from the conditions of being an extension object.
\end{proof}

\begin{defn}
Let $\stesys$ be a pre-extension object. Then
\begin{equation*}
\begin{tikzcd}
\stesyst_3
  \ar{r}{\beta^t}
  \ar{d}[swap]{\ebd[3]}
  &
\stesyst_2
  \ar{d}{\ebd[2]}
  \\
\stesysf_3
  \ar{r}{\beta_2}
  \ar{d}[swap]{\eft[3]}
  &
\stesysf_2
  \ar{d}{\eft[2]}
  \\
\stesysf_2
  \ar{r}[swap]{\beta_1}
  &
\stesysf
\end{tikzcd}
\qquad
\text{and}
\qquad
\begin{tikzcd}
\stesyst_4
  \ar{r}{\beta^t_2}
  \ar{d}[swap]{\ebd[4]}
  &
\stesyst_3
  \ar{d}{\ebd[3]}
  \\
\stesysf_4
  \ar{r}{\beta_3}
  \ar{d}[swap]{\eft[4]}
  &
\stesysf_3
  \ar{d}{\eft[3]}
  \\
\stesysf_3
  \ar{r}[swap]{\beta_2}
  &
\stesysf_2
\end{tikzcd}
\end{equation*}
assemble pre-extension homomorphisms 
\(
\boldsymbol{\beta}
  :
\famesys{\famesys{\stesys}}
  \to
\famesys{\stesys}
\) 
and
\(
\boldsymbol{\beta}_\mathbf{2}
  :
\famesys{\famesys{\famesys{\stesys}}}
  \to
\famesys{\famesys{\stesys}}
\).
\end{defn}

\begin{defn}\label{famehom}
Suppose that $f:\stesys'\to\stesys$ is a pre-extension homomorphism. Then we
define $\famehom{f}:\famesys{\stesys'}\to\famesys{\stesys}$ to consist of
\begin{equation*}
\begin{tikzcd}
\stesyst_2'
  \ar{r}{f^t_2}
  \ar{d}[swap]{\ebd[2]'}
  &
\stesyst_2
  \ar{d}{\ebd[2]}
  \\
\stesysf_2'
  \ar{r}{f_2}
  \ar{d}[swap]{\eft[2]'}
  &
\stesysf_2
  \ar{d}{\eft[2]}
  \\
\stesysf'
  \ar{r}[swap]{f_1}
  &
\stesysf
\end{tikzcd}
\end{equation*}
where we define
\begin{equation*}
f^t_2 := \pullback{f_1}{f^t}{\ectxext}{\eft\circ\ebd}.
\end{equation*}
\end{defn}

\begin{lem}
The triple $\famehom{f}$ defined in \autoref{famehom} is a pre-extension homomorphism.
\end{lem}

\begin{proof}
Note that the square
\begin{equation*}
\begin{tikzcd}
\stesysf_2'
  \ar{r}{f_2}
  \ar{d}[swap]{\efamext'}
  &
\stesysf_2
  \ar{d}{\efamext}
  \\
\stesysf'
  \ar{r}[swap]{f_1}
  &
\stesysf
\end{tikzcd}
\end{equation*}
commutes by assumption. Thus, it remains to show that the square
\begin{equation*}
\begin{tikzcd}
\stesysf_3'
  \ar{r}{f_3}
  \ar{d}[swap]{\eext{2}'}
  &
\stesysf_3
  \ar{d}{\eext{2}}
  \\
\stesysf_2'
  \ar{r}[swap]{f_2}
  &
\stesysf_2
\end{tikzcd}
\end{equation*}
commutes. It is equivalent to show that the equalities
\begin{align*}
\pullbackpr{1}{\ectxext}{\eft}\circ f_2\circ\eext{2}'
  & =
\pullbackpr{1}{\ectxext}{\eft}\circ \eext{2}\circ f_3
  \\
\pullbackpr{2}{\ectxext}{\eft}\circ f_2\circ\eext{2}'
  & =
\pullbackpr{2}{\ectxext}{\eft}\circ \eext{2}\circ f_3
\end{align*}
both hold. For the first, it is straightforward to verify that the diagram
\begin{equation*}
\begin{tikzcd}[column sep=large]
{} &
\stesysf_2'
  \ar{r}{f_2}
  \ar{dr}[near end]{\pullbackpr{1}{\ectxext'}{\eft'}}
  &
\stesysf_2
  \ar{dr}{\pullbackpr{1}{\ectxext}{\eft}}
  \\
\stesysf_3'
  \ar{ur}{\eext{2}'}
  \ar{r}[swap]{\beta_2'}
  \ar{ddr}[swap]{f_3}
  &
\stesysf_2'
  \ar{r}{\efamext'}
  \ar{dr}[swap]{f_2}
  &
\stesysf'
  \ar{r}{f_1}
  &
\stesysf
  \\
{} & {} &
\stesysf_2
  \ar{ur}[near start]{\efamext}
  \\
{} &
\stesysf_3
  \ar{r}[swap]{\eext{2}}
  \ar{ur}{\beta_2}
  &
\stesysf_2
  \ar{uur}[swap]{\pullbackpr{1}{\ectxext}{\eft}}
\end{tikzcd}
\end{equation*}
commutes. For the second, note that the diagram
\begin{equation*}
\begin{tikzcd}[column sep=large]
{} &
\stesysf_2'
  \ar{r}{f_2}
  \ar{dr}{\beta_1'}
  &
\stesysf_2
  \ar{ddr}{\pullbackpr{2}{\ectxext}{\eft}}
  \\
{} & {} &
\stesysf'
  \ar{dr}[swap,near start]{f_1}
  \\
\stesysf_3'
  \ar{uur}{\eext{2}'}
  \ar{r}{\beta_2'}
  \ar{dr}[swap]{f_3}
  &
\stesysf_2'
  \ar{r}{f_2}
  \ar{ur}{\efamext'}
  &
\stesysf_2
  \ar{r}[swap]{\efamext}
  &
\stesysf
  \\
{} &
\stesysf_3
  \ar{r}[swap]{\eext{2}}
  \ar{ur}[near start]{\beta_2}
  &
\stesysf_2
  \ar{ur}[swap]{\pullbackpr{2}{\ectxext}{\eft}}
\end{tikzcd}
\end{equation*}
commutes.
\end{proof}

\begin{lem}[Stability under retracts]\label{esys-retract}
Suppose $f:\stesys\to\stesys'$ is a pre-extension homomorphism between
pre-extension objects. If there is a pre-extension homomorphism $g:\stesys'\to
\stesys$ such that $g\circ f=\catid{\stesys}$ and $\stesys'$ is an extension
algebra, then $\stesys$ is an extension algebra.
\end{lem}

Before we start with the proof, note that we have the equalities
$g_2\circ f_2=\catid{\stesysf_2}$ and $g_3\circ f_3=\catid{\stesysf_3}$
under the hypotheses of the lemma.

\begin{proof}
Our first subgoal is to show that the square
\begin{equation*}
\begin{tikzcd}
\stesysf_2 
  \ar{r}{\efamext} 
  \ar{d}[swap]{\pullbackpr{2}{\ectxext}{\eft}} 
  & 
\stesysf 
  \ar{d}{\ectxext}
  \\
\stesysf
  \ar{r}[swap]{\ectxext} 
  & 
\stesysc
\end{tikzcd}
\end{equation*}
commutes. Note that in the diagram
\begin{equation*}
\begin{tikzcd}
  {}
  & 
\stesysf
  \ar{dd}[near start]{\ectxext}
  \ar{rr}{f_1}
  &
  &
\stesysf'
  \ar{dd}[near start]{\ectxext'}
  \ar{rr}{g_1}
  &
  &
\stesysf
  \ar{dd}{\ectxext}
  \\
\stesysf_2
  \ar{dd}[swap]{\pullbackpr{2}{\ectxext}{\eft}}
  \ar[crossing over]{rr}[swap,near start]{f_2}
  \ar{ur}{\efamext}
  &
  &
\stesysf_2'
  \ar{ur}[near start]{\efamext'}
  \ar[crossing over]{rr}[swap,near start]{g_2}
  &
  &
\stesysf_2
  \ar{ur}[swap,near start]{\efamext}
  \\
  {}
  &
\stesysc
  \ar{rr}[near start]{f_0}
  &
  &
\stesysc'
  \ar{rr}[near start]{g_0}
  &
  &
\stesysc
  \\
\stesysf 
  \ar{rr}[swap]{f_1}
  \ar{ur}{\ectxext}
  &
  &
\stesysf' 
  \ar[crossing over,leftarrow]{uu}[near end,swap]{\pullbackpr{2}{\ectxext'}{\eft'}}
  \ar{ur}[swap,near end]{\ectxext'}
  \ar{rr}[swap]{g_1}
  &
  &
\stesysf
  \ar[crossing over,leftarrow]{uu}[near end,swap]{\pullbackpr{2}{\ectxext}{\eft}}
  \ar{ur}[swap]{\ectxext}
\end{tikzcd}
\end{equation*}
all the faces minus the far left and far right face commute. Using that $g$
is a section of $f$, we can read off that also the far left face commutes,
completing our first subgoal.
 
For the second subgoal, note that also $\famehom{g}\circ\famehom{f}=
\catid{\famesys{\stesys}}$ and that $\famesys{\stesys'}$ is an extension object.
Thus we can apply what we have proven so far to conclude that the square
\begin{equation*}
\begin{tikzcd}
\stesysf_3 
  \ar{r}{\eext{2}} 
  \ar{d}[swap]{\pullbackpr{2}{\efamext}{\eft[2]}} 
  & 
\stesysf_2 
  \ar{d}{\efamext}
  \\
\stesysf_2
  \ar{r}[swap]{\efamext} 
  & 
\stesysf
\end{tikzcd}
\end{equation*}
commutes.
\end{proof}

\subsection{The change of base of (pre-)extension objects}
\label{subsection:change_of_base}
An important construction of (pre-)extension objects is the change of base. It
allows us to consider `parametrized homomorphisms', such as weakening and
substitution.

\begin{defn}
Suppose $f:\stesys\to\stesys'$ is a pre-extension homomorphism. We say that
a diagram
\begin{equation*}
\begin{tikzcd}
\stesys
  \ar{r}{f}
  \ar{d}[swap]{p}
  &
\stesys'
  \ar{d}{p'}
  \\
X \ar{r}[swap]{g}
  &
Y
\end{tikzcd}
\end{equation*}
commutes if the diagram
\begin{equation*}
\begin{tikzcd}
\stesysc
  \ar{r}{f_0}
  \ar{d}[swap]{p}
  &
\stesysc'
  \ar{d}{p'}
  \\
X \ar{r}[swap]{g}
  &
Y
\end{tikzcd}
\end{equation*}
commutes.
\end{defn}

The first goal in this subsection is to define for every (pre-)extension object 
$\stesys$ and every $p:\stesysc\rightarrow X\leftarrow Y:g$, a (pre-)extension
object $\cobesys{Y}{\stesys}{g}{p}$ with a homomorphism $\pullbackpr{2}{g}{p}:
\cobesys{Y}{\stesys}{g}{p}\to\stesys$ and a morphism $\pullbackpr{1}{g}{p}:
\pullback{Y}{\stesysc}{g}{p}\to Y$ such that for every diagram
\begin{equation*}
\begin{tikzcd}[column sep=large]
\stesys'
  \ar[bend right=10]{ddr}[swap]{p'}
  \ar[bend left=10]{rrd}{f}
  \ar[dotted]{dr}[near end]{[p',f]}
  \\
  {}&
\cobesys{Y}{\stesys}{g}{p}
  \ar{d}{\pullbackpr{1}{g}{p}}
  \ar{r}[swap]{\pullbackpr{2}{g}{p}}
  &
\stesys
  \ar{d}{p}
  \\
  {}&
Y \ar{r}[swap]{g}
  &
X
\end{tikzcd}
\end{equation*}
of which the outer square commutes, the (pre-)extension homomorphism $[p',f]$ exists
and is unique with the property that it renders the diagram commutative. We will
give the definition of $\cobesys{Y}{\stesys}{g}{p}$ in \autoref{cobesys}. After
proving that the change of base of a pre-extension algebra is indeed a
pre-extension algebra (\autoref{cobesys-preext}) and that the change of base
of an extension algebra is an extension algebra (\autoref{cobesys-ext}), we
will demonstrate the above unique existence in \autoref{cobesys-existence,%
cobesys-pullback}.

The second goal in this subsection is to follow the same procedure for
$\famesys{\famesys{\stesys}}$ to show that it is equivalent to
$\cobesys{\stesysf}{\famesys{\stesys}}{\ectxext}{\eft}$. We will do this by
verifying directly that it has the universal property of the change of base
described above, because we will use the ingredients in our definition of
weakening and substitution objects.

\begin{defn}[Change of base]\label{cobesys}
Suppose $\stesys$ is a pre-extension object in $\cat{C}$ and that 
$p:\stesysc\rightarrow X\leftarrow Y:g$.
Then we define the pre-extension object $\cobesys{Y}{\stesys}{g}{p}$ to consist of
\begin{equation*}
\begin{tikzcd}
\cobesys{Y}{\stesyst}{g}{p\circ\eft\circ\ebd}
  \ar{r}
  \ar{d}[swap]{g^\ast(\ebd)}
  &
\stesyst
  \ar{d}{\ebd}
  \\
\cobesys{Y}{\stesysf}{g}{p\circ\eft}
  \ar{r}
  \ar{d}[swap]{g^\ast(\eft)}
  &
\stesysf
  \ar{d}{\eft}
  \\
\cobesys{Y}{\stesysc}{g}{p}
  \ar{r}
  \ar{d}[swap]{\pullbackpr{1}{g}{p}}
  &
\stesysc
  \ar{d}{p}
  \\
Y \ar{r}[swap]{g}
  &
X
\end{tikzcd}
\end{equation*} 
and the operations
\begin{align*}
\cobesys{Y}{\ectxext}{g}{p} 
  & : \pullback{Y}{\stesysf}{g}{p\circ\eft}\to \pullback{Y}{\stesysc}{g}{p}\\
\cobesys{Y}{\efamext}{g}{p} 
  & : \pullback
    {\pullback{Y}{\stesysf}{g}{p\circ\eft}}
    {\pullback{Y}{\stesysf}{g}{p\circ\eft}}
    {\cobesys{Y}{\ectxext}{g}{p}}
    {g^\ast(\eft)}
  \to 
  \pullback{Y}{\stesysf}{g}{p\circ\eft}.
\end{align*}
defined by
\begin{equation*}
\cobesys{Y}{\ectxext}{g}{p} := \pullback{\catid{Y}}{\ectxext}{g}{p}
\end{equation*}
and where $\cobesys{Y}{\efamext}{g}{p}$ is defined by rendering the diagram
\begin{equation*}
\begin{tikzcd}[column sep=large]
(\cobesys{Y}{\stesysf}{g}{p\circ\eft})_2
  \ar{rr}{\pullback{\pullbackpr{2}{g}{p\circ\eft}}{\pullbackpr{2}{g}{p\circ\eft}}{\ectxext}{\eft}}
  \ar{dd}[swap]{\pullbackpr{1}{\cobesys{Y}{\ectxext}{g}{p}}{g^\ast(\eft)}}
  \ar[dotted]{dr}[swap]{\cobesys{Y}{\efamext}{g}{p}}
  &
  &
\stesysf_2
  \ar{d}{\efamext}
  \\
  {}&
\cobesys{Y}{\stesysf}{g}{p\circ\eft}
  \ar{r}{\pullbackpr{2}{g}{p\circ\eft}}
  \ar{d}[swap]{\pullbackpr{1}{g}{p\circ\eft}}
  &
\stesysf
  \ar{d}{p\circ\eft}
  \\
\cobesys{Y}{\stesysf}{g}{p\circ\eft}
  \ar{r}[swap]{\pullbackpr{1}{g}{p\circ\eft}}
  &
Y \ar{r}[swap]{g}
  &
X
\end{tikzcd}
\end{equation*} 
commutative. 
The process of obtaining the pre-extension object $\cobesys{Y}{\stesys}{g}{p}$ out of $\stesys$
and $g:Y\to X$ is also called the \emph{change of base}.
\end{defn}

\begin{lem}\label{cobesys-preext}
Any change of base of a pre-extension object is a pre-extension object.
\end{lem}

\begin{proof}
Let $\stesys$ be an extension algebra and consider $p:\stesysc\rightarrow X\leftarrow Y:g$.
We need to verify that the square
\begin{equation*}
\begin{tikzcd}[column sep=large]
(\pullback{Y}{\stesysf}{g}{p\circ\eft})_2
  \ar{r}{\cobesys{Y}{\efamext}{g}{p}} 
  \ar{d}[swap]{\pullbackpr{1}{\cobesys{Y}{\ectxext}{g}{p}}{g^\ast(\eft)}} 
  & 
\pullback{Y}{\stesysf}{g}{p\circ\eft}
  \ar{d}{g^\ast(\eft)}
  \\
\pullback{Y}{\stesysf}{g}{p\circ\eft}
  \ar{r}[swap]{g^\ast(\eft)} 
  & 
\pullback{Y}{\stesysc}{g}{p}
\end{tikzcd}
\end{equation*}
commutes. It is fairly obvious that
\begin{equation*}
\pullbackpr{1}{g}{p}\circ g^\ast(\eft)\circ (\cobesys{Y}{\efamext}{g}{p})
  =
\pullbackpr{1}{g}{p\circ\eft}\circ \pullbackpr{1}{\cobesys{Y}{\ectxext}{g}{p}}{g^\ast(\eft)}
\end{equation*}
and that the diagram
\begin{equation*}
\begin{tikzcd}
  {}&
  {}&
\pullback{Y}{\stesysf}{g}{p\circ\eft}
  \ar{rr}{g^\ast(\eft)}
  \ar{dr}[swap]{\pullbackpr{2}{g}{p\circ\eft}}
  &
  {}&
\pullback{Y}{\stesysc}{g}{p}
  \ar{ddr}{\pullbackpr{2}{g}{p}}
  \\
  {}&
  {}&
  {}&
\stesysf
  \ar{drr}[swap]{\eft}
  \\
(\pullback{y}{\stesysf}{g}{p\circ\eft})_2
  \ar{uurr}{\cobesys{Y}{\efamext}{g}{p}}
  \ar{rr}[swap,yshift=-.5ex]{\pullback{\pullbackpr{2}{g}{p\circ\eft}}{\pullbackpr{2}{g}{p\circ\eft}}{\ectxext}{\eft}}
  \ar{ddrr}[swap]{\pullbackpr{1}{\cobesys{Y}{\ectxext}{g}{p}}{g^\ast(\eft)}}
  &
  {}&
\stesysf_2
  \ar{ur}{\efamext}
  \ar{dr}[swap]{\eft[2]}
  &
  {}&
  {}&
\stesysc
  \\
  {}&
  {}&
  {}&
\stesysf
  \ar{urr}{\eft}
  \\
  {}&
  {}&
\pullback{Y}{\stesysf}{g}{p\circ\eft}
  \ar{rr}[swap]{g^\ast(\eft)}
  \ar{ur}{\pullbackpr{2}{g}{p\circ\eft}}
  &
  {}&
\pullback{Y}{\stesysc}{g}{p}
  \ar{uur}[swap]{\pullbackpr{2}{g}{p}}
\end{tikzcd}
\end{equation*}
commutes.
\end{proof}

\begin{thm}\label{cobesys-ext}
The change of base of an extension algebra is an extension algebra.
\end{thm}

\begin{proof}
Our first subgoal is to verify that the square
\begin{equation*}
\begin{tikzcd}[column sep=large]
(\pullback{Y}{\stesysf}{g}{p\circ\eft})_2
  \ar{r}{\cobesys{Y}{\efamext}{g}{p}} 
  \ar{d}[swap]{\pullbackpr{2}{\cobesys{Y}{\ectxext}{g}{p}}{g^\ast(\eft)}} 
  & 
\pullback{Y}{\stesysf}{g}{p\circ\eft}
  \ar{d}{\cobesys{Y}{\ectxext}{g}{p}}
  \\
\pullback{Y}{\stesysf}{g}{p\circ\eft}
  \ar{r}[swap]{\cobesys{Y}{\ectxext}{g}{p}} 
  & 
\pullback{Y}{\stesysc}{g}{p}
\end{tikzcd}
\end{equation*}
\end{proof}

The following construction is useful for defining extension homomorphisms into
`higher' extension objects

\begin{defn}\label{cobesys-existence}
Consider a commutative diagram
\begin{equation*}
\begin{tikzcd}
\stesys'
  \ar{r}{f}
  \ar{d}[swap]{p'}
  &
\stesys
  \ar{d}{p}
  \\
Y \ar{r}[swap]{g}
  &
X
\end{tikzcd}
\end{equation*}
Then we construct $[p,f]:\stesys'\to\cobesys{Y}{\stesys}{g}{p}$
\begin{itemize}
\item by defining $[p,f]_0:\stesysc'\to\pullback{Y}{\stesysc}{g}{p}$ be the uniqe
morphism rendering the diagram
\begin{equation*}
\begin{tikzcd}[column sep=large]
\stesysc'
  \ar[bend right=10]{ddr}[swap]{p'}
  \ar[bend left=10]{rrd}{f_0}
  \ar{dr}[near end]{[p,f]_0}
  \\
  {}&
\pullback{Y}{\stesysc}{g}{p}
  \ar{r}[swap]{\pullbackpr{2}{g}{p}}
  \ar{d}{\pullbackpr{1}{g}{p}}
  &
\stesysc
  \ar{d}{p}
  \\
  {}&
Y \ar{r}[swap]{g}
  &
X
\end{tikzcd}
\end{equation*}
commutative.
\item by defining $[p,f]_1:\stesysf'\to\pullback{Y}{\stesysf}{g}{p\circ\eft}$ be the uniqe
morphism rendering the diagram
\begin{equation*}
\begin{tikzcd}[column sep=huge]
\stesysf'
  \ar[bend right=10]{ddr}[swap]{p'\circ\eft'}
  \ar[bend left=10]{rrd}{f_1}
  \ar{dr}[near end]{[p,f]_1}
  \\
  {}&
\pullback{Y}{\stesysf}{g}{p\circ\eft}
  \ar{r}[swap]{\pullbackpr{2}{g}{p\circ\eft}}
  \ar{d}{\pullbackpr{1}{g}{p\circ\eft}}
  &
\stesysc
  \ar{d}{p\circ\eft}
  \\
  {}&
Y \ar{r}[swap]{g}
  &
X
\end{tikzcd}
\end{equation*}
commutative.
\item by defining $[p,f]^t:\stesyst'\to\pullback{Y}{\stesyst}{g}{p\circ\eft\circ\ebd}$ be the uniqe
morphism rendering the diagram
\begin{equation*}
\begin{tikzcd}[column sep=huge]
\stesyst'
  \ar[bend right=10]{ddr}[swap]{p'\circ\eft'\circ\ebd'}
  \ar[bend left=10]{rrd}{f^t}
  \ar{dr}[near end]{[p,f]^t}
  \\
  {}&
\pullback{Y}{\stesysf}{g}{p\circ\eft\circ\ebd}
  \ar{r}[swap]{\pullbackpr{2}{g}{p\circ\eft\circ\ebd}}
  \ar{d}{\pullbackpr{1}{g}{p\circ\eft\circ\ebd}}
  &
\stesysc
  \ar{d}{p\circ\eft\circ\ebd}
  \\
  {}&
Y \ar{r}[swap]{g}
  &
X
\end{tikzcd}
\end{equation*}
commutative.
\end{itemize}
\end{defn}

\begin{thm}\label{cobesys-pullback}
For every diagram
\begin{equation*}
\begin{tikzcd}[column sep=large]
\stesys'
  \ar[bend right=10]{ddr}[swap]{p'}
  \ar[bend left=10]{rrd}{f}
  \ar[dotted]{dr}[near end]{[p',f]}
  \\
  {}&
\cobesys{Y}{\stesys}{g}{p}
  \ar{d}{\pullbackpr{1}{g}{p}}
  \ar{r}[swap]{\pullbackpr{2}{g}{p}}
  &
\stesys
  \ar{d}{p}
  \\
  {}&
Y \ar{r}[swap]{g}
  &
X
\end{tikzcd}
\end{equation*}
of which the outer square commutes, the pre-extension homomorphism $[p',f]$
is unique with the property that it renders the whole diagram commutative.
\end{thm}

\begin{defn}\label{famfamstesys_into}
Consider a commutative square
\begin{equation*}
\begin{tikzcd}
\stesys'
  \ar{r}{f}
  \ar{d}[swap]{p}
  &
\famesys\stesys
  \ar{d}{\eft}
  \\
\stesysf \ar{r}[swap]{\ectxext}
  &
\stesysc
\end{tikzcd}
\end{equation*}
Then we construct
\begin{equation*}
[p,f]:\stesys'\to\famesys{\famesys{\stesys}}
\end{equation*}
as follows:
\begin{itemize}
\item let $[p,f]_0:\stesysc'\to\stesysf_2$ be the unique morphism rendering
the diagram
\begin{equation*}
\begin{tikzcd}[column sep=large]
\stesysc' 
  \ar[bend left=10]{rrd}{f_0}
  \ar[swap,bend right=10]{ddr}{p}
  \ar[dotted]{dr}[near end]{[p,f]_0}
  \\
  {}&
\stesysf_2
  \ar{r}[swap]{\pullbackpr{2}{\ectxext}{\eft}}
  \ar{d}{\eft[2]}
  &
\stesysf
  \ar{d}{\eft}
  \\
  {}&
\stesysf
  \ar{r}[swap]{\ectxext}
  &
\stesysc
\end{tikzcd}
\end{equation*}
commutative.
\item Let $[p,f]_1:\stesysf'\to\stesysf_3$ be the unique morphism rendering
the diagram
\begin{equation*}
\begin{tikzcd}[column sep=large]
\stesysf'
  \ar[bend left=10]{drr}{f_1}
  \ar[swap]{dd}{\eft'}
  \ar[dotted]{dr}[near end]{[p,f]_1}
  \\
  {}&
\stesysf_3
  \ar{r}[swap]{\pullbackpr{2}{\efamext}{\eft[2]}}
  \ar{d}{\eft[2]}
  &
\stesysf_2
  \ar{d}{\eft[2]}
  \\
\stesysc'
  \ar{r}[swap]{[p,f]_0}
  &
\stesysf_2
  \ar{r}[swap]{\efamext}
  &
\stesysf
\end{tikzcd}
\end{equation*}
commutative.
\item Let $[p,f]^t:\stesyst'\to\stesyst_3$ be the unique morphism rendering
the diagram
\begin{equation*}
\begin{tikzcd}[column sep=huge]
\stesyst'
  \ar[bend left=10]{drr}{f^t}
  \ar[swap]{dd}{\eft'\circ\ebd'}
  \ar[dotted]{dr}[near end]{[p,f]^t}
  \\
  {}&
\stesyst_3
  \ar{r}[swap]{\pullbackpr{2}{\efamext}{\eft[2]\circ\ebd[2]}}
  \ar{d}[swap]{\pullbackpr{1}{\efamext}{\eft[2]\circ\ebd[2]}}
  &
\stesyst_2
  \ar{d}{\eft[2]\circ\ebd[2]}
  \\
\stesysc'
  \ar{r}[swap]{[p,f]_0}
  &
\stesysf_2
  \ar{r}[swap]{\efamext}
  &
\stesysf
\end{tikzcd}
\end{equation*}
commutative.
\end{itemize}
\end{defn}

\begin{lem}
Under the hypotheses of \autoref{famfamstesys_into}, $[p,f]$ is a pre-extension
homomorphism. Moreover, it is the unique pre-extension homomorphism for which
the diagram
\begin{equation*}
\begin{tikzcd}[column sep=large]
\stesys' 
  \ar[bend left=10]{rrd}{f}
  \ar[swap,bend right=10]{ddr}{p}
  \ar[dotted]{dr}[near end]{[p,f]}
  \\
  {}&
\famesys{\famesys{\stesys}}
  \ar{r}[swap]{\pullbackpr{2}{\ectxext}{\eft}}
  \ar{d}{\eft[2]}
  &
\famesys{\stesys}
  \ar{d}{\eft}
  \\
  {}&
\stesysf
  \ar{r}[swap]{\ectxext}
  &
\stesysc
\end{tikzcd}
\end{equation*}
commutes.
\end{lem}

\begin{lem}
Suppose $f:\stesys\to \stesys'$ is a pre-extension homomorphism and consider a morphism
$p:\stesys'\to X$ and $g:Y\to X$. Then the change of base 
$g^\ast(f):\cobesys{Y}{\stesys}{g}{p\circ f_0}\to
\cobesys{Y}{\stesys'}{g}{p}$ is a pre-extension morphism.
\end{lem}

\begin{lem}
Let $\stesys$ be a pre-extension algebra and consider $p:\stesysc\rightarrow X\leftarrow Y:g$.
Then there is an isomorphism
\begin{equation*}
\varphi:\famesys{\cobesys{Y}{\stesys}{g}{p}}
  \simeq
\cobesys{Y}{\famesys{\stesys}}{g}{p\circ\eft}
\end{equation*}
uniquely determined by
\end{lem}

\begin{proof}
This follows from the pasting lemma for pullbacks.
\end{proof}

\subsection{Pre-weakening objects}
\begin{defn}
Let $\stesys$ be an extension object in $\cat{C}$. A pre-weakening operation
on $\stesys$ is an extension homomorphism 
$ \mathbf{w}(\stesys)
    :
  \cobesys{\stesysf}{\famesys{\stesys}}{\eft}{\eft}
    \to
  \famesys{\famesys{\stesys}}$
for which the diagram
\begin{equation*}
\begin{tikzcd}[column sep=large]
\cobesys{\stesysf}{\famesys{\stesys}}{\eft}{\eft}
  \ar{r}{\mathbf{w}(\stesys)}
  \ar{dr}[swap]{\pullbackpr{1}{\eft}{\eft}}
  &
\famesys{\famesys{\stesys}}
  \ar{d}{\eft[2]}
  \\
& \stesysf
\end{tikzcd}
\end{equation*}
commutes.
\end{defn}

\begin{defn}
Let $\stesys$ be an extension object with pre-weakening operation
$\mathbf{w}(\stesys)$. Then $\famesys{\stesys}$ has the pre-weakening operation
$\mathbf{w}(\famesys{\stesys})$ which is uniquely determined by rendering the
diagram
\begin{equation*}
\begin{tikzcd}[column sep=large]
\cobesys{\stesysf_2}{\famesys{\famesys{\stesys}}}{\eft[2]}{\eft[2]}
  \ar{rr}{%
      \pullback{\beta_1}{\boldsymbol{\beta}}{\eft}{\eft}
    }
  \ar[bend right]{ddr}[swap]{\pullbackpr{1}{\eft[2]}{\eft[2]}}
  \ar[dotted]{dr}{\mathbf{w}(\famesys{\stesys})}
  &
  {}&
\cobesys{\stesysf}{\famesys{\stesys}}{\eft}{\eft}
  \ar{r}{\mathbf{w}(\stesys)}
  &
\famesys{\famesys{\stesys}}
  \ar{d}{\boldsymbol{\beta}}
  \\
  {}&
\famesys{\famesys{\famesys{\stesys}}}
  \ar{r}{\boldsymbol{\beta}_\mathbf{2}}
  \ar{d}[swap]{\eft[3]}
  &
\famesys{\famesys{\stesys}}
  \ar{d}{\eft[2]}
  \ar{r}{\boldsymbol{\beta}}
  &
\famesys{\stesys}
  \ar{d}{\eft}
  \\
  {}&
\stesysf_2
  \ar{r}[swap]{\efamext}
  &
\stesysf
  \ar{r}[swap]{\ectxext}
  &
\stesysc
\end{tikzcd}
\end{equation*}
commutative.
\end{defn}

\begin{defn}
A pre-weakening object $\stesys$ in $\cat{C}$ is an extension object $\stesys$ 
with a pre-weakening operation 
$ \mathbf{w}(\stesys)
    :
  \cobesys{\stesysf}{\famesys{\stesys}}{\eft}{\eft}
    \to
  \famesys{\famesys{\stesys}}$
for which the diagram
\begin{equation*}
\begin{tikzcd}[column sep=15em]
\cobesys{\stesysf_2}{\famesys{\stesys}}{\eft\circ\eft[2]}{\eft}
  \ar[bend right=10]{dr}[swap]%
    { [ \pullbackpr{1}{\eft\circ\eft[2]}{\eft},%
        \mathbf{w}(\stesys)%
          \circ%
        (\pullback{\efamext}{\catid{\famesys{\stesys}}}{\eft}{\eft})%
        ]%
      }
  \ar{r}{
    [ \pullbackpr{1}{\eft\circ\eft[2]}{\eft},%
      \mathbf{w}(\stesys)%
        \circ%
      (\pullback{\eft[2]}{\catid{\famesys{\stesys}}}{\eft}{\eft})%
      ]}%
  &
\cobesys{\stesysf_2}{\famesys{\famesys{\stesys}}}{\eft[2]}{\eft[2]}
  \ar{d}{\mathbf{w}(\famesys{\stesys})}
  \\
  {}&
\famesys{\famesys{\famesys{\stesys}}}
\end{tikzcd}
\end{equation*}
commutes. This condition is called \emph{Currying for weakening}.
\end{defn}

\begin{lem}
If $\stesys$ is a pre-weakening algebra, then so is $\famesys{\stesys}$. 
\end{lem}

\begin{proof}
\end{proof}

\begin{defn}
A pre-weakening morphism between preweakening objects $\stesys$ and $\stesys'$ is an
extension homomorphism $f:\stesys\to \stesys'$ such that additionally the diagram
\begin{equation*}
\begin{tikzcd}[column sep=large]
\cobesys{\stesysf}{\famesys{\stesys}}{\eft}{\eft}
  \ar{d}[swap]{\mathbf{w}(\stesys)}
  \ar{r}{\pullback{f_1}{\famehom{f}}{\eft'}{\eft'}}
  &
\cobesys{\stesysf'}{\famesys{\stesys'}}{\eft'}{\eft'}
  \ar{d}{\mathbf{w}(\stesys')}
  \\
\famesys{\famesys{\stesys}}
  \ar{r}[swap]{\famehom{\famehom{f}}}
  &
\famesys{\famesys{\stesys'}}
\end{tikzcd}
\end{equation*}
commutes.
\end{defn}

\begin{defn}
Let $\stesys$ be a pre-weakening algebra and consider $p:\stesysc\rightarrow X\leftarrow Y:p$.
Then we define
\begin{equation*}
\mathbf{w}(\cobesys{Y}{\stesys}{g}{p})
  :
\cobesys
  { (\pullback{Y}{\stesysf}{g}{p\circ\eft})}
  { \famesys{\cobesys{Y}{\stesys}{g}{p}}}
  { g^\ast(\eft)}
  { g^\ast(\eft)}
  \to
\famesys{\famesys{\cobesys{Y}{\stesys}{g}{p}}}
\end{equation*}
to be the unique extension homomorphism rendering the diagram
\begin{equation*}
\begin{tikzcd}
\cobesys
  { \pullback{Y}{\stesysf}{g}{p\circ\eft}}
  { \famesys{\cobesys{Y}{\stesys}{g}{p}}}
  { g^\ast(\eft)}
  { g^\ast(\eft)}
  \ar[bend right]{ddr}[swap]{\pullbackpr{1}{g^\ast(\eft)}{g^\ast(\eft)}}
  \ar{rr}{%
    \pullback
      { \pullbackpr{2}{g}{p\circ\eft}}
      { \boldsymbol{\pi}_\mathbf{2}(g,p\circ\eft)}
      { \eft}
      { \eft}
    }
  \ar[dotted]{dr}{\mathbf{w}(\cobesys{Y}{\stesys}{g}{p})}
  &
  {}&
\cobesys{\stesysf}{\famesys{\stesys}}{\eft}{\eft}
  \ar{r}{\mathbf{w}(\stesys)}
  &
\famesys{\famesys{\stesys}}
  \ar{d}{\boldsymbol{\beta}}
  \\
  {}&
\famesys{\famesys{\cobesys{Y}{\stesys}{g}{p}}}
  \ar{r}{\beta}
  \ar{d}{g^\ast(\eft)_1}
  &
\famesys{\cobesys{Y}{\stesys}{g}{p}}
  \ar{r}{\boldsymbol{\pi}_2(g,p\circ\eft)}
  \ar{d}{g^\ast(\eft)}
  &
\famesys{\stesys}
  \ar{d}{\eft}
  \\
  {}&
\pullback{Y}{\stesysf}{g}{p\circ\eft}
  \ar{r}[swap]{\cobesys{Y}{\ectxext}{g}{p}}
  &
\pullback{Y}{\stesysc}{g}{p}
  \ar{r}[swap]{\pullbackpr{2}{g}{p}}
  &
\stesysc
\end{tikzcd}
\end{equation*}
commutative.
\end{defn}

\subsection{Weakening objects}
Since we have shown that the property of being a pre-weakening object is closed
under the relevant operations, we can make the following definition:

\begin{defn}
A weakening object is a pre-weakening object $\stesys$ with the property that
$\mathbf{w}(\stesys)$ is a pre-weakening morphism.
\end{defn}

\begin{defn}
A weakening homomorphism is a pre-weakening homomorphism such that the domain
and codomain are weakening objects.
\end{defn}

\begin{thm}
Suppose $\stesys$ is a weakening object, then so is $\famesys{\stesys}$
\end{thm}

\begin{thm}
The change of base of any weakening object is again a weakening object.
\end{thm}

\subsection{Projection objects}
\begin{defn}
A pre-projection object is a weakening object $\stesys$ for which there is a term
$\mathbf{i}:\stesysf\to \stesyst_2$ such that the diagram
\begin{equation*}
\begin{tikzcd}[column sep=large]
\stesysf \ar{r}{\mathbf{i}} \ar{d}[swap]{\Delta_{\eft}} & \stesyst_2 \ar{d}{\ebd[2]}\\
\pullback{\stesysf}{\stesysf}{\eft}{\eft} \ar{r}[swap]{w(\stesys)_0} & \stesysf_2
\end{tikzcd}
\end{equation*}
commutes. In this diagram $\Delta_{\eft}:\stesysf\to \pullback{\stesysf}{\stesysf}{\eft}{\eft}$ is the diagonal.
\end{defn}

\begin{defn}
A pre-projection homomorphism from $\stesys$ to $\stesys'$ is a weakening homomorphism
$f:\stesys\to \stesys'$ such that the square
\begin{equation*}
\begin{tikzcd}[column sep=large]
\stesyst_2
  \ar{r}{f^t_1}
  &
\stesyst_2'
  \\
\stesysf \ar{r}[swap]{f_1}
  \ar{u}{\mathbf{i}}
  &
\stesysf'
  \ar{u}[swap]{\mathbf{i}'}
\end{tikzcd}
\end{equation*}
commutes.
\end{defn}

\begin{lem}
The change of base of a pre-projection object is again a pre-projection object.
\end{lem}

\begin{lem}
If $CFT$ is a pre-projection object, then so is $\mathbf{F}_{CFT}$, where
$\mathbf{F}_{\mathbf{i}}$ is defined to be $F\times_{e_0,c}\mathbf{i}$ is
a pre-projection algebra.
\end{lem}

\begin{defn}
A projection algebra is a pre-projection algebra for which weakening is a
pre-projection homomorphism.
\end{defn}

\begin{cor}
The change of base of a projection object is again a projection object.
\end{cor}

\begin{cor}
If $CFT$ is a projection object, then so is $\mathbf{F}_{CFT}$, where
$\mathbf{F}_{\mathbf{i}}$ is defined to be $F\times_{e_0,c}\mathbf{i}$ is
a projection algebra.
\end{cor}

\subsection{Substitution objects}

\begin{defn}
A \emph{pre-substitution} for an extension object $\stesys$ is an
extension homomorphism
\begin{equation*}
\mathbf{s}(\stesys):\cobesys{\stesyst}{\famesys{\famesys{\stesys}}}{\ebd}{\eft[2]}\to \famesys{\stesys}
\end{equation*}
for which the square
\begin{equation*}
\begin{tikzcd}[column sep=large]
\pullback{\stesyst}{\stesysf_2}{\ebd}{\eft[2]}
  \ar{r}{s(\stesys)_0}
  \ar{d}[swap]{\ebd\circ\pullbackpr{1}{\ebd}{\eft[2]}}
  &
\stesysf 
  \ar{d}{\eft}
  \\
\stesysf 
  \ar{r}[swap]{\eft}
  &
\stesysc
\end{tikzcd}
\end{equation*}
commutes. A \emph{pre-substitution object} is an extension object
together with a pre-substitution.
\end{defn}

\begin{defn}
A \emph{pre-substitution homomorphism} is an extension homomorphism $f:\stesys'\to \stesys$
for which the square
\begin{equation*}
\begin{tikzcd}[column sep=huge]
\cobesys{\stesyst'}{\famesys{\famesys{\stesys'}}}{\ebd'}{\eft[2]'}
  \ar{r}{\pullback{f^t}{\famehom{\famehom{f}}}{\ebd}{\eft[2]}}
  \ar{d}[swap]{\mathbf{s}'(\stesys')}
  &
\cobesys{\stesyst}{\famesys{\famesys{\stesys}}}{\ebd}{\eft[2]}
  \ar{d}{\mathbf{s}(\stesys)}
  \\
\famesys{\stesys'}
  \ar{r}[swap]{\famehom{f}}
  &
\famesys{\stesys}
\end{tikzcd}
\end{equation*}
commutes.
\end{defn}

\begin{lem}
The change of base of a pre-substitution object is again a pre-substitution object.
\end{lem}

\begin{lem}
If $\stesys$ is a pre-substitution object, then so is $\famesys{\stesys}$ with
$\mathbf{s}(\famesys{\stesys})$ defined to be the unique extension homomorphism
rendering the diagram
\begin{equation*}
\begin{tikzcd}
\cobesys{\stesyst_2}{\famesys{\famesys{\famesys{\stesys}}}}{\ebd[2]}{\eft[3]}
  \ar[dotted]{dr}{\mathbf{s}(\famesys{\stesys})}
  \ar{rr}{\pullback{\pullbackpr{2}{\ectxext}{\eft\circ\ebd}}{\boldsymbol{\beta}_\mathbf{2}}{\ebd}{\eft[2]}}
  \ar{dd}[swap]{\ebd[2]\circ\pullbackpr{1}{\ebd[2]}{\eft[3]}}
  &
  {}&
\cobesys{\stesyst}{\famesys{\famesys{\stesys}}}{\ebd}{\eft[2]}
  \ar{d}{\mathbf{s}(\stesys)}
  \\
  {}&
\famesys{\famesys{\stesys}}
  \ar{r}{\boldsymbol{\beta}}
  \ar{d}{\eft[2]}
  &
\famesys{\stesys}
  \ar{d}{\eft}
  \\
\stesysf_2
  \ar{r}[swap]{\eft[2]}
  &
\stesysf
  \ar{r}[swap]{\ectxext}
  &
\stesysc
\end{tikzcd}
\end{equation*}
commutative.
\end{lem}

\begin{proof}
The requirement on pre-substitutions holds by construction.
\end{proof}

It makes sense now to consider the possibility that the pre-substitution
itself is a pre-substitution homomorphism.

\begin{defn}
A \emph{substitution object} is a pre-substitution object for which substitution is
a pre-substitution homomorphism.
\end{defn}

\begin{cor}
The change of base of a substitution object is again a substitution object.
\end{cor}

\begin{cor}
If $\stesys$ is a substitution object, then so is $\famesys{\stesys}$.
\end{cor}

\subsection{Extension objects with empty context and families}

\begin{defn}
An extension object $\stesys$ is said to have \emph{empty families} if there
is a section
\begin{equation*}
\phi_1(\stesys):\stesysc\to\stesysf
\end{equation*}
of $\eft$, satisfying the following additional properties:
\begin{enumerate}
\item $\phi_1(\stesys)$ is also a section of $\ectxext$.
\item The unique morphism $\phi_2$ rendering the diagram
\begin{equation*}
\begin{tikzcd}[column sep=huge]
\stesysf
  \ar[bend left=20]{drr}{\phi_1(\stesys)\circ\ectxext}
  \ar[equals,bend right=20]{ddr}
  \ar[dotted]{dr}{\phi_2}
  \\
  {}&
\stesysf_2
  \ar{d}{\eft[2]}
  \ar{r}{\pullbackpr{2}{\ectxext}{\eft}}
  &
\stesysf
  \ar{d}{\eft}
  \\
  {}&
\stesysf
  \ar{r}[swap]{\ectxext}
  &
\stesysc
\end{tikzcd}
\end{equation*}
commutative, is a section of $\efamext$.
\item The unique morphism $\iota_1$ rendering the diagram
\begin{equation*}
\begin{tikzcd}[column sep=huge]
\stesysf
  \ar[equals,bend left=20]{drr}
  \ar[bend right=20]{ddr}[swap]{\phi_1(\stesys)\circ\eft}
  \ar[dotted]{dr}{\iota_1}
  \\
  {}&
\stesysf_2
  \ar{d}{\eft[2]}
  \ar{r}[swap]{\pullbackpr{2}{\ectxext}{\eft}}
  &
\stesysf
  \ar{d}{\eft}
  \\
  {}&
\stesysf
  \ar{r}[swap]{\ectxext}
  &
\stesysc
\end{tikzcd}
\end{equation*}
commutative, is a section of $\efamext$.
\end{enumerate}
\end{defn}

\begin{defn}
A homomorphism of extension objects with empty families is an extension
homomorphism $f:\stesys'\to\stesys$ for which the diagram
\begin{equation*}
\begin{tikzcd}
\stesysf'
  \ar{r}{f_1}
  &
\stesysf
  \\
\stesysc'
  \ar{u}{\phi_1(\stesys')}
  \ar{r}[swap]{f_0}
  &
\stesysc
  \ar{u}[swap]{\phi_1(\stesys)}
\end{tikzcd}
\end{equation*}
commutes.
\end{defn}

\begin{lem}
Suppose $\stesys$ is an extension object with empty families. Then
$\famesys{\stesys}$ is an extension object with empty families, with
$\phi_1(\famesys{\stesys}):=\phi_2$. 
\end{lem}

\begin{lem}
Suppose $\stesys$ is an extension object with empty families and consider
$p:\stesysc\rightarrow X\leftarrow Y:p$. Then
$\cobesys{Y}{\stesys}{g}{p}$ is an extension object with empty families
with $\phi_1(\cobesys{Y}{\stesys}{g}{p}):=g^\ast(\phi_1)$.
\end{lem}

\subsection{E-objects}
\begin{defn}
An \emph{E-object} is an extension object with the structure of a projection object,
the structure of a substitution object and which has an empty context and families,
such that additionally:
\begin{enumerate}
\item substitution is a projection homomorphism
\item weakening is a substitution homomorphism
\item both weakening and substitution are empty-CF homomorphisms.
\item 
\end{enumerate}
\end{defn}


\bibliographystyle{plain}
%\phantomsection\addcontentsline{toc}{section}{References}
\bibliography{References/refs}

\end{document}
