\documentclass{article}

%%%%%%%%%%%%%%%%%%%%%%%%%%%%%%%%%%%%%%%%%%%%%%%%%%%%%%%%%%%%%%%%%%%%%%%%%%%%%%%%
%%%% PACKAGES

\usepackage[utf8]{inputenc}
\usepackage[english]{babel}

%%%% Spicing up the document
\usepackage{mathpazo}
\usepackage[scaled=0.95]{helvet}
\usepackage{courier}
\linespread{1.05} % Palatino looks better with this
\usepackage{microtype}

\usepackage{fancyhdr} % To set headers and footers
\usepackage{enumitem,mathtools,xspace,xcolor}
\usepackage{comment}
\usepackage{ifthen}
\usepackage{pifont}
\newcommand{\cmark}{\ding{51}\xspace}
\newcommand{\xmark}{\ding{55}\xspace}

\usepackage{graphicx}
\usepackage{tikz-cd}
\usepackage{tikz}
\usetikzlibrary{decorations.pathmorphing}
\usepackage[inference]{semantic}
\usepackage{booktabs}

\usepackage[hyphens]{url} % This package has to be loaded *before* hyperref
\usepackage[pagebackref,colorlinks,citecolor=darkgreen,linkcolor=darkgreen,unicode]{hyperref}
\definecolor{darkgreen}{rgb}{0,0.45,0}

% For some reason the following can't be above hyperref...
\usepackage{amssymb,amsmath,amsthm,stmaryrd,mathrsfs,wasysym}
\usepackage{aliascnt}
\usepackage[capitalize]{cleveref}

% The braket macro shouldn't be necessary
\usepackage{braket} % used for \setof{ ... } macro

%%%%%%%%%%%%%%%%%%%%%%%%%%%%%%%%%%%%%%%%%%%%%%%%%%%%%%%%%%%%%%%%%%%%%%%%%%%%%%%%
%% To include references in TOC we should use this package rather than a hack.
\usepackage{tocbibind}
%\usepackage{etoolbox}           % get \apptocmd
%\apptocmd{\thebibliography}{\addcontentsline{toc}{section}{References}}{}{} % tell bibliography to get itself into the table of contents


\begin{comment}
%%%% Header and footers
\pagestyle{fancyplain}
\setlength{\headheight}{15pt}
\renewcommand{\chaptermark}[1]{\markboth{\textsc{Chapter \thechapter. #1}}{}}
\renewcommand{\sectionmark}[1]{\markright{\textsc{\thesection\ #1}}}
\end{comment}

% TOC depth
\setcounter{tocdepth}{3}

\lhead[\fancyplain{}{{\thepage}}]%
      {\fancyplain{}{\nouppercase{\rightmark}}}
\rhead[\fancyplain{}{\nouppercase{\leftmark}}]%
      {\fancyplain{}{\thepage}}
\cfoot{\textsc{\footnotesize [Draft of \today]}}
\lfoot[]{}
\rfoot[]{}

%%%%%%%%%%%%%%%%%%%%%%%%%%%%%%%%%%%%%%%%%%%%%%%%%%%%%%%%%%%%%%%%%%%%%%%%%%%%%%%%
%%%% We mostly use the macros of the book, to keep notations
%%%% and conventions the same. Recall that when the macros file
%%%% is updated, we need to comment the lines containing the
%%%% string `[chapter]` since our article is not a book.
%%%%
%%%% Instructions for updating the macros.tex file:
%%%% - fetch the latest macros.tex file from the HoTT/book git repository.
%%%% - comment all lines containing "[chapter]" because this is not a book.
%%%% - comment the definition of pbcorner because the xypic package is not used.
%%%%
\input{macros}

\newcommand{\idsymbin}{=}

%%%%%%%%%%%%%%%%%%%%%%%%%%%%%%%%%%%%%%%%%%%%%%%%%%%%%%%%%%%%%%%%%%%%%%%%%%%%%%%%
%%%% Our commands which are not part of the macros.tex file.
%%%% We should keep these commands separate, because we will
%%%% update the macros.tex following the updates of the book.

%%%% First we redefine the \id, \eqv and \ct commands so that they accept an
%%%% arbitrary number of arguments. This is useful when writing longer strings
%%%% of equalities or equivalences.

\makeatletter

\renewcommand{\id}[3][]{
  \@ifnextchar\bgroup
    {#2 \mathbin{\idsym_{#1}} \id[#1]{#3}}
    {#2 \mathbin{\idsym_{#1}} #3}
  }

\renewcommand{\eqv}[2]{
  \@ifnextchar\bgroup
    {#1 \eqvsym \eqv{#2}}
    {#1 \eqvsym #2}
  }

\newcommand{\ctsym}{%
  \mathchoice{\mathbin{\raisebox{0.5ex}{$\displaystyle\centerdot$}}}%
             {\mathbin{\raisebox{0.5ex}{$\centerdot$}}}%
             {\mathbin{\raisebox{0.25ex}{$\scriptstyle\,\centerdot\,$}}}%
             {\mathbin{\raisebox{0.1ex}{$\scriptscriptstyle\,\centerdot\,$}}}
  }

\renewcommand{\ct}[3][]{
  \@ifnextchar\bgroup
    {#2 \mathbin{\ctsym_{#1}} \ct[#1]{#3}}
    {#2 \mathbin{\ctsym_{#1}} #3}
  }

\makeatother

%%%% We always use textstyle products and sums...
%\renewcommand{\prd}{\tprd}
%\renewcommand{\sm}{\tsm}
\makeatletter
\renewcommand{\@dprd}{\@tprd}
\renewcommand{\@dsm}{\@tsm}
\renewcommand{\@dprd@noparens}{\@tprd}
\renewcommand{\@dsm@noparens}{\@tsm}

%%%% ...with a bit more spacing
\renewcommand{\@tprd}[1]{\mathchoice{{\textstyle\prod_{(#1)}\,}}{\prod_{(#1)}\,}{\prod_{(#1)}\,}{\prod_{(#1)}\,}}
\renewcommand{\@tsm}[1]{\mathchoice{{\textstyle\sum_{(#1)}\,}}{\sum_{(#1)}\,}{\sum_{(#1)}\,}{\sum_{(#1)}\,}}

%%%%%%%%%%%%%%%%%%%%%%%%%%%%%%%%%%%%%%%%%%%%%%%%%%%%%%%%%%%%%%%%%%%%%%%%%%%%%%%%
%%%% We adjust the \prd command so that implicit arguments become possible.
%%%%
%%%% First, we have the following switch. Set it to true if implicit arguments
%%%% are desired, or to false if not. Note turning off implicit arguments
%%%% might render some parts of the text harder to comprehend, since in the
%%%% text might appear $f(x)$ where we would have $f(i,x)$ without implicit
%%%% arguments.

\newcommand{\implicitargumentson}{\boolean{true}}

%%%% If one wants to use implicit arguments in the notation for product types,
%%%% a * has to be put before the argument that has to be implicit.
%%%% For example: in $\prd{x:A}*{y:B}{u:P(y)}Q(x,y,u)$, the argument y is
%%%% implicit. Any of the arguments can be made implicit this way.

%%%% First of all, we should make the command \prd search not only for a
%%%% brace, but also for a star. We introduce an auxiliary command that
%%%% determines whether the next character is a star or brace.
\newcommand{\@ifnextchar@starorbrace}[2]
%  {\@ifnextcharamong{#1}{#2}{*}{\bgroup};}
  {\@ifnextchar*{#1}{\@ifnextchar\bgroup{#1}{#2}}}
  
%%%% When encountering the \prd command, latex should determine whether it
%%%% should print implicit argument brackets or not. So the first branching
%%%% happens right here.
\renewcommand{\prd}{\@ifnextchar*{\@iprd}{\@prd}}

\newcommand{\@prd}[1]
  {\@ifnextchar@starorbrace
    {\prd@parens{#1}}
    {\@ifnextchar\sm{\prd@parens{#1}\@eatsm}{\prd@noparens{#1}}}}
\newcommand{\@prd@parens}{\@ifnextchar*{\@iprd}{\prd@parens}}
\renewcommand{\prd@parens}[1]
  {\@ifnextchar@starorbrace
    {\@theprd{#1}\@prd@parens}
    {\@ifnextchar\sm{\@theprd{#1}\@eatsm}{\@theprd{#1}}}}
\newcommand{\@theprd}[1]
  {\mathchoice{\@dprd{#1}}{\@tprd{#1}}{\@tprd{#1}}{\@tprd{#1}}}
\renewcommand{\dprd}[1]{\@dprd{#1}\@ifnextchar@starorbrace{\dprd}{}}
\renewcommand{\tprd}[1]{\@tprd{#1}\@ifnextchar@starorbrace{\tprd}{}}

%%%% Here we tell the actual symbols to be printed.
\newcommand{\@theiprd}[1]{\mathchoice{\@diprd{#1}}{\@tiprd{#1}}{\@tiprd{#1}}{\@tiprd{#1}}}
\newcommand{\@iprd}[2]{\@ifnextchar@starorbrace%
  {\@theiprd{#2}\@prd@parens}%
  {\@ifnextchar\sm%
    {\@theiprd{#2}\@eatsm}%
    {\@theiprd{#2}}}}
\def\@tiprd#1{
  \ifthenelse{\implicitargumentson}
    {\@@tiprd{#1}\@ifnextchar\bgroup{\@tiprd}{}}
    {\@tprd{#1}}}
\def\@@tiprd#1{\mathchoice{{\textstyle\prod_{\{#1\}}\,}}{\prod_{\{#1\}}\,}{\prod_{\{#1\}}\,}{\prod_{\{#1\}}\,}}
\def\@diprd{
  \ifthenelse{\implicitargumentson}
    {\@tiprd}
    {\@tprd}}
    

%%%% And finally we need to redefine \@eatprd so that implicit arguments also
%%%% works in the scope of a dependent sum.    
\def\@eatprd\prd{\@prd@parens}

\makeatother

%%%%%%%%%%%%%%%%%%%%%%%%%%%%%%%%%%%%%%%%%%%%%%%%%%%%%%%%%%%%%%%%%%%%%%%%%%%%%%%%
%%%% Redefining the quantifiers, so that some of the longer 
%%%% formulas appear one a single line without problems

%%% Dependent products written with \forall, in the same style
\makeatletter
\def\tfall#1{\forall_{(#1)}\@ifnextchar\bgroup{\,\tfall}{\,}}
\renewcommand{\fall}{\tfall}

%%% Existential quantifier %%%
\def\texis#1{\exists_{(#1)}\@ifnextchar\bgroup{\,\texis}{\,}}
\renewcommand{\exis}{\texis}

%%% Unique existence %%%
\def\uexis#1{\exists!_{(#1)}\@ifnextchar\bgroup{\,\uexis}{\,}}
\makeatother

%%%%%%%%%%%%%%%%%%%%%%%%%%%%%%%%%%%%%%%%%%%%%%%%%%%%%%%%%%%%%%%%%%%%%%%%%%%%%%%%
%%%% UNFOLD
%%%%
%%%% For each definition in the type theory we make two versions of the macro:
%%%% the macro introducing the new notation and an @unfold version of the macro
%%%% which outputs the meaning of that new notation. Thus, we can use the
%%%% following construction to write our text. When we introduce \macro, we can
%%%% write \unfold{\macro} and the output will be the result of \macro@unfold.

\makeatletter
\newcommand{\unfold}{%
  \unfoldnext}
\newcommand{\unfoldall}[1]{%
  \begingroup%
  \renewcommand{\jhom}{\jhom@unfold}%
  \renewcommand{\jhomeq}{\jhomeq@unfold}%
  \renewcommand{\jhomdefn}{\jhomdefn@unfold}%
  \renewcommand{\jfhom}{\jfhom@unfold}%
  \renewcommand{\jcomp}{\jcomp@unfold}%
  \renewcommand{\@jcomp@nested}{\@jcomp@unfold@nested}%
  \renewcommand{\@jcomp@parens}{\@jcomp@unfold@parens}%
  \renewcommand{\tmext}{\tmext@unfold}%
  \renewcommand{\@tmext@nested}{\@tmext@unfold@nested}%
  \renewcommand{\@tmext@parens}{\@tmext@unfold@parens}%
  \renewcommand{\cprojfstf}{\cprojfstf@unfold}%
  \renewcommand{\cprojfst}{\cprojfst@unfold}%
  \renewcommand{\cprojsndf}{\cprojsndf@unfold}%
  \renewcommand{\cprojsnd}{\cprojsnd@unfold}%
  \renewcommand{\jfcomp}{\jfcomp@unfold}%
%  \renewcommand{\@jfcomp@nested}{\@jfcomp@unfold@nested}%
%  \renewcommand{\@jfcomp@parens}{\@jfcomp@unfold@parens}%
  \renewcommand{\sandwich}{\sandwich@unfold}%
  \renewcommand{\finc}{\finc@unfold}%
  \renewcommand{\jvcomp}{\jvcomp@unfold}%
  \renewcommand{\subst@type@unfold}[1]{
    \@ifnextchar\cprojfstf{\@eatdo{\cprojfstf@parens}}{%
      ##1}
    }
  #1%
  \endgroup%
  }

%%%% The following command is useful when you have checked with '\@ifnextchar'
%%%% that the next character is a macro '\firstmacro' and you want to replace
%%%% it by '\secondmacro'. To establish this, simply call for
%%%% '\@ifnextchar\firstmacro{\@eatdo{\secondmacro}}{}' with the second 
%%%% argument of \@eatdo left unspecified.
\newcommand{\@eatdo}[2]{#1}

%%%% The intention of '\unfoldnext' is to unfold only the definition of the
%%%% next character, provided that it is in the list of unfoldable macros.
\newcommand{\unfoldnext}[1]{
  \@ifnextchar\jhom{\@eatdo{\jhom@unfold}}{%
  \@ifnextchar\jhomeq{\@eatdo{\jhomeq@unfold}}{%
  \@ifnextchar\jhomdefn{\@eatdo{\jhomdefn@unfold}}{%
  \@ifnextchar\jfhom{\@eatdo{\jfhom@unfold}}{%
  \@ifnextchar\jcomp{\@eatdo{\jcomp@unfold}}{%
  \@ifnextchar\@jcomp@nested{\@eatdo{\@jcomp@unfold@nested}}{%
  \@ifnextchar\@jcomp@parens{\@eatdo{\@jcomp@unfold@parens}}{%
  \@ifnextchar\tmext{\@eatdo{\tmext@unfold}}{%
  \@ifnextchar\@tmext@nested{\@eatdo{\@tmext@unfold@nested}}{%
  \@ifnextchar\@tmext@parens{\@eatdo{\@tmext@unfold@parens}}{%
  \@ifnextchar\cprojfstf{\@eatdo{\cprojfstf@unfold}}{%
  \@ifnextchar\cprojfst{\@eatdo{\cprojfst@unfold}}{%
  \@ifnextchar\cprojsndf{\@eatdo{\cprojsndf@unfold}}{%
  \@ifnextchar\cprojsnd{\@eatdo{\cprojsnd@unfold}}{%
  \@ifnextchar\jfcomp{\@eatdo{\jfcomp@unfold}}{%
%  \@ifnextchar\@jfcomp@nested{\@eatdo{\@jfcomp@unfold@nested}}{%
%  \@ifnextchar\@jfcomp@parens{\@eatdo{\@jfcomp@unfold@parens}}{%
  \@ifnextchar\sandwich{\@eatdo{\sandwich@unfold}}{%
  \@ifnextchar\finc{\@eatdo{\finc@unfold}}{%
  \@ifnextchar\jvcomp{\@eatdo{\jvcomp@unfold}}}
  #1}
\makeatother

%%%%%%%%%%%%%%%%%%%%%%%%%%%%%%%%%%%%%%%%%%%%%%%%%%%%%%%%%%%%%%%%%%%%%%%%%%%%%%%%
%%%% A PRETTY PRINTER
%%%%
%%%% We write a \pretty command that pretty prints judgments or types by
%%%% diplaying variables and omitting explicit notation for weakening.
%%%%
%%%% This command should work similar to the \unfold command
%%%%
%%%% -- UNDER CONSTRUCTION

\makeatletter
\newcommand{\vardis}[2]{\@vardis@type #2{}(\@vardis@term #1)}
\newcommand{\@vardis}{\@ifnextchar\bgroup{\@@vardis}{}}
\newcommand{\@@vardis}[1]{\@ifnextchar\bgroup{\vardis{#1}}{#1}}
\newcommand{\@vardis@term}{\@vardis}
\newcommand{\@vardis@type}{\@ifnextchar\ctxext{\@ctxext@nested}{\@ifnextchar\ctxwk{\@ctxwk@nested}{\@vardis}}}
\newcommand{\@vardis@nested}[3]{\@vardis@parens{#2}{#3}}
\newcommand{\@vardis@parens}[2]{(\vardis{#1}{#2})}
\makeatother

\makeatletter
\newcommand{\jvctx}{\jctx}
\newcommand{\jvctxeq}{\jctxeq}

\newcommand{\cctxextcombi}[2]{\@ifnextchar\bgroup{\@cctxextcombi #1}{#1:}#2}
\newcommand{\@cctxextcombi}[4]{\cctxext{{\cctxextcombi{#1}{#3}}{\@@cctxextcombi{#1}{#2}{#4}}}}
\newcommand{\@@cctxextcombi}[3]{\@ifnextchar\bgroup{\@@@ctxextcombi #2}{#2(#1):}#3(\cctxext{#1})}
\newcommand{\@@@ctxextcombi}[8] % the 5th argument is (, the 6th is \cctxext and the 8th is ).
  {\@@ctxextcombi{#7}{#1}{#3},\@@ctxextcombi{{#7}{#3}}{#2}{#4}}
\newcommand{\cctxext}[1]{\@ifnextchar\bgroup{\@cctxext}{}#1}
\newcommand{\@cctxext}[2]{\cctxext{#1},\cctxext{#2}}

\newcommand{\jvfamcombi}[3]{
  \cctxextcombi{#1}{#2} \vdash \vardis{\cctxext{#1}}{#3}
}

\newcommand{\jvfam}{\@ifnextchar*{\@jvfamAlignTrue}{\@jvfamAlignFalse}}
\newcommand{\@jvfamAlignTrue}[4]{\jfam*{#2:#3}{\vardis{#2}{#4}}}
\newcommand{\@jvfamAlignFalse}[3]{\jfam{#1:#2}{\vardis{#1}{#3}}\quad@test}

\newcommand{\jvfameq}{\@ifnextchar*{\@jvfameqAlignTrue}{\@jvfameqAlignFalse}}
\newcommand{\@jvfameqAlignTrue}[5]{\jfameq*{#2:#3}{\vardis{#2}{#4}}{\vardis{#2}{#5}}}
\newcommand{\@jvfameqAlignFalse}[4]{\jfameq{#1:#2}{\vardis{#1}{#3}}{\vardis{#1}{#4}}\quad@test}

\newcommand{\jvtype}{\@ifnextchar*{\@jvtypeAlignTrue}{\@jvtypeAlignFalse}}
\newcommand{\@jvtypeAlignTrue}[4]{\jtype*{#2:#3}{\vardis{#2}{#4}}}
\newcommand{\@jvtypeAlignFalse}[3]{\jtype{#1:#2}{\vardis{#1}{#3}}\quad@test}

\newcommand{\jvtypeeq}{\@ifnextchar*{\@jvtypeeqAlignTrue}{\@jvtypeeqAlignFalse}}
\newcommand{\@jvtypeeqAlignTrue}[5]{\jtypeeq*{#2:#3}{\vardis{#2}{#4}}{\vardis{#2}{#5}}}
\newcommand{\@jvtypeeqAlignFalse}[4]{\jtypeeq{#1:#2}{\vardis{#1}{#3}}{\vardis{#1}{#4}}\quad@test}

\newcommand{\jvterm}{\@ifnextchar*{\@jvtermAlignTrue}{\@jvtermAlignFalse}}
\newcommand{\@jvtermAlignTrue}[5]{\jterm*{#2:#3}{\vardis{#2}{#4}}{\vardis{#2}{#5}}}
\newcommand{\@jvtermAlignFalse}[4]{\jterm{#1:#2}{\vardis{#1}{#3}}{\vardis{#1}{#4}}\quad@test}

\newcommand{\jvtermeq}{\@ifnextchar*{\@jvtermeqAlignTrue}{\@jvtermeqAlignFalse}}
\newcommand{\@jvtermeqAlignTrue}[6]{\jtermeq*{#2:#3}{\vardis{#2}{#4}}{\vardis{#2}{#5}}{\vardis{#2}{#6}}}
\newcommand{\@jvtermeqAlignFalse}[5]{\jtermeq{#1:#2}{\vardis{#1}{#3}}{\vardis{#1}{#4}}{\vardis{#1}{#5}}\quad@test}
\makeatother

%%%%%%%%%%%%%%%%%%%%%%%%%%%%%%%%%%%%%%%%%%%%%%%%%%%%%%%%%%%%%%%%%%%%%%%%%%%%%%%%
%%%%

\newcommand{\famsym}{\mathcal{F}}
\newcommand{\tmsym}{\mathcal{T}}

%%%%%%%%%%%%%%%%%%%%%%%%%%%%%%%%%%%%%%%%%%%%%%%%%%%%%%%%%%%%%%%%%%%%%%%%%%%%%%%%
%%%% JUDGMENTS
%%%%
%%%% Below we define several commands for the judgments of type theory. There
%%%% are commands
%%%% * \jctx for the judgment that something is a context.
%%%% * \jctxeq for the judgment that two contexts are the same
%%%% * \jtype for the judgment that something is a type in a context
%%%% * \jtypeeq for the judgment that two types in the same context are the same
%%%% * \jterm for the judgment that something is a term of a type in a context
%%%% * \jtermeq for the judgment that two terms of the same type are the same

\makeatletter
% We first make a generic judgment command
\newcommand{\judgment}{\@ifnextchar*{\@judgmentAT}{\@judgmentAF}}
\newcommand{\@judgmentAT}[8]{\@judgment@ctx{#2} & \vdash \@judgment@rel{#3}{#4}{#5}{#6}{#7} #8}
\newcommand{\@judgmentAF}[7]{\@judgment@ctx{#1} \vdash \@judgment@rel{#2}{#3}{#4}{#5}{#6} #7\quad@test}
\newcommand{\@judgment@ctx}{\@judgment@ext}
\newcommand{\@judgment@rel}[5]{
  { \default@ctxext #1
    }
  #2 
  { \default@ctxext #3
    }
  #4
  { \default@ctxext #5
    }}
\newcommand{\@judgment@kind}[1]{~~\textit{#1}}
\newcommand{\@judgment@ext}[1]{\default@ctxext #1}

\newcommand{\quad@test}{%
  \@ifnextchar\jctx{\quad}{%
  \@ifnextchar\jctxeq{\quad}{%
  \@ifnextchar\jvctx{\quad}{%
  \@ifnextchar\jvctxeq{\quad}{%
  \@ifnextchar\jfam{\quad}{%
  \@ifnextchar\jfameq{\quad}{%
  \@ifnextchar\jvfam{\quad}{%
  \@ifnextchar\jvfameq{\quad}{%
  \@ifnextchar\jtype{\quad}{%
  \@ifnextchar\jtypeeq{\quad}{%
  \@ifnextchar\jvtype{\quad}{%
  \@ifnextchar\jvtypeeq{\quad}{%
  \@ifnextchar\jterm{\quad}{%
  \@ifnextchar\jtermeq{\quad}{%
  \@ifnextchar\jvterm{\quad}{%
  \@ifnextchar\jvtermeq{\quad}{%
  \@ifnextchar\jhom{\quad}{%
  \@ifnextchar\jhomeq{\quad}{%
  \@ifnextchar\jfhom{\quad}{%
  \@ifnextchar\jfhomeq{\quad}{%
  }}}}}}}}}}}}}}}}}}}}}

%%%% Judgments about contexts
\newcommand{\jctx@sym}{\@judgment@kind{ctx}}

\newcommand{\jctx}{\@ifnextchar*{\@jctxAlignTrue}{\@jctxAlignFalse}}
\newcommand{\@jctxAlignTrue}[2]{\judgment*{}{}{}{}{}{#2}{\jctx@sym}}
\newcommand{\@jctxAlignFalse}[1]{\judgment{}{}{}{}{}{#1}{\jctx@sym}}

\newcommand{\jctxeq}{\@ifnextchar*{\@jctxeqAlignTrue}{\@jctxeqAlignFalse}}
\newcommand{\@jctxeqAlignTrue}[3]{\judgment*{}{#2}{\jdeq}{#3}{}{}{\jctx@sym}}
\newcommand{\@jctxeqAlignFalse}[2]{\judgment{}{#1}{\jdeq}{#2}{}{}{\jctx@sym}}

\newcommand{\jctxdefn}{\@ifnextchar*{\@jctxdefnAlignTrue}{\@jctxdefnAlignFalse}}
\newcommand{\@jctxdefnAlignTrue}[3]{\judgment*{}{#2}{\defeq}{#3}{}{}{\jctx@sym}}
\newcommand{\@jctxdefnAlignFalse}[2]{\judgment{}{#1}{\defeq}{#2}{}{}{\jctx@sym}}

%%%% Judgments about families
\newcommand{\jfam@sym}{\@judgment@kind{fam}}

\newcommand{\jfam}{\@ifnextchar*{\@jfamAlignTrue}{\@jfamAlignFalse}}
\newcommand{\@jfamAlignTrue}[3]{\judgment*{#2}{}{}{}{}{#3}{\jfam@sym}}
\newcommand{\@jfamAlignFalse}[2]{\judgment{#1}{}{}{}{}{#2}{\jfam@sym}}

\newcommand{\jfameq}{\@ifnextchar*{\@jfameqAlignTrue}{\@jfameqAlignFalse}}
\newcommand{\@jfameqAlignTrue}[4]{\judgment*{#2}{#3}{\jdeq}{#4}{}{}{\jfam@sym}}
\newcommand{\@jfameqAlignFalse}[3]{\judgment{#1}{#2}{\jdeq}{#3}{}{}{\jfam@sym}}

\newcommand{\jfamdefn}{\@ifnextchar*{\@jfamdefnAlignTrue}{\@jfamdefnAlignFalse}}
\newcommand{\@jfamdefnAlignTrue}[4]{\judgment*{#2}{#3}{\defeq}{#4}{}{}{\jfam@sym}}
\newcommand{\@jfamdefnAlignFalse}[3]{\judgment{#1}{#2}{\defeq}{#3}{}{}{\jfam@sym}}
  
%%%% Judgments about types
\newcommand{\jtype@sym}{\@judgment@kind{type}}
\newcommand{\jtype}{\@ifnextchar*{\@jtypeAlignTrue}{\@jtypeAlignFalse}}
\newcommand{\@jtypeAlignTrue}[3]{\judgment*{#2}{}{}{}{}{#3}{\jtype@sym}}
\newcommand{\@jtypeAlignFalse}[2]{\judgment{#1}{}{}{}{}{#2}{\jtype@sym}}
  
\newcommand{\jtypeeq}{\@ifnextchar*{\@jtypeeqAlignTrue}{\@jtypeeqAlignFalse}}
\newcommand{\@jtypeeqAlignTrue}[4]{\judgment*{#2}{#3}{\jdeq}{#4}{}{}{\jtype@sym}}
\newcommand{\@jtypeeqAlignFalse}[3]{\judgment{#1}{#2}{\jdeq}{#3}{}{}{\jtype@sym}}

\newcommand{\jtypedefn}{\@ifnextchar*{\@jtypedefnAlignTrue}{\@jtypedefnAlignFalse}}
\newcommand{\@jtypedefnAlignTrue}[4]{\judgment*{#2}{#3}{\defeq}{#4}{}{}{\jtype@sym}}
\newcommand{\@jtypedefnAlignFalse}[3]{\judgment{#1}{#2}{\defeq}{#3}{}{}{\jtype@sym}}
  
%%%% Judgments about terms
\newcommand{\jterm}{\@ifnextchar*{\@jtermAlignTrue}{\@jtermAlignFalse}}
\newcommand{\@jtermAlignTrue}[4]{\judgment*{#2}{}{}{#4}{:}{#3}{}}
\newcommand{\@jtermAlignFalse}[3]{\judgment{#1}{}{}{#3}{:}{#2}{}}

\newcommand{\jtermeq}{\@ifnextchar*{\@jtermeqAlignTrue}{\@jtermeqAlignFalse}}
\newcommand{\@jtermeqAlignTrue}[5]{\judgment*{#2}{#4}{\jdeq}{#5}{:}{#3}{}}
\newcommand{\@jtermeqAlignFalse}[4]{\judgment{#1}{#3}{\jdeq}{#4}{:}{#2}{}}

\newcommand{\jtermdefn}{\@ifnextchar*{\@jtermdefnAlignTrue}{\@jtermdefnAlignFalse}}
\newcommand{\@jtermdefnAlignTrue}[5]{\judgment*{#2}{#4}{\defeq}{#5}{:}{#3}{}}
\newcommand{\@jtermdefnAlignFalse}[4]{\judgment{#1}{#3}{\defeq}{#4}{:}{#2}{}}
\makeatother

%%%%%%%%%%%%%%%%%%%%%%%%%%%%%%%%%%%%%%%%%%%%%%%%%%%%%%%%%%%%%%%%%%%%%%%%%%%%%%%%
%%%% THE EMPTY CONTEXT

\newcommand{\emptysym}{[\;]}
\newcommand{\emptyc}{{\emptysym}}
\newcommand{\emptyf}[1][]{{\emptysym}_{#1}}
\newcommand{\emptytm}[1][]{\typefont{\#}_{#1}}

%%%%%%%%%%%%%%%%%%%%%%%%%%%%%%%%%%%%%%%%%%%%%%%%%%%%%%%%%%%%%%%%%%%%%%%%%%%%%%%%
%%%% CONTEXT EXTENSION 
%%%%
%%%% The context extension command.
%%%%
%%%% To get a feeling of how the command works, here are a few examples.
%%%% \ctxext{A}{B} will print A.B
%%%% \ctxext{{A}{B}}{C} will print (A.B).C
%%%% \ctxext{{{A}{B}}{C}}{{D}{E}} will print ((A.B).C).(D.E)

\makeatletter
\newcommand{\ctxext}[2]{\@ctxext@ctx #1.\@ctxext@type #2}
\newcommand{\@ctxext}{\@ifnextchar\bgroup{\@@ctxext}{}}
\newcommand{\@ctxext@ctx}{%
  \@ifnextchar\ctxext{\@ctxext@nested}{%
  \@ifnextchar\ctxwk{\@ctxwk@nested}{%
  \@ifnextchar\jcomp{\@jcomp@nested}{%
  \@ifnextchar\jvcomp{\@jvcomp@nested}{%
  \@ifnextchar\jfcomp{\@jfcomp@nested}{%
  \@ctxext}}}}}}
\newcommand{\@ctxext@type}{%
  \@ifnextchar\ctxext{\@ctxext@nested}{%
  \@ifnextchar\subst{\@subst@nested}{%
  \@ifnextchar\jcomp{\@jcomp@nested}{%
  \@ifnextchar\jvcomp{\@jvcomp@nested}{%
  \@ifnextchar\jfcomp{\@jfcomp@nested}{%
  \@ctxext}}}}}}
\newcommand{\@@ctxext}[1]{\@ifnextchar\bgroup{\@ctxext@parens{#1}}{#1}}
\newcommand{\@ctxext@parens}[2]{(\ctxext{#1}{#2})}
\newcommand{\@ctxext@nested}[3]{\@ctxext@parens{#2}{#3}}

%%%% We want that some commands accept binary trees as arguments that default
%%%% into extensions. We make the following command to realize this
\newcommand{\default@ctxext}{\@ifnextchar\bgroup{\ctxext}{}}
\newcommand{\default@ctxext@parens}{\@ifnextchar\bgroup{\@ctxext@parens}{}}
\makeatother

%%%%%%%%%%%%%%%%%%%%%%%%%%%%%%%%%%%%%%%%%%%%%%%%%%%%%%%%%%%%%%%%%%%%%%%%%%%%%%%%
%%%% SUBSTITUTION

%%%% The substitution command will act the following way
%%%%
%%%% \subst{x}{P} will print P[x]
%%%% \subst{x}{{f}{Q}} will print Q[f][x]
%%%% \subst{{x}{f}}{{x}{Q}} will print Q[x][f[x]]

\makeatletter
\newcommand{\subst}[3][]{%
  \@subst@type #3{}[\@subst@term #2]^{#1}}
\newcommand{\@subst}{%
  \@ifnextchar\bgroup{\@@subst}{}}
\newcommand{\@@subst}[1]{%
  \@ifnextchar\bgroup{\subst{#1}}{#1}}
\newcommand{\@subst@term}{%
  \@subst}
\newcommand{\@subst@type}{%
  \@ifnextchar\ctxext{\@ctxext@nested}{%
  \@ifnextchar\ctxwk{\@ctxwk@nested}{%
  \@ifnextchar\jcomp{\@jcomp@nested}{%
  \@ifnextchar\tmext{\@tmext@nested}{%
  \@ifnextchar\jvcomp{\@jvcomp@nested}{%
  \@ifnextchar\jfcomp{\@jfcomp@nested}{%
%  \@ifnextchar\mfam{\@mfam@nested}{%
%  \@ifnextchar\mtm{\@mtm@nested}}
\newcommand{\subst@type@unfold}[1]{#1}
\newcommand{\@subst@nested}[3]{%
  \@subst@parens{#2}{#3}}
\newcommand{\@subst@parens}[2]{%
  (\subst{#1}{#2})}
\makeatother

%%%%%%%%%%%%%%%%%%%%%%%%%%%%%%%%%%%%%%%%%%%%%%%%%%%%%%%%%%%%%%%%%%%%%%%%%%%%%%%%
%%%% WEAKENING

%%%% The weakening command is very much like the substitution command.

\makeatletter
\newcommand{\ctxwk}[3][]{%
  \langle\@ctxwk@act #2\rangle^{#1} \@ctxwk@pass #3}
\newcommand{\@ctxwk}{%
  \@ifnextchar\bgroup{\@@ctxwk}{}}
\newcommand{\@@ctxwk}[1]{%
  \@ifnextchar\bgroup{\ctxwk{#1}}{#1}}
\newcommand{\@ctxwk@act}{%
  \@ctxwk}
\newcommand{\@ctxwk@pass}{%
  \@ifnextchar\ctxext{\@ctxext@nested}{%
  \@ifnextchar\subst{\@subst@nested}{%
  \@ifnextchar\jcomp{\@jcomp@nested}{%
  \@ifnextchar\tmext{\@tmext@nested}{%
  \@ifnextchar\jvcomp{\@jvcomp@nested}{%
  \@ifnextchar\jfcomp{\@jfcomp@nested}{%
%  \@ifnextchar\mfam{\@mfam@nested}{%
%  \@ifnextchar\mtm{\@mtm@nested}}
\newcommand{\@ctxwk@parens}[2]{%
  (\ctxwk{#1}{#2})}
\newcommand{\@ctxwk@nested}[3]{%
  \@ctxwk@parens{#2}{#3}}
\makeatother

%%%% Not sure if we're gonna need the following.
\newcommand{\ctxwkop}[2]{%
  \ctxwk{#2}{#1}}
  
%%%%%%%%%%%%%%%%%%%%%%%%%%%%%%%%%%%%%%%%%%%%%%%%%%%%%%%%%%%%%%%%%%%%%%%%%%%%%%%%
%%%% IDENTITY TERMS

\makeatletter
\newcommand{\idtm}[1]{\typefont{id}_{\default@ctxext #1}}
\makeatother

%%%%%%%%%%%%%%%%%%%%%%%%%%%%%%%%%%%%%%%%%%%%%%%%%%%%%%%%%%%%%%%%%%%%%%%%%%%%%%%%
%%%% TERM EXTENSION
%%%%
%%%% The term extension command \tmext is slightly complicated because 
%%%% \tmext@unfold should do different things depending on whether it has two
%%%% or four arguments. Thus \tmext has a full form and a short form, where
%%%% the short form has two arguments and the full form has four. 

\makeatletter

%%%% The basic term extension commands
\newcommand{\default@tmext}{\@ifnextchar\bgroup{\tmext}{}}
\newcommand{\tmext}[2]{%
  \@ifnextchar\bgroup{\tmext@full{#1}{#2}}{\tmext@short{#1}{#2}}}
\newcommand{\tmext@full}[4]{%
  \ctxext{\tmext@testleft #3}{\tmext@testright #4}}
\newcommand{\tmext@short}[2]{%
  \ctxext{\tmext@testleft #1}{\tmext@testright #2}}
\newcommand{\tmext@testleft}{%
  \@ifnextchar\bgroup{\@tmext@parens}{%
  \@ifnextchar\tmext{\@tmext@nested}{%
  \@ifnextchar\ctxwk{\@ctxwk@nested}{%
  \@ifnextchar\jcomp{\@jcomp@nested}{%
  \@ifnextchar\jvcomp{\@jvcomp@nested}{%
  \@ifnextchar\jfcomp{\@jfcomp@nested}{%
%  \default@tmext
  }}}}}}}
\newcommand{\tmext@testright}{%
  \@ifnextchar\bgroup{\@tmext@parens}{%
  \@ifnextchar\tmext{\@tmext@nested}{%
  \@ifnextchar\subst{\@subst@nested}{%
  \@ifnextchar\jcomp{\@jcomp@nested}{%
  \@ifnextchar\jvcomp{\@jvcomp@nested}{%
  \@ifnextchar\jfcomp{\@jfcomp@nested}{%
  \@ifnextchar\cprojfst{\cprojfst@nested}{%
  \@ifnextchar\cprojsnd{\cprojsnd@nested}{%
%  \default@tmext
  }}}}}}}}}
\newcommand{\@tmext@nested}[1]{%
  \@tmext@parens}
\newcommand{\@tmext@parens}[2]{%
  \@ifnextchar\bgroup
    {\tmext@full@parens{#1}{#2}}
    {(\tmext@short{#1}{#2})}}
\newcommand{\tmext@full@parens}[4]{%
  (\tmext@full{#1}{#2}{#3}{#4})}

%%%% The unfolded term extension commands
\newcommand{\tmext@unfold}[2]{%
  \@ifnextchar\bgroup{\tmext@unfold@full{#1}{#2}}{\tmext@short{#1}{#2}}}
\newcommand{\tmext@unfold@full}[4]{%  
  \subst{#4}{{#3}{\idtm{\ctxext{#1}{#2}}}}}
\newcommand{\@tmext@unfold@nested}[1]{%
  \@tmext@unfold@parens}
\newcommand{\@tmext@unfold@parens}[4]{%
  (\tmext@unfold{#1}{#2}{#3}{#4})}
\makeatother

%%%%%%%%%%%%%%%%%%%%%%%%%%%%%%%%%%%%%%%%%%%%%%%%%%%%%%%%%%%%%%%%%%%%%%%%%%%%%%%%
%%%% JUDGMENTAL MORPHISMS

\makeatletter

%%%% The judgment that f is a morphism from A to B in context \Gamma.
\newcommand{\jhomsym}[3][]{%
  ~~\textit{hom}_{#1}(\default@ctxext #2,\default@ctxext #3)}
\newcommand{\jhom}{%
  \@ifnextchar*{\@jhomAlignTrue}{\@jhomAlignFalse}}
\newcommand{\@jhomAlignTrue}[5]{%
  \judgment*{#2}{}{}{#5}{}{}{\jhomsym{#3}{#4}}}
\newcommand{\@jhomAlignFalse}[4]{%
  \judgment{#1}{}{}{#4}{}{}{\jhomsym{#2}{#3}}}
\newcommand{\jhomeq}{%
  \@ifnextchar*{\@jhomeqAlignTrue}{\@jhomeqAlignFalse}}
\newcommand{\@jhomeqAlignTrue}[6]{%
  \judgment*{#2}{#5}{\jdeq}{#6}{}{}{\jhomsym{#3}{#4}}}
\newcommand{\@jhomeqAlignFalse}[5]{%
  \judgment{#1}{#4}{\jdeq}{#5}{}{}{\jhomsym{#2}{#3}}}
\newcommand{\jhomdefn}{%
  \@ifnextchar*{\@jhomdefnAlignTrue}{\@jhomdefnAlignFalse}}
\newcommand{\@jhomdefnAlignTrue}[6]{%
  \judgment*{#2}{#5}{\defeq}{#6}{}{}{\jhomsym{#3}{#4}}}
\newcommand{\@jhomdefnAlignFalse}[5]{%
  \judgment{#1}{#4}{\defeq}{#5}{}{}{\jhomsym{#2}{#3}}}

\newcommand{\jhom@unfold}[4]{%
  \jterm
    {{#1}{#2}}
    {\ctxwk{\default@ctxext #2}{\default@ctxext@parens #3}}
    {#4}}
\newcommand{\jhomeq@unfold}[5]{%
  \jtermeq
    {{#1}{#2}}
    {\ctxwk{\default@ctxext #2}{\default@ctxext@parens #3}}
    {#4}
    {#5}}
\newcommand{\jhomdefn@unfold}[5]{%
  \jtermdefn
    {{#1}{#2}}
    {\ctxwk{\default@ctxext #2}{\default@ctxext@parens #3}}
    {#4}
    {#5}}

%%%% Composition of morphisms
\newcommand{\jcomp}[3]{%
  \jcomp@testleft #3 \circ \jcomp@testright #2}
\newcommand{\jcomp@testleft}{%
  \@ifnextchar\jcomp{\@jcomp@nested}{%
  \@ifnextchar\ctxwk{\@ctxwk@nested}{%
  \@ifnextchar\ctxext{\@ctxext@nested}{%
  \@ifnextchar\bgroup{\@jcomp@parens}{%
  \@ifnextchar\tmext{\@tmext@nested}{%
  \@ifnextchar\jvcomp{\@jvcomp@nested}{%
  \@ifnextchar\jfcomp{\@jfcomp@nested}{%
  }}}}}}}}
\newcommand{\jcomp@testright}{%
  \@ifnextchar\jcomp{\@jcomp@nested}{%
  \@ifnextchar\subst{\@subst@nested}{%
  \@ifnextchar\ctxext{\@ctxext@nested}{%
  \@ifnextchar\bgroup{\@jcomp@parens}{%
  \@ifnextchar\tmext{\@tmext@nested}{%
  \@ifnextchar\jvcomp{\@jvcomp@nested}{%
  \@ifnextchar\jfcomp{\@jfcomp@nested}{%
  }}}}}}}}
\newcommand{\@jcomp@nested}[4]{%
  \@jcomp@parens{#2}{#3}{#4}}
\newcommand{\@jcomp@parens}[3]{%
  (\jcomp{#1}{#2}{#3})}

\newcommand{\jcomp@unfold}[3]{%
  \subst
    {\jcomp@unfold@test@preside #2}
    {\ctxwk{\default@ctxext #1}{\jcomp@unfold@test@postside #3}}}
\newcommand{\jcomp@unfold@test@preside}{%
  \@ifnextchar\bgroup{\@jcomp@unfold@parens}{}}
\newcommand{\jcomp@unfold@test@postside}{%
  \@ifnextchar\bgroup{\@jcomp@unfold@parens}{}}
\newcommand{\@jcomp@unfold@nested}[4]{%
  \@jcomp@unfold@parens{#2}{#3}{#4}}
\newcommand{\@jcomp@unfold@parens}[3]{%
  (\jcomp@unfold{#1}{#2}{#3})}

%%%% Vertical composition of morphisms.
\newcommand{\jvcomp}[3]{%
  \jcomp@testleft #2 * \jcomp@testright #3}
\newcommand{\jvcomp@testleft}{%
  \@ifnextchar\jvcomp{\@jvcomp@nested}{%
  \@ifnextchar\ctxwk{\@ctxwk@nested}{%
  \@ifnextchar\ctxext{\@ctxext@nested}{%
  \@ifnextchar\bgroup{\@jvcomp@parens}{%
  \@ifnextchar\tmext{\@tmext@nested}{%
  \@ifnextchar\jcomp{\@jcomp@nested}{%
  \@ifnextchar\jfcomp{\@jfcomp@nested}{%
  }}}}}}}}
\newcommand{\jvcomp@testright}{%
  \@ifnextchar\jvcomp{\@jvcomp@nested}{%
  \@ifnextchar\subst{\@subst@nested}{%
  \@ifnextchar\ctxext{\@ctxext@nested}{%
  \@ifnextchar\bgroup{\@jvcomp@parens}{%
  \@ifnextchar\tmext{\@tmext@nested}{%
  \@ifnextchar\jcomp{\@jcomp@nested}{%
  \@ifnextchar\jfcomp{\@jfcomp@nested}{%
  }}}}}}}}
\newcommand{\@jvcomp@nested}[4]{%
  \@jvcomp@parens{#2}{#3}{#4}}
\newcommand{\@jvcomp@parens}[3]{%
  (\jvcomp{#1}{#2}{#3})}

\newcommand{\jvcomp@unfold}[3]{%
  \tmext{}{}{\ctxwk{#1}{#2}}{#3}
  }
\newcommand{\jvcomp@unfold@test@preside}{%
  \@ifnextchar\bgroup{\@jvcomp@unfold@parens}{}}
\newcommand{\jvcomp@unfold@test@postside}{%
  \@ifnextchar\bgroup{\@jvcomp@unfold@parens}{}}
\newcommand{\@jvcomp@unfold@nested}[4]{%
  \@jvcomp@unfold@parens{#2}{#3}{#4}}
\newcommand{\@jvcomp@unfold@parens}[3]{%
  (\jvcomp@unfold{#1}{#2}{#3})}

%%%% The judgment that F is a morphism from P to Q over f in context \Gamma.
\newcommand{\jfhomsym}[3]{\jhomsym[{#1}]{#2}{#3}}
\newcommand{\jfhom}{%
  \@ifnextchar*{\jfhomAlignTrue}{\jfhomAlignFalse}}
\newcommand{\jfhomAlignTrue}[8]{
  \judgment*{#2}{}{}{#8}{}{}{\jfhomsym{#5}{#6}{#7}}}
\newcommand{\jfhomAlignFalse}[7]{
  \judgment{#1}{}{}{#7}{}{}{\jfhomsym{#4}{#5}{#6}}}
\newcommand{\jfhomeq}[8]{%
  \judgment{#1}{#7}{\jdeq}{#8}{}{}{\jhomsym[{#4}]{#5}{#6}}}
\newcommand{\jfhomdefn}[8]{%
  \judgment{#1}{#7}{\defeq}{#8}{}{}{\jhomsym[{#4}]{#5}{#6}}}
\newcommand{\jfhom@unfold}[7]{%
  \jterm
    {{{#1}{#2}}{#5}}
    {\ctxwk{\default@ctxext #5}{\jcomp{#2}{#4}{#6}}}
    {#7}}
    
\newcommand{\jfcomp}[5]{%
  \jfcomp@testleft #5 \bullet \jfcomp@testright #4}
\newcommand{\jfcomp@testleft}{%
  \@ifnextchar\bgroup{\@jfcomp@parens}{%
  \@ifnextchar\jfcomp{\@jfcomp@nested}{%
  \@ifnextchar\jcomp{\@jcomp@nested}{%
  \@ifnextchar\ctxwk{\@ctxwk@nested}{%
  \@ifnextchar\tmext{\@tmext@nested}{%
  \@ifnextchar\jvcomp{\@jvcomp@nested}{%
  }}}}}}}
\newcommand{\jfcomp@testright}{%
  \@ifnextchar\bgroup{\@jfcomp@parens}{%
  \@ifnextchar\jfcomp{\@jfcomp@nested}{%
  \@ifnextchar\jcomp{\@jcomp@nested}{%
  \@ifnextchar\subst{\@subst@nested}{%
  \@ifnextchar\tmext{\@tmext@nested}{%
  \@ifnextchar\jvcomp{\@jvcomp@nested}{%
  }}}}}}}
\newcommand{\@jfcomp@nested}[1]{%
  \@jfcomp@parens}
\newcommand{\@jfcomp@parens}[5]{%
  (\jfcomp{#1}{#2}{#3}{#4}{#5})}
  
\newcommand{\jfcomp@unfold}[5]{%
  \jcomp{#3}{#4}{{#1}{#2}{#5}}}
\makeatother

%%%%%%%%%%%%%%%%%%%%%%%%%%%%%%%%%%%%%%%%%%%%%%%%%%%%%%%%%%%%%%%%%%%%%%%%%%%%%%%%
%%%% JUDGMENTAL TRIVIAL COFIBRATIONS

\newcommand{\jtcext}{\tilde}

%%%%%%%%%%%%%%%%%%%%%%%%%%%%%%%%%%%%%%%%%%%%%%%%%%%%%%%%%%%%%%%%%%%%%%%%%%%%%%%%
%%%% CONTEXT PROJECTIONS

\makeatletter
\newcommand{\cprojgenf}[3]{%
  \typefont{pr}^{%
    \@ifnextchar\bgroup{\@ctxext@parens}{%
    \@ifnextchar\ctxext{\@ctxext@nested}{%
    }}
    #2,
    \@ifnextchar\bgroup{\@ctxext@parens}{%
    \@ifnextchar\ctxext{\@ctxext@nested}{%
    }}
    #3
    }_{#1}}
\newcommand{\cprojgen}[4]{%
  \subst{#4}{\cprojgenf{#1}{#2}{#3}}}
\newcommand{\cprojgenf@nested}[1]{%
  \cprojgenf@parens}
\newcommand{\cprojgenf@parens}[3]{%
  (\cprojgenf{#1}{#2}{#3})}
\newcommand{\cprojgen@nested}[1]{%
  \cprojgen@parens}
\newcommand{\cprojgen@parens}[4]{%
  (\cprojgen{#1}{#2}{#3}{#4})}

\newcommand{\cprojfstf}[2]{%
  \cprojgenf{0}{#1}{#2}}
\newcommand{\cprojfstf@nested}[1]{%
  \cprojfstf@parens}
\newcommand{\cprojfstf@parens}[2]{%
  (\cprojfstf{#1}{#2})}
\newcommand{\cprojfstf@unfold}[2]{%
  \ctxwk{\default@ctxext #2}\idtm{\default@ctxext #1}}

\newcommand{\cprojfst}[3]{%
  \cprojgen{0}{#1}{#2}{#3}}
\newcommand{\cprojfst@nested}[1]{%
  \cprojfst@parens}
\newcommand{\cprojfst@parens}[3]{%
  (\cprojfst{#1}{#2}{#3})}
\newcommand{\cprojfst@unfold}[3]{%
  \subst{#3}{(\cprojfstf@unfold{#1}{#2})}}

\newcommand{\cprojsndf}[2]{%
  \cprojgenf{1}{#1}{#2}}
\newcommand{\cprojsndf@nested}[1]{%
  \cprojsndf@parens}
\newcommand{\cprojsndf@parens}[2]{%
  (\cprojsndf{#1}{#2})}
\newcommand{\cprojsndf@unfold}[2]{%
  \idtm{\default@ctxext #2}}

\newcommand{\cprojsnd}[3]{%
  \cprojgen{1}{#1}{#2}{#3}}
\newcommand{\cprojsnd@nested}[1]{%
  \cprojsnd@parens}
\newcommand{\cprojsnd@parens}[3]{%
  (\cprojsnd{#1}{#2}{#3})}
\newcommand{\cprojsnd@unfold}[3]{%
  \subst{#3}{\cprojsnd@unfold{#1}{#2}}}
  
%%%% The sandwich function
\newcommand{\sandwich}[3]{\typefont{sw}^{#1,#2,#3}}
\newcommand{\sandwich@unfold}[3]{\typefont{sw}^{#1,#2,#3}}
\makeatother

%%%%%%%%%%%%%%%%%%%%%%%%%%%%%%%%%%%%%%%%%%%%%%%%%%%%%%%%%%%%%%%%%%%%%%%%%%%%%%%%
%%%% FIBER INCLUSIONS

\makeatletter
\newcommand{\finc}[2]{\typefont{in}^{#2}_{#1}}
\newcommand{\finc@unfold}[2]{\tmext{}{}{\ctxwk{\subst{x}{P}}{x}}{\idtm{\subst{x}{P}}}}
\makeatother

%%%%%%%%%%%%%%%%%%%%%%%%%%%%%%%%%%%%%%%%%%%%%%%%%%%%%%%%%%%%%%%%%%%%%%%%%%%%%%%%
%%%% THE UNIT TYPE

\makeatletter
\newcommand{\unitc}[1]{%
  \unit^0_{\default@ctxext #1}}
\newcommand{\unitct}[1]{%
  \ttt^0_{\default@ctxext #1}}
\newcommand{\unitf}[2]{%
  \unit^1_{\default@ctxext #1,\default@ctxext #2}}
\newcommand{\unitft}[2]{%
  \ttt^1_{\default@ctxext #1,\default@ctxext #2}}
\makeatother

%%%%%%%%%%%%%%%%%%%%%%%%%%%%%%%%%%%%%%%%%%%%%%%%%%%%%%%%%%%%%%%%%%%%%%%%%%%%%%%%
%%%% DEPENDENT FUNCTION TYPES

\makeatletter
\newcommand{\sprd}[2]{\Pi(\default@ctxext #1,\default@ctxext #2)}
\begin{comment}
\newcommand{\@sprd@test@cod}[2]{%
  \@ifnextchar\bgroup{\@sprd@do@cod{#1}}{%
  \Pi(\@sprd@test@dom{#1}{#2} #1,
  }}
\newcommand{\@sprd@do@cod}[4]{%
  \ctxext{\@sprd{#1}{#2}}{\@sprd{#1}{#3}}
  }
\newcommand{\@sprd}[2]{
  \@ifnextchar\bgroup{\@@sprd}{%
    \Pi(}
    #1,{#2})
  }
\newcommand{\@@sprd}[5]{%
  \sprd{#1}{\sprd{#2}{#4}}
  }
\end{comment}

\newcommand{\slam}[3]{%
  \lambda^{{\default@ctxext@parens #1},{\default@ctxext@parens #2}}
  (\default@ctxext #3)
  }
\newcommand{\sev}[1]{\tfev(#1)}

\makeatother

%%%%%%%%%%%%%%%%%%%%%%%%%%%%%%%%%%%%%%%%%%%%%%%%%%%%%%%%%%%%%%%%%%%%%%%%%%%%%%%%
%%%% NON-DEPENDENT FUNCTION TYPES

\newcommand{\jfun}[2]{#1\to#2}

%%%%%%%%%%%%%%%%%%%%%%%%%%%%%%%%%%%%%%%%%%%%%%%%%%%%%%%%%%%%%%%%%%%%%%%%%%%%%%%%
%%%% THE CONSTRUCTORS OF THE TYPE THEORY OF MODELS

\makeatletter
%%%% The initial model
\newcommand{\mctx}{%
  \mathcal{C}}

%%%% The family constructor
\newcommand{\mfam}[2][]{%
  \mathcal{F}_{\default@ctxext #2}^{#1}}
\newcommand{\@mfam@nested}[1]{\@mfam@parens}
\newcommand{\@mfam@parens}[2][]{(\mfam[#1]{#2})}

%%%% The terms constructor
\newcommand{\mtm}[2][]{%
  \mathcal{T}_{\default@ctxext #2}^{#1}}
\newcommand{\@mtm@nested}[1]{\@mtm@parens}
\newcommand{\@mtm@parens}[2][]{(\mtm[#1]{#2})}

%%%% The empty type constructor
\newcommand{\tfemp}[1]{%
  \typefont{emp}_{\default@ctxext #1}}
\newcommand{\tft}[1]{%
  \typefont{t}_{\default@ctxext #1}}

%%%% The extension constructor
\newcommand{\tfext}[1]{%
  \typefont{ext}_{\default@ctxext #1}}

%%%% The substitution constructor
\newcommand{\tfsubst}[1]{%
  \typefont{subst}_{\default@ctxext #1}}
  
%%%% The weakening constructor
\newcommand{\tfwk}[1]{%
  \typefont{wk}_{\default@ctxext #1}}

%%%% The identity function constructor
\newcommand{\tfid}[1]{%
  \typefont{idtm}_{\default@ctxext #1}}
\makeatother

%%%%%%%%%%%%%%%%%%%%%%%%%%%%%%%%%%%%%%%%%%%%%%%%%%%%%%%%%%%%%%%%%%%%%%%%%%%%%%%%

%%%% Introducing logical usage of fonts.
\newcommand{\modelfont}{\mathit} % use 'mf' in command to indicate model font
\newcommand{\typefont}{\mathsf} % use 'tf' in command to indicate type font
\newcommand{\catfont}{\mathrm} % use 'cf' in command to indicate cat font

%%%%%%%%%%%%%%%%%%%%%%%%%%%%%%%%%%%%%%%%%%%%%%%%%%%%%%%%%%%%%%%%%%%%%%%%%%%%%%%%
%%%% Some macros of the book are redefined.

\renewcommand{\UU}{\typefont{U}}
\renewcommand{\isequiv}{\typefont{isEquiv}}
\renewcommand{\happly}{\typefont{hApply}}
\renewcommand{\pairr}[1]{{\mathopen{}\langle #1\rangle\mathclose{}}}
\renewcommand{\type}{\typefont{Type}}
\renewcommand{\op}[1]{{{#1}^\typefont{op}}}
\renewcommand{\susp}{\typefont{\Sigma}}

%%%%%%%%%%%%%%%%%%%%%%%%%%%%%%%%%%%%%%%%%%%%%%%%%%%%%%%%%%%%%%%%%%%%%%%%%%%%%%%%
%%%% The following is a big unorganized list of new macros that we use in the
%%%% notes. 

\newcommand{\tfW}{\typefont{W}}
\newcommand{\tfM}{\typefont{M}}
\newcommand{\mfM}{\modelfont{M}}
\newcommand{\mfN}{\modelfont{N}}
\newcommand{\tfctx}{\typefont{ctx}}
\newcommand{\mftypfunc}[1]{{\modelfont{typ}^{#1}}}
\newcommand{\mftyp}[2]{{\mftypfunc{#1}(#2)}}
\newcommand{\tftypfunc}[1]{{\typefont{typ}^{#1}}}
\newcommand{\tftyp}[2]{{\tftypfunc{#1}(#2)}}
\newcommand{\hfibfunc}[1]{\typefont{fib}_{#1}}
\newcommand{\mappingcone}[1]{\mathcal{C}_{#1}}
\newcommand{\equifib}{\typefont{equiFib}}
\newcommand{\tfcolim}{\typefont{colim}}
\newcommand{\tflim}{\typefont{lim}}
\newcommand{\tfdiag}{\typefont{diag}}
\newcommand{\tfGraph}{\typefont{Graph}}
\newcommand{\mfGraph}{\modelfont{Graph}}
\newcommand{\unitGraph}{\unit^\mfGraph}
\newcommand{\UUGraph}{\UU^\mfGraph}
\newcommand{\tfrGraph}{\typefont{rGraph}}
\newcommand{\mfrGraph}{\modelfont{rGraph}}
\newcommand{\isfunction}{\typefont{isFunction}}
\newcommand{\tfconst}{\typefont{const}}
\newcommand{\conemap}{\typefont{coneMap}}
\newcommand{\coconemap}{\typefont{coconeMap}}
\newcommand{\tflimits}{\typefont{limits}}
\newcommand{\tfcolimits}{\typefont{colimits}}
\newcommand{\islimiting}{\typefont{isLimiting}}
\newcommand{\iscolimiting}{\typefont{isColimiting}}
\newcommand{\islimit}{\typefont{isLimit}}
\newcommand{\iscolimit}{\typefont{iscolimit}}
\newcommand{\pbcone}{\typefont{cone_{pb}}}
\newcommand{\tfinj}{\typefont{inj}}
\newcommand{\tfsurj}{\typefont{surj}}
\newcommand{\tfepi}{\typefont{epi}}
\newcommand{\tftop}{\typefont{top}}
\newcommand{\sbrck}[1]{\Vert #1\Vert}
\newcommand{\strunc}[2]{\Vert #2\Vert_{#1}}
\newcommand{\gobjclass}{{\typefont{U}^\mfGraph}}
\newcommand{\gcharmap}{\typefont{fib}}
\newcommand{\diagclass}{\typefont{T}}
\newcommand{\opdiagclass}{\op{\diagclass}}
\newcommand{\equifibclass}{\diagclass^{\eqv{}{}}}
\newcommand{\universe}{\typefont{U}}
\newcommand{\catid}[1]{{\catfont{id}_{#1}}}
\newcommand{\isleftfib}{\typefont{isLeftFib}}
\newcommand{\isrightfib}{\typefont{isRightFib}}
\newcommand{\leftLiftings}{\typefont{leftLiftings}}
\newcommand{\rightLiftings}{\typefont{rightLiftings}}
\newcommand{\psh}{\typefont{Psh}}
\newcommand{\rgclass}{\typefont{\Omega^{RG}}}
\newcommand{\terms}[2][]{\lfloor #2 \rfloor^{#1}}
\newcommand{\grconstr}[2]
             {\mathchoice % max size is textstyle size.
             {{\textstyle \int_{#1}}#2}% 
             {\int_{#1}#2}%
             {\int_{#1}#2}%
             {\int_{#1}#2}}
\newcommand{\ctxhom}[3][]{\typefont{hom}_{#1}(#2,#3)}
\newcommand{\graphcharmapfunc}[1]{\gcharmap_{#1}}
\newcommand{\graphcharmap}[2][]{\graphcharmapfunc{#1}(#2)}
\newcommand{\tfexp}[1]{\typefont{exp}_{#1}}
\newcommand{\tffamfunc}{\typefont{fam}}
\newcommand{\tffam}[1]{\tffamfunc(#1)}
\newcommand{\tfev}{\typefont{ev}}
\newcommand{\tfcomp}{\typefont{comp}}
\newcommand{\isDec}[1]{\typefont{isDecidable}(#1)}
\newcommand{\smal}{\mathcal{S}}
\renewcommand{\modal}{{\ensuremath{\ocircle}}}
\newcommand{\eqrel}{\typefont{EqRel}}
\newcommand{\piw}{\ensuremath{\Pi\typefont{W}}} %% to be used in conjunction with -pretopos.
\renewcommand{\sslash}{/\!\!/}
\newcommand{\mprd}[2]{\Pi(#1,#2)}
\newcommand{\msm}[2]{\Sigma(#1,#2)}
\newcommand{\midt}[1]{\idvartype_#1}
\newcommand{\reflf}[1]{\typefont{refl}^{#1}}
\newcommand{\tfJ}{\typefont{J}}
\newcommand{\tftrans}{\typefont{trans}}

\newcommand{\tfT}{\typefont{T}}
\newcommand{\reflsym}{{\mathsf{refl}}}
\newcommand{\strans}[2]{\ensuremath{{#1}_{*}({#2})}}
\newcommand{\eqtype}[1]{\typefont{Eq}_{#1}}
\newcommand{\eqtoid}[1]{\typefont{eqtoid}(#1)}
\newcommand{\greek}{\mathrm}
\newcommand{\product}[2]{{#1}\times{#2}}
\newcommand{\pairp}[1]{(#1)}
\newcommand{\jequalizer}[3]{\{#1|#2\jdeq #3\}}
\newcommand{\jequalizerin}[2]{\iota_{#1,#2}}
\newcommand{\tounit}[1]{{!_{#1}}}
\newcommand{\trwk}{\typefont{trwk}}
\newcommand{\trext}{\typefont{trext}}

%%%%%%%%%%%%%%%%%%%%%%%%%%%%%%%%%%%%%%%%%%%%%%%%%%%%%%%%%%%%%%%%%%%%%%%%%%%%%%%%
%%%% When investigation pointed structures we use the \pt macro.

\makeatletter
\newcommand{\pt}[1][]{*_{
  \@ifnextchar\undergraph{\@undergraph@nested}
    {\@ifnextchar\underovergraph{\@underovergraph@nested}{}}#1}}
\makeatother

%%%%%%%%%%%%%%%%%%%%%%%%%%%%%%%%%%%%%%%%%%%%%%%%%%%%%%%%%%%%%%%%%%%%%%%%%%%%%%%%
%%%% OPERATIONS ON GRAPHS
%%%%
%%%% First of all, each graph has a type of vertices and a type of edges. The
%%%% type of vertices of a graph $\Gamma$ is denoted by $\pts{\Gamma}$;
%%%% and likewise for the type of edges.

\makeatletter
\newcommand{\pts}[1]{{\@graphop@nested{#1}}_{0}}
\newcommand{\edg}[1]{{\@graphop@nested{#1}}_{1}}
\newcommand{\@graphop@nested}[1]
  {\@ifnextchar\ctxext{\@ctxext@nested}
      {\@ifnextchar\undergraph{\@undergraph@nested}
         {\@ifnextchar\underovergraph{\@underovergraph@nested}{}}}
    #1}
\makeatother

%%%% The following operations of \undergraph and \underovergraph are used to
%%%% define the free category and the free groupoid of a graph, respectively

\makeatletter
\newcommand{\@undergraphtest}[2]{\@ifnextchar({#1}{#2}}
\newcommand{\undergraph}[2]{\@undergraphtest{\@undergraph@parens{#1}{#2}}{\@undergraph{#1}{#2}}}
\newcommand{\@undergraph}[2]{{#2/#1}}
\newcommand{\@undergraph@nested}[3]{\@undergraph@parens{#2}{#3}}
\newcommand{\@undergraph@parens}[2]{(\@undergraph{#1}{#2})}
\makeatother

\makeatletter
\newcommand{\underovergraph}[2]{\@underovergraphtest{\@underovergraph@parens{#1}{#2}}{\@underovergraph{#1}{#2}}}
\newcommand{\@underovergraph}[2]{{#2}\,{\parallel}\,{#1}}
\newcommand{\@underovergraphtest}{\@undergraphtest}
\newcommand{\@underovergraph@parens}[2]{(\@underovergraph{#1}{#2})}
\newcommand{\@underovergraph@nested}[3]{\@underovergraph@parens{#2}{#3}}
\makeatother

\newcommand{\graphid}[1]{\mathrm{id}_{#1}}
\newcommand{\freecat}[1]{\mathcal{C}(#1)}
\newcommand{\freegrpd}[1]{\mathcal{G}(#1)}


%%%%%%%%%%%%%%%%%%%%%%%%%%%%%%%%%%%%%%%%%%%%%%%%%%%%%%%%%%%%%%%%%%%%%%%%%%%%%%%%
%% Some tikz macros to typeset diagrams uniformly.

\tikzset{patharrow/.style={double,double equal sign distance,-,font=\scriptsize}}
\tikzset{description/.style={fill=white,inner sep=2pt}}
\tikzset{fib/.style={->>,font=\scriptsize}}

%% Used for extra wide diagrams, e.g. when the label is too large otherwise.
\tikzset{commutative diagrams/column sep/Huge/.initial=18ex}

%%%%%%%%%%%%%%%%%%%%%%%%%%%%%%%%%%%%%%%%%%%%%%%%%%%%%%%%%%%%%%%%%%%%%%%%%%%%%%%%
%%%% New theorem environment for conjectures.

\defthm{conj}{Conjecture}{Conjectures}

%%%%%%%%%%%%%%%%%%%%%%%%%%%%%%%%%%%%%%%%%%%%%%%%%%%%%%%%%%%%%%%%%%%%%%%%%%%%%%%%
%%%% The following environment for desiderata should not be there. It is better
%%%% to use the issue tracker for desiderata.

\newenvironment{desiderata}{\begingroup\color{blue}\textbf{Desiderata.}}
{\endgroup}

%%%%%%%%%%%%%%%%%%%%%%%%%%%%%%%%%%%%%%%%%%%%%%%%%%%%%%%%%%%%%%%%%%%%%%%%%%%%%%%%
%%%% The following piece of code from tex.stackexchange:
%%%%
%%%% http://tex.stackexchange.com/a/55180/14653
%%%%
%%%% We include it so that inference rules in align environments have enough
%%%% vertical space.

\newlength\minalignvsep

\makeatletter
\def\align@preamble{%
   &\hfil
    \setboxz@h{\@lign$\m@th\displaystyle{##}$}%
    \ifnum\row@>\@ne
    \ifdim\ht\z@>\ht\strutbox@
    \dimen@\ht\z@
    \advance\dimen@\minalignvsep
    \ht\strutbox\dimen@
    \fi\fi
    \strut@
    \ifmeasuring@\savefieldlength@\fi
    \set@field
    \tabskip\z@skip
   &\setboxz@h{\@lign$\m@th\displaystyle{{}##}$}%
    \ifnum\row@>\@ne
    \ifdim\ht\z@>\ht\strutbox@
    \dimen@\ht\z@
    \advance\dimen@\minalignvsep
    \ht\strutbox@\dimen@
    \fi\fi
    \strut@
    \ifmeasuring@\savefieldlength@\fi
    \set@field
    \hfil
    \tabskip\alignsep@
}
\makeatother

\minalignvsep.2em

\allowdisplaybreaks

%%%%%%%%%%%%%%%%%%%%%%%%%%%%%%%%%%%%%%%%%%%%%%%%%%%%%%%%%%%%%%%%%%%%%%%%%%%%%%%%

\setdescription[1]{itemsep=-0.2em}


%\includeonly{Algebras/article-models-cat}

%%%%%%%%%%%%%%%%%%%%%%%%%%%%%%%%%%%%%%%%%%%%%%%%%%%%%%%%%%%%%%%%%%%%%%%%%%%%%%%%
\title{An essentially algebraic formulation of dependent type theory}
\author{Egbert Rijke}
\date{\today}

\begin{document}

\maketitle

%\tableofcontents

%\section{Introduction}
The project I propose here has its origins in the beginning of 2013, when I proved a version
of the descent theorem for homotopy colimits in type theory while I was working
with Bas Spitters to develop notions from higher category theory in the
univalent foundations. To arrive at a
notion of diagram general enough to capture the higher inductive types described
in chapter 6 of \cite{TheBook} excluding the truncations we needed type
theoretical graphs. The graphs form a model of type theory and indeed we needed
several of the basic type theoretical operations to give an efficient approach
to the descent theorem. Although it was not an issue to describe the graph model
and the sense in which it models the type constructors, 
not all of type theory is interpreted very well. To start with, context
extension isn't strictly associative if dependent pair types are used for
its interpretation. Also, if $A$ and $B$ are families of graphs
over a graph $\Gamma$, then $B$ isn't also a family of graphs over the extended
graph $\ctxext{\Gamma}{A}$ (interfering with a good interpretation of weakening).
There are no issues with the interpretation of substitution though. Actually
it is not so straightforward to describe what it models!

The graph model
shows three operations: extension, weakening and substitution. All of them
can be defined such that they act not only on contexts, but also on families
and terms. Moreover, they are all compatible with each other. In fact, along
with asserting the existence of identity morphisms (which is done in type
theory), it seems
that this structure is enough to establish some of the structure of a category.
We might even go so far as to boldly assert that
\begin{quote}
\emph{Category theory is dependent type theory without type constructors.}
\end{quote}
It is a part of this proposal to test this hypothesis, with the note that by
`category theory' we mean a theory of weak higher categories which does not
have any of the truncatedness restrictions -- in particular not those appearing
in the theory of AKS-categories as
described in \cite{TheBook} in chapter 11. We note that even Dybjers theory
of internal categories with families has such restrictions: he relies on setoids.
It needs not much arguing that these are very unnatural from a univalent point
of view.

\subsection{Ideas in the definition of internal models}
An internal model of type theory is like a category with families, but we want
to avoid having to state higher coherences. In fact, we don't even start our
definition with a category of contexts; instead we just take a \emph{type} of contexts. 
The morphisms will come from the terms, evaluation of a function at a given
term will come from substitution. We recognize three basic ingredients to models:
first there is a type of contexts; second, for every context there is a model of types in
that context and third, for every type in a given context there is a type of its
terms. Then there are three basic attributes: context extension, weakening and
substitution. Context extension provides us with families over types as well as
with an interpretation of dependent pair types. We need weakening 
so that families can depend on the same type multiple times (the way the
identity type of a type depends two times on that type) and to be able
to talk about non-dependent function types,
the morphims of our category. Substitution will give us a way
to work with fibers of families as well as composition of functions and evaluation
of functions at terms.

Because we require a \emph{model} of types in a context, all the structure
which we require at the bottom level will be required to exist higher up as well.
Thus, the model of types in a given context $\Gamma$ will have a type of contexts
itself, which can be seen as the type of types in $\Gamma$; it will have its
own notion of types in a context, its own notion of terms, context extension,
weakening and substitution together with all the structure require for it. For
instance, when $A$ is a type in context $\Gamma$ in a model $\mfM$, then there
is the model of types in context $A$, which is the model of families over $A$. 
This model is required to be \emph{definitionally equal to} the model of types
in the context $\ctxext{\Gamma}{A}$, the context extension of $\Gamma$ and $A$.
In this way we protect ourselves from the need to dig an infinitely deep structure
of models when we want to consider examples.

To give the definition of a model we shall also need to consider certain morphisms
of internal models. Those should preserve all the structure: contexts are mapped
to contexts; for every context a morphism of models mapping the model of types
in that context to the model of types in the image of that context; there should
be a mapping of terms and context extension, weakening and substitution should be
preserved. We need to consider those morphisms because we require context extension,
weakening and substitution to be of that kind, thereby respecting each other
in all possible ways.

When we have this framework set up, we can interpret the basic type constructors
such as $\Pi$, $\Sigma$ and $\idtypevar{}$.
The higher categorical structure then comes from the
result that we have an interpretation of type theory.

\subsection{The elementary theory of the category with families of categories
with families}

Thus the idea was born to give a new description of Martin-L\"of type theory
which is more faithful to the idea that type theory is an algebraic description
of higher category theory. The features we have in mind for this description
is that:
\begin{enumerate}
\item As in Martin-L\"of type theory there are the three basic judgments
      \begin{align*}
        & \Gamma\text{ is a context:} & \jctx*{\Gamma}\\
        & A\text{ is a type in context }\Gamma\text{:} & \jtype*{\Gamma}{A}\\
        & x\text{ is a term of $A$ in context }\Gamma\text{:} & \jterm*{\Gamma}{A}{x}
      \end{align*}
      together with the three accompanying judgments for judgmental equality.
      Our new description of type theory should be such that we can faithfully
      make the translation to the three assertions
      \begin{enumerate}
      \item $\Gamma$ is an object\\
      \item $A$ is a fibration over $\Gamma$\\
      \item $x$ is a section of the fibration $A$. 
      \end{enumerate}
\item There is a basic type theory which has just the operations
      \begin{description}
      \item[extension] taking the domain of the fibration $A$ over $\Gamma$.
      \item[weakening] given an object $\Gamma$, weakening assigns to an object
      $\Delta$ trivial fibration with $\Delta$ as its fibers.
      \item[substitution] taking the fiber at a point. Fibers should always
      be equivalent to homotopy fibers.
      \end{description}
      Thus we aim for a closer connection between the syntactic operations and
      the operations of models.
\item Models of this basic type theory should be weak $\omega$-categories. Thus,
      it should not be necessary to start the definition of a model with a bunch
      of contexts and substitution morphisms between them: these and the
      operations on them, such as composition and associativity thereof, should
      be a consequence of the interpretation of the basic type theory.
\item There shouldn't be an empty context and dependent function types right
      from the start. These correspond to having a terminal object and the
      models of type theory with dependent function types are the locally
      cartesian closed models.
\item Contexts should not be lists of variable declarations and all the
      operations should be explicitly invariant under context extension.
\item There should be a large zoo of extensions of the basic type theory. First
      off, we can extend the basic type theory with identity types, dependent
      function types and inductive types. We want a general description of
      homotopy colimits in type theory, which should become available once we
      know what categories (models of the basic type theory) are. Other types
      such as the truncations could be added as well.
      
      An extension of a different kind is the addition of the univalence axiom.
\end{enumerate}

\subsection{The further goals of the project}
The first aim of this project is thus to give this new description of type
theory with these properties. The second aim is to define a general notion of
an internal model and provide several examples thereof. In the definition of
internal models we envision that all the operations of type theory become
morphisms of type theory: they are compatible with each and every morphism
of type theory. The compatibility of extension, weakening and substitution is
already part of the basic type theory. The compatibility of identity types
with weakening for instance, is a form of function extensionality. The 
compatibility of dependent function types with identity types is the usual
funchtion extensionality principle. The
compatibility of a universe operation with identity types should be univalence.
The descent theorem implies that the colimit operation is also compatible with
all the type theoretic operations (sigma and id are checked).

Among the examples of internal models we should have:
\begin{description}
\item[The setoid model] This is a classical one so it should be there. It might
      be a bit different in our case. We'll want to construct a setoid model
      of the basic type theory to interpret identity types without necessarily
      interpreting dependent function types (but that should remain possible).
\item[The graph model] This is the model that leads to the first version of the
      descent theorem.
\item[Univalent unvirses] A univalent universe should be a model.
\item[Equifibered diagrams over a graph] should be the fibrations in some model.
\item[The model of all models] Very likely, we will even define this simultaneously
      with the notion of model.
\item[The model of weak $\omega$-groupoids] Since we think of models of the
      basic type theory as weak $\omega$-categories, it is not too hard to
      provide a condition on those models which enables us to talk about
      weak $\omega$-categories. In fact, we have several options for such a
      condition. We investigate those; we also investigate how they relate to
      Brunerie's weak $\omega$-groupoids. Presumably, they form an internal
      model without too much fuss (because we already have an internal model
      of internal models at this stage) and we can ask whether this models
      the univalence axiom. In fact, the model of internal models might already
      have modeled the univalence axiom. 
\end{description}

We also wish to extend the descent property to a more general class of homotopy
colimits. The general recipe has already been implicit in the first descent
property where we have used diagrams over graphs. We have the following in mind:
\begin{enumerate}
\item Give an internal description of a diagram. Probably this is the same thing
      as an internal model of the basic type theory with or without identity types.
\item Define the notion of homotopy colimit thereof.
\item Define the notion if equifibered diagrams.
\item Formulate and prove descent: the map defined by substitution
 from families over the colimit to the equifibered diagrams is an equivalence.
\end{enumerate}

Using the descent property we can view $\tfcolim$ as an operation of type theory
It should preserve all the basic operations, identity types, dependent function
types, itself, etc.

Another thing we must keep in mind is that the compatibility rules in the basic
type theory are strict. We should look to the possibility to weaken them.

\subsection{Overview of the document}
In \autoref{eg} we begin the project with exploring internal models of type theory
in type theoretical setting of \cite{TheBook}.


%\part{Type theory}

\begin{comment}
In this part we develop type theory from the ground up. We start with a type
theory without any of the basic constructors. This is the theory of contexts
families and terms which has the basic operations of extension, weakening,
substitution and identity terms. Type theory before type constructors has not
been studied very much. Dependent product types or even universes tend to make
an early appearance in just about any presentation of type theory.
A noteable exception is the theory of categories with families of Dybjer
in \cite{Dybjer1996}, which has been elaborated on further by Dybjer and
Clairambault in \cite{DybjerClairambault2011} and in unpublished work by
Awodey \cite{Awodey2013} on natural models of type theory,
which makes the connection between categories with families and representable
transformations of presheaves. In the way type theory is presented by the
Univalent Foundations Project in \cite{TheBook}, which seems to have won the race of introducing universes
as early as possible hands down, it seems entirely unfeasible to study type
theory without type constructors. Also, the 
proof-assistant \Coq\ {\color{red}(and \Agda too?)} has universes and dependent product and pair types 
built-in, making it impossible to study type theory without type constructors in 
that environment.

Nevertheless, type theory without type constructors has received some attention
contributors to the Univalent Foundations Program recently. We name two
further investigations on this topic, other than the mentioned work by Awodey.
In \cite{Garner2014}, Garner describes the combinatorial structure
of the type operations of the weakening, substitution and projection monads
(their projections are our identity terms) and suggests lots of further research
that can be done on type theory without constructors. Also, Joyal has presented
his theory of tribes, which is a categorical explanation of type theory
without type constructors.

After we have described the E-system, which is the flavor of type theory without
type constructors presented in \autoref{tt}, we will demonstrate in \autoref{ttderived}
that the theory gives rise to a rich categorical structure. The notions and
properties we derive here will be essential for the further work on internal
models for E-systems, presented in \autoref{part:models}.
\end{comment}

\section{Introduction}
In this article we define the essentially algebraic objects which correspond to dependent
type theories, which are closely related to Voevodsky's B-systems. This provides
a new axiomatization of dependent type theory, with the following properties:
\begin{itemize}
\item The meta-theory should not require anything more than rules for handling
inferences. In particular, natural numbers, or an infinite set of variables 
should not be required.
\item Finitely many sorts. The basic ingredients are contexts, families and
terms. Thus there will be 3 sorts.
\item The theory is algebraic and the operations on it are homomorphisms. From
the requirement that type theoretic operations must be homomorphisms, we can
read off which judgmental equalities to require require.
\item Finite set of rules.
\item The theory is invariant under slicing in a canonical way.
\end{itemize}

We build
dependent type theory by imposing several structures of increasing complexities
on top of each other. We begin by the theory of extension, which incorporates
a context extension and a family extension operation to the theory which merely
has three sorts. On top of the theory of extension we can formulate the theories
of weakening, substitution and extension-units (i.e.~the empty context and families).
The theory of weakening can be enriched with units playing the role of identity
morphisms, which gives a theory of projections. The three resulting theories
can then be combined into dependent type theory. Thus, the theory of dependent
types is outlined in the following diagram.

\begin{equation*}
\begin{tikzcd}[ampersand replacement=\&,column sep=0em]
{} \& \textbf{Extension} \ar[very thick,-stealth]{dr} \ar[very thick,-stealth]{d} \ar[very thick,-stealth]{dl}
  \\
\makebox[.7cm]{\textbf{Weakening}} \ar[very thick,-stealth]{d} \& \textbf{Substitution} \ar[very thick,-stealth]{dd} \& \makebox[3cm]{\textbf{Empty context and families}} \ar[very thick,-stealth]{ddl}
\\
\makebox[.7cm]{\textbf{Projection}} \ar[very thick,-stealth]{dr}
\\
{} \& \textbf{Dependent Type Theory}
\end{tikzcd}
\end{equation*}

We give two presentations of the theory simultaneously. First, we directly
define the algebras for the corresponding theories in an arbitrary category
with finite limits and second, we describe the theory in terms of inference
rules.

For the semantics of the theory of E-systems, we will assume that $\cat{C}$ is a 
category with finite limits and
whenever we write a pullback, we assume that it is chosen. Recall that for
any morphism $f:A\to B$ in a category $\cat{C}$ with chosen pullbacks, there
is a functor
\begin{equation*}
f^\ast : \cat{C}/B\to\cat{C}/A.
\end{equation*}
As usual, when $g:X\to B$ is a morphism, we will write $f^\ast(X)$ for the
domain of $f^\ast(g)$. When there is more than one morphism $X\to B$ involved,
as will be the case below, we will write $\pullback{A}{X}{f}{g}$. The projections
will be written as $\pullbackpr{1}{f}{g}$ and $\pullbackpr{2}{f}{g}$. So in this notation, a
typical pullback diagram has the following form:
\begin{equation*}
\begin{tikzcd}[column sep=large]
\pullback{A}{X}{f}{g}
  \ar{r}{\pullbackpr{1}{f}{g}}
  \ar{d}[swap]{\pullbackpr{2}{f}{g}}
  &
A \ar{d}{f}
  \\
X \ar{r}[swap]{g}
  &
B
\end{tikzcd}
\end{equation*}
Also, when we have a commutative diagram of the form
\begin{equation*}
\begin{tikzcd}
A \ar{r}{f}
  \ar{d}{a}
  &
X \ar{d}
  & 
B \ar{l}[swap]{g}
  \ar{d}{b}
  \\
A'
  \ar{r}[swap]{f'}
  &
X'
  &
B'
  \ar{l}{g'}
\end{tikzcd}
\end{equation*}
we will denote the unique map from $\pullback{A}{B}{f}{g}$ to $\pullback{A'}{B'}{f'}{g'}$
such that the diagram
\begin{equation*}
\begin{tikzcd}
  {}
  & 
\pullback{A'}{B'}{f'}{g'}
  \ar{dd}
  \ar{rr}
  &
  &
B'
  \ar{dd}{g'}
  \\
\pullback{A}{B}{f}{g}
  \ar{dd}
  \ar[crossing over]{rr}
  \ar[dotted]{ur}{\pullback{a}{b}{f'}{g'}}
  &
  &
B \ar{ur}{b}
  \\
  {}
  &
A'
  \ar{rr}
  &
  &
X'
  \\
A \ar{rr}[swap]{f}
  \ar{ur}{a}
  &
  &
X \ar[crossing over,leftarrow]{uu}[near end,swap]{g}
  \ar{ur}
\end{tikzcd}
\end{equation*}
commutes, by $\pullback{a}{b}{f'}{g'}$. In the current work, we shall
write $A\times B$ for the pullback of $A\rightarrow 1\leftarrow B$, and
$\pi_1$ and $\pi_2$ for its projections (thus, no separate choice of
cartesian products is made).

The formal theory of contexts, families and terms shall be described by means of a
declaration of valid inference rules. Informally, an inference rule is a finite list
$\mathcal{H}_1,\ldots,\mathcal{H}_n;\mathcal{J}$ of expressions called
judgments. Usually we represent an inference rule in the following
form
\begin{equation*}
\inference{\mathcal{H}_1\quad\cdots\quad\mathcal{H}_n}{\mathcal{J}}
\end{equation*}
The judgments $\mathcal{H}_1$, \ldots, $\mathcal{H}_n$ are called the
hypotheses and the judgment $\mathcal{J}$ is the conclusion of the inference
rule.

An inference rule is called \emph{valid} in either of the following cases:
\begin{description}
\item[(Axiom)] It is declared valid as an axiom;
\item[(Projection)] The inference rule
\begin{equation*}
\inference{\mathcal{H}_1\quad\cdots\quad\mathcal{H}_n}{\mathcal{H}_i}
\end{equation*}
is a valid inference rule for any $1\leq i\leq n$;
\item[(Composition)] If
\begin{equation*}
\inference{\mathcal{I}_{1k}\quad\cdots\quad\mathcal{I}_{m_kk}}{\mathcal{H}_k}
\qquad\text{and}\qquad
\inference{\mathcal{H}_1\quad\cdots\quad\mathcal{H}_n}{\mathcal{J}}
\end{equation*}
are valid inference rules for $1\leq k\leq n$, then
\begin{equation*}
\inference{\mathcal{I}_{11}\quad\cdots\quad\mathcal{I}_{m_11}\quad\cdots\quad\mathcal{I}_{1n}\quad\cdots\quad\mathcal{I}_{m_nn}}{\mathcal{J}}
\end{equation*}
is also a valid inference rule;
\item[(Cut elimination)] If
\begin{equation*}
\inference{\mathcal{H}_1\quad\cdots\quad\mathcal{H}_n}{\mathcal{H}_{n+1}}
\qquad
\text{and}
\qquad
\inference{\mathcal{H}_1\quad\cdots\quad\mathcal{H}_{n+m+1}}{\mathcal{J}}
\end{equation*}
are valid inference rules, then
\begin{equation*}
\inference{\mathcal{H}_1\quad\cdots\quad\hat{\mathcal{H}}_{n+1}\quad\cdots\quad\mathcal{H}_{n+m+1}}{\mathcal{J}}
\end{equation*}
where the hypothesis $\mathcal{H}_{n+1}$ is omitted, is a valid inference rule.
\end{description}
In this article, we shall restrict ourselves to formulating the axioms for
dependent type theory. Derivations shall be made in a follow-up.

\section{The fundamental structure of dependent type theory}

\subsection{The judgments of dependent type theory}
\label{judgments}

The theory we describe here is a theory of contexts, families of
contexts and terms thereof. The families of contexts are by some authors called
dependent contexts, but they are handled a bit differently here because they
become the primary object of study. Dependent contexts can be types; they could
be seen as atomic or indecomposable dependent contexts.

Thus we make eight kinds of judgments: ``$\Gamma$ is a context'',
``$A$ is a family of contexts over $\Gamma$'', ``$A$ is a type in context $\Gamma$''
and ``$x$ is a term of the family $A$ of contexts over $\Gamma$''. The other four
judgments are for judgmental equality.
\begin{align*}
\jalign\jctx{\Gamma} 
& \jalign\jctxeq{\Gamma}{\Gamma'}
  \\
\jalign\jfam{\Gamma}{A} 
& \jalign\jfameq{\Gamma}{A}{B}
  \\
\jalign\jterm{\Gamma}{A}{x} 
& \jalign\jtermeq{\Gamma}{A}{x}{y}.
\end{align*}

Strictly speaking, we have three different judgmental equalities in play and one
could request for a notational difference to signify that fact. For instance,
we could denote the judgmental equalities of contexts, families and terms by
$\jdeq_c$, $\jdeq_f$ and $\jdeq_t$ respectively. It will, however, always be
clear which of the three kinds of judgmental equality is meant when we assert
a judgmental equality and therefore we shall not bother to make this notational
distinction.

We note that what we call families over contexts
here could also have been named dependent contexts or telescopes, see
\cite{deBruijn1991,hofmann1995extensional}. The term family is in agreement
with the terminology scheme of \cite{TheBook}, though the reader should be
warned that the notion of familie means something slightly different there than
it does here.

The rules for judgmental equality establish that it is an equivalence relation
in all three cases (contexts, families and terms). Thus, the following inference
rules shall be required to be valid:
\bgroup\small
\begin{align*}
& \inference
  { \jctx{\Gamma}
    }
  { \jctxeq{\Gamma}{\Gamma}
    } 
& & \inference
    { \jctxeq{\Gamma}{\Delta}
      }
    { \jctxeq{\Delta}{\Gamma}
      } 
& & \inference
    { \jctxeq{\Gamma}{\Delta}
      \jctxeq{\Delta}{\greek{E}}
      }
    { \jctxeq{\Gamma}{\greek{E}}
      }
    \\
& \inference
  { \jfam{\Gamma}{A}
    }
  { \jfameq{\Gamma}{A}{A}
    } 
& & \inference
    { \jfameq{\Gamma}{A}{B}
      }
    { \jfameq{\Gamma}{B}{A}
      }
& & \inference
    { \jfameq{\Gamma}{A}{B}
      \jfameq{\Gamma}{B}{C}
      }
    { \jfameq{\Gamma}{A}{C}
      }
    \\
& \inference
  { \jterm{\Gamma}{A}{x}
    }
  { \jtermeq{\Gamma}{A}{x}{x}
    }
& & \inference
    { \jtermeq{\Gamma}{A}{x}{y}
      }
    { \jtermeq{\Gamma}{A}{y}{x}
      }
& & \inference
    { \jtermeq{\Gamma}{A}{x}{y}
      \jtermeq{\Gamma}{A}{y}{z}
      }
    { \jtermeq{\Gamma}{A}{x}{z}
      }
\end{align*}
\egroup

The following convertibility rules are responsible for the strictness
of judgmental equality, which sets it apart from equivalences or identifications:
\begin{align*}
& \inference
  { \jctxeq{\Gamma}{\Delta}
    \jfam{\Gamma}{A}
    }
  { \jfam{\Delta}{A}
    }
& & \inference
    { \jctxeq{\Gamma}{\Delta}
      \jfameq{\Gamma}{A}{B}
      }
    { \jfameq{\Delta}{A}{B}
      }
    \\
& \inference
  { \jctxeq{\Gamma}{\Delta}
    \jterm{\Gamma}{A}{x}
    }
  { \jterm{\Delta}{A}{x}
    }
& & \inference
    { \jctxeq{\Gamma}{\Delta}
      \jtermeq{\Gamma}{A}{x}{y}
      }
    { \jtermeq{\Delta}{A}{x}{y}
      }
    \\
& \inference
  { \jfameq{\Gamma}{A}{B}
    \jterm{\Gamma}{A}{x}
    }
  { \jterm{\Gamma}{B}{x}
    }
& & \inference
    { \jfameq{\Gamma}{A}{B}
      \jtermeq{\Gamma}{A}{x}{y}
      }
    { \jtermeq{\Gamma}{B}{x}{y}
      }
\end{align*}


\subsection{Semantics of the fundamental structure of dependent type theory}
For the semantics of the theory of E-systems, we will assume that $\cat{C}$ is a 
category with finite limits and
whenever we write a pullback, we assume that it is chosen. Recall that for
any morphism $f:A\to B$ in a category $\cat{C}$ with chosen pullbacks, there
is a functor
\begin{equation*}
f^\ast : \cat{C}/B\to\cat{C}/A.
\end{equation*}
As usual, when $g:X\to B$ is a morphism, we will write $f^\ast(X)$ for the
domain of $f^\ast(g)$. When there is more than one morphism $X\to B$ involved,
as will be the case below, we will write $\pullback{A}{X}{f}{g}$. The projections
will be written as $\pullbackpr{1}{f}{g}$ and $\pullbackpr{2}{f}{g}$. So in this notation, a
typical pullback diagram has the following form:
\begin{equation*}
\begin{tikzcd}[column sep=large]
\pullback{A}{X}{f}{g}
  \ar{r}{\pullbackpr{1}{f}{g}}
  \ar{d}[swap]{\pullbackpr{2}{f}{g}}
  &
A \ar{d}{f}
  \\
X \ar{r}[swap]{g}
  &
B
\end{tikzcd}
\end{equation*}
Also, when we have a commutative diagram of the form
\begin{equation*}
\begin{tikzcd}
A \ar{r}{f}
  \ar{d}{a}
  &
X \ar{d}
  & 
B \ar{l}[swap]{g}
  \ar{d}{b}
  \\
A'
  \ar{r}[swap]{f'}
  &
X'
  &
B'
  \ar{l}{g'}
\end{tikzcd}
\end{equation*}
we will denote the unique map from $\pullback{A}{B}{f}{g}$ to $\pullback{A'}{B'}{f'}{g'}$
such that the diagram
\begin{equation*}
\begin{tikzcd}
  {}
  & 
\pullback{A'}{B'}{f'}{g'}
  \ar{dd}
  \ar{rr}
  &
  &
B'
  \ar{dd}{g'}
  \\
\pullback{A}{B}{f}{g}
  \ar{dd}
  \ar[crossing over]{rr}
  \ar[dotted]{ur}{\pullback{a}{b}{f'}{g'}}
  &
  &
B \ar{ur}{b}
  \\
  {}
  &
A'
  \ar{rr}
  &
  &
X'
  \\
A \ar{rr}[swap]{f}
  \ar{ur}{a}
  &
  &
X \ar[crossing over,leftarrow]{uu}[near end,swap]{g}
  \ar{ur}
\end{tikzcd}
\end{equation*}
commutes, by $\pullback{a}{b}{f'}{g'}$. In the current work, we shall
write $A\times B$ for the pullback of $A\rightarrow 1\leftarrow B$, and
$\pi_1$ and $\pi_2$ for its projections (thus, no separate choice of
cartesian products is made).

\begin{defn}
A \emph{fundamental structure} $\stesys$ in $\cat{C}$ consists of a diagram of the form
\begin{equation*}
\begin{tikzcd}
\stesyst
  \ar{d}[swap]{\ebd}
  \\
\stesysf
  \ar{d}[swap]{\eft}
  \\
\stesysc
\end{tikzcd}
\end{equation*}
in $\cat{C}$. In this diagram, $C$ is the object of \emph{contexts}, $F$ is
the object of \emph{families}, and $T$ is the object of \emph{terms}.
\end{defn}



\section{The theory of extension}

\subsection{Extension}
\label{extension}

The operations of extension enable one to consider families over
families and higher level families alike. We will need both a context extension
operation and a family extension operation.

The inference rules introducing context and family extension are as follows:
\begin{align}
& \inference
  { \jfam{\Gamma}{A}
    }
  { \jctx{\ctxext{\Gamma}{A}}
    }
& & \inference
    { \jctxeq{\Gamma}{\Delta}
      \jfameq{\Gamma}{A}{B}
      }
    { \jctxeq{\ctxext{\Gamma}{A}}{\ctxext{\Delta}{B}}
      }
    \\
& \inference
  { \jfam{{\Gamma}{A}}{P}
    }
  { \jfam{\Gamma}{\ctxext{A}{P}}
    }
& & \inference
    { \jfameq{\Gamma}{A}{B} 
      \jfameq{{\Gamma}{A}}{P}{Q}
      }
    { \jfameq{\Gamma}{\ctxext{A}{P}}{\ctxext{B}{Q}}
      }
\end{align}
The extension operation may also be defined to act on families over families
and terms thereof. For instance, When $\jfam{{\Gamma}{A}}{P}$ is a family,
we would get a family $\jfam{{\Gamma}{A}}{{\Gamma}{P}}$ and when
$\jterm{{\Gamma}{A}}{P}{f}$ is a term, we would get a term
$\jterm{{\Gamma}{A}}{{\Gamma}{P}}{\ctxext{\Gamma}{f}}$. Then there would be
axioms stating the judgmental equalities $\ctxext{\Gamma}{P}\jdeq P$ and
$\ctxext{\Gamma}{f}\jdeq f$. Thus, it is of no actual use to introduce the
action of context extension on families and terms.

Using identity terms and substitution, it is also possible to define an 
extension operation on terms, which defines a term 
$\jterm{\Gamma}{{A}{P}}{\tmext{x}{u}}$ for every $\jterm{\Gamma}{A}{x}$ and 
$\jterm{\Gamma}{\subst{x}{P}}{u}$. Thus, there is a formal definition of the
\emph{pair} $\tmext{x}{u}$. We shall not go into this in the current paper,
but it will be very useful to have when one further develops the theory of
E-systems.

\subsubsection{Associativity of extension}
\label{comp-ee}
The inference rules asserting that extension is compatible with itself assert
that contexts are unstructured lists of type declarations. This rule is
unavoidable if we want that for a family $A$ in context $\Gamma$, a family over
$A$ is the same thing as a family over $\ctxext{\Gamma}{A}$. 
\begin{align}
& \inference
  { \jfam{\Gamma}{A}
    \jfam{{\Gamma}{A}}{P}
    }
  { \jctxeq{\ctxext{{\Gamma}{A}}{P}}{\ctxext{\Gamma}{{A}{P}}}
    }
  \\
& \inference
  { \jfam{{\Gamma}{A}}{P}
    \jfam{{{\Gamma}{A}}{P}}{Q}
    }
  { \jfameq{\Gamma}{\ctxext{A}{{P}{Q}}}{\ctxext{{A}{P}}{Q}}
    }
  \label{comp-ee-c}
\end{align}



\subsection{Pre-extension algebras}
\begin{defn}
A \emph{pre-extension algebra $\stesys$ in $\cat{C}$} consists of a fundamental structure
$\stesys$
in $\cat{C}$ together with the \emph{context extension} and \emph{family extension} operations
\begin{align*}
\ectxext &:\stesysf\to \stesysc\\
\efamext & :\stesysff\to \stesysf,
\end{align*}
respectively, such that the diagram
\begin{equation*}
\begin{tikzcd}
\stesysf_2 
  \ar{r}{\efamext} 
  \ar{d}[swap]{\eft[2]} 
  & 
\stesysf 
  \ar{d}{\eft}
  \\
\stesysf
  \ar{r}[swap]{\eft} 
  & 
\stesysc
\end{tikzcd}
\end{equation*}
commutes.
\end{defn}

In the following we give an inductive definition of $\stesysf_n$ $\eft[n]$, $\eext{n}$,
$\stesyst_n$ and $\ebd[n]$. This is more than we need in this article, because
we will only make use of the first few instances.

\begin{defn}
Set $\stesysf_0:=\stesysc$, $\stesysf_1:=\stesysf$ and $\eft[1]:=\eft$. We define
\begin{equation*}
\stesysf_{2} 
  := \pullback{\stesysf}{\stesysf}{\ectxext}{\eft}
\end{equation*}
By induction on the natural numbers $n\geq 1$, we define $\stesysf_{n+2}$, $\eft[n+1]$ and
$\eext{n+1}$ simultaneously, by
\begin{align*}
\eft[n+1] 
  & := \pullbackpr{1}{\eext{n-1}}{\eft[n]} : \stesysf_{n+2}\to\stesysf_{n+1}
  \\
\stesysf_{n+2} 
  & := \pullback{\stesysf_{n+1}}{\stesysf_{n+1}}{\eext{n}}{\eft[n+1]}
\end{align*}
As inductive hypothesis we require that the outer square in the diagram
\begin{equation*}
\begin{tikzcd}[column sep=huge]
\stesysf_{n+2}
  \ar[dotted]{dr}{\eext{n+1}}
  \ar{rr}{\pullback{\pullbackpr{2}{\eext{n-1}}{\eft[n]}}{\pullbackpr{2}{\eext{n-1}}{\eft[n]}}{\eext{n-1}}{\eft[n]}}
  \ar{dd}[swap]{\eft[n+2]}
  & 
  &
\stesysf_{n+1}
  \ar{d}{\eext{n}}
  \\
  &
\stesysf_{n+1}
  \ar{d}[swap]{\eft[n+1]}
  \ar{r}{\pullbackpr{2}{\eext{n-1}}{\eft[n]}}
  &
\stesysf_{n}
  \ar{d}{\eft[n]}
  \\
\stesysf_{n+1}
  \ar{r}[swap]{\eft[n+1]}
  &
\stesysf_{n}
  \ar{r}[swap]{\eext{n-1}}
  &
\stesysf_{n-1}
\end{tikzcd}
\end{equation*}
commutes, which is clear in the base case $n=1$. Now we define $\eext{n+1}$ to be the
unique morphism rendering the above diagram commutative.

Similarly, we define
\begin{align*}
\stesyst_{n+1} 
  & := \pullback{\stesysf_n}{\stesyst_n}{\eext{n-1}}{\eft[n]\circ\ebd[n]}
  \\
\ebd[n+1]
  & := \eext{n-1}^\ast(\ebd[n])
\end{align*}
for $n\geq 1$, where we take $\stesyst_1:=\stesyst$ and $\ebd[1]:=\ebd$.
\end{defn}

The above definition allows for the construction of a formal slice construction
for each pre-extension algebra.

\begin{defn}
Suppose $\stesys$ is a pre-extension algebra of $\cat{C}$. Then we define the 
pre-extension algebra $\famesys{\stesys}$ to consist of the fundamental 
structure
\begin{equation*}
\begin{tikzcd}
\stesyst_2
  \ar{d}{\ebd[2]}
  \\
\stesysf_2
  \ar{d}{\eft[2]}
  \\
\stesysf
\end{tikzcd}
\end{equation*}
with the extension operations
\begin{align*}
\efamext 
  & 
  : \stesysf_2\to\stesysf\\
\eext{2} & : \stesysf_3\to\stesysf_2.
\end{align*}
\end{defn}

\begin{defn}
Let $\stesys$ and $\stesys'$ be pre-extension algebras. A \emph{pre-extension 
homomorphism $f$ from $\stesys'$ to $\stesys$} is a homomorphism of fundamental
structures for which the squares
\begin{equation*}
\begin{tikzcd}
\stesysf' 
  \ar{r}{f_1}
  \ar{d}[swap]{\ectxext'}
  &
\stesysf
  \ar{d}{\ectxext}
  \\
\stesysc'
  \ar{r}[swap]{f_0}
  &
\stesysc
\end{tikzcd}
\end{equation*}
and
\begin{equation*}
\begin{tikzcd}[column sep=huge]
\stesysf'\times_{\ectxext',\eft'} \stesysf'
  \ar{r}{f_1\times_{\ectxext,\eft} f_1}
  \ar{d}[swap]{\efamext'}
  &
\stesysf\times_{\ectxext,\eft} \stesysf
  \ar{d}{\efamext}
  \\
\stesysf'
  \ar{r}[swap]{f_1}
  &
\stesysf
\end{tikzcd}
\end{equation*}
Composition and the identity homomorphism are defined in the expected way. We
define furthermore
\begin{align*}
f_2 & := \pullback{f_1}{f_1}{\ectxext}{\eft}
  \\
f_3 & := \pullback{f_2}{f_2}{\efamext}{\eft[2]}.
\end{align*}
\end{defn}

Observe that a morphism $A_{n+1}:X\to\stesysf_{n+1}$ is a family over
$A_n:=\eft[n+1]\circ A_{n+1}:X\to\stesysf_n$. So $A_{n+1}:X\to\stesysf_{n+1}$
determines all the families $A_i:X\to\stesysf_i$ for $0\leq i\leq n+1$.
However, we could also consider the sequence of families given by
$B_0:= \ectxext\circ A_1$, $B_1:=A_2$,\ldots, $B_n:=A_{n+1}$, basically by
forgetting that there was a further underlying context $A_0$.
It is useful to capture this idea by defining the morphisms which realise it,
which is what we do in slightly more generality in the following definition.

An analogy might provide some intuition with the simple combinatorics at
work here. Consider the sequence $A_0,\ldots,A_{n+1}$ as a series of
matryoshka dolls fit inside each other. $A_0$ is the outer doll, $A_{n+1}$ is
the inner doll. Then there are two other fitting sequences of matryoshka dolls
we can construct: we can either take the outer $n$ dolls, leaving out the inner
doll $A_{n+1}$; or we can pick the inner $n$ dolls, leaving out the outer doll.
Both result in fitting configurations of matryoshka dolls. The first is in
analogy with $\eft[n+1]\circ A_{n+1}$, the second is in analogy with
$\beta^0_{n+1}\circ A_{n+1}$, of which $\beta^0_{n+1}$ will be defined now.

\begin{defn}
We define $\beta^m_n:\stesysf_{n+m+1}\to\stesysf_{n+m}$ for 
$m\geq 0$ and $n\geq 1$
by
\begin{align*}
\beta^m_1 
  & := 
\pullbackpr{2}{\eext{m}}{\eft[m+1]}
  & &
  : \stesysf_{1+m+1}\to\stesysf_{1+m}
  \\
\beta^m_{n+1}
  & :=
\pullback{\beta^m_n}{\beta^m_n}{\eext{n+m-1}}{\eft[n+m]}
  & &
  : \stesysf_{n+m+2}\to\stesysf_{n+m+1}
\end{align*}
We shall denote $\beta^0_n$ by $\beta_n$.
\end{defn}

\begin{rmk}
By definition, we have the equality
\begin{equation}
\eext{n+1}=[\eft[n+1]\circ\eft[n+2],\eext{n}\circ\beta^{n-1}_2]
\end{equation}
for each $n\geq 1$.
\end{rmk}

We end this subsection by stating some properties of $\beta$, which we will
use later. The easy proofs 
\begin{lem}\label{lem:beta_beta}
Let $\stesys$ be a pre-extension algebra. Then the square
\begin{equation*}
\begin{tikzcd}
\stesysf_4
  \ar{r}{\beta^1_2}
  \ar{d}[swap]{\beta_3}
  &
\stesysf_3
  \ar{d}{\beta_2}
  \\
\stesysf_3
  \ar{r}[swap]{\beta_2}
  & 
\stesysf_2
\end{tikzcd}
\end{equation*}
commutes.
\end{lem}

Note that the square
\begin{equation*}
\begin{tikzcd}
\stesysf_3
  \ar{r}{\beta_2}
  \ar{d}[swap]{\eext{2}}
  &
\stesysf_2
  \ar{d}{\efamext}
  \\
\stesysf_2
  \ar{r}[swap]{\beta_1}
  &
\stesysf
\end{tikzcd}
\end{equation*}
commutes by definition. We have a similar result relating $\eext{3}$ and
$\eext{2}$.

\begin{lem}\label{lem:beta_ext}
Let $\stesys$ be a pre-extension algebra. Then the square
\begin{equation*}
\begin{tikzcd}
\stesysf_4
  \ar{r}{\beta_3}
  \ar{d}[swap]{\eext{3}}
  &
\stesysf_3
  \ar{d}{\eext{2}}
  \\
\stesysf_3
  \ar{r}[swap]{\beta_2}
  &
\stesysf_2
\end{tikzcd}
\end{equation*}
commutes.
\end{lem}

\begin{defn}
Let $\stesys$ be a pre-extension algebra. Then
\begin{equation*}
\begin{tikzcd}
\stesyst_3
  \ar{r}{\beta_t}
  \ar{d}[swap]{\ebd[3]}
  &
\stesyst_2
  \ar{d}{\ebd[2]}
  \\
\stesysf_3
  \ar{r}{\beta_2}
  \ar{d}[swap]{\eft[3]}
  &
\stesysf_2
  \ar{d}{\eft[2]}
  \\
\stesysf_2
  \ar{r}[swap]{\beta_1}
  &
\stesysf
\end{tikzcd}
\qquad
\text{and}
\qquad
\begin{tikzcd}
\stesyst_4
  \ar{r}{{\beta_t}_2}
  \ar{d}[swap]{\ebd[4]}
  &
\stesyst_3
  \ar{d}{\ebd[3]}
  \\
\stesysf_4
  \ar{r}{\beta_3}
  \ar{d}[swap]{\eft[4]}
  &
\stesysf_3
  \ar{d}{\eft[3]}
  \\
\stesysf_3
  \ar{r}[swap]{\beta_2}
  &
\stesysf_2
\end{tikzcd}
\end{equation*}
assemble pre-extension homomorphisms 
\(
\boldsymbol{\beta}
  :
\famesys{\famesys{\stesys}}
  \to
\famesys{\stesys}
\) 
and
\(
\boldsymbol{\beta}_\mathbf{2}
  :
\famesys{\famesys{\famesys{\stesys}}}
  \to
\famesys{\famesys{\stesys}}
\).
\end{defn}

\begin{defn}\label{famehom}
Suppose that $f:\stesys'\to\stesys$ is a pre-extension homomorphism. Then we
define $\famehom{f}:\famesys{\stesys'}\to\famesys{\stesys}$ to consist of
\begin{equation*}
\begin{tikzcd}
\stesyst_2'
  \ar{r}{{f_t}_2}
  \ar{d}[swap]{\ebd[2]'}
  &
\stesyst_2
  \ar{d}{\ebd[2]}
  \\
\stesysf_2'
  \ar{r}{f_2}
  \ar{d}[swap]{\eft[2]'}
  &
\stesysf_2
  \ar{d}{\eft[2]}
  \\
\stesysf'
  \ar{r}[swap]{f_1}
  &
\stesysf
\end{tikzcd}
\end{equation*}
where we define
\begin{equation*}
{f_t}_2 := \pullback{f_1}{f_t}{\ectxext}{\eft\circ\ebd}.
\end{equation*}
\end{defn}

\begin{lem}
The triple $\famehom{f}$ defined in \autoref{famehom} is a pre-extension homomorphism.
\end{lem}

\begin{proof}
Note that the square
\begin{equation*}
\begin{tikzcd}
\stesysf_2'
  \ar{r}{f_2}
  \ar{d}[swap]{\efamext'}
  &
\stesysf_2
  \ar{d}{\efamext}
  \\
\stesysf'
  \ar{r}[swap]{f_1}
  &
\stesysf
\end{tikzcd}
\end{equation*}
commutes by assumption. Thus, it remains to show that the square
\begin{equation*}
\begin{tikzcd}
\stesysf_3'
  \ar{r}{f_3}
  \ar{d}[swap]{\eext{2}'}
  &
\stesysf_3
  \ar{d}{\eext{2}}
  \\
\stesysf_2'
  \ar{r}[swap]{f_2}
  &
\stesysf_2
\end{tikzcd}
\end{equation*}
commutes. It is equivalent to show that the equalities
\begin{align*}
\pullbackpr{1}{\ectxext}{\eft}\circ f_2\circ\eext{2}'
  & =
\pullbackpr{1}{\ectxext}{\eft}\circ \eext{2}\circ f_3
  \\
\pullbackpr{2}{\ectxext}{\eft}\circ f_2\circ\eext{2}'
  & =
\pullbackpr{2}{\ectxext}{\eft}\circ \eext{2}\circ f_3
\end{align*}
both hold. For the first subgoal, we will verify that both the left and the
right hand side are equal to the composite $\eft[3]'\circ \eft[2]'\circ f_1$.
To this end, it is straightforward to verify that the diagrams
\begin{equation*}
\begin{tikzcd}[column sep=large]
\stesysf_3'
  \arrow[r,"{\eext{2}'}"]
  \arrow[d,swap,"{\eft[3]'}"]
  &
\stesysf_2'
  \arrow[r,"f_2"]
  \arrow[d,swap,"{\eft[2]'}"]
  &
\stesysf_2
  \arrow[d,"\pullbackpr{1}{\ectxext}{\eft}"]
  \\
\stesysf_2'
  \arrow[r,swap,"{\eft[2]'}"]
  &
\stesysf'
  \arrow[r,swap,"f_1"]
  &
\stesysf
\end{tikzcd}
\end{equation*}
and
\begin{equation*}
\begin{tikzcd}
\stesysf_3'
  \arrow[r,"f_3"]
  \arrow[d,swap,"{\eft[3]'}"]
  &
\stesysf_3
  \arrow[dr,"\eext{2}"]
  \arrow[d,swap,"{\eft[3]}"]
  \\
\stesysf_2'
  \arrow[r,"f_2"]
  \arrow[dr,swap,"{\eft[2]'}"]
  &
\stesysf_2
  \arrow[dr,"{\eft[2]}"]
  &
\stesysf_2
  \arrow[d,"\pullbackpr{1}{\ectxext}{\eft}"]
  \\
{}&
\stesysf'
  \arrow[r,swap,"f_1"]
  &
\stesysf
\end{tikzcd}
\end{equation*}
commute.
\begin{comment}
\begin{equation*}
\begin{tikzcd}[column sep=large]
{} &
\stesysf_2'
  \ar{r}{f_2}
  \ar{dr}[near end]{\pullbackpr{1}{\ectxext'}{\eft'}}
  &
\stesysf_2
  \ar{dr}{\pullbackpr{1}{\ectxext}{\eft}}
  \\
\stesysf_3'
  \ar{ur}{\eext{2}'}
  \ar{r}[swap]{\beta_2'}
  \ar{ddr}[swap]{f_3}
  &
\stesysf_2'
  \ar{r}{\efamext'}
  \ar{dr}[swap]{f_2}
  &
\stesysf'
  \ar{r}{f_1}
  &
\stesysf
  \\
{} & {} &
\stesysf_2
  \ar{ur}[near start]{\efamext}
  \\
{} &
\stesysf_3
  \ar{r}[swap]{\eext{2}}
  \ar{ur}{\beta_2}
  &
\stesysf_2
  \ar{uur}[swap]{\pullbackpr{1}{\ectxext}{\eft}}
\end{tikzcd}
\end{equation*}
commutes.
\end{comment}
For the second subgoal, we will show that the left hand side and the right hand
side are both equal to $\efamext\circ f_2\circ\beta_2'$. To this end, note that
the diagrams
\begin{equation*}
\begin{tikzcd}
\stesysf_3'
  \arrow[r,"{\eext{2}'}"]
  \arrow[d,swap,"{\beta_2'}"]
  &
\stesysf_2'
  \arrow[dr,"f_2"]
  \arrow[d,swap,"{\beta_1'}"]
  \\
\stesysf_2'
  \arrow[r,"{\eext{1}'}"]
  \arrow[dr,swap,"{f_2}"]
  &
\stesysf'
  \arrow[dr,"{f_1}"]
  &
\stesysf_2
  \arrow[d,"\pullbackpr{2}{\ectxext}{\eft}"]
  \\
{}&
\stesysf_2
  \arrow[r,swap,"\efamext"]
  &
\stesysf
\end{tikzcd}
\end{equation*}
and
\begin{equation*}
\begin{tikzcd}[column sep=large]
\stesysf_3'
  \arrow[r,"f_3"]
  \arrow[d,swap,"{\beta_2'}"]
  &
\stesysf_3
  \arrow[r,"\eext{2}"]
  \arrow[d,swap,"{\beta_2}"]
  &
\stesysf_2
  \arrow[d,"\pullbackpr{2}{\ectxext}{\eft}"]
  \\
\stesysf_2'
  \arrow[r,swap,"f_2"]
  &
\stesysf_2
  \arrow[r,swap,"\efamext"]
  &
\stesysf
\end{tikzcd}
\end{equation*}
commute.
\begin{comment}
\begin{equation*}
\begin{tikzcd}[column sep=large]
{} &
\stesysf_2'
  \ar{r}{f_2}
  \ar{dr}{\beta_1'}
  &
\stesysf_2
  \ar{ddr}{\pullbackpr{2}{\ectxext}{\eft}}
  \\
{} & {} &
\stesysf'
  \ar{dr}[swap,near start]{f_1}
  \\
\stesysf_3'
  \ar{uur}{\eext{2}'}
  \ar{r}{\beta_2'}
  \ar{dr}[swap]{f_3}
  &
\stesysf_2'
  \ar{r}{f_2}
  \ar{ur}{\efamext'}
  &
\stesysf_2
  \ar{r}[swap]{\efamext}
  &
\stesysf
  \\
{} &
\stesysf_3
  \ar{r}[swap]{\eext{2}}
  \ar{ur}[near start]{\beta_2}
  &
\stesysf_2
  \ar{ur}[swap]{\pullbackpr{2}{\ectxext}{\eft}}
\end{tikzcd}
\end{equation*}
commutes.
\end{comment}
\end{proof}

\subsection{Extension algebras}
\begin{defn} 
An extension algebra is a pre-extension algebra $\stesys$ for which 
the diagrams
\begin{equation*}
\begin{tikzcd}
\stesysf_2 
  \ar{d}[swap]{\pullbackpr{2}{\ectxext}{\eft}} 
  \ar{r}{\efamext} 
  & 
\stesysf 
  \ar{d}{\ectxext}
  \\
\stesysf 
  \ar{r}[swap]{\ectxext} 
  & 
\stesysc
\end{tikzcd}
\qquad
\begin{tikzcd}
\stesysf_3
  \ar{d}[swap]{\pullbackpr{2}{\efamext}{\eft[2]}}
  \ar{r}{\eext{2}}
  & 
\stesysf_2 
  \ar{d}{\efamext} 
  \\
\stesysf_2 
  \ar{r}[swap]{\efamext} 
  &
\stesysf
\end{tikzcd}
\end{equation*}
commute.
\end{defn}

\begin{thm}[Local extension structure]\label{famextobj}
If $\stesys$ is an extension algebra, then so is $\famesys{\stesys}$.
\end{thm}

\begin{proof}
Note that the diagram
\begin{equation*}
\begin{tikzcd}
\stesysf_3
  \ar{d}[swap]{\pullbackpr{2}{\efamext}{\eft[2]}}
  \ar{r}{\eext{2}}
  & 
\stesysf_2 
  \ar{d}{\efamext} 
  \\
\stesysf_2 
  \ar{r}[swap]{\efamext} 
  &
\stesysf
\end{tikzcd}
\end{equation*}
commutes by assumption. For the second condition, we have to show that the
diagram
\begin{equation*}
\begin{tikzcd}
\stesysf_4
  \ar{d}[swap]{\pullbackpr{2}{\eext{2}}{\eft[3]}}
  \ar{r}{\eext{3}}
  & 
\stesysf_3
  \ar{d}{\eext{2}} 
  \\
\stesysf_3
  \ar{r}[swap]{\eext{2}} 
  &
\stesysf_2
\end{tikzcd}
\end{equation*}
commutes. Since this is a question about two maps into the pullback
$\stesysf_2$, it suffices to verify that the two triangles in the pullback
diagram
\begin{equation*}
\begin{tikzcd}[column sep=huge]
\stesysf_4
  \arrow[drr,"\eext{1}\circ\eext{2}\circ\beta_3",bend left=15]
  \arrow[ddr,swap,"{\eft[2]\circ\eft[3]\circ\eft[4]}",bend right=15]
  \arrow[dr,"\alpha",densely dotted]
  \\
{}&
\stesysf_2
  \arrow[r,swap,"\pullbackpr{2}{\ectxext}{\eft}"]
  \arrow[d,swap,"{\eft[2]}"]
  &
\stesysf
  \arrow[d,"\eft"]
  \\
{}&
\stesysf
  \arrow[r,swap,"\ectxext"]
  &
\stesysc
\end{tikzcd}
\end{equation*}
commute for both $\alpha:=\eext{2}\circ\eext{3}$ and 
$\alpha:=\eext{2}\circ\pullbackpr{2}{\eext{2}}{\eft[3]}$. In other words, we
will establish the equalities
\begin{equation}
\eext{2}\circ\eext{3}
  =
[\eft[2]\circ\eft[3]\circ\eft[4],\eext{1}\circ\eext{2}\circ\beta_3]
  =
\eext{2}\circ\pullbackpr{2}{\eext{2}}{\eft[3]}.
\end{equation}
\begin{comment}
\begin{align*}
\pullbackpr{1}{\ectxext}{\eft}\circ\eext{2}\circ\eext{3}
  & =
\pullbackpr{1}{\ectxext}{\eft}\circ\eext{2}\circ\pullbackpr{2}{\eext{2}}{\eft[3]}
  \\
\pullbackpr{2}{\ectxext}{\eft}\circ\eext{2}\circ\eext{3}
  & =
\pullbackpr{2}{\ectxext}{\eft}\circ\eext{2}\circ\pullbackpr{2}{\eext{2}}{\eft[3]}.
\end{align*}
\end{comment}
It is fairly straightforward to show that both the equalities
\begin{equation*}
\eft[2]\circ\eext{2}\circ\eext{3}
  =
\eft[2]\circ\eft[3]\circ\eft[4]
\end{equation*}
and
\begin{equation*}
\eft[2]\circ\eext{2}\circ\pullbackpr{2}{\eext{2}}{\eft[3]}
  =
\eft[2]\circ\eft[3]\circ\eft[4].
\end{equation*}
hold. For the second subgoal we will show that both sides are equal to 
composite $\eext{1}\circ\eext{2}\circ\beta_3$. For the left-hand side, notice that 
the diagram
\begin{equation*}
\begin{tikzcd}
\stesysf_4
  \ar{r}{\eext{3}}
  \ar{d}[swap]{\beta_3}
  &
\stesysf_3
  \ar{r}{\eext{2}}
  \ar{d}[swap]{\beta_2}
  &
\stesysf_2
  \ar{d}{\pullbackpr{2}{\ectxext}{\eft}}
  \\
\stesysf_3
  \ar{r}[swap]{\eext{2}}
  &
\stesysf_2
  \ar{r}[swap]{\efamext}
  &
\stesysf
\end{tikzcd}
\end{equation*}
commutes by \autoref{lem:beta_ext}. For the right-hand side, notice that the
diagram
\begin{equation*}
\begin{tikzcd}[column sep=huge]
\stesysf_4
  \ar{r}{\pullbackpr{2}{\eext{2}}{\eft[3]}}
  \ar{d}[swap]{\beta_3}
  &
\stesysf_3
  \ar{dr}{\eext{2}}
  \ar{d}[swap]{\beta_2}
  \\
\stesysf_3
  \ar{r}{\pullbackpr{2}{\efamext}{\eft[2]}}
  \ar{dr}[swap]{\eext{2}}
  &
\stesysf_2
  \ar{dr}{\efamext}
  &
\stesysf_2
  \ar{d}{\pullbackpr{2}{\ectxext}{\eft}}
  \\
  {} &
\stesysf_2
  \ar{r}[swap]{\efamext}
  &
\stesysf
\end{tikzcd}
\end{equation*}
commutes (by definition, in case of the upper two squares; and by associativity,
in case of the lower square).
\end{proof}

\subsection{(Pre-)extension homomorphisms}\label{subsection:e_extension_homomorphisms}
In this subsection we start with the study of pre-extension homomorphisms, which
will include the extension homomorphisms since they will be the pre-extension
homomorphisms of which both the domain and codomain are extension algebras.
Our main examples of extension homomorphisms will be the operations of weakening
and substitution. There are some basic examples of pre-extension homomorphisms
that will be useful too, which get introduced in the this section and in
\autoref{subsection:change_of_base}. In this section, we will mainly be
interested in pre-extension homomorphisms between local pre-extension algebras.
We will end this section by proving that a retract of an extension algebra is
always an extension algebra.

\begin{defn}
A pre-extension homomorphism between extension algebras is called an extension
homomorphism.
\end{defn}

\begin{lem}
Let $\stesys$ be an extension algebra. Then
\begin{equation*}
\begin{tikzcd}[column sep=huge]
\stesyst_2
  \ar{r}{\pullbackpr{2}{\ectxext}{\eft\circ\ebd}}
  \ar{d}[swap]{\ebd[2]}
  &
\stesyst
  \ar{d}{\ebd}
  \\
\stesysf_2
  \ar{r}{\pullbackpr{2}{\ectxext}{\eft}}
  \ar{d}[swap]{\eft[2]}
  &
\stesysf
  \ar{d}{\eft}
  \\
\stesysf
  \ar{r}[swap]{\ectxext}
  &
\stesysc
\end{tikzcd}
\end{equation*}
assembles an extension homomorphism $\mathbf{e}_0:\famesys{\stesys}\to\stesys$.
Likewise, we have an extension homomorphism
$\mathbf{e}_1:\famesys{\famesys{\stesys}}\to\famesys{\stesys}$. Thus, a
pre-extension algebra is an extension algebra if and only if $\mathbf{e}_0$
and $\mathbf{e}_1$ are pre-extension homomorphisms.
\end{lem}

\begin{proof}
Immediate from the conditions of being an extension algebra.
\end{proof}

\begin{lem}[Stability under retracts]\label{esys-retract}
Suppose $f:\stesys\to\stesys'$ is a pre-extension homomorphism between
pre-extension algebras. If there is a pre-extension homomorphism $g:\stesys'\to
\stesys$ such that $g\circ f=\catid{\stesys}$ and $\stesys'$ is an extension
algebra, then $\stesys$ is an extension algebra.
\end{lem}

Before we start with the proof, note that we have the equalities
$g_2\circ f_2=\catid{\stesysf_2}$ and $g_3\circ f_3=\catid{\stesysf_3}$
under the hypotheses of the lemma.

\begin{proof}
Our first subgoal is to show that the square
\begin{equation*}
\begin{tikzcd}
\stesysf_2 
  \ar{r}{\efamext} 
  \ar{d}[swap]{\pullbackpr{2}{\ectxext}{\eft}} 
  & 
\stesysf 
  \ar{d}{\ectxext}
  \\
\stesysf
  \ar{r}[swap]{\ectxext} 
  & 
\stesysc
\end{tikzcd}
\end{equation*}
commutes. Note that in the diagram
\begin{equation*}
\begin{tikzcd}
  {}
  & 
\stesysf
  \ar{dd}[near start]{\ectxext}
  \ar{rr}{f_1}
  &
  &
\stesysf'
  \ar{dd}[near start]{\ectxext'}
  \ar{rr}{g_1}
  &
  &
\stesysf
  \ar{dd}{\ectxext}
  \\
\stesysf_2
  \ar{dd}[swap]{\pullbackpr{2}{\ectxext}{\eft}}
  \ar[crossing over]{rr}[swap,near start]{f_2}
  \ar{ur}{\efamext}
  &
  &
\stesysf_2'
  \ar{ur}[near start]{\efamext'}
  \ar[crossing over]{rr}[swap,near start]{g_2}
  &
  &
\stesysf_2
  \ar{ur}[swap,near start]{\efamext}
  \\
  {}
  &
\stesysc
  \ar{rr}[near start]{f_0}
  &
  &
\stesysc'
  \ar{rr}[near start]{g_0}
  &
  &
\stesysc
  \\
\stesysf 
  \ar{rr}[swap]{f_1}
  \ar{ur}{\ectxext}
  &
  &
\stesysf' 
  \ar[crossing over,leftarrow]{uu}[near end,swap]{\pullbackpr{2}{\ectxext'}{\eft'}}
  \ar{ur}[swap,near end]{\ectxext'}
  \ar{rr}[swap]{g_1}
  &
  &
\stesysf
  \ar[crossing over,leftarrow]{uu}[near end,swap]{\pullbackpr{2}{\ectxext}{\eft}}
  \ar{ur}[swap]{\ectxext}
\end{tikzcd}
\end{equation*}
all the faces minus the far left and far right face commute. Using that $g$
is a section of $f$, we can read off that also the far left face commutes,
completing our first subgoal.
 
For the second subgoal, note that also $\famehom{g}\circ\famehom{f}=
\catid{\famesys{\stesys}}$ and that $\famesys{\stesys'}$ is an extension algebra.
Thus we can apply what we have proven so far to conclude that the square
\begin{equation*}
\begin{tikzcd}
\stesysf_3 
  \ar{r}{\eext{2}} 
  \ar{d}[swap]{\pullbackpr{2}{\efamext}{\eft[2]}} 
  & 
\stesysf_2 
  \ar{d}{\efamext}
  \\
\stesysf_2
  \ar{r}[swap]{\efamext} 
  & 
\stesysf
\end{tikzcd}
\end{equation*}
commutes.
\end{proof}

\subsection{The change of base of (pre-)extension algebras}
\label{subsection:change_of_base}
An important construction of (pre-)extension algebras is the change of base. It
allows us to consider `parametrized homomorphisms', such as weakening and
substitution.

We will
give the definition of $\cobesys{Y}{\stesys}{g}{p}$ in \autoref{cobesys}. After
proving that the change of base of a pre-extension algebra is indeed a
pre-extension algebra (\autoref{cobesys-preext}) and that the change of base
of an extension algebra is an extension algebra (\autoref{cobesys-ext}), we
will demonstrate the above unique existence in \autoref{cobesys-existence,%
cobesys-pullback}.

The second goal in this subsection is to follow the same procedure for
$\famesys{\famesys{\stesys}}$ to show that it is equivalent to
$\cobesys{\stesysf}{\famesys{\stesys}}{\ectxext}{\eft}$. We will do this by
verifying directly that it has the universal property of the change of base
described above, because we will use the ingredients in our definition of
weakening and substitution algebras.

\begin{defn}[Change of base]\label{cobesys}
Suppose $\stesys$ is a pre-extension algebra in $\cat{C}$ and that 
$p:\stesysc\rightarrow X\leftarrow Y:g$.
Then we define the pre-extension algebra $\cobesys{Y}{\stesys}{g}{p}$ to consist of
\begin{equation*}
\begin{tikzcd}
\cobesys{Y}{\stesyst}{g}{p\circ\eft\circ\ebd}
  \ar{r}
  \ar{d}[swap]{g^\ast(\ebd)}
  &
\stesyst
  \ar{d}{\ebd}
  \\
\cobesys{Y}{\stesysf}{g}{p\circ\eft}
  \ar{r}
  \ar{d}[swap]{g^\ast(\eft)}
  &
\stesysf
  \ar{d}{\eft}
  \\
\cobesys{Y}{\stesysc}{g}{p}
  \ar{r}
  \ar{d}[swap]{\pullbackpr{1}{g}{p}}
  &
\stesysc
  \ar{d}{p}
  \\
Y \ar{r}[swap]{g}
  &
X
\end{tikzcd}
\end{equation*} 
and the operations
\begin{align*}
\cobesys{Y}{\ectxext}{g}{p} 
  & : \pullback{Y}{\stesysf}{g}{p\circ\eft}\to \pullback{Y}{\stesysc}{g}{p}\\
\cobesys{Y}{\efamext}{g}{p} 
  & : \pullback
    {\pullback{Y}{\stesysf}{g}{p\circ\eft}}
    {\pullback{Y}{\stesysf}{g}{p\circ\eft}}
    {\cobesys{Y}{\ectxext}{g}{p}}
    {g^\ast(\eft)}
  \to 
  \pullback{Y}{\stesysf}{g}{p\circ\eft}.
\end{align*}
defined by
\begin{equation*}
\cobesys{Y}{\ectxext}{g}{p} := \pullback{\catid{Y}}{\ectxext}{g}{p}
\end{equation*}
and where $\cobesys{Y}{\efamext}{g}{p}$ is defined by rendering the diagram
\begin{equation*}
\begin{tikzcd}[column sep=large]
(\cobesys{Y}{\stesysf}{g}{p\circ\eft})_2
  \ar{rr}{\pullback{\pullbackpr{2}{g}{p\circ\eft}}{\pullbackpr{2}{g}{p\circ\eft}}{\ectxext}{\eft}}
  \ar{dd}[swap]{\pullbackpr{1}{\cobesys{Y}{\ectxext}{g}{p}}{g^\ast(\eft)}}
  \ar[dotted]{dr}[swap]{\cobesys{Y}{\efamext}{g}{p}}
  &
  &
\stesysf_2
  \ar{d}{\efamext}
  \\
  {}&
\cobesys{Y}{\stesysf}{g}{p\circ\eft}
  \ar{r}{\pullbackpr{2}{g}{p\circ\eft}}
  \ar{d}[swap]{\pullbackpr{1}{g}{p\circ\eft}}
  &
\stesysf
  \ar{d}{p\circ\eft}
  \\
\cobesys{Y}{\stesysf}{g}{p\circ\eft}
  \ar{r}[swap]{\pullbackpr{1}{g}{p\circ\eft}}
  &
Y \ar{r}[swap]{g}
  &
X
\end{tikzcd}
\end{equation*} 
commutative. 
The process of obtaining the pre-extension algebra $\cobesys{Y}{\stesys}{g}{p}$ out of $\stesys$
and $g:Y\to X$ is also called the \emph{change of base}.
\end{defn}

\begin{lem}\label{cobesys-preext}
Any change of base of a pre-extension algebra is a pre-extension algebra.
\end{lem}

\begin{proof}
Let $\stesys$ be an extension algebra and consider $p:\stesysc\rightarrow X\leftarrow Y:g$.
We need to verify that the square
\begin{equation*}
\begin{tikzcd}[column sep=large]
(\pullback{Y}{\stesysf}{g}{p\circ\eft})_2
  \ar{r}{\cobesys{Y}{\efamext}{g}{p}} 
  \ar{d}[swap]{\pullbackpr{1}{\cobesys{Y}{\ectxext}{g}{p}}{g^\ast(\eft)}} 
  & 
\pullback{Y}{\stesysf}{g}{p\circ\eft}
  \ar{d}{g^\ast(\eft)}
  \\
\pullback{Y}{\stesysf}{g}{p\circ\eft}
  \ar{r}[swap]{g^\ast(\eft)} 
  & 
\pullback{Y}{\stesysc}{g}{p}
\end{tikzcd}
\end{equation*}
commutes. It is fairly obvious that
\begin{equation*}
\pullbackpr{1}{g}{p}\circ g^\ast(\eft)\circ (\cobesys{Y}{\efamext}{g}{p})
  =
\pullbackpr{1}{g}{p\circ\eft}\circ \pullbackpr{1}{\cobesys{Y}{\ectxext}{g}{p}}{g^\ast(\eft)}
\end{equation*}
and that the diagram
\begin{equation*}
\begin{tikzcd}
  {}&
  {}&
\pullback{Y}{\stesysf}{g}{p\circ\eft}
  \ar{rr}{g^\ast(\eft)}
  \ar{dr}[swap]{\pullbackpr{2}{g}{p\circ\eft}}
  &
  {}&
\pullback{Y}{\stesysc}{g}{p}
  \ar{ddr}{\pullbackpr{2}{g}{p}}
  \\
  {}&
  {}&
  {}&
\stesysf
  \ar{drr}[swap]{\eft}
  \\
(\pullback{y}{\stesysf}{g}{p\circ\eft})_2
  \ar{uurr}{\cobesys{Y}{\efamext}{g}{p}}
  \ar{rr}[swap,yshift=-.5ex]{\pullback{\pullbackpr{2}{g}{p\circ\eft}}{\pullbackpr{2}{g}{p\circ\eft}}{\ectxext}{\eft}}
  \ar{ddrr}[swap]{\pullbackpr{1}{\cobesys{Y}{\ectxext}{g}{p}}{g^\ast(\eft)}}
  &
  {}&
\stesysf_2
  \ar{ur}{\efamext}
  \ar{dr}[swap]{\eft[2]}
  &
  {}&
  {}&
\stesysc
  \\
  {}&
  {}&
  {}&
\stesysf
  \ar{urr}{\eft}
  \\
  {}&
  {}&
\pullback{Y}{\stesysf}{g}{p\circ\eft}
  \ar{rr}[swap]{g^\ast(\eft)}
  \ar{ur}{\pullbackpr{2}{g}{p\circ\eft}}
  &
  {}&
\pullback{Y}{\stesysc}{g}{p}
  \ar{uur}[swap]{\pullbackpr{2}{g}{p}}
\end{tikzcd}
\end{equation*}
commutes.
\end{proof}

\begin{thm}\label{cobesys-ext}
The change of base of an extension algebra is an extension algebra.
\end{thm}

\begin{proof}
Our first subgoal is to verify that the square
\begin{equation*}
\begin{tikzcd}[column sep=large]
(\pullback{Y}{\stesysf}{g}{p\circ\eft})_2
  \ar{r}{\cobesys{Y}{\efamext}{g}{p}} 
  \ar{d}[swap]{\pullbackpr{2}{\cobesys{Y}{\ectxext}{g}{p}}{g^\ast(\eft)}} 
  & 
\pullback{Y}{\stesysf}{g}{p\circ\eft}
  \ar{d}{\cobesys{Y}{\ectxext}{g}{p}}
  \\
\pullback{Y}{\stesysf}{g}{p\circ\eft}
  \ar{r}[swap]{\cobesys{Y}{\ectxext}{g}{p}} 
  & 
\pullback{Y}{\stesysc}{g}{p}
\end{tikzcd}
\end{equation*}
\end{proof}

The following construction is useful for defining extension homomorphisms into
`higher' extension algebras

\begin{defn}\label{cobesys-existence}
Consider a commutative diagram
\begin{equation*}
\begin{tikzcd}
\stesys'
  \ar{r}{f}
  \ar{d}[swap]{p'}
  &
\stesys
  \ar{d}{p}
  \\
Y \ar{r}[swap]{g}
  &
X
\end{tikzcd}
\end{equation*}
Then we construct $[p,f]:\stesys'\to\cobesys{Y}{\stesys}{g}{p}$
\begin{itemize}
\item by defining $[p,f]_0:\stesysc'\to\pullback{Y}{\stesysc}{g}{p}$ be the uniqe
morphism rendering the diagram
\begin{equation*}
\begin{tikzcd}[column sep=large]
\stesysc'
  \ar[bend right=10]{ddr}[swap]{p'}
  \ar[bend left=10]{rrd}{f_0}
  \ar{dr}[near end]{[p,f]_0}
  \\
  {}&
\pullback{Y}{\stesysc}{g}{p}
  \ar{r}[swap]{\pullbackpr{2}{g}{p}}
  \ar{d}{\pullbackpr{1}{g}{p}}
  &
\stesysc
  \ar{d}{p}
  \\
  {}&
Y \ar{r}[swap]{g}
  &
X
\end{tikzcd}
\end{equation*}
commutative.
\item by defining $[p,f]_1:\stesysf'\to\pullback{Y}{\stesysf}{g}{p\circ\eft}$ be the uniqe
morphism rendering the diagram
\begin{equation*}
\begin{tikzcd}[column sep=huge]
\stesysf'
  \ar[bend right=10]{ddr}[swap]{p'\circ\eft'}
  \ar[bend left=10]{rrd}{f_1}
  \ar{dr}[near end]{[p,f]_1}
  \\
  {}&
\pullback{Y}{\stesysf}{g}{p\circ\eft}
  \ar{r}[swap]{\pullbackpr{2}{g}{p\circ\eft}}
  \ar{d}{\pullbackpr{1}{g}{p\circ\eft}}
  &
\stesysc
  \ar{d}{p\circ\eft}
  \\
  {}&
Y \ar{r}[swap]{g}
  &
X
\end{tikzcd}
\end{equation*}
commutative.
\item by defining $[p,f]_t:\stesyst'\to\pullback{Y}{\stesyst}{g}{p\circ\eft\circ\ebd}$ be the uniqe
morphism rendering the diagram
\begin{equation*}
\begin{tikzcd}[column sep=huge]
\stesyst'
  \ar[bend right=10]{ddr}[swap]{p'\circ\eft'\circ\ebd'}
  \ar[bend left=10]{rrd}{f_t}
  \ar{dr}[near end]{[p,f]_t}
  \\
  {}&
\pullback{Y}{\stesysf}{g}{p\circ\eft\circ\ebd}
  \ar{r}[swap]{\pullbackpr{2}{g}{p\circ\eft\circ\ebd}}
  \ar{d}{\pullbackpr{1}{g}{p\circ\eft\circ\ebd}}
  &
\stesysc
  \ar{d}{p\circ\eft\circ\ebd}
  \\
  {}&
Y \ar{r}[swap]{g}
  &
X
\end{tikzcd}
\end{equation*}
commutative.
\end{itemize}
\end{defn}

\begin{thm}\label{cobesys-pullback}
For every diagram
\begin{equation*}
\begin{tikzcd}[column sep=large]
\stesys'
  \ar[bend right=10]{ddr}[swap]{p'}
  \ar[bend left=10]{rrd}{f}
  \ar[dotted]{dr}[near end]{[p',f]}
  \\
  {}&
\cobesys{Y}{\stesys}{g}{p}
  \ar{d}{\pullbackpr{1}{g}{p}}
  \ar{r}[swap]{\pullbackpr{2}{g}{p}}
  &
\stesys
  \ar{d}{p}
  \\
  {}&
Y \ar{r}[swap]{g}
  &
X
\end{tikzcd}
\end{equation*}
of which the outer square commutes, the pre-extension homomorphism $[p',f]$
is unique with the property that it renders the whole diagram commutative.
\end{thm}

\begin{defn}\label{famfamstesys_into}
Consider a commutative square
\begin{equation*}
\begin{tikzcd}
\stesys'
  \ar{r}{f}
  \ar{d}[swap]{p}
  &
\famesys\stesys
  \ar{d}{\eft}
  \\
\stesysf \ar{r}[swap]{\ectxext}
  &
\stesysc
\end{tikzcd}
\end{equation*}
Then we construct
\begin{equation*}
[p,f]:\stesys'\to\famesys{\famesys{\stesys}}
\end{equation*}
as follows:
\begin{itemize}
\item let $[p,f]_0:\stesysc'\to\stesysf_2$ be the unique morphism rendering
the diagram
\begin{equation*}
\begin{tikzcd}[column sep=large]
\stesysc' 
  \ar[bend left=10]{rrd}{f_0}
  \ar[swap,bend right=10]{ddr}{p}
  \ar[dotted]{dr}[near end]{[p,f]_0}
  \\
  {}&
\stesysf_2
  \ar{r}[swap]{\pullbackpr{2}{\ectxext}{\eft}}
  \ar{d}{\eft[2]}
  &
\stesysf
  \ar{d}{\eft}
  \\
  {}&
\stesysf
  \ar{r}[swap]{\ectxext}
  &
\stesysc
\end{tikzcd}
\end{equation*}
commutative.
\item Let $[p,f]_1:\stesysf'\to\stesysf_3$ be the unique morphism rendering
the diagram
\begin{equation*}
\begin{tikzcd}[column sep=large]
\stesysf'
  \ar[bend left=10]{drr}{f_1}
  \ar[swap]{dd}{\eft'}
  \ar[dotted]{dr}[near end]{[p,f]_1}
  \\
  {}&
\stesysf_3
  \ar{r}[swap]{\pullbackpr{2}{\efamext}{\eft[2]}}
  \ar{d}{\eft[2]}
  &
\stesysf_2
  \ar{d}{\eft[2]}
  \\
\stesysc'
  \ar{r}[swap]{[p,f]_0}
  &
\stesysf_2
  \ar{r}[swap]{\efamext}
  &
\stesysf
\end{tikzcd}
\end{equation*}
commutative.
\item Let $[p,f]_t:\stesyst'\to\stesyst_3$ be the unique morphism rendering
the diagram
\begin{equation*}
\begin{tikzcd}[column sep=huge]
\stesyst'
  \ar[bend left=10]{drr}{f_t}
  \ar[swap]{dd}{\eft'\circ\ebd'}
  \ar[dotted]{dr}[near end]{[p,f]_t}
  \\
  {}&
\stesyst_3
  \ar{r}[swap]{\pullbackpr{2}{\efamext}{\eft[2]\circ\ebd[2]}}
  \ar{d}[swap]{\pullbackpr{1}{\efamext}{\eft[2]\circ\ebd[2]}}
  &
\stesyst_2
  \ar{d}{\eft[2]\circ\ebd[2]}
  \\
\stesysc'
  \ar{r}[swap]{[p,f]_0}
  &
\stesysf_2
  \ar{r}[swap]{\efamext}
  &
\stesysf
\end{tikzcd}
\end{equation*}
commutative.
\end{itemize}
\end{defn}

\begin{lem}
Under the hypotheses of \autoref{famfamstesys_into}, $[p,f]$ is a pre-extension
homomorphism. Moreover, it is the unique pre-extension homomorphism for which
the diagram
\begin{equation*}
\begin{tikzcd}[column sep=large]
\stesys' 
  \ar[bend left=10]{rrd}{f}
  \ar[swap,bend right=10]{ddr}{p}
  \ar[dotted]{dr}[near end]{[p,f]}
  \\
  {}&
\famesys{\famesys{\stesys}}
  \ar{r}[swap]{\pullbackpr{2}{\ectxext}{\eft}}
  \ar{d}{\eft[2]}
  &
\famesys{\stesys}
  \ar{d}{\eft}
  \\
  {}&
\stesysf
  \ar{r}[swap]{\ectxext}
  &
\stesysc
\end{tikzcd}
\end{equation*}
commutes.
\end{lem}

\begin{lem}
Suppose $f:\stesys\to \stesys'$ is a pre-extension homomorphism and consider a morphism
$p:\stesys'\to X$ and $g:Y\to X$. Then the change of base 
$g^\ast(f):\cobesys{Y}{\stesys}{g}{p\circ f_0}\to
\cobesys{Y}{\stesys'}{g}{p}$ is a pre-extension morphism.
\end{lem}

\begin{lem}
Let $\stesys$ be a pre-extension algebra and consider $p:\stesysc\rightarrow X\leftarrow Y:g$.
Then there is an isomorphism
\begin{equation*}
\varphi:\famesys{\cobesys{Y}{\stesys}{g}{p}}
  \simeq
\cobesys{Y}{\famesys{\stesys}}{g}{p\circ\eft}
\end{equation*}
uniquely determined by
\end{lem}

\begin{proof}
This follows from the pasting lemma for pullbacks.
\end{proof}



\section{The theory and semantics of weakening}

\subsection{The theory of weakening}
\label{weakening}

When $A$ is a family in context $\Gamma$, the operation of weakening by $A$
takes a family $B$ in context $\Gamma$ and provides a family $\ctxwk{A}{B}$
in context $\ctxext{\Gamma}{A}$. The context family $\ctxwk{A}{B}$ can be seen
as the constant family over $\ctxext{\Gamma}{A}$ with value $B$. This idea will
be axiomatized in the cancellation property of weakening and substitution in
\autoref{cancellation-ws}. In \autoref{morphisms} we will take the terms 
$\unfold{\jhom{\Gamma}{A}{B}{f}}$ to be the morphisms of families from $A$ to 
$B$. These will be at the heart of the categorical structure of the theory.

The weakening operation acts on three levels: on contexts, on families and
on terms. The `action on contexts' of weakening is the action we described
above: it takes a family $B$ over $\Gamma$ to a family $\ctxwk{A}{B}$ over
$\ctxext{\Gamma}{A}$; the `action on families' of weakening takes a family
$Q$ over $\ctxext{\Gamma}{B}$ to a family $\ctxwk[\famsym]{A}{Q}$ over
$\ctxext{{\Gamma}{A}}{\ctxwk{A}{B}}$; the `action on terms' of weakening takes
a term $g$ of $Q$ to a term $\ctxwk[\tmsym]{A}{g}$ of $\ctxwk[\famsym]{A}{Q}$.
\begin{align}
& \inference
  { \jfam{\Gamma}{A}
    \jfam{\Gamma}{B}
    }
  { \jfam{{\Gamma}{A}}{\ctxwk{A}{B}}
    }
& & \inference
    { \jfameq{\Gamma}{A}{A'}
      \jfameq{\Gamma}{B}{B'}
      }
    { \jfameq{{\Gamma}{A}}{\ctxwk{A}{B}}{\ctxwk{A'}{B'}}
      }
    \\
& \inference
  { \jfam{\Gamma}{A}
    \jfam{{\Gamma}{B}}{Q}
    }
  { \jfam{{{\Gamma}{A}}{\ctxwk{A}{B}}}{\ctxwk[\famsym]{A}{Q}}
    }
& & \inference
    { \jfameq{\Gamma}{A}{A'}
      \jfameq{{\Gamma}{B}}{Q}{Q'}
      }
    { \jfameq
        {{{\Gamma}{A}}{\ctxwk{A}{B}}}
        {\ctxwk[\famsym]{A}{Q}}
        {\ctxwk[\famsym]{A'}{Q'}}
      }
    \\
& \inference
  { \jfam{\Gamma}{A}
    \jterm{{\Gamma}{B}}{Q}{g}
    }
  { \jterm{{{\Gamma}{A}}{\ctxwk{A}{B}}}{\ctxwk[\famsym]{A}{Q}}{\ctxwk[\tmsym]{A}{g}}
    }
& & \inference
    { \jfameq{\Gamma}{A}{A'}
      \jtermeq{{\Gamma}{B}}{Q}{g}{g'}
      }
    { \jtermeq
        {{{\Gamma}{A}}{\ctxwk{A}{B}}}
        {\ctxwk[\famsym]{A}{Q}}
        {\ctxwk[\tmsym]{A}{g}}
        {\ctxwk[\tmsym]{A'}{g'}}
      }
\end{align}

\subsubsection{Weakenings of extensions}
\label{comp-we}
The following rules assert that when an extended family is weakened, the
weakening distributes over the extension factors.
\begin{align}
& \inference
  { \jfam{\Gamma}{A}
    \jfam{{\Gamma}{B}}{Q}
    }
  { \jfameq
      {\ctxext{\Gamma}{A}}
      {\ctxwk{A}{\ctxext{B}{Q}}}
      {\ctxext{\ctxwk{A}{B}}{\ctxwk{A}{Q}}}
    }
  \label{comp-we-c}
  \\
& \inference
  { \jfam{\Gamma}{A}
    \jfam{{{\Gamma}{B}}{Q}}{R}
    }
  { \jfameq
      {\ctxext{{\Gamma}{A}}{\ctxwk{A}{B}}}
      {\ctxwk{A}{\ctxext{Q}{R}}}
      {\ctxext{\ctxwk{A}{Q}}{\ctxwk{A}{R}}}
    }
  \label{comp-we-f}
\end{align}
When thinking of terms of $\ctxwk{A}{B}$ as morphisms of families from $A$ to
$B$, this looks already like form of type theoretic choice. It is weaker in that
it is not stated with function types, yet it is stronger in that it states a
judgmental equality between two families.

There is also a version of this property where an extended term is weakened.
This variant is stated and proved in \autoref{comp-we-t}.

\subsubsection{Weakening of weakenings}
Weakening by a family $A$ in context $\Gamma$ brings things in context $\Gamma$
to things in context $\ctxext{\Gamma}{A}$. Since we have all the ingredients of
the theory of contexts, families and terms in the context $\Gamma$ as well it
has in particular it's own weakening by families. Suppose we have a family
$\jfam{{\Gamma}{B}}{Q}$ over $B$ in context $\Gamma$. Weakening by $Q$ brings
things from context $\ctxext{\Gamma}{B}$ to $\ctxext{{\Gamma}{B}}{Q}$. Thus
we can provide rules asserting what will happen when we first weaken by $Q$ and
then (via the action on families) by $A$. We will require the following
inference rules to be valid:
\label{comp-ww}
\begin{align}
& \inference
  { \jfam{\Gamma}{A}
    \jfam{{\Gamma}{B}}{Q}
    \jfam{{\Gamma}{B}}{R}
    }
  { \jfameq
      {{{{\Gamma}{A}}{\ctxwk{A}{B}}}{\ctxwk{A}{Q}}}
      {\ctxwk{A}{{Q}{R}}}
      {\ctxwk{{A}{Q}}{{A}{R}}}
    }
  \label{comp-ww-c}\\
& \inference
  { \jfam{\Gamma}{A}
    \jfam{{\Gamma}{B}}{Q}
    \jfam{{{\Gamma}{B}}{R}}{S}
    }
  { \jfameq
      {{{{{\Gamma}{A}}{\ctxwk{A}{B}}}{\ctxwk{A}{Q}}}{\ctxwk{A}{{Q}{R}}}}
      {\ctxwk{A}{{Q}{S}}}
      {\ctxwk{{A}{Q}}{{A}{S}}}
    }
  \label{comp-ww-f}\\
& \inference
  { \jfam{\Gamma}{A}
    \jfam{{\Gamma}{B}}{Q}
    \jterm{{{\Gamma}{B}}{R}}{S}{k}
    }
  { \jtermeq
      {{{{{\Gamma}{A}}{\ctxwk{A}{B}}}{\ctxwk{A}{Q}}}{\ctxwk{A}{{Q}{R}}}}
      {\ctxwk{A}{{Q}{S}}}
      {\ctxwk{A}{{Q}{k}}}
      {\ctxwk{{A}{Q}}{{A}{k}}}
    }
  \label{comp-ww-t}
\end{align}

\begin{rmk}
As an important special case of these inference rules we have the following
valid inference rules:
\begin{align*}
& \inference
  { \jfam{\Gamma}{A}
    \jfam{\Gamma}{B}
    \jfam{\Gamma}{C}
    }
  { \jfameq
      {{{\Gamma}{A}}{\ctxwk{A}{B}}}
      {\ctxwk{A}{{B}{C}}}
      {\ctxwk{{A}{B}}{{A}{C}}}
    }
  \\
& \inference
  { \jfam{\Gamma}{A}
    \jfam{\Gamma}{B}
    \jfam{{\Gamma}{C}}{R}
    }
  { \jfameq
      {{{{\Gamma}{A}}{\ctxwk{A}{B}}}{\ctxwk{A}{{B}{C}}}}
      {\ctxwk{A}{{B}{R}}}
      {\ctxwk{{A}{B}}{{A}{R}}}
    }
  \\
& \inference
  { \jfam{\Gamma}{A}
    \jfam{\Gamma}{B}
    \jterm{{\Gamma}{C}}{R}{h}
    }
  { \jtermeq
      {{{{\Gamma}{A}}{\ctxwk{A}{B}}}{\ctxwk{A}{{B}{C}}}}
      {\ctxwk{A}{{B}{R}}}
      {\ctxwk{A}{{B}{h}}}
      {\ctxwk{{A}{B}}{{A}{h}}}
    }
\end{align*}
Moreover, we have
\begin{equation*}
\inference
  { \jfam{\Gamma}{A}
    \jfam{\Gamma}{B}
    \jterm{\Gamma}{C}{z}
    }
  { \jtermeq
      {{{\Gamma}{A}}{\ctxwk{A}{B}}}
      {\ctxwk{A}{{B}{C}}}
      {\ctxwk{A}{{B}{z}}}
      {\ctxwk{{A}{B}}{{A}{z}}}
    }
\end{equation*}
\end{rmk}

\subsubsection{Currying for weakening}
\label{comp-ew}
The rules expressing that extension is compatible with weakening assert that
weakening by an extension is the same thing as weakening twice in the
appropriate way.
\begin{align}
& \inference
  { \jfam{\Gamma}{A}
    \jfam{{\Gamma}{A}}{P}
    \jfam{\Gamma}{B}
    }
  { \jfameq
      {{{\Gamma}{A}}{P}}
      {\ctxwk{\ctxext{A}{P}}{B}}
      {\ctxwk{P}{{A}{B}}}
    }
  \label{comp-ew-c}\\
& \inference
  { \jfam{\Gamma}{A}
    \jfam{{\Gamma}{A}}{P}
    \jfam{{\Gamma}{B}}{Q}
    }
  { \jfameq
      {{{{\Gamma}{A}}{P}}{\ctxwk{P}{{A}{B}}}}
      {\ctxwk{\ctxext{A}{P}}{Q}}
      {\ctxwk{P}{{A}{Q}}}
    }
  \label{comp-ew-f}\\
& \inference
  { \jfam{\Gamma}{A}
    \jfam{{\Gamma}{A}}{P}
    \jterm{{\Gamma}{B}}{Q}{g}
    }
  { \jtermeq
      {{{{\Gamma}{A}}{P}}{\ctxwk{P}{{A}{B}}}}
      {\ctxwk{P}{{A}{Q}}}
      {\ctxwk{\ctxext{A}{P}}{g}}
      {\ctxwk{P}{{A}{g}}}
    } 
  \label{comp-ew-t}
\end{align}


\subsection{Pre-weakening algebras}
\begin{defn}
Let $\stesys$ be an extension algebra in $\cat{C}$. A pre-weakening operation
on $\stesys$ is an extension homomorphism 
$ \mathbf{w}(\stesys)
    :
  \cobesys{\stesysf}{\famesys{\stesys}}{\eft}{\eft}
    \to
  \famesys{\famesys{\stesys}}$
for which the diagram
\begin{equation*}
\begin{tikzcd}[column sep=large]
\cobesys{\stesysf}{\famesys{\stesys}}{\eft}{\eft}
  \ar{r}{\mathbf{w}(\stesys)}
  \ar{dr}[swap]{\pullbackpr{1}{\eft}{\eft}}
  &
\famesys{\famesys{\stesys}}
  \ar{d}{\eft[2]}
  \\
& \stesysf
\end{tikzcd}
\end{equation*}
commutes.
\end{defn}

\begin{defn}
Let $\stesys$ be an extension algebra with pre-weakening operation
$\mathbf{w}(\stesys)$. Then $\famesys{\stesys}$ has the pre-weakening operation
$\mathbf{w}(\famesys{\stesys})$ which is uniquely determined by rendering the
diagram
\begin{equation*}
\begin{tikzcd}[column sep=large]
\cobesys{\stesysf_2}{\famesys{\famesys{\stesys}}}{\eft[2]}{\eft[2]}
  \ar{rr}{%
      \pullback{\beta_1}{\boldsymbol{\beta}}{\eft}{\eft}
    }
  \ar[bend right]{ddr}[swap]{\pullbackpr{1}{\eft[2]}{\eft[2]}}
  \ar[dotted]{dr}{\mathbf{w}(\famesys{\stesys})}
  &
  {}&
\cobesys{\stesysf}{\famesys{\stesys}}{\eft}{\eft}
  \ar{r}{\mathbf{w}(\stesys)}
  &
\famesys{\famesys{\stesys}}
  \ar{d}{\boldsymbol{\beta}}
  \\
  {}&
\famesys{\famesys{\famesys{\stesys}}}
  \ar{r}{\boldsymbol{\beta}_\mathbf{2}}
  \ar{d}[swap]{\eft[3]}
  &
\famesys{\famesys{\stesys}}
  \ar{d}{\eft[2]}
  \ar{r}{\boldsymbol{\beta}}
  &
\famesys{\stesys}
  \ar{d}{\eft}
  \\
  {}&
\stesysf_2
  \ar{r}[swap]{\efamext}
  &
\stesysf
  \ar{r}[swap]{\ectxext}
  &
\stesysc
\end{tikzcd}
\end{equation*}
commutative.
\end{defn}

\begin{defn}
A pre-weakening algebra $\stesys$ in $\cat{C}$ is an extension algebra $\stesys$ 
with a pre-weakening operation 
$ \mathbf{w}(\stesys)
    :
  \cobesys{\stesysf}{\famesys{\stesys}}{\eft}{\eft}
    \to
  \famesys{\famesys{\stesys}}$
for which the diagram
\begin{equation*}
\begin{tikzcd}[column sep=15em]
\cobesys{\stesysf_2}{\famesys{\stesys}}{\eft\circ\eft[2]}{\eft}
  \ar[bend right=10]{dr}[swap]%
    { [ \pullbackpr{1}{\eft\circ\eft[2]}{\eft},%
        \mathbf{w}(\stesys)%
          \circ%
        (\pullback{\efamext}{\catid{\famesys{\stesys}}}{\eft}{\eft})%
        ]%
      }
  \ar{r}{
    [ \pullbackpr{1}{\eft\circ\eft[2]}{\eft},%
      \mathbf{w}(\stesys)%
        \circ%
      (\pullback{\eft[2]}{\catid{\famesys{\stesys}}}{\eft}{\eft})%
      ]}%
  &
\cobesys{\stesysf_2}{\famesys{\famesys{\stesys}}}{\eft[2]}{\eft[2]}
  \ar{d}{\mathbf{w}(\famesys{\stesys})}
  \\
  {}&
\famesys{\famesys{\famesys{\stesys}}}
\end{tikzcd}
\end{equation*}
commutes. This condition is called \emph{Currying for weakening}.
\end{defn}

\begin{lem}
If $\stesys$ is a pre-weakening algebra, then so is $\famesys{\stesys}$. 
\end{lem}

\begin{proof}
\end{proof}

\begin{defn}
A pre-weakening morphism between preweakening algebras $\stesys$ and $\stesys'$ is an
extension homomorphism $f:\stesys\to \stesys'$ such that additionally the diagram
\begin{equation*}
\begin{tikzcd}[column sep=large]
\cobesys{\stesysf}{\famesys{\stesys}}{\eft}{\eft}
  \ar{d}[swap]{\mathbf{w}(\stesys)}
  \ar{r}{\pullback{f_1}{\famehom{f}}{\eft'}{\eft'}}
  &
\cobesys{\stesysf'}{\famesys{\stesys'}}{\eft'}{\eft'}
  \ar{d}{\mathbf{w}(\stesys')}
  \\
\famesys{\famesys{\stesys}}
  \ar{r}[swap]{\famehom{\famehom{f}}}
  &
\famesys{\famesys{\stesys'}}
\end{tikzcd}
\end{equation*}
commutes.
\end{defn}

\begin{defn}
Let $\stesys$ be a pre-weakening algebra and consider $p:\stesysc\rightarrow X\leftarrow Y:p$.
Then we define
\begin{equation*}
\mathbf{w}(\cobesys{Y}{\stesys}{g}{p})
  :
\cobesys
  { (\pullback{Y}{\stesysf}{g}{p\circ\eft})}
  { \famesys{\cobesys{Y}{\stesys}{g}{p}}}
  { g^\ast(\eft)}
  { g^\ast(\eft)}
  \to
\famesys{\famesys{\cobesys{Y}{\stesys}{g}{p}}}
\end{equation*}
to be the unique extension homomorphism rendering the diagram
\begin{equation*}
\begin{tikzcd}
\cobesys
  { \pullback{Y}{\stesysf}{g}{p\circ\eft}}
  { \famesys{\cobesys{Y}{\stesys}{g}{p}}}
  { g^\ast(\eft)}
  { g^\ast(\eft)}
  \ar[bend right]{ddr}[swap]{\pullbackpr{1}{g^\ast(\eft)}{g^\ast(\eft)}}
  \ar{rr}{%
    \pullback
      { \pullbackpr{2}{g}{p\circ\eft}}
      { \boldsymbol{\pi}_\mathbf{2}(g,p\circ\eft)}
      { \eft}
      { \eft}
    }
  \ar[dotted]{dr}{\mathbf{w}(\cobesys{Y}{\stesys}{g}{p})}
  &
  {}&
\cobesys{\stesysf}{\famesys{\stesys}}{\eft}{\eft}
  \ar{r}{\mathbf{w}(\stesys)}
  &
\famesys{\famesys{\stesys}}
  \ar{d}{\boldsymbol{\beta}}
  \\
  {}&
\famesys{\famesys{\cobesys{Y}{\stesys}{g}{p}}}
  \ar{r}{\beta}
  \ar{d}{g^\ast(\eft)_1}
  &
\famesys{\cobesys{Y}{\stesys}{g}{p}}
  \ar{r}{\boldsymbol{\pi}_2(g,p\circ\eft)}
  \ar{d}{g^\ast(\eft)}
  &
\famesys{\stesys}
  \ar{d}{\eft}
  \\
  {}&
\pullback{Y}{\stesysf}{g}{p\circ\eft}
  \ar{r}[swap]{\cobesys{Y}{\ectxext}{g}{p}}
  &
\pullback{Y}{\stesysc}{g}{p}
  \ar{r}[swap]{\pullbackpr{2}{g}{p}}
  &
\stesysc
\end{tikzcd}
\end{equation*}
commutative.
\end{defn}

\subsection{Weakening algebras}
Since we have shown that the property of being a pre-weakening algebra is closed
under the relevant operations, we can make the following definition:

\begin{defn}
A weakening algebra is a pre-weakening algebra $\stesys$ with the property that
$\mathbf{w}(\stesys)$ is a pre-weakening morphism.
\end{defn}

\begin{defn}
A weakening homomorphism is a pre-weakening homomorphism such that the domain
and codomain are weakening algebras.
\end{defn}

\begin{thm}
Suppose $\stesys$ is a weakening algebra, then so is $\famesys{\stesys}$
\end{thm}

\begin{thm}
The change of base of any weakening algebra is again a weakening algebra.
\end{thm}



\section{The theory and semantics of projections}

\subsection{The inference rules of projections}
\label{identityterms}

We formulate the theory of projections on top of the theory of weakening
described in \autoref{weakening}. In the theory of projections we introduce
the identity terms. The identity term at a family $A$ in context $\Gamma$ is
a term $\jterm{{\Gamma}{A}}{\ctxwk{A}{A}}{\idtm{A}}$, so weakening is necessary.
Together with weakening, the identity terms will provide for the projections.
The projection from $\cftext{A}{P}$ to $A$ will be the term
$\jterm{{\Gamma}{{A}{P}}}{\ctxwk{\ctxext{A}{P}}{A}}{\ctxwk{P}{\idtm{A}}}$ and
will be studied extensively in \autoref{extension-on-terms}.

The following inference rules are required to be valid:
\begin{align}
& \inference
  { \jfam{\Gamma}{A}
    }
  { \unfoldall{\jhom{\Gamma}{A}{A}{\idtm{A}}}
    }
& & \inference
    { \jfameq{\Gamma}{A}{A'}
      }
    { \unfoldall{\jhomeq{\Gamma}{A}{A}{\idtm{A}}{\idtm{A'}}}
      }
\end{align}

\subsubsection{The identity term of an extended family}
The fact that identity terms are compatible with extensions is derived in
\autoref{comp-ie}.

\subsubsection{The identity term of a weakened family}
\label{comp-wi}
The identity term of a weakened family is the weakened identity term:
\begin{equation}
\inference
  { \jfam{\Gamma}{A}
    \jfam{{\Gamma}{B}}{Q}
    }
  { \unfoldall{\jhomeq
      {{{\Gamma}{A}}{\ctxwk{A}{B}}}
      {\ctxwk{A}{Q}}
      {\ctxwk{A}{Q}}
      {\ctxwk{A}{\idtm{Q}}}
      {\idtm{\ctxwk{A}{Q}}}
      }
    }
  \label{comp-wi-t}
\end{equation}

\begin{rmk}
An important special case is the judgmental equality
\begin{equation*}
\unfoldall{\jhomeq
      {{\Gamma}{A}}
      {\ctxwk{A}{B}}
      {\ctxwk{A}{B}}
      {\ctxwk{A}{\idtm{B}}}
      {\idtm{\ctxwk{A}{B}}}
      }
\end{equation*}
for a family $\jfam{\Gamma}{B}$.
\end{rmk}


\subsection{Projection algebras}
\begin{defn}
A pre-projection algebra is a weakening algebra $\stesys$ for which there is a term
$\mathbf{i}:\stesysf\to \stesyst_2$ such that the diagram
\begin{equation*}
\begin{tikzcd}[column sep=large]
\stesysf \ar{r}{\mathbf{i}} \ar{d}[swap]{\Delta_{\eft}} & \stesyst_2 \ar{d}{\ebd[2]}\\
\pullback{\stesysf}{\stesysf}{\eft}{\eft} \ar{r}[swap]{w(\stesys)_0} & \stesysf_2
\end{tikzcd}
\end{equation*}
commutes. In this diagram $\Delta_{\eft}:\stesysf\to \pullback{\stesysf}{\stesysf}{\eft}{\eft}$ is the diagonal.
\end{defn}

\begin{defn}
A pre-projection homomorphism from $\stesys$ to $\stesys'$ is a weakening homomorphism
$f:\stesys\to \stesys'$ such that the square
\begin{equation*}
\begin{tikzcd}[column sep=large]
\stesyst_2
  \ar{r}{{f_t}_1}
  &
\stesyst_2'
  \\
\stesysf \ar{r}[swap]{f_1}
  \ar{u}{\mathbf{i}}
  &
\stesysf'
  \ar{u}[swap]{\mathbf{i}'}
\end{tikzcd}
\end{equation*}
commutes.
\end{defn}

\begin{lem}
The change of base of a pre-projection algebra is again a pre-projection algebra.
\end{lem}

\begin{lem}
If $CFT$ is a pre-projection algebra, then so is $\mathbf{F}_{CFT}$, where
$\mathbf{F}_{\mathbf{i}}$ is defined to be $F\times_{e_0,c}\mathbf{i}$ is
a pre-projection algebra.
\end{lem}

\begin{defn}
A projection algebra is a pre-projection algebra for which weakening is a
pre-projection homomorphism.
\end{defn}

\begin{cor}
The change of base of a projection algebra is again a projection algebra.
\end{cor}

\begin{cor}
If $CFT$ is a projection algebra, then so is $\mathbf{F}_{CFT}$, where
$\mathbf{F}_{\mathbf{i}}$ is defined to be $F\times_{e_0,c}\mathbf{i}$ is
a projection algebra.
\end{cor}



\subsection{Substitution algebras}

\begin{defn}
A \emph{pre-substitution} for an extension algebra $\stesys$ is an
extension homomorphism
\begin{equation*}
\mathbf{s}(\stesys):\cobesys{\stesyst}{\famesys{\famesys{\stesys}}}{\ebd}{\eft[2]}\to \famesys{\stesys}
\end{equation*}
for which the square
\begin{equation*}
\begin{tikzcd}[column sep=large]
\pullback{\stesyst}{\stesysf_2}{\ebd}{\eft[2]}
  \ar{r}{s(\stesys)_0}
  \ar{d}[swap]{\ebd\circ\pullbackpr{1}{\ebd}{\eft[2]}}
  &
\stesysf 
  \ar{d}{\eft}
  \\
\stesysf 
  \ar{r}[swap]{\eft}
  &
\stesysc
\end{tikzcd}
\end{equation*}
commutes. A \emph{pre-substitution algebra} is an extension algebra
together with a pre-substitution.
\end{defn}

\begin{defn}
A \emph{pre-substitution homomorphism} is an extension homomorphism $f:\stesys'\to \stesys$
for which the square
\begin{equation*}
\begin{tikzcd}[column sep=huge]
\cobesys{\stesyst'}{\famesys{\famesys{\stesys'}}}{\ebd'}{\eft[2]'}
  \ar{r}{\pullback{f^t}{\famehom{\famehom{f}}}{\ebd}{\eft[2]}}
  \ar{d}[swap]{\mathbf{s}'(\stesys')}
  &
\cobesys{\stesyst}{\famesys{\famesys{\stesys}}}{\ebd}{\eft[2]}
  \ar{d}{\mathbf{s}(\stesys)}
  \\
\famesys{\stesys'}
  \ar{r}[swap]{\famehom{f}}
  &
\famesys{\stesys}
\end{tikzcd}
\end{equation*}
commutes.
\end{defn}

\begin{lem}
The change of base of a pre-substitution algebra is again a pre-substitution algebra.
\end{lem}

\begin{lem}
If $\stesys$ is a pre-substitution algebra, then so is $\famesys{\stesys}$ with
$\mathbf{s}(\famesys{\stesys})$ defined to be the unique extension homomorphism
rendering the diagram
\begin{equation*}
\begin{tikzcd}
\cobesys{\stesyst_2}{\famesys{\famesys{\famesys{\stesys}}}}{\ebd[2]}{\eft[3]}
  \ar[dotted]{dr}{\mathbf{s}(\famesys{\stesys})}
  \ar{rr}{\pullback{\pullbackpr{2}{\ectxext}{\eft\circ\ebd}}{\boldsymbol{\beta}_\mathbf{2}}{\ebd}{\eft[2]}}
  \ar{dd}[swap]{\ebd[2]\circ\pullbackpr{1}{\ebd[2]}{\eft[3]}}
  &
  {}&
\cobesys{\stesyst}{\famesys{\famesys{\stesys}}}{\ebd}{\eft[2]}
  \ar{d}{\mathbf{s}(\stesys)}
  \\
  {}&
\famesys{\famesys{\stesys}}
  \ar{r}{\boldsymbol{\beta}}
  \ar{d}{\eft[2]}
  &
\famesys{\stesys}
  \ar{d}{\eft}
  \\
\stesysf_2
  \ar{r}[swap]{\eft[2]}
  &
\stesysf
  \ar{r}[swap]{\ectxext}
  &
\stesysc
\end{tikzcd}
\end{equation*}
commutative.
\end{lem}

\begin{proof}
The requirement on pre-substitutions holds by construction.
\end{proof}

It makes sense now to consider the possibility that the pre-substitution
itself is a pre-substitution homomorphism.

\begin{defn}
A \emph{substitution algebra} is a pre-substitution algebra for which substitution is
a pre-substitution homomorphism.
\end{defn}

\begin{cor}
The change of base of a substitution algebra is again a substitution algebra.
\end{cor}

\begin{cor}
If $\stesys$ is a substitution algebra, then so is $\famesys{\stesys}$.
\end{cor}


\section{The theory of the empty context and families}

\subsection{The empty context and the empty families}
\label{empty}
We introduce an empty context and an family over $\Gamma$ for every context $\Gamma$. 
It has been suggested by some to only include empty families and not an
empty context because an empty context is not necessary, but we do have several
reasons to include them. Having an empty context requires also rules asserting 
that a context is the same thing as a family over the empty context and this
gives the categorical structure on contexts for free once one has it for 
families. The main ingredients that will be missing from the theory once an
empty context is avoided are weakening by a context and identity morphisms from
a context to itself. Including these by hand also requires to formulate all the
compatibility rules involving weakening once more for the cases of weakening by
a context and identity terms at contexts. We prefer to state these rules once
and only once and including an empty context helps in this respect.

We also prefer to have our scheme of compatibility rules as symmetrical as
possible. The structure of type dependency should look exactly the same in the
default case as in any context. In that respect, having an empty family but not
an empty context seems a bit odd. Also, some sets of compatibility rules 
(like the rules stating that extension is compatible with the empty families)
will become assymetrical as a result of not including an empty context. Moreover,
we would eventually like to include a stratification of the theory by means of
a type judgment (asserting that a family in a context $\Gamma$ is a type) and
study closed types (i.e.~types in the empty context). It would be possible to
provide a notion of closed types without having an empty context, but this would
have to be formulated separately and we would have to restate all the rules for
types (if any) for closed types all over again.

One of the main uses of the empty context and the empty families is that we
get the property that the `action on contexts' of an operation is compatible
with its `action on families'.

The empty family over a context $\Gamma$ is introduced by the following rule
inference rule:
\begin{align}
& \inference
  { }
  { \jctx{\emptyc}
    }
  \\
& \inference
  { \jctx{\Gamma}
    }
  { \jfam{\Gamma}{\emptyf[\Gamma]}
    }
  \\
& \inference
  { \jctxeq{\Gamma}{\Gamma'}
    }
  { \jfameq{\Gamma}{\emptyf[\Gamma]}{\emptyf[\Gamma']}
    }
\end{align}

By regarding contexts as families of contexts over the empty context, we
enable ourselves also to speak of terms of contexts. A term of a context
$\Gamma$ is a term of the family $\Gamma$ over the empty context. These ideas
are captured in the following convertibility rules:
\begin{align}
& \inference
  { \jctx{\Gamma}
    }
  { \jfam{\emptyc}{i(\Gamma)}
    } 
  &
& \inference
  { \jctxeq{\Gamma}{\Delta}
    }
  { \jfameq{\emptyc}{i(\Gamma)}{i(\Delta)}
    }
\end{align}

The reader may wonder whether the empty family $\emptyf$ always has a
term. This shall follow from the rules stating the compatibility of extension
with the empty families in \autoref{comp-0e} below and from
identity terms (\autoref{identityterms}).

\subsubsection{Compatibility of extension with the empty context and families}
In the following set of inference rules we state that the empty context and
the empty family are neutral objects for both context extension (the first two
rules) and family extension (the last two rules).
\label{comp-e0}\label{comp-0e}
\begin{align}
& \inference
  { \jctx{\Gamma}
    }
  { \jctxeq{\ctxext{\emptyc}{i(\Gamma)}}{\Gamma}
    }
  \label{comp-0e-c}
  \\
& \inference
  { \jctx{\Gamma}
    }
  { \jctxeq{\ctxext{\Gamma}{\emptyf}}{\Gamma}
    }
  \label{comp-e0-c}\\
& \inference
  { \jfam{\Gamma}{A}
    }
  { \jfameq{\Gamma}{\ctxext{\emptyf}{A}}{A}
    }
  \label{comp-0e-f}
  \\
& \inference
  { \jfam{\Gamma}{A}
    }
  { \jfameq{\Gamma}{{A}{\emptyf}}{A}
    }
  \label{comp-e0-f}
  \\
& \inference
  { \jfam{\emptyc}{A}
    }
  { \jfameq{\emptyc}{i(\ctxext{\emptyc}{A})}{A}
    }
\end{align}

\subsubsection{Family extension restricted to the empty context}
Strictly speaking we should have used a different notation for context extension
as for family extension, because the following rule asserting that family extension
in the empty context is the same thing as context extension would look tautological
without a difference. So let us denote, only for the moment, context extension
of $\Gamma$ by $A$ by $(\ctxext{\Gamma}{A})^c$ and family extension of $A$ by
$P$ in context $\Gamma$ by $(\ctxext{A}{P})^\famsym$. 

Note that we may consider a context $\Gamma$ as a family over $\emptyc$ and
a family $\jfam{\Gamma}{A}$ as a family $\jfam{{\emptyc}{\Gamma}}{A}$. 
Therefore we will require the following rule:
\begin{align}
& \inference
  { \jfam{\Gamma}{A}
    }
  { \jctxeq{\ctxext{\Gamma}{A}}{\ctxext{i(\Gamma)}{A}}
    }
\end{align}
Note that this rule actually justifies that we have not utilized two different
notations for context extension and family extension.


\subsection{Extension algebras with empty context and families}

\begin{defn}
An extension algebra $\stesys$ is said to have \emph{empty families} if there
is a section
\begin{equation*}
\phi_1(\stesys):\stesysc\to\stesysf
\end{equation*}
of $\eft$, satisfying the following additional properties:
\begin{enumerate}
\item $\phi_1(\stesys)$ is also a section of $\ectxext$.
\item The unique morphism $\phi_2$ rendering the diagram
\begin{equation*}
\begin{tikzcd}[column sep=huge]
\stesysf
  \ar[bend left=20]{drr}{\phi_1(\stesys)\circ\ectxext}
  \ar[equals,bend right=20]{ddr}
  \ar[dotted]{dr}{\phi_2}
  \\
  {}&
\stesysf_2
  \ar{d}{\eft[2]}
  \ar{r}{\pullbackpr{2}{\ectxext}{\eft}}
  &
\stesysf
  \ar{d}{\eft}
  \\
  {}&
\stesysf
  \ar{r}[swap]{\ectxext}
  &
\stesysc
\end{tikzcd}
\end{equation*}
commutative, is a section of $\efamext$.
\item The unique morphism $\iota_1$ rendering the diagram
\begin{equation*}
\begin{tikzcd}[column sep=huge]
\stesysf
  \ar[equals,bend left=20]{drr}
  \ar[bend right=20]{ddr}[swap]{\phi_1(\stesys)\circ\eft}
  \ar[dotted]{dr}{\iota_1}
  \\
  {}&
\stesysf_2
  \ar{d}{\eft[2]}
  \ar{r}[swap]{\pullbackpr{2}{\ectxext}{\eft}}
  &
\stesysf
  \ar{d}{\eft}
  \\
  {}&
\stesysf
  \ar{r}[swap]{\ectxext}
  &
\stesysc
\end{tikzcd}
\end{equation*}
commutative, is a section of $\efamext$.
\end{enumerate}
\end{defn}

\begin{defn}
A homomorphism of extension algebras with empty families is an extension
homomorphism $f:\stesys'\to\stesys$ for which the diagram
\begin{equation*}
\begin{tikzcd}
\stesysf'
  \ar{r}{f_1}
  &
\stesysf
  \\
\stesysc'
  \ar{u}{\phi_1(\stesys')}
  \ar{r}[swap]{f_0}
  &
\stesysc
  \ar{u}[swap]{\phi_1(\stesys)}
\end{tikzcd}
\end{equation*}
commutes.
\end{defn}

\begin{lem}
Suppose $\stesys$ is an extension algebra with empty families. Then
$\famesys{\stesys}$ is an extension algebra with empty families, with
$\phi_1(\famesys{\stesys}):=\phi_2$. 
\end{lem}

\begin{lem}
Suppose $\stesys$ is an extension algebra with empty families and consider
$p:\stesysc\rightarrow X\leftarrow Y:p$. Then
$\cobesys{Y}{\stesys}{g}{p}$ is an extension algebra with empty families
with $\phi_1(\cobesys{Y}{\stesys}{g}{p}):=g^\ast(\phi_1)$.
\end{lem}



\section{Joining the theories into dependent type theory}

\subsection{Joining the theory of projections with the theory of substitution}
\label{sec:esystem-equalities}

\subsubsection{Weakening of an empty family}
The following inference rules express that when the empty family is
weakened, the result is the empty family.
\label{comp-w0}
\begin{align}
& \inference
  { \jfam{\Gamma}{A}
    }
  { \jfameq{{\Gamma}{A}}{\ctxwk{A}{\emptyf}}{\emptyf}
    }
  \label{comp-w0-c}\\
& \inference
  { \jfam{\Gamma}{A}
    \jfam{\Gamma}{B}
    }
  { \jfameq
    {{{\Gamma}{A}}{\ctxwk{A}{B}}}
    {\ctxwk[\famsym]{A}{\emptyf}}
    {\emptyf}
    }
  \label{comp-w0-f}
\end{align}

Because a family over $\Gamma$ is the same as a family over 
$\ctxext{\Gamma}{\emptyf}$ we can apply both the action on contexts and the
action on families of weakening to a family $B$ over $\Gamma$. When we apply
the action on families, we obtain a family $\ctxwk[\famsym]{A}{B}$ over the
context $\ctxext{{\Gamma}{A}}{\ctxwk{A}{\emptyf}}$. However, since we have
postulated the judgmental equalities $\ctxwk{A}{\emptyf}\jdeq\emptyf$ and
$\ctxext{{\Gamma}{A}}{\emptyf}\jdeq\ctxext{\Gamma}{A}$, we see that we can
compare $\ctxwk[\famsym]{A}{B}$ with $\ctxwk{A}{B}$. The following inference
rule postulates that these two are judgmentally equal:
\begin{equation}
\inference
{ \jfam{\Gamma}{A}
  \jfam{\Gamma}{B}
  }
{ \jfameq{{\Gamma}{A}}{\ctxwk[\famsym]{A}{B}}{\ctxwk{A}{B}}
  }
\end{equation}
Due to this rule, the action on contexts and the action on families of weakening
are compatible with each other and consequently there can be no possible
confusion when we omit the annotations $\famsym$ and $\tmsym$ alltogether. In
the future, the weakening of a family $Q$ over $\ctxext{\Gamma}{B}$ shall
be denoted just by $\ctxwk{A}{Q}$ and likewise the weakening of a term $g$ of
$Q$ shall be denoted by $\ctxwk{A}{g}$.

Because a family $B$ over $\Gamma$ can be treated as a family by the operation
of weakening, weakening also acts on the terms of $B$. The weakening of a term
$y$ of $B$ by $A$ can be seen as the constant term (or function) of the
family $\ctxwk{A}{B}$.

\subsubsection{Weakening by the empty family}
Note that we can also weaken by the empty family over $\Gamma$.
Weakening by the empty family $\emptyf$ over a context $\Gamma$ leaves families, 
their terms, families over those families and terms of those unchanged:
\label{comp-0w}
\begin{align}
& \inference
  { \jany{\Gamma}{e}
    }
  { \janyeq{\Gamma}{\ctxwk{\emptyf}{e}}{e}
    }
  \label{comp-0w-any}
\end{align}

\subsubsection{Fibers of an empty family}
The following inference rules establish that the fibers of the empty family are 
the empty families:
\label{comp-s0}
\begin{align}
& \inference
  { \jterm{\Gamma}{A}{x}
    }
  { \jfameq{\Gamma}{\subst{x}{\emptyf}}{\emptyf}
    }
  \label{comp-s0-c}
  \\
& \inference
  { \jterm{\Gamma}{A}{x}
    \jfam{{\Gamma}{A}}{P}
    }
  { \jfameq
      {{\Gamma}{\subst{x}{P}}}
      {\subst{x}{\emptyf}}
      {\emptyf}
    }
  \label{comp-s0-f}
\end{align}

We use the above rule to state the compatibility of the action on families of
substitution with the action on contexts of substitution. Note that a family
$P$ over $\ctxext{\Gamma}{A}$ may be regarded as a family over
$\ctxext{{\Gamma}{A}}{\emptyf}$. Thus, we may consider the family
$\subst[\famsym]{x}{P}$ over $\ctxext{\Gamma}{\subst{x}{\emptyf}}$. Since
$\subst{x}{\emptyf}$ is judgmentally equal to the empty family, we may compare
$\subst[\famsym]{x}{P}$ with $\subst{x}{P}$:
\begin{equation}
\inference
{ \jfam{{\Gamma}{A}}{P}
  }
{ \jfameq{\Gamma}{\subst[\famsym]{x}{P}}{\subst{x}{P}}
  }
\end{equation}
Due to this rule we need not make the annotations $\famsym$ and $\tmsym$ in
the notation for substitution anymore and thus we shall omit them from now on.
Note that because a family $P$ over $\ctxext{\Gamma}{A}$ is eligible for
application of the action on families of substitution, we may also substitute
terms of $P$. Thus, given terms $\jterm{\Gamma}{A}{x}$ and $\jterm{{\Gamma}{A}}{P}{f}$,
we get a term $\jterm{\Gamma}{\subst{x}{P}}{\subst{x}{f}}$, the \emph{value of
$f$ at $x$}.

In \autoref{morphisms} we will use a combination of weakening and substitution
to define composition of morphisms of families. However, we have to rely
on the cancellation rule stated in \autoref{cancellation-ws} before we can
meaningfully state the definition of composition.

\subsubsection{Fibers of weakenings}\label{comp-sw}
The following rules assert what happens when we first weaken by a family
$Q$ over $P$ in context $\ctxext{\Gamma}{A}$ and then substitute by a term
$x$ of $A$:
\begin{align}
& \inference
  { \jterm{\Gamma}{A}{x}
    \jfam{{{\Gamma}{A}}{P}}{Q}
    \jany{{{\Gamma}{A}}{P}}{e}
    }
  { \janyeq
      {{{\Gamma}{\subst{x}{P}}}{\subst{x}{Q}}}
      {\subst{x}{\ctxwk{Q}{e}}}
      {\ctxwk{\subst{x}{Q}}{\subst{x}{e}}}
    }
  \label{comp-sw-any}
\end{align}

\begin{rmk}
As an important special case of these inference rules we have the following
valid inference rules:
\begin{align*}
& \inference
  { \jterm{\Gamma}{A}{x}
    \jfam{{\Gamma}{A}}{P}
    \jany{{\Gamma}{A}}{e}
    }
  { \janyeq
      {{\Gamma}{\subst{x}{P}}}
      {\subst{x}{\ctxwk{P}{e}}}
      {\ctxwk{\subst{x}{P}}{\subst{x}{e}}}
    }
\end{align*}
Moreover, we get
\begin{equation*}
\inference
  { \jterm{\Gamma}{A}{x}
    \jfam{{\Gamma}{A}}{P}
    \jterm{{\Gamma}{A}}{Q}{g}
    }
  { \jtermeq
      {{\Gamma}{\subst{x}{P}}}
      {\subst{x}{\ctxwk{P}{Q}}}
      {\subst{x}{\ctxwk{P}{g}}}
      {\ctxwk{\subst{x}{P}}{\subst{x}{g}}}
    }
\end{equation*}
\end{rmk}

\subsubsection{Weakenings of fibers}
\label{comp-ws}
The following inference rules explain what happens when we first weaken by a
term $g$ of $Q$ in context $\ctxext{\Gamma}{B}$ and then weaken by a family
$A$ over $\Gamma$.
\begin{align}
& \inference
  { \jfam{\Gamma}{A}
    \jterm{{\Gamma}{B}}{Q}{g}
    \jany{{{\Gamma}{B}}{Q}}{e}
    }
  { \janyeq
      {{{\Gamma}{A}}{\ctxwk{A}{B}}}
      {\ctxwk{A}{\subst{g}{e}}}
      {\subst{\ctxwk{A}{g}}{\ctxwk{A}{e}}}
    }
  \label{comp-ws-any}
\end{align}

\begin{rmk}
As an important special case of these inference rules we have the following 
valid inference rules:
\begin{align*}
& \inference
  { \jfam{\Gamma}{A}
    \jterm{\Gamma}{B}{y}
    \jany{{\Gamma}{B}}{e}
    }
  { \janyeq
      {{\Gamma}{A}}
      {\ctxwk{A}{\subst{y}{e}}}
      {\subst{\ctxwk{A}{y}}{\ctxwk{A}{e}}}
    }
\end{align*}
Moreover, we get
\begin{equation*}
\inference
  { \jfam{\Gamma}{A}
    \jterm{\Gamma}{B}{y}
    \jterm{{\Gamma}{B}}{Q}{g}
    }
  { \jtermeq
      {{\Gamma}{A}}
      {\ctxwk{A}{\subst{y}{Q}}}
      {\ctxwk{A}{\subst{y}{g}}}
      {\subst{\ctxwk{A}{y}}{\ctxwk{A}{g}}}
    }
\end{equation*}
\end{rmk}

\subsubsection{The cancellation property of weakening and substitution}
\label{cancellation-ws}
The judgmental equalities we're about to describe assert that substituting a term
in the weakening a thing gives you the thing back. In the case of contexts we get that each fiber
$\subst{x}{\ctxwk{A}{B}}$ is just $B$ and in the case of terms we get 
that $\ctxwk{A}{y}$ is the constant function
mapping everything to $y:B$. Thus, these rules actually establish the weakening
as the weakening. After stating the rules we will describe what it means to
compose context morphisms (terms of weakened contexts).
\begin{align}
& \inference
  { \jterm{\Gamma}{A}{x}
    \jany{\Gamma}{e}
    }
  { \janyeq
      {\Gamma}
      {\subst{x}{\ctxwk{A}{e}}}
      {e}
    }
  \label{cancellation-ws-e}
\end{align}

\subsubsection{The identity term of a substituted family}
\label{comp-si}
The identity term of a substituted family is the substitution of the identity term
\begin{equation}
\inference
  { \jterm{\Gamma}{A}{x}
    \jfam{{{\Gamma}{A}}{P}}{Q}
    }
  { \unfoldall{\jhomeq
      {{\Gamma}{\subst{x}{P}}}
      {\subst{x}{Q}}
      {\subst{x}{Q}}
      {\subst{x}{\idtm{Q}}}
      {\idtm{\subst{x}{Q}}}
    }}
  \label{comp-si-t}
\end{equation}

\begin{rmk}
An important special case is the judgmental equality
\begin{equation*}
\unfoldall{\jhomeq
      {\Gamma}
      {\subst{x}{P}}
      {\subst{x}{P}}
      {\subst{x}{\idtm{P}}}
      {\idtm{\subst{x}{P}}}
    }
\end{equation*}
for a family $\jfam{{\Gamma}{A}}{P}$.
\end{rmk}

\subsubsection{The cancellation property of identity terms}
\label{cancellation-i}
Identity terms are determined by their behavior with respect to substitution combined with
weakening. The identity terms will also be subject to compatibility rules.
\begin{align}
& \inference
  { \jterm{\Gamma}{A}{x}
    }
  { \jtermeq{\Gamma}{A}{\subst{x}{\idtm{A}}}{x}
    }
  \label{cancellation-si}\\
& \inference
  { \jany{{\Gamma}{A}}{e}
    }
  { \janyeq{{\Gamma}{A}}{\subst{\idtm{A}}{\ctxwk{A}{e}}}{e}
    }
  \label{precomp-idtm-any}
\end{align}


\subsection{E-systems}
\begin{defn}
An \emph{E-system} is an extension algebra with the structure of a projection algebra,
the structure of a substitution algebra and which has an empty context and families,
such that additionally:
\begin{enumerate}
\item substitution is a projection homomorphism
\item weakening is a substitution homomorphism
\item both weakening and substitution are empty-CF homomorphisms.
\item 
\end{enumerate}
\end{defn}



\section{The relation between E-systems and B-systems}

In this section we answer the following questions posed by Voevodsky in private
correspondence.
\begin{itemize}
\item Is the construction that takes a (unital) B-system to an E-system a functor?
\item Is this functor fully faithful?
\item Is there a simple way to characterize the E-systems that arise from B-systems (the image of this functor on objects)?
\item Construct an example of an E-system that does not arise from a B-system.
\end{itemize}

\subsection{The functor from B-systems to E-systems}
Consider a B-system $\mathbb{B}$. We will construct an E-system $E(\mathbb{B})$
in the category of sets. This E-systems consists of sets $C(\mathbb{B})$,
$F(\mathbb{B})$ and $T(\mathbb{B})$ for the sorts of contexts, families and
terms respectively, with functions $\eft:F(\mathbb{B})\to C(\mathbb{B})$ and
$\ebd:T(\mathbb{B})\to T(\mathbb{B})$ and the operations of extension,
weakening, units, substitution and empty contexts and families.

The set $C(\mathbb{B})$ consists of pairs $(n,X)$, where $n\in\mathbb{N}$ and
$X\in B_n$. The set $F(\mathbb{B})$ consists of quadruples $(n,X,m,Y)$, where
$n,m\in\mathbb{N}$ such that $n<m$, $X\in B_n$ and $Y\in B_{m}$ such that
$ft^{m-n}:=ft^{n-1}\circ\cdots\circ ft^m(Y)=X$.

%\section{Derived notions of the theory of contexts, families and terms}
\label{digging_deeper}

In this section we use the framework we have developed in \autoref{tt}
to derive new notions and their properties. In particular, we will
develop the notion of \emph{extension on terms} together with the projection
maps from an extension to the `base' context of family, the 
\emph{family pullback} which is a version of pullbacks for families and thirdly
the \emph{inductive morphisms} which are morphisms of type theory that allow
to find terms of families over the codomain context by pulling them back to
the domain context and finding a term there.
\emph{This section contains no new assumptions.}

In \autoref{extension-on-terms} on the extension operation on terms we will
derive all the compatibility rules that one would expect to hold for extension
on terms. The key to most of these results is the currying operation, which
could be seen as the missing feature in the table above. The extension on terms
operation depends in an essential way on the substitution operation, on the
identity terms and therefore indirectly also on the weakening operation. Thus,
we will see here all of the features of the theory we develop in
\autoref{tt} come to the 
surface.

Next, we introduce the inclusion of the fibers $\subst{x}{P}$ into the extension
$\ctxext{A}{P}$ as a morphism in context $\Gamma$. As was the case with
extension on terms and with projections, there will be a ton of compatibility
properties which we will prove about these inclusions. 

It should be kept in mind though that in the current formulation there is no
sealed deal establishing a relationship between families over a context
and any kind of morphisms -- neither with morphisms having the `base' of the
family as its codomain nor with families into a universe (universes will be
introduced in \autoref{universes}). The only thing we know here is
that a family $P$ over $\ctxext{\Gamma}{A}$ determines a context morphism
from $\ctxext{A}{P}$ to $A$ in context $\Gamma$, the projection. 
We do not see this as
a shortcoming of the theory of contexts families and terms. Rather, such a
correspondence is a feature of a theory which does incorporate universes. The
fact that we're lacking a clear connection between families and (a specified
class of) morphisms, however, does show up in our treatment of the notion we
called familie pullbacks. For instance, we can't show that a square of families
is a pullback precisely when the corresponding square of projections is a
pullback: only the backwards direction holds. 
The discrepancies continue: ordinary pullbacks do not always exist
whereas family pullbacks do but the composition of two family pullback squares
need not be a family pullback square again whereas the pasting lemma of
ordinary pullbacks holds as usually.
We feel that pointing out what we can and can't do in the current setting is
an important aspect of developing an intuition with the system and therefore
we include this subsection even though the theory of family pullbacks
might feel a bit different than the usual theory of pullbacks.

In the last subsection we give a treatment of inductive morphisms. These
morphisms appear also in the introduction of the many inductive type
constructors in \autoref{tt_constructors} and therefore a general treatment of
the subject is insightful.

\subsection{Morphisms}\label{morphisms}
Using the rules of the compatibility of substitution with weakening and of the
compatibility of weakening with itself, we see that we can show

\begin{lem}\label{lem:prehom}
The inference rule\begin{equation*}
\inference
  { \jfam{\Gamma}{A}
    \jfam{\Gamma}{B}
    \jfam{\Gamma}{C}
    \jhom{\Gamma}{A}{B}{f}
    }
  { \jfameq
    {{\Gamma}{A}}
    {\subst{f}{\ctxwk{A}{{B}{C}}}}
    {\ctxwk{A}{C}}
    }
\end{equation*}
is valid.
\end{lem}

\begin{proof}
Let $\jfam{\Gamma}{A}$, $\jfam{\Gamma}{B}$, $\jfam{\Gamma}{C}$ and $\jhom{\Gamma}{A}{B}{f}$.
Then we have the judgmental equalities\begin{align*}
\subst{f}{\ctxwk{A}{{B}{C}}}
& \jdeq 
  \subst{f}{\ctxwk{{A}{B}}{{A}{C}}}
  \\
& \jdeq 
  \ctxwk{A}{C}.
  \qedhere
\end{align*}
\end{proof}

It follows that for $\jterm{{\Gamma}{B}}{\ctxwk{B}{C}}{g}$ we can compose $f$
with $g$ to obtain a term of $\ctxwk{A}{C}$ in context $\ctxext{\Gamma}{A}$.
In the following definition, we work with in a slightly greater generality.

\begin{defn}
We define the judgment\begin{equation*}
\jhom{\Gamma}{A}{B}{f},
\end{equation*}
which is pronounced as `$f$ is a morphism from $A$ to $B$ in context $\Gamma$',
to be the judgment\begin{equation*}
\unfold{\jhom{\Gamma}{A}{B}{f}}.
\end{equation*}
Likewise, we define the judgment\begin{equation*}
\jhomeq{\Gamma}{A}{B}{f}{f'}
\end{equation*}
to be the judgment\begin{equation*}
\unfold{\jhomeq{\Gamma}{A}{B}{f}{f'}}.
\end{equation*}
\end{defn}

\begin{defn}
Let $\jhom{\Gamma}{A}{B}{f}$ and consider a family $\jfam{{\Gamma}{B}}{Q}$,
a family $\jfam{{{\Gamma}{B}}{Q}}{R}$ and a term $\jterm{{{\Gamma}{B}}{Q}}{R}{h}$.
We define\begin{align*}
\jfamdefn*
  {{\Gamma}{A}}
  {\jcomp{A}{f}{Q}}
  {\unfold{\jcomp{A}{f}{Q}}}
  \\
\jfamdefn*
  {{{\Gamma}{A}}{\jcomp{A}{f}{Q}}}
  {\jcomp{A}{f}{R}}
  {\unfold{\jcomp{A}{f}{R}}}
  \\
\jtermdefn*
  {{{\Gamma}{A}}{\jcomp{A}{f}{Q}}}
  {\jcomp{A}{f}{R}}
  {\jcomp{A}{f}{h}}
  {\unfold{\jcomp{A}{f}{h}}}.
\end{align*}
\end{defn}

\begin{lem}
Let $\jhom{\Gamma}{A}{B}{f}$. Then the inference rules
\begin{align*}
& \inference
  { %
    }
  { \jfameq{{\Gamma}{A}}{\jcomp{A}{f}{\emptyf}}{\emptyf}
    }
  \\
& \inference
  { \jfam{{\Gamma}{B}}{Q}
    }
  { \jfameq{{{\Gamma}{B}}{\jcomp{A}{f}{Q}}}{\jcomp{A}{f}{\emptyf}}{\emptyf}
    }
\end{align*}
are valid.
\end{lem}

\begin{proof}
For the first, note that we have the judgmental equalities
\begin{equation*}
\jcomp{A}{f}{\emptyf}
  \jdeq
  \unfold{\jcomp{A}{f}{\emptyf}}
  \jdeq
  \subst{f}{\emptyf}
  \jdeq
  \emptyf.
\end{equation*}
The second judgmental equality is proven similarly.
\end{proof}

\begin{rmk}
Recall that we can treat a family $\jfam{{\Gamma}{B}}{Q}$ as a family
$\jfam{{{\Gamma}{B}}{\emptyf}}{Q}$ and that $\jcomp{A}{f}{\emptyf}\jdeq
\emptyf$. Thus we can apply composition with $f$ to terms 
$\jterm{{\Gamma}{B}}{Q}{g}$. We get
\begin{equation*}
\jtermeq
  {{\Gamma}{B}}
  {\jcomp{A}{f}{Q}}
  {\jcomp{A}{f}{g}}
  {\unfold{\jcomp{A}{f}{g}}}.
\end{equation*}
In the particular situation where we take $Q$ to be a weakened family
$\ctxwk{B}{C}$, we see that we can apply composition with $f$ to morphisms
from $B$ to $C$ and we can use \autoref{lem:prehom} to see that we get
\begin{equation*}
\jhomeq{\Gamma}{A}{C}{\jcomp{A}{f}{g}}{\unfold{\jcomp{A}{f}{g}}}
\end{equation*}
for $\jhom{\Gamma}{B}{C}{g}$. 

One might argue that the notation for composition should be reserved to only
this special case, to not confuse with common intuition of composition. It is
however very convenient to see composition as an operation of the theory of
contexts, families and terms. This allows us to follow the scheme of
compatibility rules which are provable for this form of composition. 
\end{rmk}

We have lots of compatibility properties for composition. On the one hand we
have the properties that composition with a morphism $f$ is compatible with
the empty family, extension, weakening, substitution and identity terms. On
the other hand, there are compatibility properties saying what happens when
we substitute by a composition, or when we weaken or substitute a composition.
Proving all these compatibility properties is the content of the rest of this
subsection. None of it is very difficult.

\begin{lem}
We have the following inference rule expressing that composition with $f$ is
compatible with extension:
\begin{equation*}
\inference
  { \jfam{{{{\Gamma}{B}}{Q}}{R}}{S}
    }
  { \jfameq
      {{{\Gamma}{A}}{\jcomp{A}{f}{Q}}}
      {\jcomp{A}{f}{\ctxext{R}{S}}}
      {\ctxext{\jcomp{A}{f}{R}}{\jcomp{A}{f}{S}}}
    }
\end{equation*}
\end{lem}

\begin{proof}
Let $\jfam{{{{\Gamma}{B}}{Q}}{R}}{S}$. Then we have the judgmental equalities
\begin{align*}
\jcomp{A}{f}{\ctxext{R}{S}}
& \jdeq
  \unfold{\jcomp{A}{f}{\ctxext{R}{S}}}
  \\
& \jdeq
  \subst{f}{\ctxext{\ctxwk{A}{R}}{\ctxwk{A}{S}}}
  \\
& \jdeq
  \unfoldall{\ctxext{\jcomp{A}{f}{R}}{\jcomp{A}{f}{S}}}
  \\
& \jdeq
  \ctxext{\jcomp{A}{f}{R}}{\jcomp{A}{f}{S}}.\qedhere
\end{align*}
\end{proof}

\begin{lem}
We have the following inference rules expressing that composition with $f$ is
compatible with weakening:
\begin{align*}
& \inference
  { \jfam{{{\Gamma}{B}}{Q}}{R}
    \jfam{{{\Gamma}{B}}{Q}}{S}
    }
  { \jfameq
      {{{{\Gamma}{A}}{\jcomp{A}{f}{Q}}}{\jcomp{A}{f}{R}}}
      {\jcomp{A}{f}{\ctxwk{R}{S}}}
      {\ctxwk{\jcomp{A}{f}{R}}{\jcomp{A}{f}{S}}}
    }
  \\
& \inference
  { \jfam{{{\Gamma}{B}}{Q}}{R}
    \jterm{{{\Gamma}{B}}{Q}}{S}{k}
    }
  { \jtermeq
      {{{{\Gamma}{A}}{\jcomp{A}{f}{Q}}}{\jcomp{A}{f}{R}}}
      {\jcomp{A}{f}{\ctxwk{R}{S}}}
      {\jcomp{A}{f}{\ctxwk{R}{k}}}
      {\ctxwk{\jcomp{A}{f}{R}}{\jcomp{A}{f}{k}}}
    }
\end{align*}
\end{lem}

\begin{proof}
Consider the families $\jfam{{{\Gamma}{B}}{Q}}{R}$ and 
$\jfam{{{\Gamma}{B}}{Q}}{S}$. Then we have the judgmental equalities
\begin{align*}
\jcomp{A}{f}{\ctxwk{R}{S}}
& \jdeq
  \unfold{\jcomp{A}{f}{\ctxwk{R}{S}}}
  \\
& \jdeq
  \subst{f}{\ctxwk{{A}{R}}{{A}{S}}}
  \\
& \jdeq
  \unfoldall{\ctxwk{\jcomp{A}{f}{R}}{\jcomp{A}{f}{S}}}
  \\
& \jdeq
  \ctxwk{\jcomp{A}{f}{R}}{\jcomp{A}{f}{S}}.
\end{align*}
The proof of the second property is similar.
\end{proof}

\begin{lem}
We have the following inference rules expressing that composition with $f$ is
compatible with substitution:
\begin{align*}
& \inference
  { \jterm{{{\Gamma}{B}}{Q}}{R}{h}
    \jfam{{{{\Gamma}{B}}{Q}}{R}}{S}
    }
  { \jfameq
      {{{\Gamma}{A}}{\jcomp{A}{f}{Q}}}
      {\jcomp{A}{f}{\subst{h}{S}}}
      {\subst{\jcomp{A}{f}{h}}{\jcomp{A}{f}{S}}}
    }
  \\
& \inference
  { \jterm{{{\Gamma}{B}}{Q}}{R}{h}
    \jterm{{{{\Gamma}{B}}{Q}}{R}}{S}{k}
    }
  { \jtermeq
      {{{\Gamma}{A}}{\jcomp{A}{f}{Q}}}
      {\jcomp{A}{f}{\subst{h}{S}}}
      {\jcomp{A}{f}{\subst{h}{k}}}
      {\subst{\jcomp{A}{f}{h}}{\jcomp{A}{f}{k}}}
    }
\end{align*}
\end{lem}

\begin{proof}
Let $\jterm{{{\Gamma}{B}}{Q}}{R}{h}$ and $\jfam{{{{\Gamma}{B}}{Q}}{R}}{S}$.
Then we have the judgmental equalities
\begin{align*}
\jcomp{A}{f}{\subst{h}{S}}
& \jdeq
  \unfold{\jcomp{A}{f}{\subst{h}{S}}}
  \\
& \jdeq
  \subst{f}{{\ctxwk{A}{h}}{\ctxwk{A}{S}}}
  \\
& \jdeq
  \unfoldall{\subst{\jcomp{A}{f}{h}}{\jcomp{A}{f}{S}}}
  \\
& \jdeq
  \subst{\jcomp{A}{f}{h}}{\jcomp{A}{f}{S}}.
\end{align*}
The proof of the second inference rule is similar.
\end{proof}

\begin{lem}
We have the following inference rule expressing that composition with $f$ is
compatible with identity terms:
\begin{equation*}
\inference
  { \jfam{{{\Gamma}{B}}{Q}}{R}
    }
  { \jtermeq
      {{{{\Gamma}{A}}{\jcomp{A}{f}{Q}}}{\jcomp{A}{f}{R}}}
      {\ctxwk{\jcomp{A}{f}{R}}{\jcomp{A}{f}{R}}}
      {\jcomp{A}{f}{\idtm{R}}}
      {\idtm{\jcomp{A}{f}{R}}}
    }
\end{equation*}
\end{lem}

\begin{proof}
Let $\jfam{{{\Gamma}{B}}{Q}}{R}$. We have the following judgmental equalities:
\begin{align*}
\jcomp{A}{f}{\idtm{R}}
& \jdeq
  \unfold{\jcomp{A}{f}{\idtm{R}}}
  \\
& \jdeq
  \subst{f}{\idtm{\ctxwk{A}{R}}}
  \\
& \jdeq
  \unfoldall{\idtm{\jcomp{A}{f}{R}}}
  \\
& \jdeq
  \idtm{\jcomp{A}{f}{R}}.\qedhere
\end{align*}
\end{proof}

\begin{lem}
We have the following inference rules about the situation where something is
substituted by a composition:\begin{align*}
& \inference
  { \jhom{\Gamma}{A}{B}{f}
    \jhom{\Gamma}{B}{C}{g}
    \jfam{{{\Gamma}{A}}{\ctxwk{A}{C}}}{R}
    }
  { \jfameq
      {{\Gamma}{A}}
      {\subst{\jcomp{A}{f}{g}}{R}}
      {\subst{f}{{\ctxwk{A}{g}}{\ctxwk{{A}{B}}{R}}}}
    }
  \\
& \inference
  { \jhom{\Gamma}{A}{B}{f}
    \jhom{\Gamma}{B}{C}{g}
    \jterm{{{\Gamma}{A}}{\ctxwk{A}{C}}}{R}{h}
    }
  { \jfameq
    {{\Gamma}{A}}
    {\subst{\jcomp{A}{f}{g}}{h}}
    {\subst{f}{{\ctxwk{A}{g}}{\ctxwk{{A}{B}}{h}}}}
    }
\end{align*}
We also have the following related inference rules, asserting that composition
is strictly associative:\begin{align*}
& \inference
  { \jhom{\Gamma}{A}{B}{f}
    \jhom{\Gamma}{B}{C}{g}
    \jfam{{\Gamma}{C}}{R}
    }
  { \jfameq
      {{\Gamma}{A}}
      {\jcomp{A}{{A}{f}{g}}{R}}
      {\jcomp{A}{f}{{B}{g}{R}}}
    }
  \\
& \inference
    { \jhom{\Gamma}{A}{B}{f}
      \jhom{\Gamma}{B}{C}{g}
      \jterm{{\Gamma}{C}}{R}{h}
      }
    { \jtermeq
        {{\Gamma}{A}}
        {\jcomp{A}{{A}{f}{g}}{R}}
        {\jcomp{A}{{A}{f}{g}}{h}}
        {\jcomp{A}{f}{{B}{g}{h}}}
      }
\end{align*}
\end{lem}

\begin{proof}
Consider family morphisms $\jhom{\Gamma}{A}{B}{f}$ and $\jhom{\Gamma}{B}{C}{g}$
and a family $\jfam{{{\Gamma}{A}}{\ctxwk{A}{C}}}{R}$. Then we have the judgmental
equalities\begin{align*}
\subst{\jcomp{A}{f}{g}}{R} 
& \jdeq 
  \subst{{f}{\ctxwk{A}{g}}}{R}
  \\
& \jdeq 
  \subst{{f}{\ctxwk{A}{g}}}{\subst{f}{\ctxwk{{A}{B}}{R}}}
  \\
& \jdeq 
  \subst{f}{{\ctxwk{A}{g}}{\ctxwk{{A}{B}}{R}}}
\end{align*}
The proof that 
$\subst{\jcomp{A}{f}{g}}{h}\jdeq\subst{f}{{\ctxwk{A}{g}}{\ctxwk{{A}{B}}{h}}}$
is similar.

Now suppose that $\jfam{{\Gamma}{C}}{R}$ instead. Then we have\begin{align*}
\jcomp{A}{{A}{f}{g}}{R} 
& \jdeq 
  \subst{\jcomp{A}{f}{g}}{\ctxwk{A}{R}}
  \\
& \jdeq 
  \subst{{f}{\ctxwk{A}{g}}}{\ctxwk{A}{R}}
  \\
& \jdeq 
  \subst{f}{{\ctxwk{A}{g}}{\ctxwk{{A}{B}}{{A}{R}}}}
  \\
& \jdeq 
  \subst{f}{{\ctxwk{A}{g}}{\ctxwk{A}{{B}{R}}}}
  \\
& \jdeq 
  \subst{f}{\ctxwk{A}{\subst{g}{\ctxwk{B}{R}}}}
  \\
& \jdeq 
  \subst{f}{\ctxwk{A}{\jcomp{B}{g}{R}}}
  \\
& \jdeq 
  \jcomp{A}{f}{{B}{g}{R}}.
\end{align*}
Again, the proof is similar for terms $h$ of $R$ in context $\ctxext{\Gamma}{C}$.
\end{proof}

\begin{lem}
We have the following inference rules about the compatibility of composition with
weakening:\begin{align*}
& \inference
  { \jhom{\Gamma}{A}{B}{f}
    \jhom{\Gamma}{B}{C}{g}
    \jfam{{\Gamma}{A}}{P}
    }
  { \jhomeq
      {\Gamma}
      {{A}{P}}
      {C}
      {\ctxwk{P}{\jcomp{A}{f}{g}}}
      {\jcomp{{A}{P}}{\ctxwk{P}{f}}{g}}
    }
  \\
& \inference
  { \jterm{\Gamma}{B}{y}
    \jhom{\Gamma}{B}{C}{g}
    }
  { \jhomeq
      {\Gamma}
      {A}
      {C}
      {\jcomp{A}{\ctxwk{A}{y}}{g}}
      {\ctxwk{A}{\subst{y}{g}}}
    }
  \\
& \inference
  { \jhom{\Gamma}{A}{B}{f}
    \jterm{\Gamma}{C}{z}
    }
  { \jhomeq
      {\Gamma}
      {A}
      {C}
      {\jcomp{A}{f}{\ctxwk{B}{z}}}
      {\ctxwk{A}{z}}
    }
\end{align*}
\end{lem}

\begin{proof}
Let $\jhom{\Gamma}{A}{B}{f}$, $\jhom{\Gamma}{B}{C}{g}$ and $\jfam{{\Gamma}{A}}{P}$.
Then we have the judgmental equalities\begin{align*}
\ctxwk{P}{\jcomp{A}{f}{g}} 
& \jdeq 
  \ctxwk{P}{\subst{f}{\ctxwk{A}{g}}}
  \\
& \jdeq 
  \subst{\ctxwk{P}{f}}{\ctxwk{P}{{A}{g}}}
  \\
& \jdeq 
  \subst{\ctxwk{P}{f}}{\ctxwk{\ctxext{A}{P}}{g}}
  \\
& \jdeq 
  \jcomp{{A}{P}}{\ctxwk{P}{f}}{g}.
\end{align*}
Now let $\jterm{\Gamma}{B}{y}$ and $\jhom{\Gamma}{B}{C}{g}$. Then we have the
judgmental equalities\begin{align*}
\jcomp{A}{\ctxwk{A}{y}}{g}
& \jdeq 
  \subst{\ctxwk{A}{y}}{\ctxwk{A}{g}}
  \\
& \jdeq 
  \ctxwk{A}{\subst{y}{g}}.
\end{align*}
For the third assertion, let $\jhom{\Gamma}{A}{B}{f}$ and $\jterm{\Gamma}{C}{z}$.
Then we have the judgmental equalities\begin{align*}
\jcomp{A}{f}{\ctxwk{B}{z}} 
& \jdeq 
  \subst{f}{\ctxwk{A}{{B}{z}}}
  \\
& \jdeq 
  \subst{f}{\ctxwk{{A}{B}}{{A}{z}}}
  \\
& \jdeq 
  \ctxwk{A}{z}.
  \qedhere
\end{align*}
\end{proof}

\begin{lem}
We have the following inference rules about the compatibility of composition with
substitution:\begin{align*}
& \inference
  { \jhom{{\Gamma}{A}}{P}{Q}{f}
    \jhom{{\Gamma}{A}}{Q}{R}{g}
    \jterm{\Gamma}{A}{x}
    }
  { \jhomeq
      {\Gamma}
      {\subst{x}{P}}
      {\subst{x}{R}}
      {\subst{x}{\jcomp{P}{f}{g}}}
      {\jcomp{\subst{x}{P}}{\subst{x}{f}}{\subst{x}{g}}}
    }
  \\
& \inference
  { \jhom{\Gamma}{A}{B}{f}
    \jhom{\Gamma}{B}{C}{g}
    \jterm{\Gamma}{A}{x}
    }
  { \jtermeq
      {\Gamma}
      {C}
      {\subst{x}{\jcomp{A}{f}{g}}}
      {\subst{{x}{f}}{g}}
    }
\end{align*}
\end{lem}

\begin{proof}
Let $\jhom{{\Gamma}{A}}{P}{Q}{f}$, $\jhom{{\Gamma}{A}}{Q}{R}{g}$ and 
$\jterm{\Gamma}{A}{x}$.
Then we have the judgmental equalities\begin{align*}
\subst{x}{\jcomp{A}{f}{g}}
& \jdeq 
  \subst{x}{{f}{\ctxwk{P}{g}}}
  \\
& \jdeq 
  \subst{{x}{f}}{{x}{\ctxwk{P}{g}}}
  \\
& \jdeq 
  \subst{{x}{f}}{\ctxwk{\subst{x}{P}}{\subst{x}{g}}}
  \\
& \jdeq 
  \jcomp{\subst{x}{P}}{\subst{x}{f}}{\subst{x}{g}}.
\end{align*}
Now let $\jhom{\Gamma}{A}{B}{f}$, $\jhom{\Gamma}{B}{C}{g}$ and $\jterm{\Gamma}{A}{x}$.
Then we have the judgmental equalities\begin{align*}
\subst{x}{\jcomp{A}{f}{g}}
& \jdeq 
  \subst{x}{{f}{\ctxwk{A}{g}}}
  \\
& \jdeq 
  \subst{{x}{f}}{{x}{\ctxwk{A}{g}}}
  \\
& \jdeq 
  \subst{{x}{f}}{g}.
  \qedhere
\end{align*}
\end{proof}

\subsection{Projections and extension on terms}\label{extension-on-terms}
In this subsection we consider the notion of extension on terms, which has now
become definable inside our theory. Moreover, every compatibility rule one may
dream of is provable as well, using the compatibility rules we have introduced
earlier.

\begin{defn}
When $\jterm{\Gamma}{A}{x}$ and $\jterm{\Gamma}{\subst{x}{P}}{u}$ are terms,
we define 
\begin{equation*}
\jtermdefn
  {\Gamma}
  {\ctxext{A}{P}}
  {\tmext{A}{P}{x}{u}}
  {\unfold{\tmext{A}{P}{x}{u}}}.
\end{equation*} 
\end{defn}

Thus, the term $\tmext{A}{P}{x}{u}$ is the pairing of $x$ and $u$. Note that because
we have the judgmental equality 
$\ctxwk{P}{{A}{\ctxext{A}{P}}}\jdeq\ctxwk{\ctxext{A}{P}}{\ctxext{A}{P}}$ in the
context $\ctxext{{\Gamma}{A}}{P}$, the
pairing function could just be defined as $\idtm{\ctxext{A}{P}}$. 

When we substitute by an extended term we get an equal result as when we
substitute two consecutive times, like the way currying works.

\begin{lem}\label{comp-es}
The following inference rules are valid:
\begin{align*}
& \inference
  { \jterm{\Gamma}{A}{x}
    \jterm{\Gamma}{\subst{x}{P}}{u}
    \jfam{{{\Gamma}{A}}{P}}{Q}
    }
  { \jfameq
      {\Gamma}
      {\subst{\tmext{A}{P}{x}{u}}{Q}}
      {\subst{u}{{x}{Q}}}
    }
  \\
& \inference
  { \jterm{\Gamma}{A}{x}
    \jterm{\Gamma}{\subst{x}{P}}{u}
    \jterm{{{\Gamma}{A}}{P}}{Q}{g}
    }
  { \jtermeq
      {\Gamma}
      {\subst{u}{{x}{Q}}}
      {\subst{\tmext{A}{P}{x}{u}}{g}}
      {\subst{u}{{x}{g}}}
    }
  \\
& \inference
  { \jterm{\Gamma}{A}{x}
    \jterm{\Gamma}{\subst{x}{P}}{u}
    \jfam{{{{\Gamma}{A}}{P}}{Q}}{R}
    }
  { \jfameq
      {{\Gamma}{\subst{u}{{x}{Q}}}}
      {\subst{\tmext{A}{P}{x}{u}}{R}}
      {\subst{u}{{x}{R}}}
    }
  \\
& \inference
  { \jterm{\Gamma}{A}{x}
    \jterm{\Gamma}{\subst{x}{P}}{u}
    \jterm{{{{\Gamma}{A}}{P}}{Q}}{R}{t}
    }
  { \jtermeq
      {{\Gamma}{\subst{u}{{x}{Q}}}}
      {\subst{u}{{x}{R}}}
      {\subst{\tmext{A}{P}{x}{u}}{t}}
      {\subst{u}{{x}{t}}}
    }
\end{align*}
\end{lem}

\begin{proof}
We prove only the first judgmental equality. All the others are similar.
Let $\jterm{\Gamma}{A}{x}$ and $\jterm{\Gamma}{\subst{x}{P}}{u}$
be terms and let $\jfam{{{\Gamma}{A}}{P}}{Q}$ be a family. Then we have
\begin{align*}
\subst
  {\tmext{A}{P}{x}{u}}
  {Q} 
& \jdeq 
  \subst
    {{u}{{x}{\idtm{\ctxext{A}{P}}}}}
    {Q}
  \tag{by definition}\\
& \jdeq 
  \subst
    {{u}{{x}{\idtm{\ctxext{A}{P}}}}}
    {{x}{\ctxwk{A}{Q}}}
  \tag{by \autoref{defn-ws-3}}\\
& \jdeq 
  \subst
    {{u}{{x}{\idtm{\ctxext{A}{P}}}}}
    {{u}{\ctxwk{\subst{x}{P}}{\subst{x}{\ctxwk{A}{Q}}}}}
  \tag{by \autoref{defn-ws-3}}\\
& \jdeq 
  \subst
    {{u}{{x}{\idtm{\ctxext{A}{P}}}}}
    {{u}{{x}{\ctxwk{P}{{A}{Q}}}}}
  \tag{by \autoref{comp-sw-f}}\\
& \jdeq 
  \subst
    {u}
    {{{x}{\idtm{\ctxext{A}{P}}}}{{x}{\ctxwk{{P}{{A}{Q}}}}}}
  \tag{by \autoref{comp-ss-f}}\\
& \jdeq 
  \subst
    {u}
    {{x}{{\idtm{\ctxext{A}{P}}}{\ctxwk{P}{{A}{Q}}}}}
  \tag{by \autoref{comp-ss-f}}\\
& \jdeq 
  \subst
    {u}
    {{x}{{\idtm{\ctxext{A}{P}}}{\ctxwk{\ctxext{A}{P}}{Q}}}}
  \tag{by \autoref{comp-ew-f}}\\
& \jdeq 
  \subst
    {u}
    {{x}{Q}}
  \tag{by \autoref{idfunc-wk-defn}}
\end{align*}
\end{proof}

We have seen above that the pairing function into $\ctxext{A}{P}$ is just the identity term on
$\ctxext{A}{P}$. To analyze the pairing functin a little further, we will also
need the projection maps from $\ctxext{A}{P}$ to $A$ and from $\ctxext{A}{P}$
to $P$. We will now define these and see that the identity term of an
extended family is the extension (or pairing) of the identity
functions on the components in the apropriate way.

To find out what the
apropriate way is, note that
\begin{align*}
\ctxwk{\ctxext{A}{P}}{\ctxext{A}{P}} 
& \jdeq 
  \ctxwk{P}{{A}{\ctxext{A}{P}}}
  \\
& \jdeq 
  \ctxext{\ctxwk{P}{{A}{A}}}{\ctxwk{P}{{A}{P}}}
\end{align*}
We have the term $\jterm{{\Gamma}{{A}{P}}}{\ctxwk{P}{A}}{\ctxwk{P}{\idtm{A}}}$.
Thus we need to find out what $\subst{\ctxwk{P}{\idtm{A}}}{\ctxwk{P}{{A}{P}}}$ is:
\begin{align*}
\subst{\ctxwk{P}{\idtm{A}}}{\ctxwk{P}{{A}{P}}} 
& \jdeq 
  \ctxwk{P}{\subst{\idtm{A}}{\ctxwk{A}{P}}}
  \\
& \jdeq 
  \ctxwk{P}{P},
\end{align*}
where we find the term $\idtm{P}$. Therefore we define:

\begin{defn}
Let $\jfam{\Gamma}{A}$ and $\jfam{{\Gamma}{A}}{P}$ be families. We define
\begin{align*}
\jhomdefn*
  {\Gamma}
  {{A}{P}}
  {A}
  {\cprojfstf{A}{P}}
  {\unfold{\cprojfstf{A}{P}}}
  \\
\jtermdefn*
  {\ctxext{\Gamma}{{A}{P}}}
  {\ctxwk{P}{P}}
  {\cprojsndf{A}{P}}
  {\unfold{\cprojsndf{A}{P}}}
\end{align*}
\end{defn}

The constructions of the terms $\tmext{A}{P}{x}{u}$ and $\cprojfst{A}{P}{w}$ and
$\cprojsnd{A}{P}{w}$ are subject to various rules, with all of them being
consequences of earlier introduced inference rules.

\begin{lem}\label{lem:tmext-basic}
The following inference rules expressing that pairing is a strict
inverse to the combination of decompositions, are valid:
\begin{align*}
& \inference
  { \jterm{\Gamma}{\ctxext{A}{P}}{w}
    }
  { \jtermeq
      {\Gamma}
      {\ctxext{A}{P}}
      {\tmext{A}{P}{\cprojfst{A}{P}{w}}{\cprojsnd{A}{P}{w}}}
      {w}
    }
  \\
& \inference
  { \jterm{\Gamma}{A}{x}
    \jterm{\Gamma}{\subst{x}{P}}{u}
    }
  { \jtermeq
      {\Gamma}
      {A}
      {\cprojfst{A}{P}{\tmext{A}{P}{x}{u}}}
      {x}
    }
  \\
& \inference
  { \jterm{\Gamma}{A}{x}
    \jterm{\Gamma}{\subst{x}{P}}{u}
    }
  { \jtermeq
      {\Gamma}
      {\subst{x}{P}}
      {\cprojsnd{A}{P}{\tmext{A}{P}{x}{u}}}
      {u}
    }
\end{align*}
\end{lem}

\begin{proof}
To prove the first judgmental equality, note that
\begin{align*}
w 
& \jdeq 
  \subst{w}{\idtm{\ctxext{A}{P}}} 
  \tag{by \autoref{idfunc-subst-defn}}\\
& \jdeq 
  \subst
    { w}
    { { \idtm{P}}
      { { \ctxwk{P}{\idtm{A}}}
        { \idtm{\ctxwk{\ctxext{A}{P}}{\ctxext{A}{P}}}}
        }
      }
  \tag{by \autoref{idfunc-ext-comp}}\\
& \jdeq 
  \subst
    { {w}
      {\idtm{P}}
      }
    { {w}
      { { \ctxwk{P}{\idtm{A}}
          }
        { \idtm{\ctxwk{\ctxext{A}{P}}{\ctxext{A}{P}}}
          }
        }
      }
  \tag{by \autoref{comp-ss-t}}\\
& \jdeq 
  \subst
    { {w}
      {\idtm{P}}
      }
    { { {w}
        {\ctxwk{P}{\idtm{A}}}
        }
      { {w}
        {\idtm{\ctxwk{\ctxext{A}{P}}{\ctxext{A}{P}}}}
        }
      }
  \tag{by \autoref{comp-ss-t}}\\
& \jdeq 
  \subst
    { {w}
      {\idtm{P}}
      }
    { { {w}
        {\ctxwk{P}{\idtm{A}}}
        }
      { {w}
        {\ctxwk{\ctxext{A}{P}}{\idtm{\ctxext{A}{P}}}}
        }
      }
  \tag{by \autoref{idfunc-wk-comp}}\\
& \jdeq 
  \subst
    { {w}
      {\idtm{P}}
      }
    { { {w}
        {\ctxwk{P}{\idtm{A}}}
        }
      { \idtm{\ctxext{A}{P}}
        }
      }
  \tag{by \autoref{defn-ws-4}}\\
& \jdeq 
  \tmext{A}{P}{\cprojfst{A}{P}{w}}{\cprojsnd{A}{P}{w}}
  \tag{by definition}
\end{align*}
To prove the second judgmental equality, let $\jterm{\Gamma}{A}{x}$ and
$\jterm{\Gamma}{\subst{x}{P}}{u}$. Then we have
\begin{align*}
\cprojfst{A}{P}{\tmext{A}{P}{x}{u}}
& \jdeq 
  \subst{\tmext{A}{P}{x}{u}}{\ctxwk{P}{\idtm{A}}}
  \\
& \jdeq 
  \subst{u}{{x}{\ctxwk{P}{\idtm{A}}}} 
  \\
& \jdeq 
  \subst{u}{\ctxwk{\subst{x}{P}}{\subst{x}{\idtm{A}}}}
  \\
& \jdeq 
  \subst{x}{\idtm{A}}
  \\
& \jdeq 
  x.
\end{align*}
To prove the third judgmental equality, note that
\begin{align*}
\cprojsnd{A}{P}{\tmext{A}{P}{x}{u}}
& \jdeq 
  \subst{\tmext{A}{P}{x}{u}}{\idtm{P}}
  \\
& \jdeq 
  \subst{u}{{x}{\idtm{P}}}
  \\
& \jdeq 
  \subst{u}{\idtm{\subst{x}{P}}}
  \\
& \jdeq 
  u.
  \qedhere
\end{align*}
\end{proof}

In \autoref{lem:tmext-emp,lem:tmext-ext,lem:tmext-wk,lem:tmext-subst,lem:tmext-id}
we show that term extension and the projections are
compatible with the empty families, extension,
weakening, substitution and the identity terms (in that order). \autoref{lem:tmext-id}
is in fact a generalization of the above \autoref{lem:tmext-basic}.

\begin{lem}\label{lem:tmext-emp}
The following compatibility rules for extensions of the term of the empty family
are valid:
\begin{align*}
& \inference
  { \jterm{\Gamma}{A}{x}
    }
  { \jtermeq{\Gamma}{A}{\tmext{\emptytm}{x}}{x}
    }
& & \inference
  { \jterm{\Gamma}{A}{x}
    }
  { \jtermeq{\Gamma}{A}{\tmext{x}{\emptytm}}{x}
    }
  \\
& \inference
  { \jfam{\Gamma}{A}
    }
  { \jtermeq
      {{\Gamma}{A}}
      {\emptyf}
      {\cprojfstf{\emptyf}{A}}
      {\emptytm}
    }
& & \inference
  { \jfam{\Gamma}{A}
    }
  { \jhomeq
      {\Gamma}
      {A}
      {A}
      {\cprojfstf{A}{\emptyf}}
      {\idtm{A}}
    }
  \\
& \inference
  { \jfam{\Gamma}{A}
    }
  { \jhomeq
      {\Gamma}
      {A}
      {A}
      {\cprojsndf{\emptyf}{A}}
      {\idtm{A}}
    }
& & \inference
  { \jfam{\Gamma}{A}
    }
  { \jtermeq
      {{\Gamma}{A}}
      {\emptyf}
      {\cprojsndf{A}{\emptyf}}
      {\emptytm}
    }
\end{align*}
\end{lem}

\begin{proof}
These equalities are very easy to verify. We only display a proof of the first:
\begin{equation*}
\tmext{\emptytm}{x}
\jdeq \unfold{\tmext{\emptyf}{A}{\emptytm}{x}}
\jdeq \subst{x}{\idtm{{\emptyf}{A}}}
\jdeq \subst{x}{\idtm{A}}
\jdeq x.\qedhere
\end{equation*}
\end{proof}

\begin{lem}\label{lem:tmext-ext}
The following compatibility rules for two consecutive term extensions are valid:
\begin{align*}
& \inference
  { \jterm{\Gamma}{A}{x}
    \jterm{\Gamma}{\subst{x}{P}}{u}
    \jterm{\Gamma}{\subst{\tmext{A}{P}{x}{u}}{Q}}{v}
    }
  { \jtermeq
      {\Gamma}
      {\ctxext{{A}{P}}{Q}}
      {\tmext{A}{{P}{Q}}{x}{{\subst{x}{P}}{\subst{x}{Q}}{u}{v}}}
      {\tmext{{A}{P}}{Q}{{A}{P}{x}{u}}{v}}
    }
  \\
& \inference
  { \jfam{{{\Gamma}{A}}{P}}{Q}
    }
  { \jhomeq
      {\Gamma}
      {\ctxext{{A}{P}}{Q}}
      {A}
      {\jcomp{}{\cprojfstf{{A}{P}}{Q}}{\cprojfstf{A}{P}}}
      {\cprojfstf{A}{{P}{Q}}}
    }
  \\
& \inference
  { \jfam{{{\Gamma}{A}}{P}}{Q}
    }
  { \jhomeq
      {{\Gamma}{A}}
      {{P}{Q}}
      {P}
      {\jcomp{}{\cprojfstf{{A}{P}}{Q}}{\cprojsndf{A}{P}}}
      {\jcomp{}{\cprojsndf{A}{{P}{Q}}}{\cprojfstf{P}{Q}}}
    }
  \\
& \inference
    { \jfam{{{\Gamma}{A}}{P}}{Q}
      }
    { \jhomeq
        {{{\Gamma}{A}}{P}}
        {Q}
        {Q}
        {\cprojsndf{{A}{P}}{Q}}
        {\jcomp{}{\cprojsndf{A}{{P}{Q}}}{\cprojsndf{P}{Q}}}
      }
\end{align*}
\end{lem}

\begin{proof}
Consider terms $\jterm{\Gamma}{A}{x}$, $\jterm{\Gamma}{\subst{x}{P}}{u}$ and
$\jterm{\Gamma}{\subst{u}{{x}{Q}}}{v}$. Then we have
\begin{align*}
\tmext{x}{{u}{v}}
& \jdeq 
  \subst
    {\tmext{u}{v}}{{x}{\idtm{\ctxext{A}{{P}{Q}}}}}
  \\
& \jdeq 
  \subst{v}{{u}{{x}{\idtm{\ctxext{A}{{P}{Q}}}}}}
  \\
& \jdeq 
  \subst{v}{{u}{{x}{\idtm{\ctxext{{A}{P}}{Q}}}}}
  \\
& \jdeq 
  \subst{v}{{\tmext{x}{u}}{\idtm{\ctxext{{A}{P}}{Q}}}}
  \\
& \jdeq 
  \tmext{{x}{u}}{v}.
\end{align*}
To prove the judmental equality
\begin{equation*}
\jhomeq
  {\Gamma}
  {\ctxext{{A}{P}}{Q}}
  {A}
  {\jcomp{}{\cprojfstf{{A}{P}}{Q}}{\cprojfstf{A}{P}}}
  {\cprojfstf{A}{{P}{Q}}}
\end{equation*}
note that we have the judgmental equalities
\begin{align*}
\jcomp{{{A}{P}}{Q}}{\cprojfstf{{A}{P}}{Q}}{\cprojfstf{A}{P}}
& \jdeq 
  \unfoldall{\jcomp{{{A}{P}}{Q}}{\cprojfstf{{A}{P}}{Q}}{\cprojfstf{A}{P}}}
  \\
& \jdeq 
  \subst
    {\ctxwk{Q}{\idtm{{A}{P}}}}
    {\ctxwk{Q}{{\ctxext{A}{P}}{{P}{\idtm{A}}}}}
  \\
& \jdeq
  \ctxwk{Q}{\subst{\idtm{{A}{P}}}{\ctxwk{\ctxext{A}{P}}{{P}{\idtm{A}}}}}
  \\
& \jdeq
  \ctxwk{Q}{{P}{\idtm{A}}}
  \\
& \jdeq
  \ctxwk{\ctxext{P}{Q}}{\idtm{A}}
  \\
& \jdeq
  \cprojfstf{A}{{P}{Q}}
\end{align*}
To prove the judgmental equality
\begin{equation*}
\jhomeq
  {{\Gamma}{A}}
  {{P}{Q}}
  {P}
  {\jcomp{}{\cprojfstf{{A}{P}}{Q}}{\cprojsndf{A}{P}}}
  {\jcomp{}{\cprojsndf{A}{{P}{Q}}}{\cprojfstf{P}{Q}}}
\end{equation*}
note that we have the judgmental equalities
\begin{align*}
\jcomp{{{A}{P}}{Q}}{\cprojfstf{{A}{P}}{Q}}{\cprojsndf{A}{P}}
& \jdeq 
  \unfoldall{\jcomp{{{A}{P}}{Q}}{\cprojfstf{{A}{P}}{Q}}{\cprojsndf{A}{P}}}
  \\
& \jdeq
  \subst{\ctxwk{Q}{\idtm{{A}{P}}}}{\ctxwk{Q}{{\ctxext{A}{P}}{\idtm{P}}}}
  \\
& \jdeq
  \ctxwk{Q}{\subst{\idtm{{A}{P}}}{\ctxwk{\ctxext{A}{P}}{\idtm{P}}}}
  \\
& \jdeq
  \ctxwk{Q}{\idtm{P}}
  \\
& \jdeq
  \unfoldall{\jcomp{{P}{Q}}{\cprojsndf{A}{{P}{Q}}}{\cprojfstf{P}{Q}}}
  \\
& \jdeq
  \jcomp{{P}{Q}}{\cprojsndf{A}{{P}{Q}}}{\cprojfstf{P}{Q}}
\end{align*}
To prove the judgmental equality
\begin{equation*}
\jhomeq
  {{{\Gamma}{A}}{P}}
  {Q}
  {Q}
  {\cprojsndf{{A}{P}}{Q}}
  {\jcomp{}{\cprojsndf{A}{{P}{Q}}}{\cprojsndf{P}{Q}}}
\end{equation*}
note that we have the judgmental equalities
\begin{align*}
\cprojsndf{{A}{P}}{Q}
& \jdeq
  \unfoldall{\cprojsndf{{A}{P}}{Q}}
  \\
& \jdeq 
  \unfoldall{\jcomp{{P}{Q}}{\cprojsndf{A}{{P}{Q}}}{\cprojsndf{P}{Q}}}
  \\
& \jdeq
  \jcomp{}{\cprojsndf{A}{{P}{Q}}}{\cprojsndf{P}{Q}}\qedhere
\end{align*}
\begin{comment}
%%%% This was a proof of an old version of the statement
Now consider a term $\jterm{\Gamma}{\ctxext{A}{{P}{Q}}}{w}$. Then we have
\begin{align*}
w 
& \jdeq 
  \tmext
    {A}
    {{P}{Q}}
    {\cprojfst{A}{\ctxext{P}{Q}}{w}}
    {\cprojsnd{A}{\ctxext{P}{Q}}{w}}
  \\
& \jdeq 
  \tmext
    {A}
    {{P}{Q}}
    {\cprojfst{A}{\ctxext{P}{Q}}{w}}
    { {P}
      {Q}
      {\cprojfst{P}{Q}{\cprojsnd{A}{\ctxext{P}{Q}}{w}}}
      {\cprojsnd{P}{Q}{\cprojsnd{A}{\ctxext{P}{Q}}{w}}}
      }
  \\
& \jdeq 
  \tmext
    {{A}{P}}
    {Q}
    { {} % need to provide base and family, but there's no unfold.
      {}
      {\cprojfst{A}{\ctxext{P}{Q}}{w}}
      {\cprojfst{P}{Q}{\cprojsnd{A}{\ctxext{P}{Q}}{w}}}
      }
    { \cprojsnd{P}{Q}{\cprojsnd{A}{\ctxext{P}{Q}}{w}}
      }
\end{align*}
Thus we see that 
\begin{align*}
\cprojfst{\ctxext{A}{P}}{Q}{w} 
& \jdeq 
  \tmext
    {A}
    {P}
    {\cprojfst{A}{\ctxext{P}{Q}}{w}}
    {\cprojfst{P}{Q}{\cprojsnd{A}{\ctxext{P}{Q}}{w}}}
  \\ 
\cprojsnd{\ctxext{A}{P}}{Q}{w} 
& \jdeq 
  \cprojsnd{P}{Q}{\cprojsnd{A}{\ctxext{P}{Q}}{w}},
\end{align*}
proving the fourth judgmental equality, and therefore also that
\begin{align*}
\cprojfst{A}{P}{\cprojfst{\ctxext{A}{P}}{Q}{w}} 
& \jdeq 
  \cprojfst{A}{\ctxext{P}{Q}}{w}
  \\
\cprojsnd{A}{P}{\cprojfst{\ctxext{A}{P}}{Q}{w}} 
& \jdeq 
  \cprojfst{P}{Q}{\cprojsnd{A}{\ctxext{P}{Q}}{w}},
\end{align*}
proving the second and the third judgmental equalities.
\end{comment}
\end{proof}

\begin{lem}\label{lem:tmext-wk}\label{comp-we-t}
When we weaken a term $\tmext{B}{Q}{y}{v}$ of $\ctxext{B}{Q}$ in context $\Gamma$ by
a family $A$, the term that we get is $\tmext{\ctxwk{A}{B}}{\ctxwk{A}{Q}}{\ctxwk{A}{y}}{\ctxwk{A}{v}}$. More
precisely, the following inference rules are valid:
\begin{align*}
& \inference
  { \jterm{{\Gamma}{B}}{Q}{g}
    \jterm{{\Gamma}{B}}{\subst{g}{R}}{t}
    }
  { \jtermeq
      {{{\Gamma}{A}}{\ctxwk{A}{B}}}
      {\ctxwk{A}{\ctxext{Q}{R}}}
      {\ctxwk{A}{\tmext{Q}{R}{g}{t}}}
      {\tmext{\ctxwk{A}{Q}}{\ctxwk{A}{R}}{\ctxwk{A}{g}}{\ctxwk{A}{t}}}
    }
  \\
& \inference
  { \jfam{{{\Gamma}{B}}{Q}}{R}
    }
  { \jhomeq
      {{{\Gamma}{A}}{\ctxwk{A}{B}}}
      {{\ctxwk{A}{Q}}{\ctxwk{A}{R}}}
      {\ctxwk{A}{Q}}
      {\cprojfstf{\ctxwk{A}{Q}}{\ctxwk{A}{R}}}
      {\ctxwk{A}{\cprojfstf{Q}{R}}}
    }
  \\
& \inference
  { \jfam{{{\Gamma}{B}}{Q}}{R}
    }
  { \jhomeq
      {{{{\Gamma}{A}}{\ctxwk{A}{B}}}{\ctxwk{A}{Q}}}
      {\ctxwk{A}{R}}
      {\ctxwk{A}{R}}
      {\cprojsndf{\ctxwk{A}{Q}}{\ctxwk{A}{R}}}
      {\ctxwk{A}{\cprojsndf{Q}{R}}}
    }
\end{align*}
\end{lem}

\begin{proof}
Consider $\jterm{{\Gamma}{B}}{Q}{g}$ and $\jterm{{\Gamma}{B}}{\subst{g}{R}}{t}$.
Then we have the judgmental equalities
\begin{align*}
\ctxwk{A}{\ctxext{Q}{R}{g}{t}}
& \jdeq 
  \ctxwk{A}{\subst{t}{{g}{\idtm{\ctxext{Q}{R}}}}}
  \\
& \jdeq 
  \subst{\ctxwk{A}{t}}{\ctxwk{A}{\subst{g}{\idtm{\ctxext{Q}{R}}}}}
  \\
& \jdeq 
  \subst{\ctxwk{A}{t}}{{\ctxwk{A}{g}}{\ctxwk{A}{\idtm{\ctxext{Q}{R}}}}}
  \\
& \jdeq 
  \subst{\ctxwk{A}{t}}{{\ctxwk{A}{g}}{\idtm{\ctxwk{A}{\ctxext{Q}{R}}}}}
  \\
& \jdeq 
  \subst{\ctxwk{A}{t}}{{\ctxwk{A}{g}}{\idtm{\ctxext{\ctxwk{A}{Q}}{\ctxwk{A}{R}}}}}
  \\
& \jdeq 
  \tmext{\ctxwk{A}{Q}}{\ctxwk{A}{R}}{\ctxwk{A}{g}}{\ctxwk{A}{t}}
\end{align*}
Next, we want to prove the judgmental equality
\begin{equation*}
\jhomeq
  {{{\Gamma}{A}}{\ctxwk{A}{B}}}
  {{\ctxwk{A}{Q}}{\ctxwk{A}{R}}}
  {\ctxwk{A}{Q}}
  {\cprojfstf{\ctxwk{A}{Q}}{\ctxwk{A}{R}}}
  {\ctxwk{A}{\cprojfstf{Q}{R}}}
\end{equation*}
Note that we have the judgmental equalities
\begin{align*}
\cprojfstf{\ctxwk{A}{Q}}{\ctxwk{A}{R}}
& \jdeq
  \unfoldall{\cprojfstf{\ctxwk{A}{Q}}{\ctxwk{A}{R}}}
  \\
& \jdeq
  \ctxwk{{A}{R}}{{A}{\idtm{Q}}}
  \\
& \jdeq
  \unfoldall{\ctxwk{A}{\cprojfstf{Q}{R}}}
  \\
& \jdeq
  \ctxwk{A}{\cprojfstf{Q}{R}}.
\end{align*}
Finally, we want to prove the judgmental equality
\begin{equation*}
\jhomeq
  {{{{\Gamma}{A}}{\ctxwk{A}{B}}}{\ctxwk{A}{Q}}}
  {\ctxwk{A}{R}}
  {\ctxwk{A}{R}}
  {\cprojsndf{\ctxwk{A}{Q}}{\ctxwk{A}{R}}}
  {\ctxwk{A}{\cprojsndf{Q}{R}}}
\end{equation*}
Note that we have the judgmental equalities
\begin{align*}
\cprojsndf{\ctxwk{A}{Q}}{\ctxwk{A}{R}}
& \jdeq
  \unfoldall{\cprojsndf{\ctxwk{A}{Q}}{\ctxwk{A}{R}}}
  \\
& \jdeq
  \unfoldall{\ctxwk{A}{\cprojsndf{Q}{R}}}
  \\
& \jdeq
  \ctxwk{A}{\cprojsndf{Q}{R}}.
  \qedhere
\end{align*}
\end{proof}

\begin{lem}\label{lem:tmext-subst}\label{comp-se-t}
When we substitute an extended term $\tmext{P}{Q}{f}{g}$ of $\ctxext{P}{Q}$ by a term
$x$ of $A$, the term that we get is $\tmext{\subst{x}{P}}{\subst{x}{Q}}{\subst{x}{f}}{\subst{x}{g}}$.
More precisely, the following inference rules are valid:
\begin{align*}
& \inference
  { \jterm{\Gamma}{A}{x}
    \jterm{{{\Gamma}{A}}{P}}{Q}{g}
    \jterm{{{\Gamma}{A}}{P}}{\subst{g}{R}}{t}
    }
  { \jtermeq
      {{\Gamma}{\subst{x}{P}}}
      {\ctxext{\subst{x}{Q}}{\subst{x}{R}}}
      {\subst{x}{\tmext{Q}{R}{g}{t}}}
      {\tmext{\subst{x}{Q}}{\subst{x}{R}}{\subst{x}{g}}{\subst{x}{t}}}
    }
  \\
& \inference
  { \jterm{\Gamma}{A}{x}
    \jfam{{{{\Gamma}{A}}{P}}{Q}}{R}
    }
  { \jhomeq
      {{\Gamma}{\subst{x}{P}}}
      {{\subst{x}{Q}}{\subst{x}{R}}}
      {\subst{x}{Q}}
      {\cprojfstf{\subst{x}{Q}}{\subst{x}{R}}}
      {\subst{x}{\cprojfstf{Q}{R}}}
    }
  \\
& \inference
  { \jterm{\Gamma}{A}{x}
    \jfam{{{{\Gamma}{A}}{P}}{Q}}{R}
    }
  { \jhomeq
      {{{\Gamma}{\subst{x}{P}}}{\subst{x}{Q}}}
      {\subst{x}{R}}
      {\subst{x}{R}}
      {\cprojsndf{\subst{x}{Q}}{\subst{x}{R}}}
      {\subst{x}{\cprojsndf{Q}{R}}}
    }
\end{align*}
\end{lem}

\begin{proof}
Consider $\jterm{{\Gamma}{B}}{Q}{g}$ and $\jterm{{\Gamma}{B}}{\subst{g}{R}}{t}$.
Then we have the judgmental equalities
\begin{align*}
\subst{x}{\tmext{Q}{R}{g}{t}}
& \jdeq 
  \subst{x}{{t}{{g}{\idtm{\ctxext{Q}{R}}}}}
  \\
& \jdeq 
  \subst{{x}{t}}{{x}{{g}{\idtm{\ctxext{Q}{R}}}}}
  \\
& \jdeq 
  \subst{{x}{t}}{{{x}{g}}{{x}{\idtm{\ctxext{Q}{R}}}}}
  \\
& \jdeq 
  \subst{{x}{t}}{{{x}{g}}{\idtm{\subst{x}{\ctxext{Q}{R}}}}}
  \\
& \jdeq 
  \subst{{x}{t}}{{{x}{g}}{\idtm{\ctxext{\subst{x}{Q}}{\subst{x}{R}}}}}
  \\
& \jdeq 
  \tmext{\subst{x}{Q}}{\subst{x}{R}}{\subst{x}{g}}{\subst{x}{t}}.
\end{align*}
Next, we want to prove the judgmental equality
\begin{equation*}
\jhomeq
  {{\Gamma}{\subst{x}{P}}}
  {{\subst{x}{Q}}{\subst{x}{R}}}
  {\subst{x}{Q}}
  {\cprojfstf{\subst{x}{Q}}{\subst{x}{R}}}
  {\subst{x}{\cprojfstf{Q}{R}}}
\end{equation*}
Note that we have the judgmental equalities
\begin{align*}
\cprojfstf{\subst{x}{Q}}{\subst{x}{R}}
& \jdeq
  \unfoldall{\cprojfstf{\subst{x}{Q}}{\subst{x}{R}}}
  \\
& \jdeq
  \ctxwk{\subst{x}{R}}{\subst{x}{\idtm{Q}}}
  \\
& \jdeq
  \unfoldall{\subst{x}{\cprojfstf{Q}{R}}}
  \\
& \jdeq
  \subst{x}{\cprojfstf{Q}{R}}.
\end{align*}
And finally we want to prove the judgmental equality
\begin{equation*}
\jhomeq
  {{{\Gamma}{\subst{x}{P}}}{\subst{x}{Q}}}
  {\subst{x}{R}}
  {\subst{x}{R}}
  {\cprojsndf{\subst{x}{Q}}{\subst{x}{R}}}
  {\subst{x}{\cprojsndf{Q}{R}}}
\end{equation*}
Note that we have the judgmental equalities
\begin{align*}
\cprojsndf{\subst{x}{Q}}{\subst{x}{R}}
& \jdeq
  \unfoldall{\cprojsndf{\subst{x}{Q}}{\subst{x}{R}}}
  \\
& \jdeq
  \unfoldall{\subst{x}{\cprojsndf{Q}{R}}}
  \\
& \jdeq
  \subst{x}{\cprojsndf{Q}{R}}.
  \qedhere
\end{align*}
\end{proof}

We find the following inference rule, which expresses that the identity term
is compatible with extension:

\begin{lem}\label{lem:tmext-id}\label{comp-ie}
For any $\jfam{\Gamma}{A}$ and $\jfam{{\Gamma}{A}}{P}$ we have
\begin{equation}\label{idfunc-ext-comp}
\inference
  { \jfam{\Gamma}{A}
    \jfam{{\Gamma}{A}}{P}
    }
  { \jhomeq
      {\Gamma}
      {{A}{P}}{{A}{P}}
      {\idtm{\ctxext{A}{P}}}
      { \tmext
          {\ctxwk{\ctxext{A}{P}}{A}}
          {\ctxwk{\ctxext{A}{P}}{P}}
          {\cprojfstf{A}{P}}
          {\cprojsndf{A}{P}}
        }
    }
\end{equation}
\end{lem}

\begin{proof}
Consider the families $\jfam{\Gamma}{A}$ and $\jfam{{\Gamma}{A}}{P}$. Then
we have the judgmental equalities
\begin{align*}
\tmext
  {\ctxwk{\ctxext{A}{P}}{A}}
  {\ctxwk{\ctxext{A}{P}}{P}}
  {\cprojfstf{A}{P}}
  {\cprojsndf{A}{P}}
& \jdeq 
  \unfold
  { \tmext
      {\ctxwk{\ctxext{A}{P}}{A}}
      {\ctxwk{\ctxext{A}{P}}{P}}
      {\cprojfstf{A}{P}}
      {\cprojsndf{A}{P}}
    }
  \\
& \jdeq 
  \subst
    { \idtm{P}
      }
    { {\ctxwk{P}{\idtm{A}}}
      {\idtm{\ctxwk{\ctxext{A}{P}}{\ctxext{A}{P}}}}
      }
  \\
& \jdeq 
  \subst
    { \idtm{P}
      }
    { {\ctxwk{P}{\idtm{A}}}
      {\ctxwk{\ctxext{A}{P}}{\idtm{\ctxext{A}{P}}}}
      }
  \\
& \jdeq 
  \subst
    { \idtm{P}
      }
    { {\ctxwk{P}{\idtm{A}}}
      {\ctxwk{P}{{A}{\idtm{\ctxext{A}{P}}}}}
      }
  \\
& \jdeq 
  \subst
    { \idtm{P}
      }
    {\ctxwk
      {P}
      { \subst
        {\idtm{A}}
        {\ctxwk{A}{\idtm{\ctxext{A}{P}}}}
        }
      }
  \\
& \jdeq 
  \subst
    { \idtm{A}
      }
    { \ctxwk{A}{\idtm{\ctxext{A}{P}}}
      }
  \\
& \jdeq 
  \idtm{\ctxext{A}{P}}.
  \qedhere
\end{align*}
\end{proof}

\subsection{More on morphisms}

There is also a notion of morphism \emph{over} a morphism. We will develop this
notion because it will be needed in the theory of models later on.

\begin{defn}
Let $\jhom{\Gamma}{A}{B}{f}$ be a morphism from $A$ to $B$ in context $\Gamma$
and consider $\jfam{{\Gamma}{A}}{P}$ and $\jfam{{\Gamma}{B}}{Q}$. We define the
judgment\begin{equation*}
\jfhom{\Gamma}{A}{B}{f}{P}{Q}{F},
\end{equation*}
which is pronounced as `$F$ is a morphism from $P$ to $Q$ over $f$ in context
$\Gamma$', to be the judgment\begin{equation*}
\unfold{\jfhom{\Gamma}{A}{B}{f}{P}{Q}{F}}.
\end{equation*}
\end{defn}

\begin{rmk}
The judgment $\jfhom{\Gamma}{A}{B}{f}{P}{Q}{F}$ means the same thing as\begin{equation*}
\jhom{{\Gamma}{A}}{P}{\jcomp{A}{f}{Q}}{F}.
\end{equation*}
\end{rmk}

Suppose we have morphisms $\jhom{\Gamma}{A}{B}{f}$ and $\jhom{\Gamma}{B}{C}{g}$
and that we have the morphisms $\jfhom{\Gamma}{A}{B}{f}{P}{Q}{F}$ and
$\jfhom{\Gamma}{B}{C}{g}{Q}{R}{G}$ over them. Then we have\begin{equation*}
\jhom
  {{\Gamma}{A}}
  {\jcomp{A}{f}{Q}}
  {\jcomp{A}{f}{{B}{g}{R}}}
  {\unfold{\jcomp{A}{f}{G}}}
\end{equation*}
Because we also have $\jhom{{\Gamma}{A}}{P}{\jcomp{A}{f}{Q}}{F}$, we have the
composition\begin{equation*}
\jhom
  {{\Gamma}{A}}
  {P}
  {\jcomp{A}{f}{{B}{g}{R}}}
  {\jcomp{P}{F}{\unfold{\jcomp{A}{f}{G}}}}.
\end{equation*}
Because of 
the judgmental equality $\jcomp{A}{f}{{B}{g}{R}}\jdeq
\jcomp{A}{{A}{f}{g}}{R}$, it follows that 
$\jcomp{P}{F}{\unfold{\jcomp{A}{f}{G}}}$ is a morphism from $P$ to $R$ over
$\jcomp{A}{f}{g}$. We make the following definition:

\begin{defn}
Let $\jhom{\Gamma}{A}{B}{f}$ and $\jhom{\Gamma}{B}{C}{g}$
be morphisms and let $\jfhom{\Gamma}{A}{B}{f}{P}{Q}{F}$ and
$\jfhom{\Gamma}{B}{C}{g}{Q}{R}{G}$ be morphisms over them. Then we define\begin{equation*}
\jfhomdefn
  {\Gamma}
  {A}
  {C}
  {\jcomp{A}{f}{g}}
  {P}
  {R}
  {\jfcomp{A}{f}{P}{F}{G}}
  {\unfold{\jfcomp{A}{f}{P}{F}{G}}}.
\end{equation*}
\end{defn}

This composition is also judgmentally associative.

Now let $\jfam{{\Gamma}{A}}{P}$ and $\jfam{{\Gamma}{A}}{Q}$ be families. A
morphism from $P$ to $Q$ over the identity term $\idtm{A}$ in context
$\Gamma$ is the same thing as a morphism from $P$ to $Q$ in context
$\ctxext{\Gamma}{A}$:

\begin{lem}\label{hom-over-id-is-hom}
We have the following valid inference rules:\begin{align*}
& \inference
  { \jfam{{\Gamma}{A}}{P}
    \jfam{{\Gamma}{A}}{Q}
    \jfhom{\Gamma}{A}{A}{\idtm{A}}{P}{Q}{f}
    }
  { \jhom{{\Gamma}{A}}{P}{Q}{f}
    }
  \\
& \inference
  { \jfam{{\Gamma}{A}}{P}
    \jfam{{\Gamma}{A}}{Q}
    \jhom{{\Gamma}{A}}{P}{Q}{f}
    }
  { \jfhom{\Gamma}{A}{A}{\idtm{A}}{P}{Q}{f}
    }
\end{align*}
\end{lem}

\begin{proof}
If we unfold the judgments $\jhom{{\Gamma}{A}}{P}{Q}{f}$ and
$\jfhom{\Gamma}{A}{A}{\idtm{A}}{P}{Q}{f}$, we get the judgments\begin{align*}
& \unfold{\jhom{{\Gamma}{A}}{P}{Q}{f}}
  \\
& \unfold{\jfhom{\Gamma}{A}{A}{\idtm{A}}{P}{Q}{f}}
\end{align*}
respectively. Therefore, we only need to verify that
$\ctxwk{P}{\subst{\idtm{A}}{\ctxwk{A}{Q}}}\jdeq\ctxwk{P}{Q}$, which is indeed
the case by \autoref{idfunc-wk-defn}.
\end{proof}

For the following lemma, recall that the judgment $\jhom{\Gamma}{A}{{B}{Q}}{f}$
unfolds as
\begin{equation*}
\unfold{\jhom{\Gamma}{A}{{B}{Q}}{f}}
\end{equation*}
and that we have the judgmental equality 
$ \jfameq
    {{\Gamma}{A}}
    {\ctxwk{A}{\ctxext{B}{Q}}}
    {\ctxext{\ctxwk{A}{B}}{\ctxwk{A}{Q}}}.
  $
Therefore, each morphism into an extended family can itself be described as
an extended term. The following lemma explains how this goes.

\begin{lem}
Let $\jhom{\Gamma}{A}{{B}{Q}}{f}$ be a morphism from $A$ to $\ctxext{B}{Q}$
in a context $\Gamma$. Then we have
\begin{equation*}
\jhomeq
  {\Gamma}
  {A}
  {{B}{Q}}
  {f}
  {\tmext{\jcomp{}{f}{\cprojfstf{B}{Q}}}{\jcomp{}{f}{\cprojsndf{B}{Q}}}}.
\end{equation*}
Alternatively, when $\jhom{\Gamma}{A}{B}{f_0}$ and 
$\jterm{{\Gamma}{A}}{\jcomp{}{f_0}{Q}}{f_1}$ we obtain a morphism
$\jhom{\Gamma}{A}{{B}{Q}}{\tmext{f_0}{f_1}}$ with the property that
\begin{align*}
\jhomeq*{\Gamma}{A}{B}{\jcomp{}{\tmext{f_0}{f_1}}{\cprojfstf{B}{Q}}}{f_0}\\
\jtermeq*{{\Gamma}{A}}{\jcomp{}{f_0}{Q}}{\jcomp{}{\tmext{f_0}{f_1}}{\cprojsndf{B}{Q}}}{f_1}.
\end{align*}
\end{lem}

\begin{proof}
Let $\jhom{\Gamma}{A}{{B}{Q}}{f}$ be a morphism in context $\Gamma$. Then we
have the judgmental equalities
\begin{align*}
\cprojfst{\ctxwk{A}{B}}{\ctxwk{A}{Q}}{f}
& \jdeq
  \subst{f}{\ctxwk{A}{\cprojfstf{B}{Q}}}
  \\
& \jdeq
  \jcomp{}{f}{\cprojfstf{B}{Q}}
  \\
\cprojsnd{\ctxwk{A}{B}}{\ctxwk{A}{Q}}{f}
& \jdeq
  \subst{f}{\ctxwk{A}{\cprojsndf{B}{Q}}}
  \\
& \jdeq
  \jcomp{}{f}{\cprojsndf{B}{Q}}
\end{align*}
The alternative formulation of the statement is a direct corollary.
\end{proof}

We also have the following lemma about the compatibility of pairing and composition:

\begin{lem}
The following inference rule is valid
\begin{align*}
& \inference
  { \jhom{\Gamma}{A}{B}{f}
    \jhom{\Gamma}{B}{C}{g}
    \jfam{{\Gamma}{C}}{R}
    \jterm{{\Gamma}{B}}{\subst{g}{\ctxwk{B}{R}}}{w}
    }
  { \jhomeq
      {\Gamma}
      {A}
      {{C}{R}}
      {\jcomp{A}{f}{\tmext{\ctxwk{B}{C}}{\ctxwk{B}{R}}{g}{w}}}
      {\tmext{\ctxwk{A}{C}}{\ctxwk{A}{R}}{\jcomp{A}{f}{g}}{\jcomp{A}{f}{w}}}
    }
\end{align*}
\end{lem}

\begin{proof}
Let $\jhom{\Gamma}{A}{B}{f}$, $\jhom{\Gamma}{B}{C}{g}$, $\jfam{{\Gamma}{C}}{R}$
and $\jterm{{\Gamma}{B}}{\subst{g}{\ctxwk{B}{R}}}{w}$. Then we have the
judgmental equalities
\begin{align*}
\jcomp{A}{f}{\tmext{\ctxwk{B}{C}}{\ctxwk{B}{R}}{g}{w}}
& \jdeq 
  \subst{f}{\ctxwk{A}{\tmext{\ctxwk{B}{C}}{\ctxwk{B}{R}}{g}{w}}}
  \\
& \jdeq 
  \subst
    {f}
    {\tmext{\ctxwk{A}{{B}{C}}}{\ctxwk{A}{{B}{R}}}{\ctxwk{A}{g}}{\ctxwk{A}{w}}}
  \\
& \jdeq 
  \tmext
    {\ctxwk{A}{C}}
    {\ctxwk{A}{R}}
    {\subst{f}{\ctxwk{A}{g}}}
    {\subst{f}{\ctxwk{A}{w}}}
  \\
& \jdeq 
  \tmext{\ctxwk{A}{C}}{\ctxwk{A}{R}}{\jcomp{A}{f}{g}}{\jcomp{A}{f}{w}}.
  \qedhere
\end{align*}
\end{proof}

\begin{defn}
Let $\jhom{\Gamma}{A}{B}{f}$ be a morphism from $A$ to $B$ in context $\Gamma$
and let $\jfhom{\Gamma}{A}{B}{f}{P}{Q}{F}$ be a morphism over $f$ in context 
$\Gamma$. We define
\begin{equation*}
\jhomdefn{\Gamma}{{A}{P}}{{B}{Q}}{\jvcomp{P}{f}{F}}{\unfold{\jvcomp{P}{f}{F}}}
\end{equation*}
\end{defn}

\subsection{Fiber inclusions}
We will use the insights of \autoref{extension-on-terms} to define and study
\emph{fiber inclusions}. The fiber inclusion of the \emph{fiber}
$\subst{x}{P}$ into the extension $\ctxext{A}{P}$ is a morphism
$\jhom{\Gamma}{\subst{x}{P}}{{A}{P}}{\finc{x}{P}}$, for any family
$\jfam{{\Gamma}{A}}{P}$ and any term $\jterm{\Gamma}{A}{x}$. Then we will determine
the ways in which it is compatible with the other operators. Note that in this
subsection we will focus on the compatibility properties; the fact that
the fiber inclusions also appear in a pullback diagram will be established in
\autoref{pullback}. 

\begin{defn}
Let $\jterm{\Gamma}{A}{x}$ be a term and let $\jfam{{\Gamma}{A}}{P}$ be a
family. Then we define the \emph{fiber inclusion} of $\subst{x}{P}$ into
$\ctxext{A}{P}$ in context $\Gamma$ to be the morphism
\begin{equation*}
\jhomdefn{\Gamma}{\subst{x}{P}}{{A}{P}}{\finc{x}{P}}{\unfoldnext{\finc{x}{P}}}.
\end{equation*}
\end{defn}

We have the following lemmas expressing the compatibility of the fiber
inclusions with the empty context, extension, weakening and substitution. 

\begin{lem}
The fiber inclusions are compatible with the empty families; i.e.~the following
inference rules are valid
\begin{align*}
& \inference
  { \jterm{\Gamma}{A}{x}
    }
  { \jtermeq
      {\Gamma}
      {A}
      {\finc{x}{\emptyf}}
      {x}
    }
  \\
& \inference
  { \jfam{\Gamma}{A}
    }
  { \jhomeq
      {\Gamma}
      {A}
      {A}
      {\finc{\emptytm}{A}}
      {\idtm{A}}
    }
\end{align*}
\end{lem}

\begin{lem}
The fiber inclusions are compatible with extension; i.e.~the following inference
rule is valid
\begin{equation*}
\inference
  { \jterm{\Gamma}{A}{x}
    \jterm{\Gamma}{\subst{x}{P}}{u}
    \jfam{{{\Gamma}{A}}{P}}{Q}
    }
  { \jhomeq
      {\Gamma}
      {\subst{\tmext{x}{u}}{Q}}
      {{{A}{P}}{Q}}
      {\finc{\tmext{x}{u}}{Q}}
      {\jcomp{}{\finc{u}{\subst{x}{Q}}}{\finc{x}{\ctxext{P}{Q}}}}
    }
\end{equation*}
\end{lem}

\begin{lem}
The fiber inclusions are compatible with weakening; i.e.~the following inference
rule is valid
\begin{equation*}
\inference
  { \jfam{\Gamma}{A}
    \jterm{\Gamma}{B}{y}
    \jfam{{\Gamma}{B}}{Q}
    }
  { \jhomeq
      {{\Gamma}{A}}
      {\subst{\ctxwk{A}{y}}{\ctxwk{A}{Q}}}
      {{\ctxwk{A}{B}}{\ctxwk{A}{Q}}}
      {\finc{\ctxwk{A}{y}}{\ctxwk{A}{Q}}}
      {\ctxwk{A}{\finc{y}{Q}}}
    }
\end{equation*}
\end{lem}

\begin{lem}
The fiber inclusions are compatible with substitution; i.e.~the following
inference rule is valid
\begin{equation*}
\inference
  { \jterm{\Gamma}{A}{x}
    \jfam{{{\Gamma}{A}}{P}}{Q}
    \jterm{{\Gamma}{A}}{P}{f}
    }
  { \jhomeq
      {\Gamma}
      {\subst{{x}{f}}{{x}{Q}}}
      {{\subst{x}{P}}{\subst{x}{Q}}}
      {\finc{\subst{x}{f}}{\subst{x}{Q}}}
      {\subst{x}{\finc{f}{Q}}}
    }
\end{equation*}
\end{lem}

\begin{lem}
The fiber inclusions are compatible with identity terms; i.e.~the following
inference rule is valid
\begin{equation*}
\inference
  { \jfam{{\Gamma}{A}}{P}
    }
  { \jhomeq
      {{\Gamma}{A}}
      {P}
      {\ctxwk{A}{\ctxext{A}{P}}}
      {\finc{\idtm{A}}{\ctxwk{A}{P}}}
      {\idtm{{A}{P}}}
    }
\end{equation*}
\end{lem}

\subsection{Pullback squares and family pullback squares}
\label{pullback}
Now that we have introduced the notions of morphisms, composition and identity
terms, we can develop a diagramatic style of of displaying type dependencies
combined with morphisms. We give an informal, metatheoretical definition of
such diagrams by indicating what the various components mean. The definition
is informal because we will only use such diagrams occasionally to provide a
graphical indication of the situation in which we're working. In particular,
we will not shy away from using natural numbers and trust that the reader can
figure out what we mean.

\begin{defn}
A diagram is said to be a \emph{dependency diagram in context $\Gamma$}
if it is built up according to the following steps:
\begin{itemize}
\item The arrows appearing in a dependency diagram are either ordinary, like the
arrow%
$\begin{tikzcd}[ampersand replacement = \&]
X \ar{r} \& Y,
\end{tikzcd}$
or double-headed, like
$\begin{tikzcd}[ampersand replacement = \&]
X \ar[fib]{r} \& Y.
\end{tikzcd}$
\item An ordinary arrow 
\begin{equation*}
\begin{tikzcd}
A \ar{r}{f} & B
\end{tikzcd}
\end{equation*}
between two families $A$ and $B$ of contexts over $\Gamma$ indicates that
$f$ is a morphism from $A$ to $B$ in context $\Gamma$, i.e.~that we have the
judgment $\jhom{\Gamma}{A}{B}{f}$.
\item The set of double-headed arrows must form a forest and the root of
each maximal tree of double-headed arrows is a family of contexts over $\Gamma$.
In particular, if an object is not the domain of a double-headed arrow it must
be a family of contexts over $\Gamma$.
\item A sequence of double-headed 
arrows
\begin{equation*}
\begin{tikzcd}
P_{n} \ar[fib]{r} & \cdots \ar[fib]{r} & P_1 \ar[fib]{r} & A
\end{tikzcd}
\end{equation*}
indicates that $P_1$ is a family of contexts over $\ctxext{\Gamma}{A}$, that
$P_2$ is a family of contexts over $\ctxext{{\Gamma}{A}}{P_1}$, etcetera.
\item There can be two kinds of ladders of double-headed arrows:
\begin{equation*}
\begin{tikzcd}
P_{n} \ar{r}{F_{n}} \ar[fib]{d} & Q_{n} \ar[fib]{d}\\
\vdots \ar[fib]{d} & \vdots \ar[fib]{d}\\
P_1 \ar{r}{F_1} \ar[fib]{d} & Q_1 \ar[fib]{d}\\
A \ar{r}{f} & B
\end{tikzcd}
\qquad
\begin{tikzcd}[column sep = tiny]
P_{n+m} \ar{rr}{F_{n+m}} \ar[fib]{d} & & Q_{n+m} \ar[fib]{d}\\
\vdots \ar[fib]{d} & & \vdots \ar[fib]{d}\\
P_{n+1} \ar{rr}{F_{n+1}} \ar[fib]{dr} & & Q_{n+1} \ar[fib]{dl}\\
& P_n \ar[fib]{d}\\
& \vdots \ar[fib]{d}\\
& P_1 \ar[fib]{d}\\
& A
\end{tikzcd}
\end{equation*}
The ladder on the left 
indicates that $F_1$ is a morphism from $P_1$ to $Q_1$ \emph{over} $f$,
i.e.~that the judgment $\jfhom{\Gamma}{A}{B}{f}{P_1}{Q_1}{F_1}$ holds, that
$F_2$ is a morphism from $P_2$ to $Q_2$ over
the morphism $\tmext{\ctxwk{P_1}{f}}{F_1}$ from $\ctxext{A}{P_1}$ to
$\ctxext{B}{Q_1}$, etcetera.

The ladder on the right indicates that $F_{n+1}$ is a morphism from $P_{n+1}$ to
$Q_{n+1}$ in the appropriate context, that $F_{n+2}$ is a morphism from
$P_{n+2}$ to $Q_{n+2}$ over $F_{n+1}$, etcetera.
 
Note that the object(s) at the bottom of a ladder are always families of contexts
over $\Gamma$, so that the typing of the various ingredients makes sense.
\end{itemize}
Such a diagram is said to be commutative if the subdiagram consisting of only
the normal headed arrows is commutative in the usual sense (using judgmental
equality). Note that the ladders are inherently commutative.
\end{defn}

The most basic illustrative example of a commutative dependency diagram is
the diagram
\begin{equation*}
\begin{tikzcd}
P \ar[fib]{d} \ar{r}{F} & Q \ar[fib]{d} \\
A \ar{r}{f} & B
\end{tikzcd}
\end{equation*}
indicating a morphism $F$ from $P$ to $Q$ over the morphism $f$ from $A$ to
$B$ in a context $\Gamma$.

We can just copy the usual categorical definition of a pullback square to our
current situation, but we have to require that each arrow in the pullback square
is an ordinary arrow. When families (i.e. double-headed arrows) are involved
in the diagram, we make the following definition of a family pullback:

\begin{defn}
We say that a commutative dependency diagram of the form
\begin{equation*}
\begin{tikzcd}
P \ar[fib]{d} \ar{r}{F} & Q \ar[fib]{d} \\
A \ar{r}{f} & B
\end{tikzcd}
\end{equation*}
is a \emph{family pullback} if the following inference rules are valid:
\begin{align*}
& \inference
  { \jfam{{\Gamma}{A}}{P'}
    \jfhom{\Gamma}{A}{B}{f}{P'}{Q}{F'}
    }
  { \jhom{{\Gamma}{A}}{P'}{P}{u}
    }
  \\
& \inference
  { \jfam{{\Gamma}{A}}{P'}
    \jfhom{\Gamma}{A}{B}{f}{P'}{Q}{F'}
    }
  { \jfhomeq{\Gamma}{A}{B}{f}{P'}{Q}{\jcomp{}{u}{F}}{F'}
    }
  \\
& \inference
  { \jhom{{\Gamma}{A}}{P'}{P}{v}
    \jfhomeq{\Gamma}{A}{B}{f}{P'}{Q}{\jcomp{}{v}{F}}{F'}
    }
  { \jhomeq{{\Gamma}{A}}{P'}{P}{v}{u}
    }
\end{align*}
\end{defn}

The following lemma explains that when a square involving families is a
family pullback square whenever the corresponding square involving projections is a
pullback square. There is no proof in the the opposite direction.

\begin{lem}
A square
\begin{equation}\label{eq:fpb_to_pb_eqv_fpb}
\begin{tikzcd}
P \ar[fib]{d} \ar{r}{F} & Q \ar[fib]{d} \\
A \ar{r}{f} & B
\end{tikzcd}
\end{equation}
is a family pullback square whenever the square
\begin{equation}\label{eq:fpb_to_pb_eqv_pb}
\begin{tikzcd}[column sep = large]
\ctxext{A}{P} \ar{d}[swap]{\cprojfstf{A}{P}} \ar{r}{\tmext{\ctxwk{P}{f}}{F}} & \ctxext{B}{Q} \ar{d}{\cprojfstf{B}{Q}} \\
A \ar{r}{f} & B
\end{tikzcd}
\end{equation}
is a pullback square.
\end{lem}

The family pullback of a family along any morphism always exists. It is simply given
by the precomposition of the family with the morphism. Note that this fact does
not carry over to arbitrary pullbacks.

\begin{lem}
The diagram
\begin{equation*}
\begin{tikzcd}
\jcomp{}{f}{Q} \ar[fib]{d} \ar{r}{\idtm{\jcomp{}{f}{Q}}} & Q \ar[fib]{d} \\
A \ar{r}{f} & B
\end{tikzcd}
\end{equation*}
is a family pullback diagram.
\end{lem}

\begin{proof}
The proof is a triviality because $\jhom{{\Gamma}{A}}{P'}{\jcomp{}{f}{Q}}{F'}$
is the same judgment as $\jfhom{\Gamma}{A}{B}{f}{P'}{Q}{F'}$ and
$\jcomp{}{\idtm{\jcomp{}{f}{Q}}}{F'}\jdeq F'$.
\end{proof}

For arbitrary pullbacks we have the pasting lemma as usual, but for family
pullbacks we can only derive one of the parts of the pasting lemma.

\begin{lem}
Suppose we have the diagram
\begin{equation*}
\begin{tikzcd}
P \ar{r}{F} \ar[fib]{d} & Q \ar{r}{G} \ar[fib]{d} & R \ar[fib]{d}\\
A \ar{r}{f} & B \ar{r}{g} & C
\end{tikzcd}
\end{equation*}
where the square on the right and the outer rectangle are family pullback 
diagrams. Then the square on the left is a family pullback diagram.
\end{lem}

\begin{proof}
Let $\jfam{{\Gamma}{A}}{P'}$ be a family and let $\jfhom{\Gamma}{A}{B}{f}
{P'}{Q}{F}$ be a morphism over $f$.
\begin{itemize}
\item Then we compose $F'$ with $G$ to obtain a morphism over $\jcomp{}{f}{g}$.
\item Then we get $\jhom{{\Gamma}{A}}{P'}{P}{u}$ with a uniqueness property.
      The property that $\jcomp{}{u}{F}\jdeq F'$ follows from the assumption
      that the right square is a pullback.
\item Now assume that we have another such $v$. Compose it with $F$ and $G$.
      By the assumed properties this is the same as $u$ composed with $F$ and
      $G$. By the pullback condition we now get $u\jdeq v$. 
\end{itemize}
\end{proof}

\begin{lem}
For any $\jterm{\Gamma}{A}{x}$ and any $\jfam{{\Gamma}{A}}{P}$, the square
\begin{equation*}
\begin{tikzcd}
\subst{x}{P} \ar{d} \ar{r}{\finc{x}{P}} & \ctxext{A}{P} \ar{d}{\cprojfstf{A}{P}}\\
\emptyf \ar{r}{\ctxwk{\emptyf}{x}} & A
\end{tikzcd}
\end{equation*}
is a pullback square.
\end{lem}

In the following lemma we assert that pulling back a family $Q$ 
over $\ctxext{{\Gamma}{A}}{P}$ along a fiber
inclusion $\finc{x}{P}$ gives the family $\subst{x}{Q}$ over $\ctxext{\Gamma}{\subst{x}{P}}$. 

\begin{lem}
We have the following inference rule:
\begin{equation*}
\inference
  { \jfam{{{\Gamma}{A}}{P}}{Q}
    \jterm{\Gamma}{A}{x}
    }
  { \jfameq
      {{\Gamma}{\subst{x}{P}}}
      {\jcomp{}{\finc{x}{P}}{Q}}
      {\subst{x}{Q}}
    }
\end{equation*}
\end{lem}

\begin{proof}
We have the judgmental equalities:
\begin{align*}
\jcomp{\subst{x}{P}}{\finc{x}{P}}{Q}
& \jdeq
  \subst{\tmext{\ctxwk{\subst{x}{P}}{x}}{\idtm{\subst{x}{P}}}}{\ctxwk{\subst{x}{P}}{Q}}
  \\
& \jdeq
  \subst{\idtm{\subst{x}{P}}}{{\ctxwk{\subst{x}{P}}{x}}{\ctxwk{\subst{x}{P}}{Q}}}
  \\
& \jdeq
  \subst{\idtm{\subst{x}{P}}}{\ctxwk{\subst{x}{P}}{\subst{x}{Q}}}
  \\
& \jdeq
  \subst{x}{Q}.
\end{align*}
\end{proof}

\begin{lem}
The following inference rule is valid:
\begin{equation*}
\inference
  { \jfam{{\Gamma}{A}}{P}
    \jfam{{\Gamma}{A}}{Q}
    }
  { \jfameq
      {{\Gamma}{{A}{P}}}
      {\jcomp{{A}{P}}{\cprojfstf{A}{P}}{Q}}
      {\ctxwk{P}{Q}}
    }
\end{equation*}
\end{lem}

\begin{proof}
Let $\jfam{{\Gamma}{A}}{P}$ and $\jfam{{\Gamma}{A}}{Q}$ be
families. Then we have
\begin{align*}
\jcomp{{A}{P}}{\cprojfstf{A}{P}}{Q}
& \jdeq
  \unfoldall{\jcomp{{A}{P}}{\cprojfstf{A}{P}}{Q}}
  \tag{by definition}\\
& \jdeq 
  \subst{\ctxwk{P}{\idtm{A}}}{\ctxwk{P}{{A}{Q}}} 
  \tag{by \autoref{comp-ww-f}}\\
& \jdeq 
  \ctxwk{P}{\subst{\idtm{A}}{\ctxwk{A}{Q}}} 
  \tag{by \autoref{comp-ws-f}}\\
& \jdeq 
  \ctxwk{P}{Q} 
  \tag{by \autoref{idfunc-wk-defn}}
\end{align*}
\end{proof}


\subsection{Another special case of projections}
In this subsection we investigate the special case of a projection which
appears as a morphism from $\ctxext{{A}{P}}{\ctxwk{P}{Q}}$ to $\ctxext{A}{Q}$
in context $\Gamma$, where we assume to have the families 
$\jfam{{\Gamma}{A}}{P}$ and $\jfam{{\Gamma}{A}}{Q}$. 

Note that we have the judgmental
equalities
\begin{align*}
\ctxwk{\ctxext{{A}{P}}{\ctxwk{P}{\mfam{A}}}}{\ctxext{A}{\mfam{A}}}
& \jdeq 
  \ctxext
    {\ctxwk{\ctxext{{A}{P}}{\ctxwk{P}{\mfam{A}}}}{A}}
    {\ctxwk{\ctxext{{A}{P}}{\ctxwk{P}{\mfam{A}}}}{\mfam{A}}}
  \\
& \jdeq
  \ctxext
    {\ctxwk{{P}{\mfam{A}}}{{\ctxext{A}{P}}{A}}}
    {\ctxwk{\ctxext{{A}{P}}{\ctxwk{P}{\mfam{A}}}}{\mfam{A}}}
\end{align*}
Note that we have the term $\ctxwk{{P}{\mfam{A}}}{\cprojfstf{A}{P}}$ of the
family $\ctxwk{{P}{\mfam{A}}}{{\ctxext{A}{P}}{A}}$. Therefore, we need to
find a term of type $\subst{\ctxwk{{P}{\mfam{A}}}{\cprojfstf{A}{P}}}
{\ctxwk{\ctxext{{A}{P}}{\ctxwk{P}{\mfam{A}}}}{\mfam{A}}}$. Note that we have
the judgmental equalities:
\begin{align*}
\subst
  {\ctxwk{{P}{\mfam{A}}}{\cprojfstf{A}{P}}}
  {\ctxwk{\ctxext{{A}{P}}{\ctxwk{P}{\mfam{A}}}}{\mfam{A}}}
& \jdeq
  \subst
    {\ctxwk{{P}{\mfam{A}}}{\cprojfstf{A}{P}}}
    {\ctxwk{{P}{\mfam{A}}}{{\ctxext{A}{P}}{\mfam{A}}}}
  \\
& \jdeq
  \ctxwk
    { {P}{\mfam{A}}
      }
    { \subst
        {\cprojfstf{A}{P}}
        {\ctxwk{\ctxext{A}{P}}{\mfam{A}}}
      }
  \\
& \jdeq
  \ctxwk
    { {P}{\mfam{A}}
      }
    { {P}{\mfam{A}}
      }
  \\
& \jdeq
  \ctxwk{P}{{\mfam{A}}{\mfam{A}}}
\end{align*}
We find the term $\ctxwk{P}{\idtm{\mfam{A}}}$ here. Thus we can now define
$\bar{\typefont{pr}}$ by:
\begin{equation}\label{barproj}
\jhomdefn
  {\Gamma}
  {{{A}{P}}{\mfam{A}}}
  {{A}{\mfam{A}}}
  {\bar{\typefont{pr}}}
  {\tmext{\ctxwk{{P}{\mfam{A}}}{\cprojfstf{A}{P}}}{\ctxwk{P}{\idtm{\mfam{A}}}}}
\end{equation}

\begin{lem}
We have the judgmental equality
\begin{equation*}
\jfameq
  {{{{\Gamma}{A}}{P}}{\ctxwk{P}{\mfam{A}}}}
  {\jcomp{}{\bar{\typefont{pr}}}{Q}}
  {\ctxwk{P}{Q}}
\end{equation*}
for any family $Q$ of contexts over $\ctxext{{\Gamma}{A}}{\mfam{A}}$ 
\end{lem}

\subsection{Inductive morphisms}
\begin{defn}
Let $\jhom{\Gamma}{A}{B}{f}$ be a context morphism. We say that $f$ is an 
\emph{inductive morphism} if the following inference rules are valid:
\begin{align*}
& \inference
  { \jfam{{\Gamma}{B}}{Q}
    \jterm{{\Gamma}{A}}{\jcomp{A}{f}{Q}}{g}
    }
  { \jterm{{\Gamma}{B}}{Q}{\jtcext{g}}
    }
& & \inference
  { \jfam{{\Gamma}{B}}{Q}
    \jterm{{\Gamma}{A}}{\jcomp{A}{f}{Q}}{g}
    }
  { \jtermeq{{\Gamma}{A}}{\jcomp{A}{f}{Q}}{\jcomp{A}{f}{\jtcext{g}}}{g}
    }
\end{align*}
\end{defn}

When $f$ is a context morphism from $A$ to $B$ in context $\Gamma$, 
finding a term of a family $Q$ over $\ctxext{\Gamma}{B}$ can be accomplished
by finding a term of the family $\jcomp{A}{f}{Q}$ over $\ctxext{\Gamma}{A}$.
Inductive morphisms come with a computation rule. Thus, term
constructors of inductively defined types are going to be the major source of
examples of inductive morphisms. Since identity terms behave
so nicely, they are inductive morphisms too.

\begin{lem}
The identity term
$\jhom{\Gamma}{A}{A}{\idtm{A}}$ is an inductive morphism
for each family $A$ of contexts over $\Gamma$
\end{lem}

\begin{itemize}
\item Extensions of inductive morphisms are inductive
\item Weakenings of inductive morphisms are inductive
\item Substitutions of inductive morphisms are inductive
\end{itemize}

\subsection{Extension algebras}
In this subsection our goal is to define the notion of extension algebras,
which are internal versions of the extension operation of the theory of
contexts, families and terms. In this article, their use will be mainly in
universes. The theory of extension algebras requires the full power (i.e.~all
of the ingredients) of the theory of contexts, families and terms in its
formulation.

Let $P$ be a family over an extended context $\ctxext{\Gamma}{A}$. We could
mimic extension by requiring to have terms
\begin{align*}
\jhom*{\Gamma}{{A}{P}}{A}{e}\\
\jhom*{{\Gamma}{A}}{{P}{\jcomp{}{e}{P}}}{P}{f}.
\end{align*}
Extension also satisfies the properties explained in \autoref{comp-ee}, so we
must find the two judgmental equalities for $e$ and $f$ mimicing those. 
The first of these judgmental equalities is easy to give: it says that the
following diagram commutes:
\begin{equation}\label{eq:ealg-eq1}
\begin{tikzcd}[column sep=huge]
\ctxext{A}{{P}{\jcomp{}{e}{P}}} 
  \ar{d}[swap]{\jvcomp{}{e}{\idtm{\jcomp{}{e}{P}}}
    %\tmext{\ctxwk{\jcomp{}{e}{P}}{e}}{\idtm{\jcomp{}{e}{P}}}
    } 
  \ar{r}{\jvcomp{}{\idtm{A}}{f}
    %\tmext{\ctxwk{\ctxext{P}{\jcomp{}{e}{P}}}{\idtm{A}}}{f}
    } 
  & \ctxext{A}{P} \ar{d}{e}\\
\ctxext{A}{P} \ar{r}[swap]{e} & A
\end{tikzcd}
\end{equation}
To get a feel for this judgmental equality we include the following lemma.

\begin{lem}
Let $A$, $P$, $e$ and $f$ be as above, satisfying \autoref{eq:ealg-eq1} and
let $x_0:A$,
$x_1:\subst{x_0}{P}$ and $x_2:\subst{x_1}{{x_0}{\jcomp{}{e}{P}}}$.
Then we have the judgmental equality
\begin{equation*}
\subst{{x_2}{{x_1}{f}}}{{x_0}{e}}
\jdeq
\subst{x_2}{{{x_1}{{x_0}{e}}}{e}}.
\end{equation*}
\end{lem}

\begin{proof}
The proof is a simple computation:
\begin{align*}
\subst{{x_2}{{x_1}{f}}}{{x_0}{e}}
& \jdeq
  \subst{\tmext{x_0}{\subst{x_2}{{x_1}{f}}}}{e}
  \\
& \jdeq 
  \subst{{x_2}{{x_1}{{x_0}{\jvcomp{}{\idtm{A}}{f}}}}}{e}
  \\
& \jdeq
  \subst{x_2}{{x_1}{{x_0}{\jcomp{}{\jvcomp{}{\idtm{A}}{f}}{e}}}}
  \\
& \jdeq
  \subst{x_2}{{x_1}{{x_0}{\jcomp{}{\jvcomp{}{e}{\idtm{\jcomp{}{e}{P}}}}{e}}}}
  \\
& \jdeq 
  \subst{{x_2}{{x_1}{{x_0}{\jvcomp{}{e}{\idtm{\jcomp{}{e}{P}}}}}}}{e}
  \\
& \jdeq
  \subst{\tmext{\subst{x_1}{{x_0}{e}}}{x_2}}{e}
  \\
& \jdeq
  \subst{x_2}{{{x_1}{{x_0}{e}}}{e}}.\qedhere
\end{align*}
\end{proof}

The second of the judgmental equalities is harder to describe, however. We need
to consider `higher' families, i.e.~families over families over families, and
thus we need to look at the family $\jcomp{}{e}{\jcomp{}{e}{P}}$ and find the
two dotted morphisms in the diagram
\begin{equation*}
\begin{tikzcd}
\ctxext{P}{{\jcomp{}{e}{P}}{\jcomp{}{e}{\jcomp{}{e}{P}}}}
  \ar[densely dotted]{d}
  \ar[densely dotted]{r}
& \ctxext{P}{\jcomp{}{e}{P}} \ar{d}{f}\\
\ctxext{P}{\jcomp{}{e}{P}} \ar{r}{f} & P
\end{tikzcd}
\end{equation*}
The first is easy to find. Note that we have the judgmental equality
\begin{equation*}
\ctxext{\jcomp{}{e}{P}}{\jcomp{}{e}{\jcomp{}{e}{P}}}
  \jdeq
  \jcomp{}{e}{\ctxext{P}{\jcomp{}{e}{P}}}
\end{equation*}
and therefore we may just take the morphism
\begin{equation*}
\jhom
  {{\Gamma}{A}}
  {{P}{{\jcomp{}{e}{P}}{\jcomp{}{e}{\jcomp{}{e}{P}}}}}
  {{P}{\jcomp{}{e}{P}}}
  {\jvcomp{}{\idtm{P}}{\jcomp{}{e}{f}}}.
\end{equation*}
For the other morphism we need to look at the family
$\ctxext{P}{{\jcomp{}{e}{P}}{\jcomp{}{e}{\jcomp{}{e}{P}}}}$ differently. We do
that in the following lemma.

\begin{lem}
Suppose we have $A$, $P$, $e$ and $f$ satisfying \autoref{eq:ealg-eq1}. Then
we have the judgmental equality
\begin{equation*}
\jcomp{}{e}{\jcomp{}{e}{P}}
  \jdeq
  \jcomp{}{f}{\jcomp{}{e}{P}}.
\end{equation*}
\end{lem}

\begin{proof}
The proof is a rather long computation:
\begin{align*}
\jcomp{}{e}{\jcomp{}{e}{P}}
& \jdeq
  \subst{e}
    {\ctxwk{\ctxext{A}{P}}{\jcomp{}{e}{P}}}
  \\
& \jdeq
  \subst
    {\idtm{\jcomp{}{e}{P}}}
    { \ctxwk
        {\jcomp{}{e}{P}}{\subst{e}
        {\ctxwk{\ctxext{A}{P}}{\jcomp{}{e}{P}}}}}
  \\
& \jdeq
  \subst
    {\idtm{\jcomp{}{e}{P}}}
    { {\ctxwk{\jcomp{}{e}{P}}{e}}
      {\ctxwk{\jcomp{}{e}{P}}{{\ctxext{A}{P}}{\jcomp{}{e}{P}}}}}
  \\
& \jdeq
  \subst
    {\idtm{\jcomp{}{e}{P}}}
    { {\ctxwk{\jcomp{}{e}{P}}{e}}
      {\ctxwk{\ctxext{{A}{P}}{\jcomp{}{e}{P}}}{\jcomp{}{e}{P}}}}
  \\
& \jdeq
  \subst
    {\idtm{\jcomp{}{e}{P}}}
    { {\ctxwk{\jcomp{}{e}{P}}{e}}
      {\ctxwk{\ctxext{A}{{P}{\jcomp{}{e}{P}}}}{\jcomp{}{e}{P}}}}
  \\
& \jdeq
  \subst
    {\unfold{\jvcomp{\jcomp{}{e}{P}}{e}{\idtm{\jcomp{}{e}{P}}}}}
    {\ctxwk{\ctxext{A}{{P}{\jcomp{}{e}{P}}}}{\jcomp{}{e}{P}}}
  \\
& \jdeq
  \subst
    {\jvcomp{}{e}{\idtm{\jcomp{}{e}{P}}}}
    {\ctxwk{\ctxext{A}{{P}{\jcomp{}{e}{P}}}}{\jcomp{}{e}{P}}}
  \\
& \jdeq
  \jcomp{}{\jvcomp{}{e}{\idtm{\jcomp{}{e}{P}}}}{\jcomp{}{e}{P}}
  \\
& \jdeq 
  \jcomp{}{\jcomp{}{\jvcomp{}{e}{\idtm{\jcomp{}{e}{P}}}}{e}}{P}
  \\
& \jdeq
  \jcomp{}{
    \jcomp{}{\jvcomp{}{\idtm{A}}{f}}{e}}{P}
  \\
& \jdeq
  \jcomp{}{\jvcomp{}{\idtm{A}}{f}}{
    \jcomp{}{e}{P}}
  \\
& \jdeq
  \subst
    {\tmext{\ctxwk{\ctxext{P}{\jcomp{}{e}{P}}}{\idtm{A}}}{f}}
    {\ctxwk{\ctxext{A}{{P}{\jcomp{}{e}{P}}}}{\jcomp{}{e}{P}}}
  \\
& \jdeq
  \subst
    {f}
    { {\ctxwk{\ctxext{P}{\jcomp{}{e}{P}}}{\idtm{A}}}
      {\ctxwk{\ctxext{A}{{P}{\jcomp{}{e}{P}}}}{\jcomp{}{e}{P}}}
      }
  \\
& \jdeq
  \subst
    {f}
    { {\cprojfstf{A}{\ctxext{P}{\jcomp{}{e}{P}}}}
      {\ctxwk{\ctxext{A}{{P}{\jcomp{}{e}{P}}}}{\jcomp{}{e}{P}}}
      }
  \\
& \jdeq
  \subst
    {f}
    {\jcomp{}{\cprojfstf{A}{\ctxext{P}{\jcomp{}{e}{P}}}}{\jcomp{}{e}{P}}}
  \\
& \jdeq
  \subst
    {f}
    {\ctxwk{\ctxext{P}{\jcomp{}{e}{P}}}{\jcomp{}{e}{P}}}
  \\
& \jdeq
  \jcomp{}{f}{\jcomp{}{e}{P}}.
  \qedhere
\end{align*}
\end{proof}

Now we see that we can use the morphism
\begin{equation*}
\jhom
  {{\Gamma}{A}}
  {{P}{{\jcomp{}{e}{P}}{\jcomp{}{f}{{}{e}{P}}}}}
  {{P}{\jcomp{}{e}{P}}}
  {\jvcomp{}{f}{\idtm{\jcomp{}{f}{{}{e}{P}}}}}.
\end{equation*}
Thus, the second judgmental equality we will need is that the diagram
\begin{equation}\label{eq:ealg-eq2}
\begin{tikzcd}[column sep=huge]
\ctxext{P}{{\jcomp{}{e}{P}}{\jcomp{}{e}{{}{e}{P}}}} 
  \ar{r}{\jvcomp{}{\idtm{P}}{\jcomp{}{e}{f}}}
  \ar{d}[swap]{
    \jvcomp{}{f}{\idtm{\jcomp{}{f}{{}{e}{P}}}}
%    \tmext
%      {\ctxwk{\jcomp{}{\bar{e}}{{}{e}{P}}}{f}}
%      {\idtm{\jcomp{}{\bar{e}}{{}{e}{P}}}}
    }
& \ctxext{P}{\jcomp{}{e}{P}} \ar{d}{f}\\
\ctxext{P}{\jcomp{}{e}{P}} \ar{r}[swap]{f} & P
\end{tikzcd}
\end{equation}
commutes judgmentally. Now we can confidently formulate the definition of
extension algebras.

\begin{defn}
An \emph{extension algebra in context $\Gamma$} is a quadruple $(A,P,e,f)$
consisting of a family $A$ over context $\Gamma$, a family $P$ over the context
$\ctxext{\Gamma}{A}$, a morphism $e$ from $\ctxext{A}{P}$ to $A$ in context
$\Gamma$ and a morphism $f$ from $\ctxext{P}{\jcomp{}{e}{P}}$ to $P$ in context
$\ctxext{\Gamma}{A}$, satisfying the judgmental equalities of
\autoref{eq:ealg-eq1,eq:ealg-eq2}.
\end{defn}

We also give a bit of intuition to the requirement of \autoref{eq:ealg-eq2} by
means of the following lemma.

\begin{lem}
Let $(A,P,e,f)$ be an extension algebra in context $\Gamma$ and let
$y_0:P$, $y_1:\subst{y_0}{\jcomp{}{e}{P}}$ and 
$y_2:\subst{y_1}{{y_0}{\jcomp{}{e}{\jcomp{}{e}{P}}}}$. Then we have the
judgmental equality
\begin{equation*}
\subst{{y_2}{{y_1}{\jcomp{}{e}{f}}}}{{y_0}{f}}
  \jdeq
  \subst{y_2}{{{y_1}{{y_0}{f}}}{f}}
\end{equation*}
\end{lem}

\subsection{Extension-empty algebras}
\begin{defn}
Let $P$ be a family over the extended context $\ctxext{\Gamma}{A}$, let
$\jterm{\Gamma}{A}{x}$ and $\jterm{{\Gamma}{A}}{P}{f}$ be terms. Then the
quadruple $(A,P,x,f)$ is said to be an \emph{empty algebra in context $\Gamma$}
if the following judgmental equalities hold:
\begin{align*}
\jfameq*{\Gamma}{\subst{x}{P}}{A}\\
\jtermeq*{\Gamma}{A}{\subst{x}{f}}{x}.
\end{align*}
\end{defn}


%\part{Internal Models}\label{part:models}

%\section{Internalizing the theory of contexts families and terms}
One of the guiding ideas behind the design of the theory of contexts, families
and terms was that it would have to be possible to consider internal versions
of the theory. In this section we aim for this internalization.

It would be interesting to write out a weak version of pre-universes, internal
to Martin-L\"of type theory with the function extensionality principle. 
To do this, the empty context needs to
be replaced by a contractible type, extension by dependent pair types,
judgmental equalities of terms by identifications and judgmental equalities
of families by equivalences of types. We conjecture that it is possible to
carry this out (in particular to make sure that all the constructions 
type-check). This could serve as a starting point for investigating internal
models without truncatedness assumptions and for investigating internal higher
categories. 
Moreover, one could then extend the notion of `weak' pre-universes
with the requirement that every internal morphism is weakly anodyne. This
could give an internal theory of weak higher groupoids.

\subsection{Extension algebras}
In this subsection our goal is to define the notion of extension algebras,
which are internal versions of the extension operation of the theory of
contexts, families and terms. In this article, their use will be mainly in
universes. The theory of extension algebras requires the full power (i.e.~all
of the ingredients) of the theory of contexts, families and terms in its
formulation and it is (perhaps surprisingly) quite involved to formulate it.

Let $P$ be a family over an extended context $\ctxext{\Gamma}{A}$. We could
mimic extension by requiring to have terms
\begin{align*}
\jhom*{\Gamma}{{A}{P}}{A}{\epsilon_0}\\
\jhom*{{\Gamma}{A}}{{P}{\jcomp{}{\epsilon_0}{P}}}{P}{\epsilon_1}.
\end{align*}

\begin{rmk}
Instead of looking at $\epsilon_1$ as a context morphism from $\ctxext{P}{\jcomp{}{\epsilon_0}{P}}$
to $P$, one could also look at $\epsilon_1$ as a morphism \emph{over $\cprojfstf{A}{P}$},
as indicated in the following diagram:
\begin{equation*}
\begin{tikzcd}
P
  \ar[fib]{d}
& \jcomp{}{\epsilon_0}{P}
  \ar[fib]{d}
  \ar{l}[swap]{\epsilon_1}
  \ar{r}
& P
  \ar[fib]{d}
  \\
A
& \ctxext{A}{P}
  \ar{l}{\cprojfstf{A}{P}}
  \ar{r}[swap]{\epsilon_0}
& A
\end{tikzcd}
\end{equation*}
This makes it clear that $\epsilon_1$ takes a family over an extended context as an
argument. The extended context consists of a `base part' and a `family part'. 
The output of $\epsilon_1$ is a new (extended) family over that base part. Forgetting 
the family part is what the projection takes care of.
\end{rmk}

Extension also satisfies the properties explained in \autoref{comp-ee}, so we
must find the two judgmental equalities for $\epsilon_0$ and $\epsilon_1$ mimicing those. 
The first of these judgmental equalities is easy to give: it says that the
following diagram commutes:
\begin{equation}\label{eq:extalg-eq1}
\begin{tikzcd}[column sep=huge]
\ctxext{A}{{P}{\jcomp{}{\epsilon_0}{P}}} 
  \ar{d}[swap]{\jvcomp{}{\epsilon_0}{\idtm{\jcomp{}{\epsilon_0}{P}}}
    } 
  \ar{r}{\jvcomp{}{\idtm{A}}{\epsilon_1}
    } 
  & \ctxext{A}{P} \ar{d}{\epsilon_0}\\
\ctxext{A}{P} \ar{r}[swap]{\epsilon_0} & A
\end{tikzcd}
\end{equation}
To get a feel for this judgmental equality we include the following lemma.

\begin{lem}
Let $A$, $P$, $\epsilon_0$ and $\epsilon_1$ be as above, satisfying \autoref{eq:extalg-eq1} and
let $x_0:A$,
$x_1:\subst{x_0}{P}$ and $x_2:\subst{x_1}{{x_0}{\jcomp{}{\epsilon_0}{P}}}$.
Then we have the judgmental equality
\begin{equation*}
\subst{{x_2}{{x_1}{\epsilon_1}}}{{x_0}{\epsilon_0}}
\jdeq
\subst{x_2}{{{x_1}{{x_0}{\epsilon_0}}}{\epsilon_0}}.
\end{equation*}
\end{lem}

\begin{proof}
The proof is a simple computation:
\begin{align*}
\subst{{x_2}{{x_1}{\epsilon_1}}}{{x_0}{\epsilon_0}}
& \jdeq
  \subst{\tmext{x_0}{\subst{x_2}{{x_1}{\epsilon_1}}}}{\epsilon_0}
  \\
& \jdeq 
  \subst{{x_2}{{x_1}{{x_0}{\jvcomp{}{\idtm{A}}{\epsilon_1}}}}}{\epsilon_0}
  \\
& \jdeq
  \subst{x_2}{{x_1}{{x_0}{\jcomp{}{\jvcomp{}{\idtm{A}}{\epsilon_1}}{\epsilon_0}}}}
  \\
& \jdeq
  \subst{x_2}{{x_1}{{x_0}{\jcomp{}{\jvcomp{}{\epsilon_0}{\idtm{\jcomp{}{\epsilon_0}{P}}}}{\epsilon_0}}}}
  \\
& \jdeq 
  \subst{{x_2}{{x_1}{{x_0}{\jvcomp{}{\epsilon_0}{\idtm{\jcomp{}{\epsilon_0}{P}}}}}}}{\epsilon_0}
  \\
& \jdeq
  \subst{\tmext{\subst{x_1}{{x_0}{\epsilon_0}}}{x_2}}{\epsilon_0}
  \\
& \jdeq
  \subst{x_2}{{{x_1}{{x_0}{\epsilon_0}}}{\epsilon_0}}.\qedhere
\end{align*}
\end{proof}

The second of the judgmental equalities is harder to describe, however. We need
to consider `higher' families, i.e.~families over families over families, and
thus we need to look at the family $\jcomp{}{\epsilon_0}{\jcomp{}{\epsilon_0}{P}}$ and find the
two dotted morphisms in the diagram
\begin{equation*}
\begin{tikzcd}
\ctxext{P}{{\jcomp{}{\epsilon_0}{P}}{\jcomp{}{\epsilon_0}{\jcomp{}{\epsilon_0}{P}}}}
  \ar[densely dotted]{d}
  \ar[densely dotted]{r}
& \ctxext{P}{\jcomp{}{\epsilon_0}{P}} \ar{d}{\epsilon_1}\\
\ctxext{P}{\jcomp{}{\epsilon_0}{P}} \ar{r}{\epsilon_1} & P
\end{tikzcd}
\end{equation*}
The first is easy to find. Note that we have the judgmental equality
\begin{equation*}
\ctxext{\jcomp{}{\epsilon_0}{P}}{\jcomp{}{\epsilon_0}{\jcomp{}{\epsilon_0}{P}}}
  \jdeq
  \jcomp{}{\epsilon_0}{\ctxext{P}{\jcomp{}{\epsilon_0}{P}}}
\end{equation*}
and therefore we may just take the morphism
\begin{equation*}
\jhom
  {{\Gamma}{A}}
  {{P}{{\jcomp{}{\epsilon_0}{P}}{\jcomp{}{\epsilon_0}{\jcomp{}{\epsilon_0}{P}}}}}
  {{P}{\jcomp{}{\epsilon_0}{P}}}
  {\jvcomp{}{\idtm{P}}{\jcomp{}{\epsilon_0}{\epsilon_1}}}.
\end{equation*}
For the other morphism we need to look at the family
$\ctxext{P}{{\jcomp{}{\epsilon_0}{P}}{\jcomp{}{\epsilon_0}{\jcomp{}{\epsilon_0}{P}}}}$ differently. We do
that in the following lemma.

\begin{lem}\label{lem:extalg-twins}
Suppose we have $A$, $P$, $\epsilon_0$ and $\epsilon_1$ satisfying \autoref{eq:extalg-eq1}. Then
the inference rules
\begin{align*}
& \inference
  { \jfam{{\Gamma}{A}}{Q}
    }
  { \jfameq
      {{{{\Gamma}{A}}{P}}{\jcomp{}{\epsilon_0}{P}}}
      {\jcomp{}{\epsilon_0}{\jcomp{}{\epsilon_0}{Q}}}
      {\jcomp{}{\epsilon_1}{\jcomp{}{\epsilon_0}{Q}}}
    }
  \\
& \inference
  { \jfam{{{\Gamma}{A}}{Q}}{R}
    }
  { \jfameq
      {{{{{\Gamma}{A}}{P}}{\jcomp{}{\epsilon_0}{P}}}
        {\jcomp{}{\epsilon_0}{\jcomp{}{\epsilon_0}{Q}}}}
      {\jcomp{}{\epsilon_0}{\jcomp{}{\epsilon_0}{R}}}
      {\jcomp{}{\epsilon_1}{\jcomp{}{\epsilon_0}{R}}}
    }
  \\
& \inference
  { \jterm{{{\Gamma}{A}}{Q}}{R}{h}
    }
  { \jtermeq
      {{{{{\Gamma}{A}}{P}}{\jcomp{}{\epsilon_0}{P}}}
        {\jcomp{}{\epsilon_0}{\jcomp{}{\epsilon_0}{Q}}}}
      {\jcomp{}{\epsilon_0}{\jcomp{}{\epsilon_0}{R}}}
      {\jcomp{}{\epsilon_0}{\jcomp{}{\epsilon_0}{h}}}
      {\jcomp{}{\epsilon_1}{\jcomp{}{\epsilon_0}{h}}}
    }
\end{align*}
are valid.
\end{lem}

\begin{proof}
We only prove the first inference rule. Let $\jfam{{\Gamma}{A}}{Q}$ be a family.
Then we have the judgmental equalities
\begin{align*}
\jcomp{}{\epsilon_0}{\jcomp{}{\epsilon_0}{Q}}
& \jdeq
  \jcomp{}{\idtm{\jcomp{}{\epsilon_0}{P}}}{\jcomp{}{\epsilon_0}{\jcomp{}{\epsilon_0}{Q}}}
  \tag{by \autoref{idfunc-precomp}}
  \\
& \jdeq
  \jcomp{}{\jvcomp{}{\epsilon_0}{\idtm{\jcomp{}{\epsilon_0}{P}}}}{\jcomp{}{\epsilon_0}{Q}}
  \tag{by \autoref{lem:composition-threesome}}
  \\
& \jdeq
  \jcomp{}{\jcomp{}{\jvcomp{}{\epsilon_0}{\idtm{\jcomp{}{\epsilon_0}{P}}}}{\epsilon_0}}{Q}
  \tag{by \autoref{lem:jcomp-jcomp}}
  \\
& \jdeq
  \jcomp{}{\jcomp{}{\jvcomp{}{\idtm{A}}{\epsilon_1}}{\epsilon_0}}{Q}
  \tag{by \autoref{eq:extalg-eq1}}
  \\
& \jdeq
  \jcomp{}{\jvcomp{}{\idtm{A}}{\epsilon_1}}{\jcomp{}{\epsilon_0}{Q}}
  \tag{by \autoref{lem:jcomp-jcomp}}
  \\
& \jdeq
  \jcomp{}{\epsilon_1}{\jcomp{}{\idtm{A}}{\jcomp{}{\epsilon_0}{Q}}}
  \tag{by \autoref{lem:composition-threesome}}
  \\
& \jdeq
  \jcomp{}{\epsilon_1}{\jcomp{}{\epsilon_0}{Q}}.
  \tag{by \autoref{idfunc-precomp}}
\end{align*}
\end{proof}

Now we see that we can use the morphism
\begin{equation*}
\jhom
  {{\Gamma}{A}}
  {{P}{{\jcomp{}{\epsilon_0}{P}}{\jcomp{}{\epsilon_1}{{}{\epsilon_0}{P}}}}}
  {{P}{\jcomp{}{\epsilon_0}{P}}}
  {\jvcomp{}{\epsilon_1}{\idtm{\jcomp{}{\epsilon_1}{{}{\epsilon_0}{P}}}}}.
\end{equation*}
Thus, the second judgmental equality we will need is that the diagram
\begin{equation}\label{eq:extalg-eq2}
\begin{tikzcd}[column sep=huge]
\ctxext{P}{{\jcomp{}{\epsilon_0}{P}}{\jcomp{}{\epsilon_0}{{}{\epsilon_0}{P}}}} 
  \ar{r}{\jvcomp{}{\idtm{P}}{\jcomp{}{\epsilon_0}{\epsilon_1}}}
  \ar{d}[swap]{
    \jvcomp{}{\epsilon_1}{\idtm{\jcomp{}{\epsilon_1}{{}{\epsilon_0}{P}}}}
    }
& \ctxext{P}{\jcomp{}{\epsilon_0}{P}} \ar{d}{\epsilon_1}\\
\ctxext{P}{\jcomp{}{\epsilon_0}{P}} \ar{r}[swap]{\epsilon_1} & P
\end{tikzcd}
\end{equation}
commutes judgmentally. Now we can confidently formulate the definition of
extension algebras.

\begin{defn}
An \emph{extension algebra in context $\Gamma$} is a quadruple $(A,P,\epsilon_0,\epsilon_1)$
consisting of a family $A$ over context $\Gamma$, a family $P$ over the context
$\ctxext{\Gamma}{A}$, a morphism $\epsilon_0$ from $\ctxext{A}{P}$ to $A$ in context
$\Gamma$ and a morphism $\epsilon_1$ from $\ctxext{P}{\jcomp{}{\epsilon_0}{P}}$ to $P$ in context
$\ctxext{\Gamma}{A}$, satisfying the judgmental equalities of
\autoref{eq:extalg-eq1,eq:extalg-eq2}.
\end{defn}

We also give a bit of intuition to the requirement of \autoref{eq:extalg-eq2} by
means of the following lemma.

\begin{lem}
Let $(A,P,\epsilon_0,\epsilon_1)$ be an extension algebra in context $\Gamma$ and let
$y_0:P$, $y_1:\subst{y_0}{\jcomp{}{\epsilon_0}{P}}$ and 
$y_2:\subst{y_1}{{y_0}{\jcomp{}{\epsilon_0}{\jcomp{}{\epsilon_0}{P}}}}$. Then we have the
judgmental equality
\begin{equation*}
\subst{{y_2}{{y_1}{\jcomp{}{\epsilon_0}{\epsilon_1}}}}{{y_0}{\epsilon_1}}
  \jdeq
  \subst{y_2}{{{y_1}{{y_0}{\epsilon_1}}}{\epsilon_1}}
\end{equation*}
\end{lem}

\begin{proof}
The proof is a straightforward calculation:
\begin{align*}
\subst{{y_2}{{y_1}{\jcomp{}{\epsilon_0}{\epsilon_1}}}}{{y_0}{\epsilon_1}}
& \jdeq
  \subst{\tmext{y_0}{\subst{y_2}{{y_1}{\jcomp{}{\epsilon_0}{\epsilon_1}}}}}{\epsilon_1}
  \\
& \jdeq
  \subst{{y_2}{{y_1}{{y_0}{\jvcomp{}{\idtm{P}}{\jcomp{}{\epsilon_0}{\epsilon_1}}}}}}{\epsilon_1}
  \\
& \jdeq
  \subst
    {y_2}
    { {y_1}
      { {y_0}
        {\jcomp{}{\jvcomp{}{\idtm{P}}{\jcomp{}{\epsilon_0}{\epsilon_1}}}{\epsilon_1}}
        }
      }
  \\
& \jdeq
  \subst
    {y_2}
    { {y_1}
      { {y_0}
        {\jcomp{}{\jvcomp{}{\epsilon_1}{\idtm{\jcomp{}{\epsilon_1}{{}{\epsilon_0}{P}}}}}{\epsilon_1}}
        }
      }
  \\
& \jdeq
  \subst{{y_2}{{y_1}{{y_0}{\jvcomp{}{\epsilon_1}{\idtm{\jcomp{}{\epsilon_1}{{}{\epsilon_0}{P}}}}}}}}{\epsilon_1}
  \\
& \jdeq
  \subst{\tmext{\subst{y_1}{{y_0}{\epsilon_1}}}{y_2}}{\epsilon_1}
  \\
& \jdeq
  \subst{y_2}{{{y_1}{{y_0}{\epsilon_1}}}{\epsilon_1}}.\qedhere
\end{align*}
\end{proof}

There is a trivial class of examples of extension algebras we can give right
away. More examples will be introduced by universes, later on.

\begin{eg}
Let $A$ be a family in context $\Gamma$. Then the quadruple
\begin{equation*}
(A,\emptyf,\idtm{A},\emptytm)
\end{equation*}
is an extension algebra in context $\Gamma$, as is the quadruple
\begin{equation*}
(\emptyf,A,\emptytm,\idtm{A}).
\end{equation*}
Also, the quadruple
\begin{equation*}
(A,\ctxwk{A}{A},\cprojfstf{A}{\ctxwk{A}{A}},\cprojfstf{\ctxwk{A}}{\ctxwk{A}{{A}{A}}})
\end{equation*}
is an extension algebra in context $\Gamma$.
\end{eg}

The following lemma explains how each extension algebra gives rise to infinitely
many extension algebras.

\begin{thm}
Suppose that $(A,P,\epsilon_0,\epsilon_1)$ is an extension algebra in context
$\Gamma$. Then 
$(P,\jcomp{}{\epsilon_0}{P},\epsilon_1,\jcomp{}{\epsilon_0}{\epsilon_1})$
is an extension algebra in context $\ctxext{\Gamma}{A}$.
\end{thm}

\begin{proof}
We first need to verify that $\jcomp{}{\epsilon_0}{\epsilon_1}$ is a morphism
from $\ctxext{\jcomp{}{\epsilon_0}{P}}{\jcomp{}{\epsilon_1}{{}{\epsilon_0}{P}}}$
to $\jcomp{}{\epsilon_0}{P}$. This follows from the judgmental equality
$\jcomp{}{\epsilon_1}{{}{\epsilon_0}{P}}\jdeq
\jcomp{}{\epsilon_0}{{}{\epsilon_0}{P}}$, which we have proved in
\autoref{lem:extalg-twins}. Notice how the diagram in \autoref{eq:extalg-eq2} is
of exactly the right sort, so the quadruple
$(P,\jcomp{}{\epsilon_0}{P},\epsilon_1,\jcomp{}{\epsilon_0}{\epsilon_1})$
satisfies its version of \autoref{eq:extalg-eq1}. It is left to verify that the diagram
\begin{small}
\begin{equation*}
\begin{tikzcd}[column sep=huge]
\ctxext
  { \jcomp{}{\epsilon_0}{P}
    }
  { { \jcomp{}{\epsilon_1}{%
        \jcomp{}{\epsilon_0}{P}
        }
      }
    { \jcomp{}{\epsilon_1}{%
        \jcomp{}{\epsilon_1}{%
          \jcomp{}{\epsilon_0}{P}
          }
        }
      }
    } 
  \ar{r}
    { \jvcomp{}{\idtm{\jcomp{}{\epsilon_0}{P}}}{%
        \jcomp{}{\epsilon_1}{%
          \jcomp{}{\epsilon_0}{\epsilon_1}}}}
  \ar{d}[swap]{
    \jvcomp{}{\jcomp{}{\epsilon_0}{\epsilon_1}}{%
      \idtm{
        \jcomp{}{\jcomp{}{\epsilon_0}{\epsilon_1}}{%
          \jcomp{}{\epsilon_1}{%
            \jcomp{}{\epsilon_0}{P}
            }
          }
        }
      }
    }
& \ctxext
    {\jcomp{}{\epsilon_0}{P}}
    {\jcomp{}{\epsilon_1}{\jcomp{}{\epsilon_0}{P}}} 
  \ar{d}{\jcomp{}{\epsilon_0}{\epsilon_1}}
  \\
\ctxext
  {\jcomp{}{\epsilon_0}{P}}
  {\jcomp{}{\epsilon_1}{\jcomp{}{\epsilon_0}{P}}} 
  \ar{r}[swap]{\jcomp{}{\epsilon_0}{\epsilon_1}} 
& \jcomp{}{\epsilon_0}{P}
\end{tikzcd}
\end{equation*}
\end{small}
commutes judgmentally; this diagram is the version of \autoref{eq:extalg-eq2}
for the quadruple
$(P,\jcomp{}{\epsilon_0}{P},\epsilon_1,\jcomp{}{\epsilon_0}{\epsilon_1})$. Note
that this follows from \autoref{eq:extalg-eq2} provided that we can show that
\begin{align}
\jvcomp{}{\idtm{\jcomp{}{\epsilon_0}{P}}}{%
  \jcomp{}{\epsilon_1}{%
    \jcomp{}{\epsilon_0}{\epsilon_1}}}
& \jdeq
  \jcomp{}{\epsilon_0}{%
    \jvcomp{}{\idtm{P}}{\jcomp{}{\epsilon_0}{\epsilon_1}}
    }
  \label{eq:extalg-infty1}
  \\
\jvcomp{}{\jcomp{}{\epsilon_0}{\epsilon_1}}{%
  \idtm{
    \jcomp{}{\jcomp{}{\epsilon_0}{\epsilon_1}}{%
      \jcomp{}{\epsilon_1}{%
        \jcomp{}{\epsilon_0}{P}
        }
      }
    }
  }
& \jdeq
\jcomp{}{\epsilon_0}{%
  \jvcomp{}{\epsilon_1}{%
    \idtm{
      \jcomp{}{\epsilon_1}{%
        \jcomp{}{\epsilon_0}{P}
        }
      }
    }
  }
  \label{eq:extalg-infty2}
\end{align}
Note that \autoref{eq:extalg-infty1} follows if we can show that
\begin{equation*}
\jcomp{}{\epsilon_1}{\jcomp{}{\epsilon_0}{\epsilon_1}}
  \jdeq
  \jcomp{}{\epsilon_0}{\jcomp{}{\epsilon_0}{\epsilon_1}}.
\end{equation*}
This is a special case of \autoref{lem:extalg-twins}. The second judgmental
equality, \autoref{eq:extalg-infty2}, is trivial.
\end{proof}

\begin{thm}
Let $(Q,R,\eta_0,\eta_1)$ be an extension algebra in context $\ctxext{\Gamma}{B}$ and let
$\jfam{\Gamma}{A}$. Then the quadruple
\begin{equation*}
(\ctxwk{A}{Q},\ctxwk{A}{R},\ctxwk{A}{\eta_0},\ctxwk{A}{\eta_1})
\end{equation*}
is an extension algebra in context $\ctxext{{\Gamma}{A}}{\ctxwk{A}{B}}$.
\end{thm}

\begin{proof}
The proof follows from the fact that weakening by $A$ is compatible with all
the involved operations.
\end{proof}

\begin{thm}
Let $(Q,R,\eta_0,\eta_1)$ be an extension algebra in context $\ctxext{{\Gamma}{A}}{P}$
and let $\jterm{\Gamma}{A}{x}$. Then the quadruple
\begin{equation*}
(\subst{x}{Q},\subst{x}{R},\subst{x}{\eta_0},\subst{x}{\eta_1})
\end{equation*}
is an extension algebra in context $\ctxext{\Gamma}{\subst{x}{P}}$.
\end{thm}

\begin{proof}
The proof follows from the fact that substitution with $x$ is compatible with
all the involved operations.
\end{proof}

\begin{cor}
Let $(Q,R,\eta_0,\eta_1)$ be an extension algebra in context $\ctxext{\Gamma}{B}$
and let $\jhom{\Gamma}{A}{B}{f}$. Then the quadruple
\begin{equation*}
(\jcomp{A}{f}{Q},\jcomp{A}{f}{R},\jcomp{A}{f}{\eta_0},\jcomp{A}{f}{\eta_1})
\end{equation*}
is an extension algebra in context $\ctxext{\Gamma}{A}$.
\end{cor}

\begin{defn}
An \emph{extension homomorphism} from $(A,P,\epsilon_0,\epsilon_1)$ to
$(B,Q,\eta_0,\eta_1)$ in context $\Gamma$ is a pair $(f_0,f_1)$ consisting of
\begin{align*}
\jhom*{\Gamma}{A}{B}{f_0}
  \\
\jfhom*{\Gamma}{A}{B}{f_0}{P}{Q}{f_1}
\end{align*}
for which the diagrams
\begin{equation}\label{eq:extalg-hom1}
\begin{tikzcd}
\ctxext{A}{P}
  \ar{r}{\jvcomp{}{f_0}{f_1}}
  \ar{d}[swap]{\epsilon_0}
& \ctxext{B}{Q}
  \ar{d}{\eta_0}
  \\
A
  \ar{r}[swap]{f_0}
& B
\end{tikzcd}
\end{equation}
and
\begin{equation}\label{eq:extalg-hom2}
\begin{tikzcd}[column sep=large]
\ctxext{P}{\jcomp{}{\epsilon_0}{P}}
  \ar{r}{\jvcomp{}{f_1}{\jcomp{}{\epsilon_0}{f_1}}}
  \ar{d}[swap]{\epsilon_1}
& \jcomp{}{f_0}{\ctxext{Q}{\jcomp{}{\eta_0}{Q}}}
  \ar{d}{\jcomp{}{f_0}{\eta_1}}
  \\
P
  \ar{r}[swap]{f_1}
& \jcomp{}{f_0}{Q}
\end{tikzcd}
\end{equation}
commute judgmentally.
\end{defn}

\begin{rmk}
To see that the upper morphism in the diagram of \autoref{eq:extalg-hom2} has
indeed the indicated codomain provided that the diagram of \autoref{eq:extalg-hom1}
commutes judgmentally, note that we have the judgmental equalities
\begin{align*}
\jcomp{}{f_1}{\jcomp{}{f_0}{\jcomp{}{\eta_0}{Q}}}
& \jdeq 
  \jcomp{}{\jvcomp{}{f_0}{f_1}}{\jcomp{}{\eta_0}{Q}}
  \tag{by \autoref{lem:composition-threesome}}
  \\
& \jdeq
  \jcomp{}{\jcomp{}{\jvcomp{}{f_0}{f_1}}{\eta_0}}{Q}
  \tag{by \autoref{lem:jcomp-jcomp}}
  \\
& \jdeq
  \jcomp{}{\jcomp{}{\epsilon_0}{f_0}}{Q}
  \tag{by \autoref{eq:extalg-hom1}}
  \\
& \jdeq
  \jcomp{}{\epsilon_0}{\jcomp{}{f_0}{Q}}.
  \tag{by \autoref{lem:jcomp-jcomp}}
\end{align*}
and we indeed have the morphism $\jcomp{}{\epsilon_0}{f_1}$ from 
$\jcomp{}{\epsilon_0}{P}$ to $\jcomp{}{\epsilon_0}{\jcomp{}{f_0}{Q}}$.
\end{rmk}

\begin{rmk}
I suspect that if we copy this theory of extension algebras to Martin-L\"of
type theory, with the judgmental equalities replaced by identifications, with
dependent pair types rather than those strict extensions, etcetera, then
the type of $f_1$ for which these two diagrams commute is a mere proposition.

With this notion of morphism, a term of an extension
algebra $(A,P,\epsilon_0,\epsilon_1)$ is a pair $(x_0,x_1)$ such that
$\subst{x_1}{{x_0}{\epsilon_0}}\jdeq x_0$.
\end{rmk}

\begin{comment}
Extension algebras don't come in isolation. There are also extension algebra
families and extension algebra terms. We now aim to define these and to
establish various constructions of new extension algebras out of old ones:
the empty extension algebra, and extensions, weakenings and substitutions
of extension algebras and of course the identity term as an extension algebra
term. We start with extension algebra families.

\begin{defn}
Consider an extension algebra $\mathcal{A}\defeq(A,P,e,f)$. 
An extension algebra family over $\mathcal{A}$ is likewise a quadruple
$\mathcal{B}\defeq(B,Q,g,h)$. Here we have a family $\jfam{{\Gamma}{A}}{B}$, a
family $\jfam{{{{\Gamma}{A}}{P}}{\ctxwk{P}{B}}}{Q}$ and
\begin{align*}
\jhom*{{{\Gamma}{A}}{P}}{\ctxext{\ctxwk{P}{B}}{Q}}{\jcomp{}{\epsilon_0}{B}}{g}\\
\jhom*{{{{\Gamma}{A}}{P}}{\ctxwk{P}{B}}}{\ctxext{Q}{\jcomp{}{g}{Q}}}{Q}{h}.
\end{align*}
The quadruple $(\jcomp{}{\epsilon_0}{B},Q,g,h)$ is required to be an extension algebra
in context $\ctxext{{\Gamma}{A}}{P}$.
\end{defn}

\begin{defn}
Suppose $\mathcal{A}$ is an extension algebra and $\mathcal{B}$ is an extension
algebra family over $\mathcal{A}$. A term of $\mathcal{B}$ is a pair $(x,y)$
consisting of
\begin{align*}
\jterm*{{\Gamma}{A}}{B}{x}\\
\jterm*{{{\Gamma}{A}}{P}}{\subst{\jcomp{}{\epsilon_0}{x}}{Q}}{y}
\end{align*}
such that the diagrams
\begin{equation*}
\begin{tikzcd}
\ctxext{\jcomp{}{\epsilon_0}{B}}{Q} 
  \ar{r}{g} 
  \ar[shift right=.7ex,fib]{d}
& B 
  \ar[shift right=.7ex,fib]{d} 
  \\
\ctxext{A}{P} 
  \ar[shift right=.7ex,dotted]{u}[swap]{\tmext{\jcomp{}{\epsilon_0}{x}}{y}}
  \ar{r}{e}
& A
  \ar[shift right=.7ex,dotted]{u}[swap]{x}
\end{tikzcd}
\end{equation*}
and
\begin{equation*}
\begin{tikzcd}
\jcomp{}{f}{\ctxext{Q}{\jcomp{}{g}{Q}}}
  \ar{r}{\jcomp{}{f}{h}}
  \ar[shift right=.7ex,fib]{d}
& Q
  \ar[shift right=.7ex,fib]{d}
  \\
\jcomp{}{f}{\jcomp{}{\epsilon_0}{B}}
  \ar{r}{\idtm{\jcomp{}{f}{\jcomp{}{\epsilon_0}{B}}}}
  \ar[shift right=.7ex,fib]{d}
  \ar[shift right=.7ex,dotted,mapsto]{u}[swap]{\jcomp{}{f}{y}}
& \jcomp{}{\epsilon_0}{B}
  \ar[shift right=.7ex,fib]{d}
  \ar[shift right=.7ex,dotted,mapsto]{u}[swap]{y}
  \\
\ctxext{P}{\jcomp{}{\epsilon_0}{P}}
  \ar{r}[swap]{f}
  \ar[shift right=.7ex,dotted]{u}[swap]{\jcomp{}{f}{\jcomp{}{\epsilon_0}{x}}}
& P
  \ar[shift right=.7ex,dotted]{u}[swap]{\jcomp{}{\epsilon_0}{x}}
\end{tikzcd}
\end{equation*}
commute.
\end{defn}

\begin{defn}
Suppose $\mathcal{A}$ and $\mathcal{B}$ are extension algebras in context
$\Gamma$. We define the extension algebra $\ctxwk{\mathcal{A}}{\mathcal{B}}$
to be the quadruple
\begin{equation*}
(\ctxwk{A}{B},\ctxwk{\ctxext{A}{P}}{Q},\ctxwk{\ctxext{A}{P}}{g},\ctxwk{\ctxext{A}{P}}{h}).
\end{equation*}
Note that $\ctxwk{\ctxext{A}{P}}{Q}$ is a family over $\ctxwk{\ctxext{A}{P}}{B}$,
whereas it should be a family over $\jcomp{}{\epsilon_0}{\ctxwk{A}{B}}$. These are the
same by \autoref{lem:prehom}.
\end{defn}

\begin{rmk}
Before we continue, let us explore what it means to be an extension algebra
term of the extension algebra $\ctxwk{\mathcal{A}}{\mathcal{B}}$. Such an
extension algebra term $(x,y)$ would consist of
\begin{align*}
\jterm*{{\Gamma}{A}}{\ctxwk{A}{B}}{x}\\
\jterm*{{{\Gamma}{A}}{P}}{\subst{\jcomp{}{\epsilon_0}{x}}{\ctxwk{\ctxext{A}{P}}{Q}}}{y}.
\end{align*}
Thus, $x$ is a context morphism from $A$ to $B$ and $y$ is nothing but a term
of $\jcomp{}{\jcomp{}{\epsilon_0}{x}}{Q}$. For $x$, we see that the diagram
\begin{equation*}
\begin{tikzcd}
\ctxext{B}{Q} 
  \ar{r}{g} 
& B 
  \\
\ctxext{A}{P} 
  \ar{u}{\jvcomp{}{x}{y}}
  \ar{r}{e}
& A
  \ar{u}[swap]{x}
\end{tikzcd}
\end{equation*}
commutes.
\end{rmk}
\end{comment}

\subsection{Extension-empty algebras}
\begin{defn}
Let $P$ be a family over the extended context $\ctxext{\Gamma}{A}$, let
$\jterm{\Gamma}{A}{\phi_0}$ and $\jterm{{\Gamma}{A}}{P}{\phi_1}$ be terms. Then the
quadruple $(A,P,\phi_0,\phi_1)$ is said to be an \emph{empty algebra in context $\Gamma$}
if the following judgmental equalities hold:
\begin{align}
\jfameq*{\Gamma}{\subst{\phi_0}{P}}{A}
  \label{empalg-eq1}
  \\
\jtermeq*{\Gamma}{A}{\subst{\phi_0}{\phi_1}}{\phi_0}.
  \label{empalg-eq2}
\end{align}
\end{defn}

\begin{defn}
Let $(A,P,\phi_0,\phi_1)$ and $(B,Q,\psi_0,\psi_1)$ be empty algebras in context
$\Gamma$. An empty homomorphism from $(A,P,\phi_0,\phi_1)$ to $(B,Q,\psi_0,\psi_1)$
is a pair $(f_0,f_1)$ consisting of
\begin{align*}
\jhom*{\Gamma}{A}{B}{f_0}\\
\jfhom*{\Gamma}{A}{B}{f}{P}{Q}{f_1}
\end{align*}
satisfying the judgmental equalities
\begin{align*}
\jtermeq*{\Gamma}{B}{\subst{\phi_0}{f_0}}{\psi_0}\\
\jhomeq*{\Gamma}{A}{B}{\subst{\phi_0}{f_1}}{f_0}\\
\jtermeq*{{\Gamma}{A}}{\jcomp{}{f_0}{Q}}{\subst{\phi_1}{f_1}}{\jcomp{}{f_0}{\psi_1}}.
\end{align*}
\end{defn}

Thus, the empty algebras are the kind of algebras that require that families
are compatible with contexts, just as our motivation in \autoref{empty}. We
now combine the notion of extension algebras and empty algebras.

An extension-empty algebra in context $\Gamma$ is going to be a sextuple
$(A,P,\epsilon_0,\epsilon_1,\phi_0,\phi_1)$ for which 
the quadruple $(A,P,\epsilon_0,\epsilon_1)$ is an extension algebra in context 
$\Gamma$, the quadruple $(A,P,\phi_0,\phi_1)$ is an empty algebra in context
$\Gamma$, satisfying additional judgmental equalities expressing the 
compatibility of $\epsilon_0$ and $\epsilon_1$ with $\phi_0$ and $\phi_1$.
There will be four such judgmental equalities.

We can immediately state the first two:
\begin{align}
\jtermeq*{{\Gamma}{A}}{\ctxwk{A}{A}}{\subst{\phi_0}{\epsilon_0}}{\idtm{A}}
  \label{extempalg-eq1}
  \\
\jtermeq*{{\Gamma}{A}}{\ctxwk{A}{A}}{\subst{\phi_1}{\epsilon_0}}{\idtm{A}}
  \label{extempalg-eq2}
\end{align}
To see what $\subst{\phi_0}{\epsilon_1}$ can be, we must know its type first.
It is a morphism from $\subst{\phi_0}{\ctxext{P}{\jcomp{}{\epsilon_0}{P}}}$ to
$\subst{\phi_0}{P}$. We already know that $\subst{\phi_0}{P}\jdeq A$ by
\autoref{empalg-eq1} and to compute $\subst{\phi_0}{\jcomp{}{\epsilon_0}{P}}$
we use the following lemma.

\begin{lem}\label{lem:empalg-mor}
Consider an empty algebra $(A,P,\phi_0,\phi_1)$ in context $\Gamma$
and a morphism $\jhom{\Gamma}{{A}{P}}{B}{f}$.
Then $\subst{\phi_i}{f}$ is a morphism from $A$ to $B$ in context $\Gamma$ and
the following inference rules are valid for $i$ being $0$ or $1$:
\begin{align*}
& \inference
  { \jfam{{\Gamma}{A}}{Q}
    }
  { \jfameq
      {{\Gamma}{A}}
      {\subst{\phi_i}{\jcomp{}{f}{Q}}}
      {\jcomp{}{\subst{\phi_i}{f}}{Q}}
    }
  \\
& \inference
  { \jfam{{{\Gamma}{A}}{Q}}{R}
    }
  { \jfameq
      {{{\Gamma}{A}}{\jcomp{}{\subst{\phi_i}{f}}{Q}}}
      {\subst{\phi_i}{\jcomp{}{f}{R}}}
      {\jcomp{}{\subst{\phi_i}{f}}{R}}
    }
  \\
& \inference
  { \jterm{{{\Gamma}{A}}{Q}}{R}{h}
    }
  { \jtermeq
      {{{\Gamma}{A}}{\jcomp{}{\subst{\phi_i}{f}}{Q}}}
      {\jcomp{}{\subst{\phi_i}{f}}{R}}
      {\subst{\phi_i}{\jcomp{}{f}{h}}}
      {\jcomp{}{\subst{\phi_i}{f}}{h}}
    }
\end{align*}
\end{lem}

\begin{proof}
We only prove the first inference rule in both cases.
Let $Q$ be a family over $\ctxext{\Gamma}{A}$. In the case $i=0$
 we have the judgmental equalities
\begin{align*}
\subst{\phi_0}{\jcomp{}{f}{Q}}
& \jdeq
  \subst{\phi_0}{{f}{\ctxwk{\ctxext{A}{P}}{Q}}}
  \tag{by definition}
  \\
& \jdeq
  \subst{{\phi_0}{f}}{{\phi_0}{\ctxwk{\ctxext{A}{P}}{Q}}}
  \tag{by \autoref{comp-ss-f}}
  \\
& \jdeq
  \subst{{\phi_0}{f}}{{\phi_0}{\ctxwk{P}{{A}{Q}}}}
  \tag{by \autoref{comp-ew-f}}
  \\
& \jdeq
  \subst{{\phi_0}{f}}{\ctxwk{\subst{\phi_0}{P}}{\subst{\phi_0}{\ctxwk{A}{Q}}}}
  \tag{by \autoref{comp-sw-f}}
  \\
& \jdeq
  \subst{{\phi_0}{f}}{\ctxwk{\subst{\phi_0}{P}}{Q}}
  \tag{by \autoref{cancellation-ws-f}}
  \\
& \jdeq
  \subst{{\phi_0}{f}}{\ctxwk{A}{Q}}
  \tag{by \autoref{empalg-eq1}}
  \\
& \jdeq
  \jcomp{}{\subst{\phi_0}{f}}{Q}.
  \tag{by definition}
\end{align*}
In the case $i=1$ we have the judgmental equalities
\begin{align*}
\subst{\phi_1}{\jcomp{}{f}{Q}}
& \jdeq
  \subst{\phi_1}{{f}{\ctxwk{\ctxext{A}{P}}{Q}}}
  \tag{by definition}
  \\
& \jdeq
  \subst{{\phi_1}{f}}{{\phi_1}{\ctxwk{\ctxext{A}{P}}{Q}}}
  \tag{by \autoref{comp-ss-f}}
  \\
& \jdeq
  \subst{{\phi_1}{f}}{{\phi_1}{\ctxwk{P}{{A}{Q}}}}
  \tag{by \autoref{comp-ew-f}}
  \\
& \jdeq
  \subst{{\phi_1}{f}}{\ctxwk{A}{Q}}
  \tag{by \autoref{cancellation-ws-f}}
  \\
& \jdeq
  \jcomp{}{\subst{\phi_1}{f}}{Q}.
  \tag{by definition}
\end{align*}
\end{proof}

As an immediate corollary, if we assume the judgmental equalities
\autoref{extempalg-eq1,extempalg-eq2} we get that 
\begin{equation}\label{cor:empalg-mor}
\jfameq{{\Gamma}{A}}{\subst{\phi_i}{\jcomp{}{\epsilon_0}{P}}}{P}
\end{equation}
and hence that $\subst{\phi_0}{\epsilon_1}$ is a 
morphism from $\ctxext{A}{P}$ to $A$. Thus, we can require
\begin{equation}\label{extempalg-eq3}
\jhomeq{\Gamma}{{A}{P}}{A}{\subst{\phi_0}{\epsilon_1}}{\epsilon_0}.
\end{equation}
For the final judgmental equality we need to explain the term
\begin{equation*}
\jterm
  {{{\Gamma}{A}}{\subst{\phi_1}{\jcomp{}{\epsilon_0}{P}}}}
  {\subst{\phi_1}{\ctxwk{\ctxext{P}{\jcomp{}{\epsilon_0}{P}}}{P}}}
  {\subst{\phi_1}{\epsilon_1}}.
\end{equation*}
We have already established that $\subst{\phi_1}{\jcomp{}{\epsilon_0}{P}}\jdeq
P$. We also see that 
\begin{align*}
\subst{\phi_1}{\ctxwk{\ctxext{P}{\jcomp{}{\epsilon_0}{P}}}{P}}
& \jdeq
  \subst{\phi_1}{\ctxwk{\jcomp{}{\epsilon_0}{P}}{{P}{P}}}
  \tag{by \autoref{comp-ew-f}}
  \\
& \jdeq
  \ctxwk{\subst{\phi_1}{\jcomp{}{\epsilon_0}{P}}}{\subst{\phi_1}{\ctxwk{P}{P}}}
  \tag{by \autoref{comp-sw-f}}
  \\
& \jdeq
  \ctxwk{P}{\subst{\phi_1}{\ctxwk{P}{P}}}
  \tag{by \autoref{cor:empalg-mor}}
  \\
& \jdeq
  \ctxwk{P}{P}
  \tag{by \autoref{cancellation-ws-f}}
\end{align*}
and we will therefore require that
\begin{equation}\label{extempalg-eq4}
\jtermeq{{{\Gamma}{A}}{P}}{\ctxwk{P}{P}}{\subst{\phi_1}{\epsilon_1}}{\idtm{P}}.
\end{equation}
We bring all this together in the definition of extension-empty algebras:

\begin{defn}
An \emph{extension-empty algebra in context $\Gamma$} 
is a sextuple $(A,P,\epsilon_0,\epsilon_1,\phi_0,\phi_1)$ for which 
the quadruple $(A,P,\epsilon_0,\epsilon_1)$ is an extension algebra in context 
$\Gamma$, the quadruple $(A,P,\phi_0,\phi_1)$ is an empty algebra in context
$\Gamma$, satisfying the judgmental equalities 
\autoref{extempalg-eq1,extempalg-eq2,extempalg-eq3,extempalg-eq4}.
\end{defn}

\begin{thm}
Let $(A,P,\epsilon_0,\epsilon_1,\phi_0,\phi_1)$ be an extension-empty algebra
in context $\Gamma$. Then the sextuple
\begin{equation*}
(P,\jcomp{}{\epsilon_0}{P},\epsilon_1,\jcomp{}{\epsilon_0}{\epsilon_1},\phi_1,
\jcomp{}{\epsilon_0}{\phi_1})
\end{equation*}
is an extension-empty algebra in context $\ctxext{\Gamma}{A}$. 
\end{thm}

\begin{proof}
We first check whether the quadruple $(P,\jcomp{}{\epsilon_0}{P},\phi_1,\jcomp{}{\epsilon_0}{\phi_1})$
is an empty algebra in context $\ctxext{\Gamma}{A}$. Thus we need to verify
that $\subst{\phi_1}{\jcomp{}{\epsilon_0}{P}}\jdeq P$ and that
$\subst{\phi_1}{\jcomp{}{\epsilon_0}{\phi_1}}\jdeq\phi_1$. Both these judgmental
equalities follow from \autoref{lem:empalg-mor}.

To complete the proof, we need to verify that
\autoref{extempalg-eq1,extempalg-eq2,extempalg-eq3,extempalg-eq4} hold in our
current setting.
\end{proof}

\subsection{Term-algebras and term-operations}
Until now we had no need to consider term-algebras. Extension is an operation
letting contexts and families interact and the empty object is a term which
requires that contexts and families are compatible. However, to model the
theory of contexts, families and terms we do need to take terms into account.

\begin{defn}
A term-algebra in context $\Gamma$ is a septuple 
$(A,P,T,\epsilon_0,\epsilon_1,\phi_0,\phi_1)$ such that the quadruple
$(A,P,\epsilon_0,\epsilon_1)$ is an extension algebra in context $\Gamma$, 
the quadruple $(A,P,\phi_0,\phi_1)$ is an empty algebra in context $\Gamma$,
for which we additionally have the following judgmental equality:
\begin{align*}
\jfameq*{{\Gamma}{A}}{\subst{\phi_1}{T}}{\emptyf}.
\end{align*}
\end{defn}

\begin{thm}
Let $\mathcal{A}\defeq(A,P,T,\epsilon_0,\epsilon_1,\phi_0,\phi_1)$ be a term-algebra in context
$\Gamma$. Then the septuple
\begin{equation*}
(P,\jcomp{}{\epsilon_0}{P},\jcomp{}{\epsilon_0}{T},\epsilon_1,\jcomp{}{\epsilon_0}{\epsilon_1},\phi_1,\jcomp{}{\epsilon_0}{\phi_1})
\end{equation*}
is a term-algebra in context $\ctxext{\Gamma}{A}$, called the family term-algebra
of $\mathcal{A}$.
\end{thm}

\begin{thm}
Let $\mathcal{A}\defeq(A,P,T,\epsilon_0,\epsilon_1,\phi_0,\phi_1)$ be a term-algebra in context
$\Gamma$ and let $X$ be a family in context $\Gamma$. Then
\begin{equation*}
\ctxwk{X}{\mathcal{A}}\defeq(\ctxwk{X}{A},\ctxwk{X}{P},\ctxwk{X}{T},\ctxwk{X}{\epsilon_0},\ctxwk{X}{\epsilon_1},\ctxwk{X}{\phi_0},\ctxwk{X}{\phi_1})
\end{equation*}
is a term-algebra in context $\ctxext{\Gamma}{X}$.
\end{thm}

\begin{thm}
Let $\mathcal{A}\defeq(A,P,T,\epsilon_0,\epsilon_1,\phi_0,\phi_1)$ be a term-algebra in context
$\ctxext{\Gamma}{X}$ and let $\jterm{\Gamma}{X}{x}$. Then
\begin{equation*}
\subst{x}{\mathcal{A}}\defeq
(\subst{x}{A},\subst{x}{P},\subst{x}{T},\subst{x}{\epsilon_0},\subst{x}{\epsilon_1},
\subst{x}{\phi_0},\subst{x}{\phi_1})
\end{equation*}
is a term-algebra in context $\Gamma$.
\end{thm}

\begin{defn}
Let $\mathcal{A}$ and $\mathcal{B}$ be term-algebras in context $\Gamma$. 
A term-operation from $\mathcal{A}$ to $\mathcal{B}$ is a triple $(f_0,f_1,f_2)$
consisting of
\begin{align*}
\jhom*{\Gamma}{A}{B}{f_0}
  \\
\jfhom*{\Gamma}{A}{B}{f_0}{P}{Q}{f_1}
  \\
\jfhom*{\Gamma}{{A}{P}}{{B}{Q}}{\jvcomp{P}{f_0}{f_1}}{T}{U}{f_2}
\end{align*}
such that the pair $(f_0,f_1)$ is both an extension homomorphism from
$(A,P,\epsilon_0,\epsilon_1)$ to $(B,Q,\eta_0,\eta_1)$ in context $\Gamma$
and an empty homomorphism from $(A,P,\phi_0,\phi_1)$ to $(B,Q,\psi_0,\psi_1)$
in context $\Gamma$.
\end{defn}

\subsection{Weakening term-algebras}
Weakening term-algebras will be term-algebras with certain added structure.
Although strictly speaking one could formulate a notion of weakening algebra
which only depends on extension algebras and which omits both empty objects
and terms, we shall not do so here. 

Let $(A,P,T,\epsilon_0,\epsilon_1,\phi_0,\phi_1)$ be a term-algebra in context
$\Gamma$. 

An extension-weakening algebra in context $\Gamma$ will be an octuple
\begin{equation*}
(A,P,\epsilon_0,\epsilon_1,\phi_0,\phi_1,\omega_0,\omega_1)
\end{equation*}
where $(A,P,\epsilon_0,\epsilon_1,\phi_0,\phi_1)$ is an extension-empty algebra
in context $\Gamma$ and where
\begin{align*}
\jhom*
  {{{\Gamma}{A}}{P}}
  {\ctxwk{P}{P}}
  {\jcomp{}{\epsilon_0}{P}}
  {\omega_0}
  \\
\jfhom*
  {{{\Gamma}{A}}{P}}
  {\ctxwk{P}{P}}
  {\jcomp{}{\epsilon_0}{P}}
  {\omega_0}
  {\ctxwk{P}{\jcomp{}{\epsilon_0}{P}}}
  {\jcomp{}{\epsilon_0}{\jcomp{}{\epsilon_0}{P}}}
  {\omega_1}
\end{align*}
such that $(\omega_0,\omega_1)$ is both an extension homomorphism
\begin{equation*}
\begin{tikzcd}[column sep=large]
\ctxwk{P}{(P,
  \jcomp{}{\epsilon_0}{P},
  \epsilon_1,
  \jcomp{}{\epsilon_0}{\epsilon_1})}
\ar{r}{(\omega_0,\omega_1)}
& \jcomp{}{\epsilon_0}{(P,
  \jcomp{}{\epsilon_0}{P},
  \epsilon_1,
  \jcomp{}{\epsilon_0}{\epsilon_1})}
\end{tikzcd}
\end{equation*}
and an empty homomorphism

satisfying five(?) additional judgmental equalities expressing that $\omega_0$
and $\omega_1$ are compatible with $\epsilon_1$, $\phi_1$
and with each other. Our current goal is to figure out what these are. 

Before we go into that, we develop a bit of intuition by explaining how
$\omega_0$ and $\omega_1$ act when applied to a family $x_1:\subst{x_0}{P}$
over $x_0:A$. Note that
\begin{equation*}
\jterm{{\Gamma}{\subst{x_0}{P}}}{\subst{{x_1}{{x_0}{\epsilon_0}}}{P}}{\subst{x_1}{{x_0}{\omega_0}}},
\end{equation*}
so $\subst{x_1}{{x_0}{\omega_0}}$ takes families over $x_0$ to families over
the extended $\subst{x_1}{{x_0}{\epsilon_0}}$.

To compute the type of $\subst{x_2}{{x_1}{{x_0}{\omega_1}}}$ for families
$x_1,x_2:\subst{x_0}{P}$ over $x_0:A$ we have to do a
bit more work. Note that
\begin{align*}
& \subst{x_2}{{x_1}{{x_0}{\jcomp{}{\omega_0}{\jcomp{}{\epsilon_0}{\jcomp{}{\epsilon_0}{P}}}}}}
  \\
& \jdeq
  \subst{x_2}{{x_1}{{x_0}{\unfold{\jcomp{\ctxwk{P}{P}}{\omega_0}{\jcomp{}{\epsilon_0}{\jcomp{}{\epsilon_0}{P}}}}}}}
  \tag{by definition}
  \\
& \jdeq
  \subst
    { {x_2}
      { {x_1}
        { {x_0}
          {\omega_0}
          }
        }
      }
    { {x_2}
      { {x_1}
        { {x_0}
          {\ctxwk{{P}{P}}{\jcomp{}{\epsilon_0}{\jcomp{}{\epsilon_0}{P}}}}
          }
        }
      }
  \tag{by \autoref{comp-ss-f}}
  \\
& \jdeq
  \subst
    { {x_2}
      { {x_1}
        { {x_0}
          {\omega_0}
          }
        }
      }
    { {x_2}
      { \ctxwk
          {\subst{x_1}{\ctxwk{\subst{x_0}{P}}{\subst{x_0}{P}}}}
          {\subst{x_1}{{x_0}{\jcomp{}{\epsilon_0}{\jcomp{}{\epsilon_0}{P}}}}}
        }
      }
  \tag{by \autoref{comp-sw-f}}
  \\
& \jdeq
  \subst
    { {x_2}
      { {x_1}
        { {x_0}
          {\omega_0}
          }
        }
      }
    { {x_1}
      { {x_0}
        {\jcomp{}{\epsilon_0}{\jcomp{}{\epsilon_0}{P}}}
        }
      }
  \tag{by \autoref{cancellation-ws-f}}
  \\
& \jdeq
  \subst
    { {{x_2}{{x_1}{{x_0}{\omega_0}}}}
      {{{x_1}{{x_0}{\epsilon_0}}}{\epsilon_0}}
      }
    { P
      }.
  \tag{by \autoref{comp-ss-f}}
\end{align*}
Therefore, we see that
\begin{equation*}
\jterm
  {{\Gamma}{\subst{{x_2}{{x_0}{\epsilon_0}}}{P}}}
  { \subst
      { {{x_2}{{x_1}{{x_0}{\omega_0}}}}
        {{{x_1}{{x_0}{\epsilon_0}}}{\epsilon_0}}
        }
      { P
        }
    }
  { \subst{x_2}{{x_1}{{x_0}{\omega_1}}}
    }
\end{equation*}
Thus, the term $\subst{x_2}{{x_1}{{x_0}{\omega_1}}}$ takes a family over the
extended context $\subst{x_2}{{x_0}{\epsilon_0}}$ to a family over the
weakening $\subst{x_2}{{x_1}{{x_0}{\omega_0}}}$, which is itself a family
over $\subst{x_1}{{x_0}{\epsilon_0}}$. In other words, $\omega_1$
is precisely the internalization of the action on families of weakening, as
intended.

\subsubsection{The compatibility of weakening with extension}
The first rule, expressing that $\omega_0$ is compatible with $\epsilon_1$ is that
the diagram
\begin{equation}\label{wkalg-exteq1}
\begin{tikzcd}[column sep=large]
\ctxwk{P}{\ctxext{P}{\jcomp{}{\epsilon_0}{P}}}
  \ar{r}{\jvcomp{}{\omega_0}{\omega_1}}
  \ar{d}[swap]{\ctxwk{P}{\epsilon_1}}
& \ctxext
    {\jcomp{}{\epsilon_0}{P}}
    {\jcomp{}{\epsilon_0}{\jcomp{}{\epsilon_0}{P}}}
  \ar{d}{\jcomp{}{\epsilon_0}{\epsilon_1}}
  \\
\ctxwk{P}{P}
  \ar{r}[swap]{\omega_0}
& \jcomp{}{\epsilon_0}{P}
\end{tikzcd}
\end{equation}
commutes judgmentally. For the second rule, expressing that $\omega_1$ is 
compatible with $\epsilon_1$, we wish to fill in a commutative diagram
\begin{equation*}
\begin{tikzcd}[column sep=huge]
\ctxwk
  {P}
  { \ctxext
      {\jcomp{}{\epsilon_0}{P}}
      {\jcomp{}{\epsilon_0}{\jcomp{}{\epsilon_0}{P}}}
    }
  \ar[densely dotted]{r}{\jvcomp{}{\omega_1}{?}}
  \ar{d}[swap]{\ctxwk{P}{\jcomp{}{\epsilon_0}{\epsilon_1}}}
& \jcomp{}{\omega_0}{%
    \jcomp{}{\epsilon_0}{%
      \jcomp{}{\epsilon_0}{%
        \ctxext{P}{\jcomp{}{\epsilon_0}{P}}
        }
      }
    }
  \ar{d}{ \jcomp{}{\omega_0}{%
            \jcomp{}{\epsilon_0}{%
              \jcomp{}{\epsilon_0}{\epsilon_1}
              }
            }
          }
  \\
\ctxwk{P}{\jcomp{}{\epsilon_0}{P}}
  \ar{r}[swap]{\omega_1}
& \jcomp{}{\omega_0}{%
    \jcomp{}{\epsilon_0}{%
      \jcomp{}{\epsilon_0}{P}
      }
    }
\end{tikzcd}
\end{equation*}
The morphism indicated by the question mark is a morphism from
$\ctxwk{P}{\jcomp{}{\epsilon_0}{\jcomp{}{\epsilon_0}{P}}}$ to
$%
\jcomp{}{\omega_1}{%
  \jcomp{}{\omega_0}{%
    \jcomp{}{\epsilon_0}{%
      \jcomp{}{\epsilon_0}{%
        \ctxext{P}{\jcomp{}{\epsilon_0}{P}}
        }
      }
    }
  }
$ in context $\ctxext{{\Gamma}{A}}{P}$. One would expect this morphism to be
\begin{equation*}
\begin{tikzcd}[column sep=large]
\jcomp{}{\ctxwk{P}{\epsilon_0}}{\ctxwk{P}{\jcomp{}{\epsilon_0}{P}}}
  \ar{r}{\jcomp{}{\ctxwk{P}{\epsilon_0}}{\omega_1}}
& \jcomp{}{\ctxwk{P}{\epsilon_0}}{%
    \jcomp{}{\omega_0}{%
      \jcomp{}{\epsilon_0}{%
        \jcomp{}{\epsilon_0}{%
          P
          }
        }
      }
    }
\end{tikzcd}
\end{equation*}
in the context $\ctxext{{\Gamma}{A}}{P}$, 
so we must verify the judgmental equalities
\begin{align*}
\jcomp{}{\ctxwk{P}{\epsilon_0}}{\ctxwk{P}{\jcomp{}{\epsilon_0}{P}}}
& \jdeq 
  \ctxwk{P}{\jcomp{}{\epsilon_0}{\jcomp{}{\epsilon_0}{P}}}
  \\
\jcomp{}{\ctxwk{P}{\epsilon_0}}{%
    \jcomp{}{\omega_0}{%
      \jcomp{}{\epsilon_0}{%
        \jcomp{}{\epsilon_0}{%
          P
          }
        }
      }
    }
& \jdeq
  \jcomp{}{\omega_1}{%
    \jcomp{}{\omega_0}{%
      \jcomp{}{\epsilon_0}{%
        \jcomp{}{\epsilon_0}{%
          \ctxext{P}{\jcomp{}{\epsilon_0}{P}}
          }
        }
      }
    } 
\end{align*}
of families over the context $\ctxext{{\Gamma}{A}}{P}$. The first of these
judgmental equalities is nothing special: it follows at straight from
\autoref{comp-ws-f}. We prove the second equality in the following lemma.

\begin{lem}
Let $(A,P,\epsilon_0,\epsilon_1)$ be an extension algebra and let
\begin{align*}
\jhom*
  {{{\Gamma}{A}}{P}}
  {\ctxwk{P}{P}}
  {\jcomp{}{\epsilon_0}{P}}
  {\omega_0}
  \\
\jfhom*
  {{{\Gamma}{A}}{P}}
  {\ctxwk{P}{P}}
  {\jcomp{}{\epsilon_0}{P}}
  {\omega_0}
  {\ctxwk{P}{\jcomp{}{\epsilon_0}{P}}}
  {\jcomp{}{\epsilon_0}{\jcomp{}{\epsilon_0}{P}}}
  {\omega_1}
\end{align*}
be morphisms satisfying \autoref{wkalg-exteq1}. Then the inference rules
\begin{align*}
& \inference
  { \jfam{{\Gamma}{A}}{Q}
    }
  { \jfameq
      {\blank}
      { \jcomp{}{\omega_1}{%
          \jcomp{}{\omega_0}{%
            \jcomp{}{\epsilon_0}{%
              \jcomp{}{\epsilon_0}{%
                \jcomp{}{\epsilon_0}{Q}
                }
              }
            }
          } 
        }
      { \jcomp{}{\ctxwk{P}{\epsilon_0}}{%
          \jcomp{}{\omega_0}{%
            \jcomp{}{\epsilon_0}{%
              \jcomp{}{\epsilon_0}{%
                Q
                }
              }
            }
          }
        }
      }
\end{align*}
are valid.
\end{lem}

\begin{proof}
We have the judgmental equalities
\begin{align*}
& \jcomp{}{\omega_1}{%
          \jcomp{}{\omega_0}{%
            \jcomp{}{\epsilon_0}{%
              \jcomp{}{\epsilon_0}{%
                \jcomp{}{\epsilon_0}{Q}
                }
              }
            }
          } 
  \\
& \jdeq
  \jcomp{}{\jvcomp{}{\omega_0}{\omega_1}}{%
            \jcomp{}{\epsilon_0}{%
              \jcomp{}{\epsilon_0}{%
                \jcomp{}{\epsilon_0}{Q}
                }
              }
            }
  \tag{by \autoref{lem:composition-threesome}}
  \\
& \jdeq
  \jcomp{}{\jvcomp{}{\omega_0}{\omega_1}}{%
            \jcomp{}{\epsilon_0}{%
              \jcomp{}{\epsilon_1}{%
                \jcomp{}{\epsilon_0}{Q}
                }
              }
            }
  \tag{by \autoref{lem:extalg-twins}}
  \\
& \jdeq
  \jcomp{}{\jvcomp{}{\omega_0}{\omega_1}}{%
            \jcomp{}{\jcomp{}{\epsilon_0}{\epsilon_1}}{%
              \jcomp{}{\epsilon_0}{%
                \jcomp{}{\epsilon_0}{Q}
                }
              }
            }
   \tag{by \autoref{lem:jcomp-higherjcomp}}
   \\
& \jdeq
  \jcomp{}{\jcomp{}{%
             \jvcomp{}{\omega_0}{\omega_1}}{%
               \jcomp{}{\epsilon_0}{\epsilon_1}}}{%
              \jcomp{}{\epsilon_0}{%
                \jcomp{}{\epsilon_0}{Q}
                }
              }
  \tag{by \autoref{lem:jcomp-jcomp}}
  \\
& \jdeq
  \jcomp{}{\jcomp{}{\ctxwk{P}{\epsilon_0}}{\omega_0}}{%
              \jcomp{}{\epsilon_0}{%
                \jcomp{}{\epsilon_0}{Q}
                }
              }
  \tag{by \autoref{wkalg-exteq1}}
  \\
& \jdeq
  \jcomp{}{\ctxwk{P}{\epsilon_0}}{%
    \jcomp{}{\omega_0}{%
      \jcomp{}{\epsilon_0}{%
        \jcomp{}{\epsilon_0}{%
          Q
          }
        }
      }
    }
  \tag{by \autoref{lem:jcomp-jcomp}}
\end{align*}
\end{proof}

This enables us to require that the diagram
\begin{equation}\label{wkalg-exteq2}
\begin{tikzcd}[column sep=huge]
\ctxwk
  {P}
  { \ctxext
      {\jcomp{}{\epsilon_0}{P}}
      {\jcomp{}{\epsilon_0}{\jcomp{}{\epsilon_0}{P}}}
    }
  \ar{r}{\jvcomp{}{\omega_1}{\jcomp{}{\ctxwk{P}{\epsilon_0}}{\omega_1}}}
  \ar{d}[swap]{\ctxwk{P}{\jcomp{}{\epsilon_0}{\epsilon_1}}}
& \jcomp{}{\omega_0}{%
    \jcomp{}{\epsilon_0}{%
      \jcomp{}{\epsilon_0}{%
        \ctxext{P}{\jcomp{}{\epsilon_0}{P}}
        }
      }
    }
  \ar{d}{ \jcomp{}{\omega_0}{%
            \jcomp{}{\epsilon_0}{%
              \jcomp{}{\epsilon_0}{\epsilon_1}
              }
            }
          }
  \\
\ctxwk{P}{\jcomp{}{\epsilon_0}{P}}
  \ar{r}[swap]{\omega_1}
& \jcomp{}{\omega_0}{%
    \jcomp{}{\epsilon_0}{%
      \jcomp{}{\epsilon_0}{P}
      }
    }
\end{tikzcd}
\end{equation}
commutes judgmentally.

\subsubsection{The compatibility of weakening with the empty context and family}
The first two judgmental equalities expressing that $\omega_0$ and $\omega_1$
are compatible with $\phi_1$ are easy to state:
\begin{align}
\jhomeq*{{\Gamma}{A}}{P}{P}{\subst{\phi_1}{\omega_0}}{\idtm{P}}\\
\jhomeq*{{\Gamma}{A}}{P}{P}{\subst{\phi_1}{\omega_1}}{\idtm{P}}
\end{align}

\subsubsection{The compatibility of weakening with itself}
To express the compatibility of weakening with itself, we must fill in the
following diagram:
\begin{equation*}
\begin{tikzcd}
\ctxwk{P}{{P}{P}}
  \ar{r}{\ctxwk{P}{\omega_0}}
  \ar{d}[swap]{\ctxwk{{P}{P}}{\omega_0}}
& \ctxwk{P}{\jcomp{}{\epsilon_0}{P}}
  \ar{d}{\omega_1}
  \\
\ctxwk{{P}{P}}{\jcomp{}{\epsilon_0}{P}}
  \ar[densely dotted]{r}
& \jcomp{}{\omega_0}{\jcomp{}{\epsilon_0}{\jcomp{}{\epsilon_0}{P}}}
\end{tikzcd}
\end{equation*}

\subsubsection{The definition of weakening algebras}

\subsection{Extension-substitution algebras}

Let $(A,P,T,\epsilon_0,\epsilon_1,\phi_0,\phi_1)$ be a term-algebra. To define
substitution algebras, we will consider
\begin{align*}
\jhom*
  {{{{\Gamma}{A}}{P}}{T}}
  {\ctxwk{T}{\jcomp{}{\epsilon_0}{P}}}
  {P}
  {\sigma_0}
  \\
\jfhom*
  {{{{\Gamma}{A}}{P}}{T}}
  {\ctxwk{T}{\jcomp{}{\epsilon_0}{P}}}
  {P}
  {\sigma_0}
  {\ctxwk{T}{\jcomp{}{\epsilon_0}{\jcomp{}{\epsilon_0}{P}}}}
  {\jcomp{}{\epsilon_0}{P}}
  {\sigma_1}
  \\
\jfhom*
  {{{{\Gamma}{A}}{P}}{T}}
  { \ctxext
      {\ctxwk{T}{\jcomp{}{\epsilon_0}{P}}}
      {\ctxwk{T}{\jcomp{}{\epsilon_0}{\jcomp{}{\epsilon_0}{P}}}}
    }
  {\ctxext{P}{\jcomp{}{\epsilon_0}{P}}}
  {\jvcomp{}{\sigma_0}{\sigma_1}}
  {\ctxwk{T}{\jcomp{}{\epsilon_0}{\jcomp{}{\epsilon_0}{T}}}}
  {\jcomp{}{\epsilon_0}{T}}
  {\sigma_2}
\end{align*}
for which the pair $(\sigma_0,\sigma_1)$ is required to be an extension
homomorphism and the triple $(\sigma_0,\sigma_1,\sigma_2)$ is required to
be an empty term-homomorphism.

\subsection{Pre-universes}
Pre-universes are internal versions of the theory of contexts, families and
terms. They interpret extension, the empty context, weakening, substitution
and identity terms all at once in a compatible way. Besides the compatibility
properties there will be judgmental equalities analoguous to the cancellation
properties of \autoref{cancellation-ws,cancellation-i}. Pre-universes are to
the theory of contexts, families and terms what internal categories to a
category.


%\part{Type constructors}

%\section{The dependent function constructor}
As we have done with most everything so far, we will take the point of view that
the dependent function constructor is an operation with an action on contexts,
on families and on terms. When $A$ is a family of contexts over $\Gamma$,
$\mprd{A}{\blank}$ takes things in context $\ctxext{\Gamma}{A}$ to the context
$\Gamma$. It's action on terms restricted to the empty context is usually 
denoted by $\lambda$, so we shall denote the entire action by $\lambda$. As
usual, $\lambda$ is reversed by evaluation. 

The dependent function type constructor not only acts on families $P$ of
contexts over $\ctxext{\Gamma}{A}$, but it also sends families $Q$ of contexts
over $\ctxext{{\Gamma}{A}}{P}$ to families $\mprd[\famsym]{A}{Q}$ over $\ctxext{\Gamma}
{\mprd{A}{P}}$. Moreover, we will also have a version of $\lambda$-abstraction
for terms of such families $Q$.

\begin{align*}
& \inference
    { \jfam{{\Gamma}{A}}{P}
      }
    { \jfam{\Gamma}{\mprd{A}{P}}
      }
& & \inference
      { \jfameq{\Gamma}{A}{A'}
        \jfameq{{\Gamma}{A}}{P}{P'}
        }
      { \jfameq{\Gamma}{\mprd{A}{P}}{\mprd{A'}{P'}}
        }
  \\
& \inference
    { \jfam{{{\Gamma}{A}}{P}}{Q}
      }
    { \jfam{{\Gamma}{\mprd{A}{P}}}{\mprd[\famsym]{A}{Q}}
      }
& & \inference
      { \jfameq{\Gamma}{A}{A'}
        \jfameq{{{\Gamma}{A}}{P}}{Q}{Q'}
        }
      { \jfameq{{\Gamma}{\mprd{A}{P}}}{\mprd[\famsym]{A}{Q}}{\mprd[\famsym]{A'}{Q'}}
        }
  \\
& \inference
    { \jterm{{{\Gamma}{A}}{P}}{Q}{g}
      }
    { \jterm{{\Gamma}{\mprd{A}{P}}}{\mprd[\famsym]{A}{Q}}{\slam{A}{Q}{g}}
      }
& & \inference
      { \jfameq{\Gamma}{A}{A'}
        \jtermeq{{{\Gamma}{A}}{P}}{Q}{g}{g'}
        }
      { \jtermeq{{\Gamma}{\mprd{A}{P}}}{\mprd[\famsym]{A}{Q}}{\slam{A}{Q}{g}}{\slam{A'}{Q}{g'}}
        }
\end{align*}

\begin{defn}
Let $A$ and $B$ be families of contexts over $\Gamma$. Then we define
\begin{equation*}
\jfamdefn{\Gamma}{\jfun{A}{B}}{\sprd{A}{\ctxwk{A}{B}}}.
\end{equation*}
\end{defn}

\subsection{The compatibility rules for the dependent product constructor}
In this subsection we lay out the compatibility rules which we will require
for the dependent product constructor.

\subsubsection{Dependent products are compatible with the empty context}
The empty context can appear in the domain and in the codomain of the dependent
function type constructor. We have the following inference rules explaining
what happens when the empty context appears in the domain:
\begin{align}
& \inference
    { \jfam{\Gamma}{P}
      }
    { \jfameq{\Gamma}{\mprd{\emptyf}{P}}{P}
      }
  \\
& \inference
    { \jfam{{\Gamma}{P}}{Q}
      }
    { \jfameq{{\Gamma}{P}}{\mprd[\famsym]{\emptyf}{Q}}{Q}
      }
  \\
& \inference
    { \jterm{{\Gamma}{P}}{Q}{g}
      }
    { \jtermeq{{\Gamma}{P}}{Q}{\slam{\emptyf}{Q}{g}}{g}
      }
\end{align}
We have the following infernece rules explaining what happens when the empty
context appears in the codomain:
\begin{align}
& \inference
    { \jfam{\Gamma}{A}
      }
    { \jfameq{\Gamma}{\mprd{A}{\emptyf}}{\emptyf}
      }
  \\
& \inference
    { \jfam{{\Gamma}{A}}{P}
      }
    { \jfameq{{\Gamma}{\mprd{A}{P}}}{\mprd[\famsym]{A}{\emptyf}}{\emptyf}
      }
\end{align}
Now we see that we can compare $\mprd{A}{P}$ with $\mprd[\famsym]{A}{P}$ by
seeing $P$ as a family over $\ctxext{{\Gamma}{A}}{\emptyf}$. We will require
the following inference rule to be valid:
\begin{equation}
\inference
  { \jfam{{\Gamma}{A}}{P}
    }
  { \jfameq{\Gamma}{\mprd{A}{P}}{\mprd[\famsym]{A}{P}}
    }
\end{equation}
This rule will allow us to omit the annotation indicating the action on families,
which we will do from now on. We also note that because the empty family over
$\ctxext{\Gamma}{A}$ is mapped to the empty family over $\Gamma$ by the
dependent function constructor, we obtain the important special cases of
lambda abstraction and evaluation that
\begin{align*}
& \inference
  { \jterm{{\Gamma}{A}}{P}{f}
    }
  { \jterm{\Gamma}{\mprd{A}{P}}{\slam{A}{P}{f}}
    }
\end{align*}
Thus, we retrieve what is originally meant by lambda abstraction.

\subsubsection{Dependent products are compatible with extension}
The following rules describe what happens when a dependent function constructor
is taken over an extension

\begin{align*}
& \inference
  { \jfam{{\Gamma}{A}}{P}
    \jfam{{{\Gamma}{A}}{P}}{Q}
    }
  { \jfameq
      {\Gamma}
      {\mprd{\ctxext{A}{P}}{Q}}
      {\mprd{A}{\mprd{P}{Q}}}
    }
  \\
& \inference
  { \jfam{{\Gamma}{A}}{P}
    \jfam{{{{\Gamma}{A}}{P}}{Q}}{R}
    }
  { \jfameq
      {{\Gamma}{\mprd{\ctxext{A}{P}}{Q}}}
      {\mprd{\ctxext{A}{P}}{R}}
      {\mprd{A}{\mprd{P}{R}}}
    }
  \\
& \inference
  { \jfam{{\Gamma}{A}}{P}
    \jterm{{{{\Gamma}{A}}{P}}{Q}}{R}{h}
    }
  { \jtermeq
      {{\Gamma}{\mprd{\ctxext{A}{P}}{Q}}}
      {\mprd{\ctxext{A}{P}}{R}}
      {\slam{\ctxext{A}{P}}{R}{h}}
      {\slam{A}{\mprd{P}{R}}{\slam{P}{R}{h}}}
    }
\end{align*}

When the dependent product constructor is applied to an extension, we get the
following:

\begin{align*}
& \inference
  { \jfam{{{\Gamma}{A}}{P}}{Q}
    }
  { \jfameq
      {\Gamma}
      {\mprd{A}{\ctxext{P}{Q}}}
      {\ctxext{\mprd{A}{P}}{\mprd{A}{Q}}}
    }
  \\
& \inference
  { \jfam{{{{\Gamma}{A}}{P}}{Q}}{R}
    }
  { \jfameq
      {{\Gamma}{\mprd{A}{P}}}
      {\mprd{A}{\ctxext{Q}{R}}}
      {\ctxext{\mprd{A}{Q}}{\mprd{A}{R}}}
    }
\end{align*}
We show in \autoref{lem:slam-ext} that the remaining two properties
involving lambda abstraction and evaluation are derivable.

\subsubsection{Weakening is compatible with the dependent product constructor}

\begin{align*}
& \inference
  { \jfam{\Gamma}{A}
    \jfam{{{\Gamma}{B}}{Q}}{R}
    }
  { \jfameq
      {{{\Gamma}{A}}{\ctxwk{A}{B}}}
      {\ctxwk{A}{\mprd{Q}{R}}}
      {\mprd{\ctxwk{A}{Q}}{\ctxwk{A}{R}}}
    }
  \\
& \inference
  { \jfam{\Gamma}{A}
    \jfam{{{{\Gamma}{B}}{Q}}{R}}{S}
    }
  { \jfameq
      {{{{\Gamma}{A}}{\ctxwk{A}{B}}}{\ctxwk{A}{\mprd{Q}{R}}}}
      {\ctxwk{A}{\mprd{Q}{S}}}
      {\mprd{\ctxwk{A}{Q}}{\ctxwk{A}{S}}}
    }
  \\
& \inference
  { \jfam{\Gamma}{A}
    \jterm{{{{\Gamma}{B}}{Q}}{R}}{S}{k}
    }
  { \jtermeq
      {{{{\Gamma}{A}}{\ctxwk{A}{B}}}{\ctxwk{A}{\mprd{Q}{R}}}}
      {\ctxwk{A}{\mprd{Q}{S}}}
      {\ctxwk{A}{\slam{Q}{S}{k}}}
      {\slam{\ctxwk{A}{Q}}{\ctxwk{A}{S}}{\ctxwk{A}{k}}}
    }
\end{align*}

\subsubsection{Dependent products are taken fiberwise}
The rules explaining that substitution is compatible with the dependent product
constructor assert that dependent products are taken fiberwise (as is usual).

\begin{align*}
& \inference
  { \jterm{\Gamma}{A}{x}
    \jfam{{{{\Gamma}{A}}{P}}{Q}}{R}
    }
  { \jfameq
      {{\Gamma}{\subst{x}{P}}}
      {\subst{x}{\mprd{Q}{R}}}
      {\mprd{\subst{x}{Q}}{\subst{x}{R}}}
    }
  \\
& \inference
  { \jterm{\Gamma}{A}{x}
    \jfam{{{{{\Gamma}{A}}{P}}{Q}}{R}}{S}
    }
  { \jfameq
      {{{\Gamma}{\subst{x}{P}}}{\mprd{\subst{x}{Q}}{\subst{x}{R}}}}
      {\subst{x}{\mprd{Q}{S}}}
      {\mprd{\subst{x}{Q}}{\subst{x}{S}}}
    }
  \\
& \inference
  { \jterm{\Gamma}{A}{x}
    \jterm{{{{{\Gamma}{A}}{P}}{Q}}{R}}{S}{k}
    }
  { \jtermeq
      {{{\Gamma}{\subst{x}{P}}}{\mprd{\subst{x}{Q}}{\subst{x}{R}}}}
      {\subst{x}{\mprd{Q}{S}}}
      {\subst{x}{\slam{Q}{S}{k}}}
      {\slam{\subst{x}{Q}}{\subst{x}{S}}{\subst{x}{k}}}
    }
\end{align*}

\subsubsection{The dependent function constructor is compatible with weakening}
\begin{align*}
& \inference
  { \jfam{{{\Gamma}{A}}{P}}{Q}
    \jfam{{{\Gamma}{Q}}{P}}{R}
    }
  { \jfameq
      {{{\Gamma}{\mprd{A}{P}}}{\mprd{A}{Q}}}
      {\mprd{A}{\ctxwk{Q}{R}}}
      {\ctxwk{\mprd{A}{Q}}{\mprd{A}{R}}}
    }
  \\
& \inference
  { \jfam{{{\Gamma}{A}}{P}}{Q}
    \jfam{{{{\Gamma}{P}}{Q}}{R}}{S}
    }
  { \jfameq
      {{{{\Gamma}{\mprd{A}{P}}}{\mprd{A}{Q}}}{\mprd{A}{\ctxwk{Q}{R}}}}
      {\mprd{A}{\ctxwk{Q}{S}}}
      {\ctxwk{\mprd{A}{Q}}{\mprd{A}{S}}}
    }
  \\
& \inference
  { \jfam{{{\Gamma}{A}}{P}}{Q}
    \jterm{{{{\Gamma}{P}}{Q}}{R}}{S}{k}
    }
  { \jtermeq
      {{{{\Gamma}{\mprd{A}{P}}}{\mprd{A}{Q}}}{\mprd{A}{\ctxwk{Q}{R}}}}
      {\mprd{A}{\ctxwk{Q}{S}}}
      {\slam{A}{\ctxwk{Q}{S}}{\ctxwk{Q}{k}}}
      {\ctxwk{\mprd{A}{Q}}{\slam{A}{S}{k}}}
    }
\end{align*}

\subsubsection{The dependent function constructor is compatible with substitution}
\begin{align*}
& \inference
  { \jterm{{{\Gamma}{A}}{P}}{Q}{g}
    \jfam{{{{\Gamma}{A}}{P}}{Q}}{R}
    }
  { \jfameq
      {{\Gamma}{\mprd{A}{P}}}
      {\mprd{A}{\subst{g}{R}}}
      {\subst{\slam{A}{Q}{g}}{\mprd{A}{R}}}
    }
  \\
& \inference
  { \jterm{{{\Gamma}{A}}{P}}{Q}{g}
    \jfam{{{{{\Gamma}{A}}{P}}{Q}}{R}}{S}
    }
  { \jfameq
      {{{\Gamma}{\mprd{A}{P}}}{\mprd{A}{\subst{g}{R}}}}
      {\mprd{A}{\subst{g}{S}}}
      {\subst{\slam{A}{Q}{g}}{\mprd{A}{S}}}
    }
  \\
& \inference
  { \jterm{{{\Gamma}{A}}{P}}{Q}{g}
    \jterm{{{{{\Gamma}{A}}{P}}{Q}}{R}}{S}{k}
    }
  { \jtermeq
      {{{\Gamma}{\mprd{A}{P}}}{\mprd{A}{\subst{g}{R}}}}
      {\mprd{A}{\subst{g}{S}}}
      {\slam{A}{\subst{g}{S}}{\subst{g}{k}}}
      {\subst{\slam{A}{Q}{g}}{\slam{A}{S}{k}}}
    }
  \\
& \inference
  { \jterm{{{\Gamma}{A}}{P}}{Q}{g}
    \jterm{{\Gamma}{\mprd{A}{\ctxext{{P}{Q}}{R}}}}{\mprd{A}{S}}{k}
    }
  { \jtermeq
      {{{{\Gamma}{A}}{P}}{\subst{g}{R}}}
      {\subst{g}{S}}
      {\sev{A}{\subst{g}{S}}{\subst{\slam{A}{Q}{g}}{k}}}
      {\subst{g}{\sev{A}{S}{k}}}
    }
\end{align*}

\subsubsection{The dependent function constructor is compatible with the identity terms}
\begin{align*}
& \inference
  { \jfam{{{\Gamma}{A}}{P}}{Q}
    }
  { \jtermeq
      {{{\Gamma}{\mprd{A}{P}}}{\mprd{A}{Q}}}
      {\ctxwk{\mprd{A}{Q}}{\mprd{A}{Q}}}
      {\slam{A}{\ctxwk{Q}{Q}}{\idtm{Q}}}
      {\idtm{\mprd{A}{Q}}}
    }
  \\
& \inference
  { \jfam{{{\Gamma}{A}}{P}}{Q}
    }
  { \jtermeq
      {{{{\Gamma}{A}}{P}}{Q}}
      {\ctxwk{Q}{Q}}
      {\sev{A}{\ctxwk{Q}{Q}}{\idtm{\mprd{A}{Q}}}}
      {\idtm{Q}}
    }
\end{align*}

\subsection{Consequences of the rules for the dependent function constructors}

\begin{lem}\label{lem:slam-ext}
The inference rules asserting that lambda abstraction is compatible with term
extension
\begin{align*}
& \inference
  { \jterm{{{\Gamma}{A}}{P}}{{Q}{R}}{h}
    }
  { \jtermeq
      {{\Gamma}{\mprd{A}{P}}}
      {\mprd{A}{\ctxext{Q}{R}}}
      {\cprojfst{\mprd{A}{Q}}{\mprd{A}{R}}{\slam{A}{\ctxext{Q}{R}}{h}}}
      {\slam{A}{Q}{\cprojfst{Q}{R}{h}}}
    }
  \\
& \inference
  { \jterm{{{\Gamma}{A}}{P}}{{Q}{R}}{h}
    }
  { \jtermeq
      {{\Gamma}{\mprd{A}{P}}}
      {\mprd{A}{\ctxext{Q}{R}}}
      {\cprojfst{\mprd{A}{Q}}{\mprd{A}{R}}{\slam{A}{\ctxext{Q}{R}}{h}}}
      {\slam{A}{R}{\cprojsnd{Q}{R}{h}}}
    }
\intertext{and the inference rules asserting that evaluation is compatible with
term extension}
& \inference
  { \jterm{{\Gamma}{\mprd{A}{P}}}{\mprd{A}{\ctxext{Q}{R}}}{h}
    }
  { \jtermeq
      {{{\Gamma}{A}}{P}}
      {{Q}{R}}
      {\cprojfst{Q}{R}{\sev{A}{\ctxext{Q}{R}}{h}}}
      {\sev{A}{Q}{\cprojfst{\mprd{A}{Q}}{\mprd{A}{R}}{h}}}
    }
  \\
& \inference
  { \jterm{{\Gamma}{\mprd{A}{P}}}{\mprd{A}{\ctxext{Q}{R}}}{h}
    }
  { \jtermeq
      {{{\Gamma}{A}}{P}}
      {{Q}{R}}
      {\cprojsnd{Q}{R}{\sev{A}{\ctxext{Q}{R}}{h}}}
      {\sev{A}{R}{\cprojsnd{\mprd{A}{Q}}{\mprd{A}{R}}{h}}}
    }
\end{align*}
are valid.
\end{lem}

\begin{proof}
We have the judgmental equalities
\begin{align*}
\cprojfst{\mprd{A}{Q}}{\mprd{A}{R}}{\slam{A}{\ctxext{Q}{R}}{h}}
& \jdeq
  \subst{\slam{A}{\ctxext{Q}{R}}{h}}{\ctxwk{\mprd{A}{R}}{\idtm{\mprd{A}{Q}}}}
  \\
& \jdeq 
  \subst{\slam{A}{\ctxext{Q}{R}}{h}}{\ctxwk{\mprd{A}{R}}{\slam{A}{\ctxwk{Q}{Q}}{\idtm{Q}}}}
  \\
& \jdeq
  \subst{\slam{A}{\ctxext{Q}{R}}{h}}{\slam{A}{\ctxwk{R}{{Q}{Q}}}{\ctxwk{R}{\idtm{Q}}}}
  \\
& \jdeq
  \subst{\slam{A}{\ctxext{Q}{R}}{h}}{\slam{A}{\ctxwk{\ctxext{Q}{R}}{Q}}{\ctxwk{R}{\idtm{Q}}}}
  \\
& \jdeq
  \slam{A}{Q}{\subst{h}{\ctxwk{R}{\idtm{Q}}}}
  \\
& \jdeq
  \slam{A}{Q}{\cprojfst{Q}{R}{h}}
\end{align*}
\end{proof}

\begin{rmk}
It follows from these rules that on the projections (i.e.~on the fibrations),
lambda abstraction and evaluation are each other's inverses even though we
have not even postulated that evaluation is a retraction of lambda 
abstraction. Thus, in combination with univalence this should give that
they are homotopically inverse to each other.
\end{rmk}

\subsection{The evaluation term}
Until now we have just introduced the dependent function constructor, but we 
have not treated one of its main characteristic, the evaluation map. Usually
in type theory, evaluation is introduces by a rule like
\begin{equation*}
\inference
    { \jterm{\Gamma}{\mprd{A}{P}}{f}
      }
    { \jterm{{\Gamma}{A}}{P}{\mathsf{ev}(f)}
      }
\end{equation*}
or even a rule like
\begin{equation*}
\inference
    { \jterm{\Gamma}{\mprd{A}{P}}{f}
      \jterm{\Gamma}{A}{x}
      }
    { \jterm{\Gamma}{\subst{x}{P}}{\mathsf{ev}(f,x)}
      }
\end{equation*}
involving a term of $A$. We immediately discard the latter as a viable option.
In the current setting however, there is an even better option than the first. 
We need a better option because given a family 
$\jfam{{{\Gamma}{A}}{P}}{Q}$, we will need to have a family over
$\ctxext{{\Gamma}{\mprd{A}{P}}}{\ctxwk{\mprd{A}{P}}{A}}$ with fibers
$\subst{{x}{\mathsf{ev}(f)}}{{x}{Q}}$ at $f:\mprd{A}{P}$ and $x:A$. In a 
name-free setting, such a family is impossible to obtain when using evaluation
as introduced above. The solution is to introduce evaluation via the rule
\begin{equation}\label{eq:evtm}
\inference
  { \jfam{{\Gamma}{A}}{P}
    }
  { \jterm
      {{{\Gamma}{\mprd{A}{P}}}{\ctxwk{\mprd{A}{P}}{A}}}
      {\ctxwk{\mprd{A}{P}}{P}}
      {\evtm{A}{P}}
    }
\end{equation}
The advantages are as follows:
\begin{enumerate}
\item It matches the categorical notion of locally cartesian closedness, where
the unit and the counit consist of morphisms in the category.
\item We now have the family 
\begin{equation*}
\jfam
  {{{\Gamma}{\mprd{A}{P}}}{\ctxwk{\mprd{A}{P}}{A}}}
  {\subst{\evtm{A}{P}}{\ctxwk{\mprd{A}{P}}{Q}}}
\end{equation*}
Using this family, we hope to find the judgmental equality
\begin{equation*}
\jfameq
  {{\Gamma}{\mprd{A}{P}}}
  {\mprd{A}{Q}}
  {\mprd{\ctxwk{\mprd{A}{P}}{A}}{\subst{\evtm{A}{P}}{\ctxwk{\mprd{A}{P}}{Q}}}}
\end{equation*}
expressing the action on families of the dependent function constructor in terms
of its action on contexts. This is in agreement with the usual approach of
\cite{TheBook}, where no action on families of the dependent function type
constructor is needed. We will also be able to use the family
\begin{equation*}
\jfam
  {{\Gamma}{\mprd{A}{P}}}
  {{\ctxwk{\mprd{A}{P}}{A}}{\subst{\evtm{A}{P}}{\ctxwk{\mprd{A}{P}}{Q}}}}
\end{equation*}
to formulate the compatibility of the dependent function constructor with itself.
\end{enumerate}
Of course, we will also need the following convertibility rule before we 
continue:

\begin{rmk}
It is common practice, for example in \cite{TheBook}, to omit explicit notation
for evaluation. In our setting that would mean that we would denote a term 
$\subst{f}{\evtm{A}{P}}$ of $P$ in context $\ctxext{\Gamma}{A}$ by $f$.
However, since the evaluation term is set to be an actual term of the theory,
it seems to be harmful to the exposition of the theory to copy this practice.
\end{rmk}

\begin{comment}
\subsubsection{Introducing the evaluation operator}
The operation $\ctxev{A}{P}{\blank}$ brings things in context $\ctxext{{\Gamma}{A}}{P}$ to
the context $\ctxext{{\Gamma}{\mprd{A}{P}}}{\ctxwk{\mprd{A}{P}}{A}}$.
\begin{align*}
& \inference
  { \jfam{{{\Gamma}{A}}{P}}{Q}
    }
  { \jfamdefn
      {{{\Gamma}{\mprd{A}{P}}}{\ctxwk{\mprd{A}{P}}{A}}}
      {\ctxev{A}{P}{Q}}
      {\subst{\evtm{A}{P}}{\ctxwk{\mprd{A}{P}}{Q}}}
    }
  \\
& \inference
  { \jfam{{{{\Gamma}{A}}{P}}{Q}}{R}
    }
  { \jfamdefn
      {{{{\Gamma}{\mprd{A}{P}}}{\ctxwk{\mprd{A}{P}}{A}}}{\ctxev{A}{P}{Q}}}
      {\ctxev[\famsym]{A}{P}{R}}
      {\subst{\evtm{A}{P}}{\ctxwk{\mprd{A}{P}}{R}}}
    }
  \\
& \inference
  { \jterm{{{{\Gamma}{A}}{P}}{Q}}{R}{h}
    }
  { \jtermdefn
      {{{{\Gamma}{\mprd{A}{P}}}{\ctxwk{\mprd{A}{P}}{A}}}{\ctxev{A}{P}{Q}}}
      {\ctxev[\famsym]{A}{P}{R}}
      {\ctxev[\tmsym]{A}{P}{h}}
      {\subst{\evtm{A}{P}}{\ctxwk{\mprd{A}{P}}{h}}}
    }
\end{align*}
\end{comment}

\subsubsection{The compatibility rules of the evaluation terms}

\paragraph{Compatibility with the empty context}
The following rule explains that the evaluation term is judgmentally equal to
the identity term when the domain of the dependent function constructor is the
empty family over $\Gamma$.
\begin{align*}
& \inference
    { \jfam{\Gamma}{P}
      }
    { \jtermeq
        {{\Gamma}{P}}
        {\ctxwk{P}{P}}
        {\evtm{\emptyf}{P}}
        {\idtm{P}}
      }
\end{align*}

\paragraph{Currying for the evaluation term}
We have asserted that there is a judgmental equality
\begin{equation*}
\jfameq
      {\Gamma}
      {\mprd{\ctxext{A}{P}}{Q}}
      {\mprd{A}{\mprd{P}{Q}}}
\end{equation*}
for every family $Q$ over $\ctxext{{\Gamma}{A}}{P}$. It follows that we have
the judgmental equalities
\begin{align*}
& \ctxwk{\mprd{A}{\mprd{P}{Q}}}{Q}
  \\
& \jdeq
  \subst
    { \evtm{A}{\mprd{P}{Q}}}
    { \ctxwk
        {{\mprd{A}{\mprd{P}{Q}}}{\mprd{P}{Q}}}
        {{\mprd{A}{\mprd{P}{Q}}}{Q}}
      }
  \\
& \jdeq
  \subst
    { \evtm{A}{\mprd{P}{Q}}}
    { \ctxwk
        {\mprd{A}{\mprd{P}{Q}}}
        {{\mprd{P}{Q}}{Q}}
      }
\end{align*}
Since $\evtm{P}{Q}$ is a term of $\ctxwk{\mprd{P}{Q}}{Q}$ in 
context $\ctxext{{{\Gamma}{A}}{\mprd{P}{Q}}}{\ctxwk{\mprd{P}{Q}}{P}}$ we may
require the following compatibility rule:
\begin{align}
& \inference
  { \jfam{{{\Gamma}{A}}{P}}{Q}
    }
  { \begin{array}{l}
    \ctxext
      {{\Gamma}{\mprd{\ctxext{A}{P}}{Q}}}{\ctxwk{\mprd{\ctxext{A}{P}}{Q}}{\ctxext{A}{P}}}
      \\
    \jtermeq
      {\qquad}
      {\ctxwk{\mprd{\ctxext{A}{P}}{Q}}{Q}}
      {\evtm{\ctxext{A}{P}}{Q}}
      {\subst{\evtm{A}{\mprd{P}{Q}}}{\ctxwk{\mprd{A}{\mprd{P}{Q}}}{\evtm{P}{Q}}}}
    \end{array}
    }
\end{align}
To get a feel for what it says we substitute terms in both expressions involved
in the following lemma.

\begin{lem}
Consider $f:\mprd{\ctxext{A}{P}}{Q}$, $x:A$ and $u:\subst{x}{P}$ be terms in
context $\Gamma$. Then we have the judgmental equality
\begin{equation*}
\jtermeq
  {\Gamma}
  {\subst{u}{{x}{Q}}}
  {\subst{\tmext{x}{u}}{{f}{\evtm{\ctxext{A}{P}}{Q}}}}
  {\subst{u}{{{x}{{f}{\evtm{A}{\mprd{P}{Q}}}}}{\evtm{\subst{x}{P}}{\subst{x}{Q}}}}}
\end{equation*}
is valid.
\end{lem}

\paragraph{Weakening is compatible with the evaluation term}
\begin{align*}
& \inference
  { \jfam{\Gamma}{A}
    \jfam{{{\Gamma}{B}}{Q}}{R}
    }
  { \begin{array}{l}
    \ctxext
        { { { {\Gamma}
              {A}
              }
            { \ctxwk{A}{B}
              }
            }
          { \mprd{\ctxwk{A}{Q}}{\ctxwk{A}{R}}
            }
          }
        { \ctxwk{\mprd{\ctxwk{A}{Q}}{\ctxwk{A}{R}}}{\ctxwk{A}{Q}}
          }
      \\
    \jtermeq
      {\qquad}
      {\ctxwk{\mprd{\ctxwk{A}{Q}}{\ctxwk{A}{R}}}{\ctxwk{A}{R}}}
      {\evtm{\ctxwk{A}{Q}}{\ctxwk{A}{R}}}
      {\ctxwk{A}{\evtm{Q}{R}}}  
    \end{array}
    }
\end{align*}

\paragraph{Substitution is compatible with the evaluation term}
\begin{align*}
& \inference
  { \jterm{\Gamma}{A}{x}
    \jfam{{{{\Gamma}{A}}{P}}{Q}}{R}
    }
  { \begin{array}{l}
    \ctxext
      {{{\Gamma}{\subst{x}{P}}}{\mprd{\subst{x}{Q}}{\subst{x}{R}}}}
      {\ctxwk{\mprd{\subst{x}{Q}}{\subst{x}{R}}}{\subst{x}{Q}}}
      \\
    \jtermeq
      {\qquad}
      {\ctxwk{\mprd{\subst{x}{Q}}{\subst{x}{R}}}{\subst{x}{R}}}
      {\subst{x}{\evtm{Q}{R}}}
      {\evtm{\subst{x}{Q}}{\subst{x}{R}}}
    \end{array}
    }  
\end{align*}

\paragraph{The evaluation term is compatible with weakening}
\begin{align*}
& \inference
  { \jfam{{{\Gamma}{A}}{P}}{Q}
    \jfam{{{{\Gamma}{P}}{Q}}{R}}{S}
    \jterm{{{\Gamma}{\mprd{A}{P}}}{\mprd{A}{R}}}{\mprd{A}{S}}{k}
    }
  { \jtermeq
      {{{{{\Gamma}{A}}{P}}{Q}}{\ctxwk{Q}{R}}}
      {\ctxwk{Q}{S}}
      {\sev{A}{\ctxwk{Q}{S}}{\ctxwk{\mprd{A}{Q}}{k}}}
      {\ctxwk{Q}{\sev{A}{S}{k}}}
    }
\end{align*}

\subsection{Composition of dependent functions}


%\section{Universes}\label{sec:universes}

\subsection{Generating E-algebras}

\begin{defn}
Consider a formal triple $A,P,T$ in context $\Gamma$ consisting of
$\jfam{\Gamma}{A}$, $\jfam{{\Gamma}{A}}{P}$ and $\jfam{{{\Gamma}{A}}{P}}{T}$.
Then we form the \emph{E-algebra $\genealg{A}{P}{T}$
in context $\Gamma$ generated by $A,T,P$} to be an E-algebra in context
$\Gamma$. We introduce 
\begin{equation*}
\begin{tikzcd}
T 
  \ar{r}{\cfthomt{\genealgincl{A}{P}{T}}}
  \ar[fib]{d}
& \genealgt{A}{P}{T}
  \ar[fib]{d}
  \\
P 
  \ar{r}{\cfthomf{\genealgincl{A}{P}{T}}}
  \ar[fib]{d}
& \genealgf{A}{P}{T}
  \ar[fib]{d}
  \\
A 
  \ar{r}{\cfthomc{\genealgincl{A}{P}{T}}}
& \genealgc{A}{P}{T}
\end{tikzcd}
\end{equation*}
\end{defn}

\subsection{The universe operator}\label{universes}

\begin{defn}
Let $\jfam{\Gamma}{A}$ and $\jfam{{\Gamma}{A}}{P}$. Then we define $\mathbf{U}(A,P)$ to consist of $\jfam{\Gamma}{\UU(A,P)}$ and $\jfam{{\Gamma}{\UU(A,P)}}{\tilde\UU(A,P)}$ for which there is an inclusion
\begin{equation*}
\begin{tikzcd}
P 
  \ar[fib]{d}
  \ar{r}
& \tilde\UU(A,P)
  \ar[fib]{d}
  \\
A \ar{r}
& \UU(A,P)
\end{tikzcd}
\end{equation*}
such that
\begin{equation*}
\begin{tikzcd}
\mprd{\tilde\UU(A,P)}{\ctxwk{\tilde\UU(A,P)}{{\UU(A,P)}{\tilde\UU(A,P)}}}
  \ar[fib]{d}
  \\
\mprd{\tilde\UU(A,P)}{\ctxwk{\tilde\UU(A,P)}{{\UU(A,P)}{\UU(A,P)}}}
  \ar[fib]{d}
  \\
\UU(A,P)
\end{tikzcd}
\end{equation*}
is an E-algebra
\end{defn}


%\section{Inductive constructions}

\subsection{Strong inductive morphisms}
In this subsection we investigate the notion of inductive morphisms. An 
inductive morphism $f$ from $A$ to $B$ in context $\Gamma$ is a morphism which
induces an operation which is judgmentally the inverse of composition
with $f$. We explore a notion of inductive morphism which is much stronger
than the usual notion: it pushes families over $A$ to families over $B$,
families over families over $A$ to families over families over $B$ and terms
thereof to terms of the output families over families in a manner compatible
with the empty family, extension, weakening, substitution and identity terms.
As a result, inductive morphisms will be stable under extension, weakening,
substitution and the identity term is an inductive morphism.

Inductive morphisms are useful to for inductive types which are defined using
only one (ordinary morphism as) constructor, like the unit type and dependent 
pair types.
They can't be used to define the empty type, disjoint sums,
the natural numbers or identity types. 

{\color{red} Maybe
we can solve this by writing down a type theory of inductive constructions.}

\begin{defn}
Let $\jhom{\gamma}{A}{B}{f}$ be a context morphism. We say that $f$ is an
\emph{inductive morphism} if there is an operation $\tfindf{f}$ given by
\begin{align*}
& \inference
  { \jfam{{\Gamma}{A}}{P}
    }
  { \jfam{{\Gamma}{B}}{\tfind{f}{P}}
    }
  \\
& \inference
  { \jfam{{{\Gamma}{A}}{P}}{Q}
    }
  { \jfam{{{\Gamma}{B}}{\tfind{f}{P}}}{\tfind[\famsym]{f}{Q}}
    }
  \\
& \inference
  { \jterm{{{\Gamma}{A}}{P}}{Q}{g}
    }
  { \jterm{{{\Gamma}{B}}{\tfind{f}{P}}}{\tfind[\famsym]{f}{Q}}{\tfind[\tmsym]{f}{g}}
    }
\end{align*}
for which the inference rules in the following list are valid:
\begin{enumerate}
\item The operation $\tfindf{f}$ is compatible with the empty context:
\begin{align*}
& \inference
  { 
    }
  { \jfameq{{\Gamma}{B}}{\tfind{f}{\emptyf}}{\emptyf}
    }
  \\
& \inference
  { \jfam{{\Gamma}{A}}{P}
    }
  { \jfameq{{{\Gamma}{B}}{\tfind{f}{P}}}{\tfind[\famsym]{f}{\emptyf}}{\emptyf}
    }
\end{align*}
\item The action on families $\tfindf[\famsym]{f}$ of $\tfindf{f}$ is compatible
with the action on contexts:
\begin{equation*}
\inference
  { \jfam{{\Gamma}{A}}{P}
    }
  { \jfameq{{\Gamma}{B}}{\tfind[\famsym]{f}{P}}{\tfind{f}{P}}
    }
\end{equation*}
Because of this inference rule we shall henceforth omit the annotations
$\famsym$ and $\tmsym$ from the operation $\tfindf{f}$ as usual.
\item The operation $\tfindf{f}$ is compatible with extension:
\begin{align*}
& \inference
  { \jfam{{{\Gamma}{A}}{P}}{Q}
    }
  { \jfameq
      {{\Gamma}{B}}
      {\tfind{f}{\ctxext{P}{Q}}}
      {\ctxext{\tfind{f}{P}}{\tfind{f}{Q}}}
    }
  \\
& \inference
  { \jfam{{{{\Gamma}{A}}{P}}{Q}}{R}
    }
  { \jfameq
      {{{\Gamma}{B}}{\tfind{f}{P}}}
      {\tfind{f}{\ctxext{Q}{R}}}
      {\ctxext{\tfind{f}{Q}}{\tfind{f}{R}}}
    }
\end{align*}
\item The operation $\tfindf{f}$ is compatible with weakening:
\begin{align*}
& \inference
  { \jfam{{\Gamma}{A}}{P}
    \jfam{{\Gamma}{A}}{Q}
    }
  { \jfameq
      {{{\Gamma}{B}}{\tfind{f}{P}}}
      {\tfind{f}{\ctxwk{P}{Q}}}
      {\ctxwk{\tfind{f}{P}}{\tfind{f}{Q}}}
    }
  \\
& \inference
  { \jfam{{\Gamma}{A}}{P}
    \jfam{{{\Gamma}{A}}{Q}}{R}
    }
  { \jfameq
      {{{{\Gamma}{B}}{\tfind{f}{P}}}{\ctxwk{\tfind{f}{P}}{\tfind{f}{Q}}}}
      {\tfind{f}{\ctxwk{P}{R}}}
      {\ctxwk{\tfind{f}{P}}{\tfind{f}{R}}}
    }
  \\
& \inference
  { \jfam{{\Gamma}{A}}{P}
    \jterm{{{\Gamma}{A}}{Q}}{R}{h}
    }
  { \jtermeq
      {{{{\Gamma}{B}}{\tfind{f}{P}}}{\ctxwk{\tfind{f}{P}}{\tfind{f}{Q}}}}
      {\ctxwk{\tfind{f}{P}}{\tfind{f}{R}}}
      {\tfind{f}{\ctxwk{P}{h}}}
      {\ctxwk{\tfind{f}{P}}{\tfind{f}{h}}}
    }
\end{align*}
\item We will also require that the operation $\tfindf{f}$ is compatible with
weakening by $A$:
\begin{align*}
& \inference
  { \jfam{\Gamma}{X}
    }
  { \jfameq
      {{\Gamma}{B}}
      {\tfind{f}{\ctxwk{A}{X}}}
      {\ctxwk{B}{X}}
    }
  \\
& \inference
  { \jfam{{\Gamma}{X}}{Y}
    }
  { \jfameq
      {{{\Gamma}{B}}{\ctxwk{B}{X}}}
      {\tfind{f}{\ctxwk{A}{Y}}}
      {\ctxwk{B}{Y}}
    }
  \\
& \inference
  { \jterm{{\Gamma}{X}}{Y}{y}
    }
  { \jtermeq
      {{{\Gamma}{B}}{\ctxwk{B}{X}}}
      {\ctxwk{B}{Y}}
      {\tfind{f}{\ctxwk{A}{y}}}
      {\ctxwk{B}{y}}
    }
\end{align*}
These rules assert that constant families and terms are mapped to constant
families and terms.
\item The operation $\tfindf{f}$ is compatible with substitution:
\begin{align*}
& \inference
  { \jterm{{{\Gamma}{A}}{P}}{Q}{g}
    \jfam{{{{\Gamma}{A}}{P}}{Q}}{R}
    }
  { \jfameq
      {{{\Gamma}{B}}{\tfind{f}{P}}}
      {\tfind{f}{\subst{g}{R}}}
      {\subst{\tfind{f}{g}}{\tfind{f}{R}}}
    }
  \\
& \inference
  { \jterm{{{\Gamma}{A}}{P}}{Q}{g}
    \jfam{{{{{\Gamma}{A}}{P}}{Q}}{R}}{S}
    }
  { \jfameq
      {{{{\Gamma}{B}}{\tfind{f}{P}}}{\subst{\tfind{f}{g}}{\tfind{f}{R}}}}
      {\tfind{f}{\subst{g}{S}}}
      {\subst{\tfind{f}{g}}{\tfind{f}{S}}}
    }
  \\
& \inference
  { \jterm{{{\Gamma}{A}}{P}}{Q}{g}
    \jterm{{{{{\Gamma}{A}}{P}}{Q}}{R}}{S}{k}
    }
  { \jtermeq
      {{{{\Gamma}{B}}{\tfind{f}{P}}}{\subst{\tfind{f}{g}}{\tfind{f}{R}}}}
      {\subst{\tfind{f}{g}}{\tfind{f}{S}}}
      {\tfind{f}{\subst{g}{k}}}
      {\subst{\tfind{f}{g}}{\tfind{f}{k}}}
    }
\end{align*}
\item The operation $\tfindf{f}$ is compatible with the identity terms:
\begin{equation*}
\inference
  { \jfam{{{\Gamma}{A}}{P}}{Q}
    }
  { \jtermeq
      {{{{\Gamma}{B}}{\tfind{f}{P}}}{\tfind{f}{Q}}}
      {\ctxwk{\tfind{f}{Q}}{\tfind{f}{Q}}}
      {\tfind{f}{\idtm{Q}}}
      {\idtm{\tfind{f}{Q}}}
    }
\end{equation*}
\item We will also require that $\tfindf{f}$ is compatible with $f$ itself:
\begin{equation*}
\inference
  {
    }
  { \jtermeq
      {{\Gamma}{B}}
      {\ctxwk{B}{B}}
      {\tfind{f}{f}}
      {\idtm{B}}
    }
\end{equation*}
\item Finally, we require that $\tfindf{f}$ is the right inverse of composition
with $f$:
\begin{align*}
& \inference
  { \jfam{{\Gamma}{A}}{P}
    }
  { \jfameq
      {{\Gamma}{A}}
      {\jcomp{}{f}{\tfind{f}{P}}}
      {P}
    }
  \\
& \inference
  { \jfam{{{\Gamma}{A}}{P}}{Q}
    }
  { \jfameq
      {{{\Gamma}{A}}{P}}
      {\jcomp{}{f}{\tfind{f}{Q}}}
      {Q}
    }
  \\
& \inference
  { \jterm{{{\Gamma}{A}}{P}}{Q}{g}
    }
  { \jtermeq
      {{{\Gamma}{A}}{P}}
      {Q}
      {\jcomp{}{f}{\tfind{f}{g}}}
      {g}
    }
\end{align*}
\end{enumerate}
\end{defn}

\begin{rmk}
The rules expressing that $\tfindf{f}$ is a right inverse to composition with
$f$ are usually called the `computation rules' of the induction principle.

Recall that we had announced that $\tfindf{f}$ would be an actual inverse of
composition with $f$, while we only have stated explicitly that $\tfindf{f}$
is a right inverse. We get the fact that it is also a left inverse from the
other compatibility rules. For example: given $\jfam{{\Gamma}{B}}{Q}$ we get
\begin{equation*}
\tfind{f}{\jcomp{}{f}{Q}}
  \jdeq
  \tfind{f}{\subst{f}{\ctxwk{A}{Q}}}
  \jdeq
  \subst{\tfind{f}{f}}{\tfind{f}{\ctxwk{A}{Q}}}
  \jdeq
  \subst{\idtm{B}}{\ctxwk{B}{Q}}
  \jdeq
  Q.
\end{equation*}
We thus recover the usual sort of induction principle. When $Q$ is a family
over $\ctxext{\Gamma}{B}$, all we have to do to find a term of $Q$ is to find
a term $g$ of $\jcomp{}{f}{Q}$. The result of applying $\tfindf{f}$ to $g$
will be a term of $Q$.
\end{rmk}

\begin{rmk}
These stronger rules also seem to imply that not every equivalence is going
to be an inductive morphism (when we add all the type constructors to the
theory). For instance, the interval is equivalent to the unit type. If the
unit type is defined via an inductive morphism $\emptyc\to\unit$ we get that
every family over $\unit$ is definitionally a constant type because
every family over $\emptyc$ is a weakening by the empty family. If the
equivalence from $\unit$ to the interval were inductive, this would in turn
imply that every type family over the interval is constant. However, this is
not the case because we have the family which has the unit type as a fiber
above one endpoint and the interval above the other.
\end{rmk}

\begin{lem}
The identity term
$\jhom{\Gamma}{A}{A}{\idtm{A}}$ is an inductive morphism
for each family $A$ of contexts over $\Gamma$
\end{lem}

\begin{proof}
Composition with the identity morphism is an identity operation.
\end{proof}

\begin{itemize}
\item Extensions of inductive morphisms are inductive
\item Weakenings of inductive morphisms are inductive
\item Substitutions of inductive morphisms are inductive
\end{itemize}


%\part{Categorical semantics}

%\section{E-objects in categories with finite limits}
In this section we assume that $\cat{C}$ is a finitely complete category and
whenever we write a pullback, we assume that it is chosen. Recall that for
any morphism $f:A\to B$ in a category $\cat{C}$ with chosen pullbacks, there
is a functor
\begin{equation*}
f^\ast : \cat{C}/B\to\cat{C}/A.
\end{equation*}
As usual, when $g:X\to B$ is a morphism, we will write $f^\ast(X)$ for the
domain of $f^\ast(g)$. When there is more than one morphism $X\to B$ involved,
as will be the case below, we will write $\pullback{A}{X}{f}{g}$. The projections
will be written as $\pullbackpr{1}{f}{g}$ and $\pullbackpr{2}{f}{g}$. So in this notation, a
typical pullback diagram has the following form:
\begin{equation*}
\begin{tikzcd}[column sep=large]
\pullback{A}{X}{f}{g}
  \ar{r}{\pullbackpr{1}{f}{g}}
  \ar{d}[swap]{\pullbackpr{2}{f}{g}}
  &
A \ar{d}{f}
  \\
X \ar{r}[swap]{g}
  &
B
\end{tikzcd}
\end{equation*}
Also, when we have a commutative diagram of the form
\begin{equation*}
\begin{tikzcd}
A \ar{r}{f}
  \ar{d}{a}
  &
X \ar{d}
  & 
B \ar{l}[swap]{g}
  \ar{d}{b}
  \\
A'
  \ar{r}[swap]{f'}
  &
X'
  &
B'
  \ar{l}{g'}
\end{tikzcd}
\end{equation*}
we will denote the unique map from $\pullback{A}{B}{f}{g}$ to $\pullback{A'}{B'}{f'}{g'}$
such that the diagram
\begin{equation*}
\begin{tikzcd}
  {}
  & 
\pullback{A'}{B'}{f'}{g'}
  \ar{dd}
  \ar{rr}
  &
  &
B'
  \ar{dd}{g'}
  \\
\pullback{A}{B}{f}{g}
  \ar{dd}
  \ar[crossing over]{rr}
  \ar[dotted]{ur}{\pullback{a}{b}{f'}{g'}}
  &
  &
B \ar{ur}{b}
  \\
  {}
  &
A'
  \ar{rr}
  &
  &
X'
  \\
A \ar{rr}[swap]{f}
  \ar{ur}{a}
  &
  &
X \ar[crossing over,leftarrow]{uu}[near end,swap]{g}
  \ar{ur}
\end{tikzcd}
\end{equation*}
commutes, by $\pullback{a}{b}{f'}{g'}$. In the current work, we shall
write $A\times B$ for the pullback of $A\rightarrow 1\leftarrow B$, and
$\pi_1$ and $\pi_2$ for its projections (thus, no separate choice of
cartesian products is made).


\subsection{Extension objects}
\begin{defn}
A \emph{pre-extension object $\stesys$ in $\cat{C}$} consists of a \emph{fundamental structure}, which is a diagram of the form
\begin{equation*}
\begin{tikzcd}
\stesyst
  \ar{d}[swap]{\ebd}
  \\
\stesysf
  \ar{d}[swap]{\eft}
  \\
\stesysc
\end{tikzcd}
\end{equation*}
in $\cat{C}$ together with the \emph{context extension} and \emph{family extension} operations
\begin{align*}
\ectxext &:\stesysf\to \stesysc\\
\efamext & :\stesysff\to \stesysf,
\end{align*}
respectively, such that the diagram
\begin{equation*}
\begin{tikzcd}
\stesysf_2 
  \ar{r}{\efamext} 
  \ar{d}[swap]{\eft[1]} 
  & 
\stesysf 
  \ar{d}{\eft}
  \\
\stesysf
  \ar{r}[swap]{\eft} 
  & 
\stesysc
\end{tikzcd}
\end{equation*}
commutes.
\end{defn}

\begin{defn}
We introduce the following notation:
\begin{align*}
\stesysf_2 
  & := \stesysff
  \\
\eft[1] 
  & := \pullbackpr{1}{\ectxext}{\eft} : \stesysf_2\to\stesysf
  \\
\stesysf_3 & := \pullback{\stesysf_2}{\stesysf_2}{\efamext}{\eft[1]}
  \\
\eft[2]
  & := \pullbackpr{1}{\efamext}{\eft[1]} : \stesysf_3\to\stesysf_2.
\end{align*}
Then it follows that the outer square in the diagram
\begin{equation*}
\begin{tikzcd}[column sep=large]
\stesysf_3
  \ar[dotted]{dr}{\eext{2}}
  \ar{rr}{\pullback{\pullbackpr{2}{\ectxext}{\eft}}{\pullbackpr{2}{\ectxext}{\eft}}{\ectxext}{\eft}}
  \ar{dd}[swap]{\eft[2]}
  & 
  &
\stesysf_2
  \ar{d}{\efamext}
  \\
  &
\stesysf_2
  \ar{d}[swap]{\eft[1]}
  \ar{r}{\pullbackpr{2}{\ectxext}{\eft}}
  &
\stesysf
  \ar{d}{\eft}
  \\
\stesysf_2
  \ar{r}[swap]{\eft[1]}
  &
\stesysf
  \ar{r}[swap]{\ectxext}
  &
\stesysc
\end{tikzcd}
\end{equation*}
commutes. We define $\eext{2}$ to be the unique morphism rendering the above diagram
commutative. Now we may continue to define
\begin{align*}
\stesysf_4 
  & := 
\pullback{\stesysf_3}{\stesysf_3}{\eext{2}}{\eft[2]}
  \\
\eft[3] 
  & := 
\pullbackpr{1}{\eext{2}}{\eft[2]}.
\end{align*}
Then we see that the outer square of the diagram
\begin{equation*}
\begin{tikzcd}[column sep=large]
\stesysf_4
  \ar[dotted]{dr}{\eext{3}}
  \ar{rr}{\pullback{\pullbackpr{2}{\efamext}{\eft[1]}}{\pullbackpr{2}{\efamext}{\eft[1]}}{\efamext}{\eft[1]}}
  \ar{dd}[swap]{\eft[3]}
  & 
  &
\stesysf_3
  \ar{d}{\eext{2}}
  \\
  &
\stesysf_3
  \ar{d}[swap]{\eft[2]}
  \ar{r}{\pullbackpr{2}{\efamext}{\eft[1]}}
  &
\stesysf_2
  \ar{d}{\eft[1]}
  \\
\stesysf_3
  \ar{r}[swap]{\eft[2]}
  &
\stesysf_2
  \ar{r}[swap]{\efamext}
  &
\stesysf
\end{tikzcd}
\end{equation*}
commutes,
so we may define $\eext{3}$ to be the unique map which renders the diagram. It
is straightforward to continue this process by induction, but we shall need not
go any further in this article.
\end{defn}

\begin{defn} An extension object is a pre-extension object $\stesys$ for which 
the diagrams
\begin{equation*}
\begin{tikzcd}
\stesysf_2 
  \ar{d}[swap]{\pullbackpr{2}{\ectxext}{\eft}} 
  \ar{r}{\efamext} 
  & 
\stesysf 
  \ar{d}{\ectxext}
  \\
\stesysf 
  \ar{r}[swap]{\ectxext} 
  & 
\stesysc
\end{tikzcd}
\qquad
\begin{tikzcd}
\stesysf_3
  \ar{d}[swap]{\pullbackpr{2}{\efamext}{\eft[1]}}
  \ar{r}{\eext{2}}
  & 
\stesysf_2 
  \ar{d}{\efamext} 
  \\
\stesysf_2 
  \ar{r}[swap]{\efamext} 
  &
\stesysf
\end{tikzcd}
\end{equation*}
commute.
\end{defn}

\begin{comment}
\begin{lem}
There exists an isomorphism $\alpha$ such that the triangle
\begin{equation*}
\begin{tikzcd}[column sep=tiny]
\pullback{\stesysf}{\stesysf_2}{\ectxext}{\eft\circ\eft[1]}
  \ar[dotted]{rr}{\alpha}
  \ar{dr}[swap]{\pullback{\catid{\stesysf}}{\efamext}{\ectxext}{\eft}}
  &
  &
\stesysf_3
  \ar{dl}{\eext{2}}
  \\
& \stesysf_2
\end{tikzcd}
\end{equation*}
commutes
\end{lem}

\begin{proof}
There is a unique morphism $\alpha:
\pullback{\stesysf}{\stesysf_2}{\ectxext}{\eft\circ\eft[1]}\to\stesysf_3$
rendering the diagram
\begin{equation*}
\begin{tikzcd}[column sep=large]
\pullback{\stesysf}{\stesysf_2}{\ectxext}{\eft\circ\eft[1]}
  \ar[bend left=10,yshift=.5ex]{drrr}{\pullbackpr{2}{\ectxext}{\eft}\circ\pullbackpr{2}{\ectxext}{\eft\circ\eft[1]}}
  \ar[bend right=10]{ddr}[swap]{\pullback{\catid{\stesysf}}{\eft[1]}{\ectxext}{\eft}}
  \ar[dotted]{dr}{\alpha}
  \\
& \stesysf_3
  \ar{r}{\pullbackpr{2}{\efamext}{\eft[1]}}
  \ar{d}{\eft[2]}
  &
\stesysf_2
  \ar{d}[swap]{\eft[1]}
  \ar{r}[swap]{\pullbackpr{2}{\ectxext}{\eft}}
  &
\stesysf
  \ar{d}{\eft}
  \\
{} & \stesysf_2
  \ar{r}[swap]{\efamext}
  &
\stesysf
  \ar{r}[swap]{\ectxext}
  &
\stesysc
\end{tikzcd}
\end{equation*}
\end{proof}
\end{comment}

\begin{defn}
Suppose $\stesys$ is a pre-extension object of $\cat{C}$. Then we define the pre-extension object
$\famesys{\stesys}$ to consist of the fundamental structure
\begin{equation*}
\begin{tikzcd}
\stesyst_2
  \ar{d}{\ebd[1]}
  \\
\stesysf_2
  \ar{d}{\eft[1]}
  \\
\stesysf
\end{tikzcd}
\end{equation*}
where
\begin{align*}
\stesyst_2 
  & := \pullback{\stesysf}{\stesyst}{\ectxext}{\eft\circ\ebd}
  \\
\ebd[1]
  & := \ectxext^\ast(\ebd),
\end{align*}
with the extension operations
\begin{align*}
\efamext 
  & 
  : \stesysf_2\to\stesysf\\
\eext{2} & : \stesysf_3\to\stesysf_2.
\end{align*}
\end{defn}

In \autoref{famextobj} below we shall show that $\famesys{\stesys}$ is an
extension algebra whenever $\stesys$ is an extension algebra. We shall need
a handful of lemmas to give the proof.

\begin{defn}
We define
\begin{align*}
\beta_1 
  & := 
\pullbackpr{2}{\ectxext}{\eft}
  & &
  : \stesysf_2\to\stesysf
  \\
\beta_2
  & :=
\pullback{\beta_1}{\beta_1}{\ectxext}{\eft}
  & &
  : \stesysf_3\to\stesysf_2
  \\
\beta_3
  & :=
\pullback{\beta_2}{\beta_2}{\efamext}{\eft[1]}
  & &
  : \stesysf_4\to\stesysf_3.
\end{align*}
\end{defn}

\begin{lem}
Let $\stesys$ be a pre-extension object. Then the square
\begin{equation*}
\begin{tikzcd}[column sep=10em]
\stesysf_4
  \ar{r}{\pullback{\pullbackpr{2}{\efamext}{\eft[1]}}{\pullbackpr{2}{\efamext}{\eft[1]}}{\efamext}{\eft[1]}}
  \ar{d}[swap]{\beta_3}
  &
\stesysf_3
  \ar{d}{\beta_2}
  \\
\stesysf_3
  \ar{r}[swap]{\beta_2}
  &
\stesysf_2
\end{tikzcd}
\end{equation*}
commutes.
\end{lem}

\begin{proof}
Left to the reader.
\end{proof}

\begin{comment}
\begin{proof}
It is straightforward to verify the equalities
\begin{align*}
\pullbackpr{1}{\ectxext}{\eft}\circ\beta_2\circ
  (\pullback{\pullbackpr{2}{\efamext}{\eft[1]}}{\pullbackpr{2}{\efamext}{\eft[1]}}{\efamext}{\eft[1]})
  & =
\pullbackpr{1}{\ectxext}{\eft}\circ\beta_2\circ\beta_3
  \\
\pullbackpr{2}{\ectxext}{\eft}\circ\beta_2\circ
  (\pullback{\pullbackpr{2}{\efamext}{\eft[1]}}{\pullbackpr{2}{\efamext}{\eft[1]}}{\efamext}{\eft[1]})
  & =
\pullbackpr{2}{\ectxext}{\eft}\circ\beta_2\circ\beta_3.\qedhere
\end{align*}
\end{proof}
\end{comment}

Note that the square
\begin{equation*}
\begin{tikzcd}
\stesysf_3
  \ar{r}{\beta_2}
  \ar{d}[swap]{\eext{2}}
  &
\stesysf_2
  \ar{d}{\efamext}
  \\
\stesysf_2
  \ar{r}[swap]{\beta_1}
  &
\stesysf
\end{tikzcd}
\end{equation*}
commutes by definition. We have a similar result relating $\eext{3}$ and
$\eext{2}$.

\begin{lem}
Let $\stesys$ be a pre-extension object. Then the square
\begin{equation*}
\begin{tikzcd}
\stesysf_4
  \ar{r}{\beta_3}
  \ar{d}[swap]{\eext{3}}
  &
\stesysf_3
  \ar{d}{\eext{2}}
  \\
\stesysf_3
  \ar{r}[swap]{\beta_2}
  &
\stesysf_2
\end{tikzcd}
\end{equation*}
commutes.
\end{lem}

\begin{proof}
Left to the reader.
\end{proof}
\begin{comment}
\begin{proof}
It is straightforward to verify the equalities
\begin{align*}
\pullbackpr{1}{\ectxext}{\eft}\circ\beta_2\circ\eext{3}
  & = \beta_1\circ\eft[2]\circ\eft[3]
  \\
\pullbackpr{1}{\ectxext}{\eft}\circ\eext{2}\circ\beta_3
  & = \beta_1\circ\eft[2]\circ\eft[3].
\end{align*}
Thus, it remains to verify that
\begin{equation*}
\pullbackpr{2}{\ectxext}{\eft}\circ\beta_2\circ\eext{3}
  = \pullbackpr{2}{\ectxext}{\eft}\circ\eext{2}\circ\beta_3.
\end{equation*}
It is straightforward to see that the diagram
\begin{equation*}
\begin{tikzcd}[column sep=large]
\stesysf_4
  \ar{dd}[swap]{\pullback{\pullbackpr{2}{\efamext}{\eft[1]}}{\pullbackpr{2}{\efamext}{\eft[1]}}{\efamext}{\eft[1]}}
  \ar{r}{\eext{3}}
  &
\stesysf_3
  \ar{r}{\beta_2}
  \ar{d}[swap]{\pullbackpr{2}{\efamext}{\eft[1]}}
  &
\stesysf_2
  \ar{dd}{\pullbackpr{2}{\ectxext}{\eft}}
  \\
  {} &
\stesysf_2
  \ar{dr}{\beta_1}
  \\
\stesysf_3
  \ar{ur}{\eext{2}}
  \ar{r}[swap]{\beta_2}
  &
\stesysf_2
  \ar{r}[swap]{\efamext}
  &
\stesysf
\end{tikzcd}
\end{equation*}
commutes. It is likewise straightforward to see that the diagram
\begin{equation*}
\begin{tikzcd}
\stesysf_4
  \ar{r}{\beta_3}
  \ar{d}[swap]{\pullback{\pullbackpr{2}{\efamext}{\eft[1]}}{\pullbackpr{2}{\efamext}{\eft[1]}}{\efamext}{\eft[1]}}
  &
\stesysf_3
  \ar{r}{\eext{2}}
  \ar{d}[swap]{\beta_2}
  &
\stesysf_2
  \ar{d}{\pullbackpr{2}{\ectxext}{\eft}}
  \\
\stesysf_3
  \ar{r}[swap]{\beta_2}
  &
\stesysf_2
  \ar{r}[swap]{\efamext}
  &
\stesysf
\end{tikzcd}
\end{equation*}
commutes, completing our goal.
\end{proof}
\end{comment}

\begin{thm}[Local extension structure]\label{famextobj}
If $\stesys$ is an extension object, then so is $\famesys{\stesys}$.
\end{thm}

\begin{proof}
Note that the diagram
\begin{equation*}
\begin{tikzcd}
\stesysf_3
  \ar{d}[swap]{\pullbackpr{2}{\efamext}{\eft[1]}}
  \ar{r}{\eext{2}}
  & 
\stesysf_2 
  \ar{d}{\efamext} 
  \\
\stesysf_2 
  \ar{r}[swap]{\efamext} 
  &
\stesysf
\end{tikzcd}
\end{equation*}
commutes by assumption. For the second condition, we have to show that the
diagram
\begin{equation*}
\begin{tikzcd}
\stesysf_4
  \ar{d}[swap]{\pullbackpr{2}{\eext{2}}{\eft[2]}}
  \ar{r}{\eext{3}}
  & 
\stesysf_3
  \ar{d}{\eext{2}} 
  \\
\stesysf_3
  \ar{r}[swap]{\eext{2}} 
  &
\stesysf_2
\end{tikzcd}
\end{equation*}
Since this is a question about two maps into a pullback, it suffices to verify
that
\begin{align*}
\pullbackpr{1}{\ectxext}{\eft}\circ\eext{2}\circ\eext{3}
  & =
\pullbackpr{1}{\ectxext}{\eft}\circ\eext{2}\circ\pullbackpr{2}{\eext{2}}{\eft[2]}
  \\
\pullbackpr{2}{\ectxext}{\eft}\circ\eext{2}\circ\eext{3}
  & =
\pullbackpr{2}{\ectxext}{\eft}\circ\eext{2}\circ\pullbackpr{2}{\eext{2}}{\eft[2]}.
\end{align*}
For the first equality, it is fairly straightforward to show that both the
equalities
\begin{equation*}
\pullbackpr{1}{\ectxext}{\eft}\circ\eext{2}\circ\eext{3}
  =
\eft[1]\circ\eft[2]\circ\eft[3]
\end{equation*}
and
\begin{equation*}
\pullbackpr{1}{\ectxext}{\eft}\circ\eext{2}\circ\pullbackpr{2}{\eext{2}}{\eft[2]}
  =
\eft[1]\circ\eft[2]\circ\eft[3].
\end{equation*}
hold. For the second subgoal (which is more tricky). Notice first that the
diagram
\begin{equation*}
\begin{tikzcd}
\stesysf_4
  \ar{r}{\eext{3}}
  \ar{d}[swap]{\beta_3}
  &
\stesysf_3
  \ar{r}{\eext{2}}
  \ar{d}[swap]{\beta_2}
  &
\stesysf_2
  \ar{d}{\pullbackpr{2}{\ectxext}{\eft}}
  \\
\stesysf_3
  \ar{r}[swap]{\eext{2}}
  &
\stesysf_2
  \ar{r}[swap]{\efamext}
  &
\stesysf
\end{tikzcd}
\end{equation*}
commutes. We also have the commutative diagram
\begin{equation*}
\begin{tikzcd}[column sep=large]
\stesysf_4
  \ar{r}{\pullbackpr{2}{\eext{2}}{\eft[2]}}
  \ar{d}[swap]{\beta_3}
  &
\stesysf_3
  \ar{r}{\eext{2}}
  \ar{d}[swap]{\beta_2}
  &
\stesysf_2
  \ar{d}{\pullbackpr{2}{\ectxext}{\eft}}
  \\
\stesysf_3
  \ar{r}{\pullbackpr{2}{\efamext}{\eft[1]}}
  \ar{dr}[swap]{\eext{2}}
  &
\stesysf_2
  \ar{r}{\efamext}
  &
\stesysf
  \\
  {} &
\stesysf_2
  \ar{ur}[swap]{\efamext}
\end{tikzcd}
\end{equation*}
completing the proof.
\end{proof}

\subsection{(Pre-)extension homomorphisms}\label{subsection:e_extension_homomorphisms}
In this subsection we start with the study of pre-extension homomorphisms, which
will include the extension homomorphisms since they will be the pre-extension
homomorphisms of which both the domain and codomain are extension objects.
Our main examples of extension homomorphisms will be the operations of weakening
and substitution. There are some basic examples of pre-extension homomorphisms
that will be useful too, which get introduced in the this section and in
\autoref{subsection:change_of_base}. In this section, we will mainly be
interested in pre-extension homomorphisms between local pre-extension objects.
We will end this section by proving that a retract of an extension object is
always an extension object.

\begin{defn}
Let $\stesys$ and $\stesys'$ be pre-extension algebras. A \emph{pre-extension 
homomorphism $f$ from $\stesys$ to $\stesys'$} is a triple $(f_0,f_1,f^t)$ 
consisting of morphisms
\begin{equation*}
\begin{tikzcd}
\stesyst 
  \ar{r}{f^t}
  \ar{d}[swap]{\ebd}
  &
\stesyst'
  \ar{d}{\ebd'}
  \\
\stesysf 
  \ar{r}{f_1}
  \ar{d}[swap]{\eft}
  &
\stesysf'
  \ar{d}{\eft'}
  \\
\stesysc 
  \ar{r}[swap]{f_0}
  &
\stesysc'
\end{tikzcd}
\end{equation*}
such that the indicated squares commute, for which furthermore the squares
\begin{equation*}
\begin{tikzcd}
\stesysf \ar{r}{f_1}
  \ar{d}[swap]{\ectxext}
  &
\stesysf'
  \ar{d}{\ectxext'}
  \\
\stesysc
  \ar{r}[swap]{f_0}
  &
\stesysc'
\end{tikzcd}
\end{equation*}
and
\begin{equation*}
\begin{tikzcd}[column sep=large]
\stesysf\times_{\ectxext,\eft} \stesysf
  \ar{r}{f_1\times_{\ectxext',\eft'} f_1}
  \ar{d}[swap]{\efamext}
  &
\stesysf'\times_{\ectxext',\eft'} \stesysf'
  \ar{d}{\efamext'}
  \\
\stesysf
  \ar{r}[swap]{f_1}
  &
\stesysf'
\end{tikzcd}
\end{equation*}
Composition and the identity homomorphism are defined in the expected way. We
define furthermore
\begin{align*}
f_2 & := \pullback{f_1}{f_1}{\ectxext'}{\eft'}
  \\
f_3 & := \pullback{f_2}{f_2}{\efamext'}{\eft[1]'}.
\end{align*}
\end{defn}

\begin{defn}
A pre-extension homomorphism between extension objects is called an extension
homomorphism.
\end{defn}

\begin{lem}
Let $\stesys$ be an extension object. Then
\begin{equation*}
\begin{tikzcd}[column sep=large]
\stesyst_2
  \ar{r}{\pullbackpr{2}{\ectxext}{\eft\circ\ebd}}
  \ar{d}[swap]{\ebd[1]}
  &
\stesyst
  \ar{d}{\ebd}
  \\
\stesysf_2
  \ar{r}{\pullbackpr{2}{\ectxext}{\eft}}
  \ar{d}[swap]{\eft[1]}
  &
\stesysf
  \ar{d}{\eft}
  \\
\stesysf
  \ar{r}[swap]{\ectxext}
  &
\stesysc
\end{tikzcd}
\end{equation*}
assembles an extension homomorphism $\mathbf{e}_0:\famesys{\stesys}\to\stesys$.
Likewise, we have an extension homomorphism
$\mathbf{e}_1:\famesys{\famesys{\stesys}}\to\famesys{\stesys}$. Thus, a
pre-extension object is an extension object if and only if $\mathbf{e}_0$
and $\mathbf{e}_1$ are pre-extension homomorphisms.
\end{lem}

\begin{proof}
Immediate from the conditions of being an extension object.
\end{proof}

\begin{defn}
Let $\stesys$ be a pre-extension object. Then
\begin{equation*}
\begin{tikzcd}
\stesyst_3
  \ar{r}{\beta^t}
  \ar{d}[swap]{\ebd[2]}
  &
\stesyst_2
  \ar{d}{\ebd[1]}
  \\
\stesysf_3
  \ar{r}{\beta_2}
  \ar{d}[swap]{\eft[2]}
  &
\stesysf_2
  \ar{d}{\eft[1]}
  \\
\stesysf_2
  \ar{r}[swap]{\beta_1}
  &
\stesysf
\end{tikzcd}
\qquad
\text{and}
\qquad
\begin{tikzcd}
\stesyst_4
  \ar{r}{\beta^t_2}
  \ar{d}[swap]{\ebd[3]}
  &
\stesyst_3
  \ar{d}{\ebd[2]}
  \\
\stesysf_4
  \ar{r}{\beta_3}
  \ar{d}[swap]{\eft[3]}
  &
\stesysf_3
  \ar{d}{\eft[2]}
  \\
\stesysf_3
  \ar{r}[swap]{\beta_2}
  &
\stesysf_2
\end{tikzcd}
\end{equation*}
assemble pre-extension homomorphisms 
\(
\boldsymbol{\beta}
  :
\famesys{\famesys{\stesys}}
  \to
\famesys{\stesys}
\) 
and
\(
\boldsymbol{\beta}_\mathbf{2}
  :
\famesys{\famesys{\famesys{\stesys}}}
  \to
\famesys{\famesys{\stesys}}
\).
\end{defn}

\begin{defn}\label{famehom}
Suppose that $f:\stesys'\to\stesys$ is a pre-extension homomorphism. Then we
define $\famehom{f}:\famesys{\stesys'}\to\famesys{\stesys}$ to consist of
\begin{equation*}
\begin{tikzcd}
\stesyst_2'
  \ar{r}{f^t_2}
  \ar{d}[swap]{\ebd[1]'}
  &
\stesyst_2
  \ar{d}{\ebd[1]}
  \\
\stesysf_2'
  \ar{r}{f_2}
  \ar{d}[swap]{\eft[1]'}
  &
\stesysf_2
  \ar{d}{\eft[1]}
  \\
\stesysf'
  \ar{r}[swap]{f_1}
  &
\stesysf
\end{tikzcd}
\end{equation*}
where we define
\begin{equation*}
f^t_2 := \pullback{f_1}{f^t}{\ectxext}{\eft\circ\ebd}.
\end{equation*}
\end{defn}

\begin{lem}
The triple $\famehom{f}$ defined in \autoref{famehom} is a pre-extension homomorphism.
\end{lem}

\begin{proof}
Note that the square
\begin{equation*}
\begin{tikzcd}
\stesysf_2'
  \ar{r}{f_2}
  \ar{d}[swap]{\efamext'}
  &
\stesysf_2
  \ar{d}{\efamext}
  \\
\stesysf'
  \ar{r}[swap]{f_1}
  &
\stesysf
\end{tikzcd}
\end{equation*}
commutes by assumption. Thus, it remains to show that the square
\begin{equation*}
\begin{tikzcd}
\stesysf_3'
  \ar{r}{f_3}
  \ar{d}[swap]{\eext{2}'}
  &
\stesysf_3
  \ar{d}{\eext{2}}
  \\
\stesysf_2'
  \ar{r}[swap]{f_2}
  &
\stesysf_2
\end{tikzcd}
\end{equation*}
commutes. It is equivalent to show that the equalities
\begin{align*}
\pullbackpr{1}{\ectxext}{\eft}\circ f_2\circ\eext{2}'
  & =
\pullbackpr{1}{\ectxext}{\eft}\circ \eext{2}\circ f_3
  \\
\pullbackpr{2}{\ectxext}{\eft}\circ f_2\circ\eext{2}'
  & =
\pullbackpr{2}{\ectxext}{\eft}\circ \eext{2}\circ f_3
\end{align*}
both hold. For the first, it is straightforward to verify that the diagram
\begin{equation*}
\begin{tikzcd}[column sep=large]
{} &
\stesysf_2'
  \ar{r}{f_2}
  \ar{dr}[near end]{\pullbackpr{1}{\ectxext'}{\eft'}}
  &
\stesysf_2
  \ar{dr}{\pullbackpr{1}{\ectxext}{\eft}}
  \\
\stesysf_3'
  \ar{ur}{\eext{2}'}
  \ar{r}[swap]{\beta_2'}
  \ar{ddr}[swap]{f_3}
  &
\stesysf_2'
  \ar{r}{\efamext'}
  \ar{dr}[swap]{f_2}
  &
\stesysf'
  \ar{r}{f_1}
  &
\stesysf
  \\
{} & {} &
\stesysf_2
  \ar{ur}[near start]{\efamext}
  \\
{} &
\stesysf_3
  \ar{r}[swap]{\eext{2}}
  \ar{ur}{\beta_2}
  &
\stesysf_2
  \ar{uur}[swap]{\pullbackpr{1}{\ectxext}{\eft}}
\end{tikzcd}
\end{equation*}
commutes. For the second, note that the diagram
\begin{equation*}
\begin{tikzcd}[column sep=large]
{} &
\stesysf_2'
  \ar{r}{f_2}
  \ar{dr}{\beta_1'}
  &
\stesysf_2
  \ar{ddr}{\pullbackpr{2}{\ectxext}{\eft}}
  \\
{} & {} &
\stesysf'
  \ar{dr}[swap,near start]{f_1}
  \\
\stesysf_3'
  \ar{uur}{\eext{2}'}
  \ar{r}{\beta_2'}
  \ar{dr}[swap]{f_3}
  &
\stesysf_2'
  \ar{r}{f_2}
  \ar{ur}{\efamext'}
  &
\stesysf_2
  \ar{r}[swap]{\efamext}
  &
\stesysf
  \\
{} &
\stesysf_3
  \ar{r}[swap]{\eext{2}}
  \ar{ur}[near start]{\beta_2}
  &
\stesysf_2
  \ar{ur}[swap]{\pullbackpr{2}{\ectxext}{\eft}}
\end{tikzcd}
\end{equation*}
commutes.
\end{proof}

\begin{lem}[Stability under retracts]\label{esys-retract}
Suppose $f:\stesys\to\stesys'$ is a pre-extension homomorphism between
pre-extension objects. If there is a pre-extension homomorphism $g:\stesys'\to
\stesys$ such that $g\circ f=\catid{\stesys}$ and $\stesys'$ is an extension
algebra, then $\stesys$ is an extension algebra.
\end{lem}

Before we start with the proof, note that we have the equalities
$g_2\circ f_2=\catid{\stesysf_2}$ and $g_3\circ f_3=\catid{\stesysf_3}$
under the hypotheses of the lemma.

\begin{proof}
Our first subgoal is to show that the square
\begin{equation*}
\begin{tikzcd}
\stesysf_2 
  \ar{r}{\efamext} 
  \ar{d}[swap]{\pullbackpr{2}{\ectxext}{\eft}} 
  & 
\stesysf 
  \ar{d}{\ectxext}
  \\
\stesysf
  \ar{r}[swap]{\ectxext} 
  & 
\stesysc
\end{tikzcd}
\end{equation*}
commutes. Note that in the diagram
\begin{equation*}
\begin{tikzcd}
  {}
  & 
\stesysf
  \ar{dd}[near start]{\ectxext}
  \ar{rr}{f_1}
  &
  &
\stesysf'
  \ar{dd}[near start]{\ectxext'}
  \ar{rr}{g_1}
  &
  &
\stesysf
  \ar{dd}{\ectxext}
  \\
\stesysf_2
  \ar{dd}[swap]{\pullbackpr{2}{\ectxext}{\eft}}
  \ar[crossing over]{rr}[swap,near start]{f_2}
  \ar{ur}{\efamext}
  &
  &
\stesysf_2'
  \ar{ur}[near start]{\efamext'}
  \ar[crossing over]{rr}[swap,near start]{g_2}
  &
  &
\stesysf_2
  \ar{ur}[swap,near start]{\efamext}
  \\
  {}
  &
\stesysc
  \ar{rr}[near start]{f_0}
  &
  &
\stesysc'
  \ar{rr}[near start]{g_0}
  &
  &
\stesysc
  \\
\stesysf 
  \ar{rr}[swap]{f_1}
  \ar{ur}{\ectxext}
  &
  &
\stesysf' 
  \ar[crossing over,leftarrow]{uu}[near end,swap]{\pullbackpr{2}{\ectxext'}{\eft'}}
  \ar{ur}[swap,near end]{\ectxext'}
  \ar{rr}[swap]{g_1}
  &
  &
\stesysf
  \ar[crossing over,leftarrow]{uu}[near end,swap]{\pullbackpr{2}{\ectxext}{\eft}}
  \ar{ur}[swap]{\ectxext}
\end{tikzcd}
\end{equation*}
all the faces minus the far left and far right face commute. Using that $g$
is a section of $f$, we can read off that also the far left face commutes,
completing our first subgoal.
 
For the second subgoal, note that also $\famehom{g}\circ\famehom{f}=
\catid{\famesys{\stesys}}$ and that $\famesys{\stesys'}$ is an extension object.
Thus we can apply what we have proven so far to conclude that the square
\begin{equation*}
\begin{tikzcd}
\stesysf_3 
  \ar{r}{\eext{2}} 
  \ar{d}[swap]{\pullbackpr{2}{\efamext}{\eft[1]}} 
  & 
\stesysf_2 
  \ar{d}{\efamext}
  \\
\stesysf_2
  \ar{r}[swap]{\efamext} 
  & 
\stesysf
\end{tikzcd}
\end{equation*}
commutes.
\end{proof}

\subsection{The change of base of (pre-)extension objects}
\label{subsection:change_of_base}
An important construction of (pre-)extension objects is the change of base. It
allows us to consider `parametrized homomorphisms', such as weakening and
substitution.

\begin{defn}
Suppose $f:\stesys\to\stesys'$ is a pre-extension homomorphism. We say that
a diagram
\begin{equation*}
\begin{tikzcd}
\stesys
  \ar{r}{f}
  \ar{d}[swap]{p}
  &
\stesys'
  \ar{d}{p'}
  \\
X \ar{r}[swap]{g}
  &
Y
\end{tikzcd}
\end{equation*}
commutes if the diagram
\begin{equation*}
\begin{tikzcd}
\stesysc
  \ar{r}{f_0}
  \ar{d}[swap]{p}
  &
\stesysc'
  \ar{d}{p'}
  \\
X \ar{r}[swap]{g}
  &
Y
\end{tikzcd}
\end{equation*}
commutes.
\end{defn}

The first goal in this subsection is to define for every (pre-)extension object 
$\stesys$ and every $p:\stesysc\rightarrow X\leftarrow Y:g$, a (pre-)extension
object $\cobesys{Y}{\stesys}{g}{p}$ with a homomorphism $\pullbackpr{2}{g}{p}:
\cobesys{Y}{\stesys}{g}{p}\to\stesys$ and a morphism $\pullbackpr{1}{g}{p}:
\pullback{Y}{\stesysc}{g}{p}\to Y$ such that for every diagram
\begin{equation*}
\begin{tikzcd}[column sep=large]
\stesys'
  \ar[bend right=10]{ddr}[swap]{p'}
  \ar[bend left=10]{rrd}{f}
  \ar[dotted]{dr}[near end]{[p',f]}
  \\
  {}&
\cobesys{Y}{\stesys}{g}{p}
  \ar{d}{\pullbackpr{1}{g}{p}}
  \ar{r}[swap]{\pullbackpr{2}{g}{p}}
  &
\stesys
  \ar{d}{p}
  \\
  {}&
Y \ar{r}[swap]{g}
  &
X
\end{tikzcd}
\end{equation*}
of which the outer square commutes, the (pre-)extension homomorphism $[p',f]$ exists
and is unique with the property that it renders the diagram commutative. We will
give the definition of $\cobesys{Y}{\stesys}{g}{p}$ in \autoref{cobesys}. After
proving that the change of base of a pre-extension algebra is indeed a
pre-extension algebra (\autoref{cobesys-preext}) and that the change of base
of an extension algebra is an extension algebra (\autoref{cobesys-ext}), we
will demonstrate the above unique existence in \autoref{cobesys-existence,%
cobesys-pullback}.

The second goal in this subsection is to follow the same procedure for
$\famesys{\famesys{\stesys}}$ to show that it is equivalent to
$\cobesys{\stesysf}{\famesys{\stesys}}{\ectxext}{\eft}$. We will do this by
verifying directly that it has the universal property of the change of base
described above, because we will use the ingredients in our definition of
weakening and substitution objects.

\begin{defn}[Change of base]\label{cobesys}
Suppose $\stesys$ is a pre-extension object in $\cat{C}$ and that 
$p:\stesysc\rightarrow X\leftarrow Y:g$.
Then we define the pre-extension object $\cobesys{Y}{\stesys}{g}{p}$ to consist of
\begin{equation*}
\begin{tikzcd}
\cobesys{Y}{\stesyst}{g}{p\circ\eft\circ\ebd}
  \ar{r}
  \ar{d}[swap]{g^\ast(\ebd)}
  &
\stesyst
  \ar{d}{\ebd}
  \\
\cobesys{Y}{\stesysf}{g}{p\circ\eft}
  \ar{r}
  \ar{d}[swap]{g^\ast(\eft)}
  &
\stesysf
  \ar{d}{\eft}
  \\
\cobesys{Y}{\stesysc}{g}{p}
  \ar{r}
  \ar{d}[swap]{\pullbackpr{1}{g}{p}}
  &
\stesysc
  \ar{d}{p}
  \\
Y \ar{r}[swap]{g}
  &
X
\end{tikzcd}
\end{equation*} 
and the operations
\begin{align*}
\cobesys{Y}{\ectxext}{g}{p} 
  & : \pullback{Y}{\stesysf}{g}{p\circ\eft}\to \pullback{Y}{\stesysc}{g}{p}\\
\cobesys{Y}{\efamext}{g}{p} 
  & : \pullback
    {\pullback{Y}{\stesysf}{g}{p\circ\eft}}
    {\pullback{Y}{\stesysf}{g}{p\circ\eft}}
    {\cobesys{Y}{\ectxext}{g}{p}}
    {g^\ast(\eft)}
  \to 
  \pullback{Y}{\stesysf}{g}{p\circ\eft}.
\end{align*}
defined by
\begin{equation*}
\cobesys{Y}{\ectxext}{g}{p} := \pullback{\catid{Y}}{\ectxext}{g}{p}
\end{equation*}
and where $\cobesys{Y}{\efamext}{g}{p}$ is defined by rendering the diagram
\begin{equation*}
\begin{tikzcd}[column sep=large]
(\cobesys{Y}{\stesysf}{g}{p\circ\eft})_2
  \ar{rr}{\pullback{\pullbackpr{2}{g}{p\circ\eft}}{\pullbackpr{2}{g}{p\circ\eft}}{\ectxext}{\eft}}
  \ar{dd}[swap]{\pullbackpr{1}{\cobesys{Y}{\ectxext}{g}{p}}{g^\ast(\eft)}}
  \ar[dotted]{dr}[swap]{\cobesys{Y}{\efamext}{g}{p}}
  &
  &
\stesysf_2
  \ar{d}{\efamext}
  \\
  {}&
\cobesys{Y}{\stesysf}{g}{p\circ\eft}
  \ar{r}{\pullbackpr{2}{g}{p\circ\eft}}
  \ar{d}[swap]{\pullbackpr{1}{g}{p\circ\eft}}
  &
\stesysf
  \ar{d}{p\circ\eft}
  \\
\cobesys{Y}{\stesysf}{g}{p\circ\eft}
  \ar{r}[swap]{\pullbackpr{1}{g}{p\circ\eft}}
  &
Y \ar{r}[swap]{g}
  &
X
\end{tikzcd}
\end{equation*} 
commutative. 
The process of obtaining the pre-extension object $\cobesys{Y}{\stesys}{g}{p}$ out of $\stesys$
and $g:Y\to X$ is also called the \emph{change of base}.
\end{defn}

\begin{lem}\label{cobesys-preext}
Any change of base of a pre-extension object is a pre-extension object.
\end{lem}

\begin{proof}
Let $\stesys$ be an extension algebra and consider $p:\stesysc\rightarrow X\leftarrow Y:g$.
We need to verify that the square
\begin{equation*}
\begin{tikzcd}[column sep=large]
(\pullback{Y}{\stesysf}{g}{p\circ\eft})_2
  \ar{r}{\cobesys{Y}{\efamext}{g}{p}} 
  \ar{d}[swap]{\pullbackpr{1}{\cobesys{Y}{\ectxext}{g}{p}}{g^\ast(\eft)}} 
  & 
\pullback{Y}{\stesysf}{g}{p\circ\eft}
  \ar{d}{g^\ast(\eft)}
  \\
\pullback{Y}{\stesysf}{g}{p\circ\eft}
  \ar{r}[swap]{g^\ast(\eft)} 
  & 
\pullback{Y}{\stesysc}{g}{p}
\end{tikzcd}
\end{equation*}
commutes. It is fairly obvious that
\begin{equation*}
\pullbackpr{1}{g}{p}\circ g^\ast(\eft)\circ (\cobesys{Y}{\efamext}{g}{p})
  =
\pullbackpr{1}{g}{p\circ\eft}\circ \pullbackpr{1}{\cobesys{Y}{\ectxext}{g}{p}}{g^\ast(\eft)}
\end{equation*}
and that the diagram
\begin{equation*}
\begin{tikzcd}
  {}&
  {}&
\pullback{Y}{\stesysf}{g}{p\circ\eft}
  \ar{rr}{g^\ast(\eft)}
  \ar{dr}[swap]{\pullbackpr{2}{g}{p\circ\eft}}
  &
  {}&
\pullback{Y}{\stesysc}{g}{p}
  \ar{ddr}{\pullbackpr{2}{g}{p}}
  \\
  {}&
  {}&
  {}&
\stesysf
  \ar{drr}[swap]{\eft}
  \\
(\pullback{y}{\stesysf}{g}{p\circ\eft})_2
  \ar{uurr}{\cobesys{Y}{\efamext}{g}{p}}
  \ar{rr}[swap,yshift=-.5ex]{\pullback{\pullbackpr{2}{g}{p\circ\eft}}{\pullbackpr{2}{g}{p\circ\eft}}{\ectxext}{\eft}}
  \ar{ddrr}[swap]{\pullbackpr{1}{\cobesys{Y}{\ectxext}{g}{p}}{g^\ast(\eft)}}
  &
  {}&
\stesysf_2
  \ar{ur}{\efamext}
  \ar{dr}[swap]{\eft[1]}
  &
  {}&
  {}&
\stesysc
  \\
  {}&
  {}&
  {}&
\stesysf
  \ar{urr}{\eft}
  \\
  {}&
  {}&
\pullback{Y}{\stesysf}{g}{p\circ\eft}
  \ar{rr}[swap]{g^\ast(\eft)}
  \ar{ur}{\pullbackpr{2}{g}{p\circ\eft}}
  &
  {}&
\pullback{Y}{\stesysc}{g}{p}
  \ar{uur}[swap]{\pullbackpr{2}{g}{p}}
\end{tikzcd}
\end{equation*}
commutes.
\end{proof}

\begin{thm}\label{cobesys-ext}
The change of base of an extension algebra is an extension algebra.
\end{thm}

\begin{proof}
Our first subgoal is to verify that the square
\begin{equation*}
\begin{tikzcd}[column sep=large]
(\pullback{Y}{\stesysf}{g}{p\circ\eft})_2
  \ar{r}{\cobesys{Y}{\efamext}{g}{p}} 
  \ar{d}[swap]{\pullbackpr{2}{\cobesys{Y}{\ectxext}{g}{p}}{g^\ast(\eft)}} 
  & 
\pullback{Y}{\stesysf}{g}{p\circ\eft}
  \ar{d}{\cobesys{Y}{\ectxext}{g}{p}}
  \\
\pullback{Y}{\stesysf}{g}{p\circ\eft}
  \ar{r}[swap]{\cobesys{Y}{\ectxext}{g}{p}} 
  & 
\pullback{Y}{\stesysc}{g}{p}
\end{tikzcd}
\end{equation*}
\end{proof}

The following construction is useful for defining extension homomorphisms into
`higher' extension objects

\begin{defn}\label{cobesys-existence}
Consider a commutative diagram
\begin{equation*}
\begin{tikzcd}
\stesys'
  \ar{r}{f}
  \ar{d}[swap]{p'}
  &
\stesys
  \ar{d}{p}
  \\
Y \ar{r}[swap]{g}
  &
X
\end{tikzcd}
\end{equation*}
Then we construct $[p,f]:\stesys'\to\cobesys{Y}{\stesys}{g}{p}$
\begin{itemize}
\item by defining $[p,f]_0:\stesysc'\to\pullback{Y}{\stesysc}{g}{p}$ be the uniqe
morphism rendering the diagram
\begin{equation*}
\begin{tikzcd}[column sep=large]
\stesysc'
  \ar[bend right=10]{ddr}[swap]{p'}
  \ar[bend left=10]{rrd}{f_0}
  \ar{dr}[near end]{[p,f]_0}
  \\
  {}&
\pullback{Y}{\stesysc}{g}{p}
  \ar{r}[swap]{\pullbackpr{2}{g}{p}}
  \ar{d}{\pullbackpr{1}{g}{p}}
  &
\stesysc
  \ar{d}{p}
  \\
  {}&
Y \ar{r}[swap]{g}
  &
X
\end{tikzcd}
\end{equation*}
commutative.
\item by defining $[p,f]_1:\stesysf'\to\pullback{Y}{\stesysf}{g}{p\circ\eft}$ be the uniqe
morphism rendering the diagram
\begin{equation*}
\begin{tikzcd}[column sep=huge]
\stesysf'
  \ar[bend right=10]{ddr}[swap]{p'\circ\eft'}
  \ar[bend left=10]{rrd}{f_1}
  \ar{dr}[near end]{[p,f]_1}
  \\
  {}&
\pullback{Y}{\stesysf}{g}{p\circ\eft}
  \ar{r}[swap]{\pullbackpr{2}{g}{p\circ\eft}}
  \ar{d}{\pullbackpr{1}{g}{p\circ\eft}}
  &
\stesysc
  \ar{d}{p\circ\eft}
  \\
  {}&
Y \ar{r}[swap]{g}
  &
X
\end{tikzcd}
\end{equation*}
commutative.
\item by defining $[p,f]^t:\stesyst'\to\pullback{Y}{\stesyst}{g}{p\circ\eft\circ\ebd}$ be the uniqe
morphism rendering the diagram
\begin{equation*}
\begin{tikzcd}[column sep=huge]
\stesyst'
  \ar[bend right=10]{ddr}[swap]{p'\circ\eft'\circ\ebd'}
  \ar[bend left=10]{rrd}{f^t}
  \ar{dr}[near end]{[p,f]^t}
  \\
  {}&
\pullback{Y}{\stesysf}{g}{p\circ\eft\circ\ebd}
  \ar{r}[swap]{\pullbackpr{2}{g}{p\circ\eft\circ\ebd}}
  \ar{d}{\pullbackpr{1}{g}{p\circ\eft\circ\ebd}}
  &
\stesysc
  \ar{d}{p\circ\eft\circ\ebd}
  \\
  {}&
Y \ar{r}[swap]{g}
  &
X
\end{tikzcd}
\end{equation*}
commutative.
\end{itemize}
\end{defn}

\begin{thm}\label{cobesys-pullback}
For every diagram
\begin{equation*}
\begin{tikzcd}[column sep=large]
\stesys'
  \ar[bend right=10]{ddr}[swap]{p'}
  \ar[bend left=10]{rrd}{f}
  \ar[dotted]{dr}[near end]{[p',f]}
  \\
  {}&
\cobesys{Y}{\stesys}{g}{p}
  \ar{d}{\pullbackpr{1}{g}{p}}
  \ar{r}[swap]{\pullbackpr{2}{g}{p}}
  &
\stesys
  \ar{d}{p}
  \\
  {}&
Y \ar{r}[swap]{g}
  &
X
\end{tikzcd}
\end{equation*}
of which the outer square commutes, the pre-extension homomorphism $[p',f]$
is unique with the property that it renders the whole diagram commutative.
\end{thm}

\begin{defn}\label{famfamstesys_into}
Consider a commutative square
\begin{equation*}
\begin{tikzcd}
\stesys'
  \ar{r}{f}
  \ar{d}[swap]{p}
  &
\famesys\stesys
  \ar{d}{\eft}
  \\
\stesysf \ar{r}[swap]{\ectxext}
  &
\stesysc
\end{tikzcd}
\end{equation*}
Then we construct
\begin{equation*}
|[p,f]|:\stesys'\to\famesys{\famesys{\stesys}}
\end{equation*}
as follows:
\begin{itemize}
\item let $|[p,f]|_0:\stesysc'\to\stesysf_2$ be the unique morphism rendering
the diagram
\begin{equation*}
\begin{tikzcd}[column sep=large]
\stesysc' 
  \ar[bend left=10]{rrd}{f_0}
  \ar[swap,bend right=10]{ddr}{p}
  \ar[dotted]{dr}[near end]{|[p,f]|_0}
  \\
  {}&
\stesysf_2
  \ar{r}[swap]{\pullbackpr{2}{\ectxext}{\eft}}
  \ar{d}{\eft[1]}
  &
\stesysf
  \ar{d}{\eft}
  \\
  {}&
\stesysf
  \ar{r}[swap]{\ectxext}
  &
\stesysc
\end{tikzcd}
\end{equation*}
commutative.
\item Let $|[p,f]|_1:\stesysf'\to\stesysf_3$ be the unique morphism rendering
the diagram
\begin{equation*}
\begin{tikzcd}[column sep=large]
\stesysf'
  \ar[bend left=10]{drr}{f_1}
  \ar[swap]{dd}{\eft'}
  \ar[dotted]{dr}[near end]{|[p,f]|_1}
  \\
  {}&
\stesysf_3
  \ar{r}[swap]{\pullbackpr{2}{\efamext}{\eft[1]}}
  \ar{d}{\eft[1]}
  &
\stesysf_2
  \ar{d}{\eft[1]}
  \\
\stesysc'
  \ar{r}[swap]{|[p,f]|_0}
  &
\stesysf_2
  \ar{r}[swap]{\efamext}
  &
\stesysf
\end{tikzcd}
\end{equation*}
commutative.
\item Let $|[p,f]|^t:\stesyst'\to\stesyst_3$ be the unique morphism rendering
the diagram
\begin{equation*}
\begin{tikzcd}[column sep=huge]
\stesyst'
  \ar[bend left=10]{drr}{f^t}
  \ar[swap]{dd}{\eft'\circ\ebd'}
  \ar[dotted]{dr}[near end]{|[p,f]|^t}
  \\
  {}&
\stesyst_3
  \ar{r}[swap]{\pullbackpr{2}{\efamext}{\eft[1]\circ\ebd[1]}}
  \ar{d}[swap]{\pullbackpr{1}{\efamext}{\eft[1]\circ\ebd[1]}}
  &
\stesyst_2
  \ar{d}{\eft[1]\circ\ebd[1]}
  \\
\stesysc'
  \ar{r}[swap]{|[p,f]|_0}
  &
\stesysf_2
  \ar{r}[swap]{\efamext}
  &
\stesysf
\end{tikzcd}
\end{equation*}
commutative.
\end{itemize}
\end{defn}

\begin{lem}
Under the hypotheses of \autoref{famfamstesys_into}, $|[p,f]|$ is a pre-extension
homomorphism. Moreover, it is the unique pre-extension homomorphism for which
the diagram
\begin{equation*}
\begin{tikzcd}[column sep=large]
\stesys' 
  \ar[bend left=10]{rrd}{f}
  \ar[swap,bend right=10]{ddr}{p}
  \ar[dotted]{dr}[near end]{|[p,f]|}
  \\
  {}&
\famesys{\famesys{\stesys}}
  \ar{r}[swap]{\pullbackpr{2}{\ectxext}{\eft}}
  \ar{d}{\eft[1]}
  &
\famesys{\stesys}
  \ar{d}{\eft}
  \\
  {}&
\stesysf
  \ar{r}[swap]{\ectxext}
  &
\stesysc
\end{tikzcd}
\end{equation*}
commutes.
\end{lem}

\begin{lem}
Suppose $f:\stesys\to \stesys'$ is a pre-extension homomorphism and consider a morphism
$p:\stesys'\to X$ and $g:Y\to X$. Then the change of base 
$g^\ast(f):\cobesys{Y}{\stesys}{g}{p\circ f_0}\to
\cobesys{Y}{\stesys'}{g}{p}$ is a pre-extension morphism.
\end{lem}

\begin{lem}
Let $\stesys$ be a pre-extension algebra and consider $p:\stesysc\rightarrow X\leftarrow Y:g$.
Then there is an isomorphism
\begin{equation*}
\varphi:\famesys{\cobesys{Y}{\stesys}{g}{p}}
  \simeq
\cobesys{Y}{\famesys{\stesys}}{g}{p\circ\eft}
\end{equation*}
uniquely determined by
\end{lem}

\begin{proof}
This follows from the pasting lemma for pullbacks.
\end{proof}

\subsection{Pre-weakening objects}
\begin{defn}
Let $\stesys$ be an extension object in $\cat{C}$. A pre-weakening operation
on $\stesys$ is an extension homomorphism 
$ \mathbf{w}(\stesys)
    :
  \cobesys{\stesysf}{\famesys{\stesys}}{\eft}{\eft}
    \to
  \famesys{\famesys{\stesys}}$
for which the diagram
\begin{equation*}
\begin{tikzcd}[column sep=large]
\cobesys{\stesysf}{\famesys{\stesys}}{\eft}{\eft}
  \ar{r}{\mathbf{w}(\stesys)}
  \ar{dr}[swap]{\pullbackpr{1}{\eft}{\eft}}
  &
\famesys{\famesys{\stesys}}
  \ar{d}{\eft[1]}
  \\
& \stesysf
\end{tikzcd}
\end{equation*}
commutes.
\end{defn}

\begin{defn}
Let $\stesys$ be an extension object with pre-weakening operation
$\mathbf{w}(\stesys)$. Then $\famesys{\stesys}$ has the pre-weakening operation
$\mathbf{w}(\famesys{\stesys})$ which is uniquely determined by rendering the
diagram
\begin{equation*}
\begin{tikzcd}[column sep=large]
\cobesys{\stesysf_2}{\famesys{\famesys{\stesys}}}{\eft[1]}{\eft[1]}
  \ar{rr}{%
      \pullback{\beta_1}{\boldsymbol{\beta}}{\eft}{\eft}
    }
  \ar[bend right]{ddr}[swap]{\pullbackpr{1}{\eft[1]}{\eft[1]}}
  \ar[dotted]{dr}{\mathbf{w}(\famesys{\stesys})}
  &
  {}&
\cobesys{\stesysf}{\famesys{\stesys}}{\eft}{\eft}
  \ar{r}{\mathbf{w}(\stesys)}
  &
\famesys{\famesys{\stesys}}
  \ar{d}{\boldsymbol{\beta}}
  \\
  {}&
\famesys{\famesys{\famesys{\stesys}}}
  \ar{r}{\boldsymbol{\beta}_\mathbf{2}}
  \ar{d}[swap]{\eft[2]}
  &
\famesys{\famesys{\stesys}}
  \ar{d}{\eft[1]}
  \ar{r}{\boldsymbol{\beta}}
  &
\famesys{\stesys}
  \ar{d}{\eft}
  \\
  {}&
\stesysf_2
  \ar{r}[swap]{\efamext}
  &
\stesysf
  \ar{r}[swap]{\ectxext}
  &
\stesysc
\end{tikzcd}
\end{equation*}
commutative.
\end{defn}

\begin{defn}
A pre-weakening object $\stesys$ in $\cat{C}$ is an extension object $\stesys$ 
with a pre-weakening operation 
$ \mathbf{w}(\stesys)
    :
  \cobesys{\stesysf}{\famesys{\stesys}}{\eft}{\eft}
    \to
  \famesys{\famesys{\stesys}}$
for which the diagram
\begin{equation*}
\begin{tikzcd}[column sep=15em]
\cobesys{\stesysf_2}{\famesys{\stesys}}{\eft\circ\eft[1]}{\eft}
  \ar[bend right=10]{dr}[swap]%
    { [ \pullbackpr{1}{\eft\circ\eft[1]}{\eft},%
        \mathbf{w}(\stesys)%
          \circ%
        (\pullback{\efamext}{\catid{\famesys{\stesys}}}{\eft}{\eft})%
        ]%
      }
  \ar{r}{
    [ \pullbackpr{1}{\eft\circ\eft[1]}{\eft},%
      \mathbf{w}(\stesys)%
        \circ%
      (\pullback{\eft[1]}{\catid{\famesys{\stesys}}}{\eft}{\eft})%
      ]}%
  &
\cobesys{\stesysf_2}{\famesys{\famesys{\stesys}}}{\eft[1]}{\eft[1]}
  \ar{d}{\mathbf{w}(\famesys{\stesys})}
  \\
  {}&
\famesys{\famesys{\famesys{\stesys}}}
\end{tikzcd}
\end{equation*}
commutes. This condition is called \emph{Currying for weakening}.
\end{defn}

\begin{lem}
If $\stesys$ is a pre-weakening algebra, then so is $\famesys{\stesys}$. 
\end{lem}

\begin{proof}
\end{proof}

\begin{defn}
A pre-weakening morphism between preweakening objects $\stesys$ and $\stesys'$ is an
extension homomorphism $f:\stesys\to \stesys'$ such that additionally the diagram
\begin{equation*}
\begin{tikzcd}[column sep=large]
\cobesys{\stesysf}{\famesys{\stesys}}{\eft}{\eft}
  \ar{d}[swap]{\mathbf{w}(\stesys)}
  \ar{r}{\pullback{f_1}{\famehom{f}}{\eft'}{\eft'}}
  &
\cobesys{\stesysf'}{\famesys{\stesys'}}{\eft'}{\eft'}
  \ar{d}{\mathbf{w}(\stesys')}
  \\
\famesys{\famesys{\stesys}}
  \ar{r}[swap]{\famehom{\famehom{f}}}
  &
\famesys{\famesys{\stesys'}}
\end{tikzcd}
\end{equation*}
commutes.
\end{defn}

\begin{defn}
Let $\stesys$ be a pre-weakening algebra and consider $p:\stesysc\rightarrow X\leftarrow Y:p$.
Then we define
\begin{equation*}
\mathbf{w}(\cobesys{Y}{\stesys}{g}{p})
  :
\cobesys
  { (\pullback{Y}{\stesysf}{g}{p\circ\eft})}
  { \famesys{\cobesys{Y}{\stesys}{g}{p}}}
  { g^\ast(\eft)}
  { g^\ast(\eft)}
  \to
\famesys{\famesys{\cobesys{Y}{\stesys}{g}{p}}}
\end{equation*}
to be the unique extension homomorphism rendering the diagram
\begin{equation*}
\begin{tikzcd}
\cobesys
  { \pullback{Y}{\stesysf}{g}{p\circ\eft}}
  { \famesys{\cobesys{Y}{\stesys}{g}{p}}}
  { g^\ast(\eft)}
  { g^\ast(\eft)}
  \ar[bend right]{ddr}[swap]{\pullbackpr{1}{g^\ast(\eft)}{g^\ast(\eft)}}
  \ar{rr}{%
    \pullback
      { \pullbackpr{2}{g}{p\circ\eft}}
      { \boldsymbol{\pi}_\mathbf{2}(g,p\circ\eft)}
      { \eft}
      { \eft}
    }
  \ar[dotted]{dr}{\mathbf{w}(\cobesys{Y}{\stesys}{g}{p})}
  &
  {}&
\cobesys{\stesysf}{\famesys{\stesys}}{\eft}{\eft}
  \ar{r}{\mathbf{w}(\stesys)}
  &
\famesys{\famesys{\stesys}}
  \ar{d}{\boldsymbol{\beta}}
  \\
  {}&
\famesys{\famesys{\cobesys{Y}{\stesys}{g}{p}}}
  \ar{r}{\beta}
  \ar{d}{g^\ast(\eft)_1}
  &
\famesys{\cobesys{Y}{\stesys}{g}{p}}
  \ar{r}{\boldsymbol{\pi}_2(g,p\circ\eft)}
  \ar{d}{g^\ast(\eft)}
  &
\famesys{\stesys}
  \ar{d}{\eft}
  \\
  {}&
\pullback{Y}{\stesysf}{g}{p\circ\eft}
  \ar{r}[swap]{\cobesys{Y}{\ectxext}{g}{p}}
  &
\pullback{Y}{\stesysc}{g}{p}
  \ar{r}[swap]{\pullbackpr{2}{g}{p}}
  &
\stesysc
\end{tikzcd}
\end{equation*}
commutative.
\end{defn}

\subsection{Weakening objects}
Since we have shown that the property of being a pre-weakening object is closed
under the relevant operations, we can make the following definition:

\begin{defn}
A weakening object is a pre-weakening object $\stesys$ with the property that
$\mathbf{w}(\stesys)$ is a pre-weakening morphism.
\end{defn}

\begin{defn}
A weakening homomorphism is a pre-weakening homomorphism such that the domain
and codomain are weakening objects.
\end{defn}

\begin{thm}
Suppose $\stesys$ is a weakening object, then so is $\famesys{\stesys}$
\end{thm}

\begin{thm}
The change of base of any weakening object is again a weakening object.
\end{thm}

\subsection{Projection objects}
\begin{defn}
A pre-projection object is a weakening object $\stesys$ for which there is a term
$\mathbf{i}:\stesysf\to \stesyst_2$ such that the diagram
\begin{equation*}
\begin{tikzcd}[column sep=large]
\stesysf \ar{r}{\mathbf{i}} \ar{d}[swap]{\Delta_{\eft}} & \stesyst_2 \ar{d}{\ebd[1]}\\
\pullback{\stesysf}{\stesysf}{\eft}{\eft} \ar{r}[swap]{w(\stesys)_0} & \stesysf_2
\end{tikzcd}
\end{equation*}
commutes. In this diagram $\Delta_{\eft}:\stesysf\to \pullback{\stesysf}{\stesysf}{\eft}{\eft}$ is the diagonal.
\end{defn}

\begin{defn}
A pre-projection homomorphism from $\stesys$ to $\stesys'$ is a weakening homomorphism
$f:\stesys\to \stesys'$ such that the square
\begin{equation*}
\begin{tikzcd}[column sep=large]
\stesyst_2
  \ar{r}{f^t_1}
  &
\stesyst_2'
  \\
\stesysf \ar{r}[swap]{f_1}
  \ar{u}{\mathbf{i}}
  &
\stesysf'
  \ar{u}[swap]{\mathbf{i}'}
\end{tikzcd}
\end{equation*}
commutes
\end{defn}

\begin{lem}
The change of base of a pre-projection object is again a pre-projection object.
\end{lem}

\begin{lem}
If $CFT$ is a pre-projection object, then so is $\mathbf{F}_{CFT}$, where
$\mathbf{F}_{\mathbf{i}}$ is defined to be $F\times_{e_0,c}\mathbf{i}$ is
a pre-projection algebra.
\end{lem}

\begin{defn}
A projection algebra is a pre-projection algebra for which weakening is a
pre-projection homomorphism.
\end{defn}

\begin{cor}
The change of base of a projection object is again a projection object.
\end{cor}

\begin{cor}
If $CFT$ is a projection object, then so is $\mathbf{F}_{CFT}$, where
$\mathbf{F}_{\mathbf{i}}$ is defined to be $F\times_{e_0,c}\mathbf{i}$ is
a projection algebra.
\end{cor}

\subsection{Substitution objects}

\begin{defn}
A \emph{pre-substitution} for an extension object $\stesys$ is an
extension homomorphism
\begin{equation*}
\mathbf{s}(\stesys):\cobesys{\stesyst}{\famesys{\famesys{\stesys}}}{\ebd}{\eft[1]}\to \famesys{\stesys}
\end{equation*}
for which the square
\begin{equation*}
\begin{tikzcd}[column sep=large]
\pullback{\stesyst}{\stesysf_2}{\ebd}{\eft[1]}
  \ar{r}{s(\stesys)_0}
  \ar{d}[swap]{\ebd\circ\pullbackpr{1}{\ebd}{\eft[1]}}
  &
\stesysf 
  \ar{d}{\eft}
  \\
\stesysf 
  \ar{r}[swap]{\eft}
  &
\stesysc
\end{tikzcd}
\end{equation*}
commutes. A \emph{pre-substitution object} is an extension object
together with a pre-substitution.
\end{defn}

\begin{defn}
A \emph{pre-substitution homomorphism} is an extension homomorphism $f:\stesys'\to \stesys$
for which the square
\begin{equation*}
\begin{tikzcd}[column sep=huge]
\cobesys{\stesyst'}{\famesys{\famesys{\stesys'}}}{\ebd'}{\eft[1]'}
  \ar{r}{\pullback{f^t}{\famehom{\famehom{f}}}{\ebd}{\eft[1]}}
  \ar{d}[swap]{\mathbf{s}'(\stesys')}
  &
\cobesys{\stesyst}{\famesys{\famesys{\stesys}}}{\ebd}{\eft[1]}
  \ar{d}{\mathbf{s}(\stesys)}
  \\
\famesys{\stesys'}
  \ar{r}[swap]{\famehom{f}}
  &
\famesys{\stesys}
\end{tikzcd}
\end{equation*}
commutes.
\end{defn}

\begin{lem}
The change of base of a pre-substitution object is again a pre-substitution object.
\end{lem}

\begin{lem}
If $\stesys$ is a pre-substitution object, then so is $\famesys{\stesys}$ with
$\mathbf{s}(\famesys{\stesys})$ defined to be the unique extension homomorphism
rendering the diagram
\begin{equation*}
\begin{tikzcd}
\cobesys{\stesyst_2}{\famesys{\famesys{\famesys{\stesys}}}}{\ebd[1]}{\eft[2]}
  \ar[dotted]{dr}{\mathbf{s}(\famesys{\stesys})}
  \ar{rr}{\pullback{\pullbackpr{2}{\ectxext}{\eft\circ\ebd}}{\boldsymbol{\beta}_\mathbf{2}}{\ebd}{\eft[1]}}
  \ar{dd}[swap]{\ebd[1]\circ\pullbackpr{1}{\ebd[1]}{\eft[2]}}
  &
  {}&
\cobesys{\stesyst}{\famesys{\famesys{\stesys}}}{\ebd}{\eft[1]}
  \ar{d}{\mathbf{s}(\stesys)}
  \\
  {}&
\famesys{\famesys{\stesys}}
  \ar{r}{\boldsymbol{\beta}}
  \ar{d}{\eft[1]}
  &
\famesys{\stesys}
  \ar{d}{\eft}
  \\
\stesysf_2
  \ar{r}[swap]{\eft[1]}
  &
\stesysf
  \ar{r}[swap]{\ectxext}
  &
\stesysc
\end{tikzcd}
\end{equation*}
commutative.
\end{lem}

\begin{proof}
The requirement on pre-substitutions holds by construction.
\end{proof}

It makes sense now to consider the possibility that the pre-substitution
itself is a pre-substitution homomorphism.

\begin{defn}
A \emph{substitution object} is a pre-substitution object for which substitution is
a pre-substitution homomorphism.
\end{defn}

\begin{cor}
The change of base of a substitution object is again a substitution object.
\end{cor}

\begin{cor}
If $\stesys$ is a substitution object, then so is $\famesys{\stesys}$.
\end{cor}

\subsection{Extension objects with empty context and families}

\begin{defn}
An extension object $\stesys$ is said to have \emph{empty families} if there
is a section
\begin{equation*}
\phi_1(\stesys):\stesysc\to\stesysf
\end{equation*}
of $\eft$, satisfying the following additional properties:
\begin{enumerate}
\item $\phi_1(\stesys)$ is also a section of $\ectxext$.
\item The unique morphism $\phi_2$ rendering the diagram
\begin{equation*}
\begin{tikzcd}[column sep=huge]
\stesysf
  \ar[bend left=20]{drr}{\phi_1(\stesys)\circ\ectxext}
  \ar[equals,bend right=20]{ddr}
  \ar[dotted]{dr}{\phi_2}
  \\
  {}&
\stesysf_2
  \ar{d}{\eft[1]}
  \ar{r}{\pullbackpr{2}{\ectxext}{\eft}}
  &
\stesysf
  \ar{d}{\eft}
  \\
  {}&
\stesysf
  \ar{r}[swap]{\ectxext}
  &
\stesysc
\end{tikzcd}
\end{equation*}
commutative, is a section of $\efamext$.
\item The unique morphism $\iota_1$ rendering the diagram
\begin{equation*}
\begin{tikzcd}[column sep=huge]
\stesysf
  \ar[equals,bend left=20]{drr}
  \ar[bend right=20]{ddr}[swap]{\phi_1(\stesys)\circ\eft}
  \ar[dotted]{dr}{\iota_1}
  \\
  {}&
\stesysf_2
  \ar{d}{\eft[1]}
  \ar{r}[swap]{\pullbackpr{2}{\ectxext}{\eft}}
  &
\stesysf
  \ar{d}{\eft}
  \\
  {}&
\stesysf
  \ar{r}[swap]{\ectxext}
  &
\stesysc
\end{tikzcd}
\end{equation*}
commutative, is a section of $\efamext$.
\end{enumerate}
\end{defn}

\begin{defn}
A homomorphism of extension objects with empty families is an extension
homomorphism $f:\stesys'\to\stesys$ for which the diagram
\begin{equation*}
\begin{tikzcd}
\stesysf'
  \ar{r}{f_1}
  &
\stesysf
  \\
\stesysc'
  \ar{u}{\phi_1(\stesys')}
  \ar{r}[swap]{f_0}
  &
\stesysc
  \ar{u}[swap]{\phi_1(\stesys)}
\end{tikzcd}
\end{equation*}
commutes.
\end{defn}

\begin{lem}
Suppose $\stesys$ is an extension object with empty families. Then
$\famesys{\stesys}$ is an extension object with empty families, with
$\phi_1(\famesys{\stesys}):=\phi_2$. 
\end{lem}

\begin{lem}
Suppose $\stesys$ is an extension object with empty families and consider
$p:\stesysc\rightarrow X\leftarrow Y:p$. Then
$\cobesys{Y}{\stesys}{g}{p}$ is an extension object with empty families
with $\phi_1(\cobesys{Y}{\stesys}{g}{p}):=g^\ast(\phi_1)$.
\end{lem}

\subsection{E-objects}
\begin{defn}
An \emph{E-object} is an extension object with the structure of a projection object,
the structure of a substitution object and which has an empty context and families,
such that additionally:
\begin{enumerate}
\item substitution is a projection homomorphism
\item weakening is a substitution homomorphism
\item both weakening and substitution are empty-CF homomorphisms.
\item 
\end{enumerate}
\end{defn}


\bibliographystyle{plain}
%\phantomsection\addcontentsline{toc}{section}{References}
\bibliography{References/refs}

\end{document}
