\section{Internal models of type theory}
\subsection{Models of type theory without basic constructors}\label{internal-model-contexts}
\begin{defn}\label{defn:premodel}
An internal model $\mfM$ of type theory consists of the following data. 
\begin{enumerate}
\item A type $\tfctx(\mfM)$ of \emph{contexts}.
\item A function $\mftypfunc{\mfM}$ assigning to each context $\Gamma$ of $\mfM$ an internal model
$\mftyp{\mfM}{\Gamma}$ of type theory.
\begin{defn}
The type $\tfctx(\mftyp{\mfM}{\Gamma})$ is denoted by $\mftyp{\mfM}{\Gamma}$. A
term of $\mftyp{\mfM}{\Gamma}$ is called a \emph{type in context $\Gamma$}. To indicate
that $A$ is a type in context $\Gamma$ we also write $\Gamma\vdash A:\mfM$. 
When $P:\mftyp{\mftyp{\mfM}{\Gamma}}{A}$ for some type $A$ in context $\Gamma$, we
also speak of \emph{a family $P$ over $A$ in context $\Gamma$.}
\end{defn}
\item A family $\terms{\blank}:\prd*{\Gamma:\ctx(\mfM)}\mftyp{\mfM}{\Gamma}\to\type$
assigning to every type $A$ in context $\Gamma$ the type $\terms{A}$ of its
terms.
\begin{defn}
When $A$ is a type in context $\Gamma$, we define $\Gamma\vdash x:A$ 
to mean $x:\terms{A}$.
\end{defn}
\item Context extension: a morphism
\begin{equation*}
\tfext^\Gamma:\mftyp{\mfM}{\Gamma}\to\mfM
\end{equation*}
of internal models for every context $\Gamma$.
\begin{defn}
When $A$ is a type in context $\Gamma$, we also denote
the context $\tfext^\Gamma_0(A)$ of $\mfM$ by $\ctxext{\Gamma}{A}$.
We will write
$\ctxext{{\Gamma}{A}}{P}$ for $\ctxext{{\Gamma}{A}}{P}$.
\end{defn}
\item The judgmental equalities:
\begin{align*}
\mftyp{\mftyp{\mfM}{\Gamma}}{A} & \jdeq\mftyp{\mfM}{\ctxext{\Gamma}{A}}\\
\mftyp{\tfext^\Gamma}{A} & \jdeq \modelfont{id}_{\mftyp{\mfM}{\ctxext{\Gamma}{A}}}\\
\terms[{\tfext^\Gamma}]{P} & \jdeq \idfunc[\terms{P}].
\end{align*}
\begin{rmk}
In particular, we will have the judgmental equalities:
\begin{enumerate}
\item When $A$ is a type in context $\Gamma$ we have
\begin{equation*}
\mftyp{\mftyp{\mfM}{\Gamma}}{A}\jdeq\mftyp{\mfM}{\ctxext{\Gamma}{A}},
\end{equation*}
ensuring that a context in the model $\mftyp{\mftyp{\mfM}{\Gamma}}{A}$ is the same thing as a
context in the model $\mftyp{\mfM}{\ctxext{\Gamma}{A}}$.
\item If $Q$ is a family over $P$ where $P$ is a family over $A$ in context $\Gamma$, then
\begin{equation*}
\tfext^{\protect{\mftyp{\mftyp{\mfM}{\Gamma}}{A}}}(P,Q)
\jdeq
\tfext^{\protect{\mftyp{\mfM}{\ctxext{\Gamma}{A}}}}(P,Q)
\end{equation*}
ensuring that twe two possible notion of context extension are the same.
\end{enumerate}
Other judgmental equalities will be required with the ingredients that follow.
We will not list them all.
\end{rmk}
\item Weakening: a morphism
\begin{equation*}
\tfwk^A:\mftyp{\mfM}{\Gamma}\to\mftyp{\mfM}{\ctxext{\Gamma}{A}}
\end{equation*}
of internal models for every type $A$ in context $\Gamma$. When $B$ is a type
in context $\Gamma$, we denote $\tfwk^A(B)$ by $\ctxwk{A}{B}$. 
\item Substitution: a morphism
\begin{equation*}
\tfsubst^x:\mftyp{\mfM}{\ctxext{\Gamma}{A}}\to\mftyp{\mfM}{\Gamma}
\end{equation*}
of internal models for any $x:A$ in context $\Gamma$. When $P$ is a family over $A$ in context
$\Gamma$, we denote $\tfsubst^x(P)$ also by $\subst{x}{P}$. 
\item A judgmental equality
\begin{equation*}
\tfsubst^x\circ\tfwk^A\jdeq\modelfont{id}_{\mftyp{\mfM}{\Gamma}}
\end{equation*}
for any $x:A$ and $B$ in context $\Gamma$.
\item A context $\unit^\mfM:\tfctx(\mfM)$ and a term of type $\isequiv(\ctxext{\unit^\mfM}{\blank})$. We will denote
this equivalence by $e_\unit$. The context $\unit^\mfM$ is also
called the \emph{empty context}.
\begin{defn}
For any context $\Gamma$, type $\terms{\Gamma}$ is defined to mean
$\terms{e_\unit^{-1}(\Gamma)}$. 
\end{defn}
\item A section $\unit^{\blank}:\prd{\Gamma:\ctx(\mfM)}\mftyp{\mfM}{\Gamma}$ assigning
a type $\unit^\Gamma$ in context $\Gamma$ to every context $\Gamma$ and
an identification $\alpha_\unit(\Gamma):\id{\ctxext{\Gamma}{\unit^\Gamma}}{\Gamma}$
for every context $\Gamma$. We also require that there is an identification
$\id{\trans{\alpha_\unit(\Gamma)}{\ctxwk{\unit^\Gamma}{A}}}{A}$ for every
type $A$ in context $\Gamma$.
\item For any type $A$ in context $\Gamma$, a term $\idfunc[A]:\terms{\ctxwk{A}{A}}$
\end{enumerate}
\begin{flushright}
\textsl{End of \autoref{defn:premodel}.}
\end{flushright}
\end{defn}

\begin{defn}
In an internal model $\mfM$ we define
\begin{equation*}
\ctxhom{\Delta}{\Gamma}\defeq \terms{\ctxwk{e_\unit^{-1}(\Delta)}{e_\unit^{-1}(\Gamma)}}
\end{equation*}
\end{defn}

\subsection{Morphisms of internal models}
\begin{defn}\label{defn:premodel-morphism}
A morphism $f:\mfM\to \mfN$ of internal models consists of
\begin{enumerate}
\item a function $\ctx(f):\ctx(\mfM)\to\ctx(\mfN)$. The function $\ctx(f)$ is also
denoted by $f_0$.
\item a morphism $\mftyp{f}{\Gamma}:\mftyp{\mfM}{\Gamma}\to\mftyp{\mfN}{f_0(\Gamma)}$ of internal models for every
$\Gamma:\ctx(M)$.
\item the judgmental equality
\begin{equation*}
\mftyp{\mftyp{f}{\Gamma}}{A}\jdeq\mftyp{f}{\ctxext{\Gamma}{A}}
\end{equation*}
\item a function $\terms[f]{A}:\terms[\mfM]{A}\to\terms[\mfN]{\mftyp{f}{\Gamma}_0(A)}$ for
every type $A$ in context $\Gamma$. 
\item preservation of context extension: 
\begin{align*}
\alpha^f_0 & :\id{f\circ\tfext^\Gamma}{\tfext^{f_0(\Gamma)}\circ\mftyp{f}{\Gamma}}\\
\alpha^f_1 & :\id{\mftyp{f}{\Gamma}_0\circ\tfext^A}{\tfext^{\mftyp{f}{\Gamma}_0(A)}\circ\mftyp{f}{\ctxext{\Gamma}{A}}}.
\end{align*}
\begin{comment}
the judgmental equality
\begin{equation*}
f_0(\ctxext{\Gamma}{A})\jdeq\ctxext{f_0(\Gamma)}\mftyp{f}{\Gamma}_0(A)
\end{equation*}
for every type $A$ in context $\Gamma$.
\end{comment}
\item preservation of weakening: 
\begin{align*}
\beta^f_0 & :\id{\mftyp{f}{\Gamma}\circ\tfwk^\Gamma}{\tfwk^{f_0(\Gamma)}\circ f}\\
\beta^f_1 & :\id{\mftyp{f}{\ctxext{\Gamma}{A}}\circ\tfwk^A}{\tfwk^\protect{\mftyp{f}{\Gamma}_0(A)}\circ\mftyp{f}{\Gamma}}.
\end{align*}
\begin{comment}
This gives the following three judgmental equalities:
\begin{enumerate}
\item the judgmental equality
\begin{equation*}
\mftyp{f}{\ctxext{\Gamma}{A}}_0(\ctxwk{A}{B})\jdeq\ctxwk{\mftyp{f}{\Gamma}_0(A)}{\mftyp{f}{\Gamma}_0(B)}
\end{equation*}
for every two types $A$ and $B$ in context $\Gamma$.
\item the judgmental equality
\begin{equation*}
\mftyp{f}{\ctxext{{\Gamma}{A}}{\ctxwk{A}{B}}}\circ\tfwk^A\jdeq\tfwk^{\protect{\mftyp{f}{\Gamma}_0(A)}}\circ\mftyp{f}{\ctxext{\Gamma}{B}}
\end{equation*}
\item the judgmental equality
\begin{equation*}
\terms[f]{\ctxwk{A}{B}}{{A}{y}}\jdeq\ctxwk{\mftyp{f}\Gamma_0(A)}{(\terms[f]{B}(y))}
\end{equation*}
\end{enumerate}
\end{comment}
\item preservation of substitution: 
\begin{equation*}
\gamma^f:\id{\mftyp{f}\Gamma_0\circ\tfsubst^x}{\tfsubst^\protect{\terms[f]{A}(x)}\circ\mftyp{f}{\ctxext{\Gamma}{A}}}.
\end{equation*}
\begin{comment}
\begin{enumerate}
\item the judgmental equality
\begin{equation*}
\mftyp{f}\Gamma_0(\subst{x}{P})\jdeq\subst{\terms[f]{A}(x)}{\mftyp{f}{\ctxext{\Gamma}{A}}_0(P)}
\end{equation*}
for every family $P$ over $A$ in context $\Gamma$ and every term $x:A$.
\item the judgmental equality
\begin{equation*}
\mftyp{f}\Gamma_0(\subst{x}{Q})\jdeq\subst{\terms[f]{A}(x)}{\mftyp{f}{\ctxext{{\Gamma}{A}}{P}}_0(Q)}
\end{equation*}
for every family $Q$ over $P$ over $A$ in context $\Gamma$ and every term $x:A$.
\end{enumerate}
\end{comment}
\end{enumerate}
\begin{flushright}
\textsl{End of \autoref{defn:premodel-morphism}.}
\end{flushright}
\end{defn}

\begin{defn}
Suppose that $f:\mfM\to\mfN$ and $g:\mfN\to\mfN'$ are morphisms of internal models.
We define the composition $g\circ f:\mfM\to\mfN'$ to be the morphism given by
\begin{enumerate}
\item $(g\circ f)_0\defeq g_0\circ f_0$
\item $\mftypfunc{g\circ f}(\Gamma)\defeq\mftypfunc{g}(f_0(\Gamma))\circ\mftypfunc{f}(\Gamma)$
\item $\terms[g\circ f]{A}\defeq\terms[g]{\mftyp{f}{\Gamma}_0(A)}\circ\terms[f]{A}$.
\item a definition of $\alpha^{g\circ f}_0$ and $\alpha^{g\circ f}_1$.
\item a definition of $\beta^{g\circ f}_0$ and $\beta^{f\circ f}_1$.
\item a definition of $\gamma^{g\circ f}$. 
\end{enumerate}
\end{defn}

\begin{rmk}
The requirement that a morphism of models acts on terms is reminiscent of the requirement
that a functor acts on morphisms. In this fasion, the requirement that a morphism of
models preserves substitution is the counterpart of a functor preserving composition.
\end{rmk}
Context extension, weakening and substitution are required to be such morphims
of models. Thus they must be extended for this purpose.

\begin{description}
\item[Context extension] We will extend context extension to a morphism 
$\tfext(\Gamma):\mftyp{\mfM}{\Gamma}\to M$. Thus, we already have
$\tfext(\Gamma)_0\defeq\ctxext{\Gamma}{\blank}$. We also require 
\begin{align*}
\mftyp{\tfext(\Gamma)}{A} & : \mftyp{\mftyp{\mfM}{\Gamma}}{A}\to \mftyp{\mfM}{\ctxext{\Gamma}{A}}\\
\terms[\tfext(\Gamma)]{P} & : \terms[\mftyp{\mfM}{\Gamma}]{P}\to\terms[\mfM]{\mftyp{\tfext(\Gamma)}{A}_0(P)}
\end{align*}
For both of these we take the identity function.
We require furthermore that context extension preserves context extension,
weakening and substitution:
\begin{enumerate}
\item Extension preserves extension: for every family $P$ over $A$ in context
$\Gamma$ an identification $\id{\ctxext{{\Gamma}{A}}{P}}{\ctxext{\Gamma}{{A}{P}}}$.
\item Extension preserves weakening:
\item Extension preserves substitution:
Because context extension is the identity on the levels of types and terms,
the rules are easier.
\end{enumerate}
\item[Weakening] is a morphism $\tfwk:\mftyp{\mfM}{\Gamma}\to\mftyp{\mfM}{\ctxext{\Gamma}{A}}$, so we must have
\begin{align*}
\tfwk^A_0 & : \mftyp{\mfM}{\Gamma}\to\mftyp{\mfM}{\ctxext{\Gamma}{A}}
\intertext{which is denoted by $\ctxwk{A}{\blank}$,}
\mftypfunc{\tfwk^A} & : \mftyp{\mfM}{\ctxext{\Gamma}{B}}\to\mftyp{\mfM}{\ctxext{{\Gamma}{A}}{\ctxwk{A}{B}}}\\
\terms[\tfwk^A]{B} & : \terms[\mftyp{\mfM}{\Gamma}]{B}\to\terms[\mftyp{\mfM}{\ctxext{\Gamma}{A}}]{\ctxwk{A}{B}}
\end{align*}
for a term $y:B$, the term $\terms[{\tfwk^A}]{B}(y)$ is the constant map from
$A$ to $B$, assigning $y$ to every term of $A$.
\item[Substitution] is a morphism $\tfsubst^x:\mftyp{\mfM}{\ctxext{\Gamma}{A}}
\to\mftyp{\mfM}{\Gamma}$ for every $x:A$ in context $\Gamma$, so we must have
\begin{align*}
\tfsubst^x_0 & : \mftyp{\mfM}{\ctxext{\Gamma}{A}}\to\mftyp{\mfM}{\Gamma}
\intertext{which is denoted by $\subst{x}{\blank}$,}
\mftyp{\tfsubst^x}{P} & : \mftyp{\mfM}{\ctxext{{\Gamma}{A}}{P}}\to\mftyp{\mfM}{\ctxext{\Gamma}{\subst{x}{P}}}\\
\terms[\tfsubst^x]{P} & : \terms{P}\to\terms{\subst{x}{P}}
\end{align*}
The function $\terms[\tfsubst^x]{P}$ is usually denoted by $\tfev(\blank,x)$. 
\end{description}

\subsection{The basic type constructors in internal models}

\begin{defn}
\begin{enumerate}
\item A type $\mprd{A}{P}$ for every type $A$ in context $\Gamma$ and every type
$P$ in context $\ctxext{\Gamma}{A}$.
\begin{defn}
Suppose $A$ and $B$ are types in context $\Gamma$. We define
\begin{align*}
A\to B & \defeq\mprd{A}{\ctxwk{A}{B}}.
\intertext{For any two contexts $\Delta$ and $\Gamma$, we define}
\Delta\to\Gamma & \defeq e_\unit^{-1}(\Delta)\to e_\unit^{-1}(\Gamma)
\end{align*}
and furthermore we define $\ctxhom{\Delta}{\Gamma}\defeq\terms{\Delta\to\Gamma}$.
\end{defn}
\item An equivalence $\lambda:\eqv{\terms{P}}{\terms{\mprd{A}{P}}}$ for every
family $P$ over $A$ in context $\Gamma$. When $\ctxext{\Gamma}{A}\vdash 
u:P$, we call $\lambda(u)$ the \emph{$\lambda$-abstraction of $u$.}
\begin{rmk}
Note that we get an equivalence $\eqv{\terms{\unit^\Gamma\to A}}{\terms{A}}$ 
for every type $A$ in context $\Gamma$.
\end{rmk}
\begin{rmk}
Thus we see that $\eqv{\terms{A\to B}}{\terms{\ctxwk{A}{B}}}$
by $\lambda$-abstraction. 
\end{rmk}
%\item A function $\tfev:\terms{\mprd{A}{P}}\to\prd{x:\terms{A}}\terms{\subst{x}{P}}$.
\item A term $\pi^A:\ctxext{\Gamma}{A}\to \Gamma$ for every type $A$ in context
$\Gamma$.
\item A term $\iota^x:\ctxext{\Gamma}{\subst{x}{P}}\to\ctxext{{\Gamma}{A}}{P}$
for every family $P$ of types over $A$ in context $\Gamma$ and every
term $x:A$.
\item A type $A[f]$ in context $\Delta$ for every $f:\ctxhom{\Delta}{\Gamma}$ 
and every type $A$ in context $\Gamma$. {\color{blue}Could this be defined
in terms of the substitution $\subst{x}{P}$ we already have? It seems so. If this is
indeed the case we need either some identifications or we could just omit
this part of the definition.}
\item an identification $\id{B[\pi^A]}{\ctxwk{A}{B}}$ for any two types $A$
and $B$ in context $\Gamma$.
\item A type $\msm{A}{P}$ in context $\Gamma$ for every family $P$ over $A$
in context $\Gamma$, with an equivalence $\pairr{\blank,\blank}:\eqv{\sm{x:\terms{A}}\terms{\subst{x}{P}}}
{\terms{\msm{A}{P}}}$.
\item A family $\idtypevar{A}$ over $\ctxwk{A}{A}$ in context $\ctxext{\Gamma}{A}$ for
every type $A$ in context $\Gamma$ and a term of type.
\begin{rmk}
If $A$ and $B$ are types in context $\Gamma$, we may denote the type 
$\msm{A}{\ctxwk{A}{B}}$ by $A\times B$. By the end of the current definition
there will be an identification $\id{\ctxext{{\Gamma}{A}}{P}}
{\ctxext{\Gamma}{\msm{A}{P}}}$.

There is a term $\delta:\ctxhom{\ctxext{\Gamma}{A}}{\ctxext{{\Gamma}{A}}{\ctxwk{A}{A}}}$
defined by...
\end{rmk}
\item For any family $Q$ over $\idtypevar{A}$ in context $\ctxext{{\Gamma}{A}}{\ctxwk{A}{A}}$ an equivalence
$J:\eqv{\terms{Q}}{}$
\end{enumerate}
\end{defn}

\begingroup
\color{blue}
\subsubsection*{More desiderata}
Some of which hopefully follow from more elegant or general rules, but
this list is to keep them in mind:
\begin{enumerate}
\item $\id{e_\unit^{-1}(\Gamma.A)}{\msm{e_\unit^{-1}(\Gamma)}{\trans{(\eta_\Gamma)}{A}}}$, where
$\eta_\Gamma:\id{\Gamma}{e_\unit(e_\unit^{-1}(\Gamma))}$ is the unit of the
equivalence $e_\unit$.  
\item $\id{(\msm{A}{P}\to B)}{(P\to \ctxwk{A}{B})}$, expressing that $\Sigma$ is
left adjoint to weakening.
\item $\id{(\ctxwk{A}{B}\to P)}{(B\to\mprd{A}{P})}$ expressing that $\Pi$ is
right adjoint to weakening.
\item a term $\Gamma\vdash\idfunc[A]:A\to A$ for every type $A$ in context $\Gamma$.
\item a context $\UU$ for the universe.
\item an identification $\id{\pi^A\circ \pi^P\circ\iota^x}{\pi^{\subst{x}{P}}}$ for every
family $P$ of types over $A$ in context $\Gamma$ and every term $x:A$. In other
words, the diagram
\begin{equation*}
\begin{tikzcd}
\ctxext{{\Gamma}{A}}{P} \ar{r}{\pi^P} & \ctxext{\Gamma}{A} \ar{d}{\pi^A}\\
\ctxext{\Gamma}{\subst{x}{P}} \ar{u}{\iota^x} \ar{r}[swap]{\pi^{\subst{x}{P}}} & \Gamma
\end{tikzcd}
\end{equation*}
should commute.
\begin{comment}
\item For $\Gamma:\ctx$, $\Gamma\vdash A:\type$, $\ctxext{\Gamma}{A}\vdash P
:\type$ and $\ctxext{{\Gamma}{A}}{P}\vdash Q:\type$, $x:A$ and $u:P$
we desire that $\typefont{nat{-}ev{-}subst}(Q,u,x):\id{Q[\iota^x][\tfev(u,x)]}{Q[u][x]}$.
\item A function
\begin{equation*}
\tfcomp:\terms{\mprd{P}{Q}}\to\prd{x:\terms{A}}\terms{\mprd{\subst{x}{P}}{Q[\iota^x]}}
\end{equation*}
which could possibly replace $\tfev$ because it could be that $\tfev$ is
definable in terms of $\tfcomp$. It should satisfy
\begin{equation*}
\id{\trans{\typefont{nat{-}ev{-}subst}(Q,u,x)}{\tfev(\tfcomp(f,x),\tfev(u,x))}}{\tfev(\lambda(\tfev(f,u)),x)},
\end{equation*}
which is a naturality condition. There should be an equivalence between
$\terms{\mprd{A}{P}}$ and a type consisting of those terms of $\prd{x:\terms{A}}\terms{\subst{x}{P}}$ which satisfy
a certain naturality condition. One could hope that by formulating this dependently
enough, it is possible to write down such a naturality in one line.
\end{comment}
\end{enumerate}
\endgroup

\begin{comment}
\begingroup
\color{darkgreen}
\subsubsection*{The problem with evaluation.}
...or why I'm doing such strange things. It seems that evaluation is of a
different nature. At first sight, one might expect that evaluation compares
to $\lambda$-abstraction the same way that first and second projection compare
to the pairing function $\pairr{\blank,\blank}$. However, this is not the case
because evaluation cannot be seen as an inverse to $\lambda$-abstraction. The
$\lambda$-abstraction is of type $\terms{P}\to\terms{\mprd{A}{P}}$ whereas
evaluation is of type $\terms{\mprd{A}{P}}\to\prd{x:\terms{A}}\terms{\subst{x}{P}}$. There
are a few things to note about $\tfev$:
\begin{enumerate}
\item There's no function back so in particular $\tfev$ is not an equivalence.
My guess is that there should be a naturality condition playing a role.
\item Evaluation tells how functions interact with substitution of terms.
\end{enumerate}
So it seems that evaluation is like a variable binder in the same way
that quantifiers are binding variables. If I remember right, Voevodsky mentioned
something like this at the Formal Topology conference in Ljubljana, though I
forgot what exactly he said.
\endgroup
\end{comment}

\begin{comment}
\begingroup
\color{gray}
\subsubsection*{Old stuff below:}
\begin{enumerate}
\item a context morphism $\catid{\Gamma}:\ctxhom{\Gamma}{\Gamma}$ for every
context $\Gamma$.
\item a context morphism $f\circ g:\ctxhom{E}{\Gamma}$ for every two context
morphisms $f:\ctxhom{\Delta}{\Gamma}$ and $g:\ctxhom{E}{\Delta}$. 
\item paths $\id{f\circ\catid{\Delta}}{f}$ and $\id{\catid{\Gamma}\circ f}{f}$
for every $f:\ctxhom{\Delta}{\Gamma}$.
\item a path $\id{f\circ (g\circ h)}{(f\circ g)\circ h}$ for every
$f:\ctxhom{\Delta}{\Gamma}$, $g:\ctxhom{E}{\Delta}$ and $h:\ctxhom{Z}{E}$. 
\item a context morphism $\Gamma_!:\ctxhom{\Gamma}{\unit^\mfM}$ 
for every context $\Gamma$.
\item a path $\id{(\unit^\mfM)_!}{\catid{\unit^\mfM}}$.
\item a path $\id{\Gamma_!\circ f}{\Delta_!}$ for every $f:\ctxhom{\Delta}{\Gamma}$. 
\item a path $\id{A[\catid{\Gamma}]}{A}$ for every object $A$ in context $\Gamma$.
\item a path $\id{A[f][g]}{A[f\circ g]}$ for every object $A$ in context $\Gamma$
and every $f:\ctxhom{\Delta}{\Gamma}$ and $g:\ctxhom{E}{\Delta}$. 
\end{enumerate}
\endgroup
\end{comment}

\begin{desiderata}
We should have a theorem stating that our internal model indeed interprets
the rules of type theory.
\end{desiderata}

\begin{comment}
\section{Continuity of functions}
We wish to find a condition under which we have
\begin{equation*}
\eqv{\terms{\mprd{A}{P}}}{\sm{f:\prd{x:\terms{A}}\terms{\subst{x}{P}}}\text{``$f$ is continuous''}}.
\end{equation*}
for all families $P$ over $A$. 
Needless to say, we first need to formulate what we mean by ``$f$ is continuous''.
Thus, we need to find the family $\pairr{p,u,v}\mapsto \id{\trans{p}{u}}{v}$,
which also requires us to define transportation.

Let $P$ be a family over $A$ in context $\Gamma$. Then we can weaken
$\ctxwk{A}{P}$, which is a family over $\ctxwk{A}{A}$ in context $\ctxext{\Gamma}{A}$.
We also have the family $\ctxwk{\ctxwk{A}{A}}{P}$ over $\ctxwk{A}{A}$ in context
$\ctx{\Gamma}{A}$ and thus we can form the family
\begin{equation*}
\ctxwk{\ctxwk{A}{A}}{P}\to\ctxwk{A}{P}.
\end{equation*}
This family plays the role of the family $x,y:A\vdash P(x)\to P(y):\type$.
{\color{red}Explain this by performing the relevant substitution.}
Transportation is a term of the family
\begin{equation*}
\idtypevar{A}\to\ctxwk{\ctxwk{A}{A}}{P}\to\ctxwk{A}{P}
\end{equation*}
obtained by identity elimination, as usual. Thus we obtain a term
\begin{equation*}
\mathcal{T}rans_{x,y}:\terms{\id{x}{y}\to\subst{x}{P}\to\subst{y}{P}}.
\end{equation*}
for every $x,y:A$.

Now let $x,y:A$. Then we have the type $\subst{y}{P}$
in context $\Gamma$ and the family $\ctxwk{\subst{y}{P}}{\subst{x}{P}}$
over $\subst{y}{P}$ in context $\Gamma$. We could weaken $\subst{y}{P}$ subsequently by $A$, $\ctxwk{A}{A}$
and $\idtypevar{A}$ to obtain the family $\ctxwk{\idtypevar{A}}{\ctxwk{\ctxwk{A}{A}}{\ctxwk{A}{\subst{y}{P}}}}$ over $\idtypevar{A}$ in context $\ctxext{{\Gamma}{A}}{\ctxwk{A}{A}}$. We will
denote this family simply by $\ctxwk{\idtypevar{A}}{\subst{y}{P}}$. Likewise,
we obtain the family $\ctxwk{\idtypevar{A}}{\ctxwk{\subst{y}{P}}{\subst{x}{P}}}$ over
$\ctxwk{\idtypevar{A}}{\subst{y}{P}}$ in context $\ctxext{{{\Gamma}{A}}{\ctxwk{A}{A}}}{\idtypevar{A}}$ and we obtain the family $\ctxwk{\idtypevar{A}}{\idtypevar{\subst{y}{P}}}$
over $\ctxwk{\idtypevar{A}}{\ctxwk{\subst{y}{P}}{\subst{y}{P}}}$. Now we
have the family $\subst{\ctxwk{A}{\mathcal{T}rans_{x,y}}}{\ctxwk{\idtypevar{A}}{\idtypevar{\subst{y}{P}}}}$ over
$\ctxwk{\idtypevar{A}}{\ctxwk{\subst{y}{P}}{\subst{x}{P}}}$, which is
the desired family to state the continuity condition:

\begin{defn}
A function $f:\prd{x:\terms{A}}\terms{\subst{x}{P}}$ is said to be \emph{continuous}
if there is a term of type
\begin{equation*}
\prd{x,y:\terms{A}}\terms{\subst{\ctxwk{A}{\mathcal{T}rans_{x,y}}}{\ctxwk{\idtypevar{A}}{\idtypevar{\subst{y}{P}}}}}.
\end{equation*}
\end{defn}

\begin{lem}
If all functions are ``continuous mappings of terms'' then types without terms
are empty, i.e.\ we have
\begin{equation*}
(\terms{A}\to\emptyt)\to\terms{A\to\emptyt^\mfGraph}
\end{equation*}
\end{lem}

\begin{proof}
Any function from a type without terms is vacuously continuous.
\end{proof}

We will see that in the graph model functions are not just continuous mappings
of terms, since there is a graph $\tilde{\emptyt}$ which has no terms but which
differs from the empty graph nevertheless.

\begin{desiderata}
The following may relate to functions being continuous mappings of terms (it
seems to be a very strong condition):
\begin{enumerate}
\item We have the adjunctions $\tfcolim\dashv\Delta\dashv\tflim$ in which the
two monads are idempotent (we don't even have the adjunctions for $\mfGraph$).
\item $\eqv{\terms{\Omega}}{\ctx}$
\end{enumerate}
We also need to check whether all functions in the reflexive graphs have this
property (if not, ditch this section).
\end{desiderata}
\end{comment}

\subsection*{Weak $\omega$-groupoids}

A weak omega groupoid is a model of type theory with $\Sigma$ and $\idtypevar{}$
in which there is a term of $\terms{A}$ for every type $A$ in context $\Gamma$.
