\section{Introduction}
In this paper we define a notion of internal model which is adapted to
the univalent setting in which we work. Thus, we only describe what it means
to be an internal model when the ambient type theory is univalent. Our
presentation is derived from the definition of~\cite{Dybjer1996}. There,
an internal model is defined to be a category with families with interpretations
for the basic type constructors $\Pi$, $\Sigma$ and $\idtypevar{}$. However,
a category in~\cite{Dybjer1996} has a set of objects and for every two objects
a setoid of morphisms. This is too restrictive for our purpose. Ideally, we
would start with an $\infty$-category of contexts and an $\infty$-presheaf
of families over it. This presents us with the problem that a notion of
$\infty$-category has not been formulated \emph{in} type theory. Therefore,
we make a deviation from the approach in~\cite{Dybjer1996}. We shall
start with a type $\tfctx$ of contexts, a family $\mftypfunc:\tfctx\to\type$
of families over it and a family $\terms{\blank}:\prd*{\Gamma:\tfctx}\mftyp{\mfM}{\Gamma}
\to\type$ sending a type $A$ in context $\Gamma$ to its type $\terms{A}$ of
terms; then we shall directly interpret $\Pi$-types, declaring
$\terms{A\to B}$ to be $\ctxhom[\Gamma]{A}{B}$. The idea is that by letting
$\Pi$ play a fundamental role in the definition of a category with families
$\mathcal{C}$,
the morphisms get their properties directly from the internal type theory
of $\mathcal{C}$.

\begin{comment}
We briefly list the data of which an internal model of type theory consits. We
consider internal models in the style of~\cite{Dybjer1996}. Thus we will
describe a model $\mfM$ as a category with families. However, since our setting is
univalent type theory we will deviate from~\cite{Dybjer1996} in the following
respects: first of all, we make use of typical ambiguity and hence omit
reference to a thing called \emph{sort}. 

We organize the definition of an internal model $\mfM$ as follows: in
\autoref{internal-model-contexts} we describe the category of contexts itself;
in \autoref{internal-model-families} we describe the families over a given context
and the related operations; in \autoref{internal-model-constructors} we
describe the basic constructors in the internal model.
\end{comment}

\section{Ideas in the definition}
An internal model of type theory is like a category with families, but we want
to avoid having to state higher coherences. In fact, we don't even start our
definition with a category of contexts; instead we just take a \emph{type} of contexts. 
The morphisms will come from the terms, evaluation of a function at a given
term will come from substitution. We recognize three basic ingredients to models:
first there is a type of contexts; second, for every context there is a model of types in
that context and third, for every type in a given context there is a type of its
terms. Then there are three basic attributes: context extension, weakening and
substitution. Context extension provides us with families over types as well as
with an interpretation of dependent pair types. We need weakening 
so that families can depend on the same type multiple times (the way the
identity type of a type depends two times on that type) and to be able
to talk about non-dependent function types,
the morphims of our category. Substitution will give us a way
to work with fibers of families as well as composition of functions and evaluation
of functions at terms.

Because we require a \emph{model} of types in a context, all the structure
which we require at the bottom level will be required to exist higher up as well.
Thus, the model of types in a given context $\Gamma$ will have a type of contexts
itself, which can be seen as the type of types in $\Gamma$; it will have its
own notion of types in a context, its own notion of terms, context extension,
weakening and substitution together with all the structure require for it. For
instance, when $A$ is a type in context $\Gamma$ in a model $\mfM$, then there
is the model of types in context $A$, which is the model of families over $A$. 
This model is required to be \emph{definitionally equal to} the model of types
in the context $\ctxext{\Gamma}A$, the context extension of $\Gamma$ and $A$.
In this way we protect ourselves from the need to dig an infinitely deep structure
of models when we want to consider examples.

To give the definition of a model we shall also need to consider certain morphisms
of internal models. Those should preserve all the structure: contexts are mapped
to contexts; for every context a morphism of models mapping the model of types
in that context to the model of types in the image of that context; there should
be a mapping of terms and context extension, weakening and substitution should be
preserved. We need to consider those morphisms because we require context extension,
weakening and substitution to be of that kind, thereby respecting each other
in all possible ways.

When we have this framework set up, we can interpret the basic type constructors
such as $\Pi$, $\Sigma$ and $\idtypevar{}$.
The higher categorical structure then comes from the
result that we have an interpretation of type theory.

\begingroup
\color{red}
\begin{rmk}
Currently, the definition seems to be circular. To define a model we need that
context extension, weakening and substitution be morphisms of models. A morphism
of models needs to preserve context extension, weakening and substitution.
Moreover, its definition requires the notion of composition of morphisms.

I've done it this way because it guides me to what the rules should be, trusting
that I can work my way back to give a (possibly less transparent) definition
which contains no circularities without doubt.
\end{rmk}
\endgroup

\section{Univalent universes as internal models}

Before we give the definition we illustrate the concepts that go into it in
the case of a univalent universe $\UU$.
Regardless of the definition of an internal model, a univalent universe should
be an example of one.

The type of contexts is $\UU$ itself and for every $\Gamma:\UU$, a type in
context $\Gamma$ is simply a family $A:\Gamma\to\UU$. The type $\terms{A}$ of terms of a type $A$
in context $\Gamma$ is defined to be $\prd{i:\Gamma}A(i)$. Note that $\Gamma\to
\UU$ itself also interprets type theory where a type in context $A:\Gamma\to\UU$
is a family $P:(\sm{i:\Gamma}A(i))\to\UU$. We may denote this model by
$\mftyp{\UU}{\Gamma}$. The terms of a type $P$ in context $A$ in the model
$\Gamma\to\UU$ are the terms of $\prd{i:\Gamma}{x:A(i)}P(i,x)$.

When $A$ is a type in context
$\Gamma$ we define the context extension $\ctxext{\Gamma}A$ to be
$\sm{i:\Gamma}A(i)$. Note that context extension $\ctxext{\Gamma}\blank$
can be seen acting not only as a function from the context of the model
$\Gamma\to\UU$ to the context of the model $\UU$, but it also acts on the types 
and terms: when $A$ is a type in context $\Gamma$ we may take the identity map
from $\mftyp{\Gamma\to\UU}{A}\to\mftyp{\UU}{\ctxext{\Gamma}A}$ because
$\mftyp{\Gamma\to\UU}{A}$ is taken to be $\mftyp{\UU}{\ctxext{\Gamma}A}$;
thus, $\ctxext{\Gamma}P\defeq P$ for every family $P$ over $A$ in context 
$\Gamma$. Likewise, when $P$ is a type in context $A$ in the model 
$\Gamma\to\UU$ then context extension should act on the terms of $P$ via a
function $\terms{P}\to\terms{\ctxext{\Gamma}P}$, which we take to be
the identity map once more.

Since $\mftyp{\UU}{\Gamma}$ is a model of type theory it
has it's own notion of context extension: when $P$ is a type in context $A$ in
the model $\mftyp{\UU}{\Gamma}$ then $\ctxext{A}P$ is the family
$\lam{i}\sm{x:A(i)}P(i,x):\Gamma\to\UU$. Also the context extension of
$\Gamma\to\UU$ acts trivially on the types and terms. Context extension is
analoguous to the Grothendieck construction that associates the category of
elements to a presheaf and it gives us $\Sigma$-types.

When $A$ and $B$ are types in context
$\Gamma$, the weakening $\ctxwk{A}{B}$ of $B$ along $A$ is defined to be
$\lam{\pairr{i,x}}B(i):(\sm{i:\Gamma}A(i))\to\UU$. Weakening along a type $A$ 
in context $\Gamma$ also acts on types and terms. When $Q$ is a type in context
$B$ in the model $\Gamma\to\UU$, we define $\ctxwk{A}{Q}$ to be 
$\lam{\pairr{i,x}}{y}Q(i,y)$. When $g:\terms{Q}$ we define $\ctxwk{A}{g}$ to
be $\lam{\pairr{i,x}}{y}g(i,y)$.

Note that for two types $A$ and $B$ in context $\Gamma$, the terms of
$\ctxwk{A}{B}$ are the terms of $\prd{i:\Gamma}A(i)\to B(i)$, i.e.~they are
the fiberwise maps from $A$ to $B$. We shall take the type $\terms{\ctxwk{A}{B}}$
of terms of $\ctxwk{A}{B}$ to be the type of morphisms from $A$ to $B$. Also,
any context of $\UU$ may be seen as a type in the empty context $\unit$. Thus
the type of terms of a context $\Gamma$ is $\unit\to\Gamma$, which is
equivalent to $\Gamma$. A context morphism from $\Delta$ to $\Gamma$ is a term
of $\ctxwk{\Gamma}{\Delta}$, i.e.~a function from $\Gamma$ to $\Delta$. We denote
the type $\terms{\ctxwk{\Gamma}{\Delta}}$ by $\ctxhom{\Delta}{\Gamma}$. 

When $P$ is a family over
$A$ in context $\Gamma$ and $x$ is a term of $A$ we define the type $\subst{x}{P}$
in context $\Gamma$ to be $\lam{i}P(i,x(i))$. Like extension 

We will interpret the dependent product $\mprd{A}{P}$ of a family $P$ over
$A$ in context $\Gamma$ by
\begin{equation*}
\mprd{A}{P}(i)\defeq \prd{x:A(i)}P(i,x)
\end{equation*}
With this interpretation there is an equivalence 
$\lambda:\eqv{\terms{P}}{\terms{\mprd{A}{P}}}$. This is lambda-abstraction; its
inverse being evaluation. We should note however, that the rule for evaluation
we interpret here is
\begin{equation*}
\inference{\Gamma\vdash f:\mprd{A}{P}}{\ctxext{\Gamma}A\vdash\tfev(f):P}
\end{equation*}
which is different than the usual rule
\begin{equation*}
\inference{\Gamma\vdash f:\mprd{A}{P} \qquad \Gamma\vdash a:A}{\Gamma \vdash \tfev(f,a):\subst{a}{P}}
\end{equation*}
The reason for this is that the interpretation of the usual rule would give a
function of type $\terms{\mprd{A}{P}}\to\prd{x:\terms{A}}\terms{\subst{x}{P}}$,
but this does not describe the terms of $\mprd{A}{P}$ by any means. Moreover,
since we have implemented substitution, we obtain from $\ctxext{\Gamma}A\vdash\tfev(f):P$ 
and $\Gamma\vdash x:A$ a term $\Gamma\vdash\subst{x}{\tfev(f)}:\subst{x}{P}$ and
therefore we do not loose anything with this approach.

{\color{red} point to example}

\section{A variable-free presentation of type theory}
We have seen in the example of $\UU$ that we will not need to refer to variables
when we are manipulating contexts. In fact, it is more natural not to.
This suggests that there is a presentation
of type theory where contexts do not mention variables. The first small gain is that such a presentation
would not be burdened with comments about variables being bounded or not, or fresh or
not occuring at all.

\subsection{The basic judgments}
The judgments we can make are the standard six basic judgments involving contexts, types and terms and
judgmental equality in each of these three cases.
\begin{align*}
\jctx*{\Gamma} & \jctxeq*{\Gamma}{\Gamma'}\\
\jtype*{\Gamma}{A} & \jtypeeq*{\Gamma}{A}{B}\\
\jterm*{\Gamma}{A}{x} & \jtermeq*{\Gamma}{A}{x}{y}.
\end{align*}

In contrast with standard practise, we don't assume that there is an empty context. The idea is that
if we interpret only the syntax of contexts, types and terms with extension, weakening and
substitution, we do not have a full model of type theory but we still have a 
structure resembling a category and categories need not have terminal objects. Therefore,
the rules that follow will describe only how to manipulate contexts, types and terms.
Of most of the inferences that we give there are two versions: one introducing an operation, the other asserting that the operation in question preserves judgmental equality. 

\subsection{The basic rules for judgmental equality}
The rules for judgmental equality establish that it is an equivalence relation
in all three cases (contexts, types and terms).
%\begingroup
%\renewcommand*{\arraystretch}{3}
%\begin{equation*}
%\begin{array}{ccc}
\begin{infarray}{ccc}
\inference{\jctx{\Gamma}}{\jctxeq{\Gamma}{\Gamma}} & \inference{\jctxeq{\Gamma}{\Delta}}{\jctxeq{\Delta}{\Gamma}} & \inference{\jctxeq{\Gamma}{\Delta}\qquad\jctxeq{\Delta}{\Theta}}{\jctxeq{\Gamma}{\Theta}}\\
\inference{\jtype{\Gamma}{A}}{\jtypeeq{\Gamma}{A}{A}} &
\inference{\jtypeeq{\Gamma}{A}{B}}{\jtypeeq{\Gamma}{B}{A}} & 
\inference{\jtypeeq{\Gamma}{A}{B}\qquad\jtypeeq{\Gamma}{B}{C}}{\jtypeeq{\Gamma}{A}{C}}\\
\inference{\jterm{\Gamma}{A}{x}}{\jtermeq{\Gamma}{A}{x}{x}} & 
\inference{\jtermeq{\Gamma}{A}{x}{y}}{\jtermeq{\Gamma}{A}{y}{x}} &
\inference{\jtermeq{\Gamma}{A}{x}{y}\qquad\jtermeq{\Gamma}{A}{y}{z}}{\jtermeq{\Gamma}{A}{x}{z}}
\end{infarray}
%\end{array}
%\end{equation*}
\begin{infarray}{cc}
\inference{\jctxeq{\Gamma}{\Delta}\qquad\jtype{\Gamma}{A}}{\jtype{\Delta}{A}}
& \inference{\jctxeq{\Gamma}{\Delta}\qquad\jtypeeq{\Gamma}{A}{B}}{\jtypeeq{\Delta}{A}{B}}\\
\inference{\jctxeq{\Gamma}{\Delta}\qquad\jterm{\Gamma}{A}{x}}{\jterm{\Delta}{A}{x}}
& \inference{\jctxeq{\Gamma}{\Delta}\qquad\jtermeq{\Gamma}{A}{x}{y}}{\jtermeq{\Delta}{A}{x}{y}}\\
\inference{\jtypeeq{\Gamma}{A}{B}\qquad \jterm{\Gamma}{A}{x}}{\jterm{\Gamma}{B}{x}}
& \inference{\jtypeeq{\Gamma}{A}{B}\qquad\jtermeq{\Gamma}{A}{x}{y}}{\jtermeq{\Gamma}{B}{x}{y}}
\end{infarray}

\subsection{Extension}
We introduce extension which not only extends a context $\Gamma$ and a type
$A$ over it to a context $\ctxext{\Gamma}A$, but which also extends a type $A$
in context $\Gamma$ and a family $P$ over it to a type $\ctxext{A}P$ in context
$\Gamma$. We do this to ensure that all of type theory can be done in a context.
For instance, we could say (1) that a context in context $\Gamma$ is the same thing
as a type in context $\Gamma$; (2) When $A$ is a context in this sense, a type in
context $A$ is the same thing as a family $P$ over $A$ and (3) when $P$ is a type
in context $A$ in this sense, a term of $P$ keeps its original meaning.

Note that by introducing extension on the level of types and families,
we introduce $\Sigma$-types at a very early stage.
However, we need substitution to make this precise.
\begin{infarray}{cc}
\inference{\jtype{\Gamma}{A}}{\jctx{\ctxext{\Gamma}A}}
& \inference{\jctxeq{\Gamma}{\Delta}\qquad\jtypeeq{\Gamma}{A}{B}}{\jctxeq{\ctxext{\Gamma}A}{\ctxext{\Delta}B}}\\
\inference{\jtype{\ctxext{\Gamma}A}{P}}{\jtype{\Gamma}{\ctxext{A}P}}
& \inference{\jtypeeq{\Gamma}{A}{B}\qquad\jtypeeq{\ctxext{\Gamma}A}{P}{Q}}{\jtypeeq{\Gamma}{\ctxext{A}P}{\ctxext{B}Q}}
\end{infarray}

\subsection{Weakening}
We first define weakening by a context $\Gamma$. Since weakening by $\Gamma$
should in principle be a functor, it acts on contexts, types and terms alike.
Note that it is because of the weakening $\ctxwk{\Gamma}{\Delta}$ of a context
$\Delta$ by $\Gamma$ that we can speak of context morphisms from $\Gamma$ to $\Delta$: they are the terms
of $\ctxwk{\Gamma}{\Delta}$.
\begin{infarray}{cc}
\inference{\jctx{\Gamma}\qquad\jctx{\Delta}}{\jtype{\Gamma}{\ctxwk{\Gamma}{\Delta}}} 
& \inference{\jctxeq{\Gamma}{\Gamma'}\qquad\jctxeq{\Delta}{\Delta'}}{\jtypeeq{\Gamma}{\ctxwk{\Gamma}{\Delta}}{\ctxwk{\Gamma'}{\Delta'}}}\\
\inference{\jctx{\Gamma}\qquad\jtype{\Delta}{B}}{\jtype{\ctxext{\Gamma}\ctxwk{\Gamma}{\Delta}}{\ctxwk{\Gamma}{B}}} & \inference{\jctxeq{\Gamma}{\Gamma'}\qquad\jtypeeq{\Delta}{B}{B'}}{\jtypeeq{\ctxext{\Gamma}\ctxwk{\Gamma}{\Delta}}{\ctxwk{\Gamma}{B}}{\ctxwk{\Gamma'}{B'}}}\\
\inference{\jctx{\Gamma}\qquad\jterm{\Delta}{B}{y}}{\jterm{\ctxext{\Gamma}\ctxwk{\Gamma}{\Delta}}{\ctxwk{\Gamma}{B}}{\ctxwk{\Gamma}{y}}} 
& \inference{\jctxeq{\Gamma}{\Gamma'}\qquad\jtermeq{\Delta}{B}{y}{y'}}{\jtermeq{\ctxext{\Gamma}\ctxwk{\Gamma}{\Delta}}{\ctxwk{\Gamma}{B}}{\ctxwk{\Gamma}{y}}{\ctxwk{\Gamma'}{y'}}} 
\end{infarray}

A weakening operation is also defined for types. When $A$ and $B$ are both types
 in context $\Gamma$, the weakened type $\ctxwk{A}{B}$ in context $\ctxext{\Gamma}A$
 is the family which `doesn't really depend on $A$'. The terms of $\ctxwk{A}{B}$
 are the functions from $A$ to $B$. Likewise, the terms $\ctxwk{A}y$ are the
 constant maps at $y$ from $A$ to $B$, for $y:B$.
\begin{infarray}{cc}
\inference{\jtype{\Gamma}{A}\qquad\jtype{\Gamma}{B}}{\jtype{\ctxext{\Gamma}A}{\ctxwk{A}{B}}}
& \inference{\jtypeeq{\Gamma}{A}{A'}\qquad\jtypeeq{\Gamma}{B}{B'}}{\jtypeeq{\ctxext{\Gamma}A}{\ctxwk{A}{B}}{\ctxwk{A'}{B'}}}\\
\inference{\jtype{\Gamma}{A}\qquad\jtype{\ctxext{\Gamma}B}{Q}}
{\jtype{\ctxext({\Gamma}{A})\ctxwk{A}{B}}{\ctxwk{A}{Q}}}
& \inference{\jtypeeq{\Gamma}{A}{A'}\qquad\jtypeeq{\ctxext{\Gamma}B}{Q}{Q'}}
{\jtypeeq{\ctxext({\Gamma}{A})\ctxwk{A}{B}}{\ctxwk{A}{Q}}{\ctxwk{A'}{Q'}}}\\
\inference{\jtype{\Gamma}{A}\qquad\jterm{\Gamma}{B}{y}}{\jterm{\ctxext{\Gamma}A}{\ctxwk{A}{B}}{\ctxwk{A}{y}}}
& \inference{\jtypeeq{\Gamma}{A}{A'}\qquad\jtermeq{\Gamma}{B}{y}{y'}}{\jtermeq{\ctxext{\Gamma}A}{\ctxwk{A}{B}}{\ctxwk{A}{y}}{\ctxwk{A'}{y'}}}
\end{infarray}

\subsection{Substitution}
\begin{infarray}{cc}
\inference{\jterm{\Gamma}{A}{x}\qquad\jtype{\ctxext{\Gamma}A}{P}}{\jtype{\Gamma}{\subst{x}{P}}}
& \inference{\jtermeq{\Gamma}{A}{x}{x'}\qquad\jtypeeq{\ctxext{\Gamma}A}{P}{P'}}{\jtypeeq{\Gamma}{\subst{x}{P}}{\subst{x'}{P'}}}\\
\inference{\jterm{\Gamma}{A}{x}\qquad\jtype{\ctxext({\Gamma}{A})P}{Q}}{\jtype{\ctxext{\Gamma}\subst{x}{P}}{\subst{x}{Q}}}
& \inference{\jtermeq{\Gamma}{A}{x}{x'}\qquad\jtypeeq{\ctxext({\Gamma}{A})P}{Q}{Q'}}{\jtypeeq{\ctxext{\Gamma}\subst{x}{P}}{\subst{x}{Q}}{\subst{x'}{Q'}}}\\
\inference{\jterm{\Gamma}{A}{x}\qquad\jterm{\ctxext{\Gamma}A}{P}{f}}{\jterm{\Gamma}{\subst{x}{P}}{\subst{x}{f}}}
& \inference{\jtermeq{\Gamma}{A}{x}{x'}\qquad\jtermeq{\ctxext{\Gamma}A}{P}{f}{f'}}{\jtermeq{\Gamma}{\subst{x}{P}}{\subst{x}{f}}{\subst{x'}{f'}}}\\
\inference{\jterm{\Gamma}{A}{x}\qquad\jterm{\ctxext({\Gamma}{A}){P}}{Q}{g}}{\jterm{\ctxext{\Gamma}\subst{x}{P}}{\subst{x}{Q}}{\subst{x}{g}}}
& \inference{\jtermeq{\Gamma}{A}{x}{x'}\qquad\jtermeq{\ctxext({\Gamma}{A})P}{Q}{g}{g'}}{\jtermeq{\ctxext{\Gamma}\subst{x}{P}}{\subst{x}{Q}}{\subst{x}{g}}{\subst{x'}{g'}}}
\end{infarray}

\subsection{Compatibility of extension, weakening and substitution}
We describe the compatibility of extension, weakening and substitution. These
compatibility rules are formulated in a weak sense.

Extension is compatible with extension:
\begin{infarray}{c}
\inference
{ \jtype{\ctxext{\Gamma}A}{P}
}{\jterm
  {\ctxext{\Gamma}({A}P)}
  {\ctxwk{\ctxext{\Gamma}({A}P)}{(\ctxext({\Gamma}{A})P)}}{\typefont{extext}_0(A,P)}
}\\
\inference
{ \jtype{\ctxext({\Gamma}{A})P}{Q}
}{\jterm
  {\ctxext{\Gamma}({A}({P}Q))}
  {\ctxwk{\ctxext{A}({P}Q)}{(\ctxext({A}{P})Q)}}
  {\typefont{extext}_1(A,P,Q)}
}
\end{infarray}

Weakening is compatible with extension:
\begin{infarray}{c}
\inference{
  \jctx{\Gamma}
  \qquad
  \jtype{\Delta}{B}}
  {\jterm
    {\ctxext{\Gamma}\ctxwk{\Gamma}{(\ctxext{\Delta}B)}}
    {\ctxwk{\ctxwk{\Gamma}{(\ctxext{\Delta}B)}}{\ctxext{(\ctxwk{\Gamma}{\Delta})}(\ctxwk{\Gamma}{B})}}
    {\typefont{wkext}_0(\Gamma,B)}}
\end{infarray}

\subsection{More stuff in the type theory}
\begin{infarray}{c}
\inference{\jtype{\Gamma}{A}}{\jterm{\ctxext{\Gamma}A}{\ctxwk{A}{A}}{\idfunc[A]}}
\end{infarray}

\subsection{Leibniz equality on terms}
Occasionally we shall need a weaker version of equality than judgmental equality.
Our choice here is Leibniz equality, since it implies propositional equality when
identity types are present.

Let $P$ be a family over $A$ in context $\Gamma$. Then we can weaken
$\ctxwk{A}{P}$, which is a family over $\ctxwk{A}{A}$ in context $\ctxext{\Gamma}A$.
We also have the family $\ctxwk{\ctxwk{A}{A}}{P}$ over $\ctxwk{A}{A}$ in context
$\ctxext{\Gamma}A$ and thus we have the inference
\begin{equation*}
\inference{\jtype{\ctxext{\Gamma}A}{P}}{\jtype{\ctxext(({\Gamma}{A}){\ctxwk{A}{A}})\ctxwk{A}{P}}{\ctxwk{\ctxwk{A}{P}}{(\ctxwk{\ctxwk{A}{A}}{P})}}}
\end{equation*}
This family plays the role of the family $\jtype{\Gamma,x,y:A}{P(x)\to P(y)}$
in the type theory of \cite{TheBook}.


\subsection{Trivial cofibrations and weak equivalences of types}
We describe a relation between types that expresses when they are weakly equivalent.
Weak equivalence is introduced because we need a weaker notion of judgmental 
equality which also makes sense when identity types are not present, since that
would allow us to state that context extension, weakening and substitution
commute with each other.

A term $f:\ctxwk{A}{B}$ is a trivial cofibration if it has the
property that for any fibration $Q$ over $B$,
to find a section of $Q$ it suffices to find a section of the fibration
$f^\ast Q$ over $A$. In our type theoretical setting, the rôle of fibrations
is played by families, the rôle of the function type $A\to B$ is played by
$\ctxwk{A}{B}$ and our version of the pullback $f^\ast Q$ is $\subst{f}{(\ctxwk{A}{Q})}$.

\begin{defn}
Let $f$ be a term of $\ctxwk{A}{B}$ in context $\Gamma$.
\begin{enumerate}
\item For a family $Q$ over $B$ in context $\Gamma$ we define $f^\ast Q\jdeq\subst{f}{(\ctxwk{A}{Q})}$.
\item For a term $g$ of $Q$ in context $\ctxext{\Gamma}B$ we define $f^\ast g\jdeq\subst{f}{(\ctxwk{A}{g})}$.
\end{enumerate} 
\end{defn}
\begin{rmk}
The type $f^\ast(\ctxwk{B}{C})$ in context $\ctxext{\Gamma}A$ is can be viewed as the
type of functions from $A$ to $C$ which factor through $f$. Thus there should be
a function from $f^\ast(\ctxwk{B}{C})$ to $\ctxwk{A}{C}$.
\end{rmk}

\begin{rmk}
Every term $\jterm{\ctxext{\Gamma}A}{\ctxwk{A}{B}}{f}$ allows us to infer the following:
\begin{equation*}
\inference{\jtype{\ctxext{\Gamma}B}{Q}}{\jtype{\ctxext{\Gamma}A}{f^\ast Q}}
\qquad
\inference{\jterm{\ctxext{\Gamma}B}{Q}{g}}{\jterm{\ctxext{\Gamma}A}{f^\ast Q}{f^\ast g}}
\end{equation*}
\end{rmk}


\begin{defn}
A term $f:\ctxwk{A}{B}$ in context $\ctxext{\Gamma}A$ is said to be a trivial
cofibration if we can infer
\begin{equation*}
\inference{\jterm{\ctxext{\Gamma}A}{f^\ast Q}{t}}{\jterm{\ctxext{\Gamma}B}{Q}{\tilde{t}}}\qquad
\inference{\jterm{\ctxext{\Gamma}A}{f^\ast Q}{t}}{\jtermeq{\ctxext{\Gamma}A}{f^\ast Q}{f^\ast \tilde{t}}{t}}
\end{equation*}
{\color{red}This statement should be reformulated so that it only involves a single judgment...
but I don't see directly how to do that.}
\end{defn}

We have the following theorem in the type theory of \cite{TheBook}, which supports
our claim that we may indeed speak of a trivial cofibration. 

\begin{thm}
Suppose $f:A\to B$ is a function. Then $f$ is an equivalence if and only if
for every $Q:B\to\type$ and every $g:\prd{x:A}Q(f(x))$ there is a section
$h:\prd{y:B}Q(b)$ with the property that $h\circ f\htpy g$. 
\end{thm}

\begin{proof}
We can first take $Q$ to be the constant family $\lam{y}A$. Furthermore, we may
take $g\defeq\idfunc[A]$. Then we get a term of type
\begin{equation*}
\sm{h:B\to A}h\circ f\htpy \idfunc[A],
\end{equation*}
i.e.~we get a left inverse $h$ for $f$. To show that $h$ is also a right inverse
of $f$, let $Q$ be the family $\lam{y}\id{f(h(y))}{y}$. To find a section of
$Q$, which is the homotopy we aim for, it suffices to find a section of
$Q\circ f$. In other words, we have to show that $\id{f(h(f(x)))}{f(x)}$ for
every $x:A$. This follows from the fact that $h$ is a left inverse for $f$.

The reverse direction is immediate.
\end{proof}

Another approach would be to define $f:\ctxwk{A}{B}$ to be left invertible
if there is a term $\ctxext({\Gamma}{A})P$

\subsection{The interactions of extension, weakening and substitution}

\subsection{The basic constructors of type theory}
\subsubsection{Dependent function types}
\begin{infarray}{cc}
\inference{\jtype{\ctxext{\Gamma}A}{P}}{\jtype{\Gamma}{\mprd{A}{P}}}
& \inference{\jtypeeq{\ctxext{\Gamma}A}{P}{P'}}{\jtypeeq{\Gamma}{\mprd{A}{P}}{\mprd{A}{P'}}}\\
\inference{\jterm{\ctxext{\Gamma}A}{P}{f}}{\jterm{\Gamma}{\mprd{A}{P}}{\lambda(f)}}
& \inference{\jtermeq{\ctxext{\Gamma}A}{P}{f}{f'}}{\jtermeq{\Gamma}{\mprd{A}{P}}{\lambda(f)}{\lambda(f')}}\\
\inference{\jterm{\Gamma}{\mprd{A}{P}}{g}}{\jterm{\ctxext{\Gamma}A}{P}{\tfev(g)}}
& \inference{\jtermeq{\Gamma}{\mprd{A}{P}}{g}{g'}}{\jtermeq{\ctxext{\Gamma}A}{P}{\tfev(g)}{\tfev(g')}}\\
\inference{\jterm{\ctxext{\Gamma}A}{P}{f}}{\jtermeq{\Gamma}{A}{\tfev(\lambda(f))}{f}}
\end{infarray}

With these rules we will not get the weak $\eta$-rule when identity types are present.
So it might be better to state that $\lambda$ is a trivial cofibration

\subsection{Identity types}
We need to define the diagonal function $\jterm{\ctxext{\Gamma}A}{\ctxwk{A}{(\ctxext{A}\ctxwk{A}{A})}}{\delta}$
\begin{infarray}{c}
\inference{\jtype{\Gamma}{A}}{\jtype{\ctxext({\Gamma}{A})\ctxwk{A}{A}}{\idtypevar{A}}}\\
\inference{\jtype{\Gamma}{A}}{\jterm{\ctxext{\Gamma}A}{\delta^\ast\idtypevar{A}}{\reflsym}}\\
\inference{\jtype{\ctxext(({\Gamma}{A}){\ctxwk{A}{A}})\idtypevar{A}}{P}\qquad
\jterm{\ctxext{\Gamma}A}{\pairr{\delta,\reflsym}^\ast P}{d}}
{\jterm{\ctxext(({\Gamma}{A}){\ctxwk{A}{A}})\idtypevar{A}}{P}{J}}
\end{infarray}

\section{The category with families $\mfM$}\label{internal-model-contexts}
\begin{defn}\label{defn:premodel}
An internal model $\mfM$ of type theory consists of the following data. 
\begin{enumerate}
\item A type $\tfctx(\mfM)$ of \emph{contexts}.
\item A function $\mftypfunc{\mfM}$ assigning to each context $\Gamma$ of $\mfM$ an internal model
$\mftyp{\mfM}{\Gamma}$ of type theory.
\begin{defn}
The type $\tfctx(\mftyp{\mfM}{\Gamma})$ is denoted by $\mftyp{\mfM}{\Gamma}$. A
term of $\mftyp{\mfM}{\Gamma}$ is called a \emph{type in context $\Gamma$}. To indicate
that $A$ is a type in context $\Gamma$ we also write $\Gamma\vdash A:\mfM$. 
When $P:\mftyp{\mftyp{\mfM}{\Gamma}}{A}$ for some type $A$ in context $\Gamma$, we
also speak of \emph{a family $P$ over $A$ in context $\Gamma$.}
\end{defn}
\item A family $\terms{\blank}:\prd*{\Gamma:\ctx(\mfM)}\mftyp{\mfM}{\Gamma}\to\type$
assigning to every type $A$ in context $\Gamma$ the type $\terms{A}$ of its
terms.
\begin{defn}
When $A$ is a type in context $\Gamma$, we define $\Gamma\vdash x:A$ 
to mean $x:\terms{A}$.
\end{defn}
\item Context extension: a morphism
\begin{equation*}
\tfext^\Gamma:\mftyp{\mfM}{\Gamma}\to\mfM
\end{equation*}
of internal models for every context $\Gamma$.
\begin{defn}
When $A$ is a type in context $\Gamma$, we also denote
the context $\tfext^\Gamma_0(A)$ of $\mfM$ by $\ctxext{\Gamma}A$.
We will write
$\ctxext({\Gamma}{A})P$ for $\ctxext({\Gamma}{A})P$.
\end{defn}
\item The judgmental equalities:
\begin{align*}
\mftyp{\mftyp{\mfM}{\Gamma}}{A} & \jdeq\mftyp{\mfM}{\ctxext{\Gamma}A}\\
\mftyp{\tfext^\Gamma}{A} & \jdeq \modelfont{id}_{\mftyp{\mfM}{\ctxext{\Gamma}A}}\\
\terms[{\tfext^\Gamma}]{P} & \jdeq \idfunc[\terms{P}].
\end{align*}
\begin{rmk}
In particular, we will have the judgmental equalities:
\begin{enumerate}
\item When $A$ is a type in context $\Gamma$ we have
\begin{equation*}
\mftyp{\mftyp{\mfM}{\Gamma}}{A}\jdeq\mftyp{\mfM}{\ctxext{\Gamma}A},
\end{equation*}
ensuring that a context in the model $\mftyp{\mftyp{\mfM}{\Gamma}}{A}$ is the same thing as a
context in the model $\mftyp{\mfM}{\ctxext{\Gamma}A}$.
\item If $Q$ is a family over $P$ where $P$ is a family over $A$ in context $\Gamma$, then
\begin{equation*}
\tfext^{\protect{\mftyp{\mftyp{\mfM}{\Gamma}}{A}}}(P,Q)
\jdeq
\tfext^{\protect{\mftyp{\mfM}{\ctxext{\Gamma}A}}}(P,Q)
\end{equation*}
ensuring that twe two possible notion of context extension are the same.
\end{enumerate}
Other judgmental equalities will be required with the ingredients that follow.
We will not list them all.
\end{rmk}
\item Weakening: a morphism
\begin{equation*}
\tfwk^A:\mftyp{\mfM}{\Gamma}\to\mftyp{\mfM}{\ctxext{\Gamma}A}
\end{equation*}
of internal models for every type $A$ in context $\Gamma$. When $B$ is a type
in context $\Gamma$, we denote $\tfwk^A(B)$ by $\ctxwk{A}{B}$. 
\item Substitution: a morphism
\begin{equation*}
\tfsubst^x:\mftyp{\mfM}{\ctxext{\Gamma}A}\to\mftyp{\mfM}{\Gamma}
\end{equation*}
of internal models for any $x:A$ in context $\Gamma$. When $P$ is a family over $A$ in context
$\Gamma$, we denote $\tfsubst^x(P)$ also by $\subst{x}{P}$. 
\item A judgmental equality
\begin{equation*}
\tfsubst^x\circ\tfwk^A\jdeq\modelfont{id}_{\mftyp{\mfM}{\Gamma}}
\end{equation*}
for any $x:A$ and $B$ in context $\Gamma$.
\item A context $\unit^\mfM:\tfctx(\mfM)$ and a term of type $\isequiv(\ctxext{\unit^\mfM}\blank)$. We will denote
this equivalence by $e_\unit$. The context $\unit^\mfM$ is also
called the \emph{empty context}.
\begin{defn}
For any context $\Gamma$, type $\terms{\Gamma}$ is defined to mean
$\terms{e_\unit^{-1}(\Gamma)}$. 
\end{defn}
\item A section $\unit^{\blank}:\prd{\Gamma:\ctx(\mfM)}\mftyp{\mfM}{\Gamma}$ assigning
a type $\unit^\Gamma$ in context $\Gamma$ to every context $\Gamma$ and
an identification $\alpha_\unit(\Gamma):\id{\ctxext{\Gamma}\unit^\Gamma}{\Gamma}$
for every context $\Gamma$. We also require that there is an identification
$\id{\trans{\alpha_\unit(\Gamma)}{\ctxwk{\unit^\Gamma}{A}}}{A}$ for every
type $A$ in context $\Gamma$.
\item For any type $A$ in context $\Gamma$, a term $\idfunc[A]:\terms{\ctxwk{A}{A}}$
\end{enumerate}
\begin{flushright}
\textsl{End of \autoref{defn:premodel}.}
\end{flushright}
\end{defn}

\begin{defn}
In an internal model $\mfM$ we define
\begin{equation*}
\ctxhom{\Delta}{\Gamma}\defeq \terms{\ctxwk{e_\unit^{-1}(\Delta)}{e_\unit^{-1}(\Gamma)}}
\end{equation*}
\end{defn}

\section{Morphisms of internal models}
\begin{defn}\label{defn:premodel-morphism}
A morphism $f:\mfM\to \mfN$ of internal models consists of
\begin{enumerate}
\item a function $\ctx(f):\ctx(\mfM)\to\ctx(\mfN)$. The function $\ctx(f)$ is also
denoted by $f_0$.
\item a morphism $\mftyp{f}{\Gamma}:\mftyp{\mfM}{\Gamma}\to\mftyp{\mfN}{f_0(\Gamma)}$ of internal models for every
$\Gamma:\ctx(M)$.
\item the judgmental equality
\begin{equation*}
\mftyp{\mftyp{f}{\Gamma}}{A}\jdeq\mftyp{f}{\ctxext{\Gamma}A}
\end{equation*}
\item a function $\terms[f]{A}:\terms[\mfM]{A}\to\terms[\mfN]{\mftyp{f}{\Gamma}_0(A)}$ for
every type $A$ in context $\Gamma$. 
\item preservation of context extension: 
\begin{align*}
\alpha^f_0 & :\id{f\circ\tfext^\Gamma}{\tfext^{f_0(\Gamma)}\circ\mftyp{f}{\Gamma}}\\
\alpha^f_1 & :\id{\mftyp{f}{\Gamma}_0\circ\tfext^A}{\tfext^{\mftyp{f}{\Gamma}_0(A)}\circ\mftyp{f}{\ctxext{\Gamma}A}}.
\end{align*}
\begin{comment}
the judgmental equality
\begin{equation*}
f_0(\ctxext{\Gamma}A)\jdeq\ctxext{f_0(\Gamma)}\mftyp{f}{\Gamma}_0(A)
\end{equation*}
for every type $A$ in context $\Gamma$.
\end{comment}
\item preservation of weakening: 
\begin{align*}
\beta^f_0 & :\id{\mftyp{f}{\Gamma}\circ\tfwk^\Gamma}{\tfwk^{f_0(\Gamma)}\circ f}\\
\beta^f_1 & :\id{\mftyp{f}{\ctxext{\Gamma}A}\circ\tfwk^A}{\tfwk^\protect{\mftyp{f}{\Gamma}_0(A)}\circ\mftyp{f}{\Gamma}}.
\end{align*}
\begin{comment}
This gives the following three judgmental equalities:
\begin{enumerate}
\item the judgmental equality
\begin{equation*}
\mftyp{f}{\ctxext{\Gamma}A}_0(\ctxwk{A}{B})\jdeq\ctxwk{\mftyp{f}{\Gamma}_0(A)}{\mftyp{f}{\Gamma}_0(B)}
\end{equation*}
for every two types $A$ and $B$ in context $\Gamma$.
\item the judgmental equality
\begin{equation*}
\mftyp{f}{\ctxext({\Gamma}{A})\ctxwk{A}{B}}\circ\tfwk^A\jdeq\tfwk^{\protect{\mftyp{f}{\Gamma}_0(A)}}\circ\mftyp{f}{\ctxext{\Gamma}B}
\end{equation*}
\item the judgmental equality
\begin{equation*}
\terms[f]{\ctxwk{A}{B}}(\ctxwk{A}{y})\jdeq\ctxwk{\mftyp{f}\Gamma_0(A)}{(\terms[f]{B}(y))}
\end{equation*}
\end{enumerate}
\end{comment}
\item preservation of substitution: 
\begin{equation*}
\gamma^f:\id{\mftyp{f}\Gamma_0\circ\tfsubst^x}{\tfsubst^\protect{\terms[f]{A}(x)}\circ\mftyp{f}{\ctxext{\Gamma}A}}.
\end{equation*}
\begin{comment}
\begin{enumerate}
\item the judgmental equality
\begin{equation*}
\mftyp{f}\Gamma_0(\subst{x}{P})\jdeq\subst{\terms[f]{A}(x)}{\mftyp{f}{\ctxext{\Gamma}A}_0(P)}
\end{equation*}
for every family $P$ over $A$ in context $\Gamma$ and every term $x:A$.
\item the judgmental equality
\begin{equation*}
\mftyp{f}\Gamma_0(\subst{x}{Q})\jdeq\subst{\terms[f]{A}(x)}{\mftyp{f}{\ctxext({\Gamma}{A})P}_0(Q)}
\end{equation*}
for every family $Q$ over $P$ over $A$ in context $\Gamma$ and every term $x:A$.
\end{enumerate}
\end{comment}
\end{enumerate}
\begin{flushright}
\textsl{End of \autoref{defn:premodel-morphism}.}
\end{flushright}
\end{defn}

\begin{defn}
Suppose that $f:\mfM\to\mfN$ and $g:\mfN\to\mfN'$ are morphisms of internal models.
We define the composition $g\circ f:\mfM\to\mfN'$ to be the morphism given by
\begin{enumerate}
\item $(g\circ f)_0\defeq g_0\circ f_0$
\item $\mftypfunc{g\circ f}(\Gamma)\defeq\mftypfunc{g}(f_0(\Gamma))\circ\mftypfunc{f}(\Gamma)$
\item $\terms[g\circ f]{A}\defeq\terms[g]{\mftyp{f}{\Gamma}_0(A)}\circ\terms[f]{A}$.
\item a definition of $\alpha^{g\circ f}_0$ and $\alpha^{g\circ f}_1$.
\item a definition of $\beta^{g\circ f}_0$ and $\beta^{f\circ f}_1$.
\item a definition of $\gamma^{g\circ f}$. 
\end{enumerate}
\end{defn}

\begin{rmk}
The requirement that a morphism of models acts on terms is reminiscent of the requirement
that a functor acts on morphisms. In this fasion, the requirement that a morphism of
models preserves substitution is the counterpart of a functor preserving composition.
\end{rmk}
Context extension, weakening and substitution are required to be such morphims
of models. Thus they must be extended for this purpose.

\begin{description}
\item[Context extension] We will extend context extension to a morphism 
$\tfext(\Gamma):\mftyp{\mfM}{\Gamma}\to M$. Thus, we already have
$\tfext(\Gamma)_0\defeq\ctxext{\Gamma}\blank$. We also require 
\begin{align*}
\mftyp{\tfext(\Gamma)}{A} & : \mftyp{\mftyp{\mfM}{\Gamma}}{A}\to \mftyp{\mfM}{\ctxext{\Gamma}A}\\
\terms[\tfext(\Gamma)]{P} & : \terms[\mftyp{\mfM}{\Gamma}]{P}\to\terms[\mfM]{\mftyp{\tfext(\Gamma)}{A}_0(P)}
\end{align*}
For both of these we take the identity function.
We require furthermore that context extension preserves context extension,
weakening and substitution:
\begin{enumerate}
\item Extension preserves extension: for every family $P$ over $A$ in context
$\Gamma$ an identification $\id{\ctxext({\Gamma}{A})P}{\ctxext{\Gamma}({A}P)}$.
\item Extension preserves weakening:
\item Extension preserves substitution:
Because context extension is the identity on the levels of types and terms,
the rules are easier.
\end{enumerate}
\item[Weakening] is a morphism $\tfwk:\mftyp{\mfM}{\Gamma}\to\mftyp{\mfM}{\ctxext{\Gamma}A}$, so we must have
\begin{align*}
\tfwk^A_0 & : \mftyp{\mfM}{\Gamma}\to\mftyp{\mfM}{\ctxext{\Gamma}A}
\intertext{which is denoted by $\ctxwk{A}{\blank}$,}
\mftypfunc{\tfwk^A} & : \mftyp{\mfM}{\ctxext{\Gamma}B}\to\mftyp{\mfM}{\ctxext({\Gamma}{A})\ctxwk{A}{B}}\\
\terms[\tfwk^A]{B} & : \terms[\mftyp{\mfM}{\Gamma}]{B}\to\terms[\mftyp{\mfM}{\ctxext{\Gamma}A}]{\ctxwk{A}{B}}
\end{align*}
for a term $y:B$, the term $\terms[{\tfwk^A}]{B}(y)$ is the constant map from
$A$ to $B$, assigning $y$ to every term of $A$.
\item[Substitution] is a morphism $\tfsubst^x:\mftyp{\mfM}{\ctxext{\Gamma}A}
\to\mftyp{\mfM}{\Gamma}$ for every $x:A$ in context $\Gamma$, so we must have
\begin{align*}
\tfsubst^x_0 & : \mftyp{\mfM}{\ctxext{\Gamma}A}\to\mftyp{\mfM}{\Gamma}
\intertext{which is denoted by $\subst{x}{\blank}$,}
\mftyp{\tfsubst^x}{P} & : \mftyp{\mfM}{\ctxext({\Gamma}{A})P}\to\mftyp{\mfM}{\ctxext{\Gamma}\subst{x}{P}}\\
\terms[\tfsubst^x]{P} & : \terms{P}\to\terms{\subst{x}{P}}
\end{align*}
The function $\terms[\tfsubst^x]{P}$ is usually denoted by $\tfev(\blank,x)$. 
\end{description}

\section{The basic type constructors in internal models}

\begin{defn}
\begin{enumerate}
\item A type $\mprd{A}{P}$ for every type $A$ in context $\Gamma$ and every type
$P$ in context $\ctxext{\Gamma}A$.
\begin{defn}
Suppose $A$ and $B$ are types in context $\Gamma$. We define
\begin{align*}
A\to B & \defeq\mprd{A}{\ctxwk{A}{B}}.
\intertext{For any two contexts $\Delta$ and $\Gamma$, we define}
\Delta\to\Gamma & \defeq e_\unit^{-1}(\Delta)\to e_\unit^{-1}(\Gamma)
\end{align*}
and furthermore we define $\ctxhom{\Delta}{\Gamma}\defeq\terms{\Delta\to\Gamma}$.
\end{defn}
\item An equivalence $\lambda:\eqv{\terms{P}}{\terms{\mprd{A}{P}}}$ for every
family $P$ over $A$ in context $\Gamma$. When $\ctxext{\Gamma}A\vdash 
u:P$, we call $\lambda(u)$ the \emph{$\lambda$-abstraction of $u$.}
\begin{rmk}
Note that we get an equivalence $\eqv{\terms{\unit^\Gamma\to A}}{\terms{A}}$ 
for every type $A$ in context $\Gamma$.
\end{rmk}
\begin{rmk}
Thus we see that $\eqv{\terms{A\to B}}{\terms{\ctxwk{A}{B}}}$
by $\lambda$-abstraction. 
\end{rmk}
%\item A function $\tfev:\terms{\mprd{A}{P}}\to\prd{x:\terms{A}}\terms{\subst{x}{P}}$.
\item A term $\pi^A:\ctxext{\Gamma}A\to \Gamma$ for every type $A$ in context
$\Gamma$.
\item A term $\iota^x:\ctxext{\Gamma}\subst{x}{P}\to\ctxext({\Gamma}{A})P$
for every family $P$ of types over $A$ in context $\Gamma$ and every
term $x:A$.
\item A type $A[f]$ in context $\Delta$ for every $f:\ctxhom{\Delta}{\Gamma}$ 
and every type $A$ in context $\Gamma$. {\color{blue}Could this be defined
in terms of the substitution $\subst{x}{P}$ we already have? It seems so. If this is
indeed the case we need either some identifications or we could just omit
this part of the definition.}
\item an identification $\id{B[\pi^A]}{\ctxwk{A}{B}}$ for any two types $A$
and $B$ in context $\Gamma$.
\item A type $\msm{A}{P}$ in context $\Gamma$ for every family $P$ over $A$
in context $\Gamma$, with an equivalence $\pairr{\blank,\blank}:\eqv{\sm{x:\terms{A}}\terms{\subst{x}{P}}}
{\terms{\msm{A}{P}}}$.
\item A family $\idtypevar{A}$ over $\ctxwk{A}{A}$ in context $\ctxext{\Gamma}A$ for
every type $A$ in context $\Gamma$ and a term of type.
\begin{rmk}
If $A$ and $B$ are types in context $\Gamma$, we may denote the type 
$\msm{A}{\ctxwk{A}{B}}$ by $A\times B$. By the end of the current definition
there will be an identification $\id{\ctxext({\Gamma}{A})P}
{\ctxext{\Gamma}\msm{A}{P}}$.

There is a term $\delta:\ctxhom{\ctxext{\Gamma}A}{\ctxext({\Gamma}{A})\ctxwk{A}{A}}$
defined by...
\end{rmk}
\item For any family $Q$ over $\idtypevar{A}$ in context $\ctxext({\Gamma}{A})\ctxwk{A}{A}$ an equivalence
$J:\eqv{\terms{Q}}{}$
\end{enumerate}
\end{defn}

\begingroup
\color{blue}
\subsubsection*{More desiderata}
Some of which hopefully follow from more elegant or general rules, but
this list is to keep them in mind:
\begin{enumerate}
\item $\id{e_\unit^{-1}(\Gamma.A)}{\msm{e_\unit^{-1}(\Gamma)}{\trans{(\eta_\Gamma)}{A}}}$, where
$\eta_\Gamma:\id{\Gamma}{e_\unit(e_\unit^{-1}(\Gamma))}$ is the unit of the
equivalence $e_\unit$.  
\item $\id{(\msm{A}{P}\to B)}{(P\to \ctxwk{A}{B})}$, expressing that $\Sigma$ is
left adjoint to weakening.
\item $\id{(\ctxwk{A}{B}\to P)}{(B\to\mprd{A}{P})}$ expressing that $\Pi$ is
right adjoint to weakening.
\item a term $\Gamma\vdash\idfunc[A]:A\to A$ for every type $A$ in context $\Gamma$.
\item a context $\UU$ for the universe.
\item an identification $\id{\pi^A\circ \pi^P\circ\iota^x}{\pi^{\subst{x}{P}}}$ for every
family $P$ of types over $A$ in context $\Gamma$ and every term $x:A$. In other
words, the diagram
\begin{equation*}
\begin{tikzcd}
\ctxext({\Gamma}{A})P \ar{r}{\pi^P} & \ctxext{\Gamma}A \ar{d}{\pi^A}\\
\ctxext{\Gamma}\subst{x}{P} \ar{u}{\iota^x} \ar{r}[swap]{\pi^{\subst{x}{P}}} & \Gamma
\end{tikzcd}
\end{equation*}
should commute.
\begin{comment}
\item For $\Gamma:\ctx$, $\Gamma\vdash A:\type$, $\ctxext{\Gamma}A\vdash P
:\type$ and $\ctxext({\Gamma}{A})P\vdash Q:\type$, $x:A$ and $u:P$
we desire that $\typefont{nat{-}ev{-}subst}(Q,u,x):\id{Q[\iota^x][\tfev(u,x)]}{Q[u][x]}$.
\item A function
\begin{equation*}
\tfcomp:\terms{\mprd{P}{Q}}\to\prd{x:\terms{A}}\terms{\mprd{\subst{x}{P}}{Q[\iota^x]}}
\end{equation*}
which could possibly replace $\tfev$ because it could be that $\tfev$ is
definable in terms of $\tfcomp$. It should satisfy
\begin{equation*}
\id{\trans{\typefont{nat{-}ev{-}subst}(Q,u,x)}{\tfev(\tfcomp(f,x),\tfev(u,x))}}{\tfev(\lambda(\tfev(f,u)),x)},
\end{equation*}
which is a naturality condition. There should be an equivalence between
$\terms{\mprd{A}{P}}$ and a type consisting of those terms of $\prd{x:\terms{A}}\terms{\subst{x}{P}}$ which satisfy
a certain naturality condition. One could hope that by formulating this dependently
enough, it is possible to write down such a naturality in one line.
\end{comment}
\end{enumerate}
\endgroup

\begin{comment}
\begingroup
\color{darkgreen}
\subsubsection*{The problem with evaluation.}
...or why I'm doing such strange things. It seems that evaluation is of a
different nature. At first sight, one might expect that evaluation compares
to $\lambda$-abstraction the same way that first and second projection compare
to the pairing function $\pairr{\blank,\blank}$. However, this is not the case
because evaluation cannot be seen as an inverse to $\lambda$-abstraction. The
$\lambda$-abstraction is of type $\terms{P}\to\terms{\mprd{A}{P}}$ whereas
evaluation is of type $\terms{\mprd{A}{P}}\to\prd{x:\terms{A}}\terms{\subst{x}{P}}$. There
are a few things to note about $\tfev$:
\begin{enumerate}
\item There's no function back so in particular $\tfev$ is not an equivalence.
My guess is that there should be a naturality condition playing a role.
\item Evaluation tells how functions interact with substitution of terms.
\end{enumerate}
So it seems that evaluation is like a variable binder in the same way
that quantifiers are binding variables. If I remember right, Voevodsky mentioned
something like this at the Formal Topology conference in Ljubljana, though I
forgot what exactly he said.
\endgroup
\end{comment}

\begin{comment}
\begingroup
\color{gray}
\subsubsection*{Old stuff below:}
\begin{enumerate}
\item a context morphism $\catid{\Gamma}:\ctxhom{\Gamma}{\Gamma}$ for every
context $\Gamma$.
\item a context morphism $f\circ g:\ctxhom{E}{\Gamma}$ for every two context
morphisms $f:\ctxhom{\Delta}{\Gamma}$ and $g:\ctxhom{E}{\Delta}$. 
\item paths $\id{f\circ\catid{\Delta}}{f}$ and $\id{\catid{\Gamma}\circ f}{f}$
for every $f:\ctxhom{\Delta}{\Gamma}$.
\item a path $\id{f\circ (g\circ h)}{(f\circ g)\circ h}$ for every
$f:\ctxhom{\Delta}{\Gamma}$, $g:\ctxhom{E}{\Delta}$ and $h:\ctxhom{Z}{E}$. 
\item a context morphism $\Gamma_!:\ctxhom{\Gamma}{\unit^\mfM}$ 
for every context $\Gamma$.
\item a path $\id{(\unit^\mfM)_!}{\catid{\unit^\mfM}}$.
\item a path $\id{\Gamma_!\circ f}{\Delta_!}$ for every $f:\ctxhom{\Delta}{\Gamma}$. 
\item a path $\id{A[\catid{\Gamma}]}{A}$ for every object $A$ in context $\Gamma$.
\item a path $\id{A[f][g]}{A[f\circ g]}$ for every object $A$ in context $\Gamma$
and every $f:\ctxhom{\Delta}{\Gamma}$ and $g:\ctxhom{E}{\Delta}$. 
\end{enumerate}
\endgroup
\end{comment}

\begin{desiderata}
We should have a theorem stating that our internal model indeed interprets
the rules of type theory.
\end{desiderata}

\begin{comment}
\section{Continuity of functions}
We wish to find a condition under which we have
\begin{equation*}
\eqv{\terms{\mprd{A}{P}}}{\sm{f:\prd{x:\terms{A}}\terms{\subst{x}{P}}}\text{``$f$ is continuous''}}.
\end{equation*}
for all families $P$ over $A$. 
Needless to say, we first need to formulate what we mean by ``$f$ is continuous''.
Thus, we need to find the family $\pairr{p,u,v}\mapsto \id{\trans{p}{u}}{v}$,
which also requires us to define transportation.

Let $P$ be a family over $A$ in context $\Gamma$. Then we can weaken
$\ctxwk{A}{P}$, which is a family over $\ctxwk{A}{A}$ in context $\ctxext{\Gamma}A$.
We also have the family $\ctxwk{\ctxwk{A}{A}}{P}$ over $\ctxwk{A}{A}$ in context
$\ctx{\Gamma}{A}$ and thus we can form the family
\begin{equation*}
\ctxwk{\ctxwk{A}{A}}{P}\to\ctxwk{A}{P}.
\end{equation*}
This family plays the role of the family $x,y:A\vdash P(x)\to P(y):\type$.
{\color{red}Explain this by performing the relevant substitution.}
Transportation is a term of the family
\begin{equation*}
\idtypevar{A}\to\ctxwk{\ctxwk{A}{A}}{P}\to\ctxwk{A}{P}
\end{equation*}
obtained by identity elimination, as usual. Thus we obtain a term
\begin{equation*}
\mathcal{T}rans_{x,y}:\terms{\id{x}{y}\to\subst{x}{P}\to\subst{y}{P}}.
\end{equation*}
for every $x,y:A$.

Now let $x,y:A$. Then we have the type $\subst{y}{P}$
in context $\Gamma$ and the family $\ctxwk{\subst{y}{P}}{\subst{x}{P}}$
over $\subst{y}{P}$ in context $\Gamma$. We could weaken $\subst{y}{P}$ subsequently by $A$, $\ctxwk{A}{A}$
and $\idtypevar{A}$ to obtain the family $\ctxwk{\idtypevar{A}}{\ctxwk{\ctxwk{A}{A}}{\ctxwk{A}{\subst{y}{P}}}}$ over $\idtypevar{A}$ in context $\ctxext({\Gamma}{A})\ctxwk{A}{A}$. We will
denote this family simply by $\ctxwk{\idtypevar{A}}{\subst{y}{P}}$. Likewise,
we obtain the family $\ctxwk{\idtypevar{A}}{\ctxwk{\subst{y}{P}}{\subst{x}{P}}}$ over
$\ctxwk{\idtypevar{A}}{\subst{y}{P}}$ in context $\ctxext(({\Gamma}{A}){\ctxwk{A}{A}})\idtypevar{A}$ and we obtain the family $\ctxwk{\idtypevar{A}}{\idtypevar{\subst{y}{P}}}$
over $\ctxwk{\idtypevar{A}}{\ctxwk{\subst{y}{P}}{\subst{y}{P}}}$. Now we
have the family $\subst{\ctxwk{A}{\mathcal{T}rans_{x,y}}}{\ctxwk{\idtypevar{A}}{\idtypevar{\subst{y}{P}}}}$ over
$\ctxwk{\idtypevar{A}}{\ctxwk{\subst{y}{P}}{\subst{x}{P}}}$, which is
the desired family to state the continuity condition:

\begin{defn}
A function $f:\prd{x:\terms{A}}\terms{\subst{x}{P}}$ is said to be \emph{continuous}
if there is a term of type
\begin{equation*}
\prd{x,y:\terms{A}}\terms{\subst{\ctxwk{A}{\mathcal{T}rans_{x,y}}}{\ctxwk{\idtypevar{A}}{\idtypevar{\subst{y}{P}}}}}.
\end{equation*}
\end{defn}

\begin{lem}
If all functions are ``continuous mappings of terms'' then types without terms
are empty, i.e.\ we have
\begin{equation*}
(\terms{A}\to\emptyt)\to\terms{A\to\emptyt^\mfGraph}
\end{equation*}
\end{lem}

\begin{proof}
Any function from a type without terms is vacuously continuous.
\end{proof}

We will see that in the graph model functions are not just continuous mappings
of terms, since there is a graph $\tilde{\emptyt}$ which has no terms but which
differs from the empty graph nevertheless.

\begin{desiderata}
The following may relate to functions being continuous mappings of terms (it
seems to be a very strong condition):
\begin{enumerate}
\item We have the adjunctions $\tfcolim\dashv\Delta\dashv\tflim$ in which the
two monads are idempotent (we don't even have the adjunctions for $\mfGraph$).
\item $\eqv{\terms{\Omega}}{\ctx}$
\end{enumerate}
We also need to check whether all functions in the reflexive graphs have this
property (if not, ditch this section).
\end{desiderata}
\end{comment}

\section{The graph model of type theory}
In this section we define the graph model of type theory, denoted by
$\mfGraph$.  In our presentation, we follow that of the definition internal
models. After we have established the graph model, we show how the graph
model can be seen as a presheaf model, namely over the category $\cdot
{\rightrightarrows}\cdot$.

\begin{defn}
A \emph{(directed) graph} $\Gamma$ is a pair $\pairr{\Gamma_0,\Gamma_1}$ 
consisting of a type $\Gamma_0$ of vertices and a family 
$\Gamma_1:\Gamma_0\to\Gamma_0\to\type$ of edges. The type $\ctx(\mfGraph)$
is defined to be the type of all graphs; we will usually denote it by
$\tfGraph$. Explicitly, we have
\begin{equation*}
\tfGraph\defeq\sm{\Gamma_0:\type}\Gamma_0\to\Gamma_0\to\type.
\end{equation*}
\end{defn}

\begin{eg}\label{ex:pb}
The underlying graph of the diagram
\begin{equation*}
\begin{tikzcd}
{} & A \ar{d}{f} \\
B \ar{r}[swap]{g} & C
\end{tikzcd}
\begin{comment}
\begin{tikzpicture}
\matrix (m) [std] { & A \\ B & C \\};
\draw[ar] (m-1-2) -- node[right] {$f$} (m-2-2);
\draw[ar] (m-2-1) -- node[below] {$g$} (m-2-2);
\end{tikzpicture}
\end{comment}
\end{equation*}
is $I\defeq\mathbf{3}$ and has $J(1,3)\defeq J(2,3)\defeq \unit $ 
and $J(x,y)\defeq\emptyt$ otherwise (it is defined using the induction 
principle of $\mathbf{3}$ and a universe).
\end{eg}


\begin{defn}
A \emph{graph $A$ in the context $\Gamma$} is a pair $\pairr{A_0,A_1}$ consisting
of 
\begin{align*}
A_0 & :\Gamma_0\to\type\\
A_1 & :\prd*{i,j:\Gamma_0}\Gamma_1(i,j)\to A_0(i)\to A_0(j)\to\type.
\end{align*}
Thus, for a graph $\Gamma$, the type $\mftyp{\mfGraph}{\Gamma}$ is the type
\begin{equation*}
\sm{A_0:\Gamma_0\to\type}\prd*{i,j:\Gamma_0}\Gamma_1(i,j)\to A_0(i)\to A_0(j)\to\type.
\end{equation*}
We will also write $\Gamma\vdash A:\mfGraph$ when $A$ is a graph in context
$\Gamma$.
\end{defn}

\begin{defn}
Suppose that $\Gamma\vdash A:\mfGraph$. Then we define the graph $\ctxext{\Gamma}A$
by
\begin{align*}
\ctxext{\Gamma}A_0 & \defeq \sm{i:\Gamma_0}A_0(i)\\
\ctxext{\Gamma}A_1(\pairr{i,x},\pairr{j,y}) & \defeq \sm{q:\Gamma_1(i,j)}A_1(q,x,y).
\end{align*}
\end{defn}

\begin{defn}
Suppose that $\Gamma\vdash A:\mfGraph$ and $\Gamma\vdash B:\mfGraph$. Then we
define the graph $\ctxwk{A}{B}$ in context $\ctxext{\Gamma}A$ by
\begin{align*}
(\ctxwk{A}{B})_0(\pairr{i,x}) & \defeq B_0(i)\\
(\ctxwk{A}{B})_1(\pairr{q,e},u,v) & \defeq B_1(q,u,v).
\end{align*}
\end{defn}

\begin{defn}
Suppose that $\Gamma\vdash A:\mfGraph$. A term $x$ of $A$ consists of a pair
$\pairr{x_0,x_1}$ where
\begin{align*}
x_0 & : \prd{i:\Gamma_0}A_0(i)\\
x_1 & : \prd*{i,j:\Gamma_0}{q:\Gamma_1}A_1(q,x_0(i),x_0(j)).
\end{align*}
Thus we define
\begin{equation*}
\terms{A}\defeq\sm{x_0:\prd{i:\Gamma_0}A_0(i)}\prd*{i,j:\Gamma_0}{q:\Gamma_1}A_1(q,x_0(i),x_0(j)).
\end{equation*}
We also write $\Gamma\vdash x:A$ when $x$ is a term of the graph $A$ in context
$\Gamma$.
\end{defn}

\begin{defn}
Suppose that $P$ is a family of graphs over $A$ in context $\Gamma$ and let
$x$ be a term of $A$. Then we define the graph $\subst{x}{P}$ 
in context $\Gamma$ by
\begin{align*}
\subst{x}{P}_0(i) & \defeq P_0(\pairr{i,x_0(i)})\\
\subst{x}{P}_1(q,u,v) & \defeq P_1(\pairr{q,x_1(q)},u,v)
\end{align*}
\end{defn}

\begin{rmk}
Note that we have the judgmental equality $\subst{x}{\ctxwk{A}{B}}\jdeq B$
for every two graphs $A$ and $B$ in context $\Gamma$ and every term $x$ of $A$.
\end{rmk}

\begin{defn}
The terminal graph $\unit^\mfGraph$, which we shall often denote simply 
by $\unit$, is defined by
\begin{align*}
{\unit^\mfGraph}_0 & \defeq \unit\\
{\unit^\mfGraph}_1(x,y) & \defeq \unit.
\end{align*}
\end{defn}

\begin{rmk}
We note that the function $\ctxext{\unit}\blank$ is an equivalence
from $\mftyp{\mfGraph}{\unit}$ to $\tfGraph$. It's inverse is the function which maps
a graph $\Gamma$ to the pair $\pairr{A_0,A_1}$ where $A_0$ is defined by
$A_0(\ttt)\defeq\Gamma_0$ and where $A_1$ is defined by $A_1(\ttt)\defeq
\Gamma_1$. 

Note that we only have an equivalence here, not a judgmental equality.
\end{rmk}

\subsubsection{The contexts of $\mfGraph$}


The category $\psh(\cdot{\rightrightarrows}\cdot)$ of presheaves over
$\cdot{\rightrightarrows}\cdot$ is given by
\begin{equation*}
\sm{A_0,A_1:\type}(A_1\to A_0)\times(A_1\to A_0).
\end{equation*}
To see that $\psh(\cdot{\rightrightarrows}\cdot)$ is indeed equivalent to
the type $\tfGraph$ of all graphs, note that we have the equivalences
\begin{align*}
\psh(\cdot{\rightrightarrows}\cdot) & \eqvsym \sm{A_0,A_1:\type}A_1\to A_0\times A_0\\
& \eqvsym \sm{A_0:\type} A_0\times A_0\to\type\\
& \eqvsym \tfGraph.
\end{align*}

\begin{defn}
Suppose that $\Delta\defeq \pairr{\Delta_0,\Delta_1}$ and $\Gamma\defeq \pairr{\Gamma_0,
\Gamma_1}$ are graphs. A morphism $f$ from $\Delta$ to $\Gamma$ is a pair
$\pairr{f_0,f_1}$ consisting of
\begin{align*}
f_0 & : \Delta_0\to\Gamma_0\\
f_1 & : \prd*{u,v:\Delta_0}\Delta_1(u,v)\to\Gamma_1(f_0(u),f_0(v))
\end{align*}
We write $\ctxhom{\Delta}{\Gamma}$ for the type of morphisms of graphs.
\end{defn}

The terminal graph $\unit^\mfGraph$ plays the role of the empty context.

\begin{defn}
The terminal graph $\unit^\mfGraph$, which we shall often denote simply 
by $\unit$, is defined by
\begin{align*}
{\unit^\mfGraph}_0 & \defeq \unit\\
{\unit^\mfGraph}_1(x,y) & \defeq \unit.
\end{align*}
\end{defn}

It is immediate that the type of graph morphisms $\Gamma\to\unit^\mfGraph$ is
contractible for every graph $\Gamma$. The terms of a graph $\Gamma$ are the
global elements of $\Gamma$, i.e.~they are the morphisms of type 
$\ctxhom{\unit^\mfGraph}{\Gamma}$.

\begin{defn}
Let $\Gamma$ be a graph. The type $\terms[\mfGraph]{\Gamma}$ of terms of $\Gamma$ is
defined to be
\begin{equation*}
\sm{x_0:\Gamma_0}\Gamma_1(x_0,x_0).
\end{equation*}
\end{defn}

We now turn to the description of the universe $\gobjclass$ of graphs. Note
that for the category $I\defeq 0{\rightrightarrows}1$, we have
$I/0\defeq\catid{0}$ and $I/1\defeq s\rightarrow\catid{1}\leftarrow t$, where
$s$ and $t$ are the morphisms $0\to 1$. Thus we have 
\begin{align*}
\psh(I/0) & \eqvsym \type\\
\psh(I/1) & \eqvsym \sm{X,Y:\type}X\to Y\to\type.
\end{align*}
The functors $\psh(\Sigma_s)$ and $\psh(\Sigma_t)$ map a presheaf 
$\pairr{X,Y,R}$ to $X$ and $Y$, respectively. 
Thus we obtain the following graph $\gobjclass$.

\begin{defn}
The universe $\gobjclass$ of graphs is defined to be
\begin{align*}
{\gobjclass}_0 & \defeq \type\\
{\gobjclass}_1(X,Y) & \defeq X\to Y\to\type.
\end{align*}
\end{defn}

Note that the type $\terms{\gobjclass}$ of terms of $\gobjclass$ is 
exactly $\tfGraph$. 

\subsubsection{Families of graphs}
Using $\gobjclass$, we obtain a description of what
it means to be a graph in a context $\Gamma$, where $\Gamma$ is itself a
graph. The notion of ``a graph in a context $\Gamma$'' is the interpretation
of ``a type in a context $\Gamma$''.

\begin{defn}
A \emph{graph $A$ in the context $\Gamma$} is a pair $\pairr{A_0,A_1}$ consisting
of 
\begin{align*}
A_0 & :\Gamma_0\to\type\\
A_1 & :\prd*{i,j:\Gamma_0}\Gamma_1(i,j)\to A_0(i)\to A_0(j)\to\type.
\end{align*}
We will also write $\Gamma\vdash A:\mfGraph$ when $A$ is a graph in context
$\Gamma$.
\end{defn}

\begin{rmk}
Note that every graph can be seen as a graph in context $\unit^\mfGraph$.
\end{rmk}

\begin{defn}
Let $A$ be a graph in context $\Gamma$ and let $f:\ctxhom{\Delta}{\Gamma}$
be a graph morphism. We define the graph $A[f]$ in context $\Delta$ by
\begin{align*}
A[f]_0(u) & \defeq A_0(f_0(u))\\
A[f]_1(p,x,y) & \defeq A_1(f_1(p),x,y).
\end{align*}
\end{defn}

\begin{defn}
Let $\Gamma\vdash A:\mfGraph$ and $\Gamma\vdash B:\mfGraph$ be two graphs in
the context $\Gamma$. A morphism $f$ from $A$ to $B$ is a pair $\pairr{f_0,f_1}$
consisting of
\begin{align*}
f_0 & : \prd*{i:\Gamma_0} A_0(i)\to B_0(i)\\
f_1 & : \prd*{i,j:\Gamma_0}{q:\Gamma_0}*{x:A_0(i)}*{y:A_0{j}}A_1(q,x,y)\to B_1(q,f_0(x),f_0(y)).
\end{align*}
We write $\ctxhom[\Gamma]{A}{B}$ for the type of morphisms between graphs in
context $\Gamma$.
\end{defn}

\begin{defn}
The unit graph $\unit^\Gamma$ in context $\Gamma$ consists of
\begin{align*}
(\unit^\Gamma)_0 & \defeq \lam{i}\unit\\
(\unit^\Gamma)_1(q) & \defeq \lam{x}{y}\unit.
\end{align*}
\end{defn}

As was the case with graphs, terms of graphs in a context are described by
morphisms from $\unit$. Explicating this in more detail, we get

\begin{defn}
Let $A$ be a graph in context $\Gamma$. A term $x$ of $A$ is a pair
$\pairr{x_0,x_1}$ consisting of
\begin{align*}
x_0 & : \prd{i:\Gamma_0}A_0(i)\\
x_1 & : \prd*{i,j:\Gamma_0}{q:\Gamma_1(i,j)}A_1(q,x_0(i),x_0(j)).
\end{align*}
We define $\terms[\mfGraph]{A}$ to be the type of terms of $A$.
\end{defn}

\begin{rmk}
Note that the type of terms of a graph $\Gamma$ is equivalent to the type
of terms of the graph $\Gamma$ in context $\unit^\mfGraph$.
\end{rmk}

\begin{defn}
Suppose that $\Gamma\vdash A:\mfGraph$. Then we define the graph $\ctxext{\Gamma}A$
by
\begin{align*}
(\ctxext{\Gamma}A)_0 & \defeq \sm{i:\Gamma_0}A_0(i)\\
(\ctxext{\Gamma}A)_1(\pairr{i,x},\pairr{j,y}) & \defeq \sm{q:\Gamma_1(i,j)}A_1(q,x,y).
\end{align*}
There is a morphism $\pi^A:\ctxhom{\ctxext{\Gamma}A,\Gamma}$ given by
\begin{align*}
\pi^A_0 & \defeq \lam{\pairr{i,x}}i && : (\sm{i:\Gamma_0}A_0(i))\to \Gamma_0\\
\pi^A_1(\pairr{i,x},\pairr{j,y}) & \defeq \lam{\pairr{q,e}}q && : (\sm{q:\Gamma_1(i,j)}A_1(q,x,y))\to\Gamma_1(i,j).
\end{align*}
\end{defn}

\begin{rmk}
The construction of $\ctxext{\Gamma}A$ is similar to the Grothendieck
construction of the category of elements for a presheaf.
\end{rmk}

\begin{defn}
Let $\Gamma\vdash A:\mfGraph$. A family $P$ of graphs over $A$ is by
definition a graph $P$ in the context $\ctxext{\Gamma}A$. More explicitly, a family
$P$ of graphs over $A$ in context $\Gamma$ consists of
\begin{align*}
P_0 & : \prd*{i:\Gamma_0}A_0(i)\to\type\\
P_1 & : \prd*{i,j:\Gamma_0}{q:\Gamma_1(i,j)}*{x:A_0(i)}*{y:A_0(j)} A_1(q,x,y)\to P_0(x)\to P_0(y)\to\type
%P_1 & : \prd{\pairr{i,x},\pairr{j,y}:(\ctxext{\Gamma}A)_0}(\ctxext{\Gamma}A)_1(\pairr{i,x},\pairr{j,y})\to P_0(i,x)\to P_0(j,y)\to\type.
\end{align*}
\end{defn}

\begin{defn}
When $P$ is a family of graphs over $A$ in context $\Gamma$ and $x:A$, we
get a graph $\subst{x}{P}$ in context $\Gamma$ given by
\begin{align*}
\subst{x}{P}_0(i) & \defeq P_0(x_0(i))\\
\subst{x}{P}_1(q,u,v) & \defeq P_1(\pairr{q,x_1(q)},u,v)
\end{align*}
\end{defn}

\subsubsection{The basic type constructors for graphs}
With families of graphs being available, we can give the interpretations of
dependent products, dependent sums and identity types. In the following, we
shall introduce the graph interpretations of dependent products, dependent
sums and identity types and describe the terms of the resulting graphs.\note{These definitions
should be connected to Mike's article but I don't really know how to do this}

\begin{defn}
Let $P$ be a family of graphs over $A$, where $\Gamma\vdash A:\mfGraph$. 
The dependent function graph $\mprd{A}{P}$ in context $\Gamma$ consists of
\begin{align*}
\mprd{A}{P}_0(i) & \defeq \prd{x:A_0(i)}P_0(x)\\
\mprd{A}{P}_1(q,f,g) & \defeq \prd*{x:A_0(i)}*{y:A_0(j)}{e:A_1(q,x,y)}P_1(\pairr{q,e},f(x),g(y)).
\end{align*}
\end{defn}

\begin{rmk}
A term $f:\mprd{A}{P}$ consists of
\begin{align*}
f_0 & : \prd*{i:\Gamma_0}{x:A_0(i)}P_0(x)\\
f_1 & : \prd*{i,j:\Gamma_0}{q:\Gamma_1(i,j)}*{x:A_0(i)}*{y:A_0(j)}{e:A_1(q,x,y)}P_1(\pairr{q,e},f_0(x),f_0(y))
\end{align*}
Therefore, we see that $\eqv{\terms{\mprd{A}{P}}}{\terms{P}}$. \emph{Warning:} it
is by no means the case that $\eqv{\terms{\mprd{A}{P}}}{\prd{x:\terms{A}}
\terms{\subst{x}{P}}}$ for all families $P$ over $A$. For instance, the graph
$\tilde{\emptyt}$ defined by $\tilde{\emptyt}_0\defeq\unit$ and 
$\tilde{\emptyt}_1(\ttt,\ttt)\defeq\emptyt$ has no terms and neither does
$\tilde{\emptyt}+\tilde{\emptyt}$. Nevertheless, there are two graph morphisms
from $\tilde{\emptyt}$ to $\tilde{\emptyt}+\tilde{\emptyt}$. More generally,
when $\Gamma$ is a graph such that $\Gamma_1(i,j)\jdeq\emptyt$ for all $i,j:\Gamma_0$,
then $\eqv{{\terms{\Gamma\to\Gamma'}}}{\Gamma_0\to\Gamma^\prime_0}$.
\end{rmk}

\begin{defn}
If $P$ is a family of graphs over $\Gamma$, the dependent pair graph
$\msm{A}{P}$ consists of
\begin{align*}
\msm{A}{P}_0(i) & \defeq \sm{x:A_0(i)}P_0(x)\\
\msm{A}{P}_1(q,\pairr{x,u},\pairr{y,v}) & \defeq \sm{e:A_1(q,x,y)}P_1(\pairr{q,e},u,v).
\end{align*}
\end{defn}

\begin{rmk}
A term $w:\msm{A}{P}$ consists of
\begin{align*}
w_0 & : \prd{i:\Gamma_0}\sm{x:A_0(i)}P_0(x)
\intertext{and, writing $\lam{i}\proj1(w_0(i))$ and $\lam{i}\proj2(w_0(i))$ as
$w_{00}$ and $w_{01}$ respectively,}
w_1 & : \prd*{i,j:\Gamma_0}{q:\Gamma_1(i,j)}\sm{e:A_1(q,w_{00}(i),w_{00}(j)}P_1(\pairr{q,e},w_{01}(i),w_{01}(j)).
\end{align*}
By $\choice{\infty}$ it follows that
\begin{equation*}
\eqv{\terms{\msm{A}{P}}}{\sm{x:\terms{A}}\terms{\subst{x}{P}}}.
\end{equation*}
\end{rmk}

\begin{rmk}
For any two graphs $A$ and $B$ in a context $\Gamma$, we have the graph
$B[\pi^A]$ in the context $\ctxext{\Gamma}A$, which we may just write
as $B$. Therefore, we can
consider the graphs $\mprd{A}{B}$ and $\msm{A}{B}$. As usually those shall be
denoted  by $A\to B$ and $A\times B$, respectively. 

A term $f$ of $A\to B$ consists of
\begin{align*}
f_0 & : \prd*{i:\Gamma_0}A_0(i)\to B_0(i)\\
f_1 & : \prd*{i,j:\Gamma_0}{q:\Gamma_1(i,j)}*{x:A_0(i)}*{y:A_0(j)}A_1(q,x,y)\to B_1(q,f_0(x),f_0(y)),
\end{align*}
so we see that $\terms{A\to B}\jdeq\ctxhom[\Gamma]{A}{B}\jdeq\terms{B[\pi^A]}$. 

A term $\pairr{x,y}$ of $A\times B$ consists of
\begin{align*}
\pairr{x,y}_0 & : \prd*{i:\Gamma_0} A_0(i)\times B_0(i)\\
\pairr{x,y}_1 & : \prd*{i,j:\Gamma_0}{q:\Gamma_1(i,j)} A_1(q,x_0(i),x_0(j))\times B_1(q,y_0(i),y_0(j))
\end{align*}
so we see that $\eqv{\terms{A\times B}}{\terms{A}\times\terms{B}}$. 

Note that we also obtain
a \emph{graph} $\Delta\to\Gamma$ this way, for every two graphs $\Delta$ and
$\Gamma$ seen as graphs in the context $\unit^\mfGraph$. 
The type $\terms{\Delta\to\Gamma}$ of terms of $\Delta\to\Gamma$ is just
$\ctxhom{\Delta}{\Gamma}$.
\end{rmk}

\begin{defn}
For any graph $\Gamma$ there is a graph $\tffam{\Gamma}$ in context 
$\gobjclass$ defined by
\begin{align*}
(\tffam{\Gamma})_0 & \defeq \lam{X}X\to\Gamma_0\\
(\tffam{\Gamma})_1(X,Y,R) & \defeq \lam{f}{g}\prd*{x:X}*{y:Y}R(x,y)\to\Gamma_1(f(x),g(y)).
\end{align*}
\end{defn}

\begin{rmk}
A term $D$ of $\msm{\gobjclass}{\tffam{\Gamma}}$ consists of a term
$\pairr{\Delta_0,f_0}$ of type
\begin{equation*}
\msm{\gobjclass}{\tffam{\Gamma}}_0\jdeq \sm{\Delta_0:\type}\Delta_0\to\Gamma_0
\end{equation*}
and a term $\pairr{\Delta_1,f_1}$ of type
\begin{equation*}
\msm{\gobjclass}{\tffam{\Gamma}}_1(q)\jdeq \sm{\Delta_1}
\end{equation*}
\end{rmk}

\begin{defn}
There is a graph morphism
\begin{equation*}
\graphcharmapfunc{\Gamma} : \sm{\gobjclass}
\end{equation*}
\end{defn}

\begin{defn}
Let $A$ be a graph in context $\Gamma$. We define the family $\idtypevar{A}$ over $\ctxwk{A}{A}$ in
context $\ctxext{\Gamma}A$ by
\begin{align*}
(\idtypevar{A})_0(\pairr{i,x},y) & \defeq \id{x}{y}\\
(\idtypevar{A}){}_1(\pairr{q,e},d,\alpha,\alpha') & \defeq \id{\trans{\pairr{\alpha,\alpha'}}{e}}{d}
\end{align*}
where $q:\Gamma_1(i,j)$, $e:A_1(q,x,x')$, $d:A_1(q,y,y')$, $\alpha:\id{x}{y}$
and $\alpha':\id{x'}{y'}$. The transportation along the path 
$\pairr{\alpha,\alpha'}:\id{\pairr{x,x'}}{\pairr{y,y'}}$ in $A_0(i)\times A_0(j)$
is taken with respect to the family $\lam{x}{x'}A_1(q,x,x')$.

We define the term $\refl{A}$ of the family 
$\subst{\idfunc[A]}{\idtypevar{A}}$ over $A$ in context $\Gamma$ by
\begin{align*}
(\reflf{A})_0(i) & \defeq \lam{x}\refl{x}\\
(\reflf{A})_1(q) & \defeq \lam{e}\refl{e}
\end{align*}
\end{defn}

\begin{defn}
Let $D$ be a family over $\idtypevar{A}$ in context 
$\ctxext({\Gamma}{A})\ctxwk{A}{A}$. Then we have the family
$\subst{\idfunc[A]}{D}$ over $\subst{\idfunc[A]}{\idtypevar{A}}$ in
context $\ctxext{\Gamma}A$ given by
\begin{align*}
\subst{\idfunc[A]}{D}_0(\pairr{i,x},\alpha) & \defeq D_0(\pairr{i,x,x},\alpha)\\
\subst{\idfunc[A]}{D}_1(\pairr{q,e},\gamma) & \defeq D_1(\pairr{q,e,e},\gamma).
\end{align*}
The family $\subst{\reflf{A}}{\subst{\idfunc[A]}{D}}$ over $A$ in
context $\Gamma$ is given by
\begin{align*}
\subst{\reflf{A}}{\subst{\idfunc[A]}{D}}_0(i,x) & \defeq D_0(i,x,x,\refl{x})\\
\subst{\reflf{A}}{\subst{\idfunc[A]}{D}}_1(q,e) & \defeq D_1(q,e,e,\refl{e}).
\end{align*}
To show that the identity graphs correctly interpret the identity elimination
rule, we must give a function
\begin{equation*}
\tfJ : \terms{\subst{\reflf{A}}{\subst{\idfunc[A]}{D}}}\to\terms{D}.
\end{equation*}
Note that a term $d$ of $\subst{\reflf{A}}{\subst{\idfunc[A]}{D}}$
consists of
\begin{align*}
d_0 & : \prd*{i:\Gamma_0}{x:A_0(i)}D_0(i,x,x,\refl{x})\\
d_1 & : \prd*{i,j:\Gamma_0}{q:\Gamma_1(i,j)}*{x:A_0(i)}*{y:A_0(j)}{e:A_1(q,x,y)}D_1(q,e,e,\refl{e})
\end{align*}
A simple argument using path induction reveals that terms of
$\subst{\reflf{A}}{\subst{\idfunc[A]}{D}}$ indeed yield terms of $D$. 
\end{defn}

\begin{rmk}
Using the identity graph $\idtypevar{A}$ we can describe the identity
graph $\id[A]{x}{y}$ in context $\Gamma$ for any two terms $x,y:A$. 
The graph $\id[A]{x}{y}$ in context $\Gamma$ consists of
\begin{align*}
(\id[A]{x}{y})_0(i) & \defeq \id{x_0(i)}{y_0(i)}\\
(\id[A]{x}{y})_1(q,\alpha,\beta) & \defeq \id{\trans{\pairr{\alpha,\beta}}{x_1(q)}}{y_1(q)}
\end{align*}
From this, we see that a term of $p:\id[A]{x}{y}$ consists of
\begin{align*}
p_0 & : \prd{i:\Gamma_0}\id{x_0(i)}{y_0(i)}\\
p_1 & : \prd*{i,j:\Gamma_0}{q:\Gamma_1(i,j)}\id{\trans{\pairr{p_0(i),p_0(j)}}{x_1(q)}}{y_1(q)}.
\end{align*}
\end{rmk}

\subsection{Properties of graphs}

\subsubsection{Contractibility and equivalences of graphs}

\begin{defn}
A graph $A$ in context $\Gamma$ is said to be \emph{contractible} if there
is a term of the graph
\begin{equation*}
\msm{A}{\mprd{\ctxwk{A}{A}}{\idtypevar{A}}}
\end{equation*}
in context $\Gamma$.
\end{defn}

\begin{lem}\label{lem:contractible-graphs}
Let $A$ be a graph in context $\Gamma$. The following are equivalent:
\begin{enumerate}
\item $A$ is a contractible graph.
\item Both $A_0(i)$ and $A_1(q,x,y)$ are always contractible.
\end{enumerate}
\end{lem}

\begin{proof}
Let $H:\msm{A}{\mprd{\ctxwk{A}{A}}{\idtypevar{A}}}$. Unfolding the definitions, we have
an element $H_0(i)$ of type
\begin{align*}
\msm{A}{\mprd{\ctxwk{A}{A}}{\idtypevar{A}}}_0(i) & \jdeq \sm{x:A_0(i)}\mprd{\ctxwk{A}{A}}{\idtypevar{A}}_0(i,x)\\
& \jdeq \sm{x:A_0(i)}\prd{y:A_0(i)}(\idtypevar{A})_0(\pairr{i,x},y)\\
& \jdeq \sm{x:A_0(i)}\prd{y:A_0(i)}\id{x}{y}
\end{align*}
for all $i:\Gamma_0$ and, writing $H_{00}(i)$ for $\proj1 H_0(i)$ and
$H_{01}(i)$ for $\lam{y}(\proj2 H_0(i))(y)$, we have $H_1(q)$ of type
\begin{align*}
& \msm{A}{\mprd{\ctxwk{A}{A}}{\idtypevar{A}}}_1(q,H_0(i),H_0(j)) \\
& \jdeq \sm{e:A_1(q,H_{00}(i),H_{00}(j))}\mprd{\ctxwk{A}{A}}{\idtypevar{A}}_1(\pairr{q,e},H_{01}(i),H_{01}(j))\\
& \jdeq \sm{e:A_1(q,H_{00}(i),H_{00}(j))}\prd*{x:A_0(i)}*{y:A_0(j)}{d:A_1(q,x,y)}(\idtypevar{A})_1(\pairr{q,e},d,H_{01}(x),H_{01}(y))\\
& \jdeq \sm{e:A_1(q,H_{00}(i),H_{00}(j))}\prd*{x:A_0(i)}*{y:A_0(j)}{d:A_1(q,x,y)}\id{\trans{\pairr{H_{01}(x),H_{01}(y)}}{e}}{d}.
\end{align*}
By $H_0$, it follows that each $A_0(i)$ is contractible. By the contractibility
of each $A_0(i)$, it follows that the type of $H_1(q)$ is equivalent to
\begin{equation*}
\sm{e:A_1(q,H_{00}(i),H_{00}(j))}\prd{d:A_1(q,H_{00}(i),H_{00}(j))}\id{e}{d}
\end{equation*}
which asserts that $A_1(q,H_{00}(i),H_{00}(j))$ is contractible. By 
the contractibility of each $A_0(i)$, is is equivalent to the assertion
that each $A_1(q,x,y)$ is contractible.
\end{proof}

\begin{rmk}
We address the question whether it is the case that a graph $A$ in context
$\Gamma$ is contractible if and only if $\terms{A}$ is contractible. As a
consequence of \autoref{lem:contractible-graphs}, it is indeed the case
that $\terms{A}$ is contractible whenever $A$ is. However, the converse
does not hold.

To see this, we first construct a counter example to the converse of the
weak function extensionality principle, which states that there is a function
of type
\begin{equation}
\iscontr(\prd{x:X}P(x))\to\prd{x:X}\iscontr(P(x))\label{eq:wfe-converse}
\end{equation}
for any type family $P:X\to\type$. In the proof of \autoref{thm:wfe-converse}, 
we will find a family
$P:X\to\type$ with the property that $\prd{x:X}P(x)$ is contractible and for
which there is a term $x:A$ with $P(x)$ not contractible. Disproving the
converse of the weak function extensionality principle suffices for our
purposes, because if $P:X\to\type$ is a counter example to \autoref{eq:wfe-converse},
then we can take $\Gamma\defeq\pairr{\Gamma_0,\Gamma_1}$ to be given by
$\Gamma_0\defeq X$ and $\Gamma_1(i,j)\defeq\emptyt$ and we take
$A\defeq\pairr{A_0,A_1}$ to be given by $A_0\defeq P$ and $A_1(q,u,v)
\defeq\emptyt$.
\end{rmk}

We define the family $\mathcal{T}:\Sn^1\to\UU$ by
We define $\mathcal{T}(\base)\defeq\mathbf{3}$. To define $\mathcal{T}(\lloop):
\id{\mathbf{3}}{\mathbf{3}}$ we apply the univalence axiom. Hence it suffices to find an
equivalence $\eqv{\mathbf{3}}{\mathbf{3}}$, for which we take the function
$e$ defined by
\begin{equation*}
e(x)\defeq\begin{cases}
0_\mathbf{3} & \text{if }x\jdeq 0_\mathbf{3}\\
2_\mathbf{3} & \text{if }x\jdeq 1_\mathbf{3}\\
1_\mathbf{3} & \text{if }x\jdeq 2_\mathbf{3}.
\end{cases}
\end{equation*}

\begin{lem}
The type $\terms{\mathcal{T}}\defeq\prd{x:\Sn^1}\mathcal{T}(x)$ is contractible.
\end{lem}

\begin{proof}
The type of sections of $\mathcal{T}$ is equivalent to $\sm{u:\mathcal{T}(\base)}\id{e(u)}{u}$.
If we have a term $\pairr{u,\alpha}$ of the latter type, it follows by induction
on $\mathbf{3}$ that $\id{\pairr{u,\alpha}}
{\pairr{0_\mathbf{3},\refl{0_\mathbf{3}}}}$ for all $\pairr{u,\alpha}:
\sm{x:\mathcal{T}(\base)}\id{e(u)}{u}$,
which shows that the type of sections of $\mathcal{T}$ is contractible.
\end{proof}

\begin{thm}\label{thm:wfe-converse}
There is a type family $P:A\to\type$ for which
\begin{equation*}
\neg\Big(\iscontr(\prd{x:A}P(x))\to\prd{x:A}\iscontr(P(x))\Big).
\end{equation*}
\end{thm}

\begin{proof}
The type family of our counter example is $\mathcal{T}$: the fiber $\mathcal{T}(\base)$ isn't contractible.
\end{proof}

\begin{defn}
A graph morphism $f:\ctxhom{\Delta}{\Gamma}$ is an equivalence of graphs when
$\graphcharmap[\Gamma]{f}$ is a contractible graph in the context $\Gamma$.
\end{defn}

\begin{lem}
Let $A$ and $B$ be graphs in a context $\Gamma$ and let $f:A\to B$. The following are equivalent:
\begin{enumerate}
\item $f[i]:A[i]\to B[i]$ is an equivalence of graphs for every term $i:\Gamma$.
\item $\ctxext{\Gamma}f:\ctxext{\Gamma}A\to\ctxext{\Gamma}B$ is an equivalence of graphs.
\item Both $f_0(i)$ and $f_1(q,x,y)$ are always equivalences.
\item $\terms{f}:\terms{\Delta}\to\terms{\Gamma}$ is an equivalence.
\end{enumerate}
\end{lem}

\begin{rmk}
It follows that there is an equivalence
\begin{equation*}
\ctxext{\Gamma}\msm{A}{P}\simeq\ctxext({\Gamma}{A})P
\end{equation*}
for every family $P$ of graphs over a graph $A$ in context $\Gamma$.
\end{rmk}

\subsubsection{Homotopy levels}
\begin{itemize}
\item A graph $A$ in context $\Gamma$ is of homotopy level $n$ precisely when each
$A_0(i)$ and each $A_1(q,x,y)$ are of homotopy level $n$. 
\item We can name at least three different propositions in the empty context:
\begin{enumerate}
\item $\Gamma_0\defeq\emptyt$.
\item $\Gamma_0\defeq\unit$ and $\Gamma_1(\ttt,\ttt)\defeq\emptyt$.
\item $\Gamma_0\defeq\unit$ and $\Gamma_1(\ttt,\ttt)\defeq\unit$.
\end{enumerate}
Therefore $\mfGraph$ does not satisfy the law of excluded middle.
\end{itemize}



\subsubsection{Univalence for the graph model}
In the other direction, we also obtain a graph morphism $\graphcharmap[\Gamma]{f}:
\ctxhom{\Gamma}{\gobjclass}$ for every graph morphism $f:\ctxhom{\Delta}{\Gamma}$.
In \autoref{graph-object-classifier} we will prove that the maps
$\int_\Gamma:\ctxhom{\Gamma}{\gobjclass}\to\sm{\Delta:\tfGraph}\ctxhom{\Delta}{\Gamma}$
is an equivalence with iverse $\graphcharmapfunc{\Gamma}$. 

\begin{defn}
Let $f:\Delta\to\Gamma$ be a graph morphism. We define the graph morphism
$\graphcharmap[\Gamma]{f}$ by
\begin{align*}
\graphcharmap[\Gamma]{f}_0 & \defeq \lam{i}\hfib{f_0}{i}\\
\graphcharmap[\Gamma]{f}_1(i,j) & \defeq \lam{q}\hfib{f_1(i,j)}{q}.
\end{align*}
\end{defn}


\begin{thm}\label{graph-object-classifier}
Main theorem here.
\end{thm}

\begin{comment}
To describe the object classifier for graphs, we will follow Streicher. Thus
we have to look at presheaves over $I/i$ for each object $i$ of the category
$I\defeq 0{\rightrightarrows}1$ with the morphisms named $s$ and $t$ for source
and target. The category $I/0$ is the terminal category;
the category $I/1$ looks like $\cdot{\rightarrow}\cdot{\leftarrow}\cdot$.
Therefore, we have
\begin{align*}
\type^{\op{(I/0)}} & \eqv{}{\type},\\
\type^{\op{(I/1)}} & \eqv{}{\sm{X,Y,A:\type}(A\to X)\times(A\to Y)}\\
& \eqv{}{\sm{X,Y:\type}X\to Y\to\type}.
\end{align*}
The functors $\type^\op{\Sigma_s}$ and $\type^\op{\Sigma_t}$ are given by
$\pi_1$ and $\pi_2$ respectively. This leads to our following definition
of the object classifier $\gobjclass$:

\begin{defn}
Define $\gobjclass$ to be the graph consisting of
\begin{align*}
\gobjclass_0 & \defeq  \type\\
\gobjclass_1(X,Y) & \defeq  X\to Y\to\type
\end{align*}
and define $\pointed{\gobjclass}$ by
\begin{align*}
(\pointed{\gobjclass})_0 & \defeq  \pointed{\type}\\
(\pointed{\gobjclass})_1(\pairr{X,x},\pairr{Y,y}) & \defeq  \sm{R:X\to Y\to\type}R(x,y)
\end{align*}
There is the obvious forgetful graph morphism $t:\pointed{\gobjclass}\to\gobjclass$,
given by projection on the first coordinate.

For any morphism $f:\Delta\to\Gamma$ of graphs we define a morphism
$\graphcharmap(f):\Gamma\to\gobjclass$ of graphs by
\begin{align*}
\graphcharmap(f)_0(i) & \defeq  \hfiber{f_0}{i}\\
\graphcharmap(f)_1(q,\pairr{u,\alpha},\pairr{v,\beta}) & \defeq  \sm{p:\Delta_1(u,v)}
\id{\trans{\pairr{\alpha,\beta}}{f_1(p)}}{q}
\end{align*}
where $\pairr{u,\alpha}:\graphcharmap(f)_0(i)$ and $\pairr{v,\beta}:\graphcharmap(f)_0(j)$. The
morphism $\graphcharmap(f)$ is called the \emph{characteristic map of $f$}. We obtain a function
\begin{equation*}
\graphcharmap : \big(\sm{\Delta:\tfGraph }\Delta\to\Gamma\big)\to\big(\Gamma\to\gobjclass\big)
\end{equation*}
for every graph $\Gamma$.
\end{defn}

\begin{thm}\label{thm:graph-classifier1}
The function $\graphcharmap$ is an equivalence for any graph $\Gamma$.
\end{thm}

\begin{proof}
We have to find a quasi-inverse
\begin{align*}
\Sigma : (\Gamma\to\gobjclass)\to\big(\sm{\Delta:\tfGraph}\Delta\to\Gamma\big)
\end{align*}
of $\graphcharmap$. Thus, we have to define $\Sigma_0:(\Gamma\to\gobjclass)\to\tfGraph$ and
$\Sigma_1:\prd{P:\Gamma\to\gobjclass}\Sigma_0(P)\to\Gamma$. For $P:\Gamma\to\gobjclass$ we define
\begin{align*}
\Sigma_0(P)_0 & \defeq \sm{i:\Gamma_0}P_0(i)\\
\Sigma_0(P)_1(\pairr{i,u},\pairr{j,v}) & \defeq \sm{q:\Gamma_1(i,j)}P_1(q,u,v)\\
\Sigma_1(P)_0 & \defeq \proj1\\
\Sigma_1(P)_1(\pairr{i,u},\pairr{j,v}) & \defeq \proj1.
\end{align*}
\end{proof}

\begin{thm}\label{conj:graph_classifier2}
For any graph morphism $f:\Delta\to\Gamma$, the diagram
\begin{equation*}
\begin{tikzcd}
\Delta \ar{r}{} \ar{d}[swap]{f} & \pointed{\gobjclass} \ar{d}{t} \\ 
\Gamma \ar{r}[swap]{\graphcharmap(f)} & \gobjclass 
\end{tikzcd}
\end{equation*}
is a pullback square.
\end{thm}
\note{We would like it to be a pb \emph{in} the graph model}
\end{comment}

\begingroup\color{blue}
\subsubsection{The adjunctions $\tfcolim\dashv\Delta\dashv\tflim$}
We define $\Delta:\type\to\tfGraph$ by
\begin{equation*}
\Delta(X)\defeq\pairr{X,\lam{x}{x'}\id{x}{x'}}
\end{equation*}
for $X:\type$. For $A:X\to\type$ we define $\Delta(A):\mftyp(\Delta(X))$ by
\begin{align*}
\Delta(A)_0(x) & \defeq A(x)\\
\Delta(A)_1(p,a,b) & \defeq \id{\trans{p}{a}}{b}.
\end{align*}

\begin{lem}
For any type $X$ and any graph $\Gamma$ there is an equivalence
\begin{equation*}
\eqv{(X\to\terms{\Gamma})}{\terms{\Delta(X)\to\Gamma}}.
\end{equation*}
\end{lem}

\begin{proof}
We have to find functions
\begin{align*}
\varphi & : (X\to\terms{\Gamma})\to\terms{\Delta(X)\to\Gamma}\\
\psi & : \terms{\Delta(X)\to\Gamma}\to X\to\terms{\Gamma}
\end{align*}
which are each others homotopy inverse. To define $\varphi$, let
\end{proof}
\endgroup

