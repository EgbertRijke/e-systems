\section{Pretty type theory}
In this section we shall discuss conventions to algorithmically `pretty-print' judgments of
the theory of contexts, families and terms. These will include variable
declarations in contexts and omission of explicit notation of weakening. This
will render various notationally different expressions identical and thus
we shall have to show that when this happens, the expressions under consideration
were already judgmentally equal. Pretty type theory could be seen as a
semi-formal version of the theory of contexts, families and terms and it
will facilitate translation to the informal language developed in \cite{TheBook}.

\begin{comment}
\subsection{The basic judgments}
The basic judgments of pretty type theory are the same as for structural type
theory. There are judgments for: ``$\Gamma$ is a context'',
``$A(i)$ over $i:\Gamma$ is a family over $\Gamma$'', ``$A(i)$ over $i:\Gamma$ 
is a type in context $\Gamma$''
and ``$x(i)$ is a term of $A(i)$ above $i:\Gamma$''. The other four
judgments are for judgmental equality. 

\begin{align*}
\jvctx*{\Gamma} & \jvctxeq*{\Gamma}{\Gamma'}\\
\jvfam*{i}{\Gamma}{A} & \jvfameq*{i}{\Gamma}{A}{B}\\
\jvtype*{i}{\Gamma}{A} & \jvtypeeq*{i}{\Gamma}{A}{B}\\
\jvterm*{i}{\Gamma}{A}{x} & \jvtermeq*{i}{\Gamma}{A}{x}{y}.
\end{align*}

We have the following basic inference rules that relate types and families:

\begin{small}
\begin{align*}
& \inference
  {\jvtype{i}{\Gamma}{A}}
  {\jvfam{i}{\Gamma}{A}}
& & \inference
    {\jvtypeeq{i}{\Gamma}{A}{B}}
    {\jvfameq{i}{\Gamma}{A}{B}}\\
& \inference
  {\jvtype{i}{\Gamma}{A}
   \jvfameq{i}{\Gamma}{A}{B}}
  {\jvtype{i}{\Gamma}{B}}
& & \inference
    {\jvtype{i}{\Gamma}{A}
     \jvfameq{i}{\Gamma}{A}{B}}
    {\jvtypeeq{i}{\Gamma}{A}{B}}
\end{align*}
\end{small}

\subsection{The basic rules for judgmental equality}
The rules for judgmental equality establish that it is an equivalence relation.
\bgroup\small
\begin{align*}
& \inference
  {\jvctx{\Gamma}}
  {\jvctxeq{\Gamma}{\Gamma}} 
& & \inference
    {\jvctxeq{\Gamma}{\Delta}}
    {\jvctxeq{\Delta}{\Gamma}} 
& & \inference
    {\jvctxeq{\Gamma}{\Delta}
     \jvctxeq{\Delta}{\greek{E}}}
    {\jvctxeq{\Gamma}{\greek{E}}}\\
& \inference
  {\jvfam{i}{\Gamma}{A}}
  {\jvfameq{i}{\Gamma}{A}{A}} 
& & \inference
    {\jvfameq{i}{\Gamma}{A}{B}}
    {\jvfameq{i}{\Gamma}{B}{A}}
& & \inference
    {\jvfameq{i}{\Gamma}{A}{B}
     \jvfameq{i}{\Gamma}{B}{C}}
    {\jvfameq{i}{\Gamma}{A}{C}}\\
& \inference
  {\jvterm{i}{\Gamma}{A}{x}}
  {\jvtermeq{i}{\Gamma}{A}{x}{x}}
& & \inference
    {\jvtermeq{i}{\Gamma}{A}{x}{y}}
    {\jvtermeq{i}{\Gamma}{A}{y}{x}}
& & \inference
    {\jvtermeq{i}{\Gamma}{A}{x}{y}
     \jvtermeq{i}{\Gamma}{A}{y}{z}}
    {\jvtermeq{i}{\Gamma}{A}{x}{z}}
\end{align*}
\egroup

The following convertibility rules are responsible for the strictness
of judgmental equality, which sets it apart from equivalences or identifications:

\begin{align*}
& \inference
  {\jvctxeq{\Gamma}{\Delta}
   \jvfam{i}{\Gamma}{A}}
  {\jvfam{i}{\Delta}{A}}
& & \inference
    {\jvctxeq{\Gamma}{\Delta}
     \jvfameq{i}{\Gamma}{A}{B}}
    {\jvfameq{i}{\Delta}{A}{B}}\\
& \inference
  {\jvctxeq{\Gamma}{\Delta}
   \jvterm{i}{\Gamma}{A}{x}}
  {\jvterm{i}{\Delta}{A}{x}}
& & \inference
    {\jvctxeq{\Gamma}{\Delta}
     \jvtermeq{i}{\Gamma}{A}{x}{y}}
    {\jvtermeq{i}{\Delta}{A}{x}{y}}\\
& \inference
  {\jvfameq{i}{\Gamma}{A}{B}
   \jvterm{i}{\Gamma}{A}{x}}
  {\jvterm{i}{\Gamma}{B}{x}}
& & \inference
    {\jvfameq{i}{\Gamma}{A}{B}
     \jvtermeq{i}{\Gamma}{A}{x}{y}}
    {\jvtermeq{i}{\Gamma}{B}{x}{y}}
\end{align*}

\subsection{The empty context}
The empty context looks a bit strange when we explicitly denote the terms. But
we will not do so anymore after this subsection.

\begin{align}
& \inference
  {}
  {\jctx{\emptyc}}\\
& \inference
  {\jctx{\Gamma}}
  {\jvfam{i}{\Gamma}{\emptyf[\Gamma]}}\\
& \inference
  {\jctx{\Gamma}}
  {\jvterm{i}{\Gamma}{\emptyf[\Gamma]}{\emptytm[\Gamma]}}\\
& \inference
  {\jvterm{i}{\Gamma}{\emptyf[\Gamma]}{x}}
  {\jvtermeq{i}{\Gamma}{\emptyf[\Gamma]}{x}{\emptytm[\Gamma]}}
\end{align}

Moreover, if $\Gamma$ is a context family over the
empty context, then $\Gamma$ is a context and every context is a context
family over the empty context. Note that this allows us to speak
of terms of contexts too.

\begin{align}
& \inference
  {\jctx{\Gamma}}
  {\jvfam{\nameless}{\emptyc}{\Gamma}} 
& & \inference
    {\jvfam{\nameless}{\emptyc}{\Gamma}}
    {\jctx{\Gamma}}\\
& \inference
  {\jctxeq{\Gamma}{\Delta}}
  {\jvfameq{\nameless}{\emptyc}{\Gamma}{\Delta}}
& & \inference
    {\jvfameq{\nameless}{\emptyc}{\Gamma}{\Delta}}
    {\jctxeq{\Gamma}{\Delta}}
\end{align}

\subsubsection{The empty context is compatible with itslef}
The empty context $\emptyc$ may be considered as a family of contexts over the empty
context. When we do this, we get $\emptyf[\emptyc]$.
\begin{equation}
\inference
  {}
  {\jvfameq{\nameless}{\emptyc}{\emptyc}{\emptyf[\emptyc]}}
\end{equation}
In the future, we shall denote $\emptyf[\Gamma]$ by $\emptyf$. The above rule
guarantees that this will not cause confusion. Likewise, we shall denote
$\emptytm[\Gamma]$ by $\emptytm$.

\subsection{Extension}
We introduce extension which not only extends a context $\Gamma$ and a family
$A$ over it to a context $\ctxext{\Gamma}{A}$, but which also extends a family $A$
in context $\Gamma$ and a family $P$ over it to a family $\ctxext{A}{P}$ over context
$\Gamma$. We do this to ensure that all of type theory can be done in a context.
For instance, we could say (1) that a context in context $\Gamma$ is the same thing
as a family over $\Gamma$; (2) When $A$ is a context in this sense, a family over
$A$ is the same thing as a family $P$ over $\ctxext{\Gamma}{A}$ and 
(3) when $P$ is a family over $A$ in this sense, a term of $P$ keeps its original meaning.

\begin{align}
& \inference
  {\jvfam{i}{\Gamma}{A}}
  {\jvfamcombi{{i}{x}}{{\Gamma}{A}}{P}}
& & \inference
    {\jctxeq{\Gamma}{\Delta}
     \jfameq{\Gamma}{A}{B}}
    {\jctxeq{\ctxext{\Gamma}{A}}{\ctxext{\Delta}{B}}}\\
& \inference
  {\jfam{{\Gamma}{A}}{P}}
  {\jfam{\Gamma}{\ctxext{A}{P}}}
& & \inference
    {\jfameq{\Gamma}{A}{B}
     \jfameq{{\Gamma}{A}}{P}{Q}}
    {\jfameq{\Gamma}{\ctxext{A}{P}}{\ctxext{B}{Q}}}
\end{align}

\subsubsection{Extension is compatible with the empty context}
The following rule asserts that extension by $\emptyc$ leaves the contexts unchanged.
\begin{align}
& \inference
  {\jctx{\Gamma}}
  {\jctxeq{\ctxext{\emptyc}{\Gamma}}{\Gamma}}\\
& \inference
  {\jctx{\Gamma}}
  {\jctxeq{\ctxext{\Gamma}{\emptyf}}{\Gamma}}\\
& \inference
  {\jfam{\Gamma}{A}}
  {\jfameq{\Gamma}{\ctxext{\emptyf}{A}}{A}}
\end{align}

\subsubsection{Extension is compatible with itself}
The inference rules asserting that extension is compatible with itself assert
that contexts are unstructured lists of type declarations. This rule is
unavoidable if we want that for a family $A$ in context $\Gamma$, a family over
$A$ is the same thing as a family over $\ctxext{\Gamma}{A}$. 

\begin{align}
& \inference
  {\jfam{\Gamma}{A}
   \jfam{{\Gamma}{A}}{P}}
  {\jctxeq{\ctxext{{\Gamma}{A}}{P}}{\ctxext{\Gamma}{{A}{P}}}}\\
& \inference
  {\jfam{{\Gamma}{A}}{P}
   \jfam{{{\Gamma}{A}}{P}}{Q}}
  {\jfameq{\Gamma}{\ctxext{{A}{P}}{Q}}{\ctxext{A}{{P}{Q}}}}
\end{align}
\end{comment}
