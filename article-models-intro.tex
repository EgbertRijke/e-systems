\section{Introduction}
This project has its origins in the beginning of 2013, when I proved a version
of the descent theorem for homotopy colimits in type theory while I was working
with Bas Spitters to develop notions from higher category theory in the
univalent foundations. To arrive at a
notion of diagram general enough to capture the higher inductive types described
in chapter 6 of \cite{TheBook} excluding the truncations we needed type
theoretical graphs. The graphs form a model of type theory and indeed we needed
several of the basic type theoretical operations to give an efficient approach
to the descent theorem. However it was not an issue to describe the graph model
and the sense in which it was a model, the presentation of it was unsystematic
and in that sense unsatisfactory too. 

From the experience of working with the graph model it became appearent that the
answer we had of the question of what it models, i.e.~Martin L\"of type theory
with the univalence axiom, could have been different if we were to describe a
type theory based on the mechanism on which the model was built. The particular
featueres of the graph model were that there was a type of graphs resembling a category
of contexts, for each context a notion of what it means to be a family of graphs
and a term thereof; there are the operations of context extension, weakening and
substitution. In other words, the graph model is an instance of type theretical
categories with families, an invention of Peter Dybjer.

As we know however, the type theoretical description of a category as given in
chapter 11 of \cite{TheBook} is restricted to categories which have a $1$-type
as the type of objects and for every two objects a $0$-type of morphisms.
Although categories with families are like that too, these are not very natural
restrictions when one is concerned about internal models of type theory. It
seems we're in the need of a higher categorical formulation of what an internal
model, or a category with family, is. But type theory has the tendency to take
care of it's own higher categorical structure and one could hope that type
theory is an elementary theory of the category with families of category with
families.

\subsection{Ideas in the definition of internal models}
An internal model of type theory is like a category with families, but we want
to avoid having to state higher coherences. In fact, we don't even start our
definition with a category of contexts; instead we just take a \emph{type} of contexts. 
The morphisms will come from the terms, evaluation of a function at a given
term will come from substitution. We recognize three basic ingredients to models:
first there is a type of contexts; second, for every context there is a model of types in
that context and third, for every type in a given context there is a type of its
terms. Then there are three basic attributes: context extension, weakening and
substitution. Context extension provides us with families over types as well as
with an interpretation of dependent pair types. We need weakening 
so that families can depend on the same type multiple times (the way the
identity type of a type depends two times on that type) and to be able
to talk about non-dependent function types,
the morphims of our category. Substitution will give us a way
to work with fibers of families as well as composition of functions and evaluation
of functions at terms.

Because we require a \emph{model} of types in a context, all the structure
which we require at the bottom level will be required to exist higher up as well.
Thus, the model of types in a given context $\Gamma$ will have a type of contexts
itself, which can be seen as the type of types in $\Gamma$; it will have its
own notion of types in a context, its own notion of terms, context extension,
weakening and substitution together with all the structure require for it. For
instance, when $A$ is a type in context $\Gamma$ in a model $\mfM$, then there
is the model of types in context $A$, which is the model of families over $A$. 
This model is required to be \emph{definitionally equal to} the model of types
in the context $\ctxext{\Gamma}{A}$, the context extension of $\Gamma$ and $A$.
In this way we protect ourselves from the need to dig an infinitely deep structure
of models when we want to consider examples.

To give the definition of a model we shall also need to consider certain morphisms
of internal models. Those should preserve all the structure: contexts are mapped
to contexts; for every context a morphism of models mapping the model of types
in that context to the model of types in the image of that context; there should
be a mapping of terms and context extension, weakening and substitution should be
preserved. We need to consider those morphisms because we require context extension,
weakening and substitution to be of that kind, thereby respecting each other
in all possible ways.

When we have this framework set up, we can interpret the basic type constructors
such as $\Pi$, $\Sigma$ and $\idtypevar{}$.
The higher categorical structure then comes from the
result that we have an interpretation of type theory.

\subsection{The elementary theory of the category with families of categories
with families}

Thus the idea was born to give a new description of Martin-L\"of type theory
which is more faithful to the idea that type theory is an algebraic description
of higher category theory. The features we have in mind for this description
is that:
\begin{enumerate}
\item As in Martin-L\"of type theory there are the three basic judgments
      \begin{align*}
        & \Gamma\text{ is a context:} & \jctx*{\Gamma}\\
        & A\text{ is a type in context }\Gamma\text{:} & \jtype*{\Gamma}{A}\\
        & x\text{ is a term of $A$ in context }\Gamma\text{:} & \jterm*{\Gamma}{A}{x}
      \end{align*}
      together with the three accompanying judgments for judgmental equality.
      Our new description of type theory should be such that we can faithfully
      make the translation to the three assertions
      \begin{enumerate}
      \item $\Gamma$ is an object\\
      \item $A$ is a fibration over $\Gamma$\\
      \item $x$ is a section of the fibration $A$. 
      \end{enumerate}
\item There is a basic type theory which has just the operations
      \begin{description}
      \item[extension] taking the domain of the fibration $A$ over $\Gamma$.
      \item[weakening] given an object $\Gamma$, weakening assigns to an object
      $\Delta$ trivial fibration with $\Delta$ as its fibers.
      \item[substitution] taking the fiber at a point. Fibers should always
      be equivalent to homotopy fibers.
      \end{description}
      Thus we aim for a closer connection between the syntactic operations and
      the operations of models.
\item Models of this basic type theory should be weak $\omega$-categories. Thus,
      it should not be necessary to start the definition of a model with a bunch
      of contexts and substitution morphisms between them: these and the
      operations on them, such as composition and associativity thereof, should
      be a consequence of the interpretation of the basic type theory.
\item There shouldn't be an empty context and dependent function types right
      from the start. These correspond to having a terminal object and the
      models of type theory with dependent function types are the locally
      cartesian closed models.
\item Contexts should not be lists of variable declarations and all the
      operations should be explicitly invariant under context extension.
\item There should be a large zoo of extensions of the basic type theory. First
      off, we can extend the basic type theory with identity types, dependent
      function types and inductive types. We want a general description of
      homotopy colimits in type theory, which should become available once we
      know what categories (models of the basic type theory) are. Other types
      such as the truncations could be added as well.
      
      An extension of a different kind is the addition of the univalence axiom.
\end{enumerate}

\subsection{The further goals of the project}
The first aim of this project is thus to give this new description of type
theory with these properties. The second aim is to define a general notion of
an internal model and provide several examples thereof. Among the examples
should be
\begin{enumerate}
\item The setoid model. This is a classical one so it should be there. It might
      be a bit different in our case. We'll want to construct a setoid model
      of the basic type theory to interpret identity types without necessarily
      interpreting dependent function types (but that should remain possible).
\item The graph model. This is the model that leads to the first version of the
      descent theorem.
\item A univalent universe should be a model.
\item Equifibered diagrams over a graph should be the fibrations in some model.
\item The model of all models. Very likely, we will even define this simultaneously
      with the notion of model.
\item The model of weak $\omega$-groupoids. Since we think of models of the
      basic type theory as weak $\omega$-categories, it is not too hard to
      provide a condition on those models which enables us to talk about
      weak $\omega$-categories. In fact, we have several options for such a
      condition. We investigate those; we also investigate how they relate to
      Brunerie's weak $\omega$-groupoids. Presumably, they form an internal
      model without too much fuss (because we already have an internal model
      of internal models at this stage) and we can ask whether this models
      the univalence axiom. In fact, the model of internal models might already
      have modeled the univalence axiom. 
\end{enumerate}

We also wish to extend the descent property to a more general class of homotopy
colimits. The general recipe has already been implicit in the first descent
property where we have used diagrams over graphs. We have the following in mind:
\begin{enumerate}
\item Give an internal description of a diagram. Probably this is the same thing
      as an internal model of the basic type theory with or without identity types.
\item Define the notion of homotopy colimit thereof.
\item Define the notion if equifibered diagrams.
\item Formulate and prove descent.
\end{enumerate}

Using the descent property we can view $\tfcolim$ as an operation of type theory
It should preserve all the basic operations, identity types, dependent function
types, itself, etc.
