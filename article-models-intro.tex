\section{Introduction}
In this paper we define a notion of internal model which is adapted to
the univalent setting in which we work. Thus, we only describe what it means
to be an internal model when the ambient type theory is univalent. Our
presentation is derived from the definition of~\cite{Dybjer1996}. There,
an internal model is defined to be a category with families with interpretations
for the basic type constructors $\Pi$, $\Sigma$ and $\idtypevar{}$. However,
a category in~\cite{Dybjer1996} has a set of objects and for every two objects
a setoid of morphisms. This is too restrictive for our purpose. Ideally, we
would start with an $\infty$-category of contexts and an $\infty$-presheaf
of families over it. This presents us with the problem that a notion of
$\infty$-category has not been formulated \emph{in} type theory. Therefore,
we make a deviation from the approach in~\cite{Dybjer1996}. We shall
start with a type $\tfctx$ of contexts, a family $\mftypfunc:\tfctx\to\type$
of families over it and a family $\terms{\blank}:\prd*{\Gamma:\tfctx}\mftyp{\mfM}{\Gamma}
\to\type$ sending a type $A$ in context $\Gamma$ to its type $\terms{A}$ of
terms; then we shall directly interpret $\Pi$-types, declaring
$\terms{A\to B}$ to be $\ctxhom[\Gamma]{A}{B}$. The idea is that by letting
$\Pi$ play a fundamental role in the definition of a category with families
$\mathcal{C}$,
the morphisms get their properties directly from the internal type theory
of $\mathcal{C}$.

\begin{comment}
We briefly list the data of which an internal model of type theory consits. We
consider internal models in the style of~\cite{Dybjer1996}. Thus we will
describe a model $\mfM$ as a category with families. However, since our setting is
univalent type theory we will deviate from~\cite{Dybjer1996} in the following
respects: first of all, we make use of typical ambiguity and hence omit
reference to a thing called \emph{sort}. 

We organize the definition of an internal model $\mfM$ as follows: in
\autoref{internal-model-contexts} we describe the category of contexts itself;
in \autoref{internal-model-families} we describe the families over a given context
and the related operations; in \autoref{internal-model-constructors} we
describe the basic constructors in the internal model.
\end{comment}

\subsection{Ideas in the definition}
An internal model of type theory is like a category with families, but we want
to avoid having to state higher coherences. In fact, we don't even start our
definition with a category of contexts; instead we just take a \emph{type} of contexts. 
The morphisms will come from the terms, evaluation of a function at a given
term will come from substitution. We recognize three basic ingredients to models:
first there is a type of contexts; second, for every context there is a model of types in
that context and third, for every type in a given context there is a type of its
terms. Then there are three basic attributes: context extension, weakening and
substitution. Context extension provides us with families over types as well as
with an interpretation of dependent pair types. We need weakening 
so that families can depend on the same type multiple times (the way the
identity type of a type depends two times on that type) and to be able
to talk about non-dependent function types,
the morphims of our category. Substitution will give us a way
to work with fibers of families as well as composition of functions and evaluation
of functions at terms.

Because we require a \emph{model} of types in a context, all the structure
which we require at the bottom level will be required to exist higher up as well.
Thus, the model of types in a given context $\Gamma$ will have a type of contexts
itself, which can be seen as the type of types in $\Gamma$; it will have its
own notion of types in a context, its own notion of terms, context extension,
weakening and substitution together with all the structure require for it. For
instance, when $A$ is a type in context $\Gamma$ in a model $\mfM$, then there
is the model of types in context $A$, which is the model of families over $A$. 
This model is required to be \emph{definitionally equal to} the model of types
in the context $\ctxext{\Gamma}{A}$, the context extension of $\Gamma$ and $A$.
In this way we protect ourselves from the need to dig an infinitely deep structure
of models when we want to consider examples.

To give the definition of a model we shall also need to consider certain morphisms
of internal models. Those should preserve all the structure: contexts are mapped
to contexts; for every context a morphism of models mapping the model of types
in that context to the model of types in the image of that context; there should
be a mapping of terms and context extension, weakening and substitution should be
preserved. We need to consider those morphisms because we require context extension,
weakening and substitution to be of that kind, thereby respecting each other
in all possible ways.

When we have this framework set up, we can interpret the basic type constructors
such as $\Pi$, $\Sigma$ and $\idtypevar{}$.
The higher categorical structure then comes from the
result that we have an interpretation of type theory.

\begingroup
\color{red}
\begin{rmk}
Currently, the definition seems to be circular. To define a model we need that
context extension, weakening and substitution be morphisms of models. A morphism
of models needs to preserve context extension, weakening and substitution.
Moreover, its definition requires the notion of composition of morphisms.

I've done it this way because it guides me to what the rules should be, trusting
that I can work my way back to give a (possibly less transparent) definition
which contains no circularities without doubt.
\end{rmk}
\endgroup
