\subsection{Joining the theory of projections with the theory of substitution}
\label{sec:esystem-equalities}

\subsubsection{Weakening of an empty family}
The following inference rules express that when the empty family is
weakened, the result is the empty family.
\label{comp-w0}
\begin{align}
& \inference
  { \jfam{\Gamma}{A}
    }
  { \jfameq{{\Gamma}{A}}{\ctxwk{A}{\emptyf}}{\emptyf}
    }
  \label{comp-w0-c}\\
& \inference
  { \jfam{\Gamma}{A}
    \jfam{\Gamma}{B}
    }
  { \jfameq
    {{{\Gamma}{A}}{\ctxwk{A}{B}}}
    {\ctxwk[\famsym]{A}{\emptyf}}
    {\emptyf}
    }
  \label{comp-w0-f}
\end{align}

Because a family over $\Gamma$ is the same as a family over 
$\ctxext{\Gamma}{\emptyf}$ we can apply both the action on contexts and the
action on families of weakening to a family $B$ over $\Gamma$. When we apply
the action on families, we obtain a family $\ctxwk[\famsym]{A}{B}$ over the
context $\ctxext{{\Gamma}{A}}{\ctxwk{A}{\emptyf}}$. However, since we have
postulated the judgmental equalities $\ctxwk{A}{\emptyf}\jdeq\emptyf$ and
$\ctxext{{\Gamma}{A}}{\emptyf}\jdeq\ctxext{\Gamma}{A}$, we see that we can
compare $\ctxwk[\famsym]{A}{B}$ with $\ctxwk{A}{B}$. The following inference
rule postulates that these two are judgmentally equal:
\begin{equation}
\inference
{ \jfam{\Gamma}{A}
  \jfam{\Gamma}{B}
  }
{ \jfameq{{\Gamma}{A}}{\ctxwk[\famsym]{A}{B}}{\ctxwk{A}{B}}
  }
\end{equation}
Due to this rule, the action on contexts and the action on families of weakening
are compatible with each other and consequently there can be no possible
confusion when we omit the annotations $\famsym$ and $\tmsym$ alltogether. In
the future, the weakening of a family $Q$ over $\ctxext{\Gamma}{B}$ shall
be denoted just by $\ctxwk{A}{Q}$ and likewise the weakening of a term $g$ of
$Q$ shall be denoted by $\ctxwk{A}{g}$.

Because a family $B$ over $\Gamma$ can be treated as a family by the operation
of weakening, weakening also acts on the terms of $B$. The weakening of a term
$y$ of $B$ by $A$ can be seen as the constant term (or function) of the
family $\ctxwk{A}{B}$.

\subsubsection{Weakening by the empty family}
Note that we can also weaken by the empty family over $\Gamma$.
Weakening by the empty family $\emptyf$ over a context $\Gamma$ leaves families, 
their terms, families over those families and terms of those unchanged:
\label{comp-0w}
\begin{align}
& \inference
  { \jany{\Gamma}{e}
    }
  { \janyeq{\Gamma}{\ctxwk{\emptyf}{e}}{e}
    }
  \label{comp-0w-any}
\end{align}

\subsubsection{Fibers of an empty family}
The following inference rules establish that the fibers of the empty family are 
the empty families:
\label{comp-s0}
\begin{align}
& \inference
  { \jterm{\Gamma}{A}{x}
    }
  { \jfameq{\Gamma}{\subst{x}{\emptyf}}{\emptyf}
    }
  \label{comp-s0-c}
  \\
& \inference
  { \jterm{\Gamma}{A}{x}
    \jfam{{\Gamma}{A}}{P}
    }
  { \jfameq
      {{\Gamma}{\subst{x}{P}}}
      {\subst{x}{\emptyf}}
      {\emptyf}
    }
  \label{comp-s0-f}
\end{align}

We use the above rule to state the compatibility of the action on families of
substitution with the action on contexts of substitution. Note that a family
$P$ over $\ctxext{\Gamma}{A}$ may be regarded as a family over
$\ctxext{{\Gamma}{A}}{\emptyf}$. Thus, we may consider the family
$\subst[\famsym]{x}{P}$ over $\ctxext{\Gamma}{\subst{x}{\emptyf}}$. Since
$\subst{x}{\emptyf}$ is judgmentally equal to the empty family, we may compare
$\subst[\famsym]{x}{P}$ with $\subst{x}{P}$:
\begin{equation}
\inference
{ \jfam{{\Gamma}{A}}{P}
  }
{ \jfameq{\Gamma}{\subst[\famsym]{x}{P}}{\subst{x}{P}}
  }
\end{equation}
Due to this rule we need not make the annotations $\famsym$ and $\tmsym$ in
the notation for substitution anymore and thus we shall omit them from now on.
Note that because a family $P$ over $\ctxext{\Gamma}{A}$ is eligible for
application of the action on families of substitution, we may also substitute
terms of $P$. Thus, given terms $\jterm{\Gamma}{A}{x}$ and $\jterm{{\Gamma}{A}}{P}{f}$,
we get a term $\jterm{\Gamma}{\subst{x}{P}}{\subst{x}{f}}$, the \emph{value of
$f$ at $x$}.

In \autoref{morphisms} we will use a combination of weakening and substitution
to define composition of morphisms of families. However, we have to rely
on the cancellation rule stated in \autoref{cancellation-ws} before we can
meaningfully state the definition of composition.

\subsubsection{Fibers of weakenings}\label{comp-sw}
The following rules assert what happens when we first weaken by a family
$Q$ over $P$ in context $\ctxext{\Gamma}{A}$ and then substitute by a term
$x$ of $A$:
\begin{align}
& \inference
  { \jterm{\Gamma}{A}{x}
    \jfam{{{\Gamma}{A}}{P}}{Q}
    \jany{{{\Gamma}{A}}{P}}{e}
    }
  { \janyeq
      {{{\Gamma}{\subst{x}{P}}}{\subst{x}{Q}}}
      {\subst{x}{\ctxwk{Q}{e}}}
      {\ctxwk{\subst{x}{Q}}{\subst{x}{e}}}
    }
  \label{comp-sw-any}
\end{align}

\begin{rmk}
As an important special case of these inference rules we have the following
valid inference rules:
\begin{align*}
& \inference
  { \jterm{\Gamma}{A}{x}
    \jfam{{\Gamma}{A}}{P}
    \jany{{\Gamma}{A}}{e}
    }
  { \janyeq
      {{\Gamma}{\subst{x}{P}}}
      {\subst{x}{\ctxwk{P}{e}}}
      {\ctxwk{\subst{x}{P}}{\subst{x}{e}}}
    }
\end{align*}
Moreover, we get
\begin{equation*}
\inference
  { \jterm{\Gamma}{A}{x}
    \jfam{{\Gamma}{A}}{P}
    \jterm{{\Gamma}{A}}{Q}{g}
    }
  { \jtermeq
      {{\Gamma}{\subst{x}{P}}}
      {\subst{x}{\ctxwk{P}{Q}}}
      {\subst{x}{\ctxwk{P}{g}}}
      {\ctxwk{\subst{x}{P}}{\subst{x}{g}}}
    }
\end{equation*}
\end{rmk}

\subsubsection{Weakenings of fibers}
\label{comp-ws}
The following inference rules explain what happens when we first weaken by a
term $g$ of $Q$ in context $\ctxext{\Gamma}{B}$ and then weaken by a family
$A$ over $\Gamma$.
\begin{align}
& \inference
  { \jfam{\Gamma}{A}
    \jterm{{\Gamma}{B}}{Q}{g}
    \jany{{{\Gamma}{B}}{Q}}{e}
    }
  { \janyeq
      {{{\Gamma}{A}}{\ctxwk{A}{B}}}
      {\ctxwk{A}{\subst{g}{e}}}
      {\subst{\ctxwk{A}{g}}{\ctxwk{A}{e}}}
    }
  \label{comp-ws-any}
\end{align}

\begin{rmk}
As an important special case of these inference rules we have the following 
valid inference rules:
\begin{align*}
& \inference
  { \jfam{\Gamma}{A}
    \jterm{\Gamma}{B}{y}
    \jany{{\Gamma}{B}}{e}
    }
  { \janyeq
      {{\Gamma}{A}}
      {\ctxwk{A}{\subst{y}{e}}}
      {\subst{\ctxwk{A}{y}}{\ctxwk{A}{e}}}
    }
\end{align*}
Moreover, we get
\begin{equation*}
\inference
  { \jfam{\Gamma}{A}
    \jterm{\Gamma}{B}{y}
    \jterm{{\Gamma}{B}}{Q}{g}
    }
  { \jtermeq
      {{\Gamma}{A}}
      {\ctxwk{A}{\subst{y}{Q}}}
      {\ctxwk{A}{\subst{y}{g}}}
      {\subst{\ctxwk{A}{y}}{\ctxwk{A}{g}}}
    }
\end{equation*}
\end{rmk}

\subsubsection{The cancellation property of weakening and substitution}
\label{cancellation-ws}
The judgmental equalities we're about to describe assert that substituting a term
in the weakening a thing gives you the thing back. In the case of contexts we get that each fiber
$\subst{x}{\ctxwk{A}{B}}$ is just $B$ and in the case of terms we get 
that $\ctxwk{A}{y}$ is the constant function
mapping everything to $y:B$. Thus, these rules actually establish the weakening
as the weakening. After stating the rules we will describe what it means to
compose context morphisms (terms of weakened contexts).
\begin{align}
& \inference
  { \jterm{\Gamma}{A}{x}
    \jany{\Gamma}{e}
    }
  { \janyeq
      {\Gamma}
      {\subst{x}{\ctxwk{A}{e}}}
      {e}
    }
  \label{cancellation-ws-e}
\end{align}

\subsubsection{The identity term of a substituted family}
\label{comp-si}
The identity term of a substituted family is the substitution of the identity term
\begin{equation}
\inference
  { \jterm{\Gamma}{A}{x}
    \jfam{{{\Gamma}{A}}{P}}{Q}
    }
  { \unfoldall{\jhomeq
      {{\Gamma}{\subst{x}{P}}}
      {\subst{x}{Q}}
      {\subst{x}{Q}}
      {\subst{x}{\idtm{Q}}}
      {\idtm{\subst{x}{Q}}}
    }}
  \label{comp-si-t}
\end{equation}

\begin{rmk}
An important special case is the judgmental equality
\begin{equation*}
\unfoldall{\jhomeq
      {\Gamma}
      {\subst{x}{P}}
      {\subst{x}{P}}
      {\subst{x}{\idtm{P}}}
      {\idtm{\subst{x}{P}}}
    }
\end{equation*}
for a family $\jfam{{\Gamma}{A}}{P}$.
\end{rmk}

\subsubsection{The cancellation property of identity terms}
\label{cancellation-i}
Identity terms are determined by their behavior with respect to substitution combined with
weakening. The identity terms will also be subject to compatibility rules.
\begin{align}
& \inference
  { \jterm{\Gamma}{A}{x}
    }
  { \jtermeq{\Gamma}{A}{\subst{x}{\idtm{A}}}{x}
    }
  \label{cancellation-si}\\
& \inference
  { \jany{{\Gamma}{A}}{e}
    }
  { \janyeq{{\Gamma}{A}}{\subst{\idtm{A}}{\ctxwk{A}{e}}}{e}
    }
  \label{precomp-idtm-any}
\end{align}
