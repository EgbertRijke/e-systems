\subsection{The theory of substitution}
\label{substitution}

The theory of substitution can be formulated separately from the theory of
weakening. Therefore, we will only require the theory of extension and the
empty context and families. In \autoref{sec:esystem-equalities} we will
provide the additional judgmental equality rules that will be required when
combining the theories of substitution and of projections.

Given a family $P$ over $A$ and a term $x$ of $A$, substitution gives a way to
consider the \emph{fiber $\subst{x}{P}$ of $P$ at $x$}. As was the case with
weakening, the substitution operation comes in three parts: the `action on
contexts' of substitution is the operation just described; the `action on
families' of substitution takes a family $Q$ over $\ctxext{{\Gamma}{A}}{P}$
to a family $\subst[\famsym]{x}{Q}$ over $\ctxext{\Gamma}{\subst{x}{P}}$; the
`action on terms' of substitution takes a term $g$ of $Q$ over
$\ctxext{{\Gamma}{A}}{P}$ to a term $\subst[\tmsym]{x}{g}$ of 
$\subst[\famsym]{x}{Q}$.
\begin{align}
& \inference
  { \jterm{\Gamma}{A}{x}
    \jany{{\Gamma}{A}}{e}
    }
  { \jany{\Gamma}{\subst{x}{e}}
    }
\end{align}

\subsubsection{Fibers of extensions}
\label{comp-se}
The following inference rules assert that if we take the fiber of an extended
family at a term $x$ of $A$ in context $\Gamma$, the substitution by $x$
distributes over the factors of the extension.
\begin{align}
& \inference
  { \jterm{\Gamma}{A}{x}
    \jfam{{{\Gamma}{A}}{P}}{Q}
    }
  { \jfameq
      {\Gamma}
      {\subst{x}{\ctxext{P}{Q}}}
      {\ctxext{\subst{x}{P}}{\subst{x}{Q}}}
    }
  \label{comp-se-c}
  \\
& \inference
  { \jterm{\Gamma}{A}{x}
    \jfam{{{{\Gamma}{A}}{P}}{Q}}{R}
    }
  { \jfameq
      {{\Gamma}{\subst{x}{P}}}
      {\subst{x}{\ctxext{Q}{R}}}
      {\ctxext{\subst{x}{Q}}{\subst{x}{R}}}
    }
  \label{comp-se-f}
\end{align}
There is also a version of this statement in which extended terms are considered.
This variant is stated and proved in \autoref{comp-se-t}.

\subsubsection{Fibers of fibers}
\label{comp-ss}

The following rules explain what happens when we first substitute by a term
$g$ of $Q$ over $P$ in context $\ctxext{\Gamma}{A}$ and then by a term $x$ of
$A$:
\begin{align}
& \inference
  { \jterm{\Gamma}{A}{x}
    \jterm{{{\Gamma}{A}}{P}}{Q}{g}
    \jany{{{{\Gamma}{A}}{P}}{Q}}{e}
    }
  { \janyeq
      {{{\Gamma}{\subst{x}{P}}}{\subst{x}{Q}}}
      {\subst{x}{{g}{e}}}
      {\subst{{x}{g}}{{x}{e}}}
    }
  \label{comp-ss-any}
\end{align}

\begin{rmk}
As an important special case of these inference rules we have the following
valid inference rules:
\begin{align*}
& \inference
  { \jterm{\Gamma}{A}{x}
    \jterm{{\Gamma}{A}}{P}{f}
    \jany{{{\Gamma}{A}}{P}}{e}
    }
  { \janyeq
      {\Gamma}
      {\subst{x}{{f}{e}}}
      {\subst{{x}{f}}{{x}{e}}}
    }
\end{align*}
Moreover, in the presence of empty families we get
\begin{equation*}
\inference
  { \jterm{\Gamma}{A}{x}
    \jterm{{\Gamma}{A}}{P}{f}
    \jterm{{{\Gamma}{A}}{P}}{Q}{g}
    }
  { \jtermeq
      {\Gamma}
      {\subst{x}{{f}{Q}}}
      {\subst{x}{{f}{g}}}
      {\subst{{x}{f}}{{x}{g}}}
    }
\end{equation*}
\end{rmk}
