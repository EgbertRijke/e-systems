\subsection{The empty context and the empty families}
\label{empty}
We introduce an empty context and an family over $\Gamma$ for every context $\Gamma$. 
It has been suggested by some to only include empty families and not an
empty context because an empty context is not necessary, but we do have several
reasons to include them. Having an empty context requires also rules asserting 
that a context is the same thing as a family over the empty context and this
gives the categorical structure on contexts for free once one has it for 
families. The main ingredients that will be missing from the theory once an
empty context is avoided are weakening by a context and identity morphisms from
a context to itself. Including these by hand also requires to formulate all the
compatibility rules involving weakening once more for the cases of weakening by
a context and identity terms at contexts. We prefer to state these rules once
and only once and including an empty context helps in this respect.

We also prefer to have our scheme of compatibility rules as symmetrical as
possible. The structure of type dependency should look exactly the same in the
default case as in any context. In that respect, having an empty family but not
an empty context seems a bit odd. Also, some sets of compatibility rules 
(like the rules stating that extension is compatible with the empty families)
will become assymetrical as a result of not including an empty context. Moreover,
we would eventually like to include a stratification of the theory by means of
a type judgment (asserting that a family in a context $\Gamma$ is a type) and
study closed types (i.e.~types in the empty context). It would be possible to
provide a notion of closed types without having an empty context, but this would
have to be formulated separately and we would have to restate all the rules for
types (if any) for closed types all over again.

One of the main uses of the empty context and the empty families is that we
get the property that the `action on contexts' of an operation is compatible
with its `action on families'.

The empty family over a context $\Gamma$ is introduced by the following rule
inference rule:
\begin{align}
& \inference
  { }
  { \jctx{\emptyc}
    }
  \\
& \inference
  { \jctx{\Gamma}
    }
  { \jfam{\Gamma}{\emptyf[\Gamma]}
    }
  \\
& \inference
  { \jctxeq{\Gamma}{\Gamma'}
    }
  { \jfameq{\Gamma}{\emptyf[\Gamma]}{\emptyf[\Gamma']}
    }
\end{align}

By regarding contexts as families of contexts over the empty context, we
enable ourselves also to speak of terms of contexts. A term of a context
$\Gamma$ is a term of the family $\Gamma$ over the empty context. These ideas
are captured in the following convertibility rules:
\begin{align}
& \inference
  { \jctx{\Gamma}
    }
  { \jfam{\emptyc}{i(\Gamma)}
    } 
  &
& \inference
  { \jctxeq{\Gamma}{\Delta}
    }
  { \jfameq{\emptyc}{i(\Gamma)}{i(\Delta)}
    }
\end{align}

The reader may wonder whether the empty family $\emptyf$ always has a
term. This shall follow from the rules stating the compatibility of extension
with the empty families in \autoref{comp-0e} below and from
identity terms (\autoref{identityterms}).

\subsubsection{Compatibility of extension with the empty context and families}
In the following set of inference rules we state that the empty context and
the empty family are neutral objects for both context extension (the first two
rules) and family extension (the last two rules).
\label{comp-e0}\label{comp-0e}
\begin{align}
& \inference
  { \jctx{\Gamma}
    }
  { \jctxeq{\ctxext{\emptyc}{i(\Gamma)}}{\Gamma}
    }
  \label{comp-0e-c}
  \\
& \inference
  { \jctx{\Gamma}
    }
  { \jctxeq{\ctxext{\Gamma}{\emptyf}}{\Gamma}
    }
  \label{comp-e0-c}\\
& \inference
  { \jfam{\Gamma}{A}
    }
  { \jfameq{\Gamma}{\ctxext{\emptyf}{A}}{A}
    }
  \label{comp-0e-f}
  \\
& \inference
  { \jfam{\Gamma}{A}
    }
  { \jfameq{\Gamma}{{A}{\emptyf}}{A}
    }
  \label{comp-e0-f}
  \\
& \inference
  { \jfam{\emptyc}{A}
    }
  { \jfameq{\emptyc}{i(\ctxext{\emptyc}{A})}{A}
    }
\end{align}

\subsubsection{Family extension restricted to the empty context}
Strictly speaking we should have used a different notation for context extension
as for family extension, because the following rule asserting that family extension
in the empty context is the same thing as context extension would look tautological
without a difference. So let us denote, only for the moment, context extension
of $\Gamma$ by $A$ by $(\ctxext{\Gamma}{A})^c$ and family extension of $A$ by
$P$ in context $\Gamma$ by $(\ctxext{A}{P})^\famsym$. 

Note that we may consider a context $\Gamma$ as a family over $\emptyc$ and
a family $\jfam{\Gamma}{A}$ as a family $\jfam{{\emptyc}{\Gamma}}{A}$. 
Therefore we will require the following rule:
\begin{align}
& \inference
  { \jfam{\Gamma}{A}
    }
  { \jctxeq{\ctxext{\Gamma}{A}}{\ctxext{i(\Gamma)}{A}}
    }
\end{align}
Note that this rule actually justifies that we have not utilized two different
notations for context extension and family extension.
