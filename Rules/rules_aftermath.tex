\subsection{%
  The possiblity of types in the theory of contexts, families and terms}
\label{types}

We have deliberately not spoken of types so far because we have taken the point
of view that a type in a context is nothing but a family in that context which
belongs to the class of types. We think of types as \emph{irreducible} families,
i.e.\ families which are neither empty nor the extension of two
families which are both not empty (in algebraic terminology: which
are both non-trivial). To allow ourselves to speak of types we introduce two
new judgments: the judgment that something is a type and the judgment that two
types are equal.
\begin{align*}
\jalign\jtype{\Gamma}{A} 
& \jalign\jtypeeq{\Gamma}{A}{B}
\end{align*}
But only families of contexts are eligible to be types. If $A$ is a type
in context $\Gamma$, then $A$ is also a family of contexts over $\Gamma$. 
Moreover, two types in context $\Gamma$ are judgmentally equal precisely when they are equal
as context families and if a family $B$ of contexts over $\Gamma$ is
judgmentally equal to a type $A$ in context $\Gamma$, then $B$ is a type in
context $\Gamma$. This is expressed by the following four inference rules:
\begin{align*}
& \inference
  { \jtype{\Gamma}{A}
    }
  { \jfam{\Gamma}{A}
    }
& & \inference
    { \jtypeeq{\Gamma}{A}{B}
      }
    { \jfameq{\Gamma}{A}{B}
      }
    \\
& \inference
  { \jtype{\Gamma}{A}
    \jfameq{\Gamma}{A}{B}
    }
  { \jtype{\Gamma}{B}
    }
& & \inference
    { \jtype{\Gamma}{A}
      \jfameq{\Gamma}{A}{B}
      }
    { \jtypeeq{\Gamma}{A}{B}
      }
\end{align*}
As pointed out at the beginning of this subsection, 
we do not assume that the empty family is a type, that would be like
assuming that the multiplicative unit of a ring is prime. 

We add rules asserting that a weakened type is again a type that 
substitution preserves the property of being a type:
\begin{align}
& \inference
  { \jfam{\Gamma}{B}
    \jtype{\ctxext{\Gamma}{B}}{Q}
    }
  { \jtype{\ctxext{{\Gamma}{A}}{\ctxwk{A}{B}}}{\ctxwk{A}{Q}}
    }
  \\
& \inference
  { \jterm{\Gamma}{A}{x}
    \jtype{\ctxext{{\Gamma}{A}}{P}}{Q}
    }
  { \jtype{\ctxext{\Gamma}{\subst{x}{P}}}{\subst{x}{Q}}
    }
\end{align}
With only the current rules, the possibility of making the judgment that
something is a type does not add much to the theory of contexts, families and
terms. Nevertheless, when studying models, having an interpretation for the
judgment that something is a type will allow for the possibility to study
conditions such as the one asserting that every family `factorizes' uniquely
as multiple applications of extension to types, analogous to the condition on
unique factorization domains in ring theory. All the tradional models of type
theory should translate to models with such a condition, simply because
traditionally contexts are viewed as lists of variable (and type) declarations.
