In this section we give a description of dependent type theory before type
constructors. Apart from contexts, families and terms -- which provide for the
core of the language of dependent logic -- the basic ingredients
of this theory will be the operations of extension on context and on families, 
and empty context and empty families, weakening, substitution
and the identity terms. The resulting theory can be seen as a manifestation of the 
structure underlying dependent type theory.

We will
formulate the theory of contexts, families and terms in such a way that contexts aren't defined
to be lists of variable declarations. The variable-free (a.k.a.~\emph{name-free}) approach 
we take here is rather different than those appearing in 
\cite{hofmann1995extensional,TheBook} but it has appeared in the work of Coquand
and in \cite{Dybjer1996}.
The main reason we don't let variable declarations in is that we don't see them 
in the internal models either. This way we also set out to a more algebraic 
approach of type theory and higher category theory. Thirdly, we will not have to
be burdened with superficial comments about variables being bounded or not, or 
fresh or free or not occuring at all.

In the current presentation, contexts can be seen as binary
planar trees of which the leaves are (families of) contexts. 
The judgmental equality relation on contexts is an equivalence relation which 
expresses that binary planar
trees of contexts are judgmentally equal if their leaves are, taking only
(the isomorphism class of) 
the order of the leaves into account \emph{and not the actual shape of the three}.
The intuition behind this equivalence relation is indeed that unstructured
(i.e.~unbracketed) lists such as the lists of variable declarations which
usually appear in type theoretical syntax, may be regarded as contexts.

Besides contexts, families and terms there will also be a notion of `type in
a context', which one may assume.
A context is eligible to be a \emph{closed type} and over a context, a family
may be a type. We have the intuition that `being a type'
expresses the property of `being atomic or irreducible'. With a notion of types,
one has the means to require a stratification by which every context is a finite
extension of types starting at a closed type, and every family is likewise a
finite extension of types. The idea behind this is to recover the theory of
\cite{Garner2014} {\color{red}(and possibly B-systems?)}. We stress that
we will not include axioms asserting such a stratification in the E-system
and we leave discussion on the possibility of a typing judgment until the
end of this section. 

With families of contexts being the principal things of
study, we have added a bit more generality to the theory, compared to the
theory of categories with families presented in \cite{Dybjer1996}. Here are
a couple of reasons for this extra generality:
\begin{enumerate}
\item We want to be very sure that we can interpret all the rules of type theory
in a straightforward manner in any context.
\item We do want do end up with a categorical structure on the contexts, derived
from the theory by requiring that a morphism from $\Gamma$ to $\Delta$ is a
term of the weakening of $\Delta$ by $\Gamma$ in context $\Gamma$. Thus, we
must be able to lift the context $\Delta$ to become a family of contexts over
$\Gamma$ and it is not reasonable to expect that this results in a type.
\item It is very easy in this setting to assert that an operation has an action
on very high levels of families, i.e.~when $F$ is an operation taking things in
context $\Gamma$ to things in context $\Delta$, then it takes a family $A$ in
context $\Gamma$ to $F(A)$ in context $\Delta$; a family $P$ in context
$\ctxext{\Gamma}{A}$ to a family $F(P)$ in context $\ctxext{\Delta}{(F(A))}$,
and so on. The reason it becomes simple is that we don't have to explain what
`and so on' means.
\end{enumerate}

We will formulate fairly strict rules governing the judgmental equalities,
expressing that extension, weakening and substitution are combatible with
each other in a judgmental manner. This does not, however, diminish the role
of isomorphisms or of homotopies could play in the theory once identity types
are added. Indeed, types could still have non-trivial identity relations and
the category of types in a certain context could genuinely display higher
categorical structure, or so we conjecture.

Much of the rules we state are just compatibility rules of extension, weakening
and substitution with each other. In a way, these rules assert that our contexts
are just structureless lists of contexts and that likewise terms are structureless
lists of terms. They are structureless in the sense that the order in which
they are formed by pairing up is irrelevant. We note that this causes complications
in the traditional way that categorical sematics of type theory is implemented,
where contexts become objects of the category which is supposed to model type
theory. The reason for this is that context extension will not satisfy all the
compatibility rules we're about to state. The first step to resolving this is taking
the types in the empty context as the objects.
