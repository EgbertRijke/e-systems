%%%%%%%%%%%%%%%%%%%%%%%%%%%%%%%%%%%%%%%%%%%%%%%%%%%%%%%%%%%%%%%%%%%%%%%%%%%%%%%%
%%%% PACKAGES

\usepackage[utf8]{inputenc}
\usepackage[english]{babel}

%%%% Spicing up the document
\usepackage{mathpazo}
\usepackage[scaled=0.95]{helvet}
\usepackage{courier}
\linespread{1.05} % Palatino looks better with this
\usepackage{microtype}

\usepackage{fancyhdr} % To set headers and footers
\usepackage{enumitem,mathtools,xspace,xcolor}
\usepackage{comment}
\usepackage{ifthen}

\usepackage{graphicx}
\usepackage{tikz-cd}
\usepackage{tikz}
\usetikzlibrary{decorations.pathmorphing}
\usepackage[inference]{semantic}
\usepackage{booktabs}

\usepackage[hyphens]{url} % This package has to be loaded *before* hyperref
\usepackage[pagebackref,colorlinks,citecolor=darkgreen,linkcolor=darkgreen,unicode]{hyperref}
\definecolor{darkgreen}{rgb}{0,0.45,0}

% For some reason the following can't be above hyperref...
\usepackage{amssymb,amsmath,amsthm,stmaryrd,mathrsfs,wasysym}
\usepackage{aliascnt}
\usepackage[capitalize]{cleveref}

% The braket macro shouldn't be necessary
\usepackage{braket} % used for \setof{ ... } macro

%%%%%%%%%%%%%%%%%%%%%%%%%%%%%%%%%%%%%%%%%%%%%%%%%%%%%%%%%%%%%%%%%%%%%%%%%%%%%%%%
%% To include references in TOC we should use this package rather than a hack.
\usepackage{tocbibind}
%\usepackage{etoolbox}           % get \apptocmd
%\apptocmd{\thebibliography}{\addcontentsline{toc}{section}{References}}{}{} % tell bibliography to get itself into the table of contents


\begin{comment}
%%%% Header and footers
\pagestyle{fancyplain}
\setlength{\headheight}{15pt}
\renewcommand{\chaptermark}[1]{\markboth{\textsc{Chapter \thechapter. #1}}{}}
\renewcommand{\sectionmark}[1]{\markright{\textsc{\thesection\ #1}}}
\end{comment}

% TOC depth
\setcounter{tocdepth}{3}

\lhead[\fancyplain{}{{\thepage}}]%
      {\fancyplain{}{\nouppercase{\rightmark}}}
\rhead[\fancyplain{}{\nouppercase{\leftmark}}]%
      {\fancyplain{}{\thepage}}
\cfoot{\textsc{\footnotesize [Draft of \today]}}
\lfoot[]{}
\rfoot[]{}

%%%%%%%%%%%%%%%%%%%%%%%%%%%%%%%%%%%%%%%%%%%%%%%%%%%%%%%%%%%%%%%%%%%%%%%%%%%%%%%%
%%%% We mostly use the macros of the book, to keep notations
%%%% and conventions the same. Recall that when the macros file
%%%% is updated, we need to comment the lines containing the
%%%% string `[chapter]` since our article is not a book.
%%%%
%%%% Instructions for updating the macros.tex file:
%%%% - fetch the latest macros.tex file from the HoTT/book git repository.
%%%% - comment all lines containing "[chapter]" because this is not a book.
%%%% - comment the definition of pbcorner because the xypic package is not used.
%%%%
%%%% MACROS FOR NOTATION %%%%
% Use these for any notation where there are multiple options.

%%% Notes and exercise sections
\makeatletter
\newcommand{\sectionNotes}{\phantomsection\section*{Notes}\addcontentsline{toc}{section}{Notes}\markright{\textsc{\@chapapp{} \thechapter{} Notes}}}
\newcommand{\sectionExercises}[1]{\phantomsection\section*{Exercises}\addcontentsline{toc}{section}{Exercises}\markright{\textsc{\@chapapp{} \thechapter{} Exercises}}}
\makeatother

%%% Definitional equality (used infix) %%%
\newcommand{\jdeq}{\equiv}      % An equality judgment
\let\judgeq\jdeq
%\newcommand{\defeq}{\coloneqq}  % An equality currently being defined
\newcommand{\defeq}{\vcentcolon\equiv}  % A judgmental equality currently being defined

%%% Term being defined
\newcommand{\define}[1]{\textbf{#1}}

%%% Vec (for example)

\newcommand{\Vect}{\ensuremath{\mathsf{Vec}}}
\newcommand{\Fin}{\ensuremath{\mathsf{Fin}}}
\newcommand{\fmax}{\ensuremath{\mathsf{fmax}}}
\newcommand{\seq}[1]{\langle #1\rangle}

%%% Dependent products %%%
\def\prdsym{\textstyle\prod}
%% Call the macro like \prd{x,y:A}{p:x=y} with any number of
%% arguments.  Make sure that whatever comes *after* the call doesn't
%% begin with an open-brace, or it will be parsed as another argument.
\makeatletter
% Currently the macro is configured to produce
%     {\textstyle\prod}(x:A) \; {\textstyle\prod}(y:B),\ 
% in display-math mode, and
%     \prod_{(x:A)} \prod_{y:B}
% in text-math mode.
\def\prd#1{\@ifnextchar\bgroup{\prd@parens{#1}}{\@ifnextchar\sm{\prd@parens{#1}\@eatsm}{\prd@noparens{#1}}}}
\def\prd@parens#1{\@ifnextchar\bgroup%
  {\mathchoice{\@dprd{#1}}{\@tprd{#1}}{\@tprd{#1}}{\@tprd{#1}}\prd@parens}%
  {\@ifnextchar\sm%
    {\mathchoice{\@dprd{#1}}{\@tprd{#1}}{\@tprd{#1}}{\@tprd{#1}}\@eatsm}%
    {\mathchoice{\@dprd{#1}}{\@tprd{#1}}{\@tprd{#1}}{\@tprd{#1}}}}}
\def\@eatsm\sm{\sm@parens}
\def\prd@noparens#1{\mathchoice{\@dprd@noparens{#1}}{\@tprd{#1}}{\@tprd{#1}}{\@tprd{#1}}}
% Helper macros for three styles
\def\lprd#1{\@ifnextchar\bgroup{\@lprd{#1}\lprd}{\@@lprd{#1}}}
\def\@lprd#1{\mathchoice{{\textstyle\prod}}{\prod}{\prod}{\prod}({\textstyle #1})\;}
\def\@@lprd#1{\mathchoice{{\textstyle\prod}}{\prod}{\prod}{\prod}({\textstyle #1}),\ }
\def\tprd#1{\@tprd{#1}\@ifnextchar\bgroup{\tprd}{}}
\def\@tprd#1{\mathchoice{{\textstyle\prod_{(#1)}}}{\prod_{(#1)}}{\prod_{(#1)}}{\prod_{(#1)}}}
\def\dprd#1{\@dprd{#1}\@ifnextchar\bgroup{\dprd}{}}
\def\@dprd#1{\prod_{(#1)}\,}
\def\@dprd@noparens#1{\prod_{#1}\,}

%%% Lambda abstractions.
% Each variable being abstracted over is a separate argument.  If
% there is more than one such argument, they *must* be enclosed in
% braces.  Arguments can be untyped, as in \lam{x}{y}, or typed with a
% colon, as in \lam{x:A}{y:B}. In the latter case, the colons are
% automatically noticed and (with current implementation) the space
% around the colon is reduced.  You can even give more than one variable
% the same type, as in \lam{x,y:A}.
\def\lam#1{{\lambda}\@lamarg#1:\@endlamarg\@ifnextchar\bgroup{.\,\lam}{.\,}}
\def\@lamarg#1:#2\@endlamarg{\if\relax\detokenize{#2}\relax #1\else\@lamvar{\@lameatcolon#2},#1\@endlamvar\fi}
\def\@lamvar#1,#2\@endlamvar{(#2\,{:}\,#1)}
% \def\@lamvar#1,#2{{#2}^{#1}\@ifnextchar,{.\,{\lambda}\@lamvar{#1}}{\let\@endlamvar\relax}}
\def\@lameatcolon#1:{#1}
\let\lamt\lam
% This version silently eats any typing annotation.
\def\lamu#1{{\lambda}\@lamuarg#1:\@endlamuarg\@ifnextchar\bgroup{.\,\lamu}{.\,}}
\def\@lamuarg#1:#2\@endlamuarg{#1}

%%% Dependent products written with \forall, in the same style
\def\fall#1{\forall (#1)\@ifnextchar\bgroup{.\,\fall}{.\,}}

%%% Existential quantifier %%%
\def\exis#1{\exists (#1)\@ifnextchar\bgroup{.\,\exis}{.\,}}

%%% Dependent sums %%%
\def\smsym{\textstyle\sum}
% Use in the same way as \prd
\def\sm#1{\@ifnextchar\bgroup{\sm@parens{#1}}{\@ifnextchar\prd{\sm@parens{#1}\@eatprd}{\sm@noparens{#1}}}}
\def\sm@parens#1{\@ifnextchar\bgroup%
  {\mathchoice{\@dsm{#1}}{\@tsm{#1}}{\@tsm{#1}}{\@tsm{#1}}\sm@parens}%
  {\@ifnextchar\prd%
    {\mathchoice{\@dsm{#1}}{\@tsm{#1}}{\@tsm{#1}}{\@tsm{#1}}\@eatprd}%
    {\mathchoice{\@dsm{#1}}{\@tsm{#1}}{\@tsm{#1}}{\@tsm{#1}}}}}
\def\@eatprd\prd{\prd@parens}
\def\sm@noparens#1{\mathchoice{\@dsm@noparens{#1}}{\@tsm{#1}}{\@tsm{#1}}{\@tsm{#1}}}
\def\lsm#1{\@ifnextchar\bgroup{\@lsm{#1}\lsm}{\@@lsm{#1}}}
\def\@lsm#1{\mathchoice{{\textstyle\sum}}{\sum}{\sum}{\sum}({\textstyle #1})\;}
\def\@@lsm#1{\mathchoice{{\textstyle\sum}}{\sum}{\sum}{\sum}({\textstyle #1}),\ }
\def\tsm#1{\@tsm{#1}\@ifnextchar\bgroup{\tsm}{}}
\def\@tsm#1{\mathchoice{{\textstyle\sum_{(#1)}}}{\sum_{(#1)}}{\sum_{(#1)}}{\sum_{(#1)}}}
\def\dsm#1{\@dsm{#1}\@ifnextchar\bgroup{\dsm}{}}
\def\@dsm#1{\sum_{(#1)}\,}
\def\@dsm@noparens#1{\sum_{#1}\,}

%%% W-types
\def\wtypesym{{\mathsf{W}}}
\def\wtype#1{\@ifnextchar\bgroup%
  {\mathchoice{\@twtype{#1}}{\@twtype{#1}}{\@twtype{#1}}{\@twtype{#1}}\wtype}%
  {\mathchoice{\@twtype{#1}}{\@twtype{#1}}{\@twtype{#1}}{\@twtype{#1}}}}
\def\lwtype#1{\@ifnextchar\bgroup{\@lwtype{#1}\lwtype}{\@@lwtype{#1}}}
\def\@lwtype#1{\mathchoice{{\textstyle\mathsf{W}}}{\mathsf{W}}{\mathsf{W}}{\mathsf{W}}({\textstyle #1})\;}
\def\@@lwtype#1{\mathchoice{{\textstyle\mathsf{W}}}{\mathsf{W}}{\mathsf{W}}{\mathsf{W}}({\textstyle #1}),\ }
\def\twtype#1{\@twtype{#1}\@ifnextchar\bgroup{\twtype}{}}
\def\@twtype#1{\mathchoice{{\textstyle\mathsf{W}_{(#1)}}}{\mathsf{W}_{(#1)}}{\mathsf{W}_{(#1)}}{\mathsf{W}_{(#1)}}}
\def\dwtype#1{\@dwtype{#1}\@ifnextchar\bgroup{\dwtype}{}}
\def\@dwtype#1{\mathsf{W}_{(#1)}\,}

\newcommand{\suppsym}{{\mathsf{sup}}}
\newcommand{\supp}{\ensuremath\suppsym\xspace}

\def\wtypeh#1{\@ifnextchar\bgroup%
  {\mathchoice{\@lwtypeh{#1}}{\@twtypeh{#1}}{\@twtypeh{#1}}{\@twtypeh{#1}}\wtypeh}%
  {\mathchoice{\@@lwtypeh{#1}}{\@twtypeh{#1}}{\@twtypeh{#1}}{\@twtypeh{#1}}}}
\def\lwtypeh#1{\@ifnextchar\bgroup{\@lwtypeh{#1}\lwtypeh}{\@@lwtypeh{#1}}}
\def\@lwtypeh#1{\mathchoice{{\textstyle\mathsf{W}^h}}{\mathsf{W}^h}{\mathsf{W}^h}{\mathsf{W}^h}({\textstyle #1})\;}
\def\@@lwtypeh#1{\mathchoice{{\textstyle\mathsf{W}^h}}{\mathsf{W}^h}{\mathsf{W}^h}{\mathsf{W}^h}({\textstyle #1}),\ }
\def\twtypeh#1{\@twtypeh{#1}\@ifnextchar\bgroup{\twtypeh}{}}
\def\@twtypeh#1{\mathchoice{{\textstyle\mathsf{W}^h_{(#1)}}}{\mathsf{W}^h_{(#1)}}{\mathsf{W}^h_{(#1)}}{\mathsf{W}^h_{(#1)}}}
\def\dwtypeh#1{\@dwtypeh{#1}\@ifnextchar\bgroup{\dwtypeh}{}}
\def\@dwtypeh#1{\mathsf{W}^h_{(#1)}\,}


\makeatother

% Other notations related to dependent sums
\let\setof\Set    % from package 'braket', write \setof{ x:A | P(x) }.
\newcommand{\pair}{\ensuremath{\mathsf{pair}}\xspace}
\newcommand{\tup}[2]{(#1,#2)}
\newcommand{\proj}[1]{\ensuremath{\mathsf{pr}_{#1}}\xspace}
\newcommand{\fst}{\ensuremath{\proj1}\xspace}
\newcommand{\snd}{\ensuremath{\proj2}\xspace}
\newcommand{\ac}{\ensuremath{\mathsf{ac}}\xspace} % not needed in symbol index
\newcommand{\un}{\ensuremath{\mathsf{upun}}\xspace} % not needed in symbol index, uniqueness principle for unit type

%%% recursor and induction
\newcommand{\rec}[1]{\mathsf{rec}_{#1}}
\newcommand{\ind}[1]{\mathsf{ind}_{#1}}
\newcommand{\indid}[1]{\ind{=_{#1}}} % (Martin-Lof) path induction principle for identity types
\newcommand{\indidb}[1]{\ind{=_{#1}}'} % (Paulin-Mohring) based path induction principle for identity types 

%%% the uniqueness principle for product types, formerly called surjective pairing and named \spr:
\newcommand{\uppt}{\ensuremath{\mathsf{uppt}}\xspace}

% Paths in pairs
\newcommand{\pairpath}{\ensuremath{\mathsf{pair}^{\mathord{=}}}\xspace}
% \newcommand{\projpath}[1]{\proj{#1}^{\mathord{=}}}
\newcommand{\projpath}[1]{\ensuremath{\apfunc{\proj{#1}}}\xspace}

%%% For quotients %%%
%\newcommand{\pairr}[1]{{\langle #1\rangle}}
\newcommand{\pairr}[1]{{\mathopen{}(#1)\mathclose{}}}
\newcommand{\Pairr}[1]{{\mathopen{}\left(#1\right)\mathclose{}}}

% \newcommand{\type}{\ensuremath{\mathsf{Type}}} % this command is overridden below, so it's commented out
\newcommand{\im}{\ensuremath{\mathsf{im}}} % the image

%%% 2D path operations
\newcommand{\leftwhisker}{\mathbin{{\ct}_{\ell}}}
\newcommand{\rightwhisker}{\mathbin{{\ct}_{r}}}
\newcommand{\hct}{\star}

%%% modalities %%%
\newcommand{\modal}{\ensuremath{\ocircle}}
\let\reflect\modal
\newcommand{\modaltype}{\ensuremath{\type_\modal}}
% \newcommand{\ism}[1]{\ensuremath{\mathsf{is}_{#1}}}
% \newcommand{\ismodal}{\ism{\modal}}
% \newcommand{\existsmodal}{\ensuremath{{\exists}_{\modal}}}
% \newcommand{\existsmodalunique}{\ensuremath{{\exists!}_{\modal}}}
% \newcommand{\modalfunc}{\textsf{\modal-fun}}
% \newcommand{\Ecirc}{\ensuremath{\mathsf{E}_\modal}}
% \newcommand{\Mcirc}{\ensuremath{\mathsf{M}_\modal}}
\newcommand{\mreturn}{\ensuremath{\eta}}
\let\project\mreturn
%\newcommand{\mbind}[1]{\ensuremath{\hat{#1}}}
\newcommand{\ext}{\mathsf{ext}}
%\newcommand{\mmap}[1]{\ensuremath{\bar{#1}}}
%\newcommand{\mjoin}{\ensuremath{\mreturn^{-1}}}
% Subuniverse
\renewcommand{\P}{\ensuremath{\type_{P}}\xspace}

%%% Localizations
% \newcommand{\islocal}[1]{\ensuremath{\mathsf{islocal}_{#1}}\xspace}
% \newcommand{\loc}[1]{\ensuremath{\mathcal{L}_{#1}}\xspace}

%%% Identity types %%%
\newcommand{\idsym}{{=}}
\newcommand{\id}[3][]{\ensuremath{#2 =_{#1} #3}\xspace}
\newcommand{\idtype}[3][]{\ensuremath{\mathsf{Id}_{#1}(#2,#3)}\xspace}
\newcommand{\idtypevar}[1]{\ensuremath{\mathsf{Id}_{#1}}\xspace}
% A propositional equality currently being defined
\newcommand{\defid}{\coloneqq}

%%% Dependent paths
\newcommand{\dpath}[4]{#3 =^{#1}_{#2} #4}

%%% singleton
% \newcommand{\sgl}{\ensuremath{\mathsf{sgl}}\xspace}
% \newcommand{\sctr}{\ensuremath{\mathsf{sctr}}\xspace}

%%% Reflexivity terms %%%
% \newcommand{\reflsym}{{\mathsf{refl}}}
\newcommand{\refl}[1]{\ensuremath{\mathsf{refl}_{#1}}\xspace}

%%% Path concatenation (used infix, in diagrammatic order) %%%
\newcommand{\ct}{%
  \mathchoice{\mathbin{\raisebox{0.5ex}{$\displaystyle\centerdot$}}}%
             {\mathbin{\raisebox{0.5ex}{$\centerdot$}}}%
             {\mathbin{\raisebox{0.25ex}{$\scriptstyle\,\centerdot\,$}}}%
             {\mathbin{\raisebox{0.1ex}{$\scriptscriptstyle\,\centerdot\,$}}}
}

%%% Path reversal %%%
\newcommand{\opp}[1]{\mathord{{#1}^{-1}}}
\let\rev\opp

%%% Transport (covariant) %%%
\newcommand{\trans}[2]{\ensuremath{{#1}_{*}\mathopen{}\left({#2}\right)\mathclose{}}\xspace}
\let\Trans\trans
%\newcommand{\Trans}[2]{\ensuremath{{#1}_{*}\left({#2}\right)}\xspace}
\newcommand{\transf}[1]{\ensuremath{{#1}_{*}}\xspace} % Without argument
%\newcommand{\transport}[2]{\ensuremath{\mathsf{transport}_{*} \: {#2}\xspace}}
\newcommand{\transfib}[3]{\ensuremath{\mathsf{transport}^{#1}(#2,#3)\xspace}}
\newcommand{\Transfib}[3]{\ensuremath{\mathsf{transport}^{#1}\Big(#2,\, #3\Big)\xspace}}
\newcommand{\transfibf}[1]{\ensuremath{\mathsf{transport}^{#1}\xspace}}

%%% 2D transport
\newcommand{\transtwo}[2]{\ensuremath{\mathsf{transport}^2\mathopen{}\left({#1},{#2}\right)\mathclose{}}\xspace}

%%% Constant transport
\newcommand{\transconst}[3]{\ensuremath{\mathsf{transportconst}}^{#1}_{#2}(#3)\xspace}
\newcommand{\transconstf}{\ensuremath{\mathsf{transportconst}}\xspace}

%%% Map on paths %%%
\newcommand{\mapfunc}[1]{\ensuremath{\mathsf{ap}_{#1}}\xspace} % Without argument
\newcommand{\map}[2]{\ensuremath{{#1}\mathopen{}\left({#2}\right)\mathclose{}}\xspace}
\let\Ap\map
%\newcommand{\Ap}[2]{\ensuremath{{#1}\left({#2}\right)}\xspace}
\newcommand{\mapdepfunc}[1]{\ensuremath{\mathsf{apd}_{#1}}\xspace} % Without argument
% \newcommand{\mapdep}[2]{\ensuremath{{#1}\llparenthesis{#2}\rrparenthesis}\xspace}
\newcommand{\mapdep}[2]{\ensuremath{\mapdepfunc{#1}\mathopen{}\left(#2\right)\mathclose{}}\xspace}
\let\apfunc\mapfunc
\let\ap\map
\let\apdfunc\mapdepfunc
\let\apd\mapdep

%%% 2D map on paths
\newcommand{\aptwofunc}[1]{\ensuremath{\mathsf{ap}^2_{#1}}\xspace}
\newcommand{\aptwo}[2]{\ensuremath{\aptwofunc{#1}\mathopen{}\left({#2}\right)\mathclose{}}\xspace}
\newcommand{\apdtwofunc}[1]{\ensuremath{\mathsf{apd}^2_{#1}}\xspace}
\newcommand{\apdtwo}[2]{\ensuremath{\apdtwofunc{#1}\mathopen{}\left(#2\right)\mathclose{}}\xspace}

%%% Identity functions %%%
\newcommand{\idfunc}[1][]{\ensuremath{\mathsf{id}_{#1}}\xspace}

%%% Homotopies (written infix) %%%
\newcommand{\htpy}{\sim}

%%% Other meanings of \sim
\newcommand{\bisim}{\sim}       % bisimulation
\newcommand{\eqr}{\sim}         % an equivalence relation

%%% Equivalence types %%%
\newcommand{\eqv}[2]{\ensuremath{#1 \simeq #2}\xspace}
\newcommand{\eqvspaced}[2]{\ensuremath{#1 \;\simeq\; #2}\xspace}
\newcommand{\eqvsym}{\simeq}    % infix symbol
\newcommand{\texteqv}[2]{\ensuremath{\mathsf{Equiv}(#1,#2)}\xspace}
\newcommand{\isequiv}{\ensuremath{\mathsf{isequiv}}}
\newcommand{\qinv}{\ensuremath{\mathsf{qinv}}}
\newcommand{\ishae}{\ensuremath{\mathsf{ishae}}}
\newcommand{\linv}{\ensuremath{\mathsf{linv}}}
\newcommand{\rinv}{\ensuremath{\mathsf{rinv}}}
\newcommand{\biinv}{\ensuremath{\mathsf{biinv}}}
\newcommand{\lcoh}[3]{\mathsf{lcoh}_{#1}(#2,#3)}
\newcommand{\rcoh}[3]{\mathsf{rcoh}_{#1}(#2,#3)}
\newcommand{\hfib}[2]{{\mathsf{fib}}_{#1}(#2)}

%%% Map on total spaces %%%
\newcommand{\total}[1]{\ensuremath{\mathsf{total}(#1)}}

%%% Universe types %%%
%\newcommand{\type}{\ensuremath{\mathsf{Type}}\xspace}
\newcommand{\UU}{\ensuremath{\mathcal{U}}\xspace}
\let\bbU\UU
\let\type\UU
% Universes of truncated types
\newcommand{\typele}[1]{\ensuremath{{#1}\text-\mathsf{Type}}\xspace}
\newcommand{\typeleU}[1]{\ensuremath{{#1}\text-\mathsf{Type}_\UU}\xspace}
\newcommand{\typelep}[1]{\ensuremath{{(#1)}\text-\mathsf{Type}}\xspace}
\newcommand{\typelepU}[1]{\ensuremath{{(#1)}\text-\mathsf{Type}_\UU}\xspace}
\let\ntype\typele
\let\ntypeU\typeleU
\let\ntypep\typelep
\let\ntypepU\typelepU
\renewcommand{\set}{\ensuremath{\mathsf{Set}}\xspace}
\newcommand{\setU}{\ensuremath{\mathsf{Set}_\UU}\xspace}
\newcommand{\prop}{\ensuremath{\mathsf{Prop}}\xspace}
\newcommand{\propU}{\ensuremath{\mathsf{Prop}_\UU}\xspace}
%Pointed types
\newcommand{\pointed}[1]{\ensuremath{#1_\bullet}}

%%% Ordinals and cardinals
\newcommand{\card}{\ensuremath{\mathsf{Card}}\xspace}
\newcommand{\ord}{\ensuremath{\mathsf{Ord}}\xspace}
\newcommand{\ordsl}[2]{{#1}_{/#2}}

%%% Univalence
\newcommand{\ua}{\ensuremath{\mathsf{ua}}\xspace} % the inverse of idtoeqv
\newcommand{\idtoeqv}{\ensuremath{\mathsf{idtoeqv}}\xspace}
\newcommand{\univalence}{\ensuremath{\mathsf{univalence}}\xspace} % the full axiom

%%% Truncation levels
\newcommand{\iscontr}{\ensuremath{\mathsf{isContr}}}
\newcommand{\contr}{\ensuremath{\mathsf{contr}}} % The path to the center of contraction
\newcommand{\isset}{\ensuremath{\mathsf{isSet}}}
\newcommand{\isprop}{\ensuremath{\mathsf{isProp}}}
% h-propositions
% \newcommand{\anhprop}{a mere proposition\xspace}
% \newcommand{\hprops}{mere propositions\xspace}

%%% Homotopy fibers %%%
%\newcommand{\hfiber}[2]{\ensuremath{\mathsf{hFiber}(#1,#2)}\xspace}
\let\hfiber\hfib

%%% Bracket/squash/truncation types %%%
% \newcommand{\brck}[1]{\textsf{mere}(#1)}
% \newcommand{\Brck}[1]{\textsf{mere}\Big(#1\Big)}
% \newcommand{\trunc}[2]{\tau_{#1}(#2)}
% \newcommand{\Trunc}[2]{\tau_{#1}\Big(#2\Big)}
% \newcommand{\truncf}[1]{\tau_{#1}}
%\newcommand{\trunc}[2]{\Vert #2\Vert_{#1}}
\newcommand{\trunc}[2]{\mathopen{}\left\Vert #2\right\Vert_{#1}\mathclose{}}
\newcommand{\ttrunc}[2]{\bigl\Vert #2\bigr\Vert_{#1}}
\newcommand{\Trunc}[2]{\Bigl\Vert #2\Bigr\Vert_{#1}}
\newcommand{\truncf}[1]{\Vert \blank \Vert_{#1}}
\newcommand{\tproj}[3][]{\mathopen{}\left|#3\right|_{#2}^{#1}\mathclose{}}
\newcommand{\tprojf}[2][]{|\blank|_{#2}^{#1}}
\def\pizero{\trunc0}
%\newcommand{\brck}[1]{\trunc{-1}{#1}}
%\newcommand{\Brck}[1]{\Trunc{-1}{#1}}
%\newcommand{\bproj}[1]{\tproj{-1}{#1}}
%\newcommand{\bprojf}{\tprojf{-1}}

\newcommand{\brck}[1]{\trunc{}{#1}}
\newcommand{\bbrck}[1]{\ttrunc{}{#1}}
\newcommand{\Brck}[1]{\Trunc{}{#1}}
\newcommand{\bproj}[1]{\tproj{}{#1}}
\newcommand{\bprojf}{\tprojf{}}

% Big parentheses
\newcommand{\Parens}[1]{\Bigl(#1\Bigr)}

% Projection and extension for truncations
\let\extendsmb\ext
\newcommand{\extend}[1]{\extendsmb(#1)}

%
%%% The empty type
\newcommand{\emptyt}{\ensuremath{\mathbf{0}}\xspace}

%%% The unit type
\newcommand{\unit}{\ensuremath{\mathbf{1}}\xspace}
\newcommand{\ttt}{\ensuremath{\star}\xspace}

%%% The two-element type
\newcommand{\bool}{\ensuremath{\mathbf{2}}\xspace}
\newcommand{\btrue}{{1_{\bool}}}
\newcommand{\bfalse}{{0_{\bool}}}

%%% Injections into binary sums and pushouts
\newcommand{\inlsym}{{\mathsf{inl}}}
\newcommand{\inrsym}{{\mathsf{inr}}}
\newcommand{\inl}{\ensuremath\inlsym\xspace}
\newcommand{\inr}{\ensuremath\inrsym\xspace}

%%% The segment of the interval
\newcommand{\seg}{\ensuremath{\mathsf{seg}}\xspace}

%%% Free groups
\newcommand{\freegroup}[1]{F(#1)}
\newcommand{\freegroupx}[1]{F'(#1)} % the "other" free group

%%% Glue of a pushout
\newcommand{\glue}{\mathsf{glue}}

%%% Circles and spheres
\newcommand{\Sn}{\mathbb{S}}
\newcommand{\base}{\ensuremath{\mathsf{base}}\xspace}
\newcommand{\lloop}{\ensuremath{\mathsf{loop}}\xspace}
\newcommand{\surf}{\ensuremath{\mathsf{surf}}\xspace}

%%% Suspension
\newcommand{\susp}{\Sigma}
\newcommand{\north}{\mathsf{N}}
\newcommand{\south}{\mathsf{S}}
\newcommand{\merid}{\mathsf{merid}}

%%% Blanks (shorthand for lambda abstractions)
\newcommand{\blank}{\mathord{\hspace{1pt}\text{--}\hspace{1pt}}}

%%% Nameless objects
\newcommand{\nameless}{\mathord{\hspace{1pt}\underline{\hspace{1ex}}\hspace{1pt}}}

%%% Some decorations
%\newcommand{\bbU}{\ensuremath{\mathbb{U}}\xspace}
% \newcommand{\bbB}{\ensuremath{\mathbb{B}}\xspace}
\newcommand{\bbP}{\ensuremath{\mathbb{P}}\xspace}

%%% Some categories
\newcommand{\uset}{\ensuremath{\mathcal{S}et}\xspace}
\newcommand{\ucat}{\ensuremath{{\mathcal{C}at}}\xspace}
\newcommand{\urel}{\ensuremath{\mathcal{R}el}\xspace}
\newcommand{\uhilb}{\ensuremath{\mathcal{H}ilb}\xspace}
\newcommand{\utype}{\ensuremath{\mathcal{T}\!ype}\xspace}

% Pullback corner
%\newbox\pbbox
%\setbox\pbbox=\hbox{\xy \POS(65,0)\ar@{-} (0,0) \ar@{-} (65,65)\endxy}
%\def\pb{\save[]+<3.5mm,-3.5mm>*{\copy\pbbox} \restore}

% Macros for the categories chapter
\newcommand{\inv}[1]{{#1}^{-1}}
\newcommand{\idtoiso}{\ensuremath{\mathsf{idtoiso}}\xspace}
\newcommand{\isotoid}{\ensuremath{\mathsf{isotoid}}\xspace}
\newcommand{\op}{^{\mathrm{op}}}
\newcommand{\y}{\ensuremath{\mathbf{y}}\xspace}
\newcommand{\dgr}[1]{{#1}^{\dagger}}
\newcommand{\unitaryiso}{\mathrel{\cong^\dagger}}
\newcommand{\cteqv}[2]{\ensuremath{#1 \simeq #2}\xspace}
\newcommand{\cteqvsym}{\simeq}     % Symbol for equivalence of categories

%%% Natural numbers
\newcommand{\N}{\ensuremath{\mathbb{N}}\xspace}
%\newcommand{\N}{\textbf{N}}
\let\nat\N
\newcommand{\natp}{\ensuremath{\nat'}\xspace} % alternative nat in induction chapter

\newcommand{\zerop}{\ensuremath{0'}\xspace}   % alternative zero in induction chapter
\newcommand{\suc}{\mathsf{succ}}
\newcommand{\sucp}{\ensuremath{\suc'}\xspace} % alternative suc in induction chapter
\newcommand{\add}{\mathsf{add}}
\newcommand{\ack}{\mathsf{ack}}
\newcommand{\ite}{\mathsf{iter}}
\newcommand{\assoc}{\mathsf{assoc}}
\newcommand{\dbl}{\ensuremath{\mathsf{double}}}
\newcommand{\dblp}{\ensuremath{\dbl'}\xspace} % alternative double in induction chapter


%%% Lists
\newcommand{\lst}[1]{\mathsf{List}(#1)}
\newcommand{\nil}{\mathsf{nil}}
\newcommand{\cons}{\mathsf{cons}}

%%% Vectors of given length, used in induction chapter
\newcommand{\vect}[2]{\ensuremath{\mathsf{Vec}_{#1}(#2)}\xspace}

%%% Integers
\newcommand{\Z}{\ensuremath{\mathbb{Z}}\xspace}
\newcommand{\Zsuc}{\mathsf{succ}}
\newcommand{\Zpred}{\mathsf{pred}}

%%% Rationals
\newcommand{\Q}{\ensuremath{\mathbb{Q}}\xspace}

%%% Function extensionality
\newcommand{\funext}{\mathsf{funext}}
\newcommand{\happly}{\mathsf{happly}}

%%% A naturality lemma
\newcommand{\com}[3]{\mathsf{swap}_{#1,#2}(#3)}

%%% Code/encode/decode
\newcommand{\code}{\ensuremath{\mathsf{code}}\xspace}
\newcommand{\encode}{\ensuremath{\mathsf{encode}}\xspace}
\newcommand{\decode}{\ensuremath{\mathsf{decode}}\xspace}

% Function definition with domain and codomain
\newcommand{\function}[4]{\left\{\begin{array}{rcl}#1 &
      \longrightarrow & #2 \\ #3 & \longmapsto & #4 \end{array}\right.}

%%% Cones and cocones
\newcommand{\cone}[2]{\mathsf{cone}_{#1}(#2)}
\newcommand{\cocone}[2]{\mathsf{cocone}_{#1}(#2)}
% Apply a function to a cocone
\newcommand{\composecocone}[2]{#1\circ#2}
\newcommand{\composecone}[2]{#2\circ#1}
%%% Diagrams
\newcommand{\Ddiag}{\mathscr{D}}

%%% (pointed) mapping spaces
\newcommand{\Map}{\mathsf{Map}}

%%% The interval
\newcommand{\interval}{\ensuremath{I}\xspace}
\newcommand{\izero}{\ensuremath{0_{\interval}}\xspace}
\newcommand{\ione}{\ensuremath{1_{\interval}}\xspace}

%%% Arrows
\newcommand{\epi}{\ensuremath{\twoheadrightarrow}}
\newcommand{\mono}{\ensuremath{\rightarrowtail}}

%%% Sets
\newcommand{\bin}{\ensuremath{\mathrel{\widetilde{\in}}}}

%%% Semigroup structure
\newcommand{\semigroupstrsym}{\ensuremath{\mathsf{SemigroupStr}}}
\newcommand{\semigroupstr}[1]{\ensuremath{\mathsf{SemigroupStr}}(#1)}
\newcommand{\semigroup}[0]{\ensuremath{\mathsf{Semigroup}}}

%%% Macros for the formal type theory
\newcommand{\emptyctx}{\ensuremath{\cdot}}
\newcommand{\production}{\vcentcolon\vcentcolon=}
\newcommand{\conv}{\downarrow}
\newcommand{\wfctx}[1]{#1\ \ctx}
\newcommand{\oftp}[3]{#1 \vdash #2 : #3}
\newcommand{\jdeqtp}[4]{#1 \vdash #2 \jdeq #3 : #4}
\newcommand{\judg}[2]{#1 \vdash #2}
\newcommand{\tmtp}[2]{#1 \mathord{:} #2}

% rule names
\newcommand{\form}{\textsc{form}}
\newcommand{\intro}{\textsc{intro}}
\newcommand{\elim}{\textsc{elim}}
\newcommand{\comp}{\textsc{comp}}
\newcommand{\uniq}{\textsc{uniq}}
\newcommand{\Weak}{\mathsf{Wkg}}
\newcommand{\Vble}{\mathsf{Vble}}
\newcommand{\Exch}{\mathsf{Exch}}
\newcommand{\Subst}{\mathsf{Subst}}

%%% Macros for HITs
\newcommand{\cc}{\mathsf{c}}
\newcommand{\pp}{\mathsf{p}}
\newcommand{\cct}{\widetilde{\mathsf{c}}}
\newcommand{\ppt}{\widetilde{\mathsf{p}}}
\newcommand{\Wtil}{\ensuremath{\widetilde{W}}\xspace}

%%% Macros for n-types
\newcommand{\istype}[1]{\mathsf{is}\mbox{-}{#1}\mbox{-}\mathsf{type}}
\newcommand{\nplusone}{\ensuremath{(n+1)}}
\newcommand{\nminusone}{\ensuremath{(n-1)}}
\newcommand{\fact}{\mathsf{fact}}

%%% Macros for homotopy
\newcommand{\kbar}{\overline{k}} % Used in van Kampen's theorem

%%% Macros for induction
\newcommand{\natw}{\ensuremath{\mathbf{N^w}}\xspace}
\newcommand{\zerow}{\ensuremath{0^\mathbf{w}}\xspace}
\newcommand{\sucw}{\ensuremath{\mathbf{s^w}}\xspace}
\newcommand{\nalg}{\nat\mathsf{Alg}}
\newcommand{\nhom}{\nat\mathsf{Hom}}
\newcommand{\ishinitw}{\mathsf{isHinit}_{\mathsf{W}}}
\newcommand{\ishinitn}{\mathsf{isHinit}_\nat}
\newcommand{\w}{\mathsf{W}}
\newcommand{\walg}{\w\mathsf{Alg}}
\newcommand{\whom}{\w\mathsf{Hom}}

%%% Macros for real numbers
\newcommand{\RC}{\ensuremath{\mathbb{R}_\mathsf{c}}\xspace} % Cauchy
\newcommand{\RD}{\ensuremath{\mathbb{R}_\mathsf{d}}\xspace} % Dedekind
\newcommand{\R}{\ensuremath{\mathbb{R}}\xspace}           % Either 
\newcommand{\barRD}{\ensuremath{\bar{\mathbb{R}}_\mathsf{d}}\xspace} % Dedekind completion of Dedekind

\newcommand{\close}[1]{\sim_{#1}} % Relation of closeness
\newcommand{\closesym}{\mathord\sim}
\newcommand{\rclim}{\mathsf{lim}} % HIT constructor for Cauchy reals
\newcommand{\rcrat}{\mathsf{rat}} % Embedding of rationals into Cauchy reals
\newcommand{\rceq}{\mathsf{eq}_{\RC}} % HIT path constructor
\newcommand{\CAP}{\mathcal{C}}    % The type of Cauchy approximations
\newcommand{\Qp}{\Q_{+}}
\newcommand{\apart}{\mathrel{\#}}  % apartness
\newcommand{\dcut}{\mathsf{isCut}}  % Dedekind cut
\newcommand{\cover}{\triangleleft} % inductive cover
\newcommand{\intfam}[3]{(#2, \lam{#1} #3)} % family of rational intervals

% Macros for the Cauchy reals construction
\newcommand{\bsim}{\frown}
\newcommand{\bbsim}{\smile}

\newcommand{\hapx}{\diamondsuit\approx}
\newcommand{\hapname}{\diamondsuit}
\newcommand{\hapxb}{\heartsuit\approx}
\newcommand{\hapbname}{\heartsuit}
\newcommand{\tap}[1]{\bullet\approx_{#1}\triangle}
\newcommand{\tapname}{\triangle}
\newcommand{\tapb}[1]{\bullet\approx_{#1}\square}
\newcommand{\tapbname}{\square}

%%% Macros for surreals
\newcommand{\NO}{\ensuremath{\mathsf{No}}\xspace}
\newcommand{\surr}[2]{\{\,#1\,\big|\,#2\,\}}
\newcommand{\LL}{\mathcal{L}}
\newcommand{\RR}{\mathcal{R}}
\newcommand{\noeq}{\mathsf{eq}_{\NO}} % HIT path constructor

\newcommand{\ble}{\trianglelefteqslant}
\newcommand{\blt}{\vartriangleleft}
\newcommand{\bble}{\sqsubseteq}
\newcommand{\bblt}{\sqsubset}

\newcommand{\hle}{\diamondsuit\preceq}
\newcommand{\hlt}{\diamondsuit\prec}
\newcommand{\hlname}{\diamondsuit}
\newcommand{\hleb}{\heartsuit\preceq}
\newcommand{\hltb}{\heartsuit\prec}
\newcommand{\hlbname}{\heartsuit}
% \newcommand{\tle}{(\bullet\preceq\triangle)}
% \newcommand{\tlt}{(\bullet\prec\triangle)}
\newcommand{\tle}{\triangle\preceq}
\newcommand{\tlt}{\triangle\prec}
\newcommand{\tlname}{\triangle}
% \newcommand{\tleb}{(\bullet\preceq\square)}
% \newcommand{\tltb}{(\bullet\prec\square)}
\newcommand{\tleb}{\square\preceq}
\newcommand{\tltb}{\square\prec}
\newcommand{\tlbname}{\square}

%%% Macros for set theory
\newcommand{\vset}{\mathsf{set}}  % point constructor for cummulative hierarchy V
\def\cd{\tproj0}
\newcommand{\inj}{\ensuremath{\mathsf{inj}}} % type of injections
\newcommand{\acc}{\ensuremath{\mathsf{acc}}} % accessibility

\newcommand{\atMostOne}{\mathsf{atMostOne}}

\newcommand{\power}[1]{\mathcal{P}(#1)} % power set
\newcommand{\powerp}[1]{\mathcal{P}_+(#1)} % inhabited power set

%%%% THEOREM ENVIRONMENTS %%%%

% Hyperref includes the command \autoref{...} which is like \ref{...}
% except that it automatically inserts the type of the thing you're
% referring to, e.g. it produces "Theorem 3.8" instead of just "3.8"
% (and makes the whole thing a hyperlink).  This saves a slight amount
% of typing, but more importantly it means that if you decide later on
% that 3.8 should be a Lemma or a Definition instead of a Theorem, you
% don't have to change the name in all the places you referred to it.

% The following hack improves on this by using the same counter for
% all theorem-type environments, so that after Theorem 1.1 comes
% Corollary 1.2 rather than Corollary 1.1.  This makes it much easier
% for the reader to find a particular theorem when flipping through
% the document.
\makeatletter
\def\defthm#1#2#3{%
  %% Ensure all theorem types are numbered with the same counter
  \newaliascnt{#1}{thm}
  \newtheorem{#1}[#1]{#2}
  \aliascntresetthe{#1}
  %% This command tells cleveref's \cref what to call things
  \crefname{#1}{#2}{#3}}

% Now define a bunch of theorem-type environments.
\newtheorem{thm}{Theorem}[section]
\crefname{thm}{Theorem}{Theorems}
%\defthm{prop}{Proposition}   % Probably we shouldn't use "Proposition" in this way
\defthm{cor}{Corollary}{Corollaries}
\defthm{lem}{Lemma}{Lemmas}
\defthm{axiom}{Axiom}{Axioms}
% Since definitions and theorems in type theory are synonymous, should
% we actually use the same theoremstyle for them?
\theoremstyle{definition}
\defthm{defn}{Definition}{Definitions}
\theoremstyle{remark}
\defthm{rmk}{Remark}{Remarks}
\defthm{eg}{Example}{Examples}
\defthm{egs}{Examples}{Examples}
\defthm{notes}{Notes}{Notes}
% Number exercises within chapters, with their own counter.
%\newtheorem{ex}{Exercise}[chapter]
%\crefname{ex}{Exercise}{Exercises}

% Display format for sections
\crefformat{section}{\S#2#1#3}
\Crefformat{section}{Section~#2#1#3}
\crefrangeformat{section}{\S\S#3#1#4--#5#2#6}
\Crefrangeformat{section}{Sections~#3#1#4--#5#2#6}
\crefmultiformat{section}{\S\S#2#1#3}{ and~#2#1#3}{, #2#1#3}{ and~#2#1#3}
\Crefmultiformat{section}{Sections~#2#1#3}{ and~#2#1#3}{, #2#1#3}{ and~#2#1#3}
\crefrangemultiformat{section}{\S\S#3#1#4--#5#2#6}{ and~#3#1#4--#5#2#6}{, #3#1#4--#5#2#6}{ and~#3#1#4--#5#2#6}
\Crefrangemultiformat{section}{Sections~#3#1#4--#5#2#6}{ and~#3#1#4--#5#2#6}{, #3#1#4--#5#2#6}{ and~#3#1#4--#5#2#6}

% Display format for appendices
\crefformat{appendix}{Appendix~#2#1#3}
\Crefformat{appendix}{Appendix~#2#1#3}
\crefrangeformat{appendix}{Appendices~#3#1#4--#5#2#6}
\Crefrangeformat{appendix}{Appendices~#3#1#4--#5#2#6}
\crefmultiformat{appendix}{Appendices~#2#1#3}{ and~#2#1#3}{, #2#1#3}{ and~#2#1#3}
\Crefmultiformat{appendix}{Appendices~#2#1#3}{ and~#2#1#3}{, #2#1#3}{ and~#2#1#3}
\crefrangemultiformat{appendix}{Appendices~#3#1#4--#5#2#6}{ and~#3#1#4--#5#2#6}{, #3#1#4--#5#2#6}{ and~#3#1#4--#5#2#6}
\Crefrangemultiformat{appendix}{Appendices~#3#1#4--#5#2#6}{ and~#3#1#4--#5#2#6}{, #3#1#4--#5#2#6}{ and~#3#1#4--#5#2#6}

\crefname{part}{Part}{Parts}

\crefformat{paragraph}{\S#2#1#3}
\Crefformat{paragraph}{Paragraph~#2#1#3}
\crefrangeformat{paragraph}{\S\S#3#1#4--#5#2#6}
\Crefrangeformat{paragraph}{Paragraphs~#3#1#4--#5#2#6}
\crefmultiformat{paragraph}{\S\S#2#1#3}{ and~#2#1#3}{, #2#1#3}{ and~#2#1#3}
\Crefmultiformat{paragraph}{Paragraphs~#2#1#3}{ and~#2#1#3}{, #2#1#3}{ and~#2#1#3}
\crefrangemultiformat{paragraph}{\S\S#3#1#4--#5#2#6}{ and~#3#1#4--#5#2#6}{, #3#1#4--#5#2#6}{ and~#3#1#4--#5#2#6}
\Crefrangemultiformat{paragraph}{Paragraphs~#3#1#4--#5#2#6}{ and~#3#1#4--#5#2#6}{, #3#1#4--#5#2#6}{ and~#3#1#4--#5#2#6}

% Number subsubsections
\setcounter{secnumdepth}{5}

% Display format for figures
\crefname{figure}{Figure}{Figures}

% Use cleveref instead of hyperref's \autoref
\let\autoref\cref


%%%% EQUATION NUMBERING %%%%

% The following hack uses the single theorem counter to number
% equations as well, so that we don't have both Theorem 1.1 and
% equation (1.1).
\let\c@equation\c@thm
\numberwithin{equation}{section}


%%%% ENUMERATE NUMBERING %%%%

% Number the first level of enumerates as (i), (ii), ...
\renewcommand{\theenumi}{(\roman{enumi})}
\renewcommand{\labelenumi}{\theenumi}


%%%% MARGINS %%%%

% This is a matter of personal preference, but I think the left
% margins on enumerates and itemizes are too wide.
\setitemize[1]{leftmargin=2em}
\setenumerate[1]{leftmargin=*}

% Likewise that they are too spaced out.
\setitemize[1]{itemsep=-0.2em}
\setenumerate[1]{itemsep=-0.2em}

%%% Notes %%%
\def\noteson{%
\gdef\note##1{\mbox{}\marginpar{\color{blue}\textasteriskcentered\ ##1}}}
\gdef\notesoff{\gdef\note##1{\null}}
\noteson

\newcommand{\Coq}{\textsc{Coq}\xspace}
\newcommand{\Agda}{\textsc{Agda}\xspace}
\newcommand{\NuPRL}{\textsc{NuPRL}\xspace}

%%%% CITATIONS %%%%

% \let \cite \citep

%%%% INDEX %%%%

\newcommand{\footstyle}[1]{{\hyperpage{#1}}n} % If you index something that is in a footnote
\newcommand{\defstyle}[1]{\textbf{\hyperpage{#1}}}  % Style for pageref to a definition

\newcommand{\indexdef}[1]{\index{#1|defstyle}}   % Index a definition
\newcommand{\indexfoot}[1]{\index{#1|footstyle}} % Index a term in a footnote
\newcommand{\indexsee}[2]{\index{#1|see{#2}}}    % Index "see also"


%%%% Standard phrasing or spelling of common phrases %%%%

\newcommand{\ZF}{Zermelo--Fraenkel}
\newcommand{\CZF}{Constructive \ZF{} Set Theory}

\newcommand{\LEM}[1]{\ensuremath{\mathsf{LEM}_{#1}}\xspace}
\newcommand{\choice}[1]{\ensuremath{\mathsf{AC}_{#1}}\xspace}

%%%% MISC %%%%

\newcommand{\mentalpause}{\medskip} % Use for "mental" pause, instead of \smallskip or \medskip

%% Use \symlabel instead of \label to mark a pageref that you need in the index of symbols
\newcounter{symindex}
\newcommand{\symlabel}[1]{\refstepcounter{symindex}\label{#1}}

% Local Variables:
% mode: latex
% TeX-master: "hott-online"
% End:


\newcommand{\idsymbin}{=}

%%%%%%%%%%%%%%%%%%%%%%%%%%%%%%%%%%%%%%%%%%%%%%%%%%%%%%%%%%%%%%%%%%%%%%%%%%%%%%%%
%%%% Our commands which are not part of the macros.tex file.
%%%% We should keep these commands separate, because we will
%%%% update the macros.tex following the updates of the book.

%%%% First we redefine the \id, \eqv and \ct commands so that they accept an
%%%% arbitrary number of arguments. This is useful when writing longer strings
%%%% of equalities or equivalences.

\makeatletter

\renewcommand{\id}[3][]{
  \@ifnextchar\bgroup
    {#2 \mathbin{\idsym_{#1}} \id[#1]{#3}}
    {#2 \mathbin{\idsym_{#1}} #3}
  }

\renewcommand{\eqv}[2]{
  \@ifnextchar\bgroup
    {#1 \eqvsym \eqv{#2}}
    {#1 \eqvsym #2}
  }

\newcommand{\ctsym}{%
  \mathchoice{\mathbin{\raisebox{0.5ex}{$\displaystyle\centerdot$}}}%
             {\mathbin{\raisebox{0.5ex}{$\centerdot$}}}%
             {\mathbin{\raisebox{0.25ex}{$\scriptstyle\,\centerdot\,$}}}%
             {\mathbin{\raisebox{0.1ex}{$\scriptscriptstyle\,\centerdot\,$}}}
  }

\renewcommand{\ct}[3][]{
  \@ifnextchar\bgroup
    {#2 \mathbin{\ctsym_{#1}} \ct[#1]{#3}}
    {#2 \mathbin{\ctsym_{#1}} #3}
  }

\makeatother

%%%% We always use textstyle products and sums...
%\renewcommand{\prd}{\tprd}
%\renewcommand{\sm}{\tsm}
\makeatletter
\renewcommand{\@dprd}{\@tprd}
\renewcommand{\@dsm}{\@tsm}
\renewcommand{\@dprd@noparens}{\@tprd}
\renewcommand{\@dsm@noparens}{\@tsm}

%%%% ...with a bit more spacing
\renewcommand{\@tprd}[1]{\mathchoice{{\textstyle\prod_{(#1)}\,}}{\prod_{(#1)}\,}{\prod_{(#1)}\,}{\prod_{(#1)}\,}}
\renewcommand{\@tsm}[1]{\mathchoice{{\textstyle\sum_{(#1)}\,}}{\sum_{(#1)}\,}{\sum_{(#1)}\,}{\sum_{(#1)}\,}}

%%%%%%%%%%%%%%%%%%%%%%%%%%%%%%%%%%%%%%%%%%%%%%%%%%%%%%%%%%%%%%%%%%%%%%%%%%%%%%%%
%%%% We adjust the \prd command so that implicit arguments become possible.
%%%%
%%%% First, we have the following switch. Set it to true if implicit arguments
%%%% are desired, or to false if not. Note turning off implicit arguments
%%%% might render some parts of the text harder to comprehend, since in the
%%%% text might appear $f(x)$ where we would have $f(i,x)$ without implicit
%%%% arguments.

\newcommand{\implicitargumentson}{\boolean{true}}

%%%% If one wants to use implicit arguments in the notation for product types,
%%%% a * has to be put before the argument that has to be implicit.
%%%% For example: in $\prd{x:A}*{y:B}{u:P(y)}Q(x,y,u)$, the argument y is
%%%% implicit. Any of the arguments can be made implicit this way.

%%%% First of all, we should make the command \prd search not only for a
%%%% brace, but also for a star. We introduce an auxiliary command that
%%%% determines whether the next character is a star or brace.
\newcommand{\@ifnextchar@starorbrace}[2]
%  {\@ifnextcharamong{#1}{#2}{*}{\bgroup};}
  {\@ifnextchar*{#1}{\@ifnextchar\bgroup{#1}{#2}}}
  
%%%% When encountering the \prd command, latex should determine whether it
%%%% should print implicit argument brackets or not. So the first branching
%%%% happens right here.
\renewcommand{\prd}{\@ifnextchar*{\@iprd}{\@prd}}

\newcommand{\@prd}[1]
  {\@ifnextchar@starorbrace
    {\prd@parens{#1}}
    {\@ifnextchar\sm{\prd@parens{#1}\@eatsm}{\prd@noparens{#1}}}}
\newcommand{\@prd@parens}{\@ifnextchar*{\@iprd}{\prd@parens}}
\renewcommand{\prd@parens}[1]
  {\@ifnextchar@starorbrace
    {\@theprd{#1}\@prd@parens}
    {\@ifnextchar\sm{\@theprd{#1}\@eatsm}{\@theprd{#1}}}}
\newcommand{\@theprd}[1]
  {\mathchoice{\@dprd{#1}}{\@tprd{#1}}{\@tprd{#1}}{\@tprd{#1}}}
\renewcommand{\dprd}[1]{\@dprd{#1}\@ifnextchar@starorbrace{\dprd}{}}
\renewcommand{\tprd}[1]{\@tprd{#1}\@ifnextchar@starorbrace{\tprd}{}}

%%%% Here we tell the actual symbols to be printed.
\newcommand{\@theiprd}[1]{\mathchoice{\@diprd{#1}}{\@tiprd{#1}}{\@tiprd{#1}}{\@tiprd{#1}}}
\newcommand{\@iprd}[2]{\@ifnextchar@starorbrace%
  {\@theiprd{#2}\@prd@parens}%
  {\@ifnextchar\sm%
    {\@theiprd{#2}\@eatsm}%
    {\@theiprd{#2}}}}
\def\@tiprd#1{
  \ifthenelse{\implicitargumentson}
    {\@@tiprd{#1}\@ifnextchar\bgroup{\@tiprd}{}}
    {\@tprd{#1}}}
\def\@@tiprd#1{\mathchoice{{\textstyle\prod_{\{#1\}}\,}}{\prod_{\{#1\}}\,}{\prod_{\{#1\}}\,}{\prod_{\{#1\}}\,}}
\def\@diprd{
  \ifthenelse{\implicitargumentson}
    {\@tiprd}
    {\@tprd}}
    

%%%% And finally we need to redefine \@eatprd so that implicit arguments also
%%%% works in the scope of a dependent sum.    
\def\@eatprd\prd{\@prd@parens}

\makeatother

%%%%%%%%%%%%%%%%%%%%%%%%%%%%%%%%%%%%%%%%%%%%%%%%%%%%%%%%%%%%%%%%%%%%%%%%%%%%%%%%
%%%% Redefining the quantifiers, so that some of the longer 
%%%% formulas appear one a single line without problems

%%% Dependent products written with \forall, in the same style
\makeatletter
\def\tfall#1{\forall_{(#1)}\@ifnextchar\bgroup{\,\tfall}{\,}}
\renewcommand{\fall}{\tfall}

%%% Existential quantifier %%%
\def\texis#1{\exists_{(#1)}\@ifnextchar\bgroup{\,\texis}{\,}}
\renewcommand{\exis}{\texis}

%%% Unique existence %%%
\def\uexis#1{\exists!_{(#1)}\@ifnextchar\bgroup{\,\uexis}{\,}}
\makeatother

%%%%%%%%%%%%%%%%%%%%%%%%%%%%%%%%%%%%%%%%%%%%%%%%%%%%%%%%%%%%%%%%%%%%%%%%%%%%%%%%
%%%% UNFOLD
%%%%
%%%% For each definition in the type theory we make two versions of the macro:
%%%% the macro introducing the new notation and an @unfold version of the macro
%%%% which outputs the meaning of that new notation. Thus, we can use the
%%%% following construction to write our text. When we introduce \macro, we can
%%%% write \unfold{\macro} and the output will be the result of \macro@unfold.

\makeatletter
\newcommand{\unfold}{%
  \unfoldall}
\newcommand{\unfoldall}[1]{%
  \begingroup%
  \renewcommand{\jhom}{\jhom@unfold}%
  \renewcommand{\jhomeq}{\jhomeq@unfold}%
  \renewcommand{\jhomdefn}{\jhomdefn@unfold}%
  \renewcommand{\jfhom}{\jfhom@unfold}%
  \renewcommand{\jcomp}{\jcomp@unfold}%
  \renewcommand{\@jcomp@nested}{\@jcomp@unfold@nested}%
  \renewcommand{\@jcomp@parens}{\@jcomp@unfold@parens}%
  \renewcommand{\tmext}{\tmext@unfold}%
  \renewcommand{\@tmext@nested}{\@tmext@unfold@nested}%
  \renewcommand{\@tmext@parens}{\@tmext@unfold@parens}%
  \renewcommand{\cprojfstf}{\cprojfstf@unfold}%
  \renewcommand{\cprojfst}{\cprojfst@unfold}%
  \renewcommand{\cprojsndf}{\cprojsndf@unfold}%
  \renewcommand{\cprojsnd}{\cprojsnd@unfold}%
  \renewcommand{\jfcomp}{\jfcomp@unfold}%
%  \renewcommand{\@jfcomp@nested}{\@jfcomp@unfold@nested}%
%  \renewcommand{\@jfcomp@parens}{\@jfcomp@unfold@parens}%
  \renewcommand{\sandwich}{\sandwich@unfold}%
  #1%
  \endgroup%
  }

%%%% The following command is useful when you have checked with '\@ifnextchar'
%%%% that the next character is a macro '\firstmacro' and you want to replace
%%%% it by '\secondmacro'. To establish this, simply call for
%%%% '\@ifnextchar\firstmacro{\@eatdo{\secondmacro}}{}' with the second 
%%%% argument of \@eatdo left unspecified.
\newcommand{\@eatdo}[2]{#1}

%%%% The intention of '\unfoldnext' is to unfold only the definition of the
%%%% next character, provided that it is in the list of unfoldable macros.
\newcommand{\unfoldnext}[1]{
  \@ifnextchar\jhom{\@eatdo{\jhom@unfold}}{%
  \@ifnextchar\jhomeq{\@eatdo{\jhomeq@unfold}}{%
  \@ifnextchar\jhomdefn{\@eatdo{\jhomdefn@unfold}}{%
  \@ifnextchar\jfhom{\@eatdo{\jfhom@unfold}}{%
  \@ifnextchar\jcomp{\@eatdo{\jcomp@unfold}}{%
  \@ifnextchar\@jcomp@nested{\@eatdo{\@jcomp@unfold@nested}}{%
  \@ifnextchar\@jcomp@parens{\@eatdo{\@jcomp@unfold@parens}}{%
  \@ifnextchar\tmext{\@eatdo{\tmext@unfold}}{%
  \@ifnextchar\@tmext@nested{\@eatdo{\@tmext@unfold@nested}}{%
  \@ifnextchar\@tmext@parens{\@eatdo{\@tmext@unfold@parens}}{%
  \@ifnextchar\cprojfstf{\@eatdo{\cprojfstf@unfold}}{%
  \@ifnextchar\cprojfst{\@eatdo{\cprojfst@unfold}}{%
  \@ifnextchar\cprojsndf{\@eatdo{\cprojsndf@unfold}}{%
  \@ifnextchar\cprojsnd{\@eatdo{\cprojsnd@unfold}}{%
  \@ifnexthcar\jfcomp{\@eatdo{\jfcomp@unfold}}{%
%  \@ifnextchar\@jfcomp@nested{\@eatdo{\@jfcomp@unfold@nested}}{%
%  \@ifnextchar\@jfcomp@parens{\@eatdo{\@jfcomp@unfold@parens}}{%
  \@ifnextchar\sandwich{\@eatdo{\sandwich@unfold}}{%
  #1%
  }}}}}}}}}}}}}}}}}%}}
\makeatother

%%%%%%%%%%%%%%%%%%%%%%%%%%%%%%%%%%%%%%%%%%%%%%%%%%%%%%%%%%%%%%%%%%%%%%%%%%%%%%%%
%%%% A PRETTY PRINTER
%%%%
%%%% We write a \pretty command that pretty prints judgments or types by
%%%% diplaying variables and omitting explicit notation for weakening.
%%%%
%%%% This command should work similar to the \unfold command
%%%%
%%%% -- UNDER CONSTRUCTION

\makeatletter
\newcommand{\vardis}[2]{\@vardis@type #2{}(\@vardis@term #1)}
\newcommand{\@vardis}{\@ifnextchar\bgroup{\@@vardis}{}}
\newcommand{\@@vardis}[1]{\@ifnextchar\bgroup{\vardis{#1}}{#1}}
\newcommand{\@vardis@term}{\@vardis}
\newcommand{\@vardis@type}{\@ifnextchar\ctxext{\@ctxext@nested}{\@ifnextchar\ctxwk{\@ctxwk@nested}{\@vardis}}}
\newcommand{\@vardis@nested}[3]{\@vardis@parens{#2}{#3}}
\newcommand{\@vardis@parens}[2]{(\vardis{#1}{#2})}
\makeatother

\makeatletter
\newcommand{\jvctx}{\jctx}
\newcommand{\jvctxeq}{\jctxeq}

\newcommand{\cctxextcombi}[2]{\@ifnextchar\bgroup{\@cctxextcombi #1}{#1:}#2}
\newcommand{\@cctxextcombi}[4]{\cctxext{{\cctxextcombi{#1}{#3}}{\@@cctxextcombi{#1}{#2}{#4}}}}
\newcommand{\@@cctxextcombi}[3]{\@ifnextchar\bgroup{\@@@ctxextcombi #2}{#2(#1):}#3(\cctxext{#1})}
\newcommand{\@@@ctxextcombi}[8] % the 5th argument is (, the 6th is \cctxext and the 8th is ).
  {\@@ctxextcombi{#7}{#1}{#3},\@@ctxextcombi{{#7}{#3}}{#2}{#4}}
\newcommand{\cctxext}[1]{\@ifnextchar\bgroup{\@cctxext}{}#1}
\newcommand{\@cctxext}[2]{\cctxext{#1},\cctxext{#2}}

\newcommand{\jvfamcombi}[3]{
  \cctxextcombi{#1}{#2} \vdash \vardis{\cctxext{#1}}{#3}
}

\newcommand{\jvfam}{\@ifnextchar*{\@jvfamAlignTrue}{\@jvfamAlignFalse}}
\newcommand{\@jvfamAlignTrue}[4]{\jfam*{#2:#3}{\vardis{#2}{#4}}}
\newcommand{\@jvfamAlignFalse}[3]{\jfam{#1:#2}{\vardis{#1}{#3}}\quad@test}

\newcommand{\jvfameq}{\@ifnextchar*{\@jvfameqAlignTrue}{\@jvfameqAlignFalse}}
\newcommand{\@jvfameqAlignTrue}[5]{\jfameq*{#2:#3}{\vardis{#2}{#4}}{\vardis{#2}{#5}}}
\newcommand{\@jvfameqAlignFalse}[4]{\jfameq{#1:#2}{\vardis{#1}{#3}}{\vardis{#1}{#4}}\quad@test}

\newcommand{\jvtype}{\@ifnextchar*{\@jvtypeAlignTrue}{\@jvtypeAlignFalse}}
\newcommand{\@jvtypeAlignTrue}[4]{\jtype*{#2:#3}{\vardis{#2}{#4}}}
\newcommand{\@jvtypeAlignFalse}[3]{\jtype{#1:#2}{\vardis{#1}{#3}}\quad@test}

\newcommand{\jvtypeeq}{\@ifnextchar*{\@jvtypeeqAlignTrue}{\@jvtypeeqAlignFalse}}
\newcommand{\@jvtypeeqAlignTrue}[5]{\jtypeeq*{#2:#3}{\vardis{#2}{#4}}{\vardis{#2}{#5}}}
\newcommand{\@jvtypeeqAlignFalse}[4]{\jtypeeq{#1:#2}{\vardis{#1}{#3}}{\vardis{#1}{#4}}\quad@test}

\newcommand{\jvterm}{\@ifnextchar*{\@jvtermAlignTrue}{\@jvtermAlignFalse}}
\newcommand{\@jvtermAlignTrue}[5]{\jterm*{#2:#3}{\vardis{#2}{#4}}{\vardis{#2}{#5}}}
\newcommand{\@jvtermAlignFalse}[4]{\jterm{#1:#2}{\vardis{#1}{#3}}{\vardis{#1}{#4}}\quad@test}

\newcommand{\jvtermeq}{\@ifnextchar*{\@jvtermeqAlignTrue}{\@jvtermeqAlignFalse}}
\newcommand{\@jvtermeqAlignTrue}[6]{\jtermeq*{#2:#3}{\vardis{#2}{#4}}{\vardis{#2}{#5}}{\vardis{#2}{#6}}}
\newcommand{\@jvtermeqAlignFalse}[5]{\jtermeq{#1:#2}{\vardis{#1}{#3}}{\vardis{#1}{#4}}{\vardis{#1}{#5}}\quad@test}
\makeatother

%%%%%%%%%%%%%%%%%%%%%%%%%%%%%%%%%%%%%%%%%%%%%%%%%%%%%%%%%%%%%%%%%%%%%%%%%%%%%%%%
%%%% JUDGMENTS
%%%%
%%%% Below we define several commands for the judgments of type theory. There
%%%% are commands
%%%% * \jctx for the judgment that something is a context.
%%%% * \jctxeq for the judgment that two contexts are the same
%%%% * \jtype for the judgment that something is a type in a context
%%%% * \jtypeeq for the judgment that two types in the same context are the same
%%%% * \jterm for the judgment that something is a term of a type in a context
%%%% * \jtermeq for the judgment that two terms of the same type are the same

\makeatletter
% We first make a generic judgment command
\newcommand{\judgment}{\@ifnextchar*{\@judgmentAT}{\@judgmentAF}}
\newcommand{\@judgmentAT}[8]{\@judgment@ctx{#2} & \vdash \@judgment@rel{#3}{#4}{#5}{#6}{#7} #8}
\newcommand{\@judgmentAF}[7]{\@judgment@ctx{#1} \vdash \@judgment@rel{#2}{#3}{#4}{#5}{#6} #7\quad@test}
\newcommand{\@judgment@ctx}{\@judgment@ext}
\newcommand{\@judgment@rel}[5]{\@judgment@ext{#1} #2 \@judgment@ext{#3} #4 \@judgment@ext{#5}}
\newcommand{\@judgment@kind}[1]{~~\textit{#1}}
\newcommand{\@judgment@ext}[1]{\default@ctxext#1}

\newcommand{\quad@test}{%
  \@ifnextchar\jctx{\quad}{%
  \@ifnextchar\jctxeq{\quad}{%
  \@ifnextchar\jvctx{\quad}{%
  \@ifnextchar\jvctxeq{\quad}{%
  \@ifnextchar\jfam{\quad}{%
  \@ifnextchar\jfameq{\quad}{%
  \@ifnextchar\jvfam{\quad}{%
  \@ifnextchar\jvfameq{\quad}{%
  \@ifnextchar\jtype{\quad}{%
  \@ifnextchar\jtypeeq{\quad}{%
  \@ifnextchar\jvtype{\quad}{%
  \@ifnextchar\jvtypeeq{\quad}{%
  \@ifnextchar\jterm{\quad}{%
  \@ifnextchar\jtermeq{\quad}{%
  \@ifnextchar\jvterm{\quad}{%
  \@ifnextchar\jvtermeq{\quad}{%
  \@ifnextchar\jhom{\quad}{%
  \@ifnextchar\jhomeq{\quad}{%
  \@ifnextchar\jfhom{\quad}{%
  \@ifnextchar\jfhomeq{\quad}{%
  }}}}}}}}}}}}}}}}}}}}}

%%%% Judgments about contexts
\newcommand{\jctx@sym}{\@judgment@kind{ctx}}

\newcommand{\jctx}{\@ifnextchar*{\@jctxAlignTrue}{\@jctxAlignFalse}}
\newcommand{\@jctxAlignTrue}[2]{\judgment*{}{}{}{}{}{#2}{\jctx@sym}}
\newcommand{\@jctxAlignFalse}[1]{\judgment{}{}{}{}{}{#1}{\jctx@sym}}

\newcommand{\jctxeq}{\@ifnextchar*{\@jctxeqAlignTrue}{\@jctxeqAlignFalse}}
\newcommand{\@jctxeqAlignTrue}[3]{\judgment*{}{#2}{\jdeq}{#3}{}{}{\jctx@sym}}
\newcommand{\@jctxeqAlignFalse}[2]{\judgment{}{#1}{\jdeq}{#2}{}{}{\jctx@sym}}

\newcommand{\jctxdefn}{\@ifnextchar*{\@jctxdefnAlignTrue}{\@jctxdefnAlignFalse}}
\newcommand{\@jctxdefnAlignTrue}[3]{\judgment*{}{#2}{\defeq}{#3}{}{}{\jctx@sym}}
\newcommand{\@jctxdefnAlignFalse}[2]{\judgment{}{#1}{\defeq}{#2}{}{}{\jctx@sym}}

%%%% Judgments about families
\newcommand{\jfam@sym}{\@judgment@kind{fam}}

\newcommand{\jfam}{\@ifnextchar*{\@jfamAlignTrue}{\@jfamAlignFalse}}
\newcommand{\@jfamAlignTrue}[3]{\judgment*{#2}{}{}{}{}{#3}{\jfam@sym}}
\newcommand{\@jfamAlignFalse}[2]{\judgment{#1}{}{}{}{}{#2}{\jfam@sym}}

\newcommand{\jfameq}{\@ifnextchar*{\@jfameqAlignTrue}{\@jfameqAlignFalse}}
\newcommand{\@jfameqAlignTrue}[4]{\judgment*{#2}{#3}{\jdeq}{#4}{}{}{\jfam@sym}}
\newcommand{\@jfameqAlignFalse}[3]{\judgment{#1}{#2}{\jdeq}{#3}{}{}{\jfam@sym}}

\newcommand{\jfamdefn}{\@ifnextchar*{\@jfamdefnAlignTrue}{\@jfamdefnAlignFalse}}
\newcommand{\@jfamdefnAlignTrue}[4]{\judgment*{#2}{#3}{\defeq}{#4}{}{}{\jfam@sym}}
\newcommand{\@jfamdefnAlignFalse}[3]{\judgment{#1}{#2}{\defeq}{#3}{}{}{\jfam@sym}}
  
%%%% Judgments about types
\newcommand{\jtype@sym}{\@judgment@kind{type}}
\newcommand{\jtype}{\@ifnextchar*{\@jtypeAlignTrue}{\@jtypeAlignFalse}}
\newcommand{\@jtypeAlignTrue}[3]{\judgment*{#2}{}{}{}{}{#3}{\jtype@sym}}
\newcommand{\@jtypeAlignFalse}[2]{\judgment{#1}{}{}{}{}{#2}{\jtype@sym}}
  
\newcommand{\jtypeeq}{\@ifnextchar*{\@jtypeeqAlignTrue}{\@jtypeeqAlignFalse}}
\newcommand{\@jtypeeqAlignTrue}[4]{\judgment*{#2}{#3}{\jdeq}{#4}{}{}{\jtype@sym}}
\newcommand{\@jtypeeqAlignFalse}[3]{\judgment{#1}{#2}{\jdeq}{#3}{}{}{\jtype@sym}}

\newcommand{\jtypedefn}{\@ifnextchar*{\@jtypedefnAlignTrue}{\@jtypedefnAlignFalse}}
\newcommand{\@jtypedefnAlignTrue}[4]{\judgment*{#2}{#3}{\defeq}{#4}{}{}{\jtype@sym}}
\newcommand{\@jtypedefnAlignFalse}[3]{\judgment{#1}{#2}{\defeq}{#3}{}{}{\jtype@sym}}
  
%%%% Judgments about terms
\newcommand{\jterm}{\@ifnextchar*{\@jtermAlignTrue}{\@jtermAlignFalse}}
\newcommand{\@jtermAlignTrue}[4]{\judgment*{#2}{}{}{#4}{:}{#3}{}}
\newcommand{\@jtermAlignFalse}[3]{\judgment{#1}{}{}{#3}{:}{#2}{}}

\newcommand{\jtermeq}{\@ifnextchar*{\@jtermeqAlignTrue}{\@jtermeqAlignFalse}}
\newcommand{\@jtermeqAlignTrue}[5]{\judgment*{#2}{#4}{\jdeq}{#5}{:}{#3}{}}
\newcommand{\@jtermeqAlignFalse}[4]{\judgment{#1}{#3}{\jdeq}{#4}{:}{#2}{}}

\newcommand{\jtermdefn}{\@ifnextchar*{\@jtermdefnAlignTrue}{\@jtermdefnAlignFalse}}
\newcommand{\@jtermdefnAlignTrue}[5]{\judgment*{#2}{#4}{\defeq}{#5}{:}{#3}{}}
\newcommand{\@jtermdefnAlignFalse}[4]{\judgment{#1}{#3}{\defeq}{#4}{:}{#2}{}}
\makeatother

%%%%%%%%%%%%%%%%%%%%%%%%%%%%%%%%%%%%%%%%%%%%%%%%%%%%%%%%%%%%%%%%%%%%%%%%%%%%%%%%
%%%% THE EMPTY CONTEXT

\newcommand{\emptysym}{[\;]}
\newcommand{\emptyc}{{\emptysym}}
\newcommand{\emptyf}[1][]{{\emptysym}_{#1}}
\newcommand{\emptytm}[1][]{\typefont{\#}_{#1}}

%%%%%%%%%%%%%%%%%%%%%%%%%%%%%%%%%%%%%%%%%%%%%%%%%%%%%%%%%%%%%%%%%%%%%%%%%%%%%%%%
%%%% CONTEXT EXTENSION 
%%%%
%%%% The context extension command.
%%%%
%%%% To get a feeling of how the command works, here are a few examples.
%%%% \ctxext{A}{B} will print A.B
%%%% \ctxext{{A}{B}}{C} will print (A.B).C
%%%% \ctxext{{{A}{B}}{C}}{{D}{E}} will print ((A.B).C).(D.E)

\makeatletter
\newcommand{\ctxext}[2]{\@ctxext@ctx #1.\@ctxext@type #2}
\newcommand{\@ctxext}{\@ifnextchar\bgroup{\@@ctxext}{}}
\newcommand{\@ctxext@ctx}{%
  \@ifnextchar\ctxext{\@ctxext@nested}{%
  \@ifnextchar\ctxwk{\@ctxwk@nested}{%
  \@ifnextchar\jcomp{\@jcomp@nested}{%
  \@ctxext}}}}
\newcommand{\@ctxext@type}{%
  \@ifnextchar\ctxext{\@ctxext@nested}{%
  \@ifnextchar\subst{\@subst@nested}{%
  \@ifnextchar\jcomp{\@jcomp@nested}{%
  \@ctxext}}}}
\newcommand{\@@ctxext}[1]{\@ifnextchar\bgroup{\@ctxext@parens{#1}}{#1}}
\newcommand{\@ctxext@parens}[2]{(\ctxext{#1}{#2})}
\newcommand{\@ctxext@nested}[3]{\@ctxext@parens{#2}{#3}}

%%%% We want that some commands accept binary trees as arguments that default
%%%% into extensions. We make the following command to realize this
\newcommand{\default@ctxext}{\@ifnextchar\bgroup{\ctxext}{}}
\newcommand{\default@ctxext@parens}{\@ifnextchar\bgroup{\@ctxext@parens}{}}
\makeatother

%%%%%%%%%%%%%%%%%%%%%%%%%%%%%%%%%%%%%%%%%%%%%%%%%%%%%%%%%%%%%%%%%%%%%%%%%%%%%%%%
%%%% SUBSTITUTION

%%%% The substitution command will act the following way
%%%%
%%%% \subst{x}{P} will print P[x]
%%%% \subst{x}{{f}{Q}} will print Q[f][x]
%%%% \subst{{x}{f}}{{x}{Q}} will print Q[x][f[x]]

\makeatletter
\newcommand{\subst}[2]{%
  \@subst@type #2{}[\@subst@term #1]}
\newcommand{\@subst}{%
  \@ifnextchar\bgroup{\@@subst}{}}
\newcommand{\@@subst}[1]{%
  \@ifnextchar\bgroup{\subst{#1}}{#1}}
\newcommand{\@subst@term}{%
  \@subst}
\newcommand{\@subst@type}{%
  \@ifnextchar\ctxext{\@ctxext@nested}{%
  \@ifnextchar\ctxwk{\@ctxwk@nested}{%
  \@ifnextchar\jcomp{\@jcomp@nested}{%
  \@ifnextchar\tmext{\@tmext@nested}{%
%  \@ifnextchar\mfam{\@mfam@nested}{%
%  \@ifnextchar\mtm{\@mtm@nested}}
\newcommand{\@subst@nested}[3]{%
  \@subst@parens{#2}{#3}}
\newcommand{\@subst@parens}[2]{%
  (\subst{#1}{#2})}
\makeatother

%%%%%%%%%%%%%%%%%%%%%%%%%%%%%%%%%%%%%%%%%%%%%%%%%%%%%%%%%%%%%%%%%%%%%%%%%%%%%%%%
%%%% WEAKENING

%%%% The weakening command is very much like the substitution command.

\makeatletter
\newcommand{\ctxwk}[2]{%
  \langle\@ctxwk@act #1\rangle\@ctxwk@pass #2}
\newcommand{\@ctxwk}{%
  \@ifnextchar\bgroup{\@@ctxwk}{}}
\newcommand{\@@ctxwk}[1]{%
  \@ifnextchar\bgroup{\ctxwk{#1}}{#1}}
\newcommand{\@ctxwk@act}{%
  \@ctxwk}
\newcommand{\@ctxwk@pass}{%
  \@ifnextchar\ctxext{\@ctxext@nested}{%
  \@ifnextchar\subst{\@subst@nested}{%
  \@ifnextchar\jcomp{\@jcomp@nested}{%
  \@ifnextchar\tmext{\@tmext@nested}{%
%  \@ifnextchar\mfam{\@mfam@nested}{%
%  \@ifnextchar\mtm{\@mtm@nested}}
\newcommand{\@ctxwk@parens}[2]{%
  (\ctxwk{#1}{#2})}
\newcommand{\@ctxwk@nested}[3]{%
  \@ctxwk@parens{#2}{#3}}
\makeatother

%%%% Not sure if we're gonna need the following.
\newcommand{\ctxwkop}[2]{%
  \ctxwk{#2}{#1}}
  
%%%%%%%%%%%%%%%%%%%%%%%%%%%%%%%%%%%%%%%%%%%%%%%%%%%%%%%%%%%%%%%%%%%%%%%%%%%%%%%%
%%%% IDENTITY TERMS

\makeatletter
\newcommand{\idtm}[1]{\typefont{id}_{#1}}
\makeatother

%%%%%%%%%%%%%%%%%%%%%%%%%%%%%%%%%%%%%%%%%%%%%%%%%%%%%%%%%%%%%%%%%%%%%%%%%%%%%%%%
%%%% TERM EXTENSION

\makeatletter

%%%% The basic term extension commands
\newcommand{\default@tmext}{\@ifnextchar\bgroup{\tmext}{}}
\newcommand{\tmext}[4]{%
  \ctxext{\tmext@testleft #3}{\tmext@testright #4}}
\newcommand{\tmext@testleft}{%
  \@ifnextchar\bgroup{\@tmext@parens}{%
  \@ifnextchar\tmext{\@tmext@nested}{%
  \@ifnextchar\ctxwk{\@ctxwk@nested}{%
  \@ifnextchar\jcomp{\@jcomp@nested}{%
  \default@tmext
  }}}}}
\newcommand{\tmext@testright}{%
  \@ifnextchar\bgroup{\@tmext@parens}{%
  \@ifnextchar\tmext{\@tmext@nested}{%
  \@ifnextchar\subst{\@subst@nested}{%
  \@ifnextchar\jcomp{\@jcomp@nested}{%
  \default@tmext
  }}}}}
\newcommand{\@tmext@nested}[5]{%
  \@tmext@parens{#2}{#3}{#4}{#5}}
\newcommand{\@tmext@parens}[4]{%
  (\tmext{#1}{#2}{#3}{#4})}

%%%% The unfolded term extension commands
\newcommand{\tmext@unfold}[4]{%
  \subst{#4}{{#3}{\idtm{\ctxext{#1}{#2}}}}}
\newcommand{\@tmext@unfold@nested}[1]{%
  \@tmext@unfold@parens}
\newcommand{\@tmext@unfold@parens}[4]{%
  (\tmext@unfold{#1}{#2}{#3}{#4})}
\makeatother

%%%%%%%%%%%%%%%%%%%%%%%%%%%%%%%%%%%%%%%%%%%%%%%%%%%%%%%%%%%%%%%%%%%%%%%%%%%%%%%%
%%%% JUDGMENTAL MORPHISMS

\makeatletter

%%%% The judgment that f is a morphism from A to B in context \Gamma.
\newcommand{\jhomsym}[3][]{%
  ~~\textit{hom}_{#1}(\default@ctxext #2,\default@ctxext #3)}
\newcommand{\jhom}{%
  \@ifnextchar*{\@jhomAlignTrue}{\@jhomAlignFalse}}
\newcommand{\@jhomAlignTrue}[5]{%
  \judgment*{#2}{}{}{#5}{}{}{\jhomsym{#3}{#4}}}
\newcommand{\@jhomAlignFalse}[4]{%
  \judgment{#1}{}{}{#4}{}{}{\jhomsym{#2}{#3}}}
\newcommand{\jhomeq}{%
  \@ifnextchar*{\@jhomeqAlignTrue}{\@jhomeqAlignFalse}}
\newcommand{\@jhomeqAlignTrue}[6]{%
  \judgment*{#2}{#5}{\jdeq}{#6}{}{}{\jhomsym{#3}{#4}}}
\newcommand{\@jhomeqAlignFalse}[5]{%
  \judgment{#1}{#4}{\jdeq}{#5}{}{}{\jhomsym{#2}{#3}}}
\newcommand{\jhomdefn}{%
  \@ifnextchar*{\@jhomdefnAlignTrue}{\@jhomdefnAlignFalse}}
\newcommand{\@jhomdefnAlignTrue}[6]{%
  \judgment*{#2}{#5}{\defeq}{#6}{}{}{\jhomsym{#3}{#4}}}
\newcommand{\@jhomdefnAlignFalse}[5]{%
  \judgment{#1}{#4}{\defeq}{#5}{}{}{\jhomsym{#2}{#3}}}

\newcommand{\jhom@unfold}[4]{%
  \jterm
    {{#1}{#2}}
    {\ctxwk{\default@ctxext #2}{\default@ctxext@parens #3}}
    {#4}}
\newcommand{\jhomeq@unfold}[5]{%
  \jtermeq
    {{#1}{#2}}
    {\ctxwk{\default@ctxext #2}{\default@ctxext@parens #3}}
    {#4}
    {#5}}
\newcommand{\jhomdefn@unfold}[5]{%
  \jtermdefn
    {{#1}{#2}}
    {\ctxwk{\default@ctxext #2}{\default@ctxext@parens #3}}
    {#4}
    {#5}}

%%%% Composition of morphisms
\newcommand{\jcomp}[3]{%
  \jcomp@testleft #3 \circ \jcomp@testright #2}
\newcommand{\jcomp@testleft}{%
  \@ifnextchar\jcomp{\@jcomp@nested}{%
  \@ifnextchar\ctxwk{\@ctxwk@nested}{%
  \@ifnextchar\ctxext{\@ctxext@nested}{%
  \@ifnextchar\bgroup{\@jcomp@parens}{%
  }}}}}
\newcommand{\jcomp@testright}{%
  \@ifnextchar\jcomp{\@jcomp@nested}{%
  \@ifnextchar\subst{\@subst@nested}{%
  \@ifnextchar\ctxext{\@ctxext@nested}{%
  \@ifnextchar\bgroup{\@jcomp@parens}{%
  }}}}}
\newcommand{\@jcomp@nested}[4]{%
  \@jcomp@parens{#2}{#3}{#4}}
\newcommand{\@jcomp@parens}[3]{%
  (\jcomp{#1}{#2}{#3})}

\newcommand{\jcomp@unfold}[3]{%
  \subst
    {\jcomp@unfold@test@preside #2}
    {\ctxwk{\default@ctxext #1}{\jcomp@unfold@test@postside #3}}}
\newcommand{\jcomp@unfold@test@preside}{%
  \@ifnextchar\bgroup{\@jcomp@unfold@parens}{}}
\newcommand{\jcomp@unfold@test@postside}{%
  \@ifnextchar\bgroup{\@jcomp@unfold@parens}{}}
\newcommand{\@jcomp@unfold@nested}[4]{%
  \@jcomp@unfold@parens{#2}{#3}{#4}}
\newcommand{\@jcomp@unfold@parens}[3]{%
  (\jcomp@unfold{#1}{#2}{#3})}

%%%% The judgment that F is a morphism from P to Q over f in context \Gamma.
\newcommand{\jfhomsym}[3]{\jhomsym[{#1}]{#2}{#3}}
\newcommand{\jfhom}{%
  \@ifnextchar*{\jfhomAlignTrue}{\jfhomAlignFalse}}
\newcommand{\jfhomAlignTrue}[8]{
  \judgment*{#2}{}{}{#8}{}{}{\jfhomsym{#5}{#6}{#7}}}
\newcommand{\jfhomAlignFalse}[7]{
  \judgment{#1}{}{}{#7}{}{}{\jfhomsym{#4}{#5}{#6}}}
\newcommand{\jfhomeq}[8]{%
  \judgment{#1}{#7}{\jdeq}{#8}{}{}{\jhomsym[{#4}]{#5}{#6}}}
\newcommand{\jfhomdefn}[8]{%
  \judgment{#1}{#7}{\defeq}{#8}{}{}{\jhomsym[{#4}]{#5}{#6}}}
\newcommand{\jfhom@unfold}[7]{%
  \jterm
    {{{#1}{#2}}{#5}}
    {\ctxwk{\default@ctxext #5}{\jcomp{#2}{#4}{#6}}}
    {#7}}
    
\newcommand{\jfcomp}[5]{%
  \jfcomp@testleft #5 \circ \jfcomp@testright #4}
\newcommand{\jfcomp@testleft}{%
  \@ifnextchar\bgroup{\@jfcomp@parens}{%
  \@ifnextchar\jfcomp{\@jfcomp@nested}{%
  \@ifnextchar\jcomp{\@jcomp@nested}{%
  \@ifnextchar\ctxwk{\@ctxwk@nested}{%
  \@ifnextchar\tmext{\@tmext@nested}{%
  }}}}}}
\newcommand{\jfcomp@testright}{%
  \@ifnextchar\bgroup{\@jfcomp@parens}{%
  \@ifnextchar\jfcomp{\@jfcomp@nested}{%
  \@ifnextchar\jcomp{\@jcomp@nested}{%
  \@ifnextchar\subst{\@subst@nested}{%
  \@ifnextchar\tmext{\@tmext@nested}{%
  }}}}}}
\newcommand{\@jfcomp@nested}[1]{%
  \@jfcomp@parens}
\newcommand{\@jfcomp@parens}[5]{%
  (\jfcomp{#1}{#2}{#3}{#4}{#5})}
  
\newcommand{\jfcomp@unfold}[5]{%
  \jcomp{#3}{#4}{{#1}{#2}{#5}}}
\makeatother

%%%%%%%%%%%%%%%%%%%%%%%%%%%%%%%%%%%%%%%%%%%%%%%%%%%%%%%%%%%%%%%%%%%%%%%%%%%%%%%%
%%%% JUDGMENTAL TRIVIAL COFIBRATIONS

\newcommand{\jtcext}{\tilde}

%%%%%%%%%%%%%%%%%%%%%%%%%%%%%%%%%%%%%%%%%%%%%%%%%%%%%%%%%%%%%%%%%%%%%%%%%%%%%%%%
%%%% CONTEXT PROJECTIONS

\makeatletter
\newcommand{\cprojgenf}[3]{%
  \typefont{pr}^{
    \@ifnextchar\ctxext{\@ctxext@nested}{}#2,
    \@ifnextchar\ctxext{\@ctxext@nested}{}#3}_{#1}}
\newcommand{\cprojgen}[4]{%
  \subst{#4}{\cprojgenf{#1}{#2}{#3}}}

\newcommand{\cprojfstf}[2]{%
  \cprojgenf{0}{#1}{#2}}
\newcommand{\cprojfst}[3]{%
  \cprojgen{0}{#1}{#2}{#3}}
\newcommand{\cprojsndf}[2]{%
  \cprojgenf{1}{#1}{#2}}
\newcommand{\cprojsnd}[3]{%
  \cprojgen{1}{#1}{#2}{#3}}

\newcommand{\cprojfstf@unfold}[2]{%
  \ctxwk{\default@ctxext #2}\idtm{\default@ctxext #1}}
\newcommand{\cprojfst@unfold}[3]{%
  \subst{#3}{(\cprojfstf@unfold{#1}{#2})}}
\newcommand{\cprojsndf@unfold}[2]{%
  \idtm{\default@ctxext #2}}
\newcommand{\cprojsnd@unfold}[3]{%
  \subst{#3}{\cprojsnd@unfold{#1}{#2}}}
  
%%%% The sandwich function
\newcommand{\sandwich}[3]{\typefont{sw}^{#1,#2,#3}}
\newcommand{\sandwich@unfold}[3]{\typefont{sw}^{#1,#2,#3}}
\makeatother

%%%%%%%%%%%%%%%%%%%%%%%%%%%%%%%%%%%%%%%%%%%%%%%%%%%%%%%%%%%%%%%%%%%%%%%%%%%%%%%%
%%%% THE UNIT TYPE

\makeatletter
\newcommand{\unitc}[1]{%
  \unit^0_{\default@ctxext #1}}
\newcommand{\unitct}[1]{%
  \ttt^0_{\default@ctxext #1}}
\newcommand{\unitf}[2]{%
  \unit^1_{\default@ctxext #1,\default@ctxext #2}}
\newcommand{\unitft}[2]{%
  \ttt^1_{\default@ctxext #1,\default@ctxext #2}}
\makeatother

%%%%%%%%%%%%%%%%%%%%%%%%%%%%%%%%%%%%%%%%%%%%%%%%%%%%%%%%%%%%%%%%%%%%%%%%%%%%%%%%
%%%% DEPENDENT FUNCTION TYPES

\makeatletter
\newcommand{\sprd}[2]{\Pi(\default@ctxext #1,\default@ctxext #2)}
\begin{comment}
\newcommand{\@sprd@test@cod}[2]{%
  \@ifnextchar\bgroup{\@sprd@do@cod{#1}}{%
  \Pi(\@sprd@test@dom{#1}{#2} #1,
  }}
\newcommand{\@sprd@do@cod}[4]{%
  \ctxext{\@sprd{#1}{#2}}{\@sprd{#1}{#3}}
  }
\newcommand{\@sprd}[2]{
  \@ifnextchar\bgroup{\@@sprd}{%
    \Pi(}
    #1,{#2})
  }
\newcommand{\@@sprd}[5]{%
  \sprd{#1}{\sprd{#2}{#4}}
  }
\end{comment}

\newcommand{\slam}[3]{%
  \lambda^{{\default@ctxext@parens #1},{\default@ctxext@parens #2}}
  (\default@ctxext #3)
  }
\newcommand{\sev}[1]{\tfev(#1)}

\makeatother

%%%%%%%%%%%%%%%%%%%%%%%%%%%%%%%%%%%%%%%%%%%%%%%%%%%%%%%%%%%%%%%%%%%%%%%%%%%%%%%%
%%%% NON-DEPENDENT FUNCTION TYPES

\newcommand{\jfun}[2]{#1\to#2}

%%%%%%%%%%%%%%%%%%%%%%%%%%%%%%%%%%%%%%%%%%%%%%%%%%%%%%%%%%%%%%%%%%%%%%%%%%%%%%%%
%%%% THE CONSTRUCTORS OF THE TYPE THEORY OF MODELS

\makeatletter
%%%% The initial model
\newcommand{\mctx}{%
  \mathcal{C}}

%%%% The family constructor
\newcommand{\mfam}[2][]{%
  \mathcal{F}_{\default@ctxext #2}^{#1}}
\newcommand{\@mfam@nested}[1]{\@mfam@parens}
\newcommand{\@mfam@parens}[2][]{(\mfam[#1]{#2})}

%%%% The terms constructor
\newcommand{\mtm}[2][]{%
  \mathcal{T}_{\default@ctxext #2}^{#1}}
\newcommand{\@mtm@nested}[1]{\@mtm@parens}
\newcommand{\@mtm@parens}[2][]{(\mtm[#1]{#2})}

%%%% The empty type constructor
\newcommand{\tfemp}[1]{%
  \typefont{emp}_{\default@ctxext #1}}
\newcommand{\tft}[1]{%
  \typefont{t}_{\default@ctxext #1}}

%%%% The extension constructor
\newcommand{\tfext}[1]{%
  \typefont{ext}_{\default@ctxext #1}}

%%%% The substitution constructor
\newcommand{\tfsubst}[1]{%
  \typefont{subst}_{\default@ctxext #1}}
  
%%%% The weakening constructor
\newcommand{\tfwk}[1]{%
  \typefont{wk}_{\default@ctxext #1}}

%%%% The identity function constructor
\newcommand{\tfid}[1]{%
  \typefont{idtm}_{\default@ctxext #1}}
\makeatother

%%%%%%%%%%%%%%%%%%%%%%%%%%%%%%%%%%%%%%%%%%%%%%%%%%%%%%%%%%%%%%%%%%%%%%%%%%%%%%%%

%%%% Introducing logical usage of fonts.
\newcommand{\modelfont}{\mathit} % use 'mf' in command to indicate model font
\newcommand{\typefont}{\mathsf} % use 'tf' in command to indicate type font
\newcommand{\catfont}{\mathrm} % use 'cf' in command to indicate cat font

%%%%%%%%%%%%%%%%%%%%%%%%%%%%%%%%%%%%%%%%%%%%%%%%%%%%%%%%%%%%%%%%%%%%%%%%%%%%%%%%
%%%% Some macros of the book are redefined.

\renewcommand{\UU}{\typefont{U}}
\renewcommand{\isequiv}{\typefont{isEquiv}}
\renewcommand{\happly}{\typefont{hApply}}
\renewcommand{\pairr}[1]{{\mathopen{}\langle #1\rangle\mathclose{}}}
\renewcommand{\type}{\typefont{Type}}
\renewcommand{\op}[1]{{{#1}^\typefont{op}}}
\renewcommand{\susp}{\typefont{\Sigma}}

%%%%%%%%%%%%%%%%%%%%%%%%%%%%%%%%%%%%%%%%%%%%%%%%%%%%%%%%%%%%%%%%%%%%%%%%%%%%%%%%
%%%% The following is a big unorganized list of new macros that we use in the
%%%% notes. 

\newcommand{\tfW}{\typefont{W}}
\newcommand{\tfM}{\typefont{M}}
\newcommand{\mfM}{\modelfont{M}}
\newcommand{\mfN}{\modelfont{N}}
\newcommand{\tfctx}{\typefont{ctx}}
\newcommand{\mftypfunc}[1]{{\modelfont{typ}^{#1}}}
\newcommand{\mftyp}[2]{{\mftypfunc{#1}(#2)}}
\newcommand{\tftypfunc}[1]{{\typefont{typ}^{#1}}}
\newcommand{\tftyp}[2]{{\tftypfunc{#1}(#2)}}
\newcommand{\hfibfunc}[1]{\typefont{fib}_{#1}}
\newcommand{\mappingcone}[1]{\mathcal{C}_{#1}}
\newcommand{\equifib}{\typefont{equiFib}}
\newcommand{\tfcolim}{\typefont{colim}}
\newcommand{\tflim}{\typefont{lim}}
\newcommand{\tfdiag}{\typefont{diag}}
\newcommand{\tfGraph}{\typefont{Graph}}
\newcommand{\mfGraph}{\modelfont{Graph}}
\newcommand{\unitGraph}{\unit^\mfGraph}
\newcommand{\UUGraph}{\UU^\mfGraph}
\newcommand{\tfrGraph}{\typefont{rGraph}}
\newcommand{\mfrGraph}{\modelfont{rGraph}}
\newcommand{\isfunction}{\typefont{isFunction}}
\newcommand{\tfconst}{\typefont{const}}
\newcommand{\conemap}{\typefont{coneMap}}
\newcommand{\coconemap}{\typefont{coconeMap}}
\newcommand{\tflimits}{\typefont{limits}}
\newcommand{\tfcolimits}{\typefont{colimits}}
\newcommand{\islimiting}{\typefont{isLimiting}}
\newcommand{\iscolimiting}{\typefont{isColimiting}}
\newcommand{\islimit}{\typefont{isLimit}}
\newcommand{\iscolimit}{\typefont{iscolimit}}
\newcommand{\pbcone}{\typefont{cone_{pb}}}
\newcommand{\tfinj}{\typefont{inj}}
\newcommand{\tfsurj}{\typefont{surj}}
\newcommand{\tfepi}{\typefont{epi}}
\newcommand{\tftop}{\typefont{top}}
\newcommand{\sbrck}[1]{\Vert #1\Vert}
\newcommand{\strunc}[2]{\Vert #2\Vert_{#1}}
\newcommand{\gobjclass}{{\typefont{U}^\mfGraph}}
\newcommand{\gcharmap}{\typefont{fib}}
\newcommand{\diagclass}{\typefont{T}}
\newcommand{\opdiagclass}{\op{\diagclass}}
\newcommand{\equifibclass}{\diagclass^{\eqv{}{}}}
\newcommand{\universe}{\typefont{U}}
\newcommand{\catid}[1]{{\catfont{id}_{#1}}}
\newcommand{\isleftfib}{\typefont{isLeftFib}}
\newcommand{\isrightfib}{\typefont{isRightFib}}
\newcommand{\leftLiftings}{\typefont{leftLiftings}}
\newcommand{\rightLiftings}{\typefont{rightLiftings}}
\newcommand{\psh}{\typefont{Psh}}
\newcommand{\rgclass}{\typefont{\Omega^{RG}}}
\newcommand{\terms}[2][]{\lfloor #2 \rfloor^{#1}}
\newcommand{\grconstr}[2]
             {\mathchoice % max size is textstyle size.
             {{\textstyle \int_{#1}}#2}% 
             {\int_{#1}#2}%
             {\int_{#1}#2}%
             {\int_{#1}#2}}
\newcommand{\ctxhom}[3][]{\typefont{hom}_{#1}(#2,#3)}
\newcommand{\graphcharmapfunc}[1]{\gcharmap_{#1}}
\newcommand{\graphcharmap}[2][]{\graphcharmapfunc{#1}(#2)}
\newcommand{\tfexp}[1]{\typefont{exp}_{#1}}
\newcommand{\tffamfunc}{\typefont{fam}}
\newcommand{\tffam}[1]{\tffamfunc(#1)}
\newcommand{\tfev}{\typefont{ev}}
\newcommand{\tfcomp}{\typefont{comp}}
\newcommand{\isDec}[1]{\typefont{isDecidable}(#1)}
\newcommand{\smal}{\mathcal{S}}
\renewcommand{\modal}{{\ensuremath{\ocircle}}}
\newcommand{\eqrel}{\typefont{EqRel}}
\newcommand{\piw}{\ensuremath{\Pi\typefont{W}}} %% to be used in conjunction with -pretopos.
\renewcommand{\sslash}{/\!\!/}
\newcommand{\mprd}[2]{\Pi(#1,#2)}
\newcommand{\msm}[2]{\Sigma(#1,#2)}
\newcommand{\midt}[1]{\idvartype_#1}
\newcommand{\reflf}[1]{\typefont{refl}^{#1}}
\newcommand{\tfJ}{\typefont{J}}
\newcommand{\tftrans}{\typefont{trans}}

\newcommand{\tfT}{\typefont{T}}
\newcommand{\reflsym}{{\mathsf{refl}}}
\newcommand{\strans}[2]{\ensuremath{{#1}_{*}({#2})}}
\newcommand{\eqtype}[1]{\typefont{Eq}_{#1}}
\newcommand{\eqtoid}[1]{\typefont{eqtoid}(#1)}
\newcommand{\greek}{\mathrm}
\newcommand{\product}[2]{{#1}\times{#2}}
\newcommand{\pairp}[1]{(#1)}
\newcommand{\jequalizer}[3]{\{#1|#2\jdeq #3\}}
\newcommand{\jequalizerin}[2]{\iota_{#1,#2}}
\newcommand{\tounit}[1]{{!_{#1}}}
\newcommand{\trwk}{\typefont{trwk}}
\newcommand{\trext}{\typefont{trext}}

%%%%%%%%%%%%%%%%%%%%%%%%%%%%%%%%%%%%%%%%%%%%%%%%%%%%%%%%%%%%%%%%%%%%%%%%%%%%%%%%
%%%% When investigation pointed structures we use the \pt macro.

\makeatletter
\newcommand{\pt}[1][]{*_{
  \@ifnextchar\undergraph{\@undergraph@nested}
    {\@ifnextchar\underovergraph{\@underovergraph@nested}{}}#1}}
\makeatother

%%%%%%%%%%%%%%%%%%%%%%%%%%%%%%%%%%%%%%%%%%%%%%%%%%%%%%%%%%%%%%%%%%%%%%%%%%%%%%%%
%%%% OPERATIONS ON GRAPHS
%%%%
%%%% First of all, each graph has a type of vertices and a type of edges. The
%%%% type of vertices of a graph $\Gamma$ is denoted by $\pts{\Gamma}$;
%%%% and likewise for the type of edges.

\makeatletter
\newcommand{\pts}[1]{{\@graphop@nested{#1}}_{0}}
\newcommand{\edg}[1]{{\@graphop@nested{#1}}_{1}}
\newcommand{\@graphop@nested}[1]
  {\@ifnextchar\ctxext{\@ctxext@nested}
      {\@ifnextchar\undergraph{\@undergraph@nested}
         {\@ifnextchar\underovergraph{\@underovergraph@nested}{}}}
    #1}
\makeatother

%%%% The following operations of \undergraph and \underovergraph are used to
%%%% define the free category and the free groupoid of a graph, respectively

\makeatletter
\newcommand{\@undergraphtest}[2]{\@ifnextchar({#1}{#2}}
\newcommand{\undergraph}[2]{\@undergraphtest{\@undergraph@parens{#1}{#2}}{\@undergraph{#1}{#2}}}
\newcommand{\@undergraph}[2]{{#2/#1}}
\newcommand{\@undergraph@nested}[3]{\@undergraph@parens{#2}{#3}}
\newcommand{\@undergraph@parens}[2]{(\@undergraph{#1}{#2})}
\makeatother

\makeatletter
\newcommand{\underovergraph}[2]{\@underovergraphtest{\@underovergraph@parens{#1}{#2}}{\@underovergraph{#1}{#2}}}
\newcommand{\@underovergraph}[2]{{#2}\,{\parallel}\,{#1}}
\newcommand{\@underovergraphtest}{\@undergraphtest}
\newcommand{\@underovergraph@parens}[2]{(\@underovergraph{#1}{#2})}
\newcommand{\@underovergraph@nested}[3]{\@underovergraph@parens{#2}{#3}}
\makeatother

\newcommand{\graphid}[1]{\mathrm{id}_{#1}}
\newcommand{\freecat}[1]{\mathcal{C}(#1)}
\newcommand{\freegrpd}[1]{\mathcal{G}(#1)}


%%%%%%%%%%%%%%%%%%%%%%%%%%%%%%%%%%%%%%%%%%%%%%%%%%%%%%%%%%%%%%%%%%%%%%%%%%%%%%%%
%% Some tikz macros to typeset diagrams uniformly.

\tikzset{patharrow/.style={double,double equal sign distance,-,font=\scriptsize}}
\tikzset{description/.style={fill=white,inner sep=2pt}}
\tikzset{fib/.style={->>,font=\scriptsize}}

%% Used for extra wide diagrams, e.g. when the label is too large otherwise.
\tikzset{commutative diagrams/column sep/Huge/.initial=18ex}

%%%%%%%%%%%%%%%%%%%%%%%%%%%%%%%%%%%%%%%%%%%%%%%%%%%%%%%%%%%%%%%%%%%%%%%%%%%%%%%%
%%%% New theorem environment for conjectures.

\defthm{conj}{Conjecture}{Conjectures}

%%%%%%%%%%%%%%%%%%%%%%%%%%%%%%%%%%%%%%%%%%%%%%%%%%%%%%%%%%%%%%%%%%%%%%%%%%%%%%%%
%%%% The following environment for desiderata should not be there. It is better
%%%% to use the issue tracker for desiderata.

\newenvironment{desiderata}{\begingroup\color{blue}\textbf{Desiderata.}}
{\endgroup}

%%%%%%%%%%%%%%%%%%%%%%%%%%%%%%%%%%%%%%%%%%%%%%%%%%%%%%%%%%%%%%%%%%%%%%%%%%%%%%%%
%%%% The following piece of code from tex.stackexchange:
%%%%
%%%% http://tex.stackexchange.com/a/55180/14653
%%%%
%%%% We include it so that inference rules in align environments have enough
%%%% vertical space.

\newlength\minalignvsep

\makeatletter
\def\align@preamble{%
   &\hfil
    \setboxz@h{\@lign$\m@th\displaystyle{##}$}%
    \ifnum\row@>\@ne
    \ifdim\ht\z@>\ht\strutbox@
    \dimen@\ht\z@
    \advance\dimen@\minalignvsep
    \ht\strutbox\dimen@
    \fi\fi
    \strut@
    \ifmeasuring@\savefieldlength@\fi
    \set@field
    \tabskip\z@skip
   &\setboxz@h{\@lign$\m@th\displaystyle{{}##}$}%
    \ifnum\row@>\@ne
    \ifdim\ht\z@>\ht\strutbox@
    \dimen@\ht\z@
    \advance\dimen@\minalignvsep
    \ht\strutbox@\dimen@
    \fi\fi
    \strut@
    \ifmeasuring@\savefieldlength@\fi
    \set@field
    \hfil
    \tabskip\alignsep@
}
\makeatother

\minalignvsep.2em

\allowdisplaybreaks

%%%%%%%%%%%%%%%%%%%%%%%%%%%%%%%%%%%%%%%%%%%%%%%%%%%%%%%%%%%%%%%%%%%%%%%%%%%%%%%%

\setdescription[1]{itemsep=-0.2em}
