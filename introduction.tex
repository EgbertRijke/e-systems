\section{Introduction}

\begin{comment}
Mathematical research is aided by the development of languages that help 
researchers to investigate their subject efficiently. Such a development is an
ongoing process, both for informal and formal languages, that may have both
mathematical and meta-mathematical aspects. 

There is an amount of attraction and repulsion that works simultaneously
between formal and informal languages for mathematics, which is a driving force
for the development of mathematical languages. The use of
a natural language in mathematical arguments facilitates people to engage
with mathematical constructs at a conceptual level, while a formal mathematical
language aims to encode the intuitive development. At the same time and 
especially nowadays, formal languages aim to possess a high level of rigor and
precision, so that mathematical arguments and constructions can be executed and
verified algorithmically, and informal mathematics is usually not considered
acceptable if it cannot in principle be carried out in a formal system.

It may be the case that formal languages are developed in order to be more
precise and certain about bodies of existing mathematical work, but there are
also cases where the conceptual thinking 

The book on Homotopy Type Theory, \cite{TheBook}, was written in informal
language (maybe `semi-formal' would be a more accurate description), 
partly because it was conceived that the underlying type theory might
not always remain the same, while principles and concepts developed starting
from type theory exist separately from any particular version of type 
theory, though perhaps not completely independent. 

Today, formal languages for mathematics, logic and computer science, are studied
under the name logical frameworks.
Pfenning \cite{PfenningLFintro} defines logical frameworks
to be meta-languages for the formalization of deductive system. Examples of
logical framework therefore include propositional logic, first-order predicate 
logic, Russell's ramified type theory, the lambda-calculus, dependent type 
theory, and so on. By considering models of a logical framework, it becomes
itself the subject of mathematical study. Thus, logical frameworks typically
contain laws for handling logical inferences, the specification of the sorts
and operations of the language or theory, and axioms the operations are required to
satisfy. 

While there are aspects of logical frameworks that are of meta-mathematical 
nature, such as descriptions of handling inference rules, it may be useful to
consider a mathematical language as a mathematical object, especially if one
is interested in models of the theory. For practically any theory there is a
wish to do so, so it may be useful to keep this in mind in the specification
of a logical framework. One way to do this, is to make sure that the theory
belongs to a class of theories of which the model theory is reasonably well-%
understood. Among such classes of theories are algebraic theories and
essentially algebraic theories.

The choices made in the design principles and specification of a logical 
framework, will determine both the subject of study and how effective we can be
in its study. Investigations of logical framework can therefore be seen as
`opening theories for mathematics', in analogy of opening theories for games.
Logical frameworks for mathematics are not set in stone and 
neither to they remain the same once they are declared.

\subsection*{Essentially algebraic theories}
An essentially algebraic theory is an equational theory of partial operations
which may be given the structure of a presheaf over usual ordered set of the
natural numbers, and of which the domain of each operation is declared by 
equations on the preceeding operations. The concept of essentially algebraic
theories is due to Freyd, who writes that a critical feature of essentially
algebraic theories is that their models are closed under 
colimits \cite{Aspects_of_topoi}. Many classes of theories have been described
and shown to be equivalent to essentially algebraic theories. Among them are
Partial Horn Logic and quasi-equational theories \cite{Palmgren_Vickers},
cartesian theories \cite{Freyd2002}, finite limit sketches 
\cite{Toposes_triples_theories}, left exact theories
\cite{McLarty1986}, and they can be equivalently represented as locally
presentable categories \cite{Adamek_Locally}.

In \cite{Adamek_Locally}, Ad\'amek and Rosicky
define an essentially algebraic theory to be a list 
\begin{equation*}
(S,\Sigma,a,d,c,E,\Sigma_t,\mathrm{Def})
\end{equation*} 
consisting of a set $S$ of sorts, a set $\Sigma$ of operation symbols,
an arity function $a:\Sigma\to\mathbf{Set}$ specifying the arity of each operation
symbol, a map $d_\sigma:a(\sigma)\to S$
for each $\sigma\in\Sigma$ specifying the sort of the domain of $\sigma$,
a map $c:\Sigma\to S$ specifying the sort of the codomain of each operation
symbol,
a set $E$ of $\Sigma$-equations, a set $\Sigma_t\subseteq\Sigma$ of total
operation symbols, and a function $\mathrm{Def}$ assigning to each operation
symbol $\sigma:\Pi_{i\in a(\sigma)} s_i\to s$ in $\Sigma-\Sigma_t$ a set
$\mathrm{Def}(\sigma)$ of $\Sigma_t$-equations in the standard variables
$x_i\in V_s$, $i\in a(\sigma)$. Subsequently, they show that a category of models of an
essentially algebraic theory is the same thing as a locally presentable 
category.

Every essentially algebraic theory $\mathbb{T}$ in the sense of Ad\'amek and Rosicky
determines a syntactic category $\mathcal{C}_\mathbb{T}$, which has all finite
limits, and $\mathbb{T}$ can be seen as a presentation of the category
$\mathcal{C}_\mathbb{T}$. Then there is a natural equivalence between the
category of models of $\mathbb{T}$ in a category $\cat{C}$ with finite limits,
and the category of finite limit preserving functors $\mathcal{C}_\mathbb{T}\to
\cat{C}$. Thus, in this sense an essentially algebraic theory \emph{is} just
a category with finite limits. 

Because essentially algebraic theories have a pleasant theory of models, 
phrasing a theory as an essentially algebraic one gives immediate access to
a well-developed toolbox which can be used in its study. Therefore, recognizing
that the theory of type dependency is essentially algebraic is a critical step
in the development of its theory of models.

An essentially algebraic theory can be built step by step in the following way.
First, one fixes the set $S$ of sorts. Next, one fixes a set $\Sigma_0$ of total
operations. 
\end{comment}

\begin{comment}
The theory of context algebras is a theory of type dependency in which we have a notion of family extension. They are called such because if we try to interpret a category with families as a context algebras, one of the things we have to do is to consider lists of dependent types as our notion of families. These are the objects De Bruijn called telescopes. On the other hand, the theory of context algebras is an (essentially) algebraic theory of the notion of structure. A family over a context $\Gamma$ in a context algebra is a structure in context $\Gamma$. Therefore I am also considering the name `structure algebras'. Types are a defined notion in context algebras. They are the atomic families (or structures), i.e.~the ones which do not decompose as the extension of two non-trivial structures. 
\end{comment}

\subsection*{Setting up the essentially algebraic theory of context algebras}
The first goal of this paper is to describe an essentially algebraic theory of
type dependency that is in close correspondence to the type theory of
\cite{TheBook}, while at the same time it is as economic as possible in the
demand of requirements of the meta-theory (such as the availability of a type of natural numbers). Instead of explaning
what a dependent type is via an infinite hierarchy of universes $\UU_0$,
$\UU_1$, \ldots, we describe type dependency using only the three sorts
that typically occur in type theoretical expressions: contexts, dependent contexts and 
terms. Although an infinite hierarchy of universes allows one to reduce everything
to a term (because every type is a term of some specified universe), and 
therefore use the literally same mechanisms for handling type expressions as for
term expressions in the process of compilation and verification of a 
construction, this specific setup requires the natural numbers to exist in the 
meta-theory. 

In the specification of a formal language of type theory, we would therefore
have three pairs of kinds of judgments that can be made:
\begin{align*}
\jalign\jctx{\Gamma} 
& \jalign\jctxeq{\Gamma}{\Gamma'}
  \\
\jalign\jfam{\Gamma}{A} 
& \jalign\jfameq{\Gamma}{A}{B}
  \\
\jalign\jterm{\Gamma}{A}{x} 
& \jalign\jtermeq{\Gamma}{A}{x}{y}.
\end{align*}
These are asserting that $\Gamma$ is a context; that $\Gamma$ and $\Gamma'$ are
judgementally equal contexts; that $A$ is a family in context $\Gamma$; that
$A$ and $B$ are judgmentally equal families in context $\Gamma$; that $x$ is
a term of $A$ in context $\Gamma$; and that $x$ and $y$ are judgmentally equal
terms of $A$ in context $\Gamma$. 
 
a context algebra is therefore an algebra consisting of three sorts: $C$ for contexts,
$F$ for families, and $T$ for terms. There is a map 
$\eft:F\to C$ which determines for each family, in which context it is a
family; and there is a map $\ebd:T\to F$ which determines for each term, of
which family it is a term. Thus, the backbones of a context algebra can be represented
as
\begin{equation*}
\begin{tikzcd}
T \arrow[d,"\ebd"] \\ F \arrow[d,"\eft"] \\ C
\end{tikzcd}
\end{equation*}
This structure is represented by the following set of rules governing judgmental
equality. First, judgmental equality is postulated to be an equivalence relation
in all three cases (contexts, families and terms), so we require
%\bgroup\small
\begin{align*}
& \inference
  { \jctx{\Gamma}
    }
  { \jctxeq{\Gamma}{\Gamma}
    } 
& & \inference
    { \jctxeq{\Gamma}{\Delta}
      }
    { \jctxeq{\Delta}{\Gamma}
      } 
& & \inference
    { \jctxeq{\Gamma}{\Delta}
      \jctxeq{\Delta}{\greek{E}}
      }
    { \jctxeq{\Gamma}{\greek{E}}
      }
    \\
& \inference
  { \jfam{\Gamma}{A}
    }
  { \jfameq{\Gamma}{A}{A}
    } 
& & \inference
    { \jfameq{\Gamma}{A}{B}
      }
    { \jfameq{\Gamma}{B}{A}
      }
& & \inference
    { \jfameq{\Gamma}{A}{B}
      \jfameq{\Gamma}{B}{C}
      }
    { \jfameq{\Gamma}{A}{C}
      }
    \\
& \inference
  { \jterm{\Gamma}{A}{x}
    }
  { \jtermeq{\Gamma}{A}{x}{x}
    }
& & \inference
    { \jtermeq{\Gamma}{A}{x}{y}
      }
    { \jtermeq{\Gamma}{A}{y}{x}
      }
& & \inference
    { \jtermeq{\Gamma}{A}{x}{y}
      \jtermeq{\Gamma}{A}{y}{z}
      }
    { \jtermeq{\Gamma}{A}{x}{z}
      }
\end{align*}
%\egroup
Second, there are convertibility rules assuring that the notion of being a
family in a certain context, or a term of a family, is compatible with judgmental
equality.
\begin{align*}
& \inference
  { \jctxeq{\Gamma}{\Delta}
    \jfam{\Gamma}{A}
    }
  { \jfam{\Delta}{A}
    }
& & \inference
    { \jctxeq{\Gamma}{\Delta}
      \jfameq{\Gamma}{A}{B}
      }
    { \jfameq{\Delta}{A}{B}
      }
    \\
& \inference
  { \jctxeq{\Gamma}{\Delta}
    \jterm{\Gamma}{A}{x}
    }
  { \jterm{\Delta}{A}{x}
    }
& & \inference
    { \jctxeq{\Gamma}{\Delta}
      \jtermeq{\Gamma}{A}{x}{y}
      }
    { \jtermeq{\Delta}{A}{x}{y}
      }
    \\
& \inference
  { \jfameq{\Gamma}{A}{B}
    \jterm{\Gamma}{A}{x}
    }
  { \jterm{\Gamma}{B}{x}
    }
& & \inference
    { \jfameq{\Gamma}{A}{B}
      \jtermeq{\Gamma}{A}{x}{y}
      }
    { \jtermeq{\Gamma}{B}{x}{y}
      }
\end{align*}
When $A$ is a family in context $\Gamma$, there is also a context $\ctxext{\Gamma}{A}$,
called the \emph{context extension of $A$}. Context extension allows us to consider
`families over families', where a family $P$ over $A$ in context $\Gamma$, would
simply be a family $P$ over $\ctxext{\Gamma}{A}$. Context extension is represented
by another function $\cftctxext : F\to C$.

Since we want the rules of
type theory to be valid in any context $\Gamma$, there is also a notion of family
extension. Thus, for any family $P$ over $A$ in context $\Gamma$, there is a
family $\ctxext{A}{P}$ in context $\Gamma$. In other words, we have a function
$\cftfamext : F\times_{\cftctxext,\eft} F\to F$. 

Two associativity laws will require
that context and family extension are compatible with each other: we have
the equality $\ctxext{{\Gamma}{A}}{P}\jdeq\ctxext{\Gamma}{{A}{P}}$ of contexts,
and the equality $\ctxext{{A}{P}}{Q}\jdeq\ctxext{A}{{P}{Q}}$ of families.

Finally, there is an empty context $\emptyc\in C$, and an empty family $\emptyf[\Gamma]$
in context $\Gamma$, for every $\Gamma\in C$. The empty context is such that context extension
restricted to families over the empty context is an isomorphism, and the empty
families are neutral with respect to context and family extension in the sense
that $\ctxext{\Gamma}{\emptyf[\Gamma]}\jdeq\Gamma$, $\ctxext{\emptyf[\Gamma]}{A}
\jdeq A$ and $\ctxext{A}{\emptyf[\ctxext{\Gamma}{A}]}\jdeq A$. 

One may recognize in these ingredients of context algebras a category with a
terminal object, and a set $T(f)$ of terms for every morphism $f$ (satisfying
no further axioms, so far). Ignoring the set of terms for the moment, $C$ is
the set of objects and $F$ is the set of morphisms; $\eft$ assigns the codomain
to each morphism and $\cftctxext$ assigns the domain to each  morphism. The
operation $\cftfamext$ composes two morphisms. The first associativity rule
asserts that the domain of the composition $\ctxext{A}{P}$ is the domain of
$P$, and the second associativity rule asserts that composition is associative
in the usual sense. The empty context satisfies exactly the properties of the
terminal object, and the empty family $\emptyf[\Gamma]$ satisfies the properties
of the identity morphisms. 

Thus, a context algebra $\mathbf{E}$ will have an underlying category, which may be called the
category of families of $\mathbf{E}$. The structure of type dependency will by
built on top of this category. This structure will consist of a telescopic homomorphism
$W_A$ called \emph{weakening by $A$}, for every family $A$ in context $\Gamma$,
a telescopic homomorphism $S_x$ called \emph{substitution by $x$}, for every term
$x$ of $A$ in context $\Gamma$, and units $\idtm{A}\in T(W_A(A))$ for every
family $A$ in context $\Gamma$, satisfying a handful of further axioms. 
The weakening and substitution homomorphisms
will preserve the extension structure we have discussed above; in other words,
they will be functors preserving the terminal object.

The category of B-systems will be shown to be equivalent to the category of
stratified context algebras. However, we do not arrive at the fact that the functor
$\mathcal{E}:\mathbf{Bsys}\to\mathbf{Ctx}$ is full without imposing an additional
condition on the E-homomorphisms. 
To ensure that the category of stratified context algebras with stratified 
E-homomorphisms between them is a full subcategory of the category 
$\mathbf{Ctx}$, we will require that the underlying functor of each E-homomorphism
lifts factorizations. It is
important to note that this condition is also algebraic, so that the category
of context algebras remains algebraic.

Recall that for any morphism $f:X\to Y$ in a category $\cat{C}$ there is a category
$\mathbf{fact}(f)$ of factorizations of $f$. Also, any functor $F:\cat{C}\to\cat{D}$
determines a functor $F_{\mathbf{fact}(f)}:\mathbf{fact}(f)\to\mathbf{fact}(H(f))$
for any morphism $f$ of $\cat{C}$.

\begin{defn}
A functor $F:\cat{C}\to\cat{D}$ is said to
\define{lift factorizations} if for any $f\in\cat{C}/X$, the functor
$F_{\mathbf{fact}(f)}:\mathbf{fact}(f)\to\mathbf{fact}(F(f))$ is an isomorphism
of categories. 
\end{defn}

\begin{comment}
\begin{rmk}
We may choose the property of lifting factorizations
to involve either an equivalence or an isomorphism of categories. The
isomorphism-version, would say that for every factorization $h'\circ g'$
of $F(f)$ there is a unique factorization $f\jdeq h\circ g$ in $\cat{C}$ such
that $F(g)\jdeq g'$ and $F(h)\jdeq h'$. In the version with equivalences,
the uniqueness is replaced by uniqueness up to isomorphism, and the equalites
are replaced by an isomorphism in $\mathbf{fact}(F(f))$. 

It is easier to state the isomorphism version of lifting factorizations with
inference rules and it might be easier to explain this condition on type theoretical
grounds, although the equivalence version has the advantage of being
categorical (i.e.~invariant under equivalence of categories).

In the present
context, it doesn't matter very much which one we pick: the categories in which
we're interested are all posets.
\end{rmk}
\end{comment}

\begin{eg}
For any slice category $\cat{C}/X$, the forgetful functor $\cat{C}/X\to\cat{C}$
lifts factorizations.
\end{eg}

Convention: A functor in this paper is a functor which lifts factorizations.

\subsection*{The pre-categorical nature of context algebras}
Although we start our definition of context algebras with a category with term structure,
we do not set out to have a categorical notion of context algebras, in the sense that the structure of
being a context algebra can be transported along equivalences of categories. The goal
in this paper is to describe the algebraic structures belonging to an essentially
algebraic theory of type dependency, and this structure turns out to be pre-categorical
rather than categorical. The fact that we've encountered a category at the outset
of our definition of context algebras is in a sense coincidence, rather than born out
of a desire to do type theory categorically and we would consider context algebras to
be pre-categorical models of type theory rather than categorical models of type
theory. The difference between pre-categories and categories is that equality
of objects in a pre-category is considered to be (strict) equality, whereas
equality of objects in a category is considered to be isomorphism of objects,
and we have chosen to use the former notion of equality.

\subsection*{Existing notions of categorical models of dependent type theory}

When introducing a new notion of model of dependent type theory, one necessarily
faces the task of comparing the new notion with existing notions, and there are
quite a few. 

\begin{itemize}
\item Contextual categories were introduced by Cartmell \cite{Cartmell1986} and were studied
extensively by Voevodsky in \cite{VV_Csys_univ,VV_C-systems_monad,VV_C-systems_quotients}
under the name C-systems.
\item Categories with families, introduced by Dybjer in \cite{Dybjer1996}, model a
theory of type dependency with sorts for contexts, context morphisms, types
and terms. Among the virtues of categories with families are that its theory
is essentially algebraic, and they are relatively simple to work with.

Categories with families are models of a dependent type theory with a
category of contexts, for each context $\Gamma$ a sort $\mathrm{Ty}(\Gamma)$ of
types in context $\Gamma$, and for each type $A\in\mathrm{Ty}(\Gamma)$ a sort
$\mathrm{Tm}(A)$ of terms of type $A$ in context $\Gamma$. We will see that
the sorts $\mathrm{Tm}(A)$ are superfluous, because it will always be 
naturally isomorphic to the set of sections of the projection morphism
$p_A:\ctxext{\Gamma}{A}\to\Gamma$ in the category of contexts. The evident
functor from categories with families to context algebras will forget that the contexts
form a category, and therefore it will not be faithful.
\item Comprehension categories were introduced by Jacobs in \cite{Jacobs1993}, as
a minimal notion of categorical model in type theory, without really making
precise what that means. 
Comprehension categories are models of a dependent type theory with a
category of context, and for each context $\Gamma$ a category of types.
However, it is not quite clear what kind of theory the theory of comprehension
categories is, which compromises the study of the notion itself.
\end{itemize}

There is a simple test to quickly figure out some basic differences between different
dependent type theories, asking what kind of structure remains if there are no
types at all. In a contextual category, only the terminal object remains.
However, when the presheaf $\mathrm{Ty}$ of a category with families is assumed
to be empty, or when a category $\cat{E}$ fibered over some category is assumed
to be empty, the underlying category of context remains. We shall see that
context algebras are like contextual categories in this respect: only a singleton can
be a context algebra with no types. This complicates a thorough comparison between
categories with families and comprehension categories on one side, and context algebras
on the other. We shall construct functors from the categories of categories with
families and comprehension categories directly preserving type dependency, but
these functors will not be faithful. 

Thus, it seems that both categories with families and comprehension categories
have more structure than needed to model dependent type theory.

\subsection*{Overview and conventions}
In \autoref{sec:esys_defn}, we give the categorical definition of context algebras in 
several stages. First we define weakening systems, then projection systems and
substitution systems, and finally we bundle them together into the definition
of context algebras. The structure of the definition of weakening systems is 
prototypcial: we first define the notion of pre-weakening systems, which are
categories that come with a pre-weakening structure; then we define the notion
of pre-weakening homomorphism, which is a functor preserving the weakening
structure and we note that it makes sense to ask for the property that the
functors of which the weakening system consists, are also pre-weakening
homomorphisms; and finally, adding this requirement will be give the definition
of weakening systems. We go through this process for each of the definitions
of weakening systems, projection systems, substitution systems and context algebras.

Throughout this section, our focus is on (set-valued) models of the essentially algebraic
theory of type dependency, and in fact we shall not give the syntactic
presentation of the theory. A syntactic presentation of type theory in the
usual type theoretic style and models in arbitrary categories with finite limits
have been presented by the author during the Homotopy Type Theory workshop in
Oxford, November 2014%
\footnote{See \url{http://www.qmac.ox.ac.uk/events/Talk\%20slides/Egbert\%20Rijke.pdf}}.

In \autoref{sec:esys_props}, we introduce the auxiliary notion of E'-system.
An E'-system consists of a pre-category $\cat{C}$ with a sub-pre-category
$\cat{F}$ with the same objects as $\cat{C}$, and a functorial choice of
pullbacks for any morphism $f:\Delta\to\Gamma$ in $\cat{C}$ and any
$A\in\cat{F}/\Gamma$. The goal of this section is to show that 
E'-systems and context algebras are different axiomatizations of the same structures,
and apart from that the goal is to show some equational reasoning in context algebras.
The properties proven in this section also serve to provide some of the 
intuition behind the constructions in \autoref{sec:esys_compared}.

Although we have introduced a notion of E-homomorphism, a better behaved category
of context algebras is the one where E-homomorphisms lift factorizations. The condition
that a functor lifts factorizations is also algebraic, so we still have an
essentially algebraic theory of context algebras if we require this.
In \autoref{sec:esys_compared}, where we introduce the category of context algebras
with E-homomorphisms lifting factorizations, we construct an embedding from Voevodsky's
B-systems into context algebras and prove that its image is the (full) subcategory
of stratified context algebras. 

\begin{rmk}
In the present article we use $\jdeq$ for equality, because the equality we
consider here will interpret the judgmental equality of type theory, 
and we wish to reserve
the more convenient symbol $=$ for the internal equality of context algebras, i.e.~for
the identification types, as in \cite{TheBook}. 
\end{rmk}

\begin{rmk}
Recall that the slice pre-category $\cat{C}/\Gamma/(\Delta,f)$ is isomorphic to the slice pre-category
$\cat{C}/\Delta$, in the essentially algebraic theory of pre-categories. In the following,
we shall usually identify these two categories. Recall also that for every functor
$F:\cat{C}\to\cat{D}$ and every object $\Delta$ of $\cat{C}$, there is a functor
$F/\Delta:\cat{C}/\Delta\to \cat{D}/F(\Delta)$ given by the action on morphisms of $F$. 
\end{rmk}

\begin{rmk}
When we think of a category $\cat{F}$ as a category of fibrations or families
over a context -- which will be the outset of the definition of context algebras --
it will be convenient to denote the objects of the $\cat{F}$ with capital
greek letters $\Gamma$, $\Delta$, \ldots, and the morphisms by capital latin letters
$A$, $B$, $P$, $Q$, \ldots. When $A$ is a morphism into $\Gamma$, we will denote
its domain by $\ctxext{\Gamma}{A}$, so we don't always have to introduce a name
for the domain. When $A$ is a morphism into $\Gamma$ and
$P$ is a morphism into $\ctxext{\Gamma}{A}$, it will also be convenient to denote the
composition of $P$ with $A$ by $\ctxext{A}{P}$. Note that $\ctxext{A}{P}$
is indeed the domain of $P$ in the slice category $\cat{F}/\Gamma$.
\end{rmk}

\begin{rmk}
We have chosen not to use common type theoretical notation involving special
expressions for judgments and for inference rules, because we want to focus
on the algebraic structure of type dependency rather than on its syntax.

We would also like to remark that although it appears that we introduce possibly
large families of operations (a weakening operation $W_A$ for each morphism $A$,
for instance), context algebras can actually be defined with finitely many
operations. This becomes clearer when one defines a context algebra in an arbitrary
category with finite limits.
\end{rmk}

\subsection*{Acknowledgements}
Several people have been indispensible throughout the research process.
I would like to thank 
Richard Garner for the early discussions on the inference rules of type 
dependency we had in Barcelona and in Paris; 
Joachim Kock for the many fruitful discussions and for hosting me at the Universitat
Aut\`onoma de Barcelona during the year 2013-2014;
Steve Awodey for sharing many insights and useful advice which lead to the
definition of a context algebra in arbitrary categories with finite limits;
and Vladimir Voevodsky for inviting me to present context algebras at the Homotopy Type
Theory workshop in Oxford (November 2014) and to visit the Institute for Advanced 
Study in Princeton (March 2015), where the definition presented here was
discovered.
