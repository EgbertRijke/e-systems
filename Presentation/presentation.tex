\documentclass{beamer}
\setbeamercolor{frametitle}{fg=teal!90!black}
\setbeamercolor{item}{fg=red!90!black}
\setbeamercolor{title}{fg=teal!90!black}
\setbeamercolor{block title}{fg=teal!90!black}

%%%%%%%%%%%%%%%%%%%%%%%%%%%%%%%%%%%%%%%%%%%%%%%%%%%%%%%%%%%%%%%%%%%%%%%%%%%%%%%%
%%%% PACKAGES

\usepackage[utf8]{inputenc}
\usepackage[english]{babel}

%%%% Spicing up the document
\usepackage{mathpazo}
\usepackage[scaled=0.95]{helvet}
\usepackage{courier}
\linespread{1.05} % Palatino looks better with this
\usepackage{microtype}

\usepackage{fancyhdr} % To set headers and footers
\usepackage{enumitem,mathtools,xspace,xcolor}
\usepackage{comment}
\usepackage{ifthen}
\usepackage{pifont}
\newcommand{\cmark}{\ding{51}\xspace}
\newcommand{\xmark}{\ding{55}\xspace}

\usepackage{graphicx}
\usepackage{tikz-cd}
\usepackage{tikz}
\usetikzlibrary{decorations.pathmorphing}
\usepackage[inference]{semantic}
\usepackage{booktabs}

\usepackage[hyphens]{url} % This package has to be loaded *before* hyperref
\usepackage[pagebackref,colorlinks,citecolor=darkgreen,linkcolor=darkgreen,unicode]{hyperref}
\definecolor{darkgreen}{rgb}{0,0.45,0}

% For some reason the following can't be above hyperref...
\usepackage{amssymb,amsmath,amsthm,stmaryrd,mathrsfs,wasysym}
\usepackage{aliascnt}
\usepackage[capitalize]{cleveref}

% The braket macro shouldn't be necessary
\usepackage{braket} % used for \setof{ ... } macro

%%%%%%%%%%%%%%%%%%%%%%%%%%%%%%%%%%%%%%%%%%%%%%%%%%%%%%%%%%%%%%%%%%%%%%%%%%%%%%%%
%% To include references in TOC we should use this package rather than a hack.
\usepackage{tocbibind}
%\usepackage{etoolbox}           % get \apptocmd
%\apptocmd{\thebibliography}{\addcontentsline{toc}{section}{References}}{}{} % tell bibliography to get itself into the table of contents


\begin{comment}
%%%% Header and footers
\pagestyle{fancyplain}
\setlength{\headheight}{15pt}
\renewcommand{\chaptermark}[1]{\markboth{\textsc{Chapter \thechapter. #1}}{}}
\renewcommand{\sectionmark}[1]{\markright{\textsc{\thesection\ #1}}}
\end{comment}

% TOC depth
\setcounter{tocdepth}{3}

\lhead[\fancyplain{}{{\thepage}}]%
      {\fancyplain{}{\nouppercase{\rightmark}}}
\rhead[\fancyplain{}{\nouppercase{\leftmark}}]%
      {\fancyplain{}{\thepage}}
\cfoot{\textsc{\footnotesize [Draft of \today]}}
\lfoot[]{}
\rfoot[]{}

%%%%%%%%%%%%%%%%%%%%%%%%%%%%%%%%%%%%%%%%%%%%%%%%%%%%%%%%%%%%%%%%%%%%%%%%%%%%%%%%
%%%% We mostly use the macros of the book, to keep notations
%%%% and conventions the same. Recall that when the macros file
%%%% is updated, we need to comment the lines containing the
%%%% string `[chapter]` since our article is not a book.
%%%%
%%%% Instructions for updating the macros.tex file:
%%%% - fetch the latest macros.tex file from the HoTT/book git repository.
%%%% - comment all lines containing "[chapter]" because this is not a book.
%%%% - comment the definition of pbcorner because the xypic package is not used.
%%%%
%%%% MACROS FOR NOTATION %%%%
% Use these for any notation where there are multiple options.

%%% Notes and exercise sections
\makeatletter
\newcommand{\sectionNotes}{\phantomsection\section*{Notes}\addcontentsline{toc}{section}{Notes}\markright{\textsc{\@chapapp{} \thechapter{} Notes}}}
\newcommand{\sectionExercises}[1]{\phantomsection\section*{Exercises}\addcontentsline{toc}{section}{Exercises}\markright{\textsc{\@chapapp{} \thechapter{} Exercises}}}
\makeatother

%%% Definitional equality (used infix) %%%
\newcommand{\jdeq}{\equiv}      % An equality judgment
\let\judgeq\jdeq
%\newcommand{\defeq}{\coloneqq}  % An equality currently being defined
\newcommand{\defeq}{\vcentcolon\equiv}  % A judgmental equality currently being defined

%%% Term being defined
\newcommand{\define}[1]{\textbf{#1}}

%%% Vec (for example)

\newcommand{\Vect}{\ensuremath{\mathsf{Vec}}}
\newcommand{\Fin}{\ensuremath{\mathsf{Fin}}}
\newcommand{\fmax}{\ensuremath{\mathsf{fmax}}}
\newcommand{\seq}[1]{\langle #1\rangle}

%%% Dependent products %%%
\def\prdsym{\textstyle\prod}
%% Call the macro like \prd{x,y:A}{p:x=y} with any number of
%% arguments.  Make sure that whatever comes *after* the call doesn't
%% begin with an open-brace, or it will be parsed as another argument.
\makeatletter
% Currently the macro is configured to produce
%     {\textstyle\prod}(x:A) \; {\textstyle\prod}(y:B),\ 
% in display-math mode, and
%     \prod_{(x:A)} \prod_{y:B}
% in text-math mode.
\def\prd#1{\@ifnextchar\bgroup{\prd@parens{#1}}{\@ifnextchar\sm{\prd@parens{#1}\@eatsm}{\prd@noparens{#1}}}}
\def\prd@parens#1{\@ifnextchar\bgroup%
  {\mathchoice{\@dprd{#1}}{\@tprd{#1}}{\@tprd{#1}}{\@tprd{#1}}\prd@parens}%
  {\@ifnextchar\sm%
    {\mathchoice{\@dprd{#1}}{\@tprd{#1}}{\@tprd{#1}}{\@tprd{#1}}\@eatsm}%
    {\mathchoice{\@dprd{#1}}{\@tprd{#1}}{\@tprd{#1}}{\@tprd{#1}}}}}
\def\@eatsm\sm{\sm@parens}
\def\prd@noparens#1{\mathchoice{\@dprd@noparens{#1}}{\@tprd{#1}}{\@tprd{#1}}{\@tprd{#1}}}
% Helper macros for three styles
\def\lprd#1{\@ifnextchar\bgroup{\@lprd{#1}\lprd}{\@@lprd{#1}}}
\def\@lprd#1{\mathchoice{{\textstyle\prod}}{\prod}{\prod}{\prod}({\textstyle #1})\;}
\def\@@lprd#1{\mathchoice{{\textstyle\prod}}{\prod}{\prod}{\prod}({\textstyle #1}),\ }
\def\tprd#1{\@tprd{#1}\@ifnextchar\bgroup{\tprd}{}}
\def\@tprd#1{\mathchoice{{\textstyle\prod_{(#1)}}}{\prod_{(#1)}}{\prod_{(#1)}}{\prod_{(#1)}}}
\def\dprd#1{\@dprd{#1}\@ifnextchar\bgroup{\dprd}{}}
\def\@dprd#1{\prod_{(#1)}\,}
\def\@dprd@noparens#1{\prod_{#1}\,}

%%% Lambda abstractions.
% Each variable being abstracted over is a separate argument.  If
% there is more than one such argument, they *must* be enclosed in
% braces.  Arguments can be untyped, as in \lam{x}{y}, or typed with a
% colon, as in \lam{x:A}{y:B}. In the latter case, the colons are
% automatically noticed and (with current implementation) the space
% around the colon is reduced.  You can even give more than one variable
% the same type, as in \lam{x,y:A}.
\def\lam#1{{\lambda}\@lamarg#1:\@endlamarg\@ifnextchar\bgroup{.\,\lam}{.\,}}
\def\@lamarg#1:#2\@endlamarg{\if\relax\detokenize{#2}\relax #1\else\@lamvar{\@lameatcolon#2},#1\@endlamvar\fi}
\def\@lamvar#1,#2\@endlamvar{(#2\,{:}\,#1)}
% \def\@lamvar#1,#2{{#2}^{#1}\@ifnextchar,{.\,{\lambda}\@lamvar{#1}}{\let\@endlamvar\relax}}
\def\@lameatcolon#1:{#1}
\let\lamt\lam
% This version silently eats any typing annotation.
\def\lamu#1{{\lambda}\@lamuarg#1:\@endlamuarg\@ifnextchar\bgroup{.\,\lamu}{.\,}}
\def\@lamuarg#1:#2\@endlamuarg{#1}

%%% Dependent products written with \forall, in the same style
\def\fall#1{\forall (#1)\@ifnextchar\bgroup{.\,\fall}{.\,}}

%%% Existential quantifier %%%
\def\exis#1{\exists (#1)\@ifnextchar\bgroup{.\,\exis}{.\,}}

%%% Dependent sums %%%
\def\smsym{\textstyle\sum}
% Use in the same way as \prd
\def\sm#1{\@ifnextchar\bgroup{\sm@parens{#1}}{\@ifnextchar\prd{\sm@parens{#1}\@eatprd}{\sm@noparens{#1}}}}
\def\sm@parens#1{\@ifnextchar\bgroup%
  {\mathchoice{\@dsm{#1}}{\@tsm{#1}}{\@tsm{#1}}{\@tsm{#1}}\sm@parens}%
  {\@ifnextchar\prd%
    {\mathchoice{\@dsm{#1}}{\@tsm{#1}}{\@tsm{#1}}{\@tsm{#1}}\@eatprd}%
    {\mathchoice{\@dsm{#1}}{\@tsm{#1}}{\@tsm{#1}}{\@tsm{#1}}}}}
\def\@eatprd\prd{\prd@parens}
\def\sm@noparens#1{\mathchoice{\@dsm@noparens{#1}}{\@tsm{#1}}{\@tsm{#1}}{\@tsm{#1}}}
\def\lsm#1{\@ifnextchar\bgroup{\@lsm{#1}\lsm}{\@@lsm{#1}}}
\def\@lsm#1{\mathchoice{{\textstyle\sum}}{\sum}{\sum}{\sum}({\textstyle #1})\;}
\def\@@lsm#1{\mathchoice{{\textstyle\sum}}{\sum}{\sum}{\sum}({\textstyle #1}),\ }
\def\tsm#1{\@tsm{#1}\@ifnextchar\bgroup{\tsm}{}}
\def\@tsm#1{\mathchoice{{\textstyle\sum_{(#1)}}}{\sum_{(#1)}}{\sum_{(#1)}}{\sum_{(#1)}}}
\def\dsm#1{\@dsm{#1}\@ifnextchar\bgroup{\dsm}{}}
\def\@dsm#1{\sum_{(#1)}\,}
\def\@dsm@noparens#1{\sum_{#1}\,}

%%% W-types
\def\wtypesym{{\mathsf{W}}}
\def\wtype#1{\@ifnextchar\bgroup%
  {\mathchoice{\@twtype{#1}}{\@twtype{#1}}{\@twtype{#1}}{\@twtype{#1}}\wtype}%
  {\mathchoice{\@twtype{#1}}{\@twtype{#1}}{\@twtype{#1}}{\@twtype{#1}}}}
\def\lwtype#1{\@ifnextchar\bgroup{\@lwtype{#1}\lwtype}{\@@lwtype{#1}}}
\def\@lwtype#1{\mathchoice{{\textstyle\mathsf{W}}}{\mathsf{W}}{\mathsf{W}}{\mathsf{W}}({\textstyle #1})\;}
\def\@@lwtype#1{\mathchoice{{\textstyle\mathsf{W}}}{\mathsf{W}}{\mathsf{W}}{\mathsf{W}}({\textstyle #1}),\ }
\def\twtype#1{\@twtype{#1}\@ifnextchar\bgroup{\twtype}{}}
\def\@twtype#1{\mathchoice{{\textstyle\mathsf{W}_{(#1)}}}{\mathsf{W}_{(#1)}}{\mathsf{W}_{(#1)}}{\mathsf{W}_{(#1)}}}
\def\dwtype#1{\@dwtype{#1}\@ifnextchar\bgroup{\dwtype}{}}
\def\@dwtype#1{\mathsf{W}_{(#1)}\,}

\newcommand{\suppsym}{{\mathsf{sup}}}
\newcommand{\supp}{\ensuremath\suppsym\xspace}

\def\wtypeh#1{\@ifnextchar\bgroup%
  {\mathchoice{\@lwtypeh{#1}}{\@twtypeh{#1}}{\@twtypeh{#1}}{\@twtypeh{#1}}\wtypeh}%
  {\mathchoice{\@@lwtypeh{#1}}{\@twtypeh{#1}}{\@twtypeh{#1}}{\@twtypeh{#1}}}}
\def\lwtypeh#1{\@ifnextchar\bgroup{\@lwtypeh{#1}\lwtypeh}{\@@lwtypeh{#1}}}
\def\@lwtypeh#1{\mathchoice{{\textstyle\mathsf{W}^h}}{\mathsf{W}^h}{\mathsf{W}^h}{\mathsf{W}^h}({\textstyle #1})\;}
\def\@@lwtypeh#1{\mathchoice{{\textstyle\mathsf{W}^h}}{\mathsf{W}^h}{\mathsf{W}^h}{\mathsf{W}^h}({\textstyle #1}),\ }
\def\twtypeh#1{\@twtypeh{#1}\@ifnextchar\bgroup{\twtypeh}{}}
\def\@twtypeh#1{\mathchoice{{\textstyle\mathsf{W}^h_{(#1)}}}{\mathsf{W}^h_{(#1)}}{\mathsf{W}^h_{(#1)}}{\mathsf{W}^h_{(#1)}}}
\def\dwtypeh#1{\@dwtypeh{#1}\@ifnextchar\bgroup{\dwtypeh}{}}
\def\@dwtypeh#1{\mathsf{W}^h_{(#1)}\,}


\makeatother

% Other notations related to dependent sums
\let\setof\Set    % from package 'braket', write \setof{ x:A | P(x) }.
\newcommand{\pair}{\ensuremath{\mathsf{pair}}\xspace}
\newcommand{\tup}[2]{(#1,#2)}
\newcommand{\proj}[1]{\ensuremath{\mathsf{pr}_{#1}}\xspace}
\newcommand{\fst}{\ensuremath{\proj1}\xspace}
\newcommand{\snd}{\ensuremath{\proj2}\xspace}
\newcommand{\ac}{\ensuremath{\mathsf{ac}}\xspace} % not needed in symbol index
\newcommand{\un}{\ensuremath{\mathsf{upun}}\xspace} % not needed in symbol index, uniqueness principle for unit type

%%% recursor and induction
\newcommand{\rec}[1]{\mathsf{rec}_{#1}}
\newcommand{\ind}[1]{\mathsf{ind}_{#1}}
\newcommand{\indid}[1]{\ind{=_{#1}}} % (Martin-Lof) path induction principle for identity types
\newcommand{\indidb}[1]{\ind{=_{#1}}'} % (Paulin-Mohring) based path induction principle for identity types 

%%% the uniqueness principle for product types, formerly called surjective pairing and named \spr:
\newcommand{\uppt}{\ensuremath{\mathsf{uppt}}\xspace}

% Paths in pairs
\newcommand{\pairpath}{\ensuremath{\mathsf{pair}^{\mathord{=}}}\xspace}
% \newcommand{\projpath}[1]{\proj{#1}^{\mathord{=}}}
\newcommand{\projpath}[1]{\ensuremath{\apfunc{\proj{#1}}}\xspace}

%%% For quotients %%%
%\newcommand{\pairr}[1]{{\langle #1\rangle}}
\newcommand{\pairr}[1]{{\mathopen{}(#1)\mathclose{}}}
\newcommand{\Pairr}[1]{{\mathopen{}\left(#1\right)\mathclose{}}}

% \newcommand{\type}{\ensuremath{\mathsf{Type}}} % this command is overridden below, so it's commented out
\newcommand{\im}{\ensuremath{\mathsf{im}}} % the image

%%% 2D path operations
\newcommand{\leftwhisker}{\mathbin{{\ct}_{\ell}}}
\newcommand{\rightwhisker}{\mathbin{{\ct}_{r}}}
\newcommand{\hct}{\star}

%%% modalities %%%
\newcommand{\modal}{\ensuremath{\ocircle}}
\let\reflect\modal
\newcommand{\modaltype}{\ensuremath{\type_\modal}}
% \newcommand{\ism}[1]{\ensuremath{\mathsf{is}_{#1}}}
% \newcommand{\ismodal}{\ism{\modal}}
% \newcommand{\existsmodal}{\ensuremath{{\exists}_{\modal}}}
% \newcommand{\existsmodalunique}{\ensuremath{{\exists!}_{\modal}}}
% \newcommand{\modalfunc}{\textsf{\modal-fun}}
% \newcommand{\Ecirc}{\ensuremath{\mathsf{E}_\modal}}
% \newcommand{\Mcirc}{\ensuremath{\mathsf{M}_\modal}}
\newcommand{\mreturn}{\ensuremath{\eta}}
\let\project\mreturn
%\newcommand{\mbind}[1]{\ensuremath{\hat{#1}}}
\newcommand{\ext}{\mathsf{ext}}
%\newcommand{\mmap}[1]{\ensuremath{\bar{#1}}}
%\newcommand{\mjoin}{\ensuremath{\mreturn^{-1}}}
% Subuniverse
\renewcommand{\P}{\ensuremath{\type_{P}}\xspace}

%%% Localizations
% \newcommand{\islocal}[1]{\ensuremath{\mathsf{islocal}_{#1}}\xspace}
% \newcommand{\loc}[1]{\ensuremath{\mathcal{L}_{#1}}\xspace}

%%% Identity types %%%
\newcommand{\idsym}{{=}}
\newcommand{\id}[3][]{\ensuremath{#2 =_{#1} #3}\xspace}
\newcommand{\idtype}[3][]{\ensuremath{\mathsf{Id}_{#1}(#2,#3)}\xspace}
\newcommand{\idtypevar}[1]{\ensuremath{\mathsf{Id}_{#1}}\xspace}
% A propositional equality currently being defined
\newcommand{\defid}{\coloneqq}

%%% Dependent paths
\newcommand{\dpath}[4]{#3 =^{#1}_{#2} #4}

%%% singleton
% \newcommand{\sgl}{\ensuremath{\mathsf{sgl}}\xspace}
% \newcommand{\sctr}{\ensuremath{\mathsf{sctr}}\xspace}

%%% Reflexivity terms %%%
% \newcommand{\reflsym}{{\mathsf{refl}}}
\newcommand{\refl}[1]{\ensuremath{\mathsf{refl}_{#1}}\xspace}

%%% Path concatenation (used infix, in diagrammatic order) %%%
\newcommand{\ct}{%
  \mathchoice{\mathbin{\raisebox{0.5ex}{$\displaystyle\centerdot$}}}%
             {\mathbin{\raisebox{0.5ex}{$\centerdot$}}}%
             {\mathbin{\raisebox{0.25ex}{$\scriptstyle\,\centerdot\,$}}}%
             {\mathbin{\raisebox{0.1ex}{$\scriptscriptstyle\,\centerdot\,$}}}
}

%%% Path reversal %%%
\newcommand{\opp}[1]{\mathord{{#1}^{-1}}}
\let\rev\opp

%%% Transport (covariant) %%%
\newcommand{\trans}[2]{\ensuremath{{#1}_{*}\mathopen{}\left({#2}\right)\mathclose{}}\xspace}
\let\Trans\trans
%\newcommand{\Trans}[2]{\ensuremath{{#1}_{*}\left({#2}\right)}\xspace}
\newcommand{\transf}[1]{\ensuremath{{#1}_{*}}\xspace} % Without argument
%\newcommand{\transport}[2]{\ensuremath{\mathsf{transport}_{*} \: {#2}\xspace}}
\newcommand{\transfib}[3]{\ensuremath{\mathsf{transport}^{#1}(#2,#3)\xspace}}
\newcommand{\Transfib}[3]{\ensuremath{\mathsf{transport}^{#1}\Big(#2,\, #3\Big)\xspace}}
\newcommand{\transfibf}[1]{\ensuremath{\mathsf{transport}^{#1}\xspace}}

%%% 2D transport
\newcommand{\transtwo}[2]{\ensuremath{\mathsf{transport}^2\mathopen{}\left({#1},{#2}\right)\mathclose{}}\xspace}

%%% Constant transport
\newcommand{\transconst}[3]{\ensuremath{\mathsf{transportconst}}^{#1}_{#2}(#3)\xspace}
\newcommand{\transconstf}{\ensuremath{\mathsf{transportconst}}\xspace}

%%% Map on paths %%%
\newcommand{\mapfunc}[1]{\ensuremath{\mathsf{ap}_{#1}}\xspace} % Without argument
\newcommand{\map}[2]{\ensuremath{{#1}\mathopen{}\left({#2}\right)\mathclose{}}\xspace}
\let\Ap\map
%\newcommand{\Ap}[2]{\ensuremath{{#1}\left({#2}\right)}\xspace}
\newcommand{\mapdepfunc}[1]{\ensuremath{\mathsf{apd}_{#1}}\xspace} % Without argument
% \newcommand{\mapdep}[2]{\ensuremath{{#1}\llparenthesis{#2}\rrparenthesis}\xspace}
\newcommand{\mapdep}[2]{\ensuremath{\mapdepfunc{#1}\mathopen{}\left(#2\right)\mathclose{}}\xspace}
\let\apfunc\mapfunc
\let\ap\map
\let\apdfunc\mapdepfunc
\let\apd\mapdep

%%% 2D map on paths
\newcommand{\aptwofunc}[1]{\ensuremath{\mathsf{ap}^2_{#1}}\xspace}
\newcommand{\aptwo}[2]{\ensuremath{\aptwofunc{#1}\mathopen{}\left({#2}\right)\mathclose{}}\xspace}
\newcommand{\apdtwofunc}[1]{\ensuremath{\mathsf{apd}^2_{#1}}\xspace}
\newcommand{\apdtwo}[2]{\ensuremath{\apdtwofunc{#1}\mathopen{}\left(#2\right)\mathclose{}}\xspace}

%%% Identity functions %%%
\newcommand{\idfunc}[1][]{\ensuremath{\mathsf{id}_{#1}}\xspace}

%%% Homotopies (written infix) %%%
\newcommand{\htpy}{\sim}

%%% Other meanings of \sim
\newcommand{\bisim}{\sim}       % bisimulation
\newcommand{\eqr}{\sim}         % an equivalence relation

%%% Equivalence types %%%
\newcommand{\eqv}[2]{\ensuremath{#1 \simeq #2}\xspace}
\newcommand{\eqvspaced}[2]{\ensuremath{#1 \;\simeq\; #2}\xspace}
\newcommand{\eqvsym}{\simeq}    % infix symbol
\newcommand{\texteqv}[2]{\ensuremath{\mathsf{Equiv}(#1,#2)}\xspace}
\newcommand{\isequiv}{\ensuremath{\mathsf{isequiv}}}
\newcommand{\qinv}{\ensuremath{\mathsf{qinv}}}
\newcommand{\ishae}{\ensuremath{\mathsf{ishae}}}
\newcommand{\linv}{\ensuremath{\mathsf{linv}}}
\newcommand{\rinv}{\ensuremath{\mathsf{rinv}}}
\newcommand{\biinv}{\ensuremath{\mathsf{biinv}}}
\newcommand{\lcoh}[3]{\mathsf{lcoh}_{#1}(#2,#3)}
\newcommand{\rcoh}[3]{\mathsf{rcoh}_{#1}(#2,#3)}
\newcommand{\hfib}[2]{{\mathsf{fib}}_{#1}(#2)}

%%% Map on total spaces %%%
\newcommand{\total}[1]{\ensuremath{\mathsf{total}(#1)}}

%%% Universe types %%%
%\newcommand{\type}{\ensuremath{\mathsf{Type}}\xspace}
\newcommand{\UU}{\ensuremath{\mathcal{U}}\xspace}
\let\bbU\UU
\let\type\UU
% Universes of truncated types
\newcommand{\typele}[1]{\ensuremath{{#1}\text-\mathsf{Type}}\xspace}
\newcommand{\typeleU}[1]{\ensuremath{{#1}\text-\mathsf{Type}_\UU}\xspace}
\newcommand{\typelep}[1]{\ensuremath{{(#1)}\text-\mathsf{Type}}\xspace}
\newcommand{\typelepU}[1]{\ensuremath{{(#1)}\text-\mathsf{Type}_\UU}\xspace}
\let\ntype\typele
\let\ntypeU\typeleU
\let\ntypep\typelep
\let\ntypepU\typelepU
\renewcommand{\set}{\ensuremath{\mathsf{Set}}\xspace}
\newcommand{\setU}{\ensuremath{\mathsf{Set}_\UU}\xspace}
\newcommand{\prop}{\ensuremath{\mathsf{Prop}}\xspace}
\newcommand{\propU}{\ensuremath{\mathsf{Prop}_\UU}\xspace}
%Pointed types
\newcommand{\pointed}[1]{\ensuremath{#1_\bullet}}

%%% Ordinals and cardinals
\newcommand{\card}{\ensuremath{\mathsf{Card}}\xspace}
\newcommand{\ord}{\ensuremath{\mathsf{Ord}}\xspace}
\newcommand{\ordsl}[2]{{#1}_{/#2}}

%%% Univalence
\newcommand{\ua}{\ensuremath{\mathsf{ua}}\xspace} % the inverse of idtoeqv
\newcommand{\idtoeqv}{\ensuremath{\mathsf{idtoeqv}}\xspace}
\newcommand{\univalence}{\ensuremath{\mathsf{univalence}}\xspace} % the full axiom

%%% Truncation levels
\newcommand{\iscontr}{\ensuremath{\mathsf{isContr}}}
\newcommand{\contr}{\ensuremath{\mathsf{contr}}} % The path to the center of contraction
\newcommand{\isset}{\ensuremath{\mathsf{isSet}}}
\newcommand{\isprop}{\ensuremath{\mathsf{isProp}}}
% h-propositions
% \newcommand{\anhprop}{a mere proposition\xspace}
% \newcommand{\hprops}{mere propositions\xspace}

%%% Homotopy fibers %%%
%\newcommand{\hfiber}[2]{\ensuremath{\mathsf{hFiber}(#1,#2)}\xspace}
\let\hfiber\hfib

%%% Bracket/squash/truncation types %%%
% \newcommand{\brck}[1]{\textsf{mere}(#1)}
% \newcommand{\Brck}[1]{\textsf{mere}\Big(#1\Big)}
% \newcommand{\trunc}[2]{\tau_{#1}(#2)}
% \newcommand{\Trunc}[2]{\tau_{#1}\Big(#2\Big)}
% \newcommand{\truncf}[1]{\tau_{#1}}
%\newcommand{\trunc}[2]{\Vert #2\Vert_{#1}}
\newcommand{\trunc}[2]{\mathopen{}\left\Vert #2\right\Vert_{#1}\mathclose{}}
\newcommand{\ttrunc}[2]{\bigl\Vert #2\bigr\Vert_{#1}}
\newcommand{\Trunc}[2]{\Bigl\Vert #2\Bigr\Vert_{#1}}
\newcommand{\truncf}[1]{\Vert \blank \Vert_{#1}}
\newcommand{\tproj}[3][]{\mathopen{}\left|#3\right|_{#2}^{#1}\mathclose{}}
\newcommand{\tprojf}[2][]{|\blank|_{#2}^{#1}}
\def\pizero{\trunc0}
%\newcommand{\brck}[1]{\trunc{-1}{#1}}
%\newcommand{\Brck}[1]{\Trunc{-1}{#1}}
%\newcommand{\bproj}[1]{\tproj{-1}{#1}}
%\newcommand{\bprojf}{\tprojf{-1}}

\newcommand{\brck}[1]{\trunc{}{#1}}
\newcommand{\bbrck}[1]{\ttrunc{}{#1}}
\newcommand{\Brck}[1]{\Trunc{}{#1}}
\newcommand{\bproj}[1]{\tproj{}{#1}}
\newcommand{\bprojf}{\tprojf{}}

% Big parentheses
\newcommand{\Parens}[1]{\Bigl(#1\Bigr)}

% Projection and extension for truncations
\let\extendsmb\ext
\newcommand{\extend}[1]{\extendsmb(#1)}

%
%%% The empty type
\newcommand{\emptyt}{\ensuremath{\mathbf{0}}\xspace}

%%% The unit type
\newcommand{\unit}{\ensuremath{\mathbf{1}}\xspace}
\newcommand{\ttt}{\ensuremath{\star}\xspace}

%%% The two-element type
\newcommand{\bool}{\ensuremath{\mathbf{2}}\xspace}
\newcommand{\btrue}{{1_{\bool}}}
\newcommand{\bfalse}{{0_{\bool}}}

%%% Injections into binary sums and pushouts
\newcommand{\inlsym}{{\mathsf{inl}}}
\newcommand{\inrsym}{{\mathsf{inr}}}
\newcommand{\inl}{\ensuremath\inlsym\xspace}
\newcommand{\inr}{\ensuremath\inrsym\xspace}

%%% The segment of the interval
\newcommand{\seg}{\ensuremath{\mathsf{seg}}\xspace}

%%% Free groups
\newcommand{\freegroup}[1]{F(#1)}
\newcommand{\freegroupx}[1]{F'(#1)} % the "other" free group

%%% Glue of a pushout
\newcommand{\glue}{\mathsf{glue}}

%%% Circles and spheres
\newcommand{\Sn}{\mathbb{S}}
\newcommand{\base}{\ensuremath{\mathsf{base}}\xspace}
\newcommand{\lloop}{\ensuremath{\mathsf{loop}}\xspace}
\newcommand{\surf}{\ensuremath{\mathsf{surf}}\xspace}

%%% Suspension
\newcommand{\susp}{\Sigma}
\newcommand{\north}{\mathsf{N}}
\newcommand{\south}{\mathsf{S}}
\newcommand{\merid}{\mathsf{merid}}

%%% Blanks (shorthand for lambda abstractions)
\newcommand{\blank}{\mathord{\hspace{1pt}\text{--}\hspace{1pt}}}

%%% Nameless objects
\newcommand{\nameless}{\mathord{\hspace{1pt}\underline{\hspace{1ex}}\hspace{1pt}}}

%%% Some decorations
%\newcommand{\bbU}{\ensuremath{\mathbb{U}}\xspace}
% \newcommand{\bbB}{\ensuremath{\mathbb{B}}\xspace}
\newcommand{\bbP}{\ensuremath{\mathbb{P}}\xspace}

%%% Some categories
\newcommand{\uset}{\ensuremath{\mathcal{S}et}\xspace}
\newcommand{\ucat}{\ensuremath{{\mathcal{C}at}}\xspace}
\newcommand{\urel}{\ensuremath{\mathcal{R}el}\xspace}
\newcommand{\uhilb}{\ensuremath{\mathcal{H}ilb}\xspace}
\newcommand{\utype}{\ensuremath{\mathcal{T}\!ype}\xspace}

% Pullback corner
%\newbox\pbbox
%\setbox\pbbox=\hbox{\xy \POS(65,0)\ar@{-} (0,0) \ar@{-} (65,65)\endxy}
%\def\pb{\save[]+<3.5mm,-3.5mm>*{\copy\pbbox} \restore}

% Macros for the categories chapter
\newcommand{\inv}[1]{{#1}^{-1}}
\newcommand{\idtoiso}{\ensuremath{\mathsf{idtoiso}}\xspace}
\newcommand{\isotoid}{\ensuremath{\mathsf{isotoid}}\xspace}
\newcommand{\op}{^{\mathrm{op}}}
\newcommand{\y}{\ensuremath{\mathbf{y}}\xspace}
\newcommand{\dgr}[1]{{#1}^{\dagger}}
\newcommand{\unitaryiso}{\mathrel{\cong^\dagger}}
\newcommand{\cteqv}[2]{\ensuremath{#1 \simeq #2}\xspace}
\newcommand{\cteqvsym}{\simeq}     % Symbol for equivalence of categories

%%% Natural numbers
\newcommand{\N}{\ensuremath{\mathbb{N}}\xspace}
%\newcommand{\N}{\textbf{N}}
\let\nat\N
\newcommand{\natp}{\ensuremath{\nat'}\xspace} % alternative nat in induction chapter

\newcommand{\zerop}{\ensuremath{0'}\xspace}   % alternative zero in induction chapter
\newcommand{\suc}{\mathsf{succ}}
\newcommand{\sucp}{\ensuremath{\suc'}\xspace} % alternative suc in induction chapter
\newcommand{\add}{\mathsf{add}}
\newcommand{\ack}{\mathsf{ack}}
\newcommand{\ite}{\mathsf{iter}}
\newcommand{\assoc}{\mathsf{assoc}}
\newcommand{\dbl}{\ensuremath{\mathsf{double}}}
\newcommand{\dblp}{\ensuremath{\dbl'}\xspace} % alternative double in induction chapter


%%% Lists
\newcommand{\lst}[1]{\mathsf{List}(#1)}
\newcommand{\nil}{\mathsf{nil}}
\newcommand{\cons}{\mathsf{cons}}

%%% Vectors of given length, used in induction chapter
\newcommand{\vect}[2]{\ensuremath{\mathsf{Vec}_{#1}(#2)}\xspace}

%%% Integers
\newcommand{\Z}{\ensuremath{\mathbb{Z}}\xspace}
\newcommand{\Zsuc}{\mathsf{succ}}
\newcommand{\Zpred}{\mathsf{pred}}

%%% Rationals
\newcommand{\Q}{\ensuremath{\mathbb{Q}}\xspace}

%%% Function extensionality
\newcommand{\funext}{\mathsf{funext}}
\newcommand{\happly}{\mathsf{happly}}

%%% A naturality lemma
\newcommand{\com}[3]{\mathsf{swap}_{#1,#2}(#3)}

%%% Code/encode/decode
\newcommand{\code}{\ensuremath{\mathsf{code}}\xspace}
\newcommand{\encode}{\ensuremath{\mathsf{encode}}\xspace}
\newcommand{\decode}{\ensuremath{\mathsf{decode}}\xspace}

% Function definition with domain and codomain
\newcommand{\function}[4]{\left\{\begin{array}{rcl}#1 &
      \longrightarrow & #2 \\ #3 & \longmapsto & #4 \end{array}\right.}

%%% Cones and cocones
\newcommand{\cone}[2]{\mathsf{cone}_{#1}(#2)}
\newcommand{\cocone}[2]{\mathsf{cocone}_{#1}(#2)}
% Apply a function to a cocone
\newcommand{\composecocone}[2]{#1\circ#2}
\newcommand{\composecone}[2]{#2\circ#1}
%%% Diagrams
\newcommand{\Ddiag}{\mathscr{D}}

%%% (pointed) mapping spaces
\newcommand{\Map}{\mathsf{Map}}

%%% The interval
\newcommand{\interval}{\ensuremath{I}\xspace}
\newcommand{\izero}{\ensuremath{0_{\interval}}\xspace}
\newcommand{\ione}{\ensuremath{1_{\interval}}\xspace}

%%% Arrows
\newcommand{\epi}{\ensuremath{\twoheadrightarrow}}
\newcommand{\mono}{\ensuremath{\rightarrowtail}}

%%% Sets
\newcommand{\bin}{\ensuremath{\mathrel{\widetilde{\in}}}}

%%% Semigroup structure
\newcommand{\semigroupstrsym}{\ensuremath{\mathsf{SemigroupStr}}}
\newcommand{\semigroupstr}[1]{\ensuremath{\mathsf{SemigroupStr}}(#1)}
\newcommand{\semigroup}[0]{\ensuremath{\mathsf{Semigroup}}}

%%% Macros for the formal type theory
\newcommand{\emptyctx}{\ensuremath{\cdot}}
\newcommand{\production}{\vcentcolon\vcentcolon=}
\newcommand{\conv}{\downarrow}
\newcommand{\wfctx}[1]{#1\ \ctx}
\newcommand{\oftp}[3]{#1 \vdash #2 : #3}
\newcommand{\jdeqtp}[4]{#1 \vdash #2 \jdeq #3 : #4}
\newcommand{\judg}[2]{#1 \vdash #2}
\newcommand{\tmtp}[2]{#1 \mathord{:} #2}

% rule names
\newcommand{\form}{\textsc{form}}
\newcommand{\intro}{\textsc{intro}}
\newcommand{\elim}{\textsc{elim}}
\newcommand{\comp}{\textsc{comp}}
\newcommand{\uniq}{\textsc{uniq}}
\newcommand{\Weak}{\mathsf{Wkg}}
\newcommand{\Vble}{\mathsf{Vble}}
\newcommand{\Exch}{\mathsf{Exch}}
\newcommand{\Subst}{\mathsf{Subst}}

%%% Macros for HITs
\newcommand{\cc}{\mathsf{c}}
\newcommand{\pp}{\mathsf{p}}
\newcommand{\cct}{\widetilde{\mathsf{c}}}
\newcommand{\ppt}{\widetilde{\mathsf{p}}}
\newcommand{\Wtil}{\ensuremath{\widetilde{W}}\xspace}

%%% Macros for n-types
\newcommand{\istype}[1]{\mathsf{is}\mbox{-}{#1}\mbox{-}\mathsf{type}}
\newcommand{\nplusone}{\ensuremath{(n+1)}}
\newcommand{\nminusone}{\ensuremath{(n-1)}}
\newcommand{\fact}{\mathsf{fact}}

%%% Macros for homotopy
\newcommand{\kbar}{\overline{k}} % Used in van Kampen's theorem

%%% Macros for induction
\newcommand{\natw}{\ensuremath{\mathbf{N^w}}\xspace}
\newcommand{\zerow}{\ensuremath{0^\mathbf{w}}\xspace}
\newcommand{\sucw}{\ensuremath{\mathbf{s^w}}\xspace}
\newcommand{\nalg}{\nat\mathsf{Alg}}
\newcommand{\nhom}{\nat\mathsf{Hom}}
\newcommand{\ishinitw}{\mathsf{isHinit}_{\mathsf{W}}}
\newcommand{\ishinitn}{\mathsf{isHinit}_\nat}
\newcommand{\w}{\mathsf{W}}
\newcommand{\walg}{\w\mathsf{Alg}}
\newcommand{\whom}{\w\mathsf{Hom}}

%%% Macros for real numbers
\newcommand{\RC}{\ensuremath{\mathbb{R}_\mathsf{c}}\xspace} % Cauchy
\newcommand{\RD}{\ensuremath{\mathbb{R}_\mathsf{d}}\xspace} % Dedekind
\newcommand{\R}{\ensuremath{\mathbb{R}}\xspace}           % Either 
\newcommand{\barRD}{\ensuremath{\bar{\mathbb{R}}_\mathsf{d}}\xspace} % Dedekind completion of Dedekind

\newcommand{\close}[1]{\sim_{#1}} % Relation of closeness
\newcommand{\closesym}{\mathord\sim}
\newcommand{\rclim}{\mathsf{lim}} % HIT constructor for Cauchy reals
\newcommand{\rcrat}{\mathsf{rat}} % Embedding of rationals into Cauchy reals
\newcommand{\rceq}{\mathsf{eq}_{\RC}} % HIT path constructor
\newcommand{\CAP}{\mathcal{C}}    % The type of Cauchy approximations
\newcommand{\Qp}{\Q_{+}}
\newcommand{\apart}{\mathrel{\#}}  % apartness
\newcommand{\dcut}{\mathsf{isCut}}  % Dedekind cut
\newcommand{\cover}{\triangleleft} % inductive cover
\newcommand{\intfam}[3]{(#2, \lam{#1} #3)} % family of rational intervals

% Macros for the Cauchy reals construction
\newcommand{\bsim}{\frown}
\newcommand{\bbsim}{\smile}

\newcommand{\hapx}{\diamondsuit\approx}
\newcommand{\hapname}{\diamondsuit}
\newcommand{\hapxb}{\heartsuit\approx}
\newcommand{\hapbname}{\heartsuit}
\newcommand{\tap}[1]{\bullet\approx_{#1}\triangle}
\newcommand{\tapname}{\triangle}
\newcommand{\tapb}[1]{\bullet\approx_{#1}\square}
\newcommand{\tapbname}{\square}

%%% Macros for surreals
\newcommand{\NO}{\ensuremath{\mathsf{No}}\xspace}
\newcommand{\surr}[2]{\{\,#1\,\big|\,#2\,\}}
\newcommand{\LL}{\mathcal{L}}
\newcommand{\RR}{\mathcal{R}}
\newcommand{\noeq}{\mathsf{eq}_{\NO}} % HIT path constructor

\newcommand{\ble}{\trianglelefteqslant}
\newcommand{\blt}{\vartriangleleft}
\newcommand{\bble}{\sqsubseteq}
\newcommand{\bblt}{\sqsubset}

\newcommand{\hle}{\diamondsuit\preceq}
\newcommand{\hlt}{\diamondsuit\prec}
\newcommand{\hlname}{\diamondsuit}
\newcommand{\hleb}{\heartsuit\preceq}
\newcommand{\hltb}{\heartsuit\prec}
\newcommand{\hlbname}{\heartsuit}
% \newcommand{\tle}{(\bullet\preceq\triangle)}
% \newcommand{\tlt}{(\bullet\prec\triangle)}
\newcommand{\tle}{\triangle\preceq}
\newcommand{\tlt}{\triangle\prec}
\newcommand{\tlname}{\triangle}
% \newcommand{\tleb}{(\bullet\preceq\square)}
% \newcommand{\tltb}{(\bullet\prec\square)}
\newcommand{\tleb}{\square\preceq}
\newcommand{\tltb}{\square\prec}
\newcommand{\tlbname}{\square}

%%% Macros for set theory
\newcommand{\vset}{\mathsf{set}}  % point constructor for cummulative hierarchy V
\def\cd{\tproj0}
\newcommand{\inj}{\ensuremath{\mathsf{inj}}} % type of injections
\newcommand{\acc}{\ensuremath{\mathsf{acc}}} % accessibility

\newcommand{\atMostOne}{\mathsf{atMostOne}}

\newcommand{\power}[1]{\mathcal{P}(#1)} % power set
\newcommand{\powerp}[1]{\mathcal{P}_+(#1)} % inhabited power set

%%%% THEOREM ENVIRONMENTS %%%%

% Hyperref includes the command \autoref{...} which is like \ref{...}
% except that it automatically inserts the type of the thing you're
% referring to, e.g. it produces "Theorem 3.8" instead of just "3.8"
% (and makes the whole thing a hyperlink).  This saves a slight amount
% of typing, but more importantly it means that if you decide later on
% that 3.8 should be a Lemma or a Definition instead of a Theorem, you
% don't have to change the name in all the places you referred to it.

% The following hack improves on this by using the same counter for
% all theorem-type environments, so that after Theorem 1.1 comes
% Corollary 1.2 rather than Corollary 1.1.  This makes it much easier
% for the reader to find a particular theorem when flipping through
% the document.
\makeatletter
\def\defthm#1#2#3{%
  %% Ensure all theorem types are numbered with the same counter
  \newaliascnt{#1}{thm}
  \newtheorem{#1}[#1]{#2}
  \aliascntresetthe{#1}
  %% This command tells cleveref's \cref what to call things
  \crefname{#1}{#2}{#3}}

% Now define a bunch of theorem-type environments.
\newtheorem{thm}{Theorem}[section]
\crefname{thm}{Theorem}{Theorems}
%\defthm{prop}{Proposition}   % Probably we shouldn't use "Proposition" in this way
\defthm{cor}{Corollary}{Corollaries}
\defthm{lem}{Lemma}{Lemmas}
\defthm{axiom}{Axiom}{Axioms}
% Since definitions and theorems in type theory are synonymous, should
% we actually use the same theoremstyle for them?
\theoremstyle{definition}
\defthm{defn}{Definition}{Definitions}
\theoremstyle{remark}
\defthm{rmk}{Remark}{Remarks}
\defthm{eg}{Example}{Examples}
\defthm{egs}{Examples}{Examples}
\defthm{notes}{Notes}{Notes}
% Number exercises within chapters, with their own counter.
%\newtheorem{ex}{Exercise}[chapter]
%\crefname{ex}{Exercise}{Exercises}

% Display format for sections
\crefformat{section}{\S#2#1#3}
\Crefformat{section}{Section~#2#1#3}
\crefrangeformat{section}{\S\S#3#1#4--#5#2#6}
\Crefrangeformat{section}{Sections~#3#1#4--#5#2#6}
\crefmultiformat{section}{\S\S#2#1#3}{ and~#2#1#3}{, #2#1#3}{ and~#2#1#3}
\Crefmultiformat{section}{Sections~#2#1#3}{ and~#2#1#3}{, #2#1#3}{ and~#2#1#3}
\crefrangemultiformat{section}{\S\S#3#1#4--#5#2#6}{ and~#3#1#4--#5#2#6}{, #3#1#4--#5#2#6}{ and~#3#1#4--#5#2#6}
\Crefrangemultiformat{section}{Sections~#3#1#4--#5#2#6}{ and~#3#1#4--#5#2#6}{, #3#1#4--#5#2#6}{ and~#3#1#4--#5#2#6}

% Display format for appendices
\crefformat{appendix}{Appendix~#2#1#3}
\Crefformat{appendix}{Appendix~#2#1#3}
\crefrangeformat{appendix}{Appendices~#3#1#4--#5#2#6}
\Crefrangeformat{appendix}{Appendices~#3#1#4--#5#2#6}
\crefmultiformat{appendix}{Appendices~#2#1#3}{ and~#2#1#3}{, #2#1#3}{ and~#2#1#3}
\Crefmultiformat{appendix}{Appendices~#2#1#3}{ and~#2#1#3}{, #2#1#3}{ and~#2#1#3}
\crefrangemultiformat{appendix}{Appendices~#3#1#4--#5#2#6}{ and~#3#1#4--#5#2#6}{, #3#1#4--#5#2#6}{ and~#3#1#4--#5#2#6}
\Crefrangemultiformat{appendix}{Appendices~#3#1#4--#5#2#6}{ and~#3#1#4--#5#2#6}{, #3#1#4--#5#2#6}{ and~#3#1#4--#5#2#6}

\crefname{part}{Part}{Parts}

\crefformat{paragraph}{\S#2#1#3}
\Crefformat{paragraph}{Paragraph~#2#1#3}
\crefrangeformat{paragraph}{\S\S#3#1#4--#5#2#6}
\Crefrangeformat{paragraph}{Paragraphs~#3#1#4--#5#2#6}
\crefmultiformat{paragraph}{\S\S#2#1#3}{ and~#2#1#3}{, #2#1#3}{ and~#2#1#3}
\Crefmultiformat{paragraph}{Paragraphs~#2#1#3}{ and~#2#1#3}{, #2#1#3}{ and~#2#1#3}
\crefrangemultiformat{paragraph}{\S\S#3#1#4--#5#2#6}{ and~#3#1#4--#5#2#6}{, #3#1#4--#5#2#6}{ and~#3#1#4--#5#2#6}
\Crefrangemultiformat{paragraph}{Paragraphs~#3#1#4--#5#2#6}{ and~#3#1#4--#5#2#6}{, #3#1#4--#5#2#6}{ and~#3#1#4--#5#2#6}

% Number subsubsections
\setcounter{secnumdepth}{5}

% Display format for figures
\crefname{figure}{Figure}{Figures}

% Use cleveref instead of hyperref's \autoref
\let\autoref\cref


%%%% EQUATION NUMBERING %%%%

% The following hack uses the single theorem counter to number
% equations as well, so that we don't have both Theorem 1.1 and
% equation (1.1).
\let\c@equation\c@thm
\numberwithin{equation}{section}


%%%% ENUMERATE NUMBERING %%%%

% Number the first level of enumerates as (i), (ii), ...
\renewcommand{\theenumi}{(\roman{enumi})}
\renewcommand{\labelenumi}{\theenumi}


%%%% MARGINS %%%%

% This is a matter of personal preference, but I think the left
% margins on enumerates and itemizes are too wide.
\setitemize[1]{leftmargin=2em}
\setenumerate[1]{leftmargin=*}

% Likewise that they are too spaced out.
\setitemize[1]{itemsep=-0.2em}
\setenumerate[1]{itemsep=-0.2em}

%%% Notes %%%
\def\noteson{%
\gdef\note##1{\mbox{}\marginpar{\color{blue}\textasteriskcentered\ ##1}}}
\gdef\notesoff{\gdef\note##1{\null}}
\noteson

\newcommand{\Coq}{\textsc{Coq}\xspace}
\newcommand{\Agda}{\textsc{Agda}\xspace}
\newcommand{\NuPRL}{\textsc{NuPRL}\xspace}

%%%% CITATIONS %%%%

% \let \cite \citep

%%%% INDEX %%%%

\newcommand{\footstyle}[1]{{\hyperpage{#1}}n} % If you index something that is in a footnote
\newcommand{\defstyle}[1]{\textbf{\hyperpage{#1}}}  % Style for pageref to a definition

\newcommand{\indexdef}[1]{\index{#1|defstyle}}   % Index a definition
\newcommand{\indexfoot}[1]{\index{#1|footstyle}} % Index a term in a footnote
\newcommand{\indexsee}[2]{\index{#1|see{#2}}}    % Index "see also"


%%%% Standard phrasing or spelling of common phrases %%%%

\newcommand{\ZF}{Zermelo--Fraenkel}
\newcommand{\CZF}{Constructive \ZF{} Set Theory}

\newcommand{\LEM}[1]{\ensuremath{\mathsf{LEM}_{#1}}\xspace}
\newcommand{\choice}[1]{\ensuremath{\mathsf{AC}_{#1}}\xspace}

%%%% MISC %%%%

\newcommand{\mentalpause}{\medskip} % Use for "mental" pause, instead of \smallskip or \medskip

%% Use \symlabel instead of \label to mark a pageref that you need in the index of symbols
\newcounter{symindex}
\newcommand{\symlabel}[1]{\refstepcounter{symindex}\label{#1}}

% Local Variables:
% mode: latex
% TeX-master: "hott-online"
% End:


\newcommand{\idsymbin}{=}

%%%%%%%%%%%%%%%%%%%%%%%%%%%%%%%%%%%%%%%%%%%%%%%%%%%%%%%%%%%%%%%%%%%%%%%%%%%%%%%%
%%%% Our commands which are not part of the macros.tex file.
%%%% We should keep these commands separate, because we will
%%%% update the macros.tex following the updates of the book.

%%%% First we redefine the \id, \eqv and \ct commands so that they accept an
%%%% arbitrary number of arguments. This is useful when writing longer strings
%%%% of equalities or equivalences.

\makeatletter

\renewcommand{\id}[3][]{
  \@ifnextchar\bgroup
    {#2 \mathbin{\idsym_{#1}} \id[#1]{#3}}
    {#2 \mathbin{\idsym_{#1}} #3}
  }

\renewcommand{\eqv}[2]{
  \@ifnextchar\bgroup
    {#1 \eqvsym \eqv{#2}}
    {#1 \eqvsym #2}
  }

\newcommand{\ctsym}{%
  \mathchoice{\mathbin{\raisebox{0.5ex}{$\displaystyle\centerdot$}}}%
             {\mathbin{\raisebox{0.5ex}{$\centerdot$}}}%
             {\mathbin{\raisebox{0.25ex}{$\scriptstyle\,\centerdot\,$}}}%
             {\mathbin{\raisebox{0.1ex}{$\scriptscriptstyle\,\centerdot\,$}}}
  }

\renewcommand{\ct}[3][]{
  \@ifnextchar\bgroup
    {#2 \mathbin{\ctsym_{#1}} \ct[#1]{#3}}
    {#2 \mathbin{\ctsym_{#1}} #3}
  }

\makeatother

%%%% We always use textstyle products and sums...
%\renewcommand{\prd}{\tprd}
%\renewcommand{\sm}{\tsm}
\makeatletter
\renewcommand{\@dprd}{\@tprd}
\renewcommand{\@dsm}{\@tsm}
\renewcommand{\@dprd@noparens}{\@tprd}
\renewcommand{\@dsm@noparens}{\@tsm}

%%%% ...with a bit more spacing
\renewcommand{\@tprd}[1]{\mathchoice{{\textstyle\prod_{(#1)}\,}}{\prod_{(#1)}\,}{\prod_{(#1)}\,}{\prod_{(#1)}\,}}
\renewcommand{\@tsm}[1]{\mathchoice{{\textstyle\sum_{(#1)}\,}}{\sum_{(#1)}\,}{\sum_{(#1)}\,}{\sum_{(#1)}\,}}

%%%%%%%%%%%%%%%%%%%%%%%%%%%%%%%%%%%%%%%%%%%%%%%%%%%%%%%%%%%%%%%%%%%%%%%%%%%%%%%%
%%%% We adjust the \prd command so that implicit arguments become possible.
%%%%
%%%% First, we have the following switch. Set it to true if implicit arguments
%%%% are desired, or to false if not. Note turning off implicit arguments
%%%% might render some parts of the text harder to comprehend, since in the
%%%% text might appear $f(x)$ where we would have $f(i,x)$ without implicit
%%%% arguments.

\newcommand{\implicitargumentson}{\boolean{true}}

%%%% If one wants to use implicit arguments in the notation for product types,
%%%% a * has to be put before the argument that has to be implicit.
%%%% For example: in $\prd{x:A}*{y:B}{u:P(y)}Q(x,y,u)$, the argument y is
%%%% implicit. Any of the arguments can be made implicit this way.

%%%% First of all, we should make the command \prd search not only for a
%%%% brace, but also for a star. We introduce an auxiliary command that
%%%% determines whether the next character is a star or brace.
\newcommand{\@ifnextchar@starorbrace}[2]
%  {\@ifnextcharamong{#1}{#2}{*}{\bgroup};}
  {\@ifnextchar*{#1}{\@ifnextchar\bgroup{#1}{#2}}}
  
%%%% When encountering the \prd command, latex should determine whether it
%%%% should print implicit argument brackets or not. So the first branching
%%%% happens right here.
\renewcommand{\prd}{\@ifnextchar*{\@iprd}{\@prd}}

\newcommand{\@prd}[1]
  {\@ifnextchar@starorbrace
    {\prd@parens{#1}}
    {\@ifnextchar\sm{\prd@parens{#1}\@eatsm}{\prd@noparens{#1}}}}
\newcommand{\@prd@parens}{\@ifnextchar*{\@iprd}{\prd@parens}}
\renewcommand{\prd@parens}[1]
  {\@ifnextchar@starorbrace
    {\@theprd{#1}\@prd@parens}
    {\@ifnextchar\sm{\@theprd{#1}\@eatsm}{\@theprd{#1}}}}
\newcommand{\@theprd}[1]
  {\mathchoice{\@dprd{#1}}{\@tprd{#1}}{\@tprd{#1}}{\@tprd{#1}}}
\renewcommand{\dprd}[1]{\@dprd{#1}\@ifnextchar@starorbrace{\dprd}{}}
\renewcommand{\tprd}[1]{\@tprd{#1}\@ifnextchar@starorbrace{\tprd}{}}

%%%% Here we tell the actual symbols to be printed.
\newcommand{\@theiprd}[1]{\mathchoice{\@diprd{#1}}{\@tiprd{#1}}{\@tiprd{#1}}{\@tiprd{#1}}}
\newcommand{\@iprd}[2]{\@ifnextchar@starorbrace%
  {\@theiprd{#2}\@prd@parens}%
  {\@ifnextchar\sm%
    {\@theiprd{#2}\@eatsm}%
    {\@theiprd{#2}}}}
\def\@tiprd#1{
  \ifthenelse{\implicitargumentson}
    {\@@tiprd{#1}\@ifnextchar\bgroup{\@tiprd}{}}
    {\@tprd{#1}}}
\def\@@tiprd#1{\mathchoice{{\textstyle\prod_{\{#1\}}\,}}{\prod_{\{#1\}}\,}{\prod_{\{#1\}}\,}{\prod_{\{#1\}}\,}}
\def\@diprd{
  \ifthenelse{\implicitargumentson}
    {\@tiprd}
    {\@tprd}}
    

%%%% And finally we need to redefine \@eatprd so that implicit arguments also
%%%% works in the scope of a dependent sum.    
\def\@eatprd\prd{\@prd@parens}

\makeatother

%%%%%%%%%%%%%%%%%%%%%%%%%%%%%%%%%%%%%%%%%%%%%%%%%%%%%%%%%%%%%%%%%%%%%%%%%%%%%%%%
%%%% Redefining the quantifiers, so that some of the longer 
%%%% formulas appear one a single line without problems

%%% Dependent products written with \forall, in the same style
\makeatletter
\def\tfall#1{\forall_{(#1)}\@ifnextchar\bgroup{\,\tfall}{\,}}
\renewcommand{\fall}{\tfall}

%%% Existential quantifier %%%
\def\texis#1{\exists_{(#1)}\@ifnextchar\bgroup{\,\texis}{\,}}
\renewcommand{\exis}{\texis}

%%% Unique existence %%%
\def\uexis#1{\exists!_{(#1)}\@ifnextchar\bgroup{\,\uexis}{\,}}
\makeatother

%%%%%%%%%%%%%%%%%%%%%%%%%%%%%%%%%%%%%%%%%%%%%%%%%%%%%%%%%%%%%%%%%%%%%%%%%%%%%%%%
%%%% UNFOLD
%%%%
%%%% For each definition in the type theory we make two versions of the macro:
%%%% the macro introducing the new notation and an @unfold version of the macro
%%%% which outputs the meaning of that new notation. Thus, we can use the
%%%% following construction to write our text. When we introduce \macro, we can
%%%% write \unfold{\macro} and the output will be the result of \macro@unfold.

\makeatletter
\newcommand{\unfold}{%
  \unfoldnext}
\newcommand{\unfoldall}[1]{%
  \begingroup%
  \renewcommand{\jhom}{\jhom@unfold}%
  \renewcommand{\jhomeq}{\jhomeq@unfold}%
  \renewcommand{\jhomdefn}{\jhomdefn@unfold}%
  \renewcommand{\jfhom}{\jfhom@unfold}%
  \renewcommand{\jcomp}{\jcomp@unfold}%
  \renewcommand{\@jcomp@nested}{\@jcomp@unfold@nested}%
  \renewcommand{\@jcomp@parens}{\@jcomp@unfold@parens}%
  \renewcommand{\tmext}{\tmext@unfold}%
  \renewcommand{\@tmext@nested}{\@tmext@unfold@nested}%
  \renewcommand{\@tmext@parens}{\@tmext@unfold@parens}%
  \renewcommand{\cprojfstf}{\cprojfstf@unfold}%
  \renewcommand{\cprojfst}{\cprojfst@unfold}%
  \renewcommand{\cprojsndf}{\cprojsndf@unfold}%
  \renewcommand{\cprojsnd}{\cprojsnd@unfold}%
  \renewcommand{\jfcomp}{\jfcomp@unfold}%
%  \renewcommand{\@jfcomp@nested}{\@jfcomp@unfold@nested}%
%  \renewcommand{\@jfcomp@parens}{\@jfcomp@unfold@parens}%
  \renewcommand{\sandwich}{\sandwich@unfold}%
  \renewcommand{\finc}{\finc@unfold}%
  \renewcommand{\jvcomp}{\jvcomp@unfold}%
  \renewcommand{\subst@type@unfold}[1]{
    \@ifnextchar\cprojfstf{\@eatdo{\cprojfstf@parens}}{%
      ##1}
    }
  #1%
  \endgroup%
  }

%%%% The following command is useful when you have checked with '\@ifnextchar'
%%%% that the next character is a macro '\firstmacro' and you want to replace
%%%% it by '\secondmacro'. To establish this, simply call for
%%%% '\@ifnextchar\firstmacro{\@eatdo{\secondmacro}}{}' with the second 
%%%% argument of \@eatdo left unspecified.
\newcommand{\@eatdo}[2]{#1}

%%%% The intention of '\unfoldnext' is to unfold only the definition of the
%%%% next character, provided that it is in the list of unfoldable macros.
\newcommand{\unfoldnext}[1]{
  \@ifnextchar\jhom{\@eatdo{\jhom@unfold}}{%
  \@ifnextchar\jhomeq{\@eatdo{\jhomeq@unfold}}{%
  \@ifnextchar\jhomdefn{\@eatdo{\jhomdefn@unfold}}{%
  \@ifnextchar\jfhom{\@eatdo{\jfhom@unfold}}{%
  \@ifnextchar\jcomp{\@eatdo{\jcomp@unfold}}{%
  \@ifnextchar\@jcomp@nested{\@eatdo{\@jcomp@unfold@nested}}{%
  \@ifnextchar\@jcomp@parens{\@eatdo{\@jcomp@unfold@parens}}{%
  \@ifnextchar\tmext{\@eatdo{\tmext@unfold}}{%
  \@ifnextchar\@tmext@nested{\@eatdo{\@tmext@unfold@nested}}{%
  \@ifnextchar\@tmext@parens{\@eatdo{\@tmext@unfold@parens}}{%
  \@ifnextchar\cprojfstf{\@eatdo{\cprojfstf@unfold}}{%
  \@ifnextchar\cprojfst{\@eatdo{\cprojfst@unfold}}{%
  \@ifnextchar\cprojsndf{\@eatdo{\cprojsndf@unfold}}{%
  \@ifnextchar\cprojsnd{\@eatdo{\cprojsnd@unfold}}{%
  \@ifnextchar\jfcomp{\@eatdo{\jfcomp@unfold}}{%
%  \@ifnextchar\@jfcomp@nested{\@eatdo{\@jfcomp@unfold@nested}}{%
%  \@ifnextchar\@jfcomp@parens{\@eatdo{\@jfcomp@unfold@parens}}{%
  \@ifnextchar\sandwich{\@eatdo{\sandwich@unfold}}{%
  \@ifnextchar\finc{\@eatdo{\finc@unfold}}{%
  \@ifnextchar\jvcomp{\@eatdo{\jvcomp@unfold}}}
  #1}
\makeatother

%%%%%%%%%%%%%%%%%%%%%%%%%%%%%%%%%%%%%%%%%%%%%%%%%%%%%%%%%%%%%%%%%%%%%%%%%%%%%%%%
%%%% A PRETTY PRINTER
%%%%
%%%% We write a \pretty command that pretty prints judgments or types by
%%%% diplaying variables and omitting explicit notation for weakening.
%%%%
%%%% This command should work similar to the \unfold command
%%%%
%%%% -- UNDER CONSTRUCTION

\makeatletter
\newcommand{\vardis}[2]{\@vardis@type #2{}(\@vardis@term #1)}
\newcommand{\@vardis}{\@ifnextchar\bgroup{\@@vardis}{}}
\newcommand{\@@vardis}[1]{\@ifnextchar\bgroup{\vardis{#1}}{#1}}
\newcommand{\@vardis@term}{\@vardis}
\newcommand{\@vardis@type}{\@ifnextchar\ctxext{\@ctxext@nested}{\@ifnextchar\ctxwk{\@ctxwk@nested}{\@vardis}}}
\newcommand{\@vardis@nested}[3]{\@vardis@parens{#2}{#3}}
\newcommand{\@vardis@parens}[2]{(\vardis{#1}{#2})}
\makeatother

\makeatletter
\newcommand{\jvctx}{\jctx}
\newcommand{\jvctxeq}{\jctxeq}

\newcommand{\cctxextcombi}[2]{\@ifnextchar\bgroup{\@cctxextcombi #1}{#1:}#2}
\newcommand{\@cctxextcombi}[4]{\cctxext{{\cctxextcombi{#1}{#3}}{\@@cctxextcombi{#1}{#2}{#4}}}}
\newcommand{\@@cctxextcombi}[3]{\@ifnextchar\bgroup{\@@@ctxextcombi #2}{#2(#1):}#3(\cctxext{#1})}
\newcommand{\@@@ctxextcombi}[8] % the 5th argument is (, the 6th is \cctxext and the 8th is ).
  {\@@ctxextcombi{#7}{#1}{#3},\@@ctxextcombi{{#7}{#3}}{#2}{#4}}
\newcommand{\cctxext}[1]{\@ifnextchar\bgroup{\@cctxext}{}#1}
\newcommand{\@cctxext}[2]{\cctxext{#1},\cctxext{#2}}

\newcommand{\jvfamcombi}[3]{
  \cctxextcombi{#1}{#2} \vdash \vardis{\cctxext{#1}}{#3}
}

\newcommand{\jvfam}{\@ifnextchar*{\@jvfamAlignTrue}{\@jvfamAlignFalse}}
\newcommand{\@jvfamAlignTrue}[4]{\jfam*{#2:#3}{\vardis{#2}{#4}}}
\newcommand{\@jvfamAlignFalse}[3]{\jfam{#1:#2}{\vardis{#1}{#3}}\quad@test}

\newcommand{\jvfameq}{\@ifnextchar*{\@jvfameqAlignTrue}{\@jvfameqAlignFalse}}
\newcommand{\@jvfameqAlignTrue}[5]{\jfameq*{#2:#3}{\vardis{#2}{#4}}{\vardis{#2}{#5}}}
\newcommand{\@jvfameqAlignFalse}[4]{\jfameq{#1:#2}{\vardis{#1}{#3}}{\vardis{#1}{#4}}\quad@test}

\newcommand{\jvtype}{\@ifnextchar*{\@jvtypeAlignTrue}{\@jvtypeAlignFalse}}
\newcommand{\@jvtypeAlignTrue}[4]{\jtype*{#2:#3}{\vardis{#2}{#4}}}
\newcommand{\@jvtypeAlignFalse}[3]{\jtype{#1:#2}{\vardis{#1}{#3}}\quad@test}

\newcommand{\jvtypeeq}{\@ifnextchar*{\@jvtypeeqAlignTrue}{\@jvtypeeqAlignFalse}}
\newcommand{\@jvtypeeqAlignTrue}[5]{\jtypeeq*{#2:#3}{\vardis{#2}{#4}}{\vardis{#2}{#5}}}
\newcommand{\@jvtypeeqAlignFalse}[4]{\jtypeeq{#1:#2}{\vardis{#1}{#3}}{\vardis{#1}{#4}}\quad@test}

\newcommand{\jvterm}{\@ifnextchar*{\@jvtermAlignTrue}{\@jvtermAlignFalse}}
\newcommand{\@jvtermAlignTrue}[5]{\jterm*{#2:#3}{\vardis{#2}{#4}}{\vardis{#2}{#5}}}
\newcommand{\@jvtermAlignFalse}[4]{\jterm{#1:#2}{\vardis{#1}{#3}}{\vardis{#1}{#4}}\quad@test}

\newcommand{\jvtermeq}{\@ifnextchar*{\@jvtermeqAlignTrue}{\@jvtermeqAlignFalse}}
\newcommand{\@jvtermeqAlignTrue}[6]{\jtermeq*{#2:#3}{\vardis{#2}{#4}}{\vardis{#2}{#5}}{\vardis{#2}{#6}}}
\newcommand{\@jvtermeqAlignFalse}[5]{\jtermeq{#1:#2}{\vardis{#1}{#3}}{\vardis{#1}{#4}}{\vardis{#1}{#5}}\quad@test}
\makeatother

%%%%%%%%%%%%%%%%%%%%%%%%%%%%%%%%%%%%%%%%%%%%%%%%%%%%%%%%%%%%%%%%%%%%%%%%%%%%%%%%
%%%%

\newcommand{\famsym}{\mathcal{F}}
\newcommand{\tmsym}{\mathcal{T}}

%%%%%%%%%%%%%%%%%%%%%%%%%%%%%%%%%%%%%%%%%%%%%%%%%%%%%%%%%%%%%%%%%%%%%%%%%%%%%%%%
%%%% JUDGMENTS
%%%%
%%%% Below we define several commands for the judgments of type theory. There
%%%% are commands
%%%% * \jctx for the judgment that something is a context.
%%%% * \jctxeq for the judgment that two contexts are the same
%%%% * \jtype for the judgment that something is a type in a context
%%%% * \jtypeeq for the judgment that two types in the same context are the same
%%%% * \jterm for the judgment that something is a term of a type in a context
%%%% * \jtermeq for the judgment that two terms of the same type are the same

\makeatletter
% We first make a generic judgment command
\newcommand{\judgment}{\@ifnextchar*{\@judgmentAT}{\@judgmentAF}}
\newcommand{\@judgmentAT}[8]{\@judgment@ctx{#2} & \vdash \@judgment@rel{#3}{#4}{#5}{#6}{#7} #8}
\newcommand{\@judgmentAF}[7]{\@judgment@ctx{#1} \vdash \@judgment@rel{#2}{#3}{#4}{#5}{#6} #7\quad@test}
\newcommand{\@judgment@ctx}{\@judgment@ext}
\newcommand{\@judgment@rel}[5]{
  { \default@ctxext #1
    }
  #2 
  { \default@ctxext #3
    }
  #4
  { \default@ctxext #5
    }}
\newcommand{\@judgment@kind}[1]{~~\textit{#1}}
\newcommand{\@judgment@ext}[1]{\default@ctxext #1}

\newcommand{\quad@test}{%
  \@ifnextchar\jctx{\quad}{%
  \@ifnextchar\jctxeq{\quad}{%
  \@ifnextchar\jvctx{\quad}{%
  \@ifnextchar\jvctxeq{\quad}{%
  \@ifnextchar\jfam{\quad}{%
  \@ifnextchar\jfameq{\quad}{%
  \@ifnextchar\jvfam{\quad}{%
  \@ifnextchar\jvfameq{\quad}{%
  \@ifnextchar\jtype{\quad}{%
  \@ifnextchar\jtypeeq{\quad}{%
  \@ifnextchar\jvtype{\quad}{%
  \@ifnextchar\jvtypeeq{\quad}{%
  \@ifnextchar\jterm{\quad}{%
  \@ifnextchar\jtermeq{\quad}{%
  \@ifnextchar\jvterm{\quad}{%
  \@ifnextchar\jvtermeq{\quad}{%
  \@ifnextchar\jhom{\quad}{%
  \@ifnextchar\jhomeq{\quad}{%
  \@ifnextchar\jfhom{\quad}{%
  \@ifnextchar\jfhomeq{\quad}{%
  }}}}}}}}}}}}}}}}}}}}}

%%%% Judgments about contexts
\newcommand{\jctx@sym}{\@judgment@kind{ctx}}

\newcommand{\jctx}{\@ifnextchar*{\@jctxAlignTrue}{\@jctxAlignFalse}}
\newcommand{\@jctxAlignTrue}[2]{\judgment*{}{}{}{}{}{#2}{\jctx@sym}}
\newcommand{\@jctxAlignFalse}[1]{\judgment{}{}{}{}{}{#1}{\jctx@sym}}

\newcommand{\jctxeq}{\@ifnextchar*{\@jctxeqAlignTrue}{\@jctxeqAlignFalse}}
\newcommand{\@jctxeqAlignTrue}[3]{\judgment*{}{#2}{\jdeq}{#3}{}{}{\jctx@sym}}
\newcommand{\@jctxeqAlignFalse}[2]{\judgment{}{#1}{\jdeq}{#2}{}{}{\jctx@sym}}

\newcommand{\jctxdefn}{\@ifnextchar*{\@jctxdefnAlignTrue}{\@jctxdefnAlignFalse}}
\newcommand{\@jctxdefnAlignTrue}[3]{\judgment*{}{#2}{\defeq}{#3}{}{}{\jctx@sym}}
\newcommand{\@jctxdefnAlignFalse}[2]{\judgment{}{#1}{\defeq}{#2}{}{}{\jctx@sym}}

%%%% Judgments about families
\newcommand{\jfam@sym}{\@judgment@kind{fam}}

\newcommand{\jfam}{\@ifnextchar*{\@jfamAlignTrue}{\@jfamAlignFalse}}
\newcommand{\@jfamAlignTrue}[3]{\judgment*{#2}{}{}{}{}{#3}{\jfam@sym}}
\newcommand{\@jfamAlignFalse}[2]{\judgment{#1}{}{}{}{}{#2}{\jfam@sym}}

\newcommand{\jfameq}{\@ifnextchar*{\@jfameqAlignTrue}{\@jfameqAlignFalse}}
\newcommand{\@jfameqAlignTrue}[4]{\judgment*{#2}{#3}{\jdeq}{#4}{}{}{\jfam@sym}}
\newcommand{\@jfameqAlignFalse}[3]{\judgment{#1}{#2}{\jdeq}{#3}{}{}{\jfam@sym}}

\newcommand{\jfamdefn}{\@ifnextchar*{\@jfamdefnAlignTrue}{\@jfamdefnAlignFalse}}
\newcommand{\@jfamdefnAlignTrue}[4]{\judgment*{#2}{#3}{\defeq}{#4}{}{}{\jfam@sym}}
\newcommand{\@jfamdefnAlignFalse}[3]{\judgment{#1}{#2}{\defeq}{#3}{}{}{\jfam@sym}}
  
%%%% Judgments about types
\newcommand{\jtype@sym}{\@judgment@kind{type}}
\newcommand{\jtype}{\@ifnextchar*{\@jtypeAlignTrue}{\@jtypeAlignFalse}}
\newcommand{\@jtypeAlignTrue}[3]{\judgment*{#2}{}{}{}{}{#3}{\jtype@sym}}
\newcommand{\@jtypeAlignFalse}[2]{\judgment{#1}{}{}{}{}{#2}{\jtype@sym}}
  
\newcommand{\jtypeeq}{\@ifnextchar*{\@jtypeeqAlignTrue}{\@jtypeeqAlignFalse}}
\newcommand{\@jtypeeqAlignTrue}[4]{\judgment*{#2}{#3}{\jdeq}{#4}{}{}{\jtype@sym}}
\newcommand{\@jtypeeqAlignFalse}[3]{\judgment{#1}{#2}{\jdeq}{#3}{}{}{\jtype@sym}}

\newcommand{\jtypedefn}{\@ifnextchar*{\@jtypedefnAlignTrue}{\@jtypedefnAlignFalse}}
\newcommand{\@jtypedefnAlignTrue}[4]{\judgment*{#2}{#3}{\defeq}{#4}{}{}{\jtype@sym}}
\newcommand{\@jtypedefnAlignFalse}[3]{\judgment{#1}{#2}{\defeq}{#3}{}{}{\jtype@sym}}
  
%%%% Judgments about terms
\newcommand{\jterm}{\@ifnextchar*{\@jtermAlignTrue}{\@jtermAlignFalse}}
\newcommand{\@jtermAlignTrue}[4]{\judgment*{#2}{}{}{#4}{:}{#3}{}}
\newcommand{\@jtermAlignFalse}[3]{\judgment{#1}{}{}{#3}{:}{#2}{}}

\newcommand{\jtermeq}{\@ifnextchar*{\@jtermeqAlignTrue}{\@jtermeqAlignFalse}}
\newcommand{\@jtermeqAlignTrue}[5]{\judgment*{#2}{#4}{\jdeq}{#5}{:}{#3}{}}
\newcommand{\@jtermeqAlignFalse}[4]{\judgment{#1}{#3}{\jdeq}{#4}{:}{#2}{}}

\newcommand{\jtermdefn}{\@ifnextchar*{\@jtermdefnAlignTrue}{\@jtermdefnAlignFalse}}
\newcommand{\@jtermdefnAlignTrue}[5]{\judgment*{#2}{#4}{\defeq}{#5}{:}{#3}{}}
\newcommand{\@jtermdefnAlignFalse}[4]{\judgment{#1}{#3}{\defeq}{#4}{:}{#2}{}}
\makeatother

%%%%%%%%%%%%%%%%%%%%%%%%%%%%%%%%%%%%%%%%%%%%%%%%%%%%%%%%%%%%%%%%%%%%%%%%%%%%%%%%
%%%% THE EMPTY CONTEXT

\newcommand{\emptysym}{[\;]}
\newcommand{\emptyc}{{\emptysym}}
\newcommand{\emptyf}[1][]{{\emptysym}_{#1}}
\newcommand{\emptytm}[1][]{\typefont{\#}_{#1}}

%%%%%%%%%%%%%%%%%%%%%%%%%%%%%%%%%%%%%%%%%%%%%%%%%%%%%%%%%%%%%%%%%%%%%%%%%%%%%%%%
%%%% CONTEXT EXTENSION 
%%%%
%%%% The context extension command.
%%%%
%%%% To get a feeling of how the command works, here are a few examples.
%%%% \ctxext{A}{B} will print A.B
%%%% \ctxext{{A}{B}}{C} will print (A.B).C
%%%% \ctxext{{{A}{B}}{C}}{{D}{E}} will print ((A.B).C).(D.E)

\makeatletter
\newcommand{\ctxext}[2]{\@ctxext@ctx #1.\@ctxext@type #2}
\newcommand{\@ctxext}{\@ifnextchar\bgroup{\@@ctxext}{}}
\newcommand{\@ctxext@ctx}{%
  \@ifnextchar\ctxext{\@ctxext@nested}{%
  \@ifnextchar\ctxwk{\@ctxwk@nested}{%
  \@ifnextchar\jcomp{\@jcomp@nested}{%
  \@ifnextchar\jvcomp{\@jvcomp@nested}{%
  \@ifnextchar\jfcomp{\@jfcomp@nested}{%
  \@ctxext}}}}}}
\newcommand{\@ctxext@type}{%
  \@ifnextchar\ctxext{\@ctxext@nested}{%
  \@ifnextchar\subst{\@subst@nested}{%
  \@ifnextchar\jcomp{\@jcomp@nested}{%
  \@ifnextchar\jvcomp{\@jvcomp@nested}{%
  \@ifnextchar\jfcomp{\@jfcomp@nested}{%
  \@ctxext}}}}}}
\newcommand{\@@ctxext}[1]{\@ifnextchar\bgroup{\@ctxext@parens{#1}}{#1}}
\newcommand{\@ctxext@parens}[2]{(\ctxext{#1}{#2})}
\newcommand{\@ctxext@nested}[3]{\@ctxext@parens{#2}{#3}}

%%%% We want that some commands accept binary trees as arguments that default
%%%% into extensions. We make the following command to realize this
\newcommand{\default@ctxext}{\@ifnextchar\bgroup{\ctxext}{}}
\newcommand{\default@ctxext@parens}{\@ifnextchar\bgroup{\@ctxext@parens}{}}
\makeatother

%%%%%%%%%%%%%%%%%%%%%%%%%%%%%%%%%%%%%%%%%%%%%%%%%%%%%%%%%%%%%%%%%%%%%%%%%%%%%%%%
%%%% SUBSTITUTION

%%%% The substitution command will act the following way
%%%%
%%%% \subst{x}{P} will print P[x]
%%%% \subst{x}{{f}{Q}} will print Q[f][x]
%%%% \subst{{x}{f}}{{x}{Q}} will print Q[x][f[x]]

\makeatletter
\newcommand{\subst}[3][]{%
  \@subst@type #3{}[\@subst@term #2]^{#1}}
\newcommand{\@subst}{%
  \@ifnextchar\bgroup{\@@subst}{}}
\newcommand{\@@subst}[1]{%
  \@ifnextchar\bgroup{\subst{#1}}{#1}}
\newcommand{\@subst@term}{%
  \@subst}
\newcommand{\@subst@type}{%
  \@ifnextchar\ctxext{\@ctxext@nested}{%
  \@ifnextchar\ctxwk{\@ctxwk@nested}{%
  \@ifnextchar\jcomp{\@jcomp@nested}{%
  \@ifnextchar\tmext{\@tmext@nested}{%
  \@ifnextchar\jvcomp{\@jvcomp@nested}{%
  \@ifnextchar\jfcomp{\@jfcomp@nested}{%
%  \@ifnextchar\mfam{\@mfam@nested}{%
%  \@ifnextchar\mtm{\@mtm@nested}}
\newcommand{\subst@type@unfold}[1]{#1}
\newcommand{\@subst@nested}[3]{%
  \@subst@parens{#2}{#3}}
\newcommand{\@subst@parens}[2]{%
  (\subst{#1}{#2})}
\makeatother

%%%%%%%%%%%%%%%%%%%%%%%%%%%%%%%%%%%%%%%%%%%%%%%%%%%%%%%%%%%%%%%%%%%%%%%%%%%%%%%%
%%%% WEAKENING

%%%% The weakening command is very much like the substitution command.

\makeatletter
\newcommand{\ctxwk}[3][]{%
  \langle\@ctxwk@act #2\rangle^{#1} \@ctxwk@pass #3}
\newcommand{\@ctxwk}{%
  \@ifnextchar\bgroup{\@@ctxwk}{}}
\newcommand{\@@ctxwk}[1]{%
  \@ifnextchar\bgroup{\ctxwk{#1}}{#1}}
\newcommand{\@ctxwk@act}{%
  \@ctxwk}
\newcommand{\@ctxwk@pass}{%
  \@ifnextchar\ctxext{\@ctxext@nested}{%
  \@ifnextchar\subst{\@subst@nested}{%
  \@ifnextchar\jcomp{\@jcomp@nested}{%
  \@ifnextchar\tmext{\@tmext@nested}{%
  \@ifnextchar\jvcomp{\@jvcomp@nested}{%
  \@ifnextchar\jfcomp{\@jfcomp@nested}{%
%  \@ifnextchar\mfam{\@mfam@nested}{%
%  \@ifnextchar\mtm{\@mtm@nested}}
\newcommand{\@ctxwk@parens}[2]{%
  (\ctxwk{#1}{#2})}
\newcommand{\@ctxwk@nested}[3]{%
  \@ctxwk@parens{#2}{#3}}
\makeatother

%%%% Not sure if we're gonna need the following.
\newcommand{\ctxwkop}[2]{%
  \ctxwk{#2}{#1}}
  
%%%%%%%%%%%%%%%%%%%%%%%%%%%%%%%%%%%%%%%%%%%%%%%%%%%%%%%%%%%%%%%%%%%%%%%%%%%%%%%%
%%%% IDENTITY TERMS

\makeatletter
\newcommand{\idtm}[1]{\typefont{id}_{\default@ctxext #1}}
\makeatother

%%%%%%%%%%%%%%%%%%%%%%%%%%%%%%%%%%%%%%%%%%%%%%%%%%%%%%%%%%%%%%%%%%%%%%%%%%%%%%%%
%%%% TERM EXTENSION
%%%%
%%%% The term extension command \tmext is slightly complicated because 
%%%% \tmext@unfold should do different things depending on whether it has two
%%%% or four arguments. Thus \tmext has a full form and a short form, where
%%%% the short form has two arguments and the full form has four. 

\makeatletter

%%%% The basic term extension commands
\newcommand{\default@tmext}{\@ifnextchar\bgroup{\tmext}{}}
\newcommand{\tmext}[2]{%
  \@ifnextchar\bgroup{\tmext@full{#1}{#2}}{\tmext@short{#1}{#2}}}
\newcommand{\tmext@full}[4]{%
  \ctxext{\tmext@testleft #3}{\tmext@testright #4}}
\newcommand{\tmext@short}[2]{%
  \ctxext{\tmext@testleft #1}{\tmext@testright #2}}
\newcommand{\tmext@testleft}{%
  \@ifnextchar\bgroup{\@tmext@parens}{%
  \@ifnextchar\tmext{\@tmext@nested}{%
  \@ifnextchar\ctxwk{\@ctxwk@nested}{%
  \@ifnextchar\jcomp{\@jcomp@nested}{%
  \@ifnextchar\jvcomp{\@jvcomp@nested}{%
  \@ifnextchar\jfcomp{\@jfcomp@nested}{%
%  \default@tmext
  }}}}}}}
\newcommand{\tmext@testright}{%
  \@ifnextchar\bgroup{\@tmext@parens}{%
  \@ifnextchar\tmext{\@tmext@nested}{%
  \@ifnextchar\subst{\@subst@nested}{%
  \@ifnextchar\jcomp{\@jcomp@nested}{%
  \@ifnextchar\jvcomp{\@jvcomp@nested}{%
  \@ifnextchar\jfcomp{\@jfcomp@nested}{%
  \@ifnextchar\cprojfst{\cprojfst@nested}{%
  \@ifnextchar\cprojsnd{\cprojsnd@nested}{%
%  \default@tmext
  }}}}}}}}}
\newcommand{\@tmext@nested}[1]{%
  \@tmext@parens}
\newcommand{\@tmext@parens}[2]{%
  \@ifnextchar\bgroup
    {\tmext@full@parens{#1}{#2}}
    {(\tmext@short{#1}{#2})}}
\newcommand{\tmext@full@parens}[4]{%
  (\tmext@full{#1}{#2}{#3}{#4})}

%%%% The unfolded term extension commands
\newcommand{\tmext@unfold}[2]{%
  \@ifnextchar\bgroup{\tmext@unfold@full{#1}{#2}}{\tmext@short{#1}{#2}}}
\newcommand{\tmext@unfold@full}[4]{%  
  \subst{#4}{{#3}{\idtm{\ctxext{#1}{#2}}}}}
\newcommand{\@tmext@unfold@nested}[1]{%
  \@tmext@unfold@parens}
\newcommand{\@tmext@unfold@parens}[4]{%
  (\tmext@unfold{#1}{#2}{#3}{#4})}
\makeatother

%%%%%%%%%%%%%%%%%%%%%%%%%%%%%%%%%%%%%%%%%%%%%%%%%%%%%%%%%%%%%%%%%%%%%%%%%%%%%%%%
%%%% JUDGMENTAL MORPHISMS

\makeatletter

%%%% The judgment that f is a morphism from A to B in context \Gamma.
\newcommand{\jhomsym}[3][]{%
  ~~\textit{hom}_{#1}(\default@ctxext #2,\default@ctxext #3)}
\newcommand{\jhom}{%
  \@ifnextchar*{\@jhomAlignTrue}{\@jhomAlignFalse}}
\newcommand{\@jhomAlignTrue}[5]{%
  \judgment*{#2}{}{}{#5}{}{}{\jhomsym{#3}{#4}}}
\newcommand{\@jhomAlignFalse}[4]{%
  \judgment{#1}{}{}{#4}{}{}{\jhomsym{#2}{#3}}}
\newcommand{\jhomeq}{%
  \@ifnextchar*{\@jhomeqAlignTrue}{\@jhomeqAlignFalse}}
\newcommand{\@jhomeqAlignTrue}[6]{%
  \judgment*{#2}{#5}{\jdeq}{#6}{}{}{\jhomsym{#3}{#4}}}
\newcommand{\@jhomeqAlignFalse}[5]{%
  \judgment{#1}{#4}{\jdeq}{#5}{}{}{\jhomsym{#2}{#3}}}
\newcommand{\jhomdefn}{%
  \@ifnextchar*{\@jhomdefnAlignTrue}{\@jhomdefnAlignFalse}}
\newcommand{\@jhomdefnAlignTrue}[6]{%
  \judgment*{#2}{#5}{\defeq}{#6}{}{}{\jhomsym{#3}{#4}}}
\newcommand{\@jhomdefnAlignFalse}[5]{%
  \judgment{#1}{#4}{\defeq}{#5}{}{}{\jhomsym{#2}{#3}}}

\newcommand{\jhom@unfold}[4]{%
  \jterm
    {{#1}{#2}}
    {\ctxwk{\default@ctxext #2}{\default@ctxext@parens #3}}
    {#4}}
\newcommand{\jhomeq@unfold}[5]{%
  \jtermeq
    {{#1}{#2}}
    {\ctxwk{\default@ctxext #2}{\default@ctxext@parens #3}}
    {#4}
    {#5}}
\newcommand{\jhomdefn@unfold}[5]{%
  \jtermdefn
    {{#1}{#2}}
    {\ctxwk{\default@ctxext #2}{\default@ctxext@parens #3}}
    {#4}
    {#5}}

%%%% Composition of morphisms
\newcommand{\jcomp}[3]{%
  \jcomp@testleft #3 \circ \jcomp@testright #2}
\newcommand{\jcomp@testleft}{%
  \@ifnextchar\jcomp{\@jcomp@nested}{%
  \@ifnextchar\ctxwk{\@ctxwk@nested}{%
  \@ifnextchar\ctxext{\@ctxext@nested}{%
  \@ifnextchar\bgroup{\@jcomp@parens}{%
  \@ifnextchar\tmext{\@tmext@nested}{%
  \@ifnextchar\jvcomp{\@jvcomp@nested}{%
  \@ifnextchar\jfcomp{\@jfcomp@nested}{%
  }}}}}}}}
\newcommand{\jcomp@testright}{%
  \@ifnextchar\jcomp{\@jcomp@nested}{%
  \@ifnextchar\subst{\@subst@nested}{%
  \@ifnextchar\ctxext{\@ctxext@nested}{%
  \@ifnextchar\bgroup{\@jcomp@parens}{%
  \@ifnextchar\tmext{\@tmext@nested}{%
  \@ifnextchar\jvcomp{\@jvcomp@nested}{%
  \@ifnextchar\jfcomp{\@jfcomp@nested}{%
  }}}}}}}}
\newcommand{\@jcomp@nested}[4]{%
  \@jcomp@parens{#2}{#3}{#4}}
\newcommand{\@jcomp@parens}[3]{%
  (\jcomp{#1}{#2}{#3})}

\newcommand{\jcomp@unfold}[3]{%
  \subst
    {\jcomp@unfold@test@preside #2}
    {\ctxwk{\default@ctxext #1}{\jcomp@unfold@test@postside #3}}}
\newcommand{\jcomp@unfold@test@preside}{%
  \@ifnextchar\bgroup{\@jcomp@unfold@parens}{}}
\newcommand{\jcomp@unfold@test@postside}{%
  \@ifnextchar\bgroup{\@jcomp@unfold@parens}{%
  \@ifnextchar\subst{\@subst@nested}{%
  }}}
\newcommand{\@jcomp@unfold@nested}[4]{%
  \@jcomp@unfold@parens{#2}{#3}{#4}}
\newcommand{\@jcomp@unfold@parens}[3]{%
  (\jcomp@unfold{#1}{#2}{#3})}

%%%% Vertical composition of morphisms.
\newcommand{\jvcomp}[3]{%
  \jcomp@testleft #2 * \jcomp@testright #3}
\newcommand{\jvcomp@testleft}{%
  \@ifnextchar\jvcomp{\@jvcomp@nested}{%
  \@ifnextchar\ctxwk{\@ctxwk@nested}{%
  \@ifnextchar\ctxext{\@ctxext@nested}{%
  \@ifnextchar\bgroup{\@jvcomp@parens}{%
  \@ifnextchar\tmext{\@tmext@nested}{%
  \@ifnextchar\jcomp{\@jcomp@nested}{%
  \@ifnextchar\jfcomp{\@jfcomp@nested}{%
  }}}}}}}}
\newcommand{\jvcomp@testright}{%
  \@ifnextchar\jvcomp{\@jvcomp@nested}{%
  \@ifnextchar\subst{\@subst@nested}{%
  \@ifnextchar\ctxext{\@ctxext@nested}{%
  \@ifnextchar\bgroup{\@jvcomp@parens}{%
  \@ifnextchar\tmext{\@tmext@nested}{%
  \@ifnextchar\jcomp{\@jcomp@nested}{%
  \@ifnextchar\jfcomp{\@jfcomp@nested}{%
  }}}}}}}}
\newcommand{\@jvcomp@nested}[4]{%
  \@jvcomp@parens{#2}{#3}{#4}}
\newcommand{\@jvcomp@parens}[3]{%
  (\jvcomp{#1}{#2}{#3})}

\newcommand{\jvcomp@unfold}[3]{%
  \tmext{}{}{\ctxwk{#1}{#2}}{#3}
  }
\newcommand{\jvcomp@unfold@test@preside}{%
  \@ifnextchar\bgroup{\@jvcomp@unfold@parens}{}}
\newcommand{\jvcomp@unfold@test@postside}{%
  \@ifnextchar\bgroup{\@jvcomp@unfold@parens}{}}
\newcommand{\@jvcomp@unfold@nested}[4]{%
  \@jvcomp@unfold@parens{#2}{#3}{#4}}
\newcommand{\@jvcomp@unfold@parens}[3]{%
  (\jvcomp@unfold{#1}{#2}{#3})}

%%%% The judgment that F is a morphism from P to Q over f in context \Gamma.
\newcommand{\jfhomsym}[3]{\jhomsym[{#1}]{#2}{#3}}
\newcommand{\jfhom}{%
  \@ifnextchar*{\jfhomAlignTrue}{\jfhomAlignFalse}}
\newcommand{\jfhomAlignTrue}[8]{
  \judgment*{#2}{}{}{#8}{}{}{\jfhomsym{#5}{#6}{#7}}}
\newcommand{\jfhomAlignFalse}[7]{
  \judgment{#1}{}{}{#7}{}{}{\jfhomsym{#4}{#5}{#6}}}
\newcommand{\jfhomeq}[8]{%
  \judgment{#1}{#7}{\jdeq}{#8}{}{}{\jhomsym[{#4}]{#5}{#6}}}
\newcommand{\jfhomdefn}[8]{%
  \judgment{#1}{#7}{\defeq}{#8}{}{}{\jhomsym[{#4}]{#5}{#6}}}
\newcommand{\jfhom@unfold}[7]{%
  \jterm
    {{{#1}{#2}}{#5}}
    {\ctxwk{\default@ctxext #5}{\jcomp{#2}{#4}{#6}}}
    {#7}}
    
\newcommand{\jfcomp}[5]{%
  \jfcomp@testleft #5 \bullet \jfcomp@testright #4}
\newcommand{\jfcomp@testleft}{%
  \@ifnextchar\bgroup{\@jfcomp@parens}{%
  \@ifnextchar\jfcomp{\@jfcomp@nested}{%
  \@ifnextchar\jcomp{\@jcomp@nested}{%
  \@ifnextchar\ctxwk{\@ctxwk@nested}{%
  \@ifnextchar\tmext{\@tmext@nested}{%
  \@ifnextchar\jvcomp{\@jvcomp@nested}{%
  }}}}}}}
\newcommand{\jfcomp@testright}{%
  \@ifnextchar\bgroup{\@jfcomp@parens}{%
  \@ifnextchar\jfcomp{\@jfcomp@nested}{%
  \@ifnextchar\jcomp{\@jcomp@nested}{%
  \@ifnextchar\subst{\@subst@nested}{%
  \@ifnextchar\tmext{\@tmext@nested}{%
  \@ifnextchar\jvcomp{\@jvcomp@nested}{%
  }}}}}}}
\newcommand{\@jfcomp@nested}[1]{%
  \@jfcomp@parens}
\newcommand{\@jfcomp@parens}[5]{%
  (\jfcomp{#1}{#2}{#3}{#4}{#5})}
  
\newcommand{\jfcomp@unfold}[5]{%
  \jcomp{#3}{#4}{{#1}{#2}{#5}}}
\makeatother

%%%%%%%%%%%%%%%%%%%%%%%%%%%%%%%%%%%%%%%%%%%%%%%%%%%%%%%%%%%%%%%%%%%%%%%%%%%%%%%%
%%%% JUDGMENTAL TRIVIAL COFIBRATIONS

\newcommand{\jtcext}{\tilde}

%%%%%%%%%%%%%%%%%%%%%%%%%%%%%%%%%%%%%%%%%%%%%%%%%%%%%%%%%%%%%%%%%%%%%%%%%%%%%%%%
%%%% CONTEXT PROJECTIONS

\makeatletter
\newcommand{\cprojgenf}[3]{%
  \typefont{pr}^{%
    \@ifnextchar\bgroup{\@ctxext@parens}{%
    \@ifnextchar\ctxext{\@ctxext@nested}{%
    }}
    #2,
    \@ifnextchar\bgroup{\@ctxext@parens}{%
    \@ifnextchar\ctxext{\@ctxext@nested}{%
    }}
    #3
    }_{#1}}
\newcommand{\cprojgen}[4]{%
  \subst{#4}{\cprojgenf{#1}{#2}{#3}}}
\newcommand{\cprojgenf@nested}[1]{%
  \cprojgenf@parens}
\newcommand{\cprojgenf@parens}[3]{%
  (\cprojgenf{#1}{#2}{#3})}
\newcommand{\cprojgen@nested}[1]{%
  \cprojgen@parens}
\newcommand{\cprojgen@parens}[4]{%
  (\cprojgen{#1}{#2}{#3}{#4})}

\newcommand{\cprojfstf}[2]{%
  \cprojgenf{0}{#1}{#2}}
\newcommand{\cprojfstf@nested}[1]{%
  \cprojfstf@parens}
\newcommand{\cprojfstf@parens}[2]{%
  (\cprojfstf{#1}{#2})}
\newcommand{\cprojfstf@unfold}[2]{%
  \ctxwk{\default@ctxext #2}\idtm{\default@ctxext #1}}

\newcommand{\cprojfst}[3]{%
  \cprojgen{0}{#1}{#2}{#3}}
\newcommand{\cprojfst@nested}[1]{%
  \cprojfst@parens}
\newcommand{\cprojfst@parens}[3]{%
  (\cprojfst{#1}{#2}{#3})}
\newcommand{\cprojfst@unfold}[3]{%
  \subst{#3}{(\cprojfstf@unfold{#1}{#2})}}

\newcommand{\cprojsndf}[2]{%
  \cprojgenf{1}{#1}{#2}}
\newcommand{\cprojsndf@nested}[1]{%
  \cprojsndf@parens}
\newcommand{\cprojsndf@parens}[2]{%
  (\cprojsndf{#1}{#2})}
\newcommand{\cprojsndf@unfold}[2]{%
  \idtm{\default@ctxext #2}}

\newcommand{\cprojsnd}[3]{%
  \cprojgen{1}{#1}{#2}{#3}}
\newcommand{\cprojsnd@nested}[1]{%
  \cprojsnd@parens}
\newcommand{\cprojsnd@parens}[3]{%
  (\cprojsnd{#1}{#2}{#3})}
\newcommand{\cprojsnd@unfold}[3]{%
  \subst{#3}{\cprojsnd@unfold{#1}{#2}}}
  
%%%% The sandwich function
\newcommand{\sandwich}[3]{\typefont{sw}^{#1,#2,#3}}
\newcommand{\sandwich@unfold}[3]{\typefont{sw}^{#1,#2,#3}}
\makeatother

%%%%%%%%%%%%%%%%%%%%%%%%%%%%%%%%%%%%%%%%%%%%%%%%%%%%%%%%%%%%%%%%%%%%%%%%%%%%%%%%
%%%% FIBER INCLUSIONS

\makeatletter
\newcommand{\finc}[2]{\typefont{in}^{#2}_{#1}}
\newcommand{\finc@unfold}[2]{\tmext{}{}{\ctxwk{\subst{x}{P}}{x}}{\idtm{\subst{x}{P}}}}
\makeatother

%%%%%%%%%%%%%%%%%%%%%%%%%%%%%%%%%%%%%%%%%%%%%%%%%%%%%%%%%%%%%%%%%%%%%%%%%%%%%%%%
%%%% THE UNIT TYPE

\makeatletter
\newcommand{\unitc}[1]{%
  \unit^0_{\default@ctxext #1}}
\newcommand{\unitct}[1]{%
  \ttt^0_{\default@ctxext #1}}
\newcommand{\unitf}[2]{%
  \unit^1_{\default@ctxext #1,\default@ctxext #2}}
\newcommand{\unitft}[2]{%
  \ttt^1_{\default@ctxext #1,\default@ctxext #2}}
\makeatother

%%%%%%%%%%%%%%%%%%%%%%%%%%%%%%%%%%%%%%%%%%%%%%%%%%%%%%%%%%%%%%%%%%%%%%%%%%%%%%%%
%%%% DEPENDENT FUNCTION TYPES

\makeatletter
\newcommand{\sprd}[2]{\Pi(\default@ctxext #1,\default@ctxext #2)}
\begin{comment}
\newcommand{\@sprd@test@cod}[2]{%
  \@ifnextchar\bgroup{\@sprd@do@cod{#1}}{%
  \Pi(\@sprd@test@dom{#1}{#2} #1,
  }}
\newcommand{\@sprd@do@cod}[4]{%
  \ctxext{\@sprd{#1}{#2}}{\@sprd{#1}{#3}}
  }
\newcommand{\@sprd}[2]{
  \@ifnextchar\bgroup{\@@sprd}{%
    \Pi(}
    #1,{#2})
  }
\newcommand{\@@sprd}[5]{%
  \sprd{#1}{\sprd{#2}{#4}}
  }
\end{comment}

\newcommand{\slam}[3]{%
  \lambda^{{\default@ctxext@parens #1},{\default@ctxext@parens #2}}
  (\default@ctxext #3)
  }
\newcommand{\sev}[1]{\tfev(#1)}

\makeatother

%%%%%%%%%%%%%%%%%%%%%%%%%%%%%%%%%%%%%%%%%%%%%%%%%%%%%%%%%%%%%%%%%%%%%%%%%%%%%%%%
%%%% NON-DEPENDENT FUNCTION TYPES

\newcommand{\jfun}[2]{#1\to#2}

%%%%%%%%%%%%%%%%%%%%%%%%%%%%%%%%%%%%%%%%%%%%%%%%%%%%%%%%%%%%%%%%%%%%%%%%%%%%%%%%
%%%% THE CONSTRUCTORS OF THE TYPE THEORY OF MODELS

\makeatletter
%%%% The initial model
\newcommand{\mctx}{%
  \mathcal{C}}

%%%% The family constructor
\newcommand{\mfam}[2][]{%
  \mathcal{F}_{\default@ctxext #2}^{#1}}
\newcommand{\@mfam@nested}[1]{\@mfam@parens}
\newcommand{\@mfam@parens}[2][]{(\mfam[#1]{#2})}

%%%% The terms constructor
\newcommand{\mtm}[2][]{%
  \mathcal{T}_{\default@ctxext #2}^{#1}}
\newcommand{\@mtm@nested}[1]{\@mtm@parens}
\newcommand{\@mtm@parens}[2][]{(\mtm[#1]{#2})}

%%%% The empty type constructor
\newcommand{\tfemp}[1]{%
  \typefont{emp}_{\default@ctxext #1}}
\newcommand{\tft}[1]{%
  \typefont{t}_{\default@ctxext #1}}

%%%% The extension constructor
\newcommand{\tfext}[1]{%
  \typefont{ext}_{\default@ctxext #1}}

%%%% The substitution constructor
\newcommand{\tfsubst}[1]{%
  \typefont{subst}_{\default@ctxext #1}}
  
%%%% The weakening constructor
\newcommand{\tfwk}[1]{%
  \typefont{wk}_{\default@ctxext #1}}

%%%% The identity function constructor
\newcommand{\tfid}[1]{%
  \typefont{idtm}_{\default@ctxext #1}}
\makeatother

%%%%%%%%%%%%%%%%%%%%%%%%%%%%%%%%%%%%%%%%%%%%%%%%%%%%%%%%%%%%%%%%%%%%%%%%%%%%%%%%

%%%% Introducing logical usage of fonts.
\newcommand{\modelfont}{\mathit} % use 'mf' in command to indicate model font
\newcommand{\typefont}{\mathsf} % use 'tf' in command to indicate type font
\newcommand{\catfont}{\mathrm} % use 'cf' in command to indicate cat font

%%%%%%%%%%%%%%%%%%%%%%%%%%%%%%%%%%%%%%%%%%%%%%%%%%%%%%%%%%%%%%%%%%%%%%%%%%%%%%%%
%%%% Some macros of the book are redefined.

\renewcommand{\UU}{\typefont{U}}
\renewcommand{\isequiv}{\typefont{isEquiv}}
\renewcommand{\happly}{\typefont{hApply}}
\renewcommand{\pairr}[1]{{\mathopen{}\langle #1\rangle\mathclose{}}}
\renewcommand{\type}{\typefont{Type}}
\renewcommand{\op}[1]{{{#1}^\typefont{op}}}
\renewcommand{\susp}{\typefont{\Sigma}}

%%%%%%%%%%%%%%%%%%%%%%%%%%%%%%%%%%%%%%%%%%%%%%%%%%%%%%%%%%%%%%%%%%%%%%%%%%%%%%%%
%%%% The following is a big unorganized list of new macros that we use in the
%%%% notes. 

\newcommand{\tfW}{\typefont{W}}
\newcommand{\tfM}{\typefont{M}}
\newcommand{\mfM}{\modelfont{M}}
\newcommand{\mfN}{\modelfont{N}}
\newcommand{\tfctx}{\typefont{ctx}}
\newcommand{\mftypfunc}[1]{{\modelfont{typ}^{#1}}}
\newcommand{\mftyp}[2]{{\mftypfunc{#1}(#2)}}
\newcommand{\tftypfunc}[1]{{\typefont{typ}^{#1}}}
\newcommand{\tftyp}[2]{{\tftypfunc{#1}(#2)}}
\newcommand{\hfibfunc}[1]{\typefont{fib}_{#1}}
\newcommand{\mappingcone}[1]{\mathcal{C}_{#1}}
\newcommand{\equifib}{\typefont{equiFib}}
\newcommand{\tfcolim}{\typefont{colim}}
\newcommand{\tflim}{\typefont{lim}}
\newcommand{\tfdiag}{\typefont{diag}}
\newcommand{\tfGraph}{\typefont{Graph}}
\newcommand{\mfGraph}{\modelfont{Graph}}
\newcommand{\unitGraph}{\unit^\mfGraph}
\newcommand{\UUGraph}{\UU^\mfGraph}
\newcommand{\tfrGraph}{\typefont{rGraph}}
\newcommand{\mfrGraph}{\modelfont{rGraph}}
\newcommand{\isfunction}{\typefont{isFunction}}
\newcommand{\tfconst}{\typefont{const}}
\newcommand{\conemap}{\typefont{coneMap}}
\newcommand{\coconemap}{\typefont{coconeMap}}
\newcommand{\tflimits}{\typefont{limits}}
\newcommand{\tfcolimits}{\typefont{colimits}}
\newcommand{\islimiting}{\typefont{isLimiting}}
\newcommand{\iscolimiting}{\typefont{isColimiting}}
\newcommand{\islimit}{\typefont{isLimit}}
\newcommand{\iscolimit}{\typefont{iscolimit}}
\newcommand{\pbcone}{\typefont{cone_{pb}}}
\newcommand{\tfinj}{\typefont{inj}}
\newcommand{\tfsurj}{\typefont{surj}}
\newcommand{\tfepi}{\typefont{epi}}
\newcommand{\tftop}{\typefont{top}}
\newcommand{\sbrck}[1]{\Vert #1\Vert}
\newcommand{\strunc}[2]{\Vert #2\Vert_{#1}}
\newcommand{\gobjclass}{{\typefont{U}^\mfGraph}}
\newcommand{\gcharmap}{\typefont{fib}}
\newcommand{\diagclass}{\typefont{T}}
\newcommand{\opdiagclass}{\op{\diagclass}}
\newcommand{\equifibclass}{\diagclass^{\eqv{}{}}}
\newcommand{\universe}{\typefont{U}}
\newcommand{\catid}[1]{{\catfont{id}_{#1}}}
\newcommand{\isleftfib}{\typefont{isLeftFib}}
\newcommand{\isrightfib}{\typefont{isRightFib}}
\newcommand{\leftLiftings}{\typefont{leftLiftings}}
\newcommand{\rightLiftings}{\typefont{rightLiftings}}
\newcommand{\psh}{\typefont{Psh}}
\newcommand{\rgclass}{\typefont{\Omega^{RG}}}
\newcommand{\terms}[2][]{\lfloor #2 \rfloor^{#1}}
\newcommand{\grconstr}[2]
             {\mathchoice % max size is textstyle size.
             {{\textstyle \int_{#1}}#2}% 
             {\int_{#1}#2}%
             {\int_{#1}#2}%
             {\int_{#1}#2}}
\newcommand{\ctxhom}[3][]{\typefont{hom}_{#1}(#2,#3)}
\newcommand{\graphcharmapfunc}[1]{\gcharmap_{#1}}
\newcommand{\graphcharmap}[2][]{\graphcharmapfunc{#1}(#2)}
\newcommand{\tfexp}[1]{\typefont{exp}_{#1}}
\newcommand{\tffamfunc}{\typefont{fam}}
\newcommand{\tffam}[1]{\tffamfunc(#1)}
\newcommand{\tfev}{\typefont{ev}}
\newcommand{\tfcomp}{\typefont{comp}}
\newcommand{\isDec}[1]{\typefont{isDecidable}(#1)}
\newcommand{\smal}{\mathcal{S}}
\renewcommand{\modal}{{\ensuremath{\ocircle}}}
\newcommand{\eqrel}{\typefont{EqRel}}
\newcommand{\piw}{\ensuremath{\Pi\typefont{W}}} %% to be used in conjunction with -pretopos.
\renewcommand{\sslash}{/\!\!/}
\newcommand{\mprd}[2]{\Pi(#1,#2)}
\newcommand{\msm}[2]{\Sigma(#1,#2)}
\newcommand{\midt}[1]{\idvartype_#1}
\newcommand{\reflf}[1]{\typefont{refl}^{#1}}
\newcommand{\tfJ}{\typefont{J}}
\newcommand{\tftrans}{\typefont{trans}}

\newcommand{\tfT}{\typefont{T}}
\newcommand{\reflsym}{{\mathsf{refl}}}
\newcommand{\strans}[2]{\ensuremath{{#1}_{*}({#2})}}
\newcommand{\eqtype}[1]{\typefont{Eq}_{#1}}
\newcommand{\eqtoid}[1]{\typefont{eqtoid}(#1)}
\newcommand{\greek}{\mathrm}
\newcommand{\product}[2]{{#1}\times{#2}}
\newcommand{\pairp}[1]{(#1)}
\newcommand{\jequalizer}[3]{\{#1|#2\jdeq #3\}}
\newcommand{\jequalizerin}[2]{\iota_{#1,#2}}
\newcommand{\tounit}[1]{{!_{#1}}}
\newcommand{\trwk}{\typefont{trwk}}
\newcommand{\trext}{\typefont{trext}}

%%%%%%%%%%%%%%%%%%%%%%%%%%%%%%%%%%%%%%%%%%%%%%%%%%%%%%%%%%%%%%%%%%%%%%%%%%%%%%%%
%%%% When investigation pointed structures we use the \pt macro.

\makeatletter
\newcommand{\pt}[1][]{*_{
  \@ifnextchar\undergraph{\@undergraph@nested}
    {\@ifnextchar\underovergraph{\@underovergraph@nested}{}}#1}}
\makeatother

%%%%%%%%%%%%%%%%%%%%%%%%%%%%%%%%%%%%%%%%%%%%%%%%%%%%%%%%%%%%%%%%%%%%%%%%%%%%%%%%
%%%% OPERATIONS ON GRAPHS
%%%%
%%%% First of all, each graph has a type of vertices and a type of edges. The
%%%% type of vertices of a graph $\Gamma$ is denoted by $\pts{\Gamma}$;
%%%% and likewise for the type of edges.

\makeatletter
\newcommand{\pts}[1]{{\@graphop@nested{#1}}_{0}}
\newcommand{\edg}[1]{{\@graphop@nested{#1}}_{1}}
\newcommand{\@graphop@nested}[1]
  {\@ifnextchar\ctxext{\@ctxext@nested}
      {\@ifnextchar\undergraph{\@undergraph@nested}
         {\@ifnextchar\underovergraph{\@underovergraph@nested}{}}}
    #1}
\makeatother

%%%% The following operations of \undergraph and \underovergraph are used to
%%%% define the free category and the free groupoid of a graph, respectively

\makeatletter
\newcommand{\@undergraphtest}[2]{\@ifnextchar({#1}{#2}}
\newcommand{\undergraph}[2]{\@undergraphtest{\@undergraph@parens{#1}{#2}}{\@undergraph{#1}{#2}}}
\newcommand{\@undergraph}[2]{{#2/#1}}
\newcommand{\@undergraph@nested}[3]{\@undergraph@parens{#2}{#3}}
\newcommand{\@undergraph@parens}[2]{(\@undergraph{#1}{#2})}
\makeatother

\makeatletter
\newcommand{\underovergraph}[2]{\@underovergraphtest{\@underovergraph@parens{#1}{#2}}{\@underovergraph{#1}{#2}}}
\newcommand{\@underovergraph}[2]{{#2}\,{\parallel}\,{#1}}
\newcommand{\@underovergraphtest}{\@undergraphtest}
\newcommand{\@underovergraph@parens}[2]{(\@underovergraph{#1}{#2})}
\newcommand{\@underovergraph@nested}[3]{\@underovergraph@parens{#2}{#3}}
\makeatother

\newcommand{\graphid}[1]{\mathrm{id}_{#1}}
\newcommand{\freecat}[1]{\mathcal{C}(#1)}
\newcommand{\freegrpd}[1]{\mathcal{G}(#1)}


%%%%%%%%%%%%%%%%%%%%%%%%%%%%%%%%%%%%%%%%%%%%%%%%%%%%%%%%%%%%%%%%%%%%%%%%%%%%%%%%
%% Some tikz macros to typeset diagrams uniformly.

\tikzset{patharrow/.style={double,double equal sign distance,-,font=\scriptsize}}
\tikzset{description/.style={fill=white,inner sep=2pt}}
\tikzset{fib/.style={->>,font=\scriptsize}}

%% Used for extra wide diagrams, e.g. when the label is too large otherwise.
\tikzset{commutative diagrams/column sep/Huge/.initial=18ex}

%%%%%%%%%%%%%%%%%%%%%%%%%%%%%%%%%%%%%%%%%%%%%%%%%%%%%%%%%%%%%%%%%%%%%%%%%%%%%%%%
%%%% New theorem environment for conjectures.

\defthm{conj}{Conjecture}{Conjectures}

%%%%%%%%%%%%%%%%%%%%%%%%%%%%%%%%%%%%%%%%%%%%%%%%%%%%%%%%%%%%%%%%%%%%%%%%%%%%%%%%
%%%% The following environment for desiderata should not be there. It is better
%%%% to use the issue tracker for desiderata.

\newenvironment{desiderata}{\begingroup\color{blue}\textbf{Desiderata.}}
{\endgroup}

%%%%%%%%%%%%%%%%%%%%%%%%%%%%%%%%%%%%%%%%%%%%%%%%%%%%%%%%%%%%%%%%%%%%%%%%%%%%%%%%
%%%% The following piece of code from tex.stackexchange:
%%%%
%%%% http://tex.stackexchange.com/a/55180/14653
%%%%
%%%% We include it so that inference rules in align environments have enough
%%%% vertical space.

\newlength\minalignvsep

\makeatletter
\def\align@preamble{%
   &\hfil
    \setboxz@h{\@lign$\m@th\displaystyle{##}$}%
    \ifnum\row@>\@ne
    \ifdim\ht\z@>\ht\strutbox@
    \dimen@\ht\z@
    \advance\dimen@\minalignvsep
    \ht\strutbox\dimen@
    \fi\fi
    \strut@
    \ifmeasuring@\savefieldlength@\fi
    \set@field
    \tabskip\z@skip
   &\setboxz@h{\@lign$\m@th\displaystyle{{}##}$}%
    \ifnum\row@>\@ne
    \ifdim\ht\z@>\ht\strutbox@
    \dimen@\ht\z@
    \advance\dimen@\minalignvsep
    \ht\strutbox@\dimen@
    \fi\fi
    \strut@
    \ifmeasuring@\savefieldlength@\fi
    \set@field
    \hfil
    \tabskip\alignsep@
}
\makeatother

\minalignvsep.2em

\allowdisplaybreaks

%%%%%%%%%%%%%%%%%%%%%%%%%%%%%%%%%%%%%%%%%%%%%%%%%%%%%%%%%%%%%%%%%%%%%%%%%%%%%%%%

\setdescription[1]{itemsep=-0.2em}


\title{\bf An algebraic formulation of dependent type theory}
\author{Egbert Rijke}
\institute{Carnegie Mellon University\\\texttt{erijke@andrew.cmu.edu}}
\date{November 7th -- 10th 2014}

\newcommand\subgoal[1]{\begin{huge}\begin{center}\textbf{Subgoal: {\color{teal!90!black}#1}}\end{center}\end{huge}}
\newcommand\important[1]{\textbf{\color{red!90!black}#1}}
\newcommand{\Btilde}{\skew{4}{\tilde}{B}}

\begin{document}

\begin{frame}
\titlepage
\end{frame}

\begin{frame}
\frametitle{\bf Goal and acknowledgements}
We will do two things:
\begin{itemize}
\item Give a syntactic reformulation of type theory in the usual style of
name-free type theory.
\item Define the corresponding algebraic objects, called \important{E-systems}, in a category with finite limits.
\end{itemize}
\pause
For many insights in the current presentation I am grateful to
\begin{itemize}
\item Vladimir Voevodsky
\item Richard Garner
\item Steve Awodey
\end{itemize}
\end{frame}

\begin{frame}
\frametitle{\bf Desired properties of the algebraic theory}
\begin{itemize}
\item The meta-theory should not require anything more than rules for handling
inferences. In particular, natural numbers should not be required.
\pause
\item Finitely many sorts. The basic ingredients are contexts, families and
terms. Thus there will be 3 sorts.
\pause
\item The theory is algebraic and the operations on it are homomorphisms. The
wish that operations are homomorphisms tells which judgmental equalities to
require.
\pause
\item Finite set of rules. The theory is invariant under slicing as a consequence, rather than by
convention. There will be no rule schemes.
\end{itemize}
\end{frame}

\begin{frame}
\frametitle{\bf The scope of the theory}
The requirement that the theory is essentially algebraic gives immediate access to the standard tools of algebra, such as
\begin{itemize}
\item free algebras;
\item quotients algebras.
\end{itemize}
\pause
The major source of examples of E-systems are \important{B-systems}.
\begin{itemize}
\item B-systems are stratified E-systems.
\item There is a full and faithful functor from C-systems to pre-B-systems with the B-systems as its image.
\end{itemize}
\pause
In a way similar to the definition of E-systems, it is possible to define
\important{internal E-systems}. One of the aims of this research was to have
a systematic treatment of internal models.
\begin{itemize}
\item Nothing more than plain dependent type theory is needed to express what
an internal E-system is.
\end{itemize}
\end{frame}

\begin{frame}
\frametitle{\bf Overview of the theories}
\begin{equation*}
\begin{tikzcd}[ampersand replacement=\&,column sep=0em]
{} \& \textbf{Extension} \ar[very thick,-stealth]{dr} \ar[very thick,-stealth]{d} \ar[very thick,-stealth]{dl}
  \\
\makebox[.7cm]{\textbf{Weakening}} \ar[very thick,-stealth]{d} \& \textbf{Substitution} \ar[very thick,-stealth]{dd} \& \makebox[3cm]{\textbf{Empty context and families}} \ar[very thick,-stealth]{ddl}
\\
\makebox[.7cm]{\textbf{Projection}} \ar[very thick,-stealth]{dr}
\\
{} \& \textbf{Dependent Type Theory}
\end{tikzcd}
\end{equation*}
\\[2em]
\pause
The theories of weakening (and projections), substitution and the empty context and families are formulated as three independent theories on top of the theory of extension.
\end{frame}

\begin{frame}
\frametitle{\bf The basic judgments}
\begin{align*}
\jalign\jctx{\Gamma} 
& \jalign\jctxeq{\Gamma}{\Delta}
  \\
\jalign\jfam{\Gamma}{A} 
& \jalign\jfameq{\Gamma}{A}{B}
  \\
\jalign\jterm{\Gamma}{A}{a} 
& \jalign\jtermeq{\Gamma}{A}{a}{b}.
\end{align*}
\\[\baselineskip]
There are rules expressing that all three kinds of judgmental equality are
convertible equivalence relations.

\pause
An example of a conversion rule:
\begin{align*}
& \inference
  { \jctxeq{\Gamma}{\Delta}
    \jfam{\Gamma}{A}
    }
  { \jfam{\Delta}{A}
    }
& & \inference
    { \jctxeq{\Gamma}{\Delta}
      \jfameq{\Gamma}{A}{B}
      }
    { \jfameq{\Delta}{A}{B}
      }
\end{align*}
\end{frame}

\begin{frame}
\frametitle{\bf The fundamental structure of E-systems}
The \important{fundamental structure of an E-system $\stesys$} in a category with finite limits consists of
\begin{equation*}
\begin{tikzcd}
\stesyst
  \ar{d}{\ebd}
  \\
\stesysf
  \ar{d}{\eft}
  \\
\stesysc
\end{tikzcd}
\end{equation*}
\pause
\begin{itemize}
\item The object $\stesysc$ represents the sort of contexts.
\pause
\item The object $\stesysf$ represents the sort of families.
\pause
\item Every family has a context, this is represented by $\eft:\stesysf\to\stesysc$.
\pause
\item The object $\stesyst$ represents the sort of terms.
\pause
\item Every term has a family associated to it, this is represented by
$\ebd:\stesyst\to\stesysf$.
\end{itemize}
\end{frame}

\begin{frame}
\frametitle{\bf Comparison with B-systems}
\begin{minipage}[t]{.42\textwidth}
The fundamental structure of E-systems:
\\[1.8cm]
\begin{equation*}
\begin{tikzcd}[ampersand replacement=\&]
\stesyst
  \ar{dr}[swap]{\ebd}
  \\
  {}\&
\stesysf
  \ar{d}{\eft}
  \\
  {}\&
\stesysc
\end{tikzcd}
\end{equation*}
\end{minipage}
\hfill
\begin{minipage}[t]{.42\textwidth}
The underlying structure of B-systems:
\vfill
\begin{equation*}
\begin{tikzcd}[ampersand replacement=\&]
\Btilde_2
  \ar{dr}[swap]{\ebd}
  \&
\vdots
  \ar{d}{\eft}
  \\
\Btilde_1
  \ar{dr}[swap]{\ebd}
  \&
B_2
  \ar{d}{\eft}
  \\
\Btilde_0
  \ar{dr}[swap]{\ebd}
  \&
B_1
  \ar{d}{\eft}
  \\
  {}\&
B_0
\end{tikzcd}
\end{equation*}
\end{minipage}
\end{frame}

\begin{frame}
\frametitle{\bf Rules for extension}
Introduction rules for \important{context extension}:
\begin{align*}
& \inference
  { \jfam{\Gamma}{A}
    }
  { \jctx{\ctxext{\Gamma}{A}}
    }
& & \inference
    { \jctxeq{\Gamma}{\Delta}
      \jfameq{\Gamma}{A}{B}
      }
    { \jctxeq{\ctxext{\Gamma}{A}}{\ctxext{\Delta}{B}}
      }
\end{align*}
\pause
Introduction rules for \important{family extension}:
\begin{align*}
& \inference
  { \jfam{{\Gamma}{A}}{P}
    }
  { \jfam{\Gamma}{\ctxext{A}{P}}
    }
& & \inference
    { \jfameq{\Gamma}{A}{B} 
      \jfameq{{\Gamma}{A}}{P}{Q}
      }
    { \jfameq{\Gamma}{\ctxext{A}{P}}{\ctxext{B}{Q}}
      }
\end{align*}
\pause
\important{Extension is associative}:
\begin{align*}
& \inference
  { \jfam{\Gamma}{A}
    \jfam{{\Gamma}{A}}{P}
    }
  { \jctxeq{\ctxext{{\Gamma}{A}}{P}}{\ctxext{\Gamma}{{A}{P}}}
    }
  \\
& \inference
  { \jfam{{\Gamma}{A}}{P}
    \jfam{{{\Gamma}{A}}{P}}{Q}
    }
  { \jfameq{\Gamma}{\ctxext{A}{{P}{Q}}}{\ctxext{{A}{P}}{Q}}
    }
\end{align*}
\end{frame}

\begin{frame}
\frametitle{\bf Pre-extension algebras}
A \important{pre-extension algebra $\stesys$} in a category with finite limits consists
\begin{itemize}
\item a fundamental structure $\stesys$, and
\item context extension and family extension operations
\end{itemize}
\begin{align*}
\ectxext% 
  &%
: \stesysf \to \stesysc%
  \\
\efamext% 
  &%
: \stesysff \to \stesysf,%
\end{align*}
\pause
implementing the introduction rules
\begin{align*}
& \inference
  { \jfam{\Gamma}{A}
    }
  { \jctx{\ctxext{\Gamma}{A}}
    }
& & \inference
  { \jfam{{\Gamma}{A}}{P}
    }
  { \jfam{\Gamma}{\ctxext{A}{P}}
    }
\end{align*}
\pause
Thus, we will additionally require that we have a commuting diagram
\begin{equation*}
\begin{tikzcd}[ampersand replacement=\&]
\stesysff 
  \ar{r}{\efamext} 
  \ar{d}[swap]{\pullbackpr{1}{\ectxext}{\eft}} 
  \& 
\stesysf 
  \ar{d}{\eft}
  \\
\stesysf
  \ar{r}[swap]{\eft} 
  \& 
\stesysc
\end{tikzcd}
\end{equation*}
\end{frame}

\begin{frame}
\frametitle{\bf Notation for pre-extension algebras}
We introduce the following notation:
\begin{align*}
\stesysf_2 
  & := \stesysff
  \\
\eft[2] 
  & := \pullbackpr{1}{\ectxext}{\eft} : \stesysf_2\to\stesysf
  \\
\stesysf_3 & := \pullback{\stesysf_2}{\stesysf_2}{\efamext}{\eft[2]}
  \\
\eft[3]
  & := \pullbackpr{1}{\efamext}{\eft[2]} : \stesysf_3\to\stesysf_2.
\end{align*}
\pause
Then it follows that the outer square in the diagram
\begin{equation*}
\begin{tikzcd}[ampersand replacement=\&,column sep=large]
\stesysf_3
  \ar[dotted]{dr}{\eext{2}}
  \ar{rr}{\pullback{\pullbackpr{2}{\ectxext}{\eft}}{\pullbackpr{2}{\ectxext}{\eft}}{\ectxext}{\eft}}
  \ar{dd}[swap]{\eft[3]}
  \& 
  \&
\stesysf_2
  \ar{d}{\efamext}
  \\
  \&
\stesysf_2
  \ar{d}[swap]{\eft[2]}
  \ar{r}{\pullbackpr{2}{\ectxext}{\eft}}
  \&
\stesysf
  \ar{d}{\eft}
  \\
\stesysf_2
  \ar{r}[swap]{\eft[2]}
  \&
\stesysf
  \ar{r}[swap]{\ectxext}
  \&
\stesysc
\end{tikzcd}
\end{equation*}
commutes. \pause
We define \important{$\eext{2}$} to be the unique morphism rendering the above diagram
commutative. 
\end{frame}

\begin{frame}
\frametitle{\bf Extension algebras}
An \important{extension algebra} is a pre-extension algebra $\stesys$ for which 
the diagrams
\begin{equation*}
\begin{tikzcd}[ampersand replacement=\&]
\stesysf_2 
  \ar{d}[swap]{\pullbackpr{2}{\ectxext}{\eft}} 
  \ar{r}{\efamext} 
  \& 
\stesysf 
  \ar{d}{\ectxext}
  \\
\stesysf 
  \ar{r}[swap]{\ectxext} 
  \& 
\stesysc
\end{tikzcd}
\qquad
\begin{tikzcd}[ampersand replacement=\&]
\stesysf_3
  \ar{d}[swap]{\pullbackpr{2}{\efamext}{\eft[2]}}
  \ar{r}{\eext{2}}
  \& 
\stesysf_2 
  \ar{d}{\efamext} 
  \\
\stesysf_2 
  \ar{r}[swap]{\efamext} 
  \&
\stesysf
\end{tikzcd}
\end{equation*}
commute.
\\[\baselineskip]
\begin{itemize}
\item These diagrams implement associativity of extension.
\end{itemize}
\end{frame}

\begin{frame}
\frametitle{\bf Subgoal: develop properties of extension algebras}
We need to know:
\begin{itemize}
\item What homomorphisms of (pre-)extension algebras are.
\pause
\item That each extension algebra gives locally an extension algebra (stable under slicing).
\pause
\item That the change of base of an extension is an extension algebra. Change of base allows
for \important{`parametrized extension homomorphisms'} of which weakening and substitution
are going to be examples.
\end{itemize}
\end{frame}

\begin{frame}
\frametitle{\bf Pre-extension homomorphisms}
\begin{footnotesize}
Let $\stesys$ and $\stesys'$ be pre-extension algebras. A \important{pre-extension 
homomorphism $f$ from $\stesys'$ to $\stesys$} is a triple $(f_0,f_1,f^t)$ 
consisting of morphisms
\begin{equation*}
\begin{tikzcd}[ampersand replacement=\&]
\stesyst' 
  \ar{r}{f^t}
  \ar{d}[swap]{\ebd'}
  \&
\stesyst
  \ar{d}{\ebd}
  \\
\stesysf'
  \ar{r}{f_1}
  \ar{d}[swap]{\eft'}
  \&
\stesysf
  \ar{d}{\eft}
  \\
\stesysc' 
  \ar{r}[swap]{f_0}
  \&
\stesysc
\end{tikzcd}
\end{equation*}
\pause
such that the indicated squares commute, for which furthermore the squares
\begin{equation*}
\begin{tikzcd}[ampersand replacement=\&]
\stesysf' 
  \ar{r}{f_1}
  \ar{d}[swap]{\ectxext'}
  \&
\stesysf
  \ar{d}{\ectxext}
  \\
\stesysc'
  \ar{r}[swap]{f_0}
  \&
\stesysc
\end{tikzcd}
\quad
\text{and}
\quad
\begin{tikzcd}[ampersand replacement=\&,column sep=huge]
\stesysf_2'
  \ar{r}{f_1\times_{\ectxext,\eft} f_1}
  \ar{d}[swap]{\efamext'}
  \&
\stesysf_2
  \ar{d}{\efamext}
  \\
\stesysf'
  \ar{r}[swap]{f_1}
  \&
\stesysf
\end{tikzcd}
\end{equation*}
commute.
\end{footnotesize}
\end{frame}

\begin{frame}
\frametitle{Extension homomorphisms}
\begin{definition}
A pre-extension homomorphism between extension algebras is called an \important{extension
homomorphism}.
\end{definition}
\end{frame}

\begin{frame}
\frametitle{\bf Slicing of pre-extension algebras}
Suppose that $\stesys$ is a pre-extension algebra. Then we define
the pre-extension algebra \important{$\famesys{\stesys}$} to consist of the
fundamental structure
\begin{equation*}
\begin{tikzcd}[ampersand replacement=\&]
\stesyst_2
  \ar{d}{\ebd[2]}
  \\
\stesysf_2
  \ar{d}{\eft[2]}
  \\
\stesysf
\end{tikzcd}
\end{equation*}
\pause
where
\begin{align*}
\stesyst_2 
  & := \pullback{\stesysf}{\stesyst}{\ectxext}{\eft\circ\ebd},
  &
\ebd[2]
  & := \ectxext^\ast(\ebd),
\end{align*}
with the extension operations
\begin{align*}
\efamext 
  & 
: \stesysf_2\to\stesysf,
  &
\eext{2} 
  & 
: \stesysf_3\to\stesysf_2.
\end{align*}
\end{frame}

\begin{frame}
\frametitle{\bf Local extension algebras}
\begin{theorem}
If $\stesys$ is an extension algebra, then so is $\famesys{\stesys}$.
\end{theorem}
\pause
\begin{lemma}
Let $\stesys$ be a pre-extension algebra. Then $\stesys$ is an extension
algebra if and only if we have the extension homomorphisms
$\mathbf{e}_0:\famesys{\stesys}\to\stesys$
and
$\mathbf{e}_1:\famesys{\famesys{\stesys}}\to\famesys{\stesys}$
given by
\begin{equation*}
\begin{tikzcd}[ampersand replacement=\&,column sep=huge]
\stesyst_2
  \ar{r}{\pullbackpr{2}{\ectxext}{\eft\circ\ebd}}
  \ar{d}[swap]{\ebd[2]}
  \&
\stesyst
  \ar{d}{\ebd}
  \\
\stesysf_2
  \ar{r}{\pullbackpr{2}{\ectxext}{\eft}}
  \ar{d}[swap]{\eft[2]}
  \&
\stesysf
  \ar{d}{\eft}
  \\
\stesysf
  \ar{r}[swap]{\ectxext}
  \&
\stesysc
\end{tikzcd}
\qquad\text{and}\qquad
\begin{tikzcd}[ampersand replacement=\&,column sep=huge]
\stesyst_3
  \ar{r}{\pullbackpr{2}{\efamext}{\eft[2]\circ\ebd[2]}}
  \ar{d}[swap]{\ebd[3]}
  \&
\stesyst_2
  \ar{d}{\ebd[2]}
  \\
\stesysf_3
  \ar{r}{\pullbackpr{2}{\efamext}{\eft[2]}}
  \ar{d}[swap]{\eft[3]}
  \&
\stesysf_2
  \ar{d}{\eft[2]}
  \\
\stesysf_2
  \ar{r}[swap]{\efamext}
  \&
\stesysf
\end{tikzcd}
\end{equation*}
\end{lemma}
\end{frame}

\begin{frame}
\frametitle{\bf Condensed commutative diagrams}
Suppose $f:\stesys\to\stesys'$ is a pre-extension homomorphism. We say that
a diagram
\begin{equation*}
\begin{tikzcd}[ampersand replacement=\&]
\stesys
  \ar{r}{f}
  \ar{d}[swap]{p}
  \&
\stesys'
  \ar{d}{p'}
  \\
X \ar{r}[swap]{g}
  \&
Y
\end{tikzcd}
\end{equation*}
commutes if the diagram
\begin{equation*}
\begin{tikzcd}[ampersand replacement=\&]
\stesysc
  \ar{r}{f_0}
  \ar{d}[swap]{p}
  \&
\stesysc'
  \ar{d}{p'}
  \\
X \ar{r}[swap]{g}
  \&
Y
\end{tikzcd}
\end{equation*}
commutes.
\end{frame}

\begin{frame}
\frametitle{\bf Change of base of pre-extension algebras}
Let $\stesys$ be a pre-extension algebra and consider $p:\stesysc\rightarrow X\leftarrow Y:g$.
Then there is a pre-extension algebra \important{$\cobesys{Y}{\stesys}{g}{p}$} with projections, such that for every diagram
\begin{equation*}
\begin{tikzcd}[ampersand replacement=\&,column sep=large]
\stesys'
  \ar[bend right=10]{ddr}[swap]{p'}
  \ar[bend left=10]{rrd}{f}
  \ar[dotted]{dr}[near end]{[p',f]}
  \\
  {}\&
\cobesys{Y}{\stesys}{g}{p}
  \ar{d}{\pullbackpr{1}{g}{p}}
  \ar{r}[swap]{\pullbackpr{2}{g}{p}}
  \&
\stesys
  \ar{d}{p}
  \\
  {}\&
Y \ar{r}[swap]{g}
  \&
X
\end{tikzcd}
\end{equation*}
of which the outer square commutes, the pre-extension homomorphism \important{$[p',f]$} exists
and is unique with the property that it renders the diagram commutative.
\end{frame}

\begin{frame}
\frametitle{\bf Change of base of extension algebras}
\begin{theorem}
The change of base of an extension algebra is again an extension algebra.
\end{theorem}
\pause
\begin{theorem}
Let $\stesys$ be a pre-extension algebra. Then
\begin{equation*}
\famesys{\famesys{\stesys}} \simeq \cobesys{\stesysf}{\famesys{\stesys}}{\ectxext}{\eft}.
\end{equation*}
\end{theorem}
\pause
\begin{theorem}
Let $\stesys$ be a pre-extension algebra and consider $p:\stesysc\rightarrow
X\leftarrow Y:g$. Then
\begin{equation*}
\famesys{\cobesys{Y}{\stesys}{g}{p}}\simeq
\cobesys{Y}{\famesys{\stesys}}{g}{p\circ\eft}.
\end{equation*}
\end{theorem}
\end{frame}

\begin{frame}
\frametitle{\bf Overview of the theory of weakening}
\begin{itemize}
\item In the syntax, we will introduce an operation for weakening which acts at the three
levels of fundamental structures (contexts, families and terms). \pause
\important{When $B$ is weakened by $A$, we denote this by $\ctxwk{A}{B}$.}
  \pause
\item Weakening will preserve extension, so it will be implemented as an extension
homomorphism.
  \pause
\item There will be a notion of \important{`Currying for weakening'}, which explains
what happens when something is weakened by an extended object.
  \pause
\item Pre-weakening algebras will be extension algebras with a weakening operation
satisfying an implementation of the currying condition. \pause Pre-weakening homomorphisms
will preserve this structure.
  \pause
\item Weakening will preserve itself, so weakening we will require that weakening is
a pre-weakening homomorphism.
\end{itemize}
\end{frame}

\begin{frame}
\frametitle{\bf Rules for the theory of weakening}
\begin{footnotesize}
The introduction rules for weakening:
\begin{align*}
& \inference
  { \jfam{\Gamma}{A}
    \jfam{\Gamma}{B}
    }
  { \jfam{{\Gamma}{A}}{\ctxwk{A}{B}}
    }
& & \inference
    { \jfameq{\Gamma}{A}{A'}
      \jfameq{\Gamma}{B}{B'}
      }
    { \jfameq{{\Gamma}{A}}{\ctxwk{A}{B}}{\ctxwk{A'}{B'}}
      }
\end{align*}
  \pause
\begin{align*}
& \inference
  { \jfam{\Gamma}{A}
    \jfam{{\Gamma}{B}}{Q}
    }
  { \jfam{{{\Gamma}{A}}{\ctxwk{A}{B}}}{\ctxwk{A}{Q}}
    }
& & \inference
    { \jfameq{\Gamma}{A}{A'}
      \jfameq{{\Gamma}{B}}{Q}{Q'}
      }
    { \jfameq
        {{{\Gamma}{A}}{\ctxwk{A}{B}}}
        {\ctxwk{A}{Q}}
        {\ctxwk{A'}{Q'}}
      }
    \\
& \inference
  { \jfam{\Gamma}{A}
    \jterm{{\Gamma}{B}}{Q}{g}
    }
  { \jterm{{{\Gamma}{A}}{\ctxwk{A}{B}}}{\ctxwk{A}{Q}}{\ctxwk{A}{g}}
    }
& & \inference
    { \jfameq{\Gamma}{A}{A'}
      \jtermeq{{\Gamma}{B}}{Q}{g}{g'}
      }
    { \jtermeq
        {{{\Gamma}{A}}{\ctxwk{A}{B}}}
        {\ctxwk{A}{Q}}
        {\ctxwk{A}{g}}
        {\ctxwk{A'}{g'}}
      }
\end{align*}
\pause
Weakening preserves extension:
\begin{align*}
& \inference
  { \jfam{\Gamma}{A}
    \jfam{{\Gamma}{B}}{Q}
    }
  { \jfameq
      {\ctxext{\Gamma}{A}}
      {\ctxwk{A}{\ctxext{B}{Q}}}
      {\ctxext{\ctxwk{A}{B}}{\ctxwk{A}{Q}}}
    }
  \\
& \inference
  { \jfam{\Gamma}{A}
    \jfam{{{\Gamma}{B}}{Q}}{R}
    }
  { \jfameq
      {\ctxext{{\Gamma}{A}}{\ctxwk{A}{B}}}
      {\ctxwk{A}{\ctxext{Q}{R}}}
      {\ctxext{\ctxwk{A}{Q}}{\ctxwk{A}{R}}}
    }
\end{align*}
\end{footnotesize}
\end{frame}

\begin{frame}
\frametitle{\bf The weakening operation}
Let $\stesys$ be an extension algebra. A \important{weakening operation
on $\stesys$} is an extension homomorphism 
\begin{equation*}
\mathbf{w}(\stesys)
    :
  \cobesys{\stesysf}{\famesys{\stesys}}{\eft}{\eft}
    \to
  \famesys{\famesys{\stesys}}
\end{equation*}
for which the diagram
\begin{equation*}
\begin{tikzcd}[ampersand replacement=\&,column sep=large]
\cobesys{\stesysf}{\famesys{\stesys}}{\eft}{\eft}
  \ar{r}{\mathbf{w}(\stesys)}
  \ar{dr}[swap]{\pullbackpr{1}{\eft}{\eft}}
  \&
\famesys{\famesys{\stesys}}
  \ar{d}{\eft[2]}
  \\
  {}\& 
\stesysf
\end{tikzcd}
\end{equation*}
commutes.
\end{frame}

\begin{frame}
\frametitle{\bf Rules for weakening: Currying}
\begin{align*}
& \inference
  { \jfam{\Gamma}{A}
    \jfam{{\Gamma}{A}}{P}
    \jfam{\Gamma}{B}
    }
  { \jfameq
      {{{\Gamma}{A}}{P}}
      {\ctxwk{\ctxext{A}{P}}{B}}
      {\ctxwk{P}{{A}{B}}}
    }
\end{align*}
\pause
\begin{align*}
& \inference
  { \jfam{\Gamma}{A}
    \jfam{{\Gamma}{A}}{P}
    \jfam{{\Gamma}{B}}{Q}
    }
  { \jfameq
      {{{{\Gamma}{A}}{P}}{\ctxwk{P}{{A}{B}}}}
      {\ctxwk{\ctxext{A}{P}}{Q}}
      {\ctxwk{P}{{A}{Q}}}
    }
\\
& \inference
  { \jfam{\Gamma}{A}
    \jfam{{\Gamma}{A}}{P}
    \jterm{{\Gamma}{B}}{Q}{g}
    }
  { \jtermeq
      {{{{\Gamma}{A}}{P}}{\ctxwk{P}{{A}{B}}}}
      {\ctxwk{P}{{A}{Q}}}
      {\ctxwk{\ctxext{A}{P}}{g}}
      {\ctxwk{P}{{A}{g}}}
    } 
\end{align*}
\\[\baselineskip]
\pause
To express these rules algebraically, we need to define a weakening operation on
$\famesys{\stesys}$, provided we have one on $\stesys$.
\end{frame}

\begin{frame}
\frametitle{\bf Weakening for the families algebra}
Let $\stesys$ be an extension algebra with weakening operation
$\mathbf{w}(\stesys)$. Then $\famesys{\stesys}$ has the weakening operation
$\mathbf{w}(\famesys{\stesys})$ which is uniquely determined by rendering the
diagram
\begin{small}
\begin{equation*}
\begin{tikzcd}[ampersand replacement=\&,column sep=large]
\cobesys{\stesysf_2}{\famesys{\famesys{\stesys}}}{\eft[2]}{\eft[2]}
  \ar{rr}{%
      \pullback{\beta_1}{\boldsymbol{\beta}}{\eft}{\eft}
    }
  \ar[bend right]{ddr}[swap]{\pullbackpr{1}{\eft[2]}{\eft[2]}}
  \ar[dotted]{dr}{\important{$\mathbf{w}(\famesys{\stesys})$}}
  \&
  {}\&
\cobesys{\stesysf}{\famesys{\stesys}}{\eft}{\eft}
  \ar{r}{\important{$\mathbf{w}(\stesys)$}}
  \&
\famesys{\famesys{\stesys}}
  \ar{d}{\boldsymbol{\beta}}
  \\
  {}\&
\famesys{\famesys{\famesys{\stesys}}}
  \ar{r}{\boldsymbol{\beta}_\mathbf{2}}
  \ar{d}[swap]{\eft[3]}
  \&
\famesys{\famesys{\stesys}}
  \ar{d}{\eft[2]}
  \ar{r}{\boldsymbol{\beta}}
  \&
\famesys{\stesys}
  \ar{d}{\eft}
  \\
  {}\&
\stesysf_2
  \ar{r}[swap]{\efamext}
  \&
\stesysf
  \ar{r}[swap]{\ectxext}
  \&
\stesysc
\end{tikzcd}
\end{equation*}
\end{small}
commutative.
\end{frame}

\begin{frame}
\frametitle{\bf Pre-weakening algebras}
A \important{pre-weakening algebra} is an extension algebra $\stesys$ 
with a weakening operation 
$ \mathbf{w}(\stesys)
    :
  \cobesys{\stesysf}{\famesys{\stesys}}{\eft}{\eft}
    \to
  \famesys{\famesys{\stesys}}$
for which the diagram
\begin{equation*}
\begin{tikzcd}[ampersand replacement=\&,column sep=15em]
\cobesys{\stesysf_2}{\famesys{\stesys}}{\eft\circ\eft[2]}{\eft}
  \ar[bend right=10]{dr}[swap]%
    { [ \pullbackpr{1}{\eft\circ\eft[2]}{\eft},%
        \important{$
        \mathbf{w}(\stesys)%
          \circ%
        (\pullback{\efamext}{\catid{\famesys{\stesys}}}{\eft}{\eft})$}%
        ]%
      }
  \ar{r}{
    [ \pullbackpr{1}{\eft\circ\eft[2]}{\eft},%
      \important{$\mathbf{w}(\stesys)$}%
        \circ%
      (\pullback{\eft[2]}{\catid{\famesys{\stesys}}}{\eft}{\eft})%
      ]}%
  \&
\cobesys{\stesysf_2}{\famesys{\famesys{\stesys}}}{\eft[2]}{\eft[2]}
  \ar{d}{\important{$\mathbf{w}(\famesys{\stesys})$}}
  \\
  {}\&
\famesys{\famesys{\famesys{\stesys}}}
\end{tikzcd}
\end{equation*}
commutes (implementing currying for weakening).
\end{frame}

\begin{frame}
\frametitle{\bf Pre-weakening homomorphisms}
A \important{pre-weakening homomorphism} between pre-weakening algebras $\stesys'$ and $\stesys$ is an
extension homomorphism $f:\stesys'\to \stesys$ such that additionally the diagram
\begin{equation*}
\begin{tikzcd}[ampersand replacement=\&,column sep=large]
\cobesys{\stesysf'}{\famesys{\stesys'}}{\eft'}{\eft'}
  \ar{d}[swap]{\mathbf{w}(\stesys')}
  \ar{r}{\pullback{f_1}{\famehom{f}}{\eft}{\eft}}
  \&
\cobesys{\stesysf}{\famesys{\stesys}}{\eft}{\eft}
  \ar{d}{\mathbf{w}(\stesys)}
  \\
\famesys{\famesys{\stesys'}}
  \ar{r}[swap]{\famehom{\famehom{f}}}
  \&
\famesys{\famesys{\stesys}}
\end{tikzcd}
\end{equation*}
commutes.
\end{frame}

\begin{frame}
\frametitle{\bf Change of base of pre-weakening algebras}
Let $\stesys$ be a pre-weakening algebra and consider $p:\stesysc\rightarrow X\leftarrow Y:g$.
Then we define
\begin{equation*}
\mathbf{w}(\cobesys{Y}{\stesys}{g}{p})
  :
\cobesys
  { (\pullback{Y}{\stesysf}{g}{p\circ\eft})}
  { \famesys{\cobesys{Y}{\stesys}{g}{p}}}
  { g^\ast(\eft)}
  { g^\ast(\eft)}
  \to
\famesys{\famesys{\cobesys{Y}{\stesys}{g}{p}}}
\end{equation*}
\pause
to be the unique extension homomorphism rendering the diagram
\begin{scriptsize}
\begin{equation*}
\begin{tikzcd}[ampersand replacement=\&]
\cobesys
  { \pullback{Y}{\stesysf}{g}{p\circ\eft}}
  { \famesys{\cobesys{Y}{\stesys}{g}{p}}}
  { g^\ast(\eft)}
  { g^\ast(\eft)}
  \ar[bend right]{ddr}[swap]{\pullbackpr{1}{g^\ast(\eft)}{g^\ast(\eft)}}
  \ar{rr}{%
    \pullback
      { \pullbackpr{2}{g}{p\circ\eft}}
      { \boldsymbol{\pi}_\mathbf{2}(g,p\circ\eft)}
      { \eft}
      { \eft}
    }
  \ar[dotted]{dr}{\mathbf{w}(\cobesys{Y}{\stesys}{g}{p})}
  \&
  {}\&
\cobesys{\stesysf}{\famesys{\stesys}}{\eft}{\eft}
  \ar{r}{\mathbf{w}(\stesys)}
  \&
\famesys{\famesys{\stesys}}
  \ar{d}{\boldsymbol{\beta}}
  \\
  {}\&
\famesys{\famesys{\cobesys{Y}{\stesys}{g}{p}}}
  \ar{r}{\beta}
  \ar{d}{g^\ast(\eft)_2}
  \&
\famesys{\cobesys{Y}{\stesys}{g}{p}}
  \ar{r}{\boldsymbol{\pi}_2(g,p\circ\eft)}
  \ar{d}{g^\ast(\eft)}
  \&
\famesys{\stesys}
  \ar{d}{\eft}
  \\
  {}\&
\pullback{Y}{\stesysf}{g}{p\circ\eft}
  \ar{r}[swap]{\cobesys{Y}{\ectxext}{g}{p}}
  \&
\pullback{Y}{\stesysc}{g}{p}
  \ar{r}[swap]{\pullbackpr{2}{g}{p}}
  \&
\stesysc
\end{tikzcd}
\end{equation*}
\end{scriptsize}
commutative.
\end{frame}

\begin{frame}
\frametitle{\bf Properties of pre-weakening algebras}
\begin{theorem}
If $\stesys$ is a pre-weakening algebra and $p:\stesysc\rightarrow X\leftarrow Y:g$,
then $\cobesys{Y}{\stesys}{g}{p}$ is also a pre-weakening algebra.
\end{theorem}
\pause
\begin{theorem}
If $\stesys$ is a pre-weakening algebra, then so is $\famesys{\stesys}$. 
\end{theorem}
\end{frame}

\begin{frame}
\frametitle{\bf Weakening algebras}
Thus, it makes sense to require that the weakening operation itself is a
pre-weakening homomorphism.
\begin{definition}
A \important{weakening algebra} is a pre-weakening algebra $\stesys$ with the property that
$\mathbf{w}(\stesys)$ is a pre-weakening homomorphism.
\end{definition}
\pause
\begin{theorem}
If $\stesys$ is a weakening algebra and $p:\stesysc\rightarrow X\leftarrow Y:g$,
then $\cobesys{Y}{\stesys}{g}{p}$ is also a weakening algebra.
\end{theorem}
\pause
\begin{theorem}
If $\stesys$ is a weakening algebra, then so is $\famesys{\stesys}$. 
\end{theorem}
\end{frame}

\begin{frame}
\frametitle{\bf Rules for weakening: weakening preserves itself}
The requirement that weakening is itself a pre-weakening homomorphism
is represented by the following inference rules which we impose on
weakening:
\begin{small}
\begin{align*}
& \inference
  { \jfam{\Gamma}{A}
    \jfam{{\Gamma}{B}}{Q}
    \jfam{{\Gamma}{B}}{R}
    }
  { \jfameq
      {{{{\Gamma}{A}}{\ctxwk{A}{B}}}{\ctxwk{A}{Q}}}
      {\ctxwk{A}{{Q}{R}}}
      {\ctxwk{{A}{Q}}{{A}{R}}}
    }
\end{align*}
\pause
\begin{align*}
& \inference
  { \jfam{\Gamma}{A}
    \jfam{{\Gamma}{B}}{Q}
    \jfam{{{\Gamma}{B}}{R}}{S}
    }
  { \jfameq
      {{{{{\Gamma}{A}}{\ctxwk{A}{B}}}{\ctxwk{A}{Q}}}{\ctxwk{A}{{Q}{R}}}}
      {\ctxwk{A}{{Q}{S}}}
      {\ctxwk{{A}{Q}}{{A}{S}}}
    }
\\
& \inference
  { \jfam{\Gamma}{A}
    \jfam{{\Gamma}{B}}{Q}
    \jterm{{{\Gamma}{B}}{R}}{S}{k}
    }
  { \jtermeq
      {{{{{\Gamma}{A}}{\ctxwk{A}{B}}}{\ctxwk{A}{Q}}}{\ctxwk{A}{{Q}{R}}}}
      {\ctxwk{A}{{Q}{S}}}
      {\ctxwk{A}{{Q}{k}}}
      {\ctxwk{{A}{Q}}{{A}{k}}}
    }
\end{align*}
\end{small}
\end{frame}

\begin{frame}
\frametitle{\bf Overview of the theory of projections}
\begin{itemize}
\item In the theory of projections, we will introduce units, which will eventually behave
like \important{fiberwise identity morphisms of families}.
  \pause
\item Together with weakening, units will induce all the \important{projections}.
  \pause
\item \important{Pre-projection algebras} will be weakening algebras with units.
  \pause
\item \important{Projection algebras} will be pre-projection algebras for which
weakening is a pre-projection homomorphism.
\end{itemize}
\end{frame}

\begin{frame}
\frametitle{\bf Rules for the theory of projections}
Introduction rules for units:
\begin{align*}
& \inference
  { \jfam{\Gamma}{A}
    }
  { \unfoldall{\jhom{\Gamma}{A}{A}{\idtm{A}}}
    }
& & \inference
    { \jfameq{\Gamma}{A}{A'}
      }
    { \unfoldall{\jhomeq{\Gamma}{A}{A}{\idtm{A}}{\idtm{A'}}}
      }
\end{align*}
\pause
Weakening preserves the unit:
\begin{equation*}
\inference
  { \jfam{\Gamma}{A}
    \jfam{{\Gamma}{B}}{Q}
    }
  { \unfoldall{\jhomeq
      {{{\Gamma}{A}}{\ctxwk{A}{B}}}
      {\ctxwk{A}{Q}}
      {\ctxwk{A}{Q}}
      {\ctxwk{A}{\idtm{Q}}}
      {\idtm{\ctxwk{A}{Q}}}
      }
    }
\end{equation*}
\end{frame}

\begin{frame}
\frametitle{\bf Pre-projection algebras}
A \important{pre-projection algebra} is a weakening algebra $\stesys$ for which there is a term
$\mathbf{i}(\stesys):\stesysf\to \stesyst_2$ such that the diagram
\begin{equation*}
\begin{tikzcd}[ampersand replacement=\&,column sep=large]
\stesysf 
  \ar{r}{\mathbf{i}(\stesys)} 
  \ar{d}[swap]{\Delta_{\eft}} 
  \& 
\stesyst_2 
  \ar{d}{\ebd[2]}\\
\pullback{\stesysf}{\stesysf}{\eft}{\eft} 
  \ar{r}[swap]{w(\stesys)_0} 
  \& 
\stesysf_2
\end{tikzcd}
\end{equation*}
commutes. In this diagram, $\Delta_{\eft}:\stesysf\to \pullback{\stesysf}{\stesysf}{\eft}{\eft}$ is the diagonal.
\end{frame}

\begin{frame}
\frametitle{\bf Pre-projection homomorphisms}
A \important{pre-projection homomorphism from $\stesys$ to $\stesys'$} is a weakening homomorphism
$f:\stesys'\to \stesys$ such that the square
\begin{equation*}
\begin{tikzcd}[ampersand replacement=\&,column sep=large]
\stesyst_2'
  \ar{r}{f^t_2}
  \&
\stesyst_2
  \\
\stesysf' 
  \ar{r}[swap]{f_1}
  \ar{u}{\mathbf{i}(\stesys')}
  \&
\stesysf
  \ar{u}[swap]{\mathbf{i}(\stesys)}
\end{tikzcd}
\end{equation*}
commutes.
\end{frame}

\begin{frame}
\frametitle{\bf Projection algebras}
\begin{definition}
A \important{projection algebra} is a pre-projection algebra for which the weakening operation
is a pre-projection homomorphism.
\end{definition}
\pause
\begin{definition}
A \important{projection homomorphism} is a pre-projection homomorphism between projection
algebras.
\end{definition}
\pause
\begin{theorem}
If $\stesys$ is a projection algebra and $p:\stesysc\rightarrow X\leftarrow Y:g$,
then $\cobesys{Y}{\stesys}{g}{p}$ is also a projection algebra.
\end{theorem}
\pause
\begin{theorem}
If $\stesys$ is a projection algebra, then so is $\famesys{\stesys}$. 
\end{theorem}
\end{frame}

\begin{frame}
\frametitle{\bf Overview of the theory of substitution}
\begin{itemize}
\item We will introduce an operation for substitution by a term, this allows us
to consider \important{fibers of families}. The substitution of $P$ by $a$ will
be denoted by $\subst{a}{P}$.
  \pause
\item Substitution will be compatible with extension, thus it will be implemented
as an extension homomorphism.
  \pause
\item \important{Pre-substitution algebras} will be extension algebras with a substitution
operation. \important{Pre-substitution homomorphisms} will preserve this structure.
  \pause
\item \important{Substitution algebras} will be pre-substitution algebras for which
the substitution operation itself is a pre-substitution homomorphism.
\end{itemize}
\end{frame}

\begin{frame}
\frametitle{\bf Rules for the theory of substitution}
\begin{footnotesize}
Introduction rules for substitution:
\begin{align*}
& \inference
  { \jterm{\Gamma}{A}{a}
    \jfam{{\Gamma}{A}}{P}
    }
  { \jfam{\Gamma}{\subst{a}{P}}
    }
& & \inference
    { \jtermeq{\Gamma}{A}{a}{a'}
      \jfameq{{\Gamma}{A}}{P}{P'}
      }
    { \jfameq{\Gamma}{\subst{a}{P}}{\subst{a'}{P'}}
      }
\end{align*}
\pause
\begin{align*}
& \inference
  { \jterm{\Gamma}{A}{a}
    \jfam{{{\Gamma}{A}}{P}}{Q}
    }
  { \jfam{{\Gamma}{\subst{a}{P}}}{\subst{a}{Q}}
    }
& & \inference
    { \jtermeq{\Gamma}{A}{a}{a'}
      \jfameq{{{\Gamma}{A}}{P}}{Q}{Q'}
      }
    { \jfameq{{\Gamma}{\subst{a}{P}}}{\subst{a}{Q}}{\subst{a'}{Q'}}
      }
    \\
& \inference
  { \jterm{\Gamma}{A}{a}
    \jterm{{{\Gamma}{A}}{P}}{Q}{g}
    }
  { \jterm{{\Gamma}{\subst{a}{P}}}{\subst{a}{Q}}{\subst{a}{g}}
    }
& & \inference
    { \jtermeq{\Gamma}{A}{a}{a'}
      \jtermeq{{{\Gamma}{A}}{P}}{Q}{g}{g'}
      }
    { \jtermeq
        {{\Gamma}{\subst{a}{P}}}
        {\subst{a}{Q}}
        {\subst{a}{g}}
        {\subst{a'}{g'}}
      }
\end{align*}
\pause
Substitution is compatible with extension:
\begin{align*}
& \inference
  { \jterm{\Gamma}{A}{a}
    \jfam{{{\Gamma}{A}}{P}}{Q}
    }
  { \jfameq
      {\Gamma}
      {\subst{a}{\ctxext{P}{Q}}}
      {\ctxext{\subst{a}{P}}{\subst{a}{Q}}}
    }
  \\
& \inference
  { \jterm{\Gamma}{A}{a}
    \jfam{{{{\Gamma}{A}}{P}}{Q}}{R}
    }
  { \jfameq
      {{\Gamma}{\subst{a}{P}}}
      {\subst{a}{\ctxext{Q}{R}}}
      {\ctxext{\subst{a}{Q}}{\subst{a}{R}}}
    }
\end{align*}
\end{footnotesize}
\end{frame}

\begin{frame}
\frametitle{\bf Pre-substitution algebras}
A \important{pre-substitution} for an extension algebra $\stesys$ is an
extension homomorphism
\begin{equation*}
\mathbf{s}(\stesys):\cobesys{\stesyst}{\famesys{\famesys{\stesys}}}{\ebd}{\eft[2]}\to \famesys{\stesys}
\end{equation*}
for which the square
\begin{equation*}
\begin{tikzcd}[ampersand replacement=\&,column sep=large]
\pullback{\stesyst}{\famesys{\famesys{\stesysf}}}{\ebd}{\eft[2]}
  \ar{r}{\mathbf{s}(\stesys)}
  \ar{d}[swap]{\ebd\circ\pullbackpr{1}{\ebd}{\eft[2]}}
  \&
\famesys{\stesys} 
  \ar{d}{\eft}
  \\
\stesysf 
  \ar{r}[swap]{\eft}
  \&
\stesysc
\end{tikzcd}
\end{equation*}
commutes.\\
\pause
A \important{pre-substitution algebra} is an extension algebra
together with a pre-substitution.
\end{frame}

\begin{frame}
\frametitle{\bf Pre-substitution homomorphisms}
A \important{pre-substitution homomorphism} is an extension homomorphism $f:\stesys'\to \stesys$
for which the square
\begin{equation*}
\begin{tikzcd}[ampersand replacement=\&,column sep=huge]
\cobesys{\stesyst'}{\famesys{\famesys{\stesys'}}}{\ebd'}{\eft[2]'}
  \ar{r}{\pullback{f^t}{\famehom{\famehom{f}}}{\ebd}{\eft[2]}}
  \ar{d}[swap]{\mathbf{s}(\stesys')}
  \&
\cobesys{\stesyst}{\famesys{\famesys{\stesys}}}{\ebd}{\eft[2]}
  \ar{d}{\mathbf{s}(\stesys)}
  \\
\famesys{\stesys'}
  \ar{r}[swap]{\famehom{f}}
  \&
\famesys{\stesys}
\end{tikzcd}
\end{equation*}
commutes.
\end{frame}

\begin{frame}
\frametitle{\bf The family pre-substitution algebra}
\begin{theorem}
If $\stesys$ is a pre-substitution algebra, then so is $\famesys{\stesys}$\pause~with
$\mathbf{s}(\famesys{\stesys})$ defined to be the unique extension homomorphism
rendering the diagram
\begin{equation*}
\begin{tikzcd}[ampersand replacement=\&]
\cobesys{\stesyst_2}{\famesys{\famesys{\famesys{\stesys}}}}{\ebd[2]}{\eft[3]}
  \ar[dotted]{dr}{\mathbf{s}(\famesys{\stesys})}
  \ar{rr}{\pullback{\pullbackpr{2}{\ectxext}{\eft\circ\ebd}}{\boldsymbol{\beta}_\mathbf{2}}{\ebd}{\eft[2]}}
  \ar{dd}[swap]{\ebd[2]\circ\pullbackpr{1}{\ebd[2]}{\eft[3]}}
  \&
  {}\&
\cobesys{\stesyst}{\famesys{\famesys{\stesys}}}{\ebd}{\eft[2]}
  \ar{d}{\mathbf{s}(\stesys)}
  \\
  {}\&
\famesys{\famesys{\stesys}}
  \ar{r}{\boldsymbol{\beta}}
  \ar{d}{\eft[2]}
  \&
\famesys{\stesys}
  \ar{d}{\eft}
  \\
\stesysf_2
  \ar{r}[swap]{\eft[2]}
  \&
\stesysf
  \ar{r}[swap]{\ectxext}
  \&
\stesysc
\end{tikzcd}
\end{equation*}
commutative.
\end{theorem}
\end{frame}

\begin{frame}
\frametitle{\bf Change of base of pre-substitution algebras}
\begin{theorem}
Let $\stesys$ be a pre-substitution algebra and consider
$p:\stesysc\rightarrow X\leftarrow Y:g$. Then
$\cobesys{Y}{\stesys}{g}{p}$ is a pre-substitution algebra
\pause~with $\mathbf{s}(\cobesys{Y}{\stesys}{g}{p})$ defined
to be the unique extension homomorphism rendering the diagram
\begin{footnotesize}
\begin{equation*}
\begin{tikzcd}[ampersand replacement=\&]
\cobesys
  {\pullback{Y}{\stesyst}{g}{p\circ\eft\circ\ebd}}
  {\famesys{\famesys{\cobesys{Y}{\stesys}{g}{p}}}}
  {g^\ast(\ebd)}{g^\ast(\eft[2])}
  \ar[dotted]{dr}{\mathbf{s}(\cobesys{Y}{\stesys}{g}{p})}
  \ar{rr}
  \ar[bend right]{ddr}[swap]%
    { \pullbackpr{1}{g}{p\circ\eft\circ\ebd}%
        \circ%
      \pullbackpr{1}{g^\ast(\ebd)}{g^\ast(\eft[2])}%
      }
  \&
  {}\&
\cobesys{\stesyst}{\famesys{\famesys{\stesys}}}{\ebd}{\eft[2]}
  \ar{d}{\mathbf{s}(\stesys)}
  \\
  {}\&
\famesys{\cobesys{Y}{\stesys}{g}{p}}
  \ar{r}
  \ar{d}{\pullbackpr{1}{g}{p\circ\eft}}
  \&
\famesys{\stesys}
  \ar{d}{p\circ\eft}
  \\
  {}\&
Y
  \ar{r}[swap]{g}
  \&
X
\end{tikzcd}
\end{equation*}
\end{footnotesize}
commutative. 
\end{theorem}
\end{frame}

\begin{frame}
\frametitle{\bf Substitution algebras}
Thus, it makes sense to require that the pre-substitution operation itself is a
pre-substitution homomorphism.
\begin{definition}
A \important{substitution algebra} is a pre-substitution algebra $\stesys$ with the property that
$\mathbf{S}(\stesys)$ is a pre-substitution morphism.
\end{definition}
\pause
\begin{definition}
A \important{substitution homomorphism} is a pre-substitution homomorphism between substitution
algebras.
\end{definition}
\pause
\begin{theorem}
If $\stesys$ is a substitution algebra and $p:\stesysc\rightarrow X\leftarrow Y:g$,
then $\cobesys{Y}{\stesys}{g}{p}$ is also a substitution algebra.
\end{theorem}
\pause
\begin{theorem}
If $\stesys$ is a substitution algebra, then so is $\famesys{\stesys}$. 
\end{theorem}
\end{frame}

\begin{frame}
\frametitle{\bf Rules for substitution: substitution is compatible with itself}
\begin{align*}
& \inference
  { \jterm{\Gamma}{A}{a}
    \jterm{{{\Gamma}{A}}{P}}{Q}{g}
    \jfam{{{{\Gamma}{A}}{P}}{Q}}{R}
    }
  { \jfameq
      {{{\Gamma}{\subst{a}{P}}}{\subst{a}{Q}}}
      {\subst{a}{{g}{R}}}
      {\subst{{a}{g}}{{a}{R}}}
    }
\end{align*}
\pause
\begin{align*}
& \inference
  { \jterm{\Gamma}{A}{a}
    \jterm{{{\Gamma}{A}}{P}}{Q}{g}
    \jfam{{{{{\Gamma}{A}}{P}}{Q}}{R}}{S}
    }
  { \jfameq
      {{{{\Gamma}{\subst{a}{P}}}{\subst{a}{Q}}}{\subst{a}{{g}{R}}}}
      {\subst{a}{{g}{S}}}
      {\subst{{a}{g}}{{a}{S}}}
    }
  \\
& \inference
  { \jterm{\Gamma}{A}{a}
    \jterm{{{\Gamma}{A}}{P}}{Q}{g}
    \jterm{{{{{\Gamma}{A}}{P}}{Q}}{R}}{S}{k}
    }
  { \jtermeq
      {{{{\Gamma}{\subst{a}{P}}}{\subst{a}{Q}}}{\subst{a}{{g}{R}}}}
      {\subst{a}{{g}{S}}}
      {\subst{a}{{g}{k}}}
      {\subst{{a}{g}}{{a}{k}}}
    }
\end{align*}
\end{frame}

\begin{frame}
\frametitle{\bf Joining the theories of projections and substitution}
To join the two theories we have formulated on top of the theory of extension,
we need to provide rules describing:
\begin{itemize}
\item that weakening preserves substitution (weakening is a substitution homomorphism);\pause
\item that substitution preserves weakening (substitution is a weakening homomorphism);\pause
\item that substitution preserves units (substitution is a projection homomorphism);\pause
\item that weakenings are constant families;\pause
\item everything is invariant with respect to precomposition with units.
\end{itemize}
\end{frame}

\begin{frame}
\frametitle{\bf Pre-E-systems}
A \important{pre-E-system} is an extension algebra $\stesys$ with a
weakening operation $\mathbf{w}(\stesys)$, units
$\mathbf{i}(\stesys)$ and a substitution operation
$\mathbf{s}(\stesys)$ giving it both the structure of
a projection algebra and a substitution algebra, such
that in addition
\begin{itemize}
\item Weakening is a substitution homomorphism;
\item Substitution is a projection homomorphism.
\end{itemize}
\pause
\important{Pre-E-homomorphisms} are extension homomorphisms which
are both projection homomorphisms and substitution
homomorphisms.
\end{frame}

\begin{frame}
\frametitle{\bf Rules: Weakened families are constant families}
\begin{align*}
& \inference
  { \jfam{\Gamma}{A}
    \jfam{\Gamma}{B}
    \jterm{\Gamma}{A}{a}
    }
  { \jfameq
      {\Gamma}
      {\subst{a}{\ctxwk{A}{B}}}
      {B}
    }
\end{align*}
\pause
\begin{align*}
& \inference
  { \jfam{\Gamma}{A}
    \jfam{{\Gamma}{B}}{Q}
    \jterm{\Gamma}{A}{a}
    }
  { \jfameq{{\Gamma}{B}}{\subst{a}{\ctxwk{A}{Q}}}{Q}
    }
  \\
& \inference
  { \jterm{\Gamma}{A}{a}
    \jterm{{\Gamma}{B}}{Q}{g}
    }
  { \jtermeq{{\Gamma}{B}}{Q}{\subst{a}{\ctxwk{A}{g}}}{g}
    }
\end{align*}
\end{frame}

\begin{frame}
\frametitle{\bf Weakened families are constant families}
In a pre-E-system, we say that \important{weakened families are 
constant families} if the diagram
\begin{equation*}
\begin{tikzcd}[ampersand replacement=\&,column sep=12em]
\cobesys{\stesyst}{\famesys{\stesys}}{\eft\circ\ebd}{\eft}
  \ar{r}{ [ \catid{\stesyst},
            \mathbf{w}(\stesys)
              \circ
            (\pullback{\ebd}{\catid{\famesys{\stesys}}}{\eft}{\eft})
            ]
          }
  \ar{dr}[swap]{\pullbackpr{2}{\eft\circ\ebd}{\eft}}
  \&
\cobesys
  { \stesyst}
  { \famesys{\famesys{\stesys}}
    }
  { \ebd}
  { \eft[2]}
  \ar{d}{\mathbf{s}(\stesys)}
  \\
  {}\&
\famesys{\stesys}
\end{tikzcd}
\end{equation*}
commutes.
\end{frame}

\begin{frame}
\frametitle{\bf Rules: precomposition with a unit has no effect}
\begin{align*}
& \inference
  { \jfam{{\Gamma}{A}}{P}
    }
  { \jfameq{{\Gamma}{A}}{\subst{\idtm{A}}{\ctxwk{A}{P}}}{P}
    }
\end{align*}
\pause
\begin{align*}
& \inference
  { \jfam{{{\Gamma}{A}}{P}}{Q}
    }
  { \jfameq{{{\Gamma}{A}}{P}}{\subst{\idtm{A}}{\ctxwk{A}{Q}}}{Q}
    }
  \\
& \inference
  { \jterm{{{\Gamma}{A}}{P}}{Q}{g}
    }
  { \jtermeq
      {{{\Gamma}{A}}{P}}
      {Q}
      {\subst{\idtm{A}}{\ctxwk{A}{g}}}
      {g}
    }
\end{align*}
\end{frame}

\begin{frame}
\frametitle{\bf Invariance with respect to precomposition with units}
In a pre-E-system $\stesys$ we say that \important{everything is invariant
with respect to precomposition with units} if the diagram
\begin{equation*}
\begin{tikzcd}[ampersand replacement=\&,column sep=large]
\famesys{\famesys{\stesys}}
  \ar{r}
  \ar[equals]{ddrr}
  \&
\famesys{\cobesys{\stesysf}{\famesys{\stesys}}{\eft}{\eft}}
  \ar{r}[yshift=1ex]{[\pullbackpr{1}{\eft}{\eft}\circ\eft^\ast(\eft[2]),\famehom{\mathbf{w}(\stesys)}]}
  \&
\cobesys
  {\stesysf}
  {\famesys{\famesys{\famesys{\stesys}}}}
  {w(\stesys)_0\circ\Delta_{\eft}}
  {\eft[3]}
  \ar{d}{\pullback{\mathbf{i}(\stesys)}{\catid{\famesys{\famesys{\stesys}}}]}{\ebd[2]}{\eft[3]}}
  \\
  {}\&
  {}\&
\cobesys{\stesyst_2}{\famesys{\famesys{\famesys{\stesys}}}}{\ebd[2]}{\eft[3]}
  \ar{d}{\mathbf{s}(\famesys{\stesys})}
  \\
  {}\&
  {}\&
\famesys{\famesys{\stesys}}
\end{tikzcd}
\end{equation*}
commutes.
\end{frame}

\begin{frame}
\frametitle{\bf E-systems}
An \important{E-system} is an extension algebra with\pause
\begin{itemize}
\item a weakening operation and units, giving it the structure of a projection algebra;\pause
\item a substitution operation, giving it the structure of a substitution algebra;\pause
\item (an empty context and empty families);
\end{itemize}
such that\pause
\begin{itemize}
\item Weakening is a substitution homomorphism;\pause
\item Substitution is a projection homomorphism;\pause
\item Weakened families are constant families;\pause
\item Precomposition with identity functions leaves everything invariant.\pause
\item (The requirements regarding the empty context and families according to their rules)
\end{itemize}
\end{frame}

\begin{frame}
\frametitle{\bf Towards algebraic Homotopy Type Theory}
\begin{footnotesize}
\begin{equation*}
\begin{tikzcd}[ampersand replacement=\&,column sep=0em,row sep=1.3em]
 {}\& 
\textbf{Extension} 
  \ar[very thick,-stealth]{dr} 
  \ar[very thick,-stealth]{d} 
  \ar[very thick,-stealth]{dl}
  \\
\makebox[2cm]{\textbf{Weakening}} 
  \ar[very thick,-stealth]{d} 
  \& 
\textbf{Substitution} 
  \ar[very thick,-stealth]{dd} 
  \& 
\makebox[3cm]{\textbf{Empty context and families}} 
  \ar[very thick,-stealth]{ddl}
  \\
\makebox[2cm]{\textbf{Projection}} 
  \ar[very thick,-stealth]{dr}
  \\
  {}\& 
\textbf{Dependent Type Theory}
  \ar[very thick,-stealth]{dr}
  \ar[very thick,-stealth]{dl}
  \\
\makebox[2cm]{\bf Inductive types}
  \ar[very thick,-stealth]{ddr}
  \&\&
\makebox[2cm]{\bf $\Pi$-types}
  \ar[very thick,-stealth]{d}
  \\
  {}\&
  {}\&
\makebox[2cm]{\bf Universes}
  \ar[very thick,-stealth]{dl}
  \\
  {}\&
\mbox{\bf Univalent mathematics(?)}
  \ar[very thick,-stealth]{d}
  \\
  {}\&
\mbox{\bf Higher inductive types}
\end{tikzcd}
\end{equation*}
\end{footnotesize}
\end{frame}

\begin{frame}
\frametitle{\bf Appendix: weakening and substitution preserve each other}
\begin{footnotesize}
Weakening preserves substitution:
\begin{align*}
& \inference
  { \jfam{\Gamma}{A}
    \jterm{{\Gamma}{B}}{Q}{g}
    \jfam{{{\Gamma}{B}}{Q}}{R}
    }
  { \jfameq
      {{{\Gamma}{A}}{\ctxwk{A}{B}}}
      {\ctxwk{A}{\subst{g}{R}}}
      {\subst{\ctxwk{A}{g}}{\ctxwk{A}{R}}}
    }
  \\
& \inference
  { \jfam{\Gamma}{A}
    \jterm{{\Gamma}{B}}{Q}{g}
    \jfam{{{{\Gamma}{B}}{Q}}{R}}{S}
    }
  { \jfameq
      {{{{\Gamma}{A}}{\ctxwk{A}{B}}}{\ctxwk{A}{\subst{g}{R}}}}
      {\ctxwk{A}{\subst{g}{S}}}
      {\subst{\ctxwk{A}{g}}{\ctxwk{A}{S}}}
    }
  \\
& \inference
  { \jfam{\Gamma}{A}
    \jterm{{\Gamma}{B}}{Q}{g}
    \jterm{{{{\Gamma}{B}}{Q}}{R}}{S}{k}
    }
  { \jtermeq
      {{{{\Gamma}{A}}{\ctxwk{A}{B}}}{\ctxwk{A}{\subst{g}{R}}}}
      {\ctxwk{A}{\subst{g}{S}}}
      {\ctxwk{A}{\subst{g}{k}}}
      {\subst{\ctxwk{A}{g}}{\ctxwk{A}{k}}}
    }
\end{align*}
\pause
Substitution preserves weakening:
\begin{align*}
& \inference
  { \jterm{\Gamma}{A}{a}
    \jfam{{{\Gamma}{A}}{P}}{Q}
    \jfam{{{\Gamma}{A}}{P}}{R}
    }
  { \jfameq
      {{{\Gamma}{\subst{a}{P}}}{\subst{a}{Q}}}
      {\subst{a}{\ctxwk{Q}{R}}}
      {\ctxwk{\subst{a}{Q}}{\subst{a}{R}}}
    }
  \\
& \inference
  { \jterm{\Gamma}{A}{a}
    \jfam{{{\Gamma}{A}}{P}}{Q}
    \jfam{{{{\Gamma}{A}}{P}}{R}}{S}
    }
  { \jfameq
      {{{{\Gamma}{\subst{a}{P}}}{\subst{a}{Q}}}{\subst{a}{\ctxwk{Q}{R}}}}
      {\subst{a}{\ctxwk{Q}{S}}}
      {\ctxwk{\subst{a}{Q}}{\subst{a}{S}}}
    }
  \\
& \inference
  { \jterm{\Gamma}{A}{a}
    \jfam{{{\Gamma}{A}}{P}}{Q}
    \jterm{{{{\Gamma}{A}}{P}}{R}}{S}{k}
    }
  { \jtermeq
      {{{{\Gamma}{\subst{a}{P}}}{\subst{a}{Q}}}{\subst{a}{\ctxwk{Q}{R}}}}
      {\subst{a}{\ctxwk{Q}{S}}}
      {\subst{a}{\ctxwk{Q}{k}}}
      {\ctxwk{\subst{a}{Q}}{\subst{a}{k}}}
    }
\end{align*}
\end{footnotesize}
\end{frame}

\begin{frame}
\frametitle{\bf Appendix: substitution preserves units}
\begin{equation*}
\inference
  { \jterm{\Gamma}{A}{a}
    \jfam{{{\Gamma}{A}}{P}}{Q}
    }
  { \unfoldall{\jhomeq
      {{\Gamma}{\subst{a}{P}}}
      {\subst{a}{Q}}
      {\subst{a}{Q}}
      {\subst{a}{\idtm{Q}}}
      {\idtm{\subst{a}{Q}}}
    }}
\end{equation*}
\end{frame}

\begin{frame}
\frametitle{\bf Appendix: the empty context and family}
\begin{small}
Introduction rules for the empty context and family:
\begin{equation*}
\inference
  { }
  { \jctx{\emptyc}
    }
\qquad
\inference
  { \jctx{\Gamma}
    }
  { \jfam{\Gamma}{\emptyf[\Gamma]}
    }
\qquad
\inference
  { \jctxeq{\Gamma}{\Gamma'}
    }
  { \jfameq{\Gamma}{\emptyf[\Gamma]}{\emptyf[\Gamma']}
    }
\end{equation*}
\pause
Every context induces a family in the empty context:
\begin{align*}
& \inference
  { \jctx{\Gamma}
    }
  { \jfam{\emptyc}{i(\Gamma)}
    } 
  &
& \inference
  { \jctxeq{\Gamma}{\Delta}
    }
  { \jfameq{\emptyc}{i(\Gamma)}{i(\Delta)}
    }
\end{align*}
\pause
Compatibility rules for empty context and family with extension:
\begin{align*}
& \inference
  { \jctx{\Gamma}
    }
  { \jctxeq{\ctxext{\emptyc}{i(\Gamma)}}{\Gamma}
    }
& 
& \inference
  { \jctx{\Gamma}
    }
  { \jctxeq{\ctxext{\Gamma}{\emptyf}}{\Gamma}
    }
  \\
& \inference
  { \jfam{\Gamma}{A}
    }
  { \jfameq{\Gamma}{\ctxext{\emptyf}{A}}{A}
    }
&
& \inference
  { \jfam{\Gamma}{A}
    }
  { \jfameq{\Gamma}{{A}{\emptyf}}{A}
    }
  \\
& \inference
  { \jfam{\emptyc}{A}
    }
  { \jfameq{\emptyc}{i(\ctxext{\emptyc}{A})}{A}
    }
\end{align*}
\pause
and finally
\begin{align*}
& \inference
  { \jfam{\Gamma}{A}
    }
  { \jctxeq{\ctxext{\Gamma}{A}}{\ctxext{i(\Gamma)}{A}}
    }
& &
\inference
  { }
  { \jfameq{\emptyc}{i(\emptyc)}{\emptyf[\emptyc]}}
\end{align*}
\end{small}
\end{frame}

\begin{frame}
\frametitle{\bf Appendix: weakening and the empty families}
Weakening by the empty family:
\begin{align*}
& \inference
  { \jfam{\Gamma}{B}
    }
  { \jfameq{\Gamma}{\ctxwk{\emptyf}{B}}{B}
    }
  \\
& \inference
  { \jfam{{\Gamma}{B}}{Q}
    }
  { \jfameq{{\Gamma}{B}}{\ctxwk{\emptyf}{Q}}{Q}
    }
  \\
& \inference
  { \jterm{{\Gamma}{B}}{Q}{g}
    }
  { \jtermeq{{\Gamma}{B}}{Q}{\ctxwk{\emptyf}{g}}{g}
    }
\end{align*}
\pause
Weakening of the empty family:
\begin{align*}
& \inference
  { \jfam{\Gamma}{A}
    }
  { \jfameq{{\Gamma}{A}}{\ctxwk{A}{\emptyf}}{\emptyf}
    }
  \\
& \inference
  { \jfam{\Gamma}{A}
    \jfam{\Gamma}{B}
    }
  { \jfameq
    {{{\Gamma}{A}}{\ctxwk{A}{B}}}
    {\ctxwk{A}{\emptyf}}
    {\emptyf}
    }
\end{align*}
\pause
Compatibility of weakening of a family with weakening of a context:
\begin{equation*}
\inference
{ \jfam{\Gamma}{A}
  \jfam{\Gamma}{B}
  }
{ \jfameq{{\Gamma}{A}}{\ctxwk{A}{B}}{\ctxwk{A}{B}}
  }
\end{equation*}
\end{frame}

\begin{frame}
\frametitle{\bf Appendix: substitution of an empty family}
\begin{align*}
& \inference
  { \jterm{\Gamma}{A}{a}
    }
  { \jfameq{\Gamma}{\subst{a}{\emptyf}}{\emptyf}
    }
  \\
& \inference
  { \jterm{\Gamma}{A}{a}
    \jfam{{\Gamma}{A}}{P}
    }
  { \jfameq
      {{\Gamma}{\subst{a}{P}}}
      {\subst{a}{\emptyf}}
      {\emptyf}
    }
\end{align*}
Compatibility of substitution of a family with substitution of a context:
\begin{equation*}
\inference
{ \jfam{{\Gamma}{A}}{P}
  }
{ \jfameq{\Gamma}{\subst{a}{P}}{\subst{a}{P}}
  }
\end{equation*}
\end{frame}

\begin{frame}
\frametitle{\bf Appendix: units act as identity functions}
\begin{equation*}
\inference
  { \jterm{\Gamma}{A}{a}
    }
  { \jtermeq{\Gamma}{A}{\subst{a}{\idtm{A}}}{a}
    }
\end{equation*}
\\[\baselineskip]
Note that the empty family is needed to have this rule.
\end{frame}
\end{document}
