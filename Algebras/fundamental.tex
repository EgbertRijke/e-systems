\subsection{Semantics of the fundamental structure of dependent type theory}
For the semantics of the theory of E-systems, we will assume that $\cat{C}$ is a 
category with finite limits and
whenever we write a pullback, we assume that it is chosen. Recall that for
any morphism $f:A\to B$ in a category $\cat{C}$ with chosen pullbacks, there
is a functor
\begin{equation*}
f^\ast : \cat{C}/B\to\cat{C}/A.
\end{equation*}
As usual, when $g:X\to B$ is a morphism, we will write $f^\ast(X)$ for the
domain of $f^\ast(g)$. When there is more than one morphism $X\to B$ involved,
as will be the case below, we will write $\pullback{A}{X}{f}{g}$. The projections
will be written as $\pullbackpr{1}{f}{g}$ and $\pullbackpr{2}{f}{g}$. So in this notation, a
typical pullback diagram has the following form:
\begin{equation*}
\begin{tikzcd}[column sep=large]
\pullback{A}{X}{f}{g}
  \ar{r}{\pullbackpr{1}{f}{g}}
  \ar{d}[swap]{\pullbackpr{2}{f}{g}}
  &
A \ar{d}{f}
  \\
X \ar{r}[swap]{g}
  &
B
\end{tikzcd}
\end{equation*}
Also, when we have a commutative diagram of the form
\begin{equation*}
\begin{tikzcd}
A \ar{r}{f}
  \ar{d}{a}
  &
X \ar{d}
  & 
B \ar{l}[swap]{g}
  \ar{d}{b}
  \\
A'
  \ar{r}[swap]{f'}
  &
X'
  &
B'
  \ar{l}{g'}
\end{tikzcd}
\end{equation*}
we will denote the unique map from $\pullback{A}{B}{f}{g}$ to $\pullback{A'}{B'}{f'}{g'}$
such that the diagram
\begin{equation*}
\begin{tikzcd}
  {}
  & 
\pullback{A'}{B'}{f'}{g'}
  \ar{dd}
  \ar{rr}
  &
  &
B'
  \ar{dd}{g'}
  \\
\pullback{A}{B}{f}{g}
  \ar{dd}
  \ar[crossing over]{rr}
  \ar[dotted]{ur}{\pullback{a}{b}{f'}{g'}}
  &
  &
B \ar{ur}{b}
  \\
  {}
  &
A'
  \ar{rr}
  &
  &
X'
  \\
A \ar{rr}[swap]{f}
  \ar{ur}{a}
  &
  &
X \ar[crossing over,leftarrow]{uu}[near end,swap]{g}
  \ar{ur}
\end{tikzcd}
\end{equation*}
commutes, by $\pullback{a}{b}{f'}{g'}$. In the current work, we shall
write $A\times B$ for the pullback of $A\rightarrow 1\leftarrow B$, and
$\pi_1$ and $\pi_2$ for its projections (thus, no separate choice of
cartesian products is made).

\begin{defn}
A \emph{fundamental structure} $\stesys$ in $\cat{C}$ consists of a diagram of the form
\begin{equation*}
\begin{tikzcd}
\stesyst
  \ar{d}[swap]{\ebd}
  \\
\stesysf
  \ar{d}[swap]{\eft}
  \\
\stesysc
\end{tikzcd}
\end{equation*}
in $\cat{C}$. In this diagram, $C$ is the object of \emph{contexts}, $F$ is
the object of \emph{families}, and $T$ is the object of \emph{terms}.
\end{defn}
