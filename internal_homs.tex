\section{Elementary properties of E-systems}\label{sec:esys_props}

The theory of E-systems could be seen as a sort of category theory with only
projections and their sections. We elaborate on this point of view by defining
the pre-category of internal morphisms, which Lawvere might have called the
category of terms. 
In the pre-category of internal morphisms over a context $\Gamma$,
which we define in \autoref{subsec:im_cat},
a morphism $f:A\to B$ is a term of the constant family $W_A(B)$ over 
$\ctxext{\Gamma}{A}$. Rather than directly defining composition, we will define
for every $f:A\to B$ in context $\Gamma$, a pre-composition operation
$f_\ast$, which is going to be an E-homomorphism from 
$\cat{F}/\ctxext{\Gamma}{B}$ to $\cat{F}/\ctxext{\Gamma}{A}$. This allows us
to consider diagrams of the following form:
\begin{equation*}
\begin{tikzcd}
P \arrow[d,->>] \arrow[r,"g"] & Q \arrow[d,->>] \\
A \arrow[r,swap,"f"] & B
\end{tikzcd}
\end{equation*}
where we consider the morphism $g$ over $f$ to be an internal morphism from 
$P$ to $f_\ast(Q)$ in context $\ctxext{\Gamma}{A}$, capturing the
essence of being a morphism from the family $P$ over $A$ to the family
$Q$ over $B$, reindexed by $f$. 
Whenever we draw such diagrams, a double headed arrow will always
refer to a family, and an ordinary arrow will refer to an internal morphism,
possibly over another internal morphism.

A morphism $g$ over $f$ can be vertically composed with $f$ so
that we obtain a morphism $\jvcomp{P}{f}{g}:\ctxext{A}{P}\to\ctxext{B}{Q}$ in 
context $\Gamma$, and it can be horizontally composed with a morphism $k:Q\to R$
over $h:B\to C$ so that we obtain a morphism $\jfcomp{A}{f}{P}{g}{k}:P\to R$ 
over $\jcomp{A}{f}{h}:A\to C$. 
In \autoref{sec:interchange}, we will show interchange laws between these
sorts of composition. The interchange laws have as a consequence that if we
assert the commutativity of a large composed diagram.

The pre-category of internal morphisms forms a pre-category of which the families form
a subpre-category. This, together with the pre-composition operation described
earlier, gives rise to the auxiliary notion of E'-systems, which we define
in \autoref{sec:E'}. The thread running through this whole section is proving
that the notions of E-system and E'-system are the same. In \autoref{sec:E'}
we show directly that any E'-system is an E-system. We end this section by
showing that the converse also holds.

Before we begin this section, we introduce some more convenient notation.

\begin{defn}
Let $A\in\cat{F}/\Gamma$. Recall that $W_A:\cat{F}/\Gamma\to\cat{F}/\ctxext{\Gamma}{A}$ acts on objects,
morphisms and terms. We introduce the infix form of weakening by $A\in\cat{F}/\Gamma$ to be
$\ctxwk{A}{\blank}$. Thus, we will write
\begin{align*}
\ctxwk{A}{B} & \defeq W_A(B) & & \text{for $B\in\cat{F}/\Gamma$} \\
\ctxwk{A}{Q} & \defeq W_A(Q) & & \text{for $B\in\cat{F}/\Gamma$ and $Q\in\cat{F}/\ctxext{\Gamma}{B}$}\\
\ctxwk{A}{g} & \defeq W_A(g) & & \text{for $B\in\cat{F}/\Gamma$, $Q\in\cat{F}/\ctxext{\Gamma}{B}$ and $g\in T(Q)$}
\end{align*}
\end{defn}

\begin{defn}
Let $x\in T(A)$ for a family $A\in\cat{F}/\Gamma$. The infix form of substitution
by $x$ is taken to be $\subst{x}{\blank}$. Thus, we will write
\begin{align*}
\subst{x}{P} & \defeq S_x(P) & & \text{for $P\in\cat{F}/\ctxext{\Gamma}{A}$} \\
\subst{x}{Q} & \defeq S_x(Q) & & \text{for $P\in\cat{F}/\ctxext{\Gamma}{A}$ and $Q\in\cat{F}/\ctxext{{\Gamma}{A}}{P}$} \\
\subst{x}{g} & \defeq S_x(g) & & \text{for $P\in\cat{F}/\ctxext{\Gamma}{A}$, $Q\in\cat{F}/\ctxext{{\Gamma}{A}}{P}$ and $g\in T(Q)$}
\end{align*}
\end{defn}

\subsection{E'-systems}\label{sec:E'}

E'-systems can be thought of as strict versions of tribes%
\footnote{Tribes are an invention of Joyal, see \url{http://ncatlab.org/homotopytypetheory/files/Joyal.pdf}}.

\begin{defn}\label{defn:E'sys}
An \define{E'-system} consists of a pre-category $\cat{C}$ and a subpre-category $\cat{F}$ with
the same objects as $\cat{C}$, such that
\begin{enumerate}
\item $\cat{F}$ has a terminal object.
\item For any $f:\Delta\to\Gamma$ in $\cat{C}$ and any $A\in\cat{F}/\Gamma$, a
functorial choice of a pullback square 
\begin{equation*}
\begin{tikzcd}[column sep=large]
\ctxext{\Delta}{f_\ast(A)} \arrow[r,"{\pi_2(f,A)}"] \arrow[d,fib,swap,"{f_\ast(A)}"] & \ctxext{\Gamma}{A} \arrow[d,fib,"A"] \\
\Delta \arrow[r,"f"] & \Gamma
\end{tikzcd}
\end{equation*}
such that $f_\ast(A)\in\cat{F}/\Delta$. Explicitly, the functoriality
requirement is that
\begin{enumerate}
\item For any $f:\Delta\to\Gamma$, one has 
$f_\ast(\catid{\Gamma})\jdeq\catid{\Delta}$ and
\begin{equation*}
\pi_2(f,\catid{\Gamma})\jdeq f.
\end{equation*}
\item \label{defn:E'sys:b} For any $A\in\cat{F}/\Gamma$, one has $(\catid{\Gamma})_\ast(A)\jdeq A$ and
\begin{equation*}
\pi_2(\catid{\Gamma},A)\jdeq\catid{\ctxext{\Gamma}{A}}
\end{equation*}
\item \label{defn:E'sys:c} For any $f:\Delta\to\Gamma$, $g:\greek{E}\to\Delta$ and $A\in\cat{F}/\Gamma$, one
has $(f\circ g)_\ast(A)\jdeq g_\ast(f_\ast(A))$ and 
\begin{equation*}
\pi_2(f\circ g,A)\jdeq \pi_2(f,A)\circ\pi_2(g,f_\ast(A))
\end{equation*}
\item For any $P\in\cat{F}/\ctxext{\Gamma}{A}$ and $f:\Delta\to\Gamma$, one has
$f_\ast(P)\jdeq\pi_2(f,A)_\ast(P)$ and 
\begin{equation*}
\pi_2(f,\ctxext{A}{P}) \jdeq \pi_2(\pi_2(f,A),P)
\end{equation*}
\end{enumerate}
\end{enumerate}
For any $f:\Delta\to\Gamma$ we shall write $f_\ast$ for the induced functor
$\cat{F}/\Gamma\to\cat{F}/\Delta$. 
\end{defn}

\begin{rmk}
Without a terminal object in $\cat{F}$, the pre-category $\cat{C}$ does not correspond
directly to anything in an E-system.
\end{rmk}

\begin{rmk}
For E'-systems, we do not require that every pullback of $A\in\cat{F}/\Gamma$ along $f:\Delta\to\Gamma$
is in $\cat{F}/\Delta$, just that there is a choice of such a pullback which is
in $\cat{F}/\Delta$. Thus, the definition of E'-system is slightly different from
Taylor's notion of display map category. Furthermore, the class of display maps in
a display map category is not required to be closed under composition, and neither
is there assumed to exist an object $T$ such that for every object $X$ there is a 
unique display map $X\to T$.
\end{rmk}

\begin{rmk}
Also, we do not require that $\cat{C}$ has a terminal object, and neither do
we require that $\cat{F}$ contains all the isomorphisms of $\cat{C}$. These are
essential differences with Joyal's notion of tribes.   
\end{rmk}

\begin{defn}
For any $\Gamma\in\cat{C}$, define $\cat{C}_\cat{F}/\Gamma$ to be the full
subpre-category of $\cat{C}/\Gamma$ with the same objects as $\cat{F}/\Gamma$. 
\end{defn}

\begin{lem}\label{lem:pb_selfdistributive}
Let $\mathbb{E}$ be an E'-system. Then $f_\ast$ can be extended to a functor from
$\cat{C}_\cat{F}/\Gamma$ to $\cat{C}_\cat{F}/\Delta$, for each $f:\Delta\to\Gamma$, and moreover the diagram
\begin{equation*}
\begin{tikzcd}[column sep=large]
\cat{C}_\cat{F}/\ctxext{\Gamma}{B} \arrow[r,"{f_\ast/B}"] \arrow[d,swap,"{g_\ast}"]
& \cat{C}_\cat{F}/\ctxext{\Delta}{f_\ast(B)} \arrow[d,"{(f_\ast(g))_\ast}"] \\
\cat{C}_\cat{F}/\ctxext{\Gamma}{A} \arrow[r,swap,"{f_\ast/A}"]
& \cat{C}_\cat{F}/\ctxext{\Delta}{f_\ast(A)}
\end{tikzcd}
\end{equation*}
commutes for every $g:A\to B$ in $\cat{C}_\cat{F}/\Gamma$.
\end{lem}

\begin{proof}
It is straightforward to use the universal property of the pullback to extend
$f_\ast$ to a functor from $\cat{C}_\cat{F}/\Gamma$ to $\cat{C}_\cat{F}/\Delta$.
To see that the diagram commutes, we calculate
\begin{align*}
f_\ast/A\circ g_\ast
& \jdeq
\pi_2(f,A)_\ast\circ g_\ast \\
& \jdeq
(g\circ\pi_2(f,A))_\ast \\
& \jdeq
(\pi_2(f,B)\circ f_\ast(g))_\ast \\
& \jdeq
(f_\ast(g))_\ast\circ\pi_2(f,B)_\ast \\
& \jdeq
(f_\ast(g))_\ast\circ f_\ast/B.\qedhere
\end{align*}
\end{proof}

\begin{thm}
An E-system is the same thing as an E'-system.
\end{thm}

\begin{proof}
We define for any $A\in\cat{F}/\Gamma$ the set
\begin{equation*}
T(A)\defeq\{x:\Gamma\to\ctxext{\Gamma}{A}\mid A\circ x\jdeq\catid{\Gamma}\}.
\end{equation*}
Since each functor $f_\ast$ is extended to all of $\cat{C}_{\cat{F}}/\Gamma$,
it follows immediately that those are functors with term structure.

We define for any $A\in\cat{F}/\Gamma$, the functor $W_A\defeq A_\ast:\cat{F}/\Gamma\to\cat{F}/\ctxext{\Gamma}{A}$.
Likewise, we define for any $x\in T(A)$, the functor $S_x\defeq x_\ast:\cat{F}/\ctxext{\Gamma}{A}\to\cat{F}/\Gamma$.
We also define $\tfid{A}:T(W_A(A))$ by the universal property of pullbacks:
\begin{equation*}
\begin{tikzcd}[column sep=large]
\ctxext{\Gamma}{A} \arrow[drr,bend left=15,equals] \arrow[ddr,bend right=15,equals] \arrow[dr,densely dotted,near end,"{\tfid{A}}"] \\
& \ctxext{{\Gamma}{A}}{W_A(A)} \arrow[r,swap,"{\pi_2(A,A)}"] \arrow[d,fib] & \ctxext{\Gamma}{A} \arrow[d,fib] \\
& \ctxext{\Gamma}{A} \arrow[r,swap,"A"] & \Gamma
\end{tikzcd}
\end{equation*}
As an immediate consequence of \autoref{lem:pb_selfdistributive}, we get that
each functor $W_A$ and $S_x$ is both a weakening functor and a substitution functor.
Since each $W_A$ and $S_x$ preserves identity morphisms, it also preserves the
terms $\tfid{A}$. 

It remains to verify the rest of the conditions of E-systems.
\begin{itemize}
\item To show that weakened families are constant, note that
\begin{equation*}
S_x\circ W_A\jdeq x_\ast\circ A_\ast\jdeq (A\circ x)_\ast\jdeq (\catid{\Gamma})_\ast\jdeq \catid{\cat{C}_{\cat{F}}/\Gamma}.
\end{equation*}
\item The identity terms are neutral for pre-composition:
\begin{equation*}
S_{\tfid{A}}\circ W_A/A \jdeq S_{\tfid{A}}\circ \pi_2(A,A)_\ast \jdeq (\pi_2(A,A)\circ\tfid{A})_\ast \jdeq
(\catid{\ctxext{\Gamma}{A}})_\ast \jdeq \catid{\cat{C}_{\cat{F}}/\ctxext{\Gamma}{A}} 
\end{equation*}
\item The identity terms behave like identity functions: by the universal property,
$S_x(\tfid{A})$ is the unique section of $A$ such that the square
\begin{equation*}
\begin{tikzcd}[column sep=6em]
\Gamma \arrow[d,swap,"{S_x(\tfid{A})}"] \arrow[r,"{\pi_2(x,\catid{\ctxext{\Gamma}{A}})}"] & \ctxext{\Gamma}{A} \arrow[d,"{\tfid{A}}"] \\
\ctxext{\Gamma}{A} \arrow[r,swap,"{\pi_2(x,W_A(A))}"] & \ctxext{{\Gamma}{A}}{W_A(A)}
\end{tikzcd}
\end{equation*}
commutes. Thus, it suffices to show that this square also commutes with $x$ in the place of
$S_x(\tfid{A})$. Note that $\pi_2(x,\catid{\ctxext{\Gamma}{A}})\jdeq x$. 
Since $\ctxext{{\Gamma}{A}}{W_A(A)}$ is itself a pullback, it suffices
and it is straightforward to verify the equalities
\begin{align*}
W_A(A)\circ\pi_2(x,W_A(A))\circ x & \jdeq W_A(A)\circ\idtm{A}\circ x \\
\pi_2(A,A)\circ\pi_2(x,W_A(A))\circ x & \jdeq \pi_2(A,A)\circ\idtm{A}\circ x.\qedhere
\end{align*}
\end{itemize}
\end{proof}



\begin{comment}
We then demonstrate how to define a category internal to an E-system. The main
point of this lies mainly in demonstrating how the operations of E-systems can
be used to define a category, because it is no surprise that a category can be
defined in type theory. Nevertheless, it is noteworthy that we need no further 
conditions on E-systems.
This is unlike the situation in category theory, where we need
to assume that the ambient category has finite limits in order to say what an
internal category is. In fact, it is possible to define an E-system internal
to an E-system, just using the algebraic structure of an E-system, and actually
defining an internal category was the hardest part of that definition.
\end{comment}

\subsection{The pre-category of internal morphisms}\label{subsec:im_cat}

\begin{defn}
For every $A,B\in\cat{F}/\Gamma$ we define the set 
\begin{equation*}
\thom{A}{B}\defeq T(\ctxwk{A}{B}).
\end{equation*}
An element $f\in \thom{A}{B}$ is called an \define{internal morphism in context
$\Gamma$}. We sometimes write $\jhom{\Gamma}{A}{B}{f}$ to indicate that $f$ is
an internal morphism over $\Gamma$, or we may draw a diagram of the form
\begin{equation*}
\begin{tikzcd}[column sep=tiny]
A \arrow[rr,"f"] \arrow[dr,fib] & & B \arrow[dl,fib] \\
& \Gamma
\end{tikzcd}
\end{equation*}
but we shall generally omit the arrows down to $\Gamma$ and say instead that we have
a diagram in context $\Gamma$. 
\end{defn}

\begin{rmk}
Note that $\thom{\catid{\Gamma}}{A}\jdeq T(A)$ for any $A\in\cat{F}/\Gamma$, 
because we have 
$W_{\catid{\ctxext{\Gamma}{A}}}\jdeq \catid{\cat{F}/\ctxext{\Gamma}{A}}$.

Note also that $\thom{\ctxext{A}{P}}{B}\jdeq \thom{P}{\ctxwk{A}{B}}$ 
for any $P\in\cat{F}/\ctxext{\Gamma}{A}$ and $B\in\cat{F}/\Gamma$,
because $W_{\ctxext{A}{P}}\jdeq W_P\circ W_A$.
Once we have established a pre-category of which the morphisms are given by 
$\thom{\blank}{\blank}$, we therefore get that 
\begin{equation*}
\ctxext{A}{(\blank)}\dashv W_A.
\end{equation*}
The right adjoint to weakening by $A$, if it exists, will be the dependent
product $\Pi_A$. 
\end{rmk}

\begin{defn}
Let $A,B\in\cat{F}/\Gamma$.
For any $f\in\thom{A}{B}$ we define the \define{pre-composition} E-homomorphism
\begin{equation*}
f_\ast \defeq S_f\circ W_A/B : \cat{F}/\ctxext{\Gamma}{B}\to\cat{F}/\ctxext{\Gamma}{A}.
\end{equation*}
Furthermore, for $g\in\thom{B}{C}$ we may consider $W_B(C)\in\cat{F}/\ctxext{\Gamma}{B}$ as a morphism from $W_B(C)$ to $\catid{\ctxext{\Gamma}{B}}$. Therefore, $f_\ast$ acts on the terms of
$W_B(C)$, so it makes sense to define $g\circ f\defeq f_\ast(g)$. In fact, it makes
sense to let $(\blank)\circ f$ be the infix notation for $f_\ast$. 
\end{defn}

\begin{rmk}
Note that for $\jhom{\Gamma}{B}{C}{g}$, 
we have $g\circ f\in T(S_f(W_A/B(W_B(C))))$, whereas we would like that
$g\circ f\in\thom{A}{C}$. More generally, we can show that
\begin{equation*}
S_f\circ (W_A/B)\circ W_B\jdeq W_A.
\end{equation*}
Since weakening is a weakening homomorphism, we have
\begin{equation*}
S_f\circ (W_A/B)\circ W_B\jdeq S_f\circ W_{W_A(B)}\circ W_A
\end{equation*}
By \autoref{tTf:StWf_id} of \autoref{defn:esystems} it we get that
\begin{equation*}
S_f\circ W_{W_A(B)}\circ W_A\jdeq W_A.
\end{equation*}
\end{rmk}

\begin{rmk}
To see how this extended notion of precomposition works, suppose we have a diagram
\begin{equation*}
\begin{tikzcd}[column sep=tiny]
Q \arrow[rr,"g"] \arrow[dr,fib] & & R \arrow[dl,fib] \\
& B
\end{tikzcd}
\end{equation*}
in context $\Gamma$, i.e.~$\jhom{\ctxext{\Gamma}{B}}{Q}{R}{g}$, and let $\jhom{\Gamma}{A}{B}{f}$. Precomposing by $f$ gives
us a diagram
\begin{equation*}
\begin{tikzcd}[column sep=tiny]
\jcomp{A}{f}{Q} \arrow[rr,"\jcomp{A}{f}{g}"] \arrow[dr,fib] & & \jcomp{A}{f}{R} \arrow[dl,fib] \\
& A
\end{tikzcd}
\end{equation*}
in context $\Gamma$.
\end{rmk}

\begin{rmk}
Note that \autoref{f:SidfWff_id} of \autoref{defn:esystems} asserts precisely
that ${(\tfid{A})}_\ast\jdeq \catid{\cat{F}/\ctxext{\Gamma}{A}}$ for any $A\in\cat{F}/\Gamma$. In
particular, it follows that $g\circ\tfid{A}\jdeq g$ for any $g\in\thom{A}{B}$
\end{rmk}

\begin{lem}\label{lem:compcomp}
For any $f\in\thom{A}{B}$ and $g\in\thom{B}{C}$ we have $f_\ast\circ g_\ast\jdeq (g\circ f)_\ast$. 
\end{lem}

\begin{proof}
\begin{align*}
f_\ast\circ g_\ast & \jdeq S_f\circ (W_A/B) \circ S_g \circ (W_B/C)\\
& \jdeq S_f\circ S_{W_A(g)}\circ(W_A/\ctxext{B}{W_B(C)})\circ (W_B/C) \\
& \jdeq S_{S_f(W_A(g))}\circ (S_f/W_A(W_B(C)))\circ (W_A/\ctxext{B}{W_B(C)}) \circ W_B/C \\
& \jdeq S_{S_f(W_A(g))}\circ ((S_f\circ (W_A/B) \circ W_B)/C) \\
& \jdeq S_{S_f(W_A(g))}\circ ((S_f\circ W_{W_A(B)}\circ W_A)/C) \\
& \jdeq S_{S_f(W_A(g))}\circ W_A/C \\
& \jdeq (g\circ f)_\ast.\qedhere
\end{align*}
\end{proof}

\begin{thm}
For each object $\Gamma$ of an E-system $\mathbb{E}$, we have a pre-category
$\mathbb{C}_I(\mathbb{E},\Gamma)$ of internal morphisms of $\mathbb{E}$,
consisting of the same objects as $\cat{F}/\Gamma$ and the hom-sets 
$\thom{A}{B}$. 
\end{thm}

\begin{proof}
The fact that composition is associative is a direct corollary of
\autoref{lem:compcomp}. The axiom $(\tfid{A})_\ast\jdeq\catid{\ctxext{\Gamma}{A}}$
implies that the identity morphisms satisfy the right identity law. It remains
to show that $\tfid{B}\circ f\jdeq f$. This is a simple calculation:
\begin{equation*}
\tfid{B}\circ f 
  \jdeq
\subst{f}{\ctxwk{A}{\tfid{B}}}
  \jdeq
\subst{f}{\tfid{\ctxwk{A}{B}}}
  \jdeq
f.\qedhere
\end{equation*}
\end{proof}

\begin{rmk}
Since any E-homomorphism is both a substitution homomorphism and a weakening
homomorphism, it follows immediately that any E-homomorphism preserves any
$f_\ast$, i.e.~for any $f\in\thom{A}{B}$ in $\mathcal{E}$ and any E-homomorphism
$F:\mathcal{E}\to\mathcal{D}$, the diagram
\begin{equation*}
\begin{tikzcd}[column sep=large]
\mathcal{E}/\ctxext{\Gamma}{A} \arrow[r,"F/\ctxext{\Gamma}{A}"] & \mathcal{D}/F(\ctxext{\Gamma}{A})
  \\
\mathcal{E}/\ctxext{\Gamma}{B} \arrow[u,"f_\ast"] \arrow[r,swap,"F/\ctxext{\Gamma}{B}"] & \mathcal{D}/F(\ctxext{\Gamma}{B}) \arrow[u,swap,"F(f)_\ast"]
\end{tikzcd}
\end{equation*}
commutes. This applies to weakening, substitution and pre-composition, and it
follows that for any E-homomorphism $F:\mathcal{E}\to\mathcal{D}$ and any
$\Gamma\in\mathcal{E}$, we have a functor $\mathbb{C}_I(F,\Gamma):
\mathbb{C}_I(\mathcal{E},\Gamma)\to\mathbb{C}_I(\mathcal{D},F(\Gamma))$.
\end{rmk}

The following fact will be useful:

\begin{lem}\label{lem:compW_W}
Let $f\in\thom{A}{B}$ be an internal morphism in context $\Gamma$. Then one has
\begin{equation*}
f_\ast\circ W_B\jdeq W_A.
\end{equation*}
\end{lem}

\begin{proof}
The proof is a simple calculation:
\begin{equation*}
f_\ast\circ W_B\jdeq S_f\circ W_A/B\circ W_B\jdeq S_f\circ W_{W_A(B)}\circ W_A
\jdeq W_A.\qedhere
\end{equation*}
\end{proof}

\subsection{Pairing and the projections}

The composition $\ctxext{A}{P}$ of $A\in\cat{F}/\Gamma$ and $P\in\cat{F}/\ctxext{\Gamma}{A}$
behaves like a strict $\Sigma$-type. We may define the pairing term
$\typefont{pair}^{A,P}\defeq\tfid{\ctxext{A}{P}}\in T(W_P(W_A(\ctxext{A}{P})))$
and the projections and prove several useful properties about them. The strictness
is found, among other things, in the fact that we can prove judgmental $\eta$-equality,
and that pairing is strictly associative.

In this section we will make more extensive use of the infix form of the
weakening and substitution operations.

\begin{defn}
Let $x\in T(A)$ and $u\in T(S_x(P))$ for $A\in\cat{F}/\Gamma$ and $P\in\cat{F}/\ctxext{\Gamma}{A}$. 
We define the \define{term extension of $x$ and $u$} to be
\begin{equation*}
\tmext{x}{u}\defeq \subst{u}{{x}{\tfid{\ctxext{A}{P}}}}\in T(\ctxext{A}{P}).
\end{equation*}
\end{defn}

To prove anything about the term $\tmext{x}{u}$, we need the following property.

\begin{thm}\label{subst_by_tmext}
Let $x\in T(A)$ and $u\in T(S_x(P))$ for $A\in\cat{F}/\Gamma$ and $P\in\cat{F}/\ctxext{\Gamma}{A}$.
Then we have
\begin{equation*}
S_{\tmext{x}{u}}\jdeq S_u\circ (S_x/P):\cat{F}/\ctxext{\Gamma}{{A}{P}}\to\cat{F}/\Gamma
\end{equation*}
\end{thm}

\begin{proof}
\begin{align*}
S_{\tmext{x}{u}} & \jdeq S_{S_u(S_x(\tfid{\ctxext{A}{P}}))} \\
& \jdeq S_{S_u(S_x(\tfid{\ctxext{A}{P}}))}\circ (S_u\circ W_{S_x(P)})\circ (S_x\circ W_{A}) \\
& \jdeq S_u \circ S_{S_x(\tfid{\ctxext{A}{P}})} \circ W_{S_x(P)}\circ S_x\circ W_A \\
& \jdeq S_u \circ (S_x/P) \circ S_{\tfid{\ctxext{A}{P}}} \circ W_P\circ W_A \\
& \jdeq S_u \circ (S_x/P) \circ S_{\tfid{\ctxext{A}{P}}} \circ W_{\ctxext{A}{P}}\\
& \jdeq S_u \circ (S_x/P).\qedhere
\end{align*}
\end{proof}

\begin{defn}
Let $A\in\cat{F}/\Gamma$ and $P\in\cat{F}/\ctxext{\Gamma}{A}$. We define
\begin{align*}
\cprojfstf{A}{P} & \defeq \ctxwk{P}{\tfid{A}}\in\thom{\ctxext{A}{P}}{A}\\
\cprojsndf{A}{P} & \defeq \tfid{P}\in\thomd{\cprojfstf{A}{P}}{\ctxext{A}{P}}{P}
\end{align*}
\end{defn}

\begin{cor}\label{cor:pairing_ehom}
Let $F:\mathcal{E}\to\mathcal{D}$ be an E-homomorphism and let $x\in T(A)$ and
$u\in T(S_x(P))$. Then we have
\begin{equation*}
F(\tmext{x}{u})\jdeq \tmext{F(x)}{F(u)}.
\end{equation*}
Also, we have
\begin{align*}
F(\cprojfstf{A}{P}) & \jdeq \cprojfstf{F(A)}{F(P)} \\
F(\cprojsndf{A}{P}) & \jdeq \cprojsndf{F(A)}{F(P)}.
\end{align*}
\end{cor}

\begin{thm}\label{thm:famext_up}
For any $A\in\cat{F}/\Gamma$ and $P\in\cat{F}/\ctxext{\Gamma}{A}$,
the map
\begin{equation*}
(x,u)\mapsto \tmext{x}{u}:\big(\bigsqcup\nolimits_{x\in T(A)} T(\subst{x}{P})\big)\to T(\ctxext{A}{P}).
\end{equation*}
is an isomorphism.  
\end{thm}

\begin{proof}
Lets write $\varphi$ for the indicated map. The inverse $\psi$ of $\varphi$ is defined
by $w\mapsto (\subst{w}{\cprojfstf{A}{P}},\subst{w}{\cprojsndf{A}{P}})$. We have
to show that for any $x\in T(A)$ and $u\in T(\subst{x}{P})$, one has
\begin{align*}
\subst{\tmext{x}{u}}{\cprojfstf{A}{P}} & \jdeq x\\
\subst{\tmext{x}{u}}{\cprojsndf{A}{P}} & \jdeq u,
\end{align*}
and that for any $w\in T(\ctxext{A}{P})$, one has
\begin{equation*}
\tmext{\subst{w}{\cprojfstf{A}{P}}}{\subst{w}{\cprojsndf{A}{P}}} \jdeq w.
\end{equation*}
To show that $\subst{\tmext{x}{u}}{\cprojfstf{A}{P}} \jdeq x$, we use that
$S_{\tmext{x}{u}}\jdeq S_u\circ S_x/P$ to show that
\begin{align*}
\subst{\tmext{x}{u}}{\cprojfstf{A}{P}}
  & \jdeq 
\subst{\tmext{x}{u}}{\ctxwk{P}{\tfid{A}}} \\
  & \jdeq
\subst{u}{{x}{\ctxwk{P}{\tfid{A}}}} \tag{By \autoref{subst_by_tmext}}\\
  & \jdeq
\subst{u}{\ctxwk{\subst{x}{P}}{\subst{x}{\tfid{A}}}} \\
  & \jdeq
\subst{x}{\tfid{A}} \\
  & \jdeq
x
\end{align*}
To show that $\subst{\tmext{x}{u}}{\cprojsndf{A}{P}}\jdeq u$, note that
\begin{equation*}
\subst{\tmext{x}{u}}{\cprojsndf{A}{P}}
  \jdeq
\subst{u}{{x}{\tfid{P}}}
  \jdeq
\subst{u}{\tfid{\subst{x}{P}}}
  \jdeq
u
\end{equation*}
Therefore, $\psi$ is a left-inverse of $\varphi$. To show that it is also a
right inverse, note that \autoref{cor:pairing_ehom} gives us that
\begin{equation*}
\tmext{\cprojfst{A}{P}{w}}{\cprojsnd{A}{P}{w}}
  \jdeq
\subst{w}{(\tmext{\cprojfstf{A}{P}}{\cprojsndf{A}{P}})}.
\end{equation*}
Thus, it suffices to show that $\tmext{\cprojfstf{A}{P}}{\cprojsndf{A}{P}}\jdeq
\tfid{\ctxext{A}{P}}$, which is straightforward.
\end{proof}

\begin{lem}\label{lem:tmext_assoc}
For any $x\in T(A)$, $u\in T(S_x(P))$ and $v\in T(S_{\tmext{x}{u}}(Q))$ we have
\begin{equation*}
\tmext{{x}{u}}{v}\jdeq \tmext{x}{{u}{v}}\in T(\ctxext{{A}{P}}{Q}).
\end{equation*}
\begin{comment}
Also, we have
\begin{align*}
\cprojfstf{A}{P}\circ\cprojfstf{\ctxext{A}{P}}{Q} & \jdeq\cprojfstf{A}{\ctxext{P}{Q}} \\
\cprojsndf{A}{P}\circ\cprojfstf{\ctxext{A}{P}}{Q} & \jdeq \cprojfstf{P}{Q}\circ\cprojsndf{A}{\ctxext{P}{Q}} \\
\cprojsndf{\ctxext{A}{P}}{Q} & \jdeq \cprojsndf{P}{Q}\circ\cprojsndf{A}{\ctxext{P}{Q}}.
\end{align*}
\end{comment}
\end{lem}

\begin{proof}
By \autoref{subst_by_tmext}, we have $S_v\circ S_{\tmext{x}{u}}\jdeq S_v\circ S_u\circ S_x \jdeq S_{\tmext{u}{v}}\circ S_x$,
so associativity of term extension follows.
\end{proof}

\begin{comment}
\begin{lem}
Let $f\in\thom{A}{\ctxext{B}{Q}}$ Then we have
\begin{equation*}
f \jdeq \tmext{\jcomp{A}{f}{\cprojfstf{B}{Q}}}{\jcomp{A}{f}{\cprojsndf{B}{Q}}}.
\end{equation*}
Alternatively, when $f_0\in\thom{A}{B}$ and $f_1\in\thom{A}{\jcomp{A}{f}{Q}}$,
then we have $\tmext{f_0}{f_1}\in\thom{A}{\ctxext{B}{Q}}$ and
\begin{align*}
\jcomp{A}{\tmext{f_0}{f_1}}{\cprojfstf{B}{Q}} & \jdeq f_0 \\
\jcomp{A}{\tmext{f_0}{f_1}}{\cprojsndf{B}{Q}} & \jdeq f_1.
\end{align*}
\end{lem}

\begin{proof}
Straightforward
\end{proof}
\end{comment}

We end this section by showing that weakening is the same operation as pre-composition
with a first projection.

\begin{thm}\label{precomp_by_proj}
Precomposition with $\cprojfstf{A}{P}$ is weakening by $P$, i.e.
\begin{equation*}
(\cprojfstf{A}{P})_\ast \jdeq W_P.
\end{equation*}
\end{thm}

\begin{proof}
\begin{align*}
(\cprojfstf{A}{P})_\ast & \jdeq S_{\cprojfstf{A}{P}}\circ W_{\ctxext{A}{P}} \\
& \jdeq S_{W_P(\tfid{A})}\circ W_P\circ W_A \\
& \jdeq W_P\circ S_{\tfid{A}}\circ W_A \\
& \jdeq W_P.\qedhere 
\end{align*}
\end{proof}

\subsection{The interchange laws}\label{sec:interchange}

We are now in the position to define vertical and horizontal composition, and
prove properties of them.

\begin{defn}
Let $\jhom{\Gamma}{A}{B}{f}$ and $\jfhom{\Gamma}{A}{B}{f}{P}{Q}{F}$. Then we
define 
\begin{equation*}
\jhomdefn{\Gamma}{{A}{P}}{{B}{Q}}{\jvcomp{P}{f}{F}}{\unfold{\jvcomp{P}{f}{F}}}.
\end{equation*}
Whenever we say that we have a diagram of the form
\begin{equation*}
\begin{tikzcd}
R \arrow[r,"f_2"] \arrow[d,fib] &
S \arrow[d,fib] \\
P \arrow[r,"f_1"] \arrow[d,fib] &
Q \arrow[d,fib] \\
A \arrow[r,"f_0"] &
B
\end{tikzcd}
\end{equation*}
we mean that we have
$f_0\in\thom{A}{B}$, $f_1\in\thomd{f_0}{P}{Q}$ and 
$f_2\in\thomd{\jvcomp{P}{f_0}{f_1}}{R}{S}$.
\end{defn}

\begin{cor}
Let $\jhom{\Gamma}{A}{B}{f}$ and $\jfhom{\Gamma}{A}{B}{f}{P}{Q}{F}$. Then we
have
\begin{equation*}
H(\jvcomp{P}{f}{F})\jdeq \jvcomp{H(P)}{H(f)}{H(F)}
\end{equation*}
for any E-homomorphism $H:\mathbb{E}\to\mathbb{D}$.
\end{cor}

\begin{lem}
Vertical composition is associative.
\end{lem}

\begin{proof}
Consider the diagram
\begin{equation*}
\begin{tikzcd}
R \arrow[r,"f_2"] \arrow[d,fib] &
S \arrow[d,fib] \\
P \arrow[r,"f_1"] \arrow[d,fib] &
Q \arrow[d,fib] \\
A \arrow[r,"f_0"] &
B
\end{tikzcd}
\end{equation*}
in context $\Gamma$.
Because weakening distributes over term extension, and term extension is
associative, we have
\begin{align*}
\jvcomp{R}{\jvcomp{P}{f_0}{f_1}}{f_2}
  & \jdeq
\tmext{\ctxwk{R}{\tmext{\ctxwk{P}{f_0}}{f_1}}}{f_2}
  \\
  & \jdeq
\tmext{\tmext{\ctxwk{R}{{P}{f_0}}}{\ctxwk{R}{f_1}}}{f_2}
  \\
  & \jdeq
\tmext{\ctxwk{\ctxext{P}{R}}{f_0}}{\tmext{\ctxwk{R}{f_1}}{f_2}}
  \tag{By \autoref{lem:tmext_assoc}}
  \\
  & \jdeq
\jvcomp{{P}{R}}{f_0}{\jvcomp{R}{f_1}{f_2}}.\qedhere
\end{align*}
\end{proof}

\begin{defn}
Let $\jhom{\Gamma}{A}{B}{f}$ and $\jfhom{\Gamma}{A}{B}{f}{P}{Q}{F}$. Then we
define the E-homomorphism 
\begin{equation*}
F_\circledast\defeq F_\ast\circ (f_\ast/Q):
\mathbb{E}/\ctxext{{\Gamma}{B}}{Q}\to\mathbb{E}/\ctxext{{\Gamma}{A}}{P}.
\end{equation*}
The infix notation of $F_\circledast$ is taken to be $\jfcomp{\blank}{\blank}{\blank}{F}{\blank}$.
\end{defn}

\begin{lem}\label{lem:three-composition}
Let $\jhom{\Gamma}{A}{B}{f}$ and $\jfhom{\Gamma}{A}{B}{f}{P}{Q}{F}$. Then we
have the equality
\begin{equation*}
F_\circledast\jdeq (\jvcomp{P}{f}{F})_\ast.
\end{equation*}
\end{lem}

\begin{proof}
\begin{align*}
F_\ast\circ (f_\ast/Q)
  & \jdeq
S_F\circ W_P\circ S_f/(W_A(Q))\circ W_A/\ctxext{B}{Q}
  \\
  & \jdeq
S_F\circ S_{W_P(f)}/W_P(W_A(Q))\circ W_P/W_A(\ctxext{B}{Q})\circ W_A/\ctxext{B}{Q}
  \\
  & \jdeq
S_F\circ S_{W_P(f)}/W_P(W_A(Q))\circ W_{\ctxext{A}{P}}/\ctxext{B}{Q}
  \\
  & \jdeq
S_{\tmext{W_P(f)}{F}}\circ W_{\ctxext{A}{P}}/\ctxext{B}{Q}
  \tag{By \autoref{subst_by_tmext}}
  \\
  & \jdeq
(\jvcomp{P}{f}{F})_\ast.\qedhere
\end{align*}
\end{proof}

We are now in position to state and prove the interchange law of horizontal
and vertical composition.

\begin{thm}\label{thm:interchange}
Consider the diagram
\begin{equation*}
\begin{tikzcd}
P \arrow[r,"F"] \arrow[d,fib] &
Q \arrow[r,"G"] \arrow[d,fib] &
R \arrow[d,fib] \\
A \arrow[r,"f"] &
B \arrow[r,"g"] &
C
\end{tikzcd}
\end{equation*}
in context $\Gamma$. Then the equality
\begin{equation*}
{\jcomp{{A}{P}}{\jvcomp{P}{f}{F}}{\jvcomp{Q}{g}{G}}}
  \jdeq
{\jvcomp{P}{\jcomp{A}{f}{g}}{\jfcomp{A}{f}{P}{F}{G}}}
\end{equation*}
of morphisms from $\ctxext{A}{P}$ to $\ctxext{C}{R}$ in context $\Gamma$ holds.
\end{thm}

\begin{proof}
By \autoref{lem:three-composition}, we have
\begin{align*}
\jcomp{{A}{P}}{\jvcomp{P}{f}{F}}{\jvcomp{Q}{g}{G}}
  & \jdeq
\jcomp{P}{F}{\jcomp{A}{f}{\tmext{\ctxwk{Q}{g}}{G}}}
  \tag{By \autoref{lem:three-composition}}
  \\
  & \jdeq
\jcomp{P}{F}{\tmext{\ctxwk{\jcomp{A}{f}{Q}}{\jcomp{A}{f}{g}}}{\jcomp{A}{f}{G}}}
  \\
  & \jdeq
\tmext{\jcomp{P}{F}{\ctxwk{\jcomp{A}{f}{Q}}{\jcomp{A}{f}{g}}}}{\jcomp{P}{F}{{A}{f}{G}}}
  \\
  & \jdeq
\tmext{\jcomp{P}{F}{\ctxwk{\jcomp{A}{f}{Q}}{\jcomp{A}{f}{g}}}}{\jfcomp{A}{f}{P}{F}{G}}
  \\
  & \jdeq
\tmext{\ctxwk{P}{\jcomp{A}{f}{g}}}{\jfcomp{A}{f}{P}{F}{G}}
  \tag{By \autoref{lem:compW_W}}
  \\
  & \jdeq
\jvcomp{P}{\jcomp{A}{f}{g}}{\jfcomp{A}{f}{P}{F}{G}}.\qedhere
\end{align*}
\end{proof}

\begin{thm}
Consider the diagram
\begin{equation*}
\begin{tikzcd}
P \arrow[r,"F"] \arrow[d,fib] &
Q \arrow[r,"G"] \arrow[d,fib] &
R \arrow[d,fib] \\
A \arrow[r,"f"] &
B \arrow[r,"g"] &
C
\end{tikzcd}
\end{equation*}
in context $\Gamma$. 
Then $F_\circledast\circ G_\circledast\jdeq (\jfcomp{A}{f}{P}{F}{G})_\circledast$.
In other words the composition $\jfcomp{}{}{}{\blank}{\blank}$ is associative.
\end{thm}

\begin{proof}
\begin{align*}
F_\circledast\circ G_\circledast
  & \jdeq
(\jvcomp{P}{f}{F})_\ast\circ(\jvcomp{Q}{g}{G})_\ast 
  \tag{By \autoref{lem:three-composition}}\\
  & \jdeq
(\jcomp{\ctxext{A}{P}}{\jvcomp{P}{f}{F}}{\jvcomp{Q}{g}{G}})_\ast 
  \tag{By \autoref{lem:compcomp}}\\
  & \jdeq
(\jvcomp{}{\jcomp{A}{f}{g}}{\jfcomp{A}{f}{P}{F}{G}})_\ast 
  \tag{By \autoref{thm:interchange}} \\
  & \jdeq
(\jfcomp{A}{f}{P}{F}{G})_\circledast.
  \tag{By \autoref{lem:three-composition}}
\end{align*}
\end{proof}

\begin{lem}
Let $\jhom{\Gamma}{A}{B}{f}$ and $\jfhom{\Gamma}{A}{B}{f}{P}{Q}{F}$. Then
$\jvcomp{P}{f}{F}$ is the unique morphism from $\ctxext{A}{P}$ to $\ctxext{B}{Q}$
with the property that both the diagram
\begin{equation*}
\begin{tikzcd}
\ctxext{A}{P}
  \ar{r}{\jvcomp{P}{f}{F}}
  \ar{d}[swap]{\cprojfstf{A}{P}}
& \ctxext{B}{Q}
  \ar{d}{\cprojfstf{B}{Q}}
  \\
A \ar{r}[swap]{f}
& B
\end{tikzcd}
\end{equation*}
commutes, and $\jcomp{}{\jvcomp{P}{f}{F}}{\cprojsndf{B}{Q}}\jdeq F$.
\end{lem}

\begin{proof}
We first note that
\begin{align*}
\jcomp{}{\jvcomp{P}{f}{F}}{\cprojfstf{B}{Q}}
  & \jdeq
\jcomp{P}{F}{\jcomp{A}{f}{\cprojfstf{B}{Q}}} 
  \tag{By \autoref{lem:three-composition}} \\
  & \jdeq
\jcomp{P}{F}{\ctxwk{\jcomp{A}{f}{Q}}{\jcomp{A}{f}{\idtm{B}}}} \\
  & \jdeq
\jcomp{P}{F}{\ctxwk{\jcomp{A}{f}{Q}}{f}} \\
  & \jdeq
\ctxwk{P} f \tag{By \autoref{lem:compW_W}} \\
  & \jdeq
\jcomp{\ctxext{A}{P}}{\cprojfstf{A}{P}}{f}.
  \tag{By \autoref{precomp_by_proj}}
\end{align*}
Also, we have
\begin{align*}
\jcomp{}{\jvcomp{P}{f}{F}}{\cprojsndf{B}{Q}}
  & \jdeq
\jcomp{P}{F}{\jcomp{A}{f}{\idtm{Q}}}
  \tag{By \autoref{lem:three-composition}}
  \\
  & \jdeq
\jcomp{P}{F}{\idtm{\jcomp{A}{f}{Q}}}
  \\
  & \jdeq
F.
\end{align*}
Thus, we conclude that $\jvcomp{P}{f}{F}$ has indeed the stated property. For
the uniqueness, let $G:\ctxext{A}{P}\to\ctxext{B}{Q}$ be a morphism such that
$\jcomp{}{G}{\cprojfstf{B}{Q}}\jdeq\jcomp{}{\cprojfstf{A}{P}}{f}$ and
$\jcomp{}{G}{\cprojsndf{B}{Q}}\jdeq F$. Then it follows that
\begin{equation*}
G \jdeq \tmext{\jcomp{}{\cprojfstf{A}{P}}{f}}{F}
  \jdeq \tmext{\ctxwk{P}{f}}{F}
  \jdeq \jvcomp{P}{f}{F}.\qedhere
\end{equation*}
\end{proof}

\subsection{E-systems are E'-systems}

\begin{lem}
Consider $A\in\cat{F}/\Gamma$ and $Q\in\cat{F}/\ctxext{\Gamma}{B}$. Then there
is an isomorphism
\begin{equation*}
\{h\in\thom{A}{\ctxext{B}{Q}}\mid \jcomp{A}{h}{\cprojfstf{B}{Q}}\jdeq f\}\cong T(\jcomp{A}{f}{Q}).
\end{equation*}
for any $f\in\thom{A}{B}$.
\end{lem}

\begin{proof}
We have the isomorphism
\begin{align*}
\thom{A}{\ctxext{B}{Q}} 
& \jdeq T(\ctxwk{A}{\ctxext{B}{Q}}) \\
& \jdeq T(\ctxext{\ctxwk{A}{B}}{\ctxwk{A}{Q}}) \\
& \cong \bigsqcup\nolimits_{(f\in T(\ctxwk{A}{B}))}T(\subst{f}{\ctxwk{A}{Q}})
\tag{By \autoref{thm:famext_up}} \\
& \jdeq \bigsqcup\nolimits_{(f\in\thom{A}{B})}T(\jcomp{A}{f}{Q}).
\end{align*}
Also, we find $\cprojfst{\ctxwk{A}{B}}{\ctxwk{A}{Q}}{h}\jdeq\subst{h}{\ctxwk{A}{\cprojfstf{B}{Q}}}\jdeq
\jcomp{A}{h}{\cprojfstf{B}{Q}}$.
\end{proof}

\begin{thm}
E-systems are the same things as E'-systems.
\end{thm}

\begin{proof}
For any $f:A\to B$ in context $\Gamma$ and any $Q\in\cat{F}/B$, we define
\begin{equation*}
\pi_2(f,Q)\defeq \jvcomp{\jcomp{A}{f}{Q}}{f}{\tfid{\jcomp{A}{f}{Q}}}
  : \ctxext{A}{\jcomp{A}{f}{Q}}\to\ctxext{B}{Q}.
\end{equation*}
Then we have the commuting diagram
\begin{equation*}
\begin{tikzcd}[column sep=huge]
\ctxext{A}{\jcomp{A}{f}{Q}} \arrow[r,"{\pi_2(f,Q)}"] \arrow[d,swap,"{\cprojfstf{A}{\jcomp{A}{f}{Q}}}"] & \ctxext{B}{Q} \arrow[d,"{\cprojfstf{B}{Q}}"] \\
A \arrow[r,"f"] & B
\end{tikzcd}
\end{equation*}
in context $\Gamma$. To show that this is a pullback diagram, consider a
morphism $g:X\to A$ in context $\Gamma$. Then we have the isomorphisms
\begin{equation*}
\{h\in\thom{X}{\ctxext{B}{Q}}\mid \jcomp{X}{h}{\cprojfstf{B}{Q}}\jdeq\jcomp{X}{g}{f}\}
  \cong
T((\jcomp{X}{g}{f})_\ast(Q))
  \cong
\{u\in\thom{X}{\ctxext{A}{\jcomp{A}{f}{Q}}}\mid\jcomp{X}{u}{\cprojfstf{A}{\jcomp{A}{f}{Q}}}\jdeq g\}
\end{equation*}
Thus, we find for every $h:X\to\ctxext{B}{Q}$ satisfying
$\jcomp{X}{h}{\cprojfstf{B}{Q}}\jdeq\jcomp{X}{g}{f}$, a unique morphism
$u:X\to\ctxext{A}{\jcomp{A}{f}{Q}}$ satisfying
$\jcomp{X}{u}{\cprojfstf{A}{\jcomp{A}{f}{Q}}}\jdeq g$. It is easy to verify
that $\jcomp{X}{u}{\pi_2(f,Q)}\jdeq h$, so the universal property of pullbacks
holds. The functoriality conditions follow immediately from the interchange
laws proven in \autoref{sec:interchange}.

We have described functors from E'-systems to E-systems and vice versa. It is
a straightforward check that these are mutual inverses.
\end{proof}

\begin{comment}
\section{Dependent morphisms}

\begin{defn}
Let $f_0\in\thom{A}{B}$ be an internal morphism and let $P\in\cat{F}/\ctxext{\Gamma}{A}$ and
$Q\in\cat{F}/\ctxext{\Gamma}{B}$. We define
\begin{equation*}
\thomd{f_0}{P}{Q} \defeq \thom{P}{{f_0}_\ast(Q)}
\end{equation*}
We indicate that $f_1\in\thomd{f_0}{P}{Q}$ by drawing the diagram
\begin{equation*}
\begin{tikzcd}
P \arrow[r,"f_1"] \arrow[d,->>] & Q \arrow[d,->>] \\
A \arrow[r,swap,"f_0"] & B.
\end{tikzcd}
\end{equation*}
\end{defn}

\begin{defn}
Consider the diagram
\begin{equation*}
\begin{tikzcd}
P \arrow[r,"f_1"] \arrow[d,->>] & Q \arrow[d,->>] \arrow[r,"g_1"] & R \arrow[d,->>] \\
A \arrow[r,swap,"f_0"] & B \arrow[r,"g_0"] & C.
\end{tikzcd}
\end{equation*}
We define
\begin{align*}
\jvcomp{P}{f_0}{f_1} & \defeq\unfold\jvcomp{P}{f_0}{f_1} & & \in \thom{\ctxext{A}{P}}{\ctxext{B}{Q}} \\
\jfcomp{A}{f_0}{P}{f_1}{g_1} & \defeq \unfold\jfcomp{A}{f_0}{P}{f_1}{g_1} & & \in \thomd{\jcomp{A}{f_0}{g_0}}{P}{R}.
\end{align*}
\end{defn}

\begin{rmk}
Consider
\begin{equation*}
\begin{tikzcd}
P \arrow[r,"f_1"] \arrow[d,->>] & Q \arrow[d,->>] \\
A \arrow[r,swap,"f_0"] & B.
\end{tikzcd}
\end{equation*}
Then the diagram
\begin{equation*}
\begin{tikzcd}[column sep=large]
\ctxext{A}{P} \arrow[r,"\jvcomp{P}{f_0}{f_1}"] \arrow[d,swap,"\cprojfstf{A}{P}"] & \ctxext{B}{Q} \arrow[d,"\cprojfstf{B}{Q}"] \\
A \arrow[r,swap,"f_0"] & B
\end{tikzcd}
\end{equation*}
commutes. Also, we have
\begin{equation*}
S_{\tmext{x}{u}}(\jvcomp{P}{f_0}{f_1}) \jdeq \tmext{S_x(f_0)}{S_{\tmext{x}{u}}(f_1)}
\end{equation*}
for each $x\in T(A)$ and $u\in T(S_x(P))$.
\end{rmk}

\section{Natural transformations of E-homomorphisms}
\begin{rmk}
When $\mathcal{E}$ is an E-system with empty context, then we have the category
structure on the contexts given by $\thom{\Gamma}{\Delta}= T(W_\Gamma(\Delta))$.
\end{rmk}

\begin{defn}
A \emph{natural transformation} of E-homomorphisms from $F:\mathcal{E}\to\mathcal{D}$
to $G:\mathcal{E}\to\mathcal{D}$ preserving the empty context consists of
\begin{enumerate}
\item A t-morphism $\tau0_A\in\thom{F(\Gamma)}{G(\Gamma)}$ for every $\Gamma\in\mathcal{E}$.
\item A commuting square square
\begin{equation*}
\begin{tikzcd}
F(A) \arrow[r,"\tau1_A"] \arrow[d,twoheadrightarrow] & G(A) \arrow[d,twoheadrightarrow] \\
F(\Gamma) \arrow[r,swap,"\tau0_\Gamma"] & G(\Gamma)
\end{tikzcd}
\end{equation*}
for every family $P$ over $A$ of solid arrows, such that the diagram
\begin{equation*}
\begin{tikzcd}
F(A) \arrow[r,"\tau1_A"] & G(A) \\
F(\Gamma) \arrow[r,swap,"\tau0_\Gamma"] \arrow[densely dotted,u,"F(x)"] & G(\Gamma) \arrow[densely dotted,u,swap,"G(x)"]
\end{tikzcd}
\end{equation*}
commutes for every $x\in T(A)$.
\end{enumerate}
such that
\begin{enumerate}
\item the square
\begin{equation*}
\begin{tikzcd}[column sep=huge]
W_{F(A)}(F(B)) \arrow[r,"W_{F(A)}(\tau1_B)"] \arrow[d,->>] & W_{G(A)}(G(B)) \arrow[d,->>] \\
F(A) \arrow[r,"\tau1_A"] & G(A)
\end{tikzcd}
\end{equation*}
commutes for every $A,B\in\mathcal{E}/\Gamma$. Note that 
\begin{align*}
W_{F(A)}(\tau1_B)
  & \in 
T(W_{F(A)}(W_{F(B)}(G(B))))
  \\
  & =
T(W_{W_{F(A)}(F(B))}(W_{F(A)}(G(B))))
\end{align*}
\end{enumerate}
\end{defn}

\section{Indexed E-systems}
\begin{defn}
Let $\cat{A}$ be a category. An $\cat{A}$-indexed E-system is an internal
E-system in the category $Psh(\cat{A})$.
\end{defn}

\section{Examples of adjunctions}

\begin{lem}
For each $A\in\mathcal{E}/\Gamma$, $W_A$ is left adjoint to $\Pi_A$. 
\end{lem}

\begin{proof}
The unit $\eta:1_{\cat{F}/\ctxext{\Gamma}{A}}\Rightarrow W_A\circ\Pi_A$ is given by
\begin{equation*}
\eta_P:\lam{f}{x} ev(\lambda(f),x)
\end{equation*}
$\eta_P\in\thom{P}{W_A(\Pi_A(P))}$
\end{proof}

\begin{lem}
Suppose $\cat{D}$ is a retract of a pre-category $\cat{F}$, i.e. there are functors
$S:\cat{D}\to\cat{F}$ and $R:\cat{F}\to\cat{D}$ with a natural isomorphism
$\tau:R\circ S \simeq \catid{\cat{D}}$, and suppose that $\cat{F}$ is a weakening
category. Then $\cat{D}$ is a weakening system too.
\end{lem}

\begin{proof}
Let $f:\Delta\to \Gamma$ be a morphism in $\cat{D}$. Then we define $W_f$ to be the
composition
\begin{equation*}
\begin{tikzcd}[column sep=3.2em]
\cat{D}/\Gamma
  \arrow[r,"S/\Gamma"]
  &
\cat{F}/S(\Gamma)
  \arrow[r,"W_{S(f)}"]
  &
\cat{F}/S(\Delta)
  \arrow[r,"R/S(\Delta)"]
  &
\cat{D}/R(S(\Delta))
  \arrow[r,"\tau_\Delta\circ{-}"]
  &
\cat{D}/\Delta
\end{tikzcd}
\end{equation*}
We first verify whether this assembles a pre-weakening structure on $\cat{D}$.
\begin{enumerate}
\item Let $\Delta$ be an object of $\cat{D}$. Then $W_{\catid{\Delta}}$ is the composition
\begin{equation*}
\begin{tikzcd}[column sep=3.2em]
\cat{D}/\Delta
  \arrow[r,"S/\Delta"]
  &
\cat{F}/S(\Delta)
  \arrow[r,"R/S(\Delta)"]
  &
\cat{D}/R(S(\Delta))
  \arrow[r,"{-}\circ\tau_\Delta"]
  &
\cat{D}/\Delta
\end{tikzcd}
\end{equation*}
This functor maps $g:\Delta'\to \Delta$ consecutively to $S(g)$, to $R(S(g))$, to
$\tau_\Delta\circ R(S(g))$. Note that $\tau_\Delta\circ R(S(g))=g\circ \tau_{\Delta'}$ by
naturality.
\end{enumerate}
\end{proof}

\section{Weak E-systems}

\begin{rmk}
Note that $W$ is a strict $2$-functor frm $\cat{F}$ to $\mathbb{Cat}$ of which
the value at an object $\Gamma$ is specified to be $\cat{F}/\Gamma$, and which
is required to preserve the final object. We could have considered $W$ to be a 
pseudo-functor from $\cat{F}$ to $\mathbb{Cat}$ instead. A coherence theorem
by Power \cite{Power89} then allows for the strictification of this pseudo-functor
into a (strict) weakening structure. (Does this map $\Gamma$ to $\cat{F}/\Gamma$
and final to final?). Thus, we know what a weak pre-weakening system has to be.
The following questions need to be answered:
\begin{itemize}
\item What is a weak weakening/projection/substitution system.
\item What is a weak E-system? I.e.~what are the coherence rules for something
to be a weakening/projection/substitution homomorphism?
\item Can they all be strictified? Are there more examples of weak E-systems?
\end{itemize}
\end{rmk}
\end{comment}
