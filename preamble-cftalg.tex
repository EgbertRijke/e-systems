%%%%%%%%%%%%%%%%%%%%%%%%%%%%%%%%%%%%%%%%%%%%%%%%%%%%%%%%%%%%%%%%%%%%%%%%%%%%%%%%
%%%% ALGEBRAS FOR THE THEORY OF CONTEXTS, FAMILIES AND TERMS

\makeatletter
\newcommand{\algfont}{\mathbf}

%%%% EXTENSION ALGEBRAS
\newcommand{\extalg}{\algfont}
\newcommand{\extalg@sym}{\@judgment@kind{ExtAlg}}
\newcommand{\jextalg}[2]{\judgment{#1}{}{}{}{}{\extalg{#2}}{\extalg@sym}}

\newcommand{\jextalgctx}[2]{%
  \jfam{#1}{\cftalgc{\cftalg{#2}}}}

\newcommand{\jextalgfam}[2]{%
  \jfam{{#1}{\cftalgc{\cftalg{#2}}}}{\cftalgf{\cftalg{#2}}}}

\newcommand{\jextalgtm}[2]{%
  \jfam{{{#1}{\cftalgc{\cftalg{#2}}}}{\cftalgf{\cftalg{#2}}}}{\cftalgt{\cftalg{#2}}}}

\newcommand{\jextalgctxext}[2]{%
  \jhom{#1}{{\cftalgc{\cftalg{#2}}}{\cftalgf{\cftalg{#2}}}}{\cftalgc{\cftalg{#2}}}{\cftctxext[\cftalg{#2}]}}

\newcommand{\jextalgfamext}[2]{%
  \jhom{{#1}{\cftalgc{\cftalg{#2}}}}{{\cftalgf{\cftalg{#2}}}{\jcomp{}{\cftctxext[\cftalg{#2}]}{\cftalgf{\cftalg{#2}}}}}{\cftalgf{\cftalg{#2}}}{\cftfamext[\cftalg{#2}]}}
  
\newcommand{\extalgc}[1]{{\renewcommand\extalg[1]{##1}#1}}
\newcommand{\extalgf}[1]{\mathcal{F}_{#1}}
\newcommand{\extalgt}[1]{\mathcal{T}_{#1}}

%%%% EXTENSION TERMS
\newcommand{\cftext}[2][]{\epsilon_{#1}^{#2}}
\newcommand{\cftctxext}[1][]{\cftext[0]{#1}}
\newcommand{\cftfamext}[1][]{\cftext[1]{#1}}

%%%% THE FAMILY EXTENSION ALGEBRA
\newcommand{\extfamalg}[1]{\algfont{F}_{#1}}

%%%% EXTENSION HOMOMORPHISMS
\newcommand{\exthom}[1]{\algfont{#1}}
\newcommand{\exthomc}[1]{{\renewcommand\exthom[1]{##1}#1}}
\newcommand{\exthomf}[1]{\mathcal{F}_{#1}}
\newcommand{\exthomt}[1]{\mathcal{T}_{#1}}
\newcommand{\extfamhom}[1]{\algfont{F}_{#1}}

%%%% CFT-ALGEBRAS
\newcommand{\cftalg}{\algfont}

%%%% The intended use of `\cftctx` is `\cftctx{\cftalg{#1}}`.
\newcommand{\cftalgc}[1]{\@gobble #1}

%%%% The intended use of `\cftfam` is `\cftfam{\cftalg{#1}}`.
\newcommand{\cftalgf}[1]{\mathcal{F}_{#1}}
\newcommand{\cftalgt}[1]{\mathcal{T}_{#1}}

\newcommand{\cftemp}[2][]{\phi_{#1}^{#2}}
\newcommand{\cftempc}{\cftemp[0]}
\newcommand{\cftempf}{\cftemp[1]}

\newcommand{\cftfamalg}[1]{\algfont{F}_{#1}}

%%%% CFT-HOMOMORPHISMS
\newcommand{\cfthom}{\algfont}

\newcommand{\cfthomc}[1]{%
  { \renewcommand{\cfthom}[1]{##1}
    #1
    }
  }

\newcommand{\cfthomf}[1]{\mathcal{F}_{#1}}

\newcommand{\cfthomcomp}[2]{#2\circ #1}

\newcommand{\cfthomt}[1]{\mathcal{T}_{#1}}

\newcommand{\cftfamhom}[1]{\algfont{F}_{#1}}

\newcommand{\cftidhom}[1]{\algfont{id}_{#1}}

%%%% THE WEAKENING TERM
\newcommand{\cftwk}[1]{\boldsymbol\omega^{#1}}
\newcommand{\cftwkc}[1]{\omega^{#1}}
\newcommand{\cftwkf}[1]{\mathcal{F}_{\boldsymbol\omega^{#1}}}
\newcommand{\cftwkt}[1]{\mathcal{T}_{\boldsymbol\omega^{#1}}}

%%%% THE SUBSTITUTION TERM
\newcommand{\cftsubst}[1]{\boldsymbol\sigma^{#1}}
\newcommand{\cftsubstc}[1]{\sigma^{#1}}
\newcommand{\cftsubstf}[1]{\mathcal{F}_{\cftsubst{#1}}}
\newcommand{\cftsubstt}[1]{\mathcal{T}_{\cftsubst{#1}}}

%%%% THE IDENTITY TERM
\newcommand{\cftidtm}[1]{\boldsymbol\iota^{#1}}

\makeatother

