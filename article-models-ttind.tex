\section{Inductive constructions}

\subsection{Strong inductive morphisms}
In this subsection we investigate the notion of inductive morphisms. An 
inductive morphism $f$ from $A$ to $B$ in context $\Gamma$ is a morphism which
induces an operation which is judgmentally the inverse of composition
with $f$. We explore a notion of inductive morphism which is much stronger
than the usual notion: it pushes families over $A$ to families over $B$,
families over families over $A$ to families over families over $B$ and terms
thereof to terms of the output families over families in a manner compatible
with the empty family, extension, weakening, substitution and identity terms.
As a result, inductive morphisms will be stable under extension, weakening,
substitution and the identity term is an inductive morphism.

Inductive morphisms are useful to for inductive types which are defined using
only one (ordinary morphism as) constructor, like the unit type and dependent 
pair types.
They can't be used to define the empty type, disjoint sums,
the natural numbers or identity types. 

{\color{red} Maybe
we can solve this by writing down a type theory of inductive constructions.}

\begin{defn}
Let $\jhom{\gamma}{A}{B}{f}$ be a context morphism. We say that $f$ is an
\emph{inductive morphism} if there is an operation $\tfindf{f}$ given by
\begin{align*}
& \inference
  { \jfam{{\Gamma}{A}}{P}
    }
  { \jfam{{\Gamma}{B}}{\tfind{f}{P}}
    }
  \\
& \inference
  { \jfam{{{\Gamma}{A}}{P}}{Q}
    }
  { \jfam{{{\Gamma}{B}}{\tfind{f}{P}}}{\tfind[\famsym]{f}{Q}}
    }
  \\
& \inference
  { \jterm{{{\Gamma}{A}}{P}}{Q}{g}
    }
  { \jterm{{{\Gamma}{B}}{\tfind{f}{P}}}{\tfind[\famsym]{f}{Q}}{\tfind[\tmsym]{f}{g}}
    }
\end{align*}
for which the inference rules in the following list are valid:
\begin{enumerate}
\item The operation $\tfindf{f}$ is compatible with the empty context:
\begin{align*}
& \inference
  { 
    }
  { \jfameq{{\Gamma}{B}}{\tfind{f}{\emptyf}}{\emptyf}
    }
  \\
& \inference
  { \jfam{{\Gamma}{A}}{P}
    }
  { \jfameq{{{\Gamma}{B}}{\tfind{f}{P}}}{\tfind[\famsym]{f}{\emptyf}}{\emptyf}
    }
\end{align*}
\item The action on families $\tfindf[\famsym]{f}$ of $\tfindf{f}$ is compatible
with the action on contexts:
\begin{equation*}
\inference
  { \jfam{{\Gamma}{A}}{P}
    }
  { \jfameq{{\Gamma}{B}}{\tfind[\famsym]{f}{P}}{\tfind{f}{P}}
    }
\end{equation*}
Because of this inference rule we shall henceforth omit the annotations
$\famsym$ and $\tmsym$ from the operation $\tfindf{f}$ as usual.
\item The operation $\tfindf{f}$ is compatible with extension:
\begin{align*}
& \inference
  { \jfam{{{\Gamma}{A}}{P}}{Q}
    }
  { \jfameq
      {{\Gamma}{B}}
      {\tfind{f}{\ctxext{P}{Q}}}
      {\ctxext{\tfind{f}{P}}{\tfind{f}{Q}}}
    }
  \\
& \inference
  { \jfam{{{{\Gamma}{A}}{P}}{Q}}{R}
    }
  { \jfameq
      {{{\Gamma}{B}}{\tfind{f}{P}}}
      {\tfind{f}{\ctxext{Q}{R}}}
      {\ctxext{\tfind{f}{Q}}{\tfind{f}{R}}}
    }
\end{align*}
\item The operation $\tfindf{f}$ is compatible with weakening:
\begin{align*}
& \inference
  { \jfam{{\Gamma}{A}}{P}
    \jfam{{\Gamma}{A}}{Q}
    }
  { \jfameq
      {{{\Gamma}{B}}{\tfind{f}{P}}}
      {\tfind{f}{\ctxwk{P}{Q}}}
      {\ctxwk{\tfind{f}{P}}{\tfind{f}{Q}}}
    }
  \\
& \inference
  { \jfam{{\Gamma}{A}}{P}
    \jfam{{{\Gamma}{A}}{Q}}{R}
    }
  { \jfameq
      {{{{\Gamma}{B}}{\tfind{f}{P}}}{\ctxwk{\tfind{f}{P}}{\tfind{f}{Q}}}}
      {\tfind{f}{\ctxwk{P}{R}}}
      {\ctxwk{\tfind{f}{P}}{\tfind{f}{R}}}
    }
  \\
& \inference
  { \jfam{{\Gamma}{A}}{P}
    \jterm{{{\Gamma}{A}}{Q}}{R}{h}
    }
  { \jtermeq
      {{{{\Gamma}{B}}{\tfind{f}{P}}}{\ctxwk{\tfind{f}{P}}{\tfind{f}{Q}}}}
      {\ctxwk{\tfind{f}{P}}{\tfind{f}{R}}}
      {\tfind{f}{\ctxwk{P}{h}}}
      {\ctxwk{\tfind{f}{P}}{\tfind{f}{h}}}
    }
\end{align*}
\item We will also require that the operation $\tfindf{f}$ is compatible with
weakening by $A$:
\begin{align*}
& \inference
  { \jfam{\Gamma}{X}
    }
  { \jfameq
      {{\Gamma}{B}}
      {\tfind{f}{\ctxwk{A}{X}}}
      {\ctxwk{B}{X}}
    }
  \\
& \inference
  { \jfam{{\Gamma}{X}}{Y}
    }
  { \jfameq
      {{{\Gamma}{B}}{\ctxwk{B}{X}}}
      {\tfind{f}{\ctxwk{A}{Y}}}
      {\ctxwk{B}{Y}}
    }
  \\
& \inference
  { \jterm{{\Gamma}{X}}{Y}{y}
    }
  { \jtermeq
      {{{\Gamma}{B}}{\ctxwk{B}{X}}}
      {\ctxwk{B}{Y}}
      {\tfind{f}{\ctxwk{A}{y}}}
      {\ctxwk{B}{y}}
    }
\end{align*}
These rules assert that constant families and terms are mapped to constant
families and terms.
\item The operation $\tfindf{f}$ is compatible with substitution:
\begin{align*}
& \inference
  { \jterm{{{\Gamma}{A}}{P}}{Q}{g}
    \jfam{{{{\Gamma}{A}}{P}}{Q}}{R}
    }
  { \jfameq
      {{{\Gamma}{B}}{\tfind{f}{P}}}
      {\tfind{f}{\subst{g}{R}}}
      {\subst{\tfind{f}{g}}{\tfind{f}{R}}}
    }
  \\
& \inference
  { \jterm{{{\Gamma}{A}}{P}}{Q}{g}
    \jfam{{{{{\Gamma}{A}}{P}}{Q}}{R}}{S}
    }
  { \jfameq
      {{{{\Gamma}{B}}{\tfind{f}{P}}}{\subst{\tfind{f}{g}}{\tfind{f}{R}}}}
      {\tfind{f}{\subst{g}{S}}}
      {\subst{\tfind{f}{g}}{\tfind{f}{S}}}
    }
  \\
& \inference
  { \jterm{{{\Gamma}{A}}{P}}{Q}{g}
    \jterm{{{{{\Gamma}{A}}{P}}{Q}}{R}}{S}{k}
    }
  { \jtermeq
      {{{{\Gamma}{B}}{\tfind{f}{P}}}{\subst{\tfind{f}{g}}{\tfind{f}{R}}}}
      {\subst{\tfind{f}{g}}{\tfind{f}{S}}}
      {\tfind{f}{\subst{g}{k}}}
      {\subst{\tfind{f}{g}}{\tfind{f}{k}}}
    }
\end{align*}
\item The operation $\tfindf{f}$ is compatible with the identity terms:
\begin{equation*}
\inference
  { \jfam{{{\Gamma}{A}}{P}}{Q}
    }
  { \jtermeq
      {{{{\Gamma}{B}}{\tfind{f}{P}}}{\tfind{f}{Q}}}
      {\ctxwk{\tfind{f}{Q}}{\tfind{f}{Q}}}
      {\tfind{f}{\idtm{Q}}}
      {\idtm{\tfind{f}{Q}}}
    }
\end{equation*}
\item We will also require that $\tfindf{f}$ is compatible with $f$ itself:
\begin{equation*}
\inference
  {
    }
  { \jtermeq
      {{\Gamma}{B}}
      {\ctxwk{B}{B}}
      {\tfind{f}{f}}
      {\idtm{B}}
    }
\end{equation*}
\item Finally, we require that $\tfindf{f}$ is the right inverse of composition
with $f$:
\begin{align*}
& \inference
  { \jfam{{\Gamma}{A}}{P}
    }
  { \jfameq
      {{\Gamma}{A}}
      {\jcomp{}{f}{\tfind{f}{P}}}
      {P}
    }
  \\
& \inference
  { \jfam{{{\Gamma}{A}}{P}}{Q}
    }
  { \jfameq
      {{{\Gamma}{A}}{P}}
      {\jcomp{}{f}{\tfind{f}{Q}}}
      {Q}
    }
  \\
& \inference
  { \jterm{{{\Gamma}{A}}{P}}{Q}{g}
    }
  { \jtermeq
      {{{\Gamma}{A}}{P}}
      {Q}
      {\jcomp{}{f}{\tfind{f}{g}}}
      {g}
    }
\end{align*}
\end{enumerate}
\end{defn}

\begin{rmk}
The rules expressing that $\tfindf{f}$ is a right inverse to composition with
$f$ are usually called the `computation rules' of the induction principle.

Recall that we had announced that $\tfindf{f}$ would be an actual inverse of
composition with $f$, while we only have stated explicitly that $\tfindf{f}$
is a right inverse. We get the fact that it is also a left inverse from the
other compatibility rules. For example: given $\jfam{{\Gamma}{B}}{Q}$ we get
\begin{equation*}
\tfind{f}{\jcomp{}{f}{Q}}
  \jdeq
  \tfind{f}{\subst{f}{\ctxwk{A}{Q}}}
  \jdeq
  \subst{\tfind{f}{f}}{\tfind{f}{\ctxwk{A}{Q}}}
  \jdeq
  \subst{\idtm{B}}{\ctxwk{B}{Q}}
  \jdeq
  Q.
\end{equation*}
We thus recover the usual sort of induction principle. When $Q$ is a family
over $\ctxext{\Gamma}{B}$, all we have to do to find a term of $Q$ is to find
a term $g$ of $\jcomp{}{f}{Q}$. The result of applying $\tfindf{f}$ to $g$
will be a term of $Q$.
\end{rmk}

\begin{rmk}
These stronger rules also seem to imply that not every equivalence is going
to be an inductive morphism (when we add all the type constructors to the
theory). For instance, the interval is equivalent to the unit type. If the
unit type is defined via an inductive morphism $\emptyc\to\unit$ we get that
every family over $\unit$ is definitionally a constant type because
every family over $\emptyc$ is a weakening by the empty family. If the
equivalence from $\unit$ to the interval were inductive, this would in turn
imply that every type family over the interval is constant. However, this is
not the case because we have the family which has the unit type as a fiber
above one endpoint and the interval above the other.
\end{rmk}

\begin{lem}
The identity term
$\jhom{\Gamma}{A}{A}{\idtm{A}}$ is an inductive morphism
for each family $A$ of contexts over $\Gamma$
\end{lem}

\begin{proof}
Composition with the identity morphism is an identity operation.
\end{proof}

\begin{itemize}
\item Extensions of inductive morphisms are inductive
\item Weakenings of inductive morphisms are inductive
\item Substitutions of inductive morphisms are inductive
\end{itemize}
