\section{B-systems}

\subsection{B-frames and stratified categories with term structure}

\begin{defn}
A \define{B-frame} is a collection of data of the following form:
\begin{enumerate}
\item for all $n\in\mathbb{N}$ two sets $B_n$ and $\tilde{B}_n$. 
\item for all $n\in\mathbb{N}$ maps of the form
\begin{align*}
\eft[n] & : B_{n+1}\to B_n \\
\ebd[n] & : \tilde{B}_n\to B_n.
\end{align*}
For $m,n\in\mathbb{N}$, we denote the composition $\eft[n]\circ\cdots\circ\eft[n+m]:B_{n+m+1}\to B_n$ by $\eft[n]^m$. 
\item $B_0$ is a singleton $\{\pt\}$.
\end{enumerate} 
A \define{homomorphism $H:\mathbb{B}\to\mathbb{A}$ of B-frames} consists of maps
$H_n:B_n\to A_n$ and $\tilde{H}_n:\tilde{B}_n\to\tilde{A}_n$ such that
\begin{align*}
\eft(H(X)) & \jdeq H(\eft(X)) \\
\ebd(\tilde{H}(x)) & \jdeq H(\ebd(x))
\end{align*}
for any $X\in B_n$ and $x\in\tilde{B}_n$. The category of B-frames is
denoted by $\mathbf{Bfr}$. 
\end{defn}

There is a forgetful functor from B-frames to rooted trees, which is established in the following definition.

\begin{defn}
We define a functor $\mathcal{E}:\mathbf{Bfr}\to\mathbf{Cat}$.
\end{defn}

\begin{constr}
Let $\mathbb{B}$ be a B-frame.
The set of objects of $\mathcal{E}(\mathbb{B})$ is taken to be the set $\bigsqcup_{(n\in\mathbb{N})}B_n$. 
The partial order on $\bigsqcup_{(n\in\mathbb{N})}B_n$ as an inductive family of propositions, by the clauses
\begin{equation*}
(n,X)\leq (n,X) \qquad\text{and}\qquad
((m,\eft(Y))\leq (n,X))\to((m+1,Y)\leq (n,X)).
\end{equation*}
This defines a partial order, hence we obtain a category.

For any homomorphism $H:\mathbb{B}\to\mathbb{A}$, an order-preserving map $\mathcal{E}(H):\mathcal{E}(\mathbb{B})\to\mathcal{E}(\mathbb{A})$ is defined in the obvious way. 
\end{constr}

We introduce \emph{stratified} pre-categories to characterize the image of this forgetful functor.

\begin{defn}
A pre-category $\cat{C}$ with terminal object is said to be \define{stratified} if there exists a 
\emph{stratification functor}
\begin{equation*}
L : \cat{C}\to (\mathbb{N},\geq)
\end{equation*}
such that
\begin{enumerate}
\item $L(X)\jdeq 0$ if and only if $X$ is terminal, and for any $f:X\to Y$ we have
$L(X)\jdeq L(Y)$ if and only if $X\jdeq Y$ and $f\jdeq\catid{X}$. 
\item every morphism $f:X\to Y$ in $\cat{C}$, where $L(X)\jdeq
n+m+1$ and $L(Y)\jdeq n$, has a unique factorization 
\begin{equation*}
\begin{tikzcd}
X \arrow[r,"f_m"] & X_m \arrow[r,"f_{m-1}"] & \cdots \arrow[r,"f_1"] & X_1 \arrow[r,"f_0"] & Y
\end{tikzcd}
\end{equation*}
where $L(X_i)\jdeq n+i$.
\end{enumerate}
A functor $F:\cat{C}\to\cat{D}$ between stratified pre-categories is said to be stratified 
if $L_{\cat{C}}\jdeq L_{\cat{D}}\circ F$.

We define $\mathbf{Cat_s}$ to be the category of stratified pre-categories, and stratified functors between them.
\end{defn}

%\begin{rmk}
%The categorical version of being stratified, is that there exists a factofibration
%$L:\cat{C}\to (N,\geq)$ which preserves the terminal object.
%\end{rmk}

\begin{lem}
Any stratified pre-category is a rooted tree.
\end{lem}

\begin{proof}
Let $\cat{C}$ be a stratified pre-category. The root of the tree is going to be the terminal
object of $\cat{C}$. Note that objects of $\cat{C}$ have no endomorphisms other
than the identity morphisms. Therefore, to show that $\cat{C}$ is a tree,
it suffices to show that for every object $X$ with $L(X)>0$ there is
a unique object $\eft(X)$ with $L(\eft(X))\jdeq L(X)-1$, with a morphism $p_X:X\to \eft(X)$.
The existence follows from the existence of the factorization of $X\to 1$. The 
uniqueness follows from the observation that for any arrow
$f:X\to Y$ with $L(Y)\jdeq L(X)-1$, we can factorize $X\to 1$ through $Y$. Since
factorizations are unique, we will have $Y\jdeq \eft(X)$, and $p_X\jdeq f$. 
\end{proof}

\begin{lem}
The category of small stratified categories is a full subcategory of the category
of small pre-categories with terminal objects and between them functors which lift 
factorizations in the sense of \autoref{defn:lift_factorizations}, and preserve the terminal object.
\end{lem}

\begin{proof}
Let $F:\cat{C}\to\cat{D}$ be a functor between stratified pre-categories which lifts
factorizations and 
which preserves terminal objects. We show by induction that $F$ is stratified, i.e.~that $L_{\mathcal{C}}(X)\jdeq L_{\mathcal{D}}(F(X))$, for any object $X$ of $\mathcal{C}$.

For the base case, we observe that since $F$ preserves the terminal object, $L(X)\jdeq 0$ implies $L(F(X))\jdeq 0$.

For the inductive step, suppose that for $n\in\mathbb{N}$, we have that $L(X)\jdeq n$ implies
$L(F(X))\jdeq n$, for any object $X$ of $\mathcal{C}$, and let $Y$ be such that $L(Y)\jdeq n+1$ with
$p_Y:Y\to\eft(Y)$. Then the poset $\mathbf{fact}(p_Y)$ is isomorphic to the
poset $\mathbf{2}$ with two objects, one of which is smaller than the other. By the
assumption that $F$ lifts factorizations, the poset $\mathbf{fact}(F(p_Y))$
is also isomorphic to $\mathbf{2}$. Thus, the only factorization
of $F(Y)\to\unit$ starts with $F(p_Y):H(Y)\to F(\eft(Y))$, which implies that
$L(F(Y))\jdeq n+1$. 
\end{proof}

\begin{cor}
A pre-category can only be stratified in at most one way. That is, being stratified is a property of pre-categories.
\end{cor}

\begin{thm}
The functor $\mathcal{E}:\mathbf{Bfr}\to\mathbf{Cat}$ factors through the forgetful functor $\mathbf{Cat_s}\to\mathbf{Cat}$.
\end{thm}

\begin{constr}
Let $\mathbb{B}$ be a B-frame. Note that for any $X\in B_{n+1}$ there is a unique $\Gamma\in
B_n$ such that $(n+1,X)\leq (n,\Gamma)$, so $\mathcal{E}(\mathbb{B})$ is stratified.

To show that the functor $\mathcal{E}(H):\mathcal{E}(\mathbb{B})\to \mathcal{E}(\mathbb{A})$ is stratified for
any homomorphism $H:\mathbb{B}\to\mathbb{A}$ of B-frames, note that
by \autoref{lem:strat_full} it suffices to show that each $\mathcal{E}(H)$ 
lifts factorizations.
Let $H:\mathbb{B}\to\mathbb{A}$ be a homomorphism of B-frames, and let
$(m,X):(n+m,X)\leq (n,\Gamma)$. Then the pre-category $\mathbf{fact}_{(m,X)}$ of
factorizations of $(m,X)$ is a finite linear order with
$m+1$ elements, and so is the category $\mathbf{fact}_{(m,H(X))}$. 
\end{constr}

\begin{rmk} 
It is in fact the case that $\mathcal{E}(\mathbb{B})$ is the free pre-category generated by the
graph with $\bigsqcup_{(n\in\mathbb{N})}B_n$ as the set of vertices, and
for each object of the form $(n+1,X)$ an edge to $(n,\eft[n](X))$. 
Essentially all $\mathcal{E}$ does, is forgetting about the term structure 
$\tilde{B}$ of $\mathbb{B}$.
%
%Furthermore, it is useful to note that for any $X\in B_n$, we get an 
%isomorphism $\mathcal{E}(\mathbb{B}/X)\cong  \mathcal{E}(\mathbb{B})/(n,X)$ of rooted trees,
%natural in $X$. Therefore we will usually not distinguish between $\mathcal{E}(\mathbb{B})/X$
%and $\mathcal{E}(\mathbb{B})/(n,X)$. 
\end{rmk}

\subsection{Non-unital pre-B-systems}

For every B-frame $\mathbb{B}$ and any $X\in B_n$, there is a B-frame
$\mathbb{B}/X$ given by
\begin{align*}
(B/X)_{m} & \jdeq \{Y\in B_{n+m}\mid\eft^{m}(Y)\jdeq X\}\\
(\tilde{B}/X)_m & \jdeq \{y\in \tilde{B}_{n+m}\mid\eft^m(\ebd(y))\jdeq X\}.
\end{align*}
Also, for any homomorphism $H:\mathbb{B}\to\mathbb{A}$ of B-frames and any
$X\in B_n$, there is a homomorphism $H/X:\mathbb{B}/X\to\mathbb{A}/H(X)$
defined in the obvious way.

Note that for $X\in B_n$ and $Y\in B_{n+m}$ such that $\eft^m(Y)\jdeq X$, 
we have an isomorphism $(\mathbb{B}/X)/Y\cong B/Y$ of B-frames, constructed in the obvious way, which is natural in the sense that for any homomorphism $H:\mathbb{B}\to\mathbb{A}$ of B-frames, the square
\begin{equation*}
\begin{tikzcd}
(\mathbb{B}/X)/Y \arrow[r,"\cong"] \arrow[d,swap,"(H/X)/Y"] &  \mathbb{B}/Y \arrow[d,"H/Y"] \\
(\mathbb{A}/H(X))/H(Y) \arrow[r,"\cong"] & \mathbb{A}/H(Y)
\end{tikzcd}
\end{equation*}
commutes.

\begin{defn}
A \define{non-unital pre-B-system} $\mathbb{B}$ consists of a B-frame $\mathbb{B}$ and homomorphisms
\begin{align*}
W_{X} & : \mathbb{B}/\eft(X)\to\mathbb{B}/X\\
S_{x} & : \mathbb{B}/\ebd(x)\to\mathbb{B}/\eft(\ebd(x))
\end{align*}
of B-frames, for any $X\in B_{n+1}$ and any $x\in\tilde{B}_{n+1}$. 

A \define{non-unital pre-B-homomorphism} $H:\mathbb{B}\to\mathbb{A}$ is a homomorphism of B-frames
for which the diagrams
\begin{equation*}
\begin{tikzcd}[column sep=huge]
\mathbb{B}/X \arrow[r,"H/X"] & \mathbb{A}/H(X) \\
\mathbb{B}/\eft(X) \arrow[u,"W_X"] \arrow[r,swap,"H/\eft(X)"] & \mathbb{A}/\eft(H(X)) \arrow[u,swap,"W_{H(X)}"]
\end{tikzcd}
\end{equation*}
and
\begin{equation*}
\begin{tikzcd}[column sep=huge]
\mathbb{B}/\ebd(x) \arrow[r,"H/\ebd(x)"] \arrow[d,swap,"S_x"] & \mathbb{A}/\ebd(H(X)) \arrow[d,"S_{\tilde{H}(x)}"] \\
\mathbb{B}/\eft(\ebd(x)) \arrow[r,swap,"H/\eft(\ebd(x))"] & \mathbb{A}/\eft(\ebd(H(X)))
\end{tikzcd}
\end{equation*}
of homomorphisms of B-frames commute for every 
$X\in B_n$ and $x\in\tilde{B}_n$. The category of non-unital pre-B-systems is denoted
by $\mathbf{B0sys}$.
\end{defn}

\begin{comment}
\begin{lem}
When $\mathbb{B}$ has the structure of a pre-B-system, then so does each $\mathbb{B}/X$.
\end{lem}

\begin{proof}
Straightforward.
\end{proof}
\end{comment}

\begin{defn}
Let $\mathbb{B}$ be a pre-B-system. Then $U(\mathbb{B})$ can be given the structure
of a pre-E-system. We write $\mathcal{E}(\mathbb{B})$ for the resulting
pre-E-system. Thus we get a functor $\mathcal{E}:\mathbf{B0sys}\to\mathbf{E0sys}_s$.
\end{defn}

\begin{constr}
We begin by \emph{simultaneously} defining the term structure and the pre-substitution
structure on $U(\mathbb{B})$ by induction on the natural numbers. More precisely,
for any $m\in\mathbb{N}$ we will define, for any $(n+m,X)\leq (n,\Gamma)$ a set
$T(m,X)$ and for any $x\in T(m,X)$ a homomorphism 
$S_x:\mathbb{B}/X\to\mathbb{B}/\Gamma$ of B-frames. Then it follows
immediately that we get stratified functors $U(S_x):U(\mathbb{B})/X\to U(\mathbb{B})/\Gamma$
defining the pre-substitution structure of $\mathcal{E}(\mathbb{B})$.

In the base case, we define for $(n,X)\in U(\mathbb{B})$, the set
$T(0,X)\defeq \{\pt[X]\}$ and $S_{\pt[X]}\defeq \catid{\mathbb{B}/X}$. 

%We also define for
%$(n+1,X)\leq (n,\Gamma)$ the set $T(1,X)\jdeq\{x\in B_{n+1}\mid \ebd(x)\jdeq X\}$.
%Note that for $x\in T(1,X)$, the homomorphism $S_x:\mathbb{B}/X \to\mathbb{B}/\Gamma$
%is already assumed to exist, because $\mathbb{B}$ is a pre-B-system.

For the inductive step, suppose that for $m\in\mathbb{N}$ we have sets $T(m,X)$
for any $(n+m,X)\leq (n,\Gamma)$ and homomorphisms $S_x:\mathbb{B}/X \to \mathbb{B}/\Gamma$
of B-frames, for any $x\in T(m,X)$. Then we define
\begin{equation*}
T(m+1,X)\defeq\bigsqcup_{x\in T(m,\eft(X))} \{u\in \tilde{B}_{n+1}\mid \ebd(u)\jdeq S_x(X)\}.
\end{equation*}
For $(x,u)\in T(m+1,X)$ we define the homomorphism $S_{(x,u)}:\mathbb{B}/X\to
\mathbb{B}/\Gamma$ as
\begin{equation*}
S_{(x,u)}\defeq S_u\circ S_x/X,
\end{equation*}
where $S_u$ is taken from the structure of $\mathbb{B}$ of being a pre-B-system.
It remains to define a pre-weakening structure on $U(\mathbb{B})$. To do this,
it suffices to define homomorphisms $W_{(m,X)}:\mathbb{B}/\Gamma \to\mathbb{B}/X$
of B-frames, for any $(n+m,X)\leq (n,\Gamma)$. We define
\begin{equation*}
W_{(m,X)}\defeq W_X\circ\cdots\circ W_{\eft^{m-1}(X)}.\qedhere
\end{equation*}
\end{constr}

\begin{thm}
The functor $\mathcal{E}$ is full and faithful.
\end{thm}

\begin{proof}
We have to show that the map $\mathrm{hom}(\mathbb{B},\mathbb{A})\to
\mathrm{hom}(\mathcal{E}(\mathbb{B}),\mathcal{E}(\mathbb{A}))$ is a bijection
for any two pre-B-systems $\mathbb{B}$ and $\mathbb{A}$.

To show that this map is injective, suppose $F,G:\mathbb{B}\to\mathbb{A}$ are 
pre-B-homomorphisms such that $\mathcal{E}(F)\jdeq\mathcal{E}(G)$. Then we have
$F_n(X)\jdeq\mathcal{E}(F)(n,X)\jdeq \mathcal{E}(G)(n,X)\jdeq G_n(X)$ for each
object $x\in B_n$. Also, we have $\tilde{F}_n(x)\jdeq T_{\mathcal{E}(F)}(n,x)
\jdeq T_{\mathcal{E}(G)}(n,x)\jdeq \tilde{G}_n(x)$ for each
$x\in\tilde{B}_n$. This shows that $F\jdeq G$, so we conclude that $\mathcal{E}$
is faithful.  

To show that the map $\mathrm{hom}(\mathbb{B},\mathbb{A})\to
\mathrm{hom}(\mathcal{E}(\mathbb{B}),\mathcal{E}(\mathbb{A}))$ 
is surjective, let $H:\mathcal{E}(\mathbb{B})\to
\mathcal{E}(\mathbb{A})$. Since both $\mathcal{E}(\mathbb{B})$ and
$\mathcal{E}(\mathbb{A})$ are stratified, we get from
\autoref{lem:strat_full} that $H$ is a stratified pre-E-homomorphism. 
Because $H$ is stratified, we get maps $H_n:B_n\to A_n$ for any $n\in\mathbb{N}$.
The term structure of $H$ gives maps $\tilde{H}_n:
\tilde{B}_n\to\tilde{A}_n$. This defines a pre-B-homomorphism since $H$ 
preserves the pre-weakening and pre-substitution structures of $\mathcal{E}(\mathbb{B})$.
It is immediate from the definition of $H_n$ and
$\tilde{H}_n$ that this pre-B-homomorphism is mapped back to $H$, so we conclude
that $\mathcal{E}$ is full.
\end{proof}
