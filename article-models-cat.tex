\section{E-objects in categories with finite limits}
In this section we assume that $\cat{C}$ is a finitely complete category and
whenever we write a pullback, we assume that it is chosen. Recall that for
any morphism $f:A\to B$ in a category $\cat{C}$ with chosen pullbacks, there
is a functor
\begin{equation*}
f^\ast : \cat{C}/B\to\cat{C}/A.
\end{equation*}
As usual, when $g:X\to B$ is a morphism, we will write $f^\ast(X)$ for the
domain of $f^\ast(g)$. When there is more than one morphism $X\to B$ involved,
as will be the case below, we will write $A\times_{f,g}X$.

\subsection{Extension objects}
\begin{defn}
An extension object $CFT$ in $\cat{C}$ consists of objects $C$, $F$ and $T$ and the
following morphisms:
\begin{align*}
c &:F\to C\\
t &:T\to F\\
e_0 &:F\to C\\
e_1 & :F\times_{e_0,c} F\to F
\end{align*}
for which the following diagrams commute:
\begin{equation*}
\begin{tikzcd}
F\times_{e_0,c} F \ar{r}{e_1} \ar{d}[swap]{e_0^\ast(c)} & F \ar{d}{c}\\
F \ar{r}[swap]{c} & C
\end{tikzcd}
\end{equation*}
\begin{equation*}
\begin{tikzcd}
F\times_{e_0,c} F \ar{d}[swap]{e_0^\ast(e_0)} \ar{r}{e_1} & F \ar{d}{e_0}\\
F \ar{r}[swap]{e_0} & C
\end{tikzcd}
\end{equation*}
\begin{equation*}
\begin{tikzcd}
(F\times_{e_0,c} F)\times_{e_1,e_0^\ast(c)}(F\times_{e_0,c} F) 
  \ar{d}[swap]{e_0^\ast(e_1)}
  \ar{r}{e_1^\ast(e_1)}
& F\times_{e_0,c} F \ar{d}{e_1} \\
F\times_{e_0,c} F \ar{r}{e_1} & F
\end{tikzcd}
\end{equation*}
\end{defn}

\begin{defn}
An extension homomorphism $f:CFT\to CFT'$ consists of
\begin{align*}
f_0 & : C\to C'\\
f_1 & : F\to F'\\
f_2 & : T\to T'\\
\end{align*}
for which the squares
\begin{equation*}
\begin{tikzcd}
F \ar{d}[swap]{c} \ar{r}{f_1} & F' \ar{d}{c'}\\
C \ar{r}[swap]{f_0} & C'
\end{tikzcd}
\end{equation*}
\begin{equation*}
\begin{tikzcd}
F \ar{d}[swap]{e_0} \ar{r}{f_1} & F' \ar{d}{e_0'}\\
C \ar{r}[swap]{f_0} & C'
\end{tikzcd}
\end{equation*}
\begin{equation*}
\begin{tikzcd}
F\times_{e_0,c} F \ar{d}[swap]{e_1} \ar{r}{e_0^\ast(f_1)} & F'\times_{e_0',c'}F' \ar{d}{e_1'}\\
F \ar{r}[swap]{f_1} & F'
\end{tikzcd}
\end{equation*}
commute.
\end{defn}

\begin{defn}
Suppose $CFT$ is an extension object of $\cat{C}$. Then we define the extension object
$\mathbf{F}_{CFT}$ to consist of the objects $F$, $F\times_{e_0,c} F$, $F\times_{e_0,t\circ c} T$
with the operations
\begin{align*}
e_0^\ast(c) & : F\times_{e_0,c} F\to F\\
e_0^\ast(t) & : F\times_{e_0,t\circ c} T\to F\times_{e_0,c} F\\
e_1 & : F\times_{e_0,c} F\to F\\
e_0^\ast(e_1) & : F\times_{e_1,e_0^\ast(c)}(F\times_{e_0,c} F)\to F\times_{e_1,e_0^\ast(c)} F
\end{align*}
\end{defn}

It requires proof that this is indeed an extension object of $\cat{C}$. However, the above
construction is a special case of the following construction of new extension objects out
of old ones with $\catid{C}:C\rightarrow C\leftarrow F:e_0$. Thus we postpone the proof
that we have an extension object to the more general case. 

\begin{defn}
Suppose $CFT$ is an extension object of $\cat{C}$ and that $f:C\rightarrow X\leftarrow D:g$.
Then we define the extension object $D\times_{g,f} CFT$ to consist of the objects $D\times_{g,f} C$,
$D\times_{g,f\circ c} F$, $D\times_{g,t\circ f\circ c} T$ and the operations
\begin{align*}
g^\ast(c) & : D\times_{g,f\circ c} F\to D\times_{g,f} C\\
g^\ast(t) & : D\times_{g,f\circ c\circ t} T\to D\times_{g,f\circ c} F\\
g^\ast(e_0) & : D\times_{g,f\circ c} F\to D\times_{g,f} C\\
g^\ast(e_1) & : (D\times_{g,f\circ c} F) \times_{g^\ast(e_0),g^\ast(c)} (D\times_{g,f\circ c} F)\to D\times_{g,f\circ c} F.
\end{align*}
\end{defn}

\begin{proof}
First we need to verify that the square
\begin{equation*}
\begin{tikzcd}
(D\times_{g,f\circ c} F) \times_{g^\ast(e_0),g^\ast(c)} (D\times_{g,f\circ c} F)
  \ar{r}{g^\ast(e_1)} 
  \ar{d}[swap]{(g^\ast(e_0))^\ast(g^\ast(c))} 
  & 
D\times_{g,f\circ c} F
  \ar{d}{g^\ast(c)}
  \\
D\times_{g,f\circ c} F
  \ar{r}[swap]{g^\ast(c)} 
  & 
D\times_{g,f} C
\end{tikzcd}
\end{equation*}
commutes. To see this, note that we have a commutative diagram
\begin{equation*}
\begin{tikzcd}
  &
(D\times_{g,f\circ c} F) \times_{g^\ast(e_0),g^\ast(c)} (D\times_{g,f\circ c} F)
  \ar{dr}{g^\ast(e_1)}
  \ar{dl}[swap]{(g^\ast(e_0))^\ast(g^\ast(c))}
  \ar{dd}{\simeq}
  \\
D\times_{g,f\circ c} F
  &
  &
D\times_{g,f\circ c} F
  \\
  &
D\times_{g,f\circ c\circ e_0^\ast(c)}(F\times_{e_0,c} F)
  \ar{ul}{g^\ast(e_0^\ast(c))}
  \ar{ur}[swap]{g^\ast(e_1)}
\end{tikzcd}
\end{equation*}
where the downwards arrow is an isomorphism. Using this observation, we see that
it is equivalent to show that the square
\begin{equation*}
\begin{tikzcd}
D\times_{g,f\circ c\circ e_0^\ast(c)}(F\times_{e_0,c} F)
  \ar{r}{g^\ast(e_1)} 
  \ar{d}[swap]{g^\ast(e_0^\ast(c))} 
  & 
D\times_{g,f\circ c} F
  \ar{d}{g^\ast(c)}
  \\
D\times_{g,f\circ c} F
  \ar{r}[swap]{g^\ast(c)} 
  & 
D\times_{g,f} C
\end{tikzcd}
\end{equation*}
commutes. But this is just a pullback along $g$ of a square which commutes by assumption.
\end{proof}

\subsection{Weakening objects}
\begin{defn}
A weakening object $CFT$ in $\cat{C}$ is an extension object $CFT$ with the
following additional structure:
\begin{align*}
w_0 & : F\times F\to F\times_{e_0,c} F\\
w_1 & : F\times(F\times_{e_0,c} F) \to F\times_{e_0,c \circ e_0^\ast(c)} (F\times_{e_0,c}F)\\
w_2 & : F\times(F\times_{e_0,t\circ c} T) \to F\times_{e_0,c}(F\times_{e_0,t\circ c} T)
\end{align*}
for which the following diagrams commute:
\begin{equation*}
\begin{tikzcd}
F\times F \ar{r}{w_0} \ar{dr}[swap]{\pi_1} & F\times_{e_0,c}F \ar{d}{e_0^\ast(c)}\\
& F
\end{tikzcd}
\end{equation*}
\begin{equation*}
\begin{tikzcd}
F\times(F\times_{e_0,c} F) 
  \ar{d}[swap]{\catid{F}\times e_0^\ast(c)}
  \ar{r}{w_1}
  &
F\times_{e_0,c\circ e_0^\ast(c)} (F\times_{e_0,c}F)
  \ar{d}{e_0^\ast(e_0^\ast(c))}
  \\
F\times F
  \ar{r}[swap]{w_0}
  &
F\times_{e_0,c} F
\end{tikzcd}
\end{equation*}
\begin{equation*}
\begin{tikzcd}
F\times(F\times_{e_0,t\circ c} T)
  \ar{r}{w_2}
  \ar{d}[swap]{\catid{F}\times e_0^\ast(t)}
  &
F\times_{e_0,c}(F\times_{e_0,t\circ c} T)
  \ar{d}{e_0^\ast(e_0^\ast(t))}
  \\
F\times(F\times_{e_0,c} F)
  \ar{r}[swap]{w_1}
  &
F\times_{e_0,c\circ e_0^\ast(c)} (F\times_{e_0,c}F)
\end{tikzcd}
\end{equation*}
\end{defn}
