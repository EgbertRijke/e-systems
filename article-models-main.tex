\documentclass{article}

%%%%%%%%%%%%%%%%%%%%%%%%%%%%%%%%%%%%%%%%%%%%%%%%%%%%%%%%%%%%%%%%%%%%%%%%%%%%%%%%
%%%% PACKAGES

\usepackage[utf8]{inputenc}
\usepackage[english]{babel}

%%%% Spicing up the document
\usepackage{mathpazo}
\usepackage[scaled=0.95]{helvet}
\usepackage{courier}
\linespread{1.05} % Palatino looks better with this
\usepackage{microtype}

\usepackage{fancyhdr} % To set headers and footers
\usepackage{enumitem,mathtools,xspace,xcolor}
\usepackage{comment}
\usepackage{ifthen}
\usepackage{pifont}
\newcommand{\cmark}{\ding{51}\xspace}
\newcommand{\xmark}{\ding{55}\xspace}

\usepackage{graphicx}
\usepackage{tikz-cd}
\usepackage{tikz}
\usetikzlibrary{decorations.pathmorphing}
\usepackage[inference]{semantic}
\usepackage{booktabs}

\usepackage[hyphens]{url} % This package has to be loaded *before* hyperref
\usepackage[pagebackref,colorlinks,citecolor=darkgreen,linkcolor=darkgreen,unicode]{hyperref}
\definecolor{darkgreen}{rgb}{0,0.45,0}

% For some reason the following can't be above hyperref...
\usepackage{amssymb,amsmath,amsthm,stmaryrd,mathrsfs,wasysym}
\usepackage{aliascnt}
\usepackage[capitalize]{cleveref}

% The braket macro shouldn't be necessary
\usepackage{braket} % used for \setof{ ... } macro

%%%%%%%%%%%%%%%%%%%%%%%%%%%%%%%%%%%%%%%%%%%%%%%%%%%%%%%%%%%%%%%%%%%%%%%%%%%%%%%%
%% To include references in TOC we should use this package rather than a hack.
\usepackage{tocbibind}
%\usepackage{etoolbox}           % get \apptocmd
%\apptocmd{\thebibliography}{\addcontentsline{toc}{section}{References}}{}{} % tell bibliography to get itself into the table of contents


\begin{comment}
%%%% Header and footers
\pagestyle{fancyplain}
\setlength{\headheight}{15pt}
\renewcommand{\chaptermark}[1]{\markboth{\textsc{Chapter \thechapter. #1}}{}}
\renewcommand{\sectionmark}[1]{\markright{\textsc{\thesection\ #1}}}
\end{comment}

% TOC depth
\setcounter{tocdepth}{3}

\lhead[\fancyplain{}{{\thepage}}]%
      {\fancyplain{}{\nouppercase{\rightmark}}}
\rhead[\fancyplain{}{\nouppercase{\leftmark}}]%
      {\fancyplain{}{\thepage}}
\cfoot{\textsc{\footnotesize [Draft of \today]}}
\lfoot[]{}
\rfoot[]{}

%%%%%%%%%%%%%%%%%%%%%%%%%%%%%%%%%%%%%%%%%%%%%%%%%%%%%%%%%%%%%%%%%%%%%%%%%%%%%%%%
%%%% We mostly use the macros of the book, to keep notations
%%%% and conventions the same. Recall that when the macros file
%%%% is updated, we need to comment the lines containing the
%%%% string `[chapter]` since our article is not a book.
%%%%
%%%% Instructions for updating the macros.tex file:
%%%% - fetch the latest macros.tex file from the HoTT/book git repository.
%%%% - comment all lines containing "[chapter]" because this is not a book.
%%%% - comment the definition of pbcorner because the xypic package is not used.
%%%%
%%%% MACROS FOR NOTATION %%%%
% Use these for any notation where there are multiple options.

%%% Notes and exercise sections
\makeatletter
\newcommand{\sectionNotes}{\phantomsection\section*{Notes}\addcontentsline{toc}{section}{Notes}\markright{\textsc{\@chapapp{} \thechapter{} Notes}}}
\newcommand{\sectionExercises}[1]{\phantomsection\section*{Exercises}\addcontentsline{toc}{section}{Exercises}\markright{\textsc{\@chapapp{} \thechapter{} Exercises}}}
\makeatother

%%% Definitional equality (used infix) %%%
\newcommand{\jdeq}{\equiv}      % An equality judgment
\let\judgeq\jdeq
%\newcommand{\defeq}{\coloneqq}  % An equality currently being defined
\newcommand{\defeq}{\vcentcolon\equiv}  % A judgmental equality currently being defined

%%% Term being defined
\newcommand{\define}[1]{\textbf{#1}}

%%% Vec (for example)

\newcommand{\Vect}{\ensuremath{\mathsf{Vec}}}
\newcommand{\Fin}{\ensuremath{\mathsf{Fin}}}
\newcommand{\fmax}{\ensuremath{\mathsf{fmax}}}
\newcommand{\seq}[1]{\langle #1\rangle}

%%% Dependent products %%%
\def\prdsym{\textstyle\prod}
%% Call the macro like \prd{x,y:A}{p:x=y} with any number of
%% arguments.  Make sure that whatever comes *after* the call doesn't
%% begin with an open-brace, or it will be parsed as another argument.
\makeatletter
% Currently the macro is configured to produce
%     {\textstyle\prod}(x:A) \; {\textstyle\prod}(y:B),\ 
% in display-math mode, and
%     \prod_{(x:A)} \prod_{y:B}
% in text-math mode.
\def\prd#1{\@ifnextchar\bgroup{\prd@parens{#1}}{\@ifnextchar\sm{\prd@parens{#1}\@eatsm}{\prd@noparens{#1}}}}
\def\prd@parens#1{\@ifnextchar\bgroup%
  {\mathchoice{\@dprd{#1}}{\@tprd{#1}}{\@tprd{#1}}{\@tprd{#1}}\prd@parens}%
  {\@ifnextchar\sm%
    {\mathchoice{\@dprd{#1}}{\@tprd{#1}}{\@tprd{#1}}{\@tprd{#1}}\@eatsm}%
    {\mathchoice{\@dprd{#1}}{\@tprd{#1}}{\@tprd{#1}}{\@tprd{#1}}}}}
\def\@eatsm\sm{\sm@parens}
\def\prd@noparens#1{\mathchoice{\@dprd@noparens{#1}}{\@tprd{#1}}{\@tprd{#1}}{\@tprd{#1}}}
% Helper macros for three styles
\def\lprd#1{\@ifnextchar\bgroup{\@lprd{#1}\lprd}{\@@lprd{#1}}}
\def\@lprd#1{\mathchoice{{\textstyle\prod}}{\prod}{\prod}{\prod}({\textstyle #1})\;}
\def\@@lprd#1{\mathchoice{{\textstyle\prod}}{\prod}{\prod}{\prod}({\textstyle #1}),\ }
\def\tprd#1{\@tprd{#1}\@ifnextchar\bgroup{\tprd}{}}
\def\@tprd#1{\mathchoice{{\textstyle\prod_{(#1)}}}{\prod_{(#1)}}{\prod_{(#1)}}{\prod_{(#1)}}}
\def\dprd#1{\@dprd{#1}\@ifnextchar\bgroup{\dprd}{}}
\def\@dprd#1{\prod_{(#1)}\,}
\def\@dprd@noparens#1{\prod_{#1}\,}

%%% Lambda abstractions.
% Each variable being abstracted over is a separate argument.  If
% there is more than one such argument, they *must* be enclosed in
% braces.  Arguments can be untyped, as in \lam{x}{y}, or typed with a
% colon, as in \lam{x:A}{y:B}. In the latter case, the colons are
% automatically noticed and (with current implementation) the space
% around the colon is reduced.  You can even give more than one variable
% the same type, as in \lam{x,y:A}.
\def\lam#1{{\lambda}\@lamarg#1:\@endlamarg\@ifnextchar\bgroup{.\,\lam}{.\,}}
\def\@lamarg#1:#2\@endlamarg{\if\relax\detokenize{#2}\relax #1\else\@lamvar{\@lameatcolon#2},#1\@endlamvar\fi}
\def\@lamvar#1,#2\@endlamvar{(#2\,{:}\,#1)}
% \def\@lamvar#1,#2{{#2}^{#1}\@ifnextchar,{.\,{\lambda}\@lamvar{#1}}{\let\@endlamvar\relax}}
\def\@lameatcolon#1:{#1}
\let\lamt\lam
% This version silently eats any typing annotation.
\def\lamu#1{{\lambda}\@lamuarg#1:\@endlamuarg\@ifnextchar\bgroup{.\,\lamu}{.\,}}
\def\@lamuarg#1:#2\@endlamuarg{#1}

%%% Dependent products written with \forall, in the same style
\def\fall#1{\forall (#1)\@ifnextchar\bgroup{.\,\fall}{.\,}}

%%% Existential quantifier %%%
\def\exis#1{\exists (#1)\@ifnextchar\bgroup{.\,\exis}{.\,}}

%%% Dependent sums %%%
\def\smsym{\textstyle\sum}
% Use in the same way as \prd
\def\sm#1{\@ifnextchar\bgroup{\sm@parens{#1}}{\@ifnextchar\prd{\sm@parens{#1}\@eatprd}{\sm@noparens{#1}}}}
\def\sm@parens#1{\@ifnextchar\bgroup%
  {\mathchoice{\@dsm{#1}}{\@tsm{#1}}{\@tsm{#1}}{\@tsm{#1}}\sm@parens}%
  {\@ifnextchar\prd%
    {\mathchoice{\@dsm{#1}}{\@tsm{#1}}{\@tsm{#1}}{\@tsm{#1}}\@eatprd}%
    {\mathchoice{\@dsm{#1}}{\@tsm{#1}}{\@tsm{#1}}{\@tsm{#1}}}}}
\def\@eatprd\prd{\prd@parens}
\def\sm@noparens#1{\mathchoice{\@dsm@noparens{#1}}{\@tsm{#1}}{\@tsm{#1}}{\@tsm{#1}}}
\def\lsm#1{\@ifnextchar\bgroup{\@lsm{#1}\lsm}{\@@lsm{#1}}}
\def\@lsm#1{\mathchoice{{\textstyle\sum}}{\sum}{\sum}{\sum}({\textstyle #1})\;}
\def\@@lsm#1{\mathchoice{{\textstyle\sum}}{\sum}{\sum}{\sum}({\textstyle #1}),\ }
\def\tsm#1{\@tsm{#1}\@ifnextchar\bgroup{\tsm}{}}
\def\@tsm#1{\mathchoice{{\textstyle\sum_{(#1)}}}{\sum_{(#1)}}{\sum_{(#1)}}{\sum_{(#1)}}}
\def\dsm#1{\@dsm{#1}\@ifnextchar\bgroup{\dsm}{}}
\def\@dsm#1{\sum_{(#1)}\,}
\def\@dsm@noparens#1{\sum_{#1}\,}

%%% W-types
\def\wtypesym{{\mathsf{W}}}
\def\wtype#1{\@ifnextchar\bgroup%
  {\mathchoice{\@twtype{#1}}{\@twtype{#1}}{\@twtype{#1}}{\@twtype{#1}}\wtype}%
  {\mathchoice{\@twtype{#1}}{\@twtype{#1}}{\@twtype{#1}}{\@twtype{#1}}}}
\def\lwtype#1{\@ifnextchar\bgroup{\@lwtype{#1}\lwtype}{\@@lwtype{#1}}}
\def\@lwtype#1{\mathchoice{{\textstyle\mathsf{W}}}{\mathsf{W}}{\mathsf{W}}{\mathsf{W}}({\textstyle #1})\;}
\def\@@lwtype#1{\mathchoice{{\textstyle\mathsf{W}}}{\mathsf{W}}{\mathsf{W}}{\mathsf{W}}({\textstyle #1}),\ }
\def\twtype#1{\@twtype{#1}\@ifnextchar\bgroup{\twtype}{}}
\def\@twtype#1{\mathchoice{{\textstyle\mathsf{W}_{(#1)}}}{\mathsf{W}_{(#1)}}{\mathsf{W}_{(#1)}}{\mathsf{W}_{(#1)}}}
\def\dwtype#1{\@dwtype{#1}\@ifnextchar\bgroup{\dwtype}{}}
\def\@dwtype#1{\mathsf{W}_{(#1)}\,}

\newcommand{\suppsym}{{\mathsf{sup}}}
\newcommand{\supp}{\ensuremath\suppsym\xspace}

\def\wtypeh#1{\@ifnextchar\bgroup%
  {\mathchoice{\@lwtypeh{#1}}{\@twtypeh{#1}}{\@twtypeh{#1}}{\@twtypeh{#1}}\wtypeh}%
  {\mathchoice{\@@lwtypeh{#1}}{\@twtypeh{#1}}{\@twtypeh{#1}}{\@twtypeh{#1}}}}
\def\lwtypeh#1{\@ifnextchar\bgroup{\@lwtypeh{#1}\lwtypeh}{\@@lwtypeh{#1}}}
\def\@lwtypeh#1{\mathchoice{{\textstyle\mathsf{W}^h}}{\mathsf{W}^h}{\mathsf{W}^h}{\mathsf{W}^h}({\textstyle #1})\;}
\def\@@lwtypeh#1{\mathchoice{{\textstyle\mathsf{W}^h}}{\mathsf{W}^h}{\mathsf{W}^h}{\mathsf{W}^h}({\textstyle #1}),\ }
\def\twtypeh#1{\@twtypeh{#1}\@ifnextchar\bgroup{\twtypeh}{}}
\def\@twtypeh#1{\mathchoice{{\textstyle\mathsf{W}^h_{(#1)}}}{\mathsf{W}^h_{(#1)}}{\mathsf{W}^h_{(#1)}}{\mathsf{W}^h_{(#1)}}}
\def\dwtypeh#1{\@dwtypeh{#1}\@ifnextchar\bgroup{\dwtypeh}{}}
\def\@dwtypeh#1{\mathsf{W}^h_{(#1)}\,}


\makeatother

% Other notations related to dependent sums
\let\setof\Set    % from package 'braket', write \setof{ x:A | P(x) }.
\newcommand{\pair}{\ensuremath{\mathsf{pair}}\xspace}
\newcommand{\tup}[2]{(#1,#2)}
\newcommand{\proj}[1]{\ensuremath{\mathsf{pr}_{#1}}\xspace}
\newcommand{\fst}{\ensuremath{\proj1}\xspace}
\newcommand{\snd}{\ensuremath{\proj2}\xspace}
\newcommand{\ac}{\ensuremath{\mathsf{ac}}\xspace} % not needed in symbol index
\newcommand{\un}{\ensuremath{\mathsf{upun}}\xspace} % not needed in symbol index, uniqueness principle for unit type

%%% recursor and induction
\newcommand{\rec}[1]{\mathsf{rec}_{#1}}
\newcommand{\ind}[1]{\mathsf{ind}_{#1}}
\newcommand{\indid}[1]{\ind{=_{#1}}} % (Martin-Lof) path induction principle for identity types
\newcommand{\indidb}[1]{\ind{=_{#1}}'} % (Paulin-Mohring) based path induction principle for identity types 

%%% the uniqueness principle for product types, formerly called surjective pairing and named \spr:
\newcommand{\uppt}{\ensuremath{\mathsf{uppt}}\xspace}

% Paths in pairs
\newcommand{\pairpath}{\ensuremath{\mathsf{pair}^{\mathord{=}}}\xspace}
% \newcommand{\projpath}[1]{\proj{#1}^{\mathord{=}}}
\newcommand{\projpath}[1]{\ensuremath{\apfunc{\proj{#1}}}\xspace}

%%% For quotients %%%
%\newcommand{\pairr}[1]{{\langle #1\rangle}}
\newcommand{\pairr}[1]{{\mathopen{}(#1)\mathclose{}}}
\newcommand{\Pairr}[1]{{\mathopen{}\left(#1\right)\mathclose{}}}

% \newcommand{\type}{\ensuremath{\mathsf{Type}}} % this command is overridden below, so it's commented out
\newcommand{\im}{\ensuremath{\mathsf{im}}} % the image

%%% 2D path operations
\newcommand{\leftwhisker}{\mathbin{{\ct}_{\ell}}}
\newcommand{\rightwhisker}{\mathbin{{\ct}_{r}}}
\newcommand{\hct}{\star}

%%% modalities %%%
\newcommand{\modal}{\ensuremath{\ocircle}}
\let\reflect\modal
\newcommand{\modaltype}{\ensuremath{\type_\modal}}
% \newcommand{\ism}[1]{\ensuremath{\mathsf{is}_{#1}}}
% \newcommand{\ismodal}{\ism{\modal}}
% \newcommand{\existsmodal}{\ensuremath{{\exists}_{\modal}}}
% \newcommand{\existsmodalunique}{\ensuremath{{\exists!}_{\modal}}}
% \newcommand{\modalfunc}{\textsf{\modal-fun}}
% \newcommand{\Ecirc}{\ensuremath{\mathsf{E}_\modal}}
% \newcommand{\Mcirc}{\ensuremath{\mathsf{M}_\modal}}
\newcommand{\mreturn}{\ensuremath{\eta}}
\let\project\mreturn
%\newcommand{\mbind}[1]{\ensuremath{\hat{#1}}}
\newcommand{\ext}{\mathsf{ext}}
%\newcommand{\mmap}[1]{\ensuremath{\bar{#1}}}
%\newcommand{\mjoin}{\ensuremath{\mreturn^{-1}}}
% Subuniverse
\renewcommand{\P}{\ensuremath{\type_{P}}\xspace}

%%% Localizations
% \newcommand{\islocal}[1]{\ensuremath{\mathsf{islocal}_{#1}}\xspace}
% \newcommand{\loc}[1]{\ensuremath{\mathcal{L}_{#1}}\xspace}

%%% Identity types %%%
\newcommand{\idsym}{{=}}
\newcommand{\id}[3][]{\ensuremath{#2 =_{#1} #3}\xspace}
\newcommand{\idtype}[3][]{\ensuremath{\mathsf{Id}_{#1}(#2,#3)}\xspace}
\newcommand{\idtypevar}[1]{\ensuremath{\mathsf{Id}_{#1}}\xspace}
% A propositional equality currently being defined
\newcommand{\defid}{\coloneqq}

%%% Dependent paths
\newcommand{\dpath}[4]{#3 =^{#1}_{#2} #4}

%%% singleton
% \newcommand{\sgl}{\ensuremath{\mathsf{sgl}}\xspace}
% \newcommand{\sctr}{\ensuremath{\mathsf{sctr}}\xspace}

%%% Reflexivity terms %%%
% \newcommand{\reflsym}{{\mathsf{refl}}}
\newcommand{\refl}[1]{\ensuremath{\mathsf{refl}_{#1}}\xspace}

%%% Path concatenation (used infix, in diagrammatic order) %%%
\newcommand{\ct}{%
  \mathchoice{\mathbin{\raisebox{0.5ex}{$\displaystyle\centerdot$}}}%
             {\mathbin{\raisebox{0.5ex}{$\centerdot$}}}%
             {\mathbin{\raisebox{0.25ex}{$\scriptstyle\,\centerdot\,$}}}%
             {\mathbin{\raisebox{0.1ex}{$\scriptscriptstyle\,\centerdot\,$}}}
}

%%% Path reversal %%%
\newcommand{\opp}[1]{\mathord{{#1}^{-1}}}
\let\rev\opp

%%% Transport (covariant) %%%
\newcommand{\trans}[2]{\ensuremath{{#1}_{*}\mathopen{}\left({#2}\right)\mathclose{}}\xspace}
\let\Trans\trans
%\newcommand{\Trans}[2]{\ensuremath{{#1}_{*}\left({#2}\right)}\xspace}
\newcommand{\transf}[1]{\ensuremath{{#1}_{*}}\xspace} % Without argument
%\newcommand{\transport}[2]{\ensuremath{\mathsf{transport}_{*} \: {#2}\xspace}}
\newcommand{\transfib}[3]{\ensuremath{\mathsf{transport}^{#1}(#2,#3)\xspace}}
\newcommand{\Transfib}[3]{\ensuremath{\mathsf{transport}^{#1}\Big(#2,\, #3\Big)\xspace}}
\newcommand{\transfibf}[1]{\ensuremath{\mathsf{transport}^{#1}\xspace}}

%%% 2D transport
\newcommand{\transtwo}[2]{\ensuremath{\mathsf{transport}^2\mathopen{}\left({#1},{#2}\right)\mathclose{}}\xspace}

%%% Constant transport
\newcommand{\transconst}[3]{\ensuremath{\mathsf{transportconst}}^{#1}_{#2}(#3)\xspace}
\newcommand{\transconstf}{\ensuremath{\mathsf{transportconst}}\xspace}

%%% Map on paths %%%
\newcommand{\mapfunc}[1]{\ensuremath{\mathsf{ap}_{#1}}\xspace} % Without argument
\newcommand{\map}[2]{\ensuremath{{#1}\mathopen{}\left({#2}\right)\mathclose{}}\xspace}
\let\Ap\map
%\newcommand{\Ap}[2]{\ensuremath{{#1}\left({#2}\right)}\xspace}
\newcommand{\mapdepfunc}[1]{\ensuremath{\mathsf{apd}_{#1}}\xspace} % Without argument
% \newcommand{\mapdep}[2]{\ensuremath{{#1}\llparenthesis{#2}\rrparenthesis}\xspace}
\newcommand{\mapdep}[2]{\ensuremath{\mapdepfunc{#1}\mathopen{}\left(#2\right)\mathclose{}}\xspace}
\let\apfunc\mapfunc
\let\ap\map
\let\apdfunc\mapdepfunc
\let\apd\mapdep

%%% 2D map on paths
\newcommand{\aptwofunc}[1]{\ensuremath{\mathsf{ap}^2_{#1}}\xspace}
\newcommand{\aptwo}[2]{\ensuremath{\aptwofunc{#1}\mathopen{}\left({#2}\right)\mathclose{}}\xspace}
\newcommand{\apdtwofunc}[1]{\ensuremath{\mathsf{apd}^2_{#1}}\xspace}
\newcommand{\apdtwo}[2]{\ensuremath{\apdtwofunc{#1}\mathopen{}\left(#2\right)\mathclose{}}\xspace}

%%% Identity functions %%%
\newcommand{\idfunc}[1][]{\ensuremath{\mathsf{id}_{#1}}\xspace}

%%% Homotopies (written infix) %%%
\newcommand{\htpy}{\sim}

%%% Other meanings of \sim
\newcommand{\bisim}{\sim}       % bisimulation
\newcommand{\eqr}{\sim}         % an equivalence relation

%%% Equivalence types %%%
\newcommand{\eqv}[2]{\ensuremath{#1 \simeq #2}\xspace}
\newcommand{\eqvspaced}[2]{\ensuremath{#1 \;\simeq\; #2}\xspace}
\newcommand{\eqvsym}{\simeq}    % infix symbol
\newcommand{\texteqv}[2]{\ensuremath{\mathsf{Equiv}(#1,#2)}\xspace}
\newcommand{\isequiv}{\ensuremath{\mathsf{isequiv}}}
\newcommand{\qinv}{\ensuremath{\mathsf{qinv}}}
\newcommand{\ishae}{\ensuremath{\mathsf{ishae}}}
\newcommand{\linv}{\ensuremath{\mathsf{linv}}}
\newcommand{\rinv}{\ensuremath{\mathsf{rinv}}}
\newcommand{\biinv}{\ensuremath{\mathsf{biinv}}}
\newcommand{\lcoh}[3]{\mathsf{lcoh}_{#1}(#2,#3)}
\newcommand{\rcoh}[3]{\mathsf{rcoh}_{#1}(#2,#3)}
\newcommand{\hfib}[2]{{\mathsf{fib}}_{#1}(#2)}

%%% Map on total spaces %%%
\newcommand{\total}[1]{\ensuremath{\mathsf{total}(#1)}}

%%% Universe types %%%
%\newcommand{\type}{\ensuremath{\mathsf{Type}}\xspace}
\newcommand{\UU}{\ensuremath{\mathcal{U}}\xspace}
\let\bbU\UU
\let\type\UU
% Universes of truncated types
\newcommand{\typele}[1]{\ensuremath{{#1}\text-\mathsf{Type}}\xspace}
\newcommand{\typeleU}[1]{\ensuremath{{#1}\text-\mathsf{Type}_\UU}\xspace}
\newcommand{\typelep}[1]{\ensuremath{{(#1)}\text-\mathsf{Type}}\xspace}
\newcommand{\typelepU}[1]{\ensuremath{{(#1)}\text-\mathsf{Type}_\UU}\xspace}
\let\ntype\typele
\let\ntypeU\typeleU
\let\ntypep\typelep
\let\ntypepU\typelepU
\renewcommand{\set}{\ensuremath{\mathsf{Set}}\xspace}
\newcommand{\setU}{\ensuremath{\mathsf{Set}_\UU}\xspace}
\newcommand{\prop}{\ensuremath{\mathsf{Prop}}\xspace}
\newcommand{\propU}{\ensuremath{\mathsf{Prop}_\UU}\xspace}
%Pointed types
\newcommand{\pointed}[1]{\ensuremath{#1_\bullet}}

%%% Ordinals and cardinals
\newcommand{\card}{\ensuremath{\mathsf{Card}}\xspace}
\newcommand{\ord}{\ensuremath{\mathsf{Ord}}\xspace}
\newcommand{\ordsl}[2]{{#1}_{/#2}}

%%% Univalence
\newcommand{\ua}{\ensuremath{\mathsf{ua}}\xspace} % the inverse of idtoeqv
\newcommand{\idtoeqv}{\ensuremath{\mathsf{idtoeqv}}\xspace}
\newcommand{\univalence}{\ensuremath{\mathsf{univalence}}\xspace} % the full axiom

%%% Truncation levels
\newcommand{\iscontr}{\ensuremath{\mathsf{isContr}}}
\newcommand{\contr}{\ensuremath{\mathsf{contr}}} % The path to the center of contraction
\newcommand{\isset}{\ensuremath{\mathsf{isSet}}}
\newcommand{\isprop}{\ensuremath{\mathsf{isProp}}}
% h-propositions
% \newcommand{\anhprop}{a mere proposition\xspace}
% \newcommand{\hprops}{mere propositions\xspace}

%%% Homotopy fibers %%%
%\newcommand{\hfiber}[2]{\ensuremath{\mathsf{hFiber}(#1,#2)}\xspace}
\let\hfiber\hfib

%%% Bracket/squash/truncation types %%%
% \newcommand{\brck}[1]{\textsf{mere}(#1)}
% \newcommand{\Brck}[1]{\textsf{mere}\Big(#1\Big)}
% \newcommand{\trunc}[2]{\tau_{#1}(#2)}
% \newcommand{\Trunc}[2]{\tau_{#1}\Big(#2\Big)}
% \newcommand{\truncf}[1]{\tau_{#1}}
%\newcommand{\trunc}[2]{\Vert #2\Vert_{#1}}
\newcommand{\trunc}[2]{\mathopen{}\left\Vert #2\right\Vert_{#1}\mathclose{}}
\newcommand{\ttrunc}[2]{\bigl\Vert #2\bigr\Vert_{#1}}
\newcommand{\Trunc}[2]{\Bigl\Vert #2\Bigr\Vert_{#1}}
\newcommand{\truncf}[1]{\Vert \blank \Vert_{#1}}
\newcommand{\tproj}[3][]{\mathopen{}\left|#3\right|_{#2}^{#1}\mathclose{}}
\newcommand{\tprojf}[2][]{|\blank|_{#2}^{#1}}
\def\pizero{\trunc0}
%\newcommand{\brck}[1]{\trunc{-1}{#1}}
%\newcommand{\Brck}[1]{\Trunc{-1}{#1}}
%\newcommand{\bproj}[1]{\tproj{-1}{#1}}
%\newcommand{\bprojf}{\tprojf{-1}}

\newcommand{\brck}[1]{\trunc{}{#1}}
\newcommand{\bbrck}[1]{\ttrunc{}{#1}}
\newcommand{\Brck}[1]{\Trunc{}{#1}}
\newcommand{\bproj}[1]{\tproj{}{#1}}
\newcommand{\bprojf}{\tprojf{}}

% Big parentheses
\newcommand{\Parens}[1]{\Bigl(#1\Bigr)}

% Projection and extension for truncations
\let\extendsmb\ext
\newcommand{\extend}[1]{\extendsmb(#1)}

%
%%% The empty type
\newcommand{\emptyt}{\ensuremath{\mathbf{0}}\xspace}

%%% The unit type
\newcommand{\unit}{\ensuremath{\mathbf{1}}\xspace}
\newcommand{\ttt}{\ensuremath{\star}\xspace}

%%% The two-element type
\newcommand{\bool}{\ensuremath{\mathbf{2}}\xspace}
\newcommand{\btrue}{{1_{\bool}}}
\newcommand{\bfalse}{{0_{\bool}}}

%%% Injections into binary sums and pushouts
\newcommand{\inlsym}{{\mathsf{inl}}}
\newcommand{\inrsym}{{\mathsf{inr}}}
\newcommand{\inl}{\ensuremath\inlsym\xspace}
\newcommand{\inr}{\ensuremath\inrsym\xspace}

%%% The segment of the interval
\newcommand{\seg}{\ensuremath{\mathsf{seg}}\xspace}

%%% Free groups
\newcommand{\freegroup}[1]{F(#1)}
\newcommand{\freegroupx}[1]{F'(#1)} % the "other" free group

%%% Glue of a pushout
\newcommand{\glue}{\mathsf{glue}}

%%% Circles and spheres
\newcommand{\Sn}{\mathbb{S}}
\newcommand{\base}{\ensuremath{\mathsf{base}}\xspace}
\newcommand{\lloop}{\ensuremath{\mathsf{loop}}\xspace}
\newcommand{\surf}{\ensuremath{\mathsf{surf}}\xspace}

%%% Suspension
\newcommand{\susp}{\Sigma}
\newcommand{\north}{\mathsf{N}}
\newcommand{\south}{\mathsf{S}}
\newcommand{\merid}{\mathsf{merid}}

%%% Blanks (shorthand for lambda abstractions)
\newcommand{\blank}{\mathord{\hspace{1pt}\text{--}\hspace{1pt}}}

%%% Nameless objects
\newcommand{\nameless}{\mathord{\hspace{1pt}\underline{\hspace{1ex}}\hspace{1pt}}}

%%% Some decorations
%\newcommand{\bbU}{\ensuremath{\mathbb{U}}\xspace}
% \newcommand{\bbB}{\ensuremath{\mathbb{B}}\xspace}
\newcommand{\bbP}{\ensuremath{\mathbb{P}}\xspace}

%%% Some categories
\newcommand{\uset}{\ensuremath{\mathcal{S}et}\xspace}
\newcommand{\ucat}{\ensuremath{{\mathcal{C}at}}\xspace}
\newcommand{\urel}{\ensuremath{\mathcal{R}el}\xspace}
\newcommand{\uhilb}{\ensuremath{\mathcal{H}ilb}\xspace}
\newcommand{\utype}{\ensuremath{\mathcal{T}\!ype}\xspace}

% Pullback corner
%\newbox\pbbox
%\setbox\pbbox=\hbox{\xy \POS(65,0)\ar@{-} (0,0) \ar@{-} (65,65)\endxy}
%\def\pb{\save[]+<3.5mm,-3.5mm>*{\copy\pbbox} \restore}

% Macros for the categories chapter
\newcommand{\inv}[1]{{#1}^{-1}}
\newcommand{\idtoiso}{\ensuremath{\mathsf{idtoiso}}\xspace}
\newcommand{\isotoid}{\ensuremath{\mathsf{isotoid}}\xspace}
\newcommand{\op}{^{\mathrm{op}}}
\newcommand{\y}{\ensuremath{\mathbf{y}}\xspace}
\newcommand{\dgr}[1]{{#1}^{\dagger}}
\newcommand{\unitaryiso}{\mathrel{\cong^\dagger}}
\newcommand{\cteqv}[2]{\ensuremath{#1 \simeq #2}\xspace}
\newcommand{\cteqvsym}{\simeq}     % Symbol for equivalence of categories

%%% Natural numbers
\newcommand{\N}{\ensuremath{\mathbb{N}}\xspace}
%\newcommand{\N}{\textbf{N}}
\let\nat\N
\newcommand{\natp}{\ensuremath{\nat'}\xspace} % alternative nat in induction chapter

\newcommand{\zerop}{\ensuremath{0'}\xspace}   % alternative zero in induction chapter
\newcommand{\suc}{\mathsf{succ}}
\newcommand{\sucp}{\ensuremath{\suc'}\xspace} % alternative suc in induction chapter
\newcommand{\add}{\mathsf{add}}
\newcommand{\ack}{\mathsf{ack}}
\newcommand{\ite}{\mathsf{iter}}
\newcommand{\assoc}{\mathsf{assoc}}
\newcommand{\dbl}{\ensuremath{\mathsf{double}}}
\newcommand{\dblp}{\ensuremath{\dbl'}\xspace} % alternative double in induction chapter


%%% Lists
\newcommand{\lst}[1]{\mathsf{List}(#1)}
\newcommand{\nil}{\mathsf{nil}}
\newcommand{\cons}{\mathsf{cons}}

%%% Vectors of given length, used in induction chapter
\newcommand{\vect}[2]{\ensuremath{\mathsf{Vec}_{#1}(#2)}\xspace}

%%% Integers
\newcommand{\Z}{\ensuremath{\mathbb{Z}}\xspace}
\newcommand{\Zsuc}{\mathsf{succ}}
\newcommand{\Zpred}{\mathsf{pred}}

%%% Rationals
\newcommand{\Q}{\ensuremath{\mathbb{Q}}\xspace}

%%% Function extensionality
\newcommand{\funext}{\mathsf{funext}}
\newcommand{\happly}{\mathsf{happly}}

%%% A naturality lemma
\newcommand{\com}[3]{\mathsf{swap}_{#1,#2}(#3)}

%%% Code/encode/decode
\newcommand{\code}{\ensuremath{\mathsf{code}}\xspace}
\newcommand{\encode}{\ensuremath{\mathsf{encode}}\xspace}
\newcommand{\decode}{\ensuremath{\mathsf{decode}}\xspace}

% Function definition with domain and codomain
\newcommand{\function}[4]{\left\{\begin{array}{rcl}#1 &
      \longrightarrow & #2 \\ #3 & \longmapsto & #4 \end{array}\right.}

%%% Cones and cocones
\newcommand{\cone}[2]{\mathsf{cone}_{#1}(#2)}
\newcommand{\cocone}[2]{\mathsf{cocone}_{#1}(#2)}
% Apply a function to a cocone
\newcommand{\composecocone}[2]{#1\circ#2}
\newcommand{\composecone}[2]{#2\circ#1}
%%% Diagrams
\newcommand{\Ddiag}{\mathscr{D}}

%%% (pointed) mapping spaces
\newcommand{\Map}{\mathsf{Map}}

%%% The interval
\newcommand{\interval}{\ensuremath{I}\xspace}
\newcommand{\izero}{\ensuremath{0_{\interval}}\xspace}
\newcommand{\ione}{\ensuremath{1_{\interval}}\xspace}

%%% Arrows
\newcommand{\epi}{\ensuremath{\twoheadrightarrow}}
\newcommand{\mono}{\ensuremath{\rightarrowtail}}

%%% Sets
\newcommand{\bin}{\ensuremath{\mathrel{\widetilde{\in}}}}

%%% Semigroup structure
\newcommand{\semigroupstrsym}{\ensuremath{\mathsf{SemigroupStr}}}
\newcommand{\semigroupstr}[1]{\ensuremath{\mathsf{SemigroupStr}}(#1)}
\newcommand{\semigroup}[0]{\ensuremath{\mathsf{Semigroup}}}

%%% Macros for the formal type theory
\newcommand{\emptyctx}{\ensuremath{\cdot}}
\newcommand{\production}{\vcentcolon\vcentcolon=}
\newcommand{\conv}{\downarrow}
\newcommand{\wfctx}[1]{#1\ \ctx}
\newcommand{\oftp}[3]{#1 \vdash #2 : #3}
\newcommand{\jdeqtp}[4]{#1 \vdash #2 \jdeq #3 : #4}
\newcommand{\judg}[2]{#1 \vdash #2}
\newcommand{\tmtp}[2]{#1 \mathord{:} #2}

% rule names
\newcommand{\form}{\textsc{form}}
\newcommand{\intro}{\textsc{intro}}
\newcommand{\elim}{\textsc{elim}}
\newcommand{\comp}{\textsc{comp}}
\newcommand{\uniq}{\textsc{uniq}}
\newcommand{\Weak}{\mathsf{Wkg}}
\newcommand{\Vble}{\mathsf{Vble}}
\newcommand{\Exch}{\mathsf{Exch}}
\newcommand{\Subst}{\mathsf{Subst}}

%%% Macros for HITs
\newcommand{\cc}{\mathsf{c}}
\newcommand{\pp}{\mathsf{p}}
\newcommand{\cct}{\widetilde{\mathsf{c}}}
\newcommand{\ppt}{\widetilde{\mathsf{p}}}
\newcommand{\Wtil}{\ensuremath{\widetilde{W}}\xspace}

%%% Macros for n-types
\newcommand{\istype}[1]{\mathsf{is}\mbox{-}{#1}\mbox{-}\mathsf{type}}
\newcommand{\nplusone}{\ensuremath{(n+1)}}
\newcommand{\nminusone}{\ensuremath{(n-1)}}
\newcommand{\fact}{\mathsf{fact}}

%%% Macros for homotopy
\newcommand{\kbar}{\overline{k}} % Used in van Kampen's theorem

%%% Macros for induction
\newcommand{\natw}{\ensuremath{\mathbf{N^w}}\xspace}
\newcommand{\zerow}{\ensuremath{0^\mathbf{w}}\xspace}
\newcommand{\sucw}{\ensuremath{\mathbf{s^w}}\xspace}
\newcommand{\nalg}{\nat\mathsf{Alg}}
\newcommand{\nhom}{\nat\mathsf{Hom}}
\newcommand{\ishinitw}{\mathsf{isHinit}_{\mathsf{W}}}
\newcommand{\ishinitn}{\mathsf{isHinit}_\nat}
\newcommand{\w}{\mathsf{W}}
\newcommand{\walg}{\w\mathsf{Alg}}
\newcommand{\whom}{\w\mathsf{Hom}}

%%% Macros for real numbers
\newcommand{\RC}{\ensuremath{\mathbb{R}_\mathsf{c}}\xspace} % Cauchy
\newcommand{\RD}{\ensuremath{\mathbb{R}_\mathsf{d}}\xspace} % Dedekind
\newcommand{\R}{\ensuremath{\mathbb{R}}\xspace}           % Either 
\newcommand{\barRD}{\ensuremath{\bar{\mathbb{R}}_\mathsf{d}}\xspace} % Dedekind completion of Dedekind

\newcommand{\close}[1]{\sim_{#1}} % Relation of closeness
\newcommand{\closesym}{\mathord\sim}
\newcommand{\rclim}{\mathsf{lim}} % HIT constructor for Cauchy reals
\newcommand{\rcrat}{\mathsf{rat}} % Embedding of rationals into Cauchy reals
\newcommand{\rceq}{\mathsf{eq}_{\RC}} % HIT path constructor
\newcommand{\CAP}{\mathcal{C}}    % The type of Cauchy approximations
\newcommand{\Qp}{\Q_{+}}
\newcommand{\apart}{\mathrel{\#}}  % apartness
\newcommand{\dcut}{\mathsf{isCut}}  % Dedekind cut
\newcommand{\cover}{\triangleleft} % inductive cover
\newcommand{\intfam}[3]{(#2, \lam{#1} #3)} % family of rational intervals

% Macros for the Cauchy reals construction
\newcommand{\bsim}{\frown}
\newcommand{\bbsim}{\smile}

\newcommand{\hapx}{\diamondsuit\approx}
\newcommand{\hapname}{\diamondsuit}
\newcommand{\hapxb}{\heartsuit\approx}
\newcommand{\hapbname}{\heartsuit}
\newcommand{\tap}[1]{\bullet\approx_{#1}\triangle}
\newcommand{\tapname}{\triangle}
\newcommand{\tapb}[1]{\bullet\approx_{#1}\square}
\newcommand{\tapbname}{\square}

%%% Macros for surreals
\newcommand{\NO}{\ensuremath{\mathsf{No}}\xspace}
\newcommand{\surr}[2]{\{\,#1\,\big|\,#2\,\}}
\newcommand{\LL}{\mathcal{L}}
\newcommand{\RR}{\mathcal{R}}
\newcommand{\noeq}{\mathsf{eq}_{\NO}} % HIT path constructor

\newcommand{\ble}{\trianglelefteqslant}
\newcommand{\blt}{\vartriangleleft}
\newcommand{\bble}{\sqsubseteq}
\newcommand{\bblt}{\sqsubset}

\newcommand{\hle}{\diamondsuit\preceq}
\newcommand{\hlt}{\diamondsuit\prec}
\newcommand{\hlname}{\diamondsuit}
\newcommand{\hleb}{\heartsuit\preceq}
\newcommand{\hltb}{\heartsuit\prec}
\newcommand{\hlbname}{\heartsuit}
% \newcommand{\tle}{(\bullet\preceq\triangle)}
% \newcommand{\tlt}{(\bullet\prec\triangle)}
\newcommand{\tle}{\triangle\preceq}
\newcommand{\tlt}{\triangle\prec}
\newcommand{\tlname}{\triangle}
% \newcommand{\tleb}{(\bullet\preceq\square)}
% \newcommand{\tltb}{(\bullet\prec\square)}
\newcommand{\tleb}{\square\preceq}
\newcommand{\tltb}{\square\prec}
\newcommand{\tlbname}{\square}

%%% Macros for set theory
\newcommand{\vset}{\mathsf{set}}  % point constructor for cummulative hierarchy V
\def\cd{\tproj0}
\newcommand{\inj}{\ensuremath{\mathsf{inj}}} % type of injections
\newcommand{\acc}{\ensuremath{\mathsf{acc}}} % accessibility

\newcommand{\atMostOne}{\mathsf{atMostOne}}

\newcommand{\power}[1]{\mathcal{P}(#1)} % power set
\newcommand{\powerp}[1]{\mathcal{P}_+(#1)} % inhabited power set

%%%% THEOREM ENVIRONMENTS %%%%

% Hyperref includes the command \autoref{...} which is like \ref{...}
% except that it automatically inserts the type of the thing you're
% referring to, e.g. it produces "Theorem 3.8" instead of just "3.8"
% (and makes the whole thing a hyperlink).  This saves a slight amount
% of typing, but more importantly it means that if you decide later on
% that 3.8 should be a Lemma or a Definition instead of a Theorem, you
% don't have to change the name in all the places you referred to it.

% The following hack improves on this by using the same counter for
% all theorem-type environments, so that after Theorem 1.1 comes
% Corollary 1.2 rather than Corollary 1.1.  This makes it much easier
% for the reader to find a particular theorem when flipping through
% the document.
\makeatletter
\def\defthm#1#2#3{%
  %% Ensure all theorem types are numbered with the same counter
  \newaliascnt{#1}{thm}
  \newtheorem{#1}[#1]{#2}
  \aliascntresetthe{#1}
  %% This command tells cleveref's \cref what to call things
  \crefname{#1}{#2}{#3}}

% Now define a bunch of theorem-type environments.
\newtheorem{thm}{Theorem}[section]
\crefname{thm}{Theorem}{Theorems}
%\defthm{prop}{Proposition}   % Probably we shouldn't use "Proposition" in this way
\defthm{cor}{Corollary}{Corollaries}
\defthm{lem}{Lemma}{Lemmas}
\defthm{axiom}{Axiom}{Axioms}
% Since definitions and theorems in type theory are synonymous, should
% we actually use the same theoremstyle for them?
\theoremstyle{definition}
\defthm{defn}{Definition}{Definitions}
\theoremstyle{remark}
\defthm{rmk}{Remark}{Remarks}
\defthm{eg}{Example}{Examples}
\defthm{egs}{Examples}{Examples}
\defthm{notes}{Notes}{Notes}
% Number exercises within chapters, with their own counter.
%\newtheorem{ex}{Exercise}[chapter]
%\crefname{ex}{Exercise}{Exercises}

% Display format for sections
\crefformat{section}{\S#2#1#3}
\Crefformat{section}{Section~#2#1#3}
\crefrangeformat{section}{\S\S#3#1#4--#5#2#6}
\Crefrangeformat{section}{Sections~#3#1#4--#5#2#6}
\crefmultiformat{section}{\S\S#2#1#3}{ and~#2#1#3}{, #2#1#3}{ and~#2#1#3}
\Crefmultiformat{section}{Sections~#2#1#3}{ and~#2#1#3}{, #2#1#3}{ and~#2#1#3}
\crefrangemultiformat{section}{\S\S#3#1#4--#5#2#6}{ and~#3#1#4--#5#2#6}{, #3#1#4--#5#2#6}{ and~#3#1#4--#5#2#6}
\Crefrangemultiformat{section}{Sections~#3#1#4--#5#2#6}{ and~#3#1#4--#5#2#6}{, #3#1#4--#5#2#6}{ and~#3#1#4--#5#2#6}

% Display format for appendices
\crefformat{appendix}{Appendix~#2#1#3}
\Crefformat{appendix}{Appendix~#2#1#3}
\crefrangeformat{appendix}{Appendices~#3#1#4--#5#2#6}
\Crefrangeformat{appendix}{Appendices~#3#1#4--#5#2#6}
\crefmultiformat{appendix}{Appendices~#2#1#3}{ and~#2#1#3}{, #2#1#3}{ and~#2#1#3}
\Crefmultiformat{appendix}{Appendices~#2#1#3}{ and~#2#1#3}{, #2#1#3}{ and~#2#1#3}
\crefrangemultiformat{appendix}{Appendices~#3#1#4--#5#2#6}{ and~#3#1#4--#5#2#6}{, #3#1#4--#5#2#6}{ and~#3#1#4--#5#2#6}
\Crefrangemultiformat{appendix}{Appendices~#3#1#4--#5#2#6}{ and~#3#1#4--#5#2#6}{, #3#1#4--#5#2#6}{ and~#3#1#4--#5#2#6}

\crefname{part}{Part}{Parts}

\crefformat{paragraph}{\S#2#1#3}
\Crefformat{paragraph}{Paragraph~#2#1#3}
\crefrangeformat{paragraph}{\S\S#3#1#4--#5#2#6}
\Crefrangeformat{paragraph}{Paragraphs~#3#1#4--#5#2#6}
\crefmultiformat{paragraph}{\S\S#2#1#3}{ and~#2#1#3}{, #2#1#3}{ and~#2#1#3}
\Crefmultiformat{paragraph}{Paragraphs~#2#1#3}{ and~#2#1#3}{, #2#1#3}{ and~#2#1#3}
\crefrangemultiformat{paragraph}{\S\S#3#1#4--#5#2#6}{ and~#3#1#4--#5#2#6}{, #3#1#4--#5#2#6}{ and~#3#1#4--#5#2#6}
\Crefrangemultiformat{paragraph}{Paragraphs~#3#1#4--#5#2#6}{ and~#3#1#4--#5#2#6}{, #3#1#4--#5#2#6}{ and~#3#1#4--#5#2#6}

% Number subsubsections
\setcounter{secnumdepth}{5}

% Display format for figures
\crefname{figure}{Figure}{Figures}

% Use cleveref instead of hyperref's \autoref
\let\autoref\cref


%%%% EQUATION NUMBERING %%%%

% The following hack uses the single theorem counter to number
% equations as well, so that we don't have both Theorem 1.1 and
% equation (1.1).
\let\c@equation\c@thm
\numberwithin{equation}{section}


%%%% ENUMERATE NUMBERING %%%%

% Number the first level of enumerates as (i), (ii), ...
\renewcommand{\theenumi}{(\roman{enumi})}
\renewcommand{\labelenumi}{\theenumi}


%%%% MARGINS %%%%

% This is a matter of personal preference, but I think the left
% margins on enumerates and itemizes are too wide.
\setitemize[1]{leftmargin=2em}
\setenumerate[1]{leftmargin=*}

% Likewise that they are too spaced out.
\setitemize[1]{itemsep=-0.2em}
\setenumerate[1]{itemsep=-0.2em}

%%% Notes %%%
\def\noteson{%
\gdef\note##1{\mbox{}\marginpar{\color{blue}\textasteriskcentered\ ##1}}}
\gdef\notesoff{\gdef\note##1{\null}}
\noteson

\newcommand{\Coq}{\textsc{Coq}\xspace}
\newcommand{\Agda}{\textsc{Agda}\xspace}
\newcommand{\NuPRL}{\textsc{NuPRL}\xspace}

%%%% CITATIONS %%%%

% \let \cite \citep

%%%% INDEX %%%%

\newcommand{\footstyle}[1]{{\hyperpage{#1}}n} % If you index something that is in a footnote
\newcommand{\defstyle}[1]{\textbf{\hyperpage{#1}}}  % Style for pageref to a definition

\newcommand{\indexdef}[1]{\index{#1|defstyle}}   % Index a definition
\newcommand{\indexfoot}[1]{\index{#1|footstyle}} % Index a term in a footnote
\newcommand{\indexsee}[2]{\index{#1|see{#2}}}    % Index "see also"


%%%% Standard phrasing or spelling of common phrases %%%%

\newcommand{\ZF}{Zermelo--Fraenkel}
\newcommand{\CZF}{Constructive \ZF{} Set Theory}

\newcommand{\LEM}[1]{\ensuremath{\mathsf{LEM}_{#1}}\xspace}
\newcommand{\choice}[1]{\ensuremath{\mathsf{AC}_{#1}}\xspace}

%%%% MISC %%%%

\newcommand{\mentalpause}{\medskip} % Use for "mental" pause, instead of \smallskip or \medskip

%% Use \symlabel instead of \label to mark a pageref that you need in the index of symbols
\newcounter{symindex}
\newcommand{\symlabel}[1]{\refstepcounter{symindex}\label{#1}}

% Local Variables:
% mode: latex
% TeX-master: "hott-online"
% End:


\newcommand{\idsymbin}{=}

%%%%%%%%%%%%%%%%%%%%%%%%%%%%%%%%%%%%%%%%%%%%%%%%%%%%%%%%%%%%%%%%%%%%%%%%%%%%%%%%
%%%% Our commands which are not part of the macros.tex file.
%%%% We should keep these commands separate, because we will
%%%% update the macros.tex following the updates of the book.

%%%% First we redefine the \id, \eqv and \ct commands so that they accept an
%%%% arbitrary number of arguments. This is useful when writing longer strings
%%%% of equalities or equivalences.

\makeatletter

\renewcommand{\id}[3][]{
  \@ifnextchar\bgroup
    {#2 \mathbin{\idsym_{#1}} \id[#1]{#3}}
    {#2 \mathbin{\idsym_{#1}} #3}
  }

\renewcommand{\eqv}[2]{
  \@ifnextchar\bgroup
    {#1 \eqvsym \eqv{#2}}
    {#1 \eqvsym #2}
  }

\newcommand{\ctsym}{%
  \mathchoice{\mathbin{\raisebox{0.5ex}{$\displaystyle\centerdot$}}}%
             {\mathbin{\raisebox{0.5ex}{$\centerdot$}}}%
             {\mathbin{\raisebox{0.25ex}{$\scriptstyle\,\centerdot\,$}}}%
             {\mathbin{\raisebox{0.1ex}{$\scriptscriptstyle\,\centerdot\,$}}}
  }

\renewcommand{\ct}[3][]{
  \@ifnextchar\bgroup
    {#2 \mathbin{\ctsym_{#1}} \ct[#1]{#3}}
    {#2 \mathbin{\ctsym_{#1}} #3}
  }

\makeatother

%%%% We always use textstyle products and sums...
%\renewcommand{\prd}{\tprd}
%\renewcommand{\sm}{\tsm}
\makeatletter
\renewcommand{\@dprd}{\@tprd}
\renewcommand{\@dsm}{\@tsm}
\renewcommand{\@dprd@noparens}{\@tprd}
\renewcommand{\@dsm@noparens}{\@tsm}

%%%% ...with a bit more spacing
\renewcommand{\@tprd}[1]{\mathchoice{{\textstyle\prod_{(#1)}\,}}{\prod_{(#1)}\,}{\prod_{(#1)}\,}{\prod_{(#1)}\,}}
\renewcommand{\@tsm}[1]{\mathchoice{{\textstyle\sum_{(#1)}\,}}{\sum_{(#1)}\,}{\sum_{(#1)}\,}{\sum_{(#1)}\,}}

%%%%%%%%%%%%%%%%%%%%%%%%%%%%%%%%%%%%%%%%%%%%%%%%%%%%%%%%%%%%%%%%%%%%%%%%%%%%%%%%
%%%% We adjust the \prd command so that implicit arguments become possible.
%%%%
%%%% First, we have the following switch. Set it to true if implicit arguments
%%%% are desired, or to false if not. Note turning off implicit arguments
%%%% might render some parts of the text harder to comprehend, since in the
%%%% text might appear $f(x)$ where we would have $f(i,x)$ without implicit
%%%% arguments.

\newcommand{\implicitargumentson}{\boolean{true}}

%%%% If one wants to use implicit arguments in the notation for product types,
%%%% a * has to be put before the argument that has to be implicit.
%%%% For example: in $\prd{x:A}*{y:B}{u:P(y)}Q(x,y,u)$, the argument y is
%%%% implicit. Any of the arguments can be made implicit this way.

%%%% First of all, we should make the command \prd search not only for a
%%%% brace, but also for a star. We introduce an auxiliary command that
%%%% determines whether the next character is a star or brace.
\newcommand{\@ifnextchar@starorbrace}[2]
%  {\@ifnextcharamong{#1}{#2}{*}{\bgroup};}
  {\@ifnextchar*{#1}{\@ifnextchar\bgroup{#1}{#2}}}
  
%%%% When encountering the \prd command, latex should determine whether it
%%%% should print implicit argument brackets or not. So the first branching
%%%% happens right here.
\renewcommand{\prd}{\@ifnextchar*{\@iprd}{\@prd}}

\newcommand{\@prd}[1]
  {\@ifnextchar@starorbrace
    {\prd@parens{#1}}
    {\@ifnextchar\sm{\prd@parens{#1}\@eatsm}{\prd@noparens{#1}}}}
\newcommand{\@prd@parens}{\@ifnextchar*{\@iprd}{\prd@parens}}
\renewcommand{\prd@parens}[1]
  {\@ifnextchar@starorbrace
    {\@theprd{#1}\@prd@parens}
    {\@ifnextchar\sm{\@theprd{#1}\@eatsm}{\@theprd{#1}}}}
\newcommand{\@theprd}[1]
  {\mathchoice{\@dprd{#1}}{\@tprd{#1}}{\@tprd{#1}}{\@tprd{#1}}}
\renewcommand{\dprd}[1]{\@dprd{#1}\@ifnextchar@starorbrace{\dprd}{}}
\renewcommand{\tprd}[1]{\@tprd{#1}\@ifnextchar@starorbrace{\tprd}{}}

%%%% Here we tell the actual symbols to be printed.
\newcommand{\@theiprd}[1]{\mathchoice{\@diprd{#1}}{\@tiprd{#1}}{\@tiprd{#1}}{\@tiprd{#1}}}
\newcommand{\@iprd}[2]{\@ifnextchar@starorbrace%
  {\@theiprd{#2}\@prd@parens}%
  {\@ifnextchar\sm%
    {\@theiprd{#2}\@eatsm}%
    {\@theiprd{#2}}}}
\def\@tiprd#1{
  \ifthenelse{\implicitargumentson}
    {\@@tiprd{#1}\@ifnextchar\bgroup{\@tiprd}{}}
    {\@tprd{#1}}}
\def\@@tiprd#1{\mathchoice{{\textstyle\prod_{\{#1\}}\,}}{\prod_{\{#1\}}\,}{\prod_{\{#1\}}\,}{\prod_{\{#1\}}\,}}
\def\@diprd{
  \ifthenelse{\implicitargumentson}
    {\@tiprd}
    {\@tprd}}
    

%%%% And finally we need to redefine \@eatprd so that implicit arguments also
%%%% works in the scope of a dependent sum.    
\def\@eatprd\prd{\@prd@parens}

\makeatother

%%%%%%%%%%%%%%%%%%%%%%%%%%%%%%%%%%%%%%%%%%%%%%%%%%%%%%%%%%%%%%%%%%%%%%%%%%%%%%%%
%%%% Redefining the quantifiers, so that some of the longer 
%%%% formulas appear one a single line without problems

%%% Dependent products written with \forall, in the same style
\makeatletter
\def\tfall#1{\forall_{(#1)}\@ifnextchar\bgroup{\,\tfall}{\,}}
\renewcommand{\fall}{\tfall}

%%% Existential quantifier %%%
\def\texis#1{\exists_{(#1)}\@ifnextchar\bgroup{\,\texis}{\,}}
\renewcommand{\exis}{\texis}

%%% Unique existence %%%
\def\uexis#1{\exists!_{(#1)}\@ifnextchar\bgroup{\,\uexis}{\,}}
\makeatother

%%%%%%%%%%%%%%%%%%%%%%%%%%%%%%%%%%%%%%%%%%%%%%%%%%%%%%%%%%%%%%%%%%%%%%%%%%%%%%%%
%%%% UNFOLD
%%%%
%%%% For each definition in the type theory we make two versions of the macro:
%%%% the macro introducing the new notation and an @unfold version of the macro
%%%% which outputs the meaning of that new notation. Thus, we can use the
%%%% following construction to write our text. When we introduce \macro, we can
%%%% write \unfold{\macro} and the output will be the result of \macro@unfold.

\makeatletter
\newcommand{\unfold}{%
  \unfoldnext}
\newcommand{\unfoldall}[1]{%
  \begingroup%
  \renewcommand{\jhom}{\jhom@unfold}%
  \renewcommand{\jhomeq}{\jhomeq@unfold}%
  \renewcommand{\jhomdefn}{\jhomdefn@unfold}%
  \renewcommand{\jfhom}{\jfhom@unfold}%
  \renewcommand{\jcomp}{\jcomp@unfold}%
  \renewcommand{\@jcomp@nested}{\@jcomp@unfold@nested}%
  \renewcommand{\@jcomp@parens}{\@jcomp@unfold@parens}%
  \renewcommand{\tmext}{\tmext@unfold}%
  \renewcommand{\@tmext@nested}{\@tmext@unfold@nested}%
  \renewcommand{\@tmext@parens}{\@tmext@unfold@parens}%
  \renewcommand{\cprojfstf}{\cprojfstf@unfold}%
  \renewcommand{\cprojfst}{\cprojfst@unfold}%
  \renewcommand{\cprojsndf}{\cprojsndf@unfold}%
  \renewcommand{\cprojsnd}{\cprojsnd@unfold}%
  \renewcommand{\jfcomp}{\jfcomp@unfold}%
%  \renewcommand{\@jfcomp@nested}{\@jfcomp@unfold@nested}%
%  \renewcommand{\@jfcomp@parens}{\@jfcomp@unfold@parens}%
  \renewcommand{\sandwich}{\sandwich@unfold}%
  \renewcommand{\finc}{\finc@unfold}%
  \renewcommand{\jvcomp}{\jvcomp@unfold}%
  \renewcommand{\subst@type@unfold}[1]{
    \@ifnextchar\cprojfstf{\@eatdo{\cprojfstf@parens}}{%
      ##1}
    }
  #1%
  \endgroup%
  }

%%%% The following command is useful when you have checked with '\@ifnextchar'
%%%% that the next character is a macro '\firstmacro' and you want to replace
%%%% it by '\secondmacro'. To establish this, simply call for
%%%% '\@ifnextchar\firstmacro{\@eatdo{\secondmacro}}{}' with the second 
%%%% argument of \@eatdo left unspecified.
\newcommand{\@eatdo}[2]{#1}

%%%% The intention of '\unfoldnext' is to unfold only the definition of the
%%%% next character, provided that it is in the list of unfoldable macros.
\newcommand{\unfoldnext}[1]{
  \@ifnextchar\jhom{\@eatdo{\jhom@unfold}}{%
  \@ifnextchar\jhomeq{\@eatdo{\jhomeq@unfold}}{%
  \@ifnextchar\jhomdefn{\@eatdo{\jhomdefn@unfold}}{%
  \@ifnextchar\jfhom{\@eatdo{\jfhom@unfold}}{%
  \@ifnextchar\jcomp{\@eatdo{\jcomp@unfold}}{%
  \@ifnextchar\@jcomp@nested{\@eatdo{\@jcomp@unfold@nested}}{%
  \@ifnextchar\@jcomp@parens{\@eatdo{\@jcomp@unfold@parens}}{%
  \@ifnextchar\tmext{\@eatdo{\tmext@unfold}}{%
  \@ifnextchar\@tmext@nested{\@eatdo{\@tmext@unfold@nested}}{%
  \@ifnextchar\@tmext@parens{\@eatdo{\@tmext@unfold@parens}}{%
  \@ifnextchar\cprojfstf{\@eatdo{\cprojfstf@unfold}}{%
  \@ifnextchar\cprojfst{\@eatdo{\cprojfst@unfold}}{%
  \@ifnextchar\cprojsndf{\@eatdo{\cprojsndf@unfold}}{%
  \@ifnextchar\cprojsnd{\@eatdo{\cprojsnd@unfold}}{%
  \@ifnextchar\jfcomp{\@eatdo{\jfcomp@unfold}}{%
%  \@ifnextchar\@jfcomp@nested{\@eatdo{\@jfcomp@unfold@nested}}{%
%  \@ifnextchar\@jfcomp@parens{\@eatdo{\@jfcomp@unfold@parens}}{%
  \@ifnextchar\sandwich{\@eatdo{\sandwich@unfold}}{%
  \@ifnextchar\finc{\@eatdo{\finc@unfold}}{%
  \@ifnextchar\jvcomp{\@eatdo{\jvcomp@unfold}}}
  #1}
\makeatother

%%%%%%%%%%%%%%%%%%%%%%%%%%%%%%%%%%%%%%%%%%%%%%%%%%%%%%%%%%%%%%%%%%%%%%%%%%%%%%%%
%%%% A PRETTY PRINTER
%%%%
%%%% We write a \pretty command that pretty prints judgments or types by
%%%% diplaying variables and omitting explicit notation for weakening.
%%%%
%%%% This command should work similar to the \unfold command
%%%%
%%%% -- UNDER CONSTRUCTION

\makeatletter
\newcommand{\vardis}[2]{\@vardis@type #2{}(\@vardis@term #1)}
\newcommand{\@vardis}{\@ifnextchar\bgroup{\@@vardis}{}}
\newcommand{\@@vardis}[1]{\@ifnextchar\bgroup{\vardis{#1}}{#1}}
\newcommand{\@vardis@term}{\@vardis}
\newcommand{\@vardis@type}{\@ifnextchar\ctxext{\@ctxext@nested}{\@ifnextchar\ctxwk{\@ctxwk@nested}{\@vardis}}}
\newcommand{\@vardis@nested}[3]{\@vardis@parens{#2}{#3}}
\newcommand{\@vardis@parens}[2]{(\vardis{#1}{#2})}
\makeatother

\makeatletter
\newcommand{\jvctx}{\jctx}
\newcommand{\jvctxeq}{\jctxeq}

\newcommand{\cctxextcombi}[2]{\@ifnextchar\bgroup{\@cctxextcombi #1}{#1:}#2}
\newcommand{\@cctxextcombi}[4]{\cctxext{{\cctxextcombi{#1}{#3}}{\@@cctxextcombi{#1}{#2}{#4}}}}
\newcommand{\@@cctxextcombi}[3]{\@ifnextchar\bgroup{\@@@ctxextcombi #2}{#2(#1):}#3(\cctxext{#1})}
\newcommand{\@@@ctxextcombi}[8] % the 5th argument is (, the 6th is \cctxext and the 8th is ).
  {\@@ctxextcombi{#7}{#1}{#3},\@@ctxextcombi{{#7}{#3}}{#2}{#4}}
\newcommand{\cctxext}[1]{\@ifnextchar\bgroup{\@cctxext}{}#1}
\newcommand{\@cctxext}[2]{\cctxext{#1},\cctxext{#2}}

\newcommand{\jvfamcombi}[3]{
  \cctxextcombi{#1}{#2} \vdash \vardis{\cctxext{#1}}{#3}
}

\newcommand{\jvfam}{\@ifnextchar*{\@jvfamAlignTrue}{\@jvfamAlignFalse}}
\newcommand{\@jvfamAlignTrue}[4]{\jfam*{#2:#3}{\vardis{#2}{#4}}}
\newcommand{\@jvfamAlignFalse}[3]{\jfam{#1:#2}{\vardis{#1}{#3}}\quad@test}

\newcommand{\jvfameq}{\@ifnextchar*{\@jvfameqAlignTrue}{\@jvfameqAlignFalse}}
\newcommand{\@jvfameqAlignTrue}[5]{\jfameq*{#2:#3}{\vardis{#2}{#4}}{\vardis{#2}{#5}}}
\newcommand{\@jvfameqAlignFalse}[4]{\jfameq{#1:#2}{\vardis{#1}{#3}}{\vardis{#1}{#4}}\quad@test}

\newcommand{\jvtype}{\@ifnextchar*{\@jvtypeAlignTrue}{\@jvtypeAlignFalse}}
\newcommand{\@jvtypeAlignTrue}[4]{\jtype*{#2:#3}{\vardis{#2}{#4}}}
\newcommand{\@jvtypeAlignFalse}[3]{\jtype{#1:#2}{\vardis{#1}{#3}}\quad@test}

\newcommand{\jvtypeeq}{\@ifnextchar*{\@jvtypeeqAlignTrue}{\@jvtypeeqAlignFalse}}
\newcommand{\@jvtypeeqAlignTrue}[5]{\jtypeeq*{#2:#3}{\vardis{#2}{#4}}{\vardis{#2}{#5}}}
\newcommand{\@jvtypeeqAlignFalse}[4]{\jtypeeq{#1:#2}{\vardis{#1}{#3}}{\vardis{#1}{#4}}\quad@test}

\newcommand{\jvterm}{\@ifnextchar*{\@jvtermAlignTrue}{\@jvtermAlignFalse}}
\newcommand{\@jvtermAlignTrue}[5]{\jterm*{#2:#3}{\vardis{#2}{#4}}{\vardis{#2}{#5}}}
\newcommand{\@jvtermAlignFalse}[4]{\jterm{#1:#2}{\vardis{#1}{#3}}{\vardis{#1}{#4}}\quad@test}

\newcommand{\jvtermeq}{\@ifnextchar*{\@jvtermeqAlignTrue}{\@jvtermeqAlignFalse}}
\newcommand{\@jvtermeqAlignTrue}[6]{\jtermeq*{#2:#3}{\vardis{#2}{#4}}{\vardis{#2}{#5}}{\vardis{#2}{#6}}}
\newcommand{\@jvtermeqAlignFalse}[5]{\jtermeq{#1:#2}{\vardis{#1}{#3}}{\vardis{#1}{#4}}{\vardis{#1}{#5}}\quad@test}
\makeatother

%%%%%%%%%%%%%%%%%%%%%%%%%%%%%%%%%%%%%%%%%%%%%%%%%%%%%%%%%%%%%%%%%%%%%%%%%%%%%%%%
%%%%

\newcommand{\famsym}{\mathcal{F}}
\newcommand{\tmsym}{\mathcal{T}}

%%%%%%%%%%%%%%%%%%%%%%%%%%%%%%%%%%%%%%%%%%%%%%%%%%%%%%%%%%%%%%%%%%%%%%%%%%%%%%%%
%%%% JUDGMENTS
%%%%
%%%% Below we define several commands for the judgments of type theory. There
%%%% are commands
%%%% * \jctx for the judgment that something is a context.
%%%% * \jctxeq for the judgment that two contexts are the same
%%%% * \jtype for the judgment that something is a type in a context
%%%% * \jtypeeq for the judgment that two types in the same context are the same
%%%% * \jterm for the judgment that something is a term of a type in a context
%%%% * \jtermeq for the judgment that two terms of the same type are the same

\makeatletter
% We first make a generic judgment command
\newcommand{\judgment}{\@ifnextchar*{\@judgmentAT}{\@judgmentAF}}
\newcommand{\@judgmentAT}[8]{\@judgment@ctx{#2} & \vdash \@judgment@rel{#3}{#4}{#5}{#6}{#7} #8}
\newcommand{\@judgmentAF}[7]{\@judgment@ctx{#1} \vdash \@judgment@rel{#2}{#3}{#4}{#5}{#6} #7\quad@test}
\newcommand{\@judgment@ctx}{\@judgment@ext}
\newcommand{\@judgment@rel}[5]{
  { \default@ctxext #1
    }
  #2 
  { \default@ctxext #3
    }
  #4
  { \default@ctxext #5
    }}
\newcommand{\@judgment@kind}[1]{~~\textit{#1}}
\newcommand{\@judgment@ext}[1]{\default@ctxext #1}

\newcommand{\quad@test}{%
  \@ifnextchar\jctx{\quad}{%
  \@ifnextchar\jctxeq{\quad}{%
  \@ifnextchar\jvctx{\quad}{%
  \@ifnextchar\jvctxeq{\quad}{%
  \@ifnextchar\jfam{\quad}{%
  \@ifnextchar\jfameq{\quad}{%
  \@ifnextchar\jvfam{\quad}{%
  \@ifnextchar\jvfameq{\quad}{%
  \@ifnextchar\jtype{\quad}{%
  \@ifnextchar\jtypeeq{\quad}{%
  \@ifnextchar\jvtype{\quad}{%
  \@ifnextchar\jvtypeeq{\quad}{%
  \@ifnextchar\jterm{\quad}{%
  \@ifnextchar\jtermeq{\quad}{%
  \@ifnextchar\jvterm{\quad}{%
  \@ifnextchar\jvtermeq{\quad}{%
  \@ifnextchar\jhom{\quad}{%
  \@ifnextchar\jhomeq{\quad}{%
  \@ifnextchar\jfhom{\quad}{%
  \@ifnextchar\jfhomeq{\quad}{%
  }}}}}}}}}}}}}}}}}}}}}

%%%% Judgments about contexts
\newcommand{\jctx@sym}{\@judgment@kind{ctx}}

\newcommand{\jctx}{\@ifnextchar*{\@jctxAlignTrue}{\@jctxAlignFalse}}
\newcommand{\@jctxAlignTrue}[2]{\judgment*{}{}{}{}{}{#2}{\jctx@sym}}
\newcommand{\@jctxAlignFalse}[1]{\judgment{}{}{}{}{}{#1}{\jctx@sym}}

\newcommand{\jctxeq}{\@ifnextchar*{\@jctxeqAlignTrue}{\@jctxeqAlignFalse}}
\newcommand{\@jctxeqAlignTrue}[3]{\judgment*{}{#2}{\jdeq}{#3}{}{}{\jctx@sym}}
\newcommand{\@jctxeqAlignFalse}[2]{\judgment{}{#1}{\jdeq}{#2}{}{}{\jctx@sym}}

\newcommand{\jctxdefn}{\@ifnextchar*{\@jctxdefnAlignTrue}{\@jctxdefnAlignFalse}}
\newcommand{\@jctxdefnAlignTrue}[3]{\judgment*{}{#2}{\defeq}{#3}{}{}{\jctx@sym}}
\newcommand{\@jctxdefnAlignFalse}[2]{\judgment{}{#1}{\defeq}{#2}{}{}{\jctx@sym}}

%%%% Judgments about families
\newcommand{\jfam@sym}{\@judgment@kind{fam}}

\newcommand{\jfam}{\@ifnextchar*{\@jfamAlignTrue}{\@jfamAlignFalse}}
\newcommand{\@jfamAlignTrue}[3]{\judgment*{#2}{}{}{}{}{#3}{\jfam@sym}}
\newcommand{\@jfamAlignFalse}[2]{\judgment{#1}{}{}{}{}{#2}{\jfam@sym}}

\newcommand{\jfameq}{\@ifnextchar*{\@jfameqAlignTrue}{\@jfameqAlignFalse}}
\newcommand{\@jfameqAlignTrue}[4]{\judgment*{#2}{#3}{\jdeq}{#4}{}{}{\jfam@sym}}
\newcommand{\@jfameqAlignFalse}[3]{\judgment{#1}{#2}{\jdeq}{#3}{}{}{\jfam@sym}}

\newcommand{\jfamdefn}{\@ifnextchar*{\@jfamdefnAlignTrue}{\@jfamdefnAlignFalse}}
\newcommand{\@jfamdefnAlignTrue}[4]{\judgment*{#2}{#3}{\defeq}{#4}{}{}{\jfam@sym}}
\newcommand{\@jfamdefnAlignFalse}[3]{\judgment{#1}{#2}{\defeq}{#3}{}{}{\jfam@sym}}
  
%%%% Judgments about types
\newcommand{\jtype@sym}{\@judgment@kind{type}}
\newcommand{\jtype}{\@ifnextchar*{\@jtypeAlignTrue}{\@jtypeAlignFalse}}
\newcommand{\@jtypeAlignTrue}[3]{\judgment*{#2}{}{}{}{}{#3}{\jtype@sym}}
\newcommand{\@jtypeAlignFalse}[2]{\judgment{#1}{}{}{}{}{#2}{\jtype@sym}}
  
\newcommand{\jtypeeq}{\@ifnextchar*{\@jtypeeqAlignTrue}{\@jtypeeqAlignFalse}}
\newcommand{\@jtypeeqAlignTrue}[4]{\judgment*{#2}{#3}{\jdeq}{#4}{}{}{\jtype@sym}}
\newcommand{\@jtypeeqAlignFalse}[3]{\judgment{#1}{#2}{\jdeq}{#3}{}{}{\jtype@sym}}

\newcommand{\jtypedefn}{\@ifnextchar*{\@jtypedefnAlignTrue}{\@jtypedefnAlignFalse}}
\newcommand{\@jtypedefnAlignTrue}[4]{\judgment*{#2}{#3}{\defeq}{#4}{}{}{\jtype@sym}}
\newcommand{\@jtypedefnAlignFalse}[3]{\judgment{#1}{#2}{\defeq}{#3}{}{}{\jtype@sym}}
  
%%%% Judgments about terms
\newcommand{\jterm}{\@ifnextchar*{\@jtermAlignTrue}{\@jtermAlignFalse}}
\newcommand{\@jtermAlignTrue}[4]{\judgment*{#2}{}{}{#4}{:}{#3}{}}
\newcommand{\@jtermAlignFalse}[3]{\judgment{#1}{}{}{#3}{:}{#2}{}}

\newcommand{\jtermeq}{\@ifnextchar*{\@jtermeqAlignTrue}{\@jtermeqAlignFalse}}
\newcommand{\@jtermeqAlignTrue}[5]{\judgment*{#2}{#4}{\jdeq}{#5}{:}{#3}{}}
\newcommand{\@jtermeqAlignFalse}[4]{\judgment{#1}{#3}{\jdeq}{#4}{:}{#2}{}}

\newcommand{\jtermdefn}{\@ifnextchar*{\@jtermdefnAlignTrue}{\@jtermdefnAlignFalse}}
\newcommand{\@jtermdefnAlignTrue}[5]{\judgment*{#2}{#4}{\defeq}{#5}{:}{#3}{}}
\newcommand{\@jtermdefnAlignFalse}[4]{\judgment{#1}{#3}{\defeq}{#4}{:}{#2}{}}
\makeatother

%%%%%%%%%%%%%%%%%%%%%%%%%%%%%%%%%%%%%%%%%%%%%%%%%%%%%%%%%%%%%%%%%%%%%%%%%%%%%%%%
%%%% THE EMPTY CONTEXT

\newcommand{\emptysym}{[\;]}
\newcommand{\emptyc}{{\emptysym}}
\newcommand{\emptyf}[1][]{{\emptysym}_{#1}}
\newcommand{\emptytm}[1][]{\typefont{\#}_{#1}}

%%%%%%%%%%%%%%%%%%%%%%%%%%%%%%%%%%%%%%%%%%%%%%%%%%%%%%%%%%%%%%%%%%%%%%%%%%%%%%%%
%%%% CONTEXT EXTENSION 
%%%%
%%%% The context extension command.
%%%%
%%%% To get a feeling of how the command works, here are a few examples.
%%%% \ctxext{A}{B} will print A.B
%%%% \ctxext{{A}{B}}{C} will print (A.B).C
%%%% \ctxext{{{A}{B}}{C}}{{D}{E}} will print ((A.B).C).(D.E)

\makeatletter
\newcommand{\ctxext}[2]{\@ctxext@ctx #1.\@ctxext@type #2}
\newcommand{\@ctxext}{\@ifnextchar\bgroup{\@@ctxext}{}}
\newcommand{\@ctxext@ctx}{%
  \@ifnextchar\ctxext{\@ctxext@nested}{%
  \@ifnextchar\ctxwk{\@ctxwk@nested}{%
  \@ifnextchar\jcomp{\@jcomp@nested}{%
  \@ifnextchar\jvcomp{\@jvcomp@nested}{%
  \@ifnextchar\jfcomp{\@jfcomp@nested}{%
  \@ctxext}}}}}}
\newcommand{\@ctxext@type}{%
  \@ifnextchar\ctxext{\@ctxext@nested}{%
  \@ifnextchar\subst{\@subst@nested}{%
  \@ifnextchar\jcomp{\@jcomp@nested}{%
  \@ifnextchar\jvcomp{\@jvcomp@nested}{%
  \@ifnextchar\jfcomp{\@jfcomp@nested}{%
  \@ctxext}}}}}}
\newcommand{\@@ctxext}[1]{\@ifnextchar\bgroup{\@ctxext@parens{#1}}{#1}}
\newcommand{\@ctxext@parens}[2]{(\ctxext{#1}{#2})}
\newcommand{\@ctxext@nested}[3]{\@ctxext@parens{#2}{#3}}

%%%% We want that some commands accept binary trees as arguments that default
%%%% into extensions. We make the following command to realize this
\newcommand{\default@ctxext}{\@ifnextchar\bgroup{\ctxext}{}}
\newcommand{\default@ctxext@parens}{\@ifnextchar\bgroup{\@ctxext@parens}{}}
\makeatother

%%%%%%%%%%%%%%%%%%%%%%%%%%%%%%%%%%%%%%%%%%%%%%%%%%%%%%%%%%%%%%%%%%%%%%%%%%%%%%%%
%%%% SUBSTITUTION

%%%% The substitution command will act the following way
%%%%
%%%% \subst{x}{P} will print P[x]
%%%% \subst{x}{{f}{Q}} will print Q[f][x]
%%%% \subst{{x}{f}}{{x}{Q}} will print Q[x][f[x]]

\makeatletter
\newcommand{\subst}[3][]{%
  \@subst@type #3{}[\@subst@term #2]^{#1}}
\newcommand{\@subst}{%
  \@ifnextchar\bgroup{\@@subst}{}}
\newcommand{\@@subst}[1]{%
  \@ifnextchar\bgroup{\subst{#1}}{#1}}
\newcommand{\@subst@term}{%
  \@subst}
\newcommand{\@subst@type}{%
  \@ifnextchar\ctxext{\@ctxext@nested}{%
  \@ifnextchar\ctxwk{\@ctxwk@nested}{%
  \@ifnextchar\jcomp{\@jcomp@nested}{%
  \@ifnextchar\tmext{\@tmext@nested}{%
  \@ifnextchar\jvcomp{\@jvcomp@nested}{%
  \@ifnextchar\jfcomp{\@jfcomp@nested}{%
%  \@ifnextchar\mfam{\@mfam@nested}{%
%  \@ifnextchar\mtm{\@mtm@nested}}
\newcommand{\subst@type@unfold}[1]{#1}
\newcommand{\@subst@nested}[3]{%
  \@subst@parens{#2}{#3}}
\newcommand{\@subst@parens}[2]{%
  (\subst{#1}{#2})}
\makeatother

%%%%%%%%%%%%%%%%%%%%%%%%%%%%%%%%%%%%%%%%%%%%%%%%%%%%%%%%%%%%%%%%%%%%%%%%%%%%%%%%
%%%% WEAKENING

%%%% The weakening command is very much like the substitution command.

\makeatletter
\newcommand{\ctxwk}[3][]{%
  \langle\@ctxwk@act #2\rangle^{#1} \@ctxwk@pass #3}
\newcommand{\@ctxwk}{%
  \@ifnextchar\bgroup{\@@ctxwk}{}}
\newcommand{\@@ctxwk}[1]{%
  \@ifnextchar\bgroup{\ctxwk{#1}}{#1}}
\newcommand{\@ctxwk@act}{%
  \@ctxwk}
\newcommand{\@ctxwk@pass}{%
  \@ifnextchar\ctxext{\@ctxext@nested}{%
  \@ifnextchar\subst{\@subst@nested}{%
  \@ifnextchar\jcomp{\@jcomp@nested}{%
  \@ifnextchar\tmext{\@tmext@nested}{%
  \@ifnextchar\jvcomp{\@jvcomp@nested}{%
  \@ifnextchar\jfcomp{\@jfcomp@nested}{%
%  \@ifnextchar\mfam{\@mfam@nested}{%
%  \@ifnextchar\mtm{\@mtm@nested}}
\newcommand{\@ctxwk@parens}[2]{%
  (\ctxwk{#1}{#2})}
\newcommand{\@ctxwk@nested}[3]{%
  \@ctxwk@parens{#2}{#3}}
\makeatother

%%%% Not sure if we're gonna need the following.
\newcommand{\ctxwkop}[2]{%
  \ctxwk{#2}{#1}}
  
%%%%%%%%%%%%%%%%%%%%%%%%%%%%%%%%%%%%%%%%%%%%%%%%%%%%%%%%%%%%%%%%%%%%%%%%%%%%%%%%
%%%% IDENTITY TERMS

\makeatletter
\newcommand{\idtm}[1]{\typefont{id}_{\default@ctxext #1}}
\makeatother

%%%%%%%%%%%%%%%%%%%%%%%%%%%%%%%%%%%%%%%%%%%%%%%%%%%%%%%%%%%%%%%%%%%%%%%%%%%%%%%%
%%%% TERM EXTENSION
%%%%
%%%% The term extension command \tmext is slightly complicated because 
%%%% \tmext@unfold should do different things depending on whether it has two
%%%% or four arguments. Thus \tmext has a full form and a short form, where
%%%% the short form has two arguments and the full form has four. 

\makeatletter

%%%% The basic term extension commands
\newcommand{\default@tmext}{\@ifnextchar\bgroup{\tmext}{}}
\newcommand{\tmext}[2]{%
  \@ifnextchar\bgroup{\tmext@full{#1}{#2}}{\tmext@short{#1}{#2}}}
\newcommand{\tmext@full}[4]{%
  \ctxext{\tmext@testleft #3}{\tmext@testright #4}}
\newcommand{\tmext@short}[2]{%
  \ctxext{\tmext@testleft #1}{\tmext@testright #2}}
\newcommand{\tmext@testleft}{%
  \@ifnextchar\bgroup{\@tmext@parens}{%
  \@ifnextchar\tmext{\@tmext@nested}{%
  \@ifnextchar\ctxwk{\@ctxwk@nested}{%
  \@ifnextchar\jcomp{\@jcomp@nested}{%
  \@ifnextchar\jvcomp{\@jvcomp@nested}{%
  \@ifnextchar\jfcomp{\@jfcomp@nested}{%
%  \default@tmext
  }}}}}}}
\newcommand{\tmext@testright}{%
  \@ifnextchar\bgroup{\@tmext@parens}{%
  \@ifnextchar\tmext{\@tmext@nested}{%
  \@ifnextchar\subst{\@subst@nested}{%
  \@ifnextchar\jcomp{\@jcomp@nested}{%
  \@ifnextchar\jvcomp{\@jvcomp@nested}{%
  \@ifnextchar\jfcomp{\@jfcomp@nested}{%
  \@ifnextchar\cprojfst{\cprojfst@nested}{%
  \@ifnextchar\cprojsnd{\cprojsnd@nested}{%
%  \default@tmext
  }}}}}}}}}
\newcommand{\@tmext@nested}[1]{%
  \@tmext@parens}
\newcommand{\@tmext@parens}[2]{%
  \@ifnextchar\bgroup
    {\tmext@full@parens{#1}{#2}}
    {(\tmext@short{#1}{#2})}}
\newcommand{\tmext@full@parens}[4]{%
  (\tmext@full{#1}{#2}{#3}{#4})}

%%%% The unfolded term extension commands
\newcommand{\tmext@unfold}[2]{%
  \@ifnextchar\bgroup{\tmext@unfold@full{#1}{#2}}{\tmext@short{#1}{#2}}}
\newcommand{\tmext@unfold@full}[4]{%  
  \subst{#4}{{#3}{\idtm{\ctxext{#1}{#2}}}}}
\newcommand{\@tmext@unfold@nested}[1]{%
  \@tmext@unfold@parens}
\newcommand{\@tmext@unfold@parens}[4]{%
  (\tmext@unfold{#1}{#2}{#3}{#4})}
\makeatother

%%%%%%%%%%%%%%%%%%%%%%%%%%%%%%%%%%%%%%%%%%%%%%%%%%%%%%%%%%%%%%%%%%%%%%%%%%%%%%%%
%%%% JUDGMENTAL MORPHISMS

\makeatletter

%%%% The judgment that f is a morphism from A to B in context \Gamma.
\newcommand{\jhomsym}[3][]{%
  ~~\textit{hom}_{#1}(\default@ctxext #2,\default@ctxext #3)}
\newcommand{\jhom}{%
  \@ifnextchar*{\@jhomAlignTrue}{\@jhomAlignFalse}}
\newcommand{\@jhomAlignTrue}[5]{%
  \judgment*{#2}{}{}{#5}{}{}{\jhomsym{#3}{#4}}}
\newcommand{\@jhomAlignFalse}[4]{%
  \judgment{#1}{}{}{#4}{}{}{\jhomsym{#2}{#3}}}
\newcommand{\jhomeq}{%
  \@ifnextchar*{\@jhomeqAlignTrue}{\@jhomeqAlignFalse}}
\newcommand{\@jhomeqAlignTrue}[6]{%
  \judgment*{#2}{#5}{\jdeq}{#6}{}{}{\jhomsym{#3}{#4}}}
\newcommand{\@jhomeqAlignFalse}[5]{%
  \judgment{#1}{#4}{\jdeq}{#5}{}{}{\jhomsym{#2}{#3}}}
\newcommand{\jhomdefn}{%
  \@ifnextchar*{\@jhomdefnAlignTrue}{\@jhomdefnAlignFalse}}
\newcommand{\@jhomdefnAlignTrue}[6]{%
  \judgment*{#2}{#5}{\defeq}{#6}{}{}{\jhomsym{#3}{#4}}}
\newcommand{\@jhomdefnAlignFalse}[5]{%
  \judgment{#1}{#4}{\defeq}{#5}{}{}{\jhomsym{#2}{#3}}}

\newcommand{\jhom@unfold}[4]{%
  \jterm
    {{#1}{#2}}
    {\ctxwk{\default@ctxext #2}{\default@ctxext@parens #3}}
    {#4}}
\newcommand{\jhomeq@unfold}[5]{%
  \jtermeq
    {{#1}{#2}}
    {\ctxwk{\default@ctxext #2}{\default@ctxext@parens #3}}
    {#4}
    {#5}}
\newcommand{\jhomdefn@unfold}[5]{%
  \jtermdefn
    {{#1}{#2}}
    {\ctxwk{\default@ctxext #2}{\default@ctxext@parens #3}}
    {#4}
    {#5}}

%%%% Composition of morphisms
\newcommand{\jcomp}[3]{%
  \jcomp@testleft #3 \circ \jcomp@testright #2}
\newcommand{\jcomp@testleft}{%
  \@ifnextchar\jcomp{\@jcomp@nested}{%
  \@ifnextchar\ctxwk{\@ctxwk@nested}{%
  \@ifnextchar\ctxext{\@ctxext@nested}{%
  \@ifnextchar\bgroup{\@jcomp@parens}{%
  \@ifnextchar\tmext{\@tmext@nested}{%
  \@ifnextchar\jvcomp{\@jvcomp@nested}{%
  \@ifnextchar\jfcomp{\@jfcomp@nested}{%
  }}}}}}}}
\newcommand{\jcomp@testright}{%
  \@ifnextchar\jcomp{\@jcomp@nested}{%
  \@ifnextchar\subst{\@subst@nested}{%
  \@ifnextchar\ctxext{\@ctxext@nested}{%
  \@ifnextchar\bgroup{\@jcomp@parens}{%
  \@ifnextchar\tmext{\@tmext@nested}{%
  \@ifnextchar\jvcomp{\@jvcomp@nested}{%
  \@ifnextchar\jfcomp{\@jfcomp@nested}{%
  }}}}}}}}
\newcommand{\@jcomp@nested}[4]{%
  \@jcomp@parens{#2}{#3}{#4}}
\newcommand{\@jcomp@parens}[3]{%
  (\jcomp{#1}{#2}{#3})}

\newcommand{\jcomp@unfold}[3]{%
  \subst
    {\jcomp@unfold@test@preside #2}
    {\ctxwk{\default@ctxext #1}{\jcomp@unfold@test@postside #3}}}
\newcommand{\jcomp@unfold@test@preside}{%
  \@ifnextchar\bgroup{\@jcomp@unfold@parens}{}}
\newcommand{\jcomp@unfold@test@postside}{%
  \@ifnextchar\bgroup{\@jcomp@unfold@parens}{%
  \@ifnextchar\subst{\@subst@nested}{%
  }}}
\newcommand{\@jcomp@unfold@nested}[4]{%
  \@jcomp@unfold@parens{#2}{#3}{#4}}
\newcommand{\@jcomp@unfold@parens}[3]{%
  (\jcomp@unfold{#1}{#2}{#3})}

%%%% Vertical composition of morphisms.
\newcommand{\jvcomp}[3]{%
  \jcomp@testleft #2 * \jcomp@testright #3}
\newcommand{\jvcomp@testleft}{%
  \@ifnextchar\jvcomp{\@jvcomp@nested}{%
  \@ifnextchar\ctxwk{\@ctxwk@nested}{%
  \@ifnextchar\ctxext{\@ctxext@nested}{%
  \@ifnextchar\bgroup{\@jvcomp@parens}{%
  \@ifnextchar\tmext{\@tmext@nested}{%
  \@ifnextchar\jcomp{\@jcomp@nested}{%
  \@ifnextchar\jfcomp{\@jfcomp@nested}{%
  }}}}}}}}
\newcommand{\jvcomp@testright}{%
  \@ifnextchar\jvcomp{\@jvcomp@nested}{%
  \@ifnextchar\subst{\@subst@nested}{%
  \@ifnextchar\ctxext{\@ctxext@nested}{%
  \@ifnextchar\bgroup{\@jvcomp@parens}{%
  \@ifnextchar\tmext{\@tmext@nested}{%
  \@ifnextchar\jcomp{\@jcomp@nested}{%
  \@ifnextchar\jfcomp{\@jfcomp@nested}{%
  }}}}}}}}
\newcommand{\@jvcomp@nested}[4]{%
  \@jvcomp@parens{#2}{#3}{#4}}
\newcommand{\@jvcomp@parens}[3]{%
  (\jvcomp{#1}{#2}{#3})}

\newcommand{\jvcomp@unfold}[3]{%
  \tmext{}{}{\ctxwk{#1}{#2}}{#3}
  }
\newcommand{\jvcomp@unfold@test@preside}{%
  \@ifnextchar\bgroup{\@jvcomp@unfold@parens}{}}
\newcommand{\jvcomp@unfold@test@postside}{%
  \@ifnextchar\bgroup{\@jvcomp@unfold@parens}{}}
\newcommand{\@jvcomp@unfold@nested}[4]{%
  \@jvcomp@unfold@parens{#2}{#3}{#4}}
\newcommand{\@jvcomp@unfold@parens}[3]{%
  (\jvcomp@unfold{#1}{#2}{#3})}

%%%% The judgment that F is a morphism from P to Q over f in context \Gamma.
\newcommand{\jfhomsym}[3]{\jhomsym[{#1}]{#2}{#3}}
\newcommand{\jfhom}{%
  \@ifnextchar*{\jfhomAlignTrue}{\jfhomAlignFalse}}
\newcommand{\jfhomAlignTrue}[8]{
  \judgment*{#2}{}{}{#8}{}{}{\jfhomsym{#5}{#6}{#7}}}
\newcommand{\jfhomAlignFalse}[7]{
  \judgment{#1}{}{}{#7}{}{}{\jfhomsym{#4}{#5}{#6}}}
\newcommand{\jfhomeq}[8]{%
  \judgment{#1}{#7}{\jdeq}{#8}{}{}{\jhomsym[{#4}]{#5}{#6}}}
\newcommand{\jfhomdefn}[8]{%
  \judgment{#1}{#7}{\defeq}{#8}{}{}{\jhomsym[{#4}]{#5}{#6}}}
\newcommand{\jfhom@unfold}[7]{%
  \jterm
    {{{#1}{#2}}{#5}}
    {\ctxwk{\default@ctxext #5}{\jcomp{#2}{#4}{#6}}}
    {#7}}
    
\newcommand{\jfcomp}[5]{%
  \jfcomp@testleft #5 \bullet \jfcomp@testright #4}
\newcommand{\jfcomp@testleft}{%
  \@ifnextchar\bgroup{\@jfcomp@parens}{%
  \@ifnextchar\jfcomp{\@jfcomp@nested}{%
  \@ifnextchar\jcomp{\@jcomp@nested}{%
  \@ifnextchar\ctxwk{\@ctxwk@nested}{%
  \@ifnextchar\tmext{\@tmext@nested}{%
  \@ifnextchar\jvcomp{\@jvcomp@nested}{%
  }}}}}}}
\newcommand{\jfcomp@testright}{%
  \@ifnextchar\bgroup{\@jfcomp@parens}{%
  \@ifnextchar\jfcomp{\@jfcomp@nested}{%
  \@ifnextchar\jcomp{\@jcomp@nested}{%
  \@ifnextchar\subst{\@subst@nested}{%
  \@ifnextchar\tmext{\@tmext@nested}{%
  \@ifnextchar\jvcomp{\@jvcomp@nested}{%
  }}}}}}}
\newcommand{\@jfcomp@nested}[1]{%
  \@jfcomp@parens}
\newcommand{\@jfcomp@parens}[5]{%
  (\jfcomp{#1}{#2}{#3}{#4}{#5})}
  
\newcommand{\jfcomp@unfold}[5]{%
  \jcomp{#3}{#4}{{#1}{#2}{#5}}}
\makeatother

%%%%%%%%%%%%%%%%%%%%%%%%%%%%%%%%%%%%%%%%%%%%%%%%%%%%%%%%%%%%%%%%%%%%%%%%%%%%%%%%
%%%% JUDGMENTAL TRIVIAL COFIBRATIONS

\newcommand{\jtcext}{\tilde}

%%%%%%%%%%%%%%%%%%%%%%%%%%%%%%%%%%%%%%%%%%%%%%%%%%%%%%%%%%%%%%%%%%%%%%%%%%%%%%%%
%%%% CONTEXT PROJECTIONS

\makeatletter
\newcommand{\cprojgenf}[3]{%
  \typefont{pr}^{%
    \@ifnextchar\bgroup{\@ctxext@parens}{%
    \@ifnextchar\ctxext{\@ctxext@nested}{%
    }}
    #2,
    \@ifnextchar\bgroup{\@ctxext@parens}{%
    \@ifnextchar\ctxext{\@ctxext@nested}{%
    }}
    #3
    }_{#1}}
\newcommand{\cprojgen}[4]{%
  \subst{#4}{\cprojgenf{#1}{#2}{#3}}}
\newcommand{\cprojgenf@nested}[1]{%
  \cprojgenf@parens}
\newcommand{\cprojgenf@parens}[3]{%
  (\cprojgenf{#1}{#2}{#3})}
\newcommand{\cprojgen@nested}[1]{%
  \cprojgen@parens}
\newcommand{\cprojgen@parens}[4]{%
  (\cprojgen{#1}{#2}{#3}{#4})}

\newcommand{\cprojfstf}[2]{%
  \cprojgenf{0}{#1}{#2}}
\newcommand{\cprojfstf@nested}[1]{%
  \cprojfstf@parens}
\newcommand{\cprojfstf@parens}[2]{%
  (\cprojfstf{#1}{#2})}
\newcommand{\cprojfstf@unfold}[2]{%
  \ctxwk{\default@ctxext #2}\idtm{\default@ctxext #1}}

\newcommand{\cprojfst}[3]{%
  \cprojgen{0}{#1}{#2}{#3}}
\newcommand{\cprojfst@nested}[1]{%
  \cprojfst@parens}
\newcommand{\cprojfst@parens}[3]{%
  (\cprojfst{#1}{#2}{#3})}
\newcommand{\cprojfst@unfold}[3]{%
  \subst{#3}{(\cprojfstf@unfold{#1}{#2})}}

\newcommand{\cprojsndf}[2]{%
  \cprojgenf{1}{#1}{#2}}
\newcommand{\cprojsndf@nested}[1]{%
  \cprojsndf@parens}
\newcommand{\cprojsndf@parens}[2]{%
  (\cprojsndf{#1}{#2})}
\newcommand{\cprojsndf@unfold}[2]{%
  \idtm{\default@ctxext #2}}

\newcommand{\cprojsnd}[3]{%
  \cprojgen{1}{#1}{#2}{#3}}
\newcommand{\cprojsnd@nested}[1]{%
  \cprojsnd@parens}
\newcommand{\cprojsnd@parens}[3]{%
  (\cprojsnd{#1}{#2}{#3})}
\newcommand{\cprojsnd@unfold}[3]{%
  \subst{#3}{\cprojsnd@unfold{#1}{#2}}}
  
%%%% The sandwich function
\newcommand{\sandwich}[3]{\typefont{sw}^{#1,#2,#3}}
\newcommand{\sandwich@unfold}[3]{\typefont{sw}^{#1,#2,#3}}
\makeatother

%%%%%%%%%%%%%%%%%%%%%%%%%%%%%%%%%%%%%%%%%%%%%%%%%%%%%%%%%%%%%%%%%%%%%%%%%%%%%%%%
%%%% FIBER INCLUSIONS

\makeatletter
\newcommand{\finc}[2]{\typefont{in}^{#2}_{#1}}
\newcommand{\finc@unfold}[2]{\tmext{}{}{\ctxwk{\subst{x}{P}}{x}}{\idtm{\subst{x}{P}}}}
\makeatother

%%%%%%%%%%%%%%%%%%%%%%%%%%%%%%%%%%%%%%%%%%%%%%%%%%%%%%%%%%%%%%%%%%%%%%%%%%%%%%%%
%%%% THE UNIT TYPE

\makeatletter
\newcommand{\unitc}[1]{%
  \unit^0_{\default@ctxext #1}}
\newcommand{\unitct}[1]{%
  \ttt^0_{\default@ctxext #1}}
\newcommand{\unitf}[2]{%
  \unit^1_{\default@ctxext #1,\default@ctxext #2}}
\newcommand{\unitft}[2]{%
  \ttt^1_{\default@ctxext #1,\default@ctxext #2}}
\makeatother

%%%%%%%%%%%%%%%%%%%%%%%%%%%%%%%%%%%%%%%%%%%%%%%%%%%%%%%%%%%%%%%%%%%%%%%%%%%%%%%%
%%%% DEPENDENT FUNCTION TYPES

\makeatletter
\newcommand{\sprd}[2]{\Pi(\default@ctxext #1,\default@ctxext #2)}
\begin{comment}
\newcommand{\@sprd@test@cod}[2]{%
  \@ifnextchar\bgroup{\@sprd@do@cod{#1}}{%
  \Pi(\@sprd@test@dom{#1}{#2} #1,
  }}
\newcommand{\@sprd@do@cod}[4]{%
  \ctxext{\@sprd{#1}{#2}}{\@sprd{#1}{#3}}
  }
\newcommand{\@sprd}[2]{
  \@ifnextchar\bgroup{\@@sprd}{%
    \Pi(}
    #1,{#2})
  }
\newcommand{\@@sprd}[5]{%
  \sprd{#1}{\sprd{#2}{#4}}
  }
\end{comment}

\newcommand{\slam}[3]{%
  \lambda^{{\default@ctxext@parens #1},{\default@ctxext@parens #2}}
  (\default@ctxext #3)
  }
\newcommand{\sev}[1]{\tfev(#1)}

\makeatother

%%%%%%%%%%%%%%%%%%%%%%%%%%%%%%%%%%%%%%%%%%%%%%%%%%%%%%%%%%%%%%%%%%%%%%%%%%%%%%%%
%%%% NON-DEPENDENT FUNCTION TYPES

\newcommand{\jfun}[2]{#1\to#2}

%%%%%%%%%%%%%%%%%%%%%%%%%%%%%%%%%%%%%%%%%%%%%%%%%%%%%%%%%%%%%%%%%%%%%%%%%%%%%%%%
%%%% THE CONSTRUCTORS OF THE TYPE THEORY OF MODELS

\makeatletter
%%%% The initial model
\newcommand{\mctx}{%
  \mathcal{C}}

%%%% The family constructor
\newcommand{\mfam}[2][]{%
  \mathcal{F}_{\default@ctxext #2}^{#1}}
\newcommand{\@mfam@nested}[1]{\@mfam@parens}
\newcommand{\@mfam@parens}[2][]{(\mfam[#1]{#2})}

%%%% The terms constructor
\newcommand{\mtm}[2][]{%
  \mathcal{T}_{\default@ctxext #2}^{#1}}
\newcommand{\@mtm@nested}[1]{\@mtm@parens}
\newcommand{\@mtm@parens}[2][]{(\mtm[#1]{#2})}

%%%% The empty type constructor
\newcommand{\tfemp}[1]{%
  \typefont{emp}_{\default@ctxext #1}}
\newcommand{\tft}[1]{%
  \typefont{t}_{\default@ctxext #1}}

%%%% The extension constructor
\newcommand{\tfext}[1]{%
  \typefont{ext}_{\default@ctxext #1}}

%%%% The substitution constructor
\newcommand{\tfsubst}[1]{%
  \typefont{subst}_{\default@ctxext #1}}
  
%%%% The weakening constructor
\newcommand{\tfwk}[1]{%
  \typefont{wk}_{\default@ctxext #1}}

%%%% The identity function constructor
\newcommand{\tfid}[1]{%
  \typefont{idtm}_{\default@ctxext #1}}
\makeatother

%%%%%%%%%%%%%%%%%%%%%%%%%%%%%%%%%%%%%%%%%%%%%%%%%%%%%%%%%%%%%%%%%%%%%%%%%%%%%%%%

%%%% Introducing logical usage of fonts.
\newcommand{\modelfont}{\mathit} % use 'mf' in command to indicate model font
\newcommand{\typefont}{\mathsf} % use 'tf' in command to indicate type font
\newcommand{\catfont}{\mathrm} % use 'cf' in command to indicate cat font

%%%%%%%%%%%%%%%%%%%%%%%%%%%%%%%%%%%%%%%%%%%%%%%%%%%%%%%%%%%%%%%%%%%%%%%%%%%%%%%%
%%%% Some macros of the book are redefined.

\renewcommand{\UU}{\typefont{U}}
\renewcommand{\isequiv}{\typefont{isEquiv}}
\renewcommand{\happly}{\typefont{hApply}}
\renewcommand{\pairr}[1]{{\mathopen{}\langle #1\rangle\mathclose{}}}
\renewcommand{\type}{\typefont{Type}}
\renewcommand{\op}[1]{{{#1}^\typefont{op}}}
\renewcommand{\susp}{\typefont{\Sigma}}

%%%%%%%%%%%%%%%%%%%%%%%%%%%%%%%%%%%%%%%%%%%%%%%%%%%%%%%%%%%%%%%%%%%%%%%%%%%%%%%%
%%%% The following is a big unorganized list of new macros that we use in the
%%%% notes. 

\newcommand{\tfW}{\typefont{W}}
\newcommand{\tfM}{\typefont{M}}
\newcommand{\mfM}{\modelfont{M}}
\newcommand{\mfN}{\modelfont{N}}
\newcommand{\tfctx}{\typefont{ctx}}
\newcommand{\mftypfunc}[1]{{\modelfont{typ}^{#1}}}
\newcommand{\mftyp}[2]{{\mftypfunc{#1}(#2)}}
\newcommand{\tftypfunc}[1]{{\typefont{typ}^{#1}}}
\newcommand{\tftyp}[2]{{\tftypfunc{#1}(#2)}}
\newcommand{\hfibfunc}[1]{\typefont{fib}_{#1}}
\newcommand{\mappingcone}[1]{\mathcal{C}_{#1}}
\newcommand{\equifib}{\typefont{equiFib}}
\newcommand{\tfcolim}{\typefont{colim}}
\newcommand{\tflim}{\typefont{lim}}
\newcommand{\tfdiag}{\typefont{diag}}
\newcommand{\tfGraph}{\typefont{Graph}}
\newcommand{\mfGraph}{\modelfont{Graph}}
\newcommand{\unitGraph}{\unit^\mfGraph}
\newcommand{\UUGraph}{\UU^\mfGraph}
\newcommand{\tfrGraph}{\typefont{rGraph}}
\newcommand{\mfrGraph}{\modelfont{rGraph}}
\newcommand{\isfunction}{\typefont{isFunction}}
\newcommand{\tfconst}{\typefont{const}}
\newcommand{\conemap}{\typefont{coneMap}}
\newcommand{\coconemap}{\typefont{coconeMap}}
\newcommand{\tflimits}{\typefont{limits}}
\newcommand{\tfcolimits}{\typefont{colimits}}
\newcommand{\islimiting}{\typefont{isLimiting}}
\newcommand{\iscolimiting}{\typefont{isColimiting}}
\newcommand{\islimit}{\typefont{isLimit}}
\newcommand{\iscolimit}{\typefont{iscolimit}}
\newcommand{\pbcone}{\typefont{cone_{pb}}}
\newcommand{\tfinj}{\typefont{inj}}
\newcommand{\tfsurj}{\typefont{surj}}
\newcommand{\tfepi}{\typefont{epi}}
\newcommand{\tftop}{\typefont{top}}
\newcommand{\sbrck}[1]{\Vert #1\Vert}
\newcommand{\strunc}[2]{\Vert #2\Vert_{#1}}
\newcommand{\gobjclass}{{\typefont{U}^\mfGraph}}
\newcommand{\gcharmap}{\typefont{fib}}
\newcommand{\diagclass}{\typefont{T}}
\newcommand{\opdiagclass}{\op{\diagclass}}
\newcommand{\equifibclass}{\diagclass^{\eqv{}{}}}
\newcommand{\universe}{\typefont{U}}
\newcommand{\catid}[1]{{\catfont{id}_{#1}}}
\newcommand{\isleftfib}{\typefont{isLeftFib}}
\newcommand{\isrightfib}{\typefont{isRightFib}}
\newcommand{\leftLiftings}{\typefont{leftLiftings}}
\newcommand{\rightLiftings}{\typefont{rightLiftings}}
\newcommand{\psh}{\typefont{Psh}}
\newcommand{\rgclass}{\typefont{\Omega^{RG}}}
\newcommand{\terms}[2][]{\lfloor #2 \rfloor^{#1}}
\newcommand{\grconstr}[2]
             {\mathchoice % max size is textstyle size.
             {{\textstyle \int_{#1}}#2}% 
             {\int_{#1}#2}%
             {\int_{#1}#2}%
             {\int_{#1}#2}}
\newcommand{\ctxhom}[3][]{\typefont{hom}_{#1}(#2,#3)}
\newcommand{\graphcharmapfunc}[1]{\gcharmap_{#1}}
\newcommand{\graphcharmap}[2][]{\graphcharmapfunc{#1}(#2)}
\newcommand{\tfexp}[1]{\typefont{exp}_{#1}}
\newcommand{\tffamfunc}{\typefont{fam}}
\newcommand{\tffam}[1]{\tffamfunc(#1)}
\newcommand{\tfev}{\typefont{ev}}
\newcommand{\tfcomp}{\typefont{comp}}
\newcommand{\isDec}[1]{\typefont{isDecidable}(#1)}
\newcommand{\smal}{\mathcal{S}}
\renewcommand{\modal}{{\ensuremath{\ocircle}}}
\newcommand{\eqrel}{\typefont{EqRel}}
\newcommand{\piw}{\ensuremath{\Pi\typefont{W}}} %% to be used in conjunction with -pretopos.
\renewcommand{\sslash}{/\!\!/}
\newcommand{\mprd}[2]{\Pi(#1,#2)}
\newcommand{\msm}[2]{\Sigma(#1,#2)}
\newcommand{\midt}[1]{\idvartype_#1}
\newcommand{\reflf}[1]{\typefont{refl}^{#1}}
\newcommand{\tfJ}{\typefont{J}}
\newcommand{\tftrans}{\typefont{trans}}

\newcommand{\tfT}{\typefont{T}}
\newcommand{\reflsym}{{\mathsf{refl}}}
\newcommand{\strans}[2]{\ensuremath{{#1}_{*}({#2})}}
\newcommand{\eqtype}[1]{\typefont{Eq}_{#1}}
\newcommand{\eqtoid}[1]{\typefont{eqtoid}(#1)}
\newcommand{\greek}{\mathrm}
\newcommand{\product}[2]{{#1}\times{#2}}
\newcommand{\pairp}[1]{(#1)}
\newcommand{\jequalizer}[3]{\{#1|#2\jdeq #3\}}
\newcommand{\jequalizerin}[2]{\iota_{#1,#2}}
\newcommand{\tounit}[1]{{!_{#1}}}
\newcommand{\trwk}{\typefont{trwk}}
\newcommand{\trext}{\typefont{trext}}

%%%%%%%%%%%%%%%%%%%%%%%%%%%%%%%%%%%%%%%%%%%%%%%%%%%%%%%%%%%%%%%%%%%%%%%%%%%%%%%%
%%%% When investigation pointed structures we use the \pt macro.

\makeatletter
\newcommand{\pt}[1][]{*_{
  \@ifnextchar\undergraph{\@undergraph@nested}
    {\@ifnextchar\underovergraph{\@underovergraph@nested}{}}#1}}
\makeatother

%%%%%%%%%%%%%%%%%%%%%%%%%%%%%%%%%%%%%%%%%%%%%%%%%%%%%%%%%%%%%%%%%%%%%%%%%%%%%%%%
%%%% OPERATIONS ON GRAPHS
%%%%
%%%% First of all, each graph has a type of vertices and a type of edges. The
%%%% type of vertices of a graph $\Gamma$ is denoted by $\pts{\Gamma}$;
%%%% and likewise for the type of edges.

\makeatletter
\newcommand{\pts}[1]{{\@graphop@nested{#1}}_{0}}
\newcommand{\edg}[1]{{\@graphop@nested{#1}}_{1}}
\newcommand{\@graphop@nested}[1]
  {\@ifnextchar\ctxext{\@ctxext@nested}
      {\@ifnextchar\undergraph{\@undergraph@nested}
         {\@ifnextchar\underovergraph{\@underovergraph@nested}{}}}
    #1}
\makeatother

%%%% The following operations of \undergraph and \underovergraph are used to
%%%% define the free category and the free groupoid of a graph, respectively

\makeatletter
\newcommand{\@undergraphtest}[2]{\@ifnextchar({#1}{#2}}
\newcommand{\undergraph}[2]{\@undergraphtest{\@undergraph@parens{#1}{#2}}{\@undergraph{#1}{#2}}}
\newcommand{\@undergraph}[2]{{#2/#1}}
\newcommand{\@undergraph@nested}[3]{\@undergraph@parens{#2}{#3}}
\newcommand{\@undergraph@parens}[2]{(\@undergraph{#1}{#2})}
\makeatother

\makeatletter
\newcommand{\underovergraph}[2]{\@underovergraphtest{\@underovergraph@parens{#1}{#2}}{\@underovergraph{#1}{#2}}}
\newcommand{\@underovergraph}[2]{{#2}\,{\parallel}\,{#1}}
\newcommand{\@underovergraphtest}{\@undergraphtest}
\newcommand{\@underovergraph@parens}[2]{(\@underovergraph{#1}{#2})}
\newcommand{\@underovergraph@nested}[3]{\@underovergraph@parens{#2}{#3}}
\makeatother

\newcommand{\graphid}[1]{\mathrm{id}_{#1}}
\newcommand{\freecat}[1]{\mathcal{C}(#1)}
\newcommand{\freegrpd}[1]{\mathcal{G}(#1)}


%%%%%%%%%%%%%%%%%%%%%%%%%%%%%%%%%%%%%%%%%%%%%%%%%%%%%%%%%%%%%%%%%%%%%%%%%%%%%%%%
%% Some tikz macros to typeset diagrams uniformly.

\tikzset{patharrow/.style={double,double equal sign distance,-,font=\scriptsize}}
\tikzset{description/.style={fill=white,inner sep=2pt}}
\tikzset{fib/.style={->>,font=\scriptsize}}

%% Used for extra wide diagrams, e.g. when the label is too large otherwise.
\tikzset{commutative diagrams/column sep/Huge/.initial=18ex}

%%%%%%%%%%%%%%%%%%%%%%%%%%%%%%%%%%%%%%%%%%%%%%%%%%%%%%%%%%%%%%%%%%%%%%%%%%%%%%%%
%%%% New theorem environment for conjectures.

\defthm{conj}{Conjecture}{Conjectures}

%%%%%%%%%%%%%%%%%%%%%%%%%%%%%%%%%%%%%%%%%%%%%%%%%%%%%%%%%%%%%%%%%%%%%%%%%%%%%%%%
%%%% The following environment for desiderata should not be there. It is better
%%%% to use the issue tracker for desiderata.

\newenvironment{desiderata}{\begingroup\color{blue}\textbf{Desiderata.}}
{\endgroup}

%%%%%%%%%%%%%%%%%%%%%%%%%%%%%%%%%%%%%%%%%%%%%%%%%%%%%%%%%%%%%%%%%%%%%%%%%%%%%%%%
%%%% The following piece of code from tex.stackexchange:
%%%%
%%%% http://tex.stackexchange.com/a/55180/14653
%%%%
%%%% We include it so that inference rules in align environments have enough
%%%% vertical space.

\newlength\minalignvsep

\makeatletter
\def\align@preamble{%
   &\hfil
    \setboxz@h{\@lign$\m@th\displaystyle{##}$}%
    \ifnum\row@>\@ne
    \ifdim\ht\z@>\ht\strutbox@
    \dimen@\ht\z@
    \advance\dimen@\minalignvsep
    \ht\strutbox\dimen@
    \fi\fi
    \strut@
    \ifmeasuring@\savefieldlength@\fi
    \set@field
    \tabskip\z@skip
   &\setboxz@h{\@lign$\m@th\displaystyle{{}##}$}%
    \ifnum\row@>\@ne
    \ifdim\ht\z@>\ht\strutbox@
    \dimen@\ht\z@
    \advance\dimen@\minalignvsep
    \ht\strutbox@\dimen@
    \fi\fi
    \strut@
    \ifmeasuring@\savefieldlength@\fi
    \set@field
    \hfil
    \tabskip\alignsep@
}
\makeatother

\minalignvsep.2em

\allowdisplaybreaks

%%%%%%%%%%%%%%%%%%%%%%%%%%%%%%%%%%%%%%%%%%%%%%%%%%%%%%%%%%%%%%%%%%%%%%%%%%%%%%%%

\setdescription[1]{itemsep=-0.2em}


%%%%%%%%%%%%%%%%%%%%%%%%%%%%%%%%%%%%%%%%%%%%%%%%%%%%%%%%%%%%%%%%%%%%%%%%%%%%%%%%
\title{Internal models of type theory}
\author{Egbert Rijke}
\date{\today}

\begin{document}

\maketitle

\tableofcontents

\begin{comment}
\section*{Introduction\phantomsection\addcontentsline{toc}{section}{Introduction}}
This paper contributes to the univalent homotopy type theory~\cite{kapulkin2012univalence,awodey2012type,pelayo2012homotopy}
program which connects and extends a number of topics such as type theory, (elementary) topos
theory~\cite{MacLaneMoerdijk,johnstone:elephant}, homotopy theory, higher category theory and higher topos
theory~\cite{rezk2010toposes,lurie2009higher}.
A key missing ingredient is a first-order (`elementary') definition of a higher topos. It has been
conjectured~\cite{awodey2012type,shulman2012univalence} that homotopy type theory can be used as the internal language, and thus may
provide an elementary definition of a higher topos; see also~\cite{LumsdaineWarren}.

A main ingredient in the theory of higher toposes is the descent property, which
is described in~\cite{rezk2010toposes,lurie2009higher}. The (weak) descent property
of an ordinary topos $\mathcal{E}$ is the property that
\begin{enumerate}
\item For any family $\{X_i\}_{i\in I}$ of objects of $\mathcal{E}$ with coproduct
$X:=\coprod_{i\in I}X_i$ and any function $f:Y\to X$ we get an isomorphism
\begin{equation*}
Y\cong\coprod_{i\in I} Y\times_X X_i
\end{equation*}
given by the canonical map.
\item Conversely, for any family $\{f_i:X_i\to Y_i\}_{i\in I}$ of maps inducing
$f:\coprod_{i\in I}X_i\to\coprod_{i\in I}Y_i$, the canonical map determines an
isomporphism
\begin{equation*}
Y_i\cong Y\times_X X_i.
\end{equation*}
\item
\item
\end{enumerate}
In model toposes, the descent property can be strengthened considerably, but
ordinary limits and colimits are to be replaced by \emph{homotopy} limits and
colimits. Since the only notion of limits and colimits in type theory is that
of homotopy limit and colimit, this is no restriction for us. The descent
property of model toposes as formulated by Rezk~\cite{rezk2010toposes} is: Let
$X:\mathbf{I}\to\mathcal{E}$ be a functor from a small category into a
model topos $\mathcal{E}$, with homotopy colimit $\bar{X}$. Then we have
\begin{enumerate}
\item Let $\bar{Y}$ be an object and let $f:\bar{Y}\to\bar{X}$ 
be any map in $\mathcal{E}$. Define
the functor $Y:\mathbf{I}\to\mathcal{E}$ by $Y(i):=Y\times_{\bar{X}} X_i$. Then
the canonical map $\typefont{hocolim}(Y)\to\bar{Y}$ is a weak equivalence.
\item Let $Y:\mathbf{I}\to\mathcal{E}$ be a functor and let $f:Y\to X$ be an
equifibered transformation from $Y$ to $X$. Define $\bar{f}:\bar{Y}\to\bar{X}$
be the homotopy colimit of $f$. Then the canonical map 
$Y_i\to \bar{Y}\times^h_{\bar{X}} X_i$ is a weak equivalence.
\end{enumerate}
These two properties are two sides of the same coin and in our type theoretical
setting it is natural to stress this. Given a diagram $D$, we will be able
to summarize the entire descent property in an equivalence between the type
$\typefont{equiFib}(D)$ of equifibered diagrams over $D$ and the type of functions
into $\typefont{colim}(D)$. We prove the descent property in 
theorem~\ref{thm:descent}.

We use universes ala Russell, universe polymorphism and typical ambiguity and hence create the illusion that $\type:\type$;
see~\cite{TheBook}. There are only a few places where we need to be explicit about universe levels.
As is common in HoTT, we will freely use the axiom of function extensionality.
Using this axiom, all limits can be constructed in type theory~\cite{Avigad:limits}.
We use higher inductive types (HITs) to construct co-limits; see~\cite{LumsdaineShulman,Shulman:HIT}.
Using the univalence axiom one can show that $\type$ is an object classifier~\cite{RijkeSpitters:Sets}.
An important feature property of $\infty$-toposes is the descent property. We prove this property in theorem~\ref{thm:descent}.

In the previous paragraph we have used informal categorical terminology such as `limits' to refer to what would be interpreted as homotopy
limits in the $\infty$-category of $\infty$-groupoids. We will continue to use this naive $\infty$-category style without any rigorous
claims to their interpretation in the model.

The paper is organized as follows. We start by defining limits and colimits of over graph. This is the most general case, as we
do not include the possibility of including higher coherence conditions.
We introduce diagrams and (co)limits in section~\ref{sec:diagram}. In section~\ref{sec:suspensions} we study suspensions.
Functions with contractible mapping cones are the same as epimorphisms, as we will see in section~\ref{sec:mapping}.
Finally, we prove the descent property.

\subsection{Notation}
A typical way we write a type in this article is:
\begin{equation}\label{eq:notation_example}
\prd{x,y:A}{p:B(x,y)}C(x,y,p)\to D(x,y,p).
\end{equation}
Assuming that $A$, $B$, $C$ and $D$ have been introduced, this type denotes a
dependent product, where the product ranges over $x,y:A$ and $p:B(x,y)$ and
the factors are the (non-dependent) function types $C(x,y,p)\to D(x,y,p)$. A
more common notation for the same type in mathematics would be
\begin{equation*}
\prod_{x,y:A}\prod_{p:B(x,y)}C(x,y,p)\to D(x,y,p),
\end{equation*}
like a big product of function spaces. Nevertheless we have chosen to use this
less common notation, because the types $A$ or $B$ can become quite complex.
% and formulas would become messy as as complex information starts `dancing
% around'. 

Another thing we notice in formula \eqref{eq:notation_example} is the use of
curly braces around $x,y:A$. This indicates that whenever we have a term
\begin{equation*}
f:\prd{x,y:A}{p:B(x,y)}C(x,y,p)\to D(x,y,p),
\end{equation*}
we will omit reference to $x$ and $y$ whenever $f$ is applied to $x,y:A$ and
$p:B(x,y)$. In other words, we will write down $f(p):C(x,y,p)\to D(x,y,p)$
instead of $f(x,y,p)$. We will do this whenever it is possible to retrieve
the terms $x$ and $y$ from the term $p$, and in many similar situations like
this. Back to formula \autoref{eq:notation_example}, it is quite possible 
that we would have used the curly braces in the introduction
of $C$ and $D$ as well, and if we had done that we would have written $C(p)$
and $D(p)$.
\end{comment}

\section{Introduction}
In this paper we define a notion of internal model which is adapted to
the univalent setting in which we work. Thus, we only describe what it means
to be an internal model when the ambient type theory is univalent. Our
presentation is derived from the definition of~\cite{Dybjer1996}. There,
an internal model is defined to be a category with families with interpretations
for the basic type constructors $\Pi$, $\Sigma$ and $\idtypevar{}$. However,
a category in~\cite{Dybjer1996} has a set of objects and for every two objects
a setoid of morphisms. This is too restrictive for our purpose. Ideally, we
would start with an $\infty$-category of contexts and an $\infty$-presheaf
of families over it. This presents us with the problem that a notion of
$\infty$-category has not been formulated \emph{in} type theory. Therefore,
we make a deviation from the approach in~\cite{Dybjer1996}. We shall
start with a type $\tfctx$ of contexts, a family $\mftypfunc:\tfctx\to\type$
of families over it and a family $\terms{\blank}:\prd*{\Gamma:\tfctx}\mftyp{\mfM}{\Gamma}
\to\type$ sending a type $A$ in context $\Gamma$ to its type $\terms{A}$ of
terms; then we shall directly interpret $\Pi$-types, declaring
$\terms{A\to B}$ to be $\ctxhom[\Gamma]{A}{B}$. The idea is that by letting
$\Pi$ play a fundamental role in the definition of a category with families
$\mathcal{C}$,
the morphisms get their properties directly from the internal type theory
of $\mathcal{C}$.

\begin{comment}
We briefly list the data of which an internal model of type theory consits. We
consider internal models in the style of~\cite{Dybjer1996}. Thus we will
describe a model $\mfM$ as a category with families. However, since our setting is
univalent type theory we will deviate from~\cite{Dybjer1996} in the following
respects: first of all, we make use of typical ambiguity and hence omit
reference to a thing called \emph{sort}. 

We organize the definition of an internal model $\mfM$ as follows: in
\autoref{internal-model-contexts} we describe the category of contexts itself;
in \autoref{internal-model-families} we describe the families over a given context
and the related operations; in \autoref{internal-model-constructors} we
describe the basic constructors in the internal model.
\end{comment}

\subsection{Ideas in the definition}
An internal model of type theory is like a category with families, but we want
to avoid having to state higher coherences. In fact, we don't even start our
definition with a category of contexts; instead we just take a \emph{type} of contexts. 
The morphisms will come from the terms, evaluation of a function at a given
term will come from substitution. We recognize three basic ingredients to models:
first there is a type of contexts; second, for every context there is a model of types in
that context and third, for every type in a given context there is a type of its
terms. Then there are three basic attributes: context extension, weakening and
substitution. Context extension provides us with families over types as well as
with an interpretation of dependent pair types. We need weakening 
so that families can depend on the same type multiple times (the way the
identity type of a type depends two times on that type) and to be able
to talk about non-dependent function types,
the morphims of our category. Substitution will give us a way
to work with fibers of families as well as composition of functions and evaluation
of functions at terms.

Because we require a \emph{model} of types in a context, all the structure
which we require at the bottom level will be required to exist higher up as well.
Thus, the model of types in a given context $\Gamma$ will have a type of contexts
itself, which can be seen as the type of types in $\Gamma$; it will have its
own notion of types in a context, its own notion of terms, context extension,
weakening and substitution together with all the structure require for it. For
instance, when $A$ is a type in context $\Gamma$ in a model $\mfM$, then there
is the model of types in context $A$, which is the model of families over $A$. 
This model is required to be \emph{definitionally equal to} the model of types
in the context $\ctxext{\Gamma}A$, the context extension of $\Gamma$ and $A$.
In this way we protect ourselves from the need to dig an infinitely deep structure
of models when we want to consider examples.

To give the definition of a model we shall also need to consider certain morphisms
of internal models. Those should preserve all the structure: contexts are mapped
to contexts; for every context a morphism of models mapping the model of types
in that context to the model of types in the image of that context; there should
be a mapping of terms and context extension, weakening and substitution should be
preserved. We need to consider those morphisms because we require context extension,
weakening and substitution to be of that kind, thereby respecting each other
in all possible ways.

When we have this framework set up, we can interpret the basic type constructors
such as $\Pi$, $\Sigma$ and $\idtypevar{}$.
The higher categorical structure then comes from the
result that we have an interpretation of type theory.

\begingroup
\color{red}
\begin{rmk}
Currently, the definition seems to be circular. To define a model we need that
context extension, weakening and substitution be morphisms of models. A morphism
of models needs to preserve context extension, weakening and substitution.
Moreover, its definition requires the notion of composition of morphisms.

I've done it this way because it guides me to what the rules should be, trusting
that I can work my way back to give a (possibly less transparent) definition
which contains no circularities without doubt.
\end{rmk}
\endgroup


\section{Empirical evidence of what an internal model should be}\label{eg}

\subsection{Univalent universes as internal models}

Before we give the definition we illustrate the concepts that go into it in
the case of a univalent universe $\UU$.
Regardless of the definition of an internal model, a univalent universe should
be an example of one.

The type of contexts is $\UU$ itself and for every $\Gamma:\UU$, a type in
context $\Gamma$ is simply a family $A:\Gamma\to\UU$. The type $\terms{A}$ of terms of a type $A$
in context $\Gamma$ is defined to be $\prd{i:\Gamma}A(i)$. Note that $\Gamma\to
\UU$ itself also interprets type theory where a type in context $A:\Gamma\to\UU$
is a family $P:(\sm{i:\Gamma}A(i))\to\UU$. We may denote this model by
$\mftyp{\UU}{\Gamma}$. The terms of a type $P$ in context $A$ in the model
$\Gamma\to\UU$ are the terms of $\prd{i:\Gamma}{x:A(i)}P(i,x)$.

When $A$ is a type in context
$\Gamma$ we define the context extension $\ctxext{\Gamma}{A}$ to be
$\sm{i:\Gamma}A(i)$. Note that context extension $\ctxext{\Gamma}{\blank}$
can be seen acting not only as a function from the context of the model
$\Gamma\to\UU$ to the context of the model $\UU$, but it also acts on the types 
and terms: when $A$ is a type in context $\Gamma$ we may take the identity map
from $\mftyp{\Gamma\to\UU}{A}\to\mftyp{\UU}{\ctxext{\Gamma}{A}}$ because
$\mftyp{\Gamma\to\UU}{A}$ is taken to be $\mftyp{\UU}{\ctxext{\Gamma}{A}}$;
thus, $\ctxext{\Gamma}{P}\defeq P$ for every family $P$ over $A$ in context 
$\Gamma$. Likewise, when $P$ is a type in context $A$ in the model 
$\Gamma\to\UU$ then context extension should act on the terms of $P$ via a
function $\terms{P}\to\terms{\ctxext{\Gamma}{P}}$, which we take to be
the identity map once more.

Since $\mftyp{\UU}{\Gamma}$ is a model of type theory it
has it's own notion of context extension: when $P$ is a type in context $A$ in
the model $\mftyp{\UU}{\Gamma}$ then $\ctxext{A}{P}$ is the family
$\lam{i}\sm{x:A(i)}P(i,x):\Gamma\to\UU$. Also the context extension of
$\Gamma\to\UU$ acts trivially on the types and terms. Context extension is
analoguous to the Grothendieck construction that associates the category of
elements to a presheaf and it gives us $\Sigma$-types.

When $A$ and $B$ are types in context
$\Gamma$, the weakening $\ctxwk{A}{B}$ of $B$ along $A$ is defined to be
$\lam{\pairr{i,x}}B(i):(\sm{i:\Gamma}A(i))\to\UU$. Weakening along a type $A$ 
in context $\Gamma$ also acts on types and terms. When $Q$ is a type in context
$B$ in the model $\Gamma\to\UU$, we define $\ctxwk{A}{Q}$ to be 
$\lam{\pairr{i,x}}{y}Q(i,y)$. When $g:\terms{Q}$ we define $\ctxwk{A}{g}$ to
be $\lam{\pairr{i,x}}{y}g(i,y)$.

Note that for two types $A$ and $B$ in context $\Gamma$, the terms of
$\ctxwk{A}{B}$ are the terms of $\prd{i:\Gamma}A(i)\to B(i)$, i.e.~they are
the fiberwise maps from $A$ to $B$. We shall take the type $\terms{\ctxwk{A}{B}}$
of terms of $\ctxwk{A}{B}$ to be the type of morphisms from $A$ to $B$. Also,
any context of $\UU$ may be seen as a type in the empty context $\unit$. Thus
the type of terms of a context $\Gamma$ is $\unit\to\Gamma$, which is
equivalent to $\Gamma$. A context morphism from $\Delta$ to $\Gamma$ is a term
of $\ctxwk{\Gamma}{\Delta}$, i.e.~a function from $\Gamma$ to $\Delta$. We denote
the type $\terms{\ctxwk{\Gamma}{\Delta}}$ by $\ctxhom{\Delta}{\Gamma}$. 

When $P$ is a family over
$A$ in context $\Gamma$ and $x$ is a term of $A$ we define the type $\subst{x}{P}$
in context $\Gamma$ to be $\lam{i}P(i,x(i))$. Like extension 

We will interpret the dependent product $\mprd{A}{P}$ of a family $P$ over
$A$ in context $\Gamma$ by
\begin{equation*}
\mprd{A}{P}(i)\defeq \prd{x:A(i)}P(i,x)
\end{equation*}
With this interpretation there is an equivalence 
$\lambda:\eqv{\terms{P}}{\terms{\mprd{A}{P}}}$. This is lambda-abstraction; its
inverse being evaluation. We should note however, that the rule for evaluation
we interpret here is
\begin{equation*}
\inference{\Gamma\vdash f:\mprd{A}{P}}{\ctxext{\Gamma}{A}\vdash\tfev(f):P}
\end{equation*}
which is different than the usual rule
\begin{equation*}
\inference{\Gamma\vdash f:\mprd{A}{P} \qquad \Gamma\vdash a:A}{\Gamma \vdash \tfev(f,a):\subst{a}{P}}
\end{equation*}
The reason for this is that the interpretation of the usual rule would give a
function of type $\terms{\mprd{A}{P}}\to\prd{x:\terms{A}}\terms{\subst{x}{P}}$,
but this does not describe the terms of $\mprd{A}{P}$ by any means. Moreover,
since we have implemented substitution, we obtain from $\ctxext{\Gamma}{A}\vdash\tfev(f):P$ 
and $\Gamma\vdash x:A$ a term $\Gamma\vdash\subst{x}{\tfev(f)}:\subst{x}{P}$ and
therefore we do not loose anything with this approach.

{\color{red} point to example}

\subsection{The graph model of type theory}
In this section we define the graph model of type theory, denoted by
$\mfGraph$.  In our presentation, we follow that of the definition internal
models. After we have established the graph model, we show how the graph
model can be seen as a presheaf model, namely over the category $\cdot
{\rightrightarrows}\cdot$.

\begin{defn}
A \emph{(directed) graph} $\Gamma$ is a pair $\pairr{\Gamma_0,\Gamma_1}$ 
consisting of a type $\Gamma_0$ of vertices and a family 
$\Gamma_1:\Gamma_0\to\Gamma_0\to\type$ of edges. The type $\ctx(\mfGraph)$
is defined to be the type of all graphs; we will usually denote it by
$\tfGraph$. Explicitly, we have
\begin{equation*}
\tfGraph\defeq\sm{\Gamma_0:\type}\Gamma_0\to\Gamma_0\to\type.
\end{equation*}
\end{defn}

\begin{eg}\label{ex:pb}
The underlying graph of the diagram
\begin{equation*}
\begin{tikzcd}
{} & A \ar{d}{f} \\
B \ar{r}[swap]{g} & C
\end{tikzcd}
\begin{comment}
\begin{tikzpicture}
\matrix (m) [std] { & A \\ B & C \\};
\draw[ar] (m-1-2) -- node[right] {$f$} (m-2-2);
\draw[ar] (m-2-1) -- node[below] {$g$} (m-2-2);
\end{tikzpicture}
\end{comment}
\end{equation*}
is $I\defeq\mathbf{3}$ and has $J(1,3)\defeq J(2,3)\defeq \unit $ 
and $J(x,y)\defeq\emptyt$ otherwise (it is defined using the induction 
principle of $\mathbf{3}$ and a universe).
\end{eg}

\begin{defn}
A \emph{family $A$ of graphs in the context $\Gamma$} is a pair $\pairr{A_0,A_1}$ consisting
of 
\begin{align*}
A_0 & :\Gamma_0\to\type\\
A_1 & :\prd*{i,j:\Gamma_0}\Gamma_1(i,j)\to A_0(i)\to A_0(j)\to\type.
\end{align*}
Thus, for a graph $\Gamma$, the type $\mftyp{\mfGraph}{\Gamma}$ is the type
\begin{equation*}
\sm{A_0:\Gamma_0\to\type}\prd*{i,j:\Gamma_0}\Gamma_1(i,j)\to A_0(i)\to A_0(j)\to\type.
\end{equation*}
We will also write $\Gamma\vdash A:\mfGraph$ when $A$ is a graph in context
$\Gamma$.
\end{defn}

\begin{defn}
Suppose that $\Gamma\vdash A:\mfGraph$. A term $x$ of $A$ consists of a pair
$\pairr{x_0,x_1}$ where
\begin{align*}
x_0 & : \prd{i:\Gamma_0}A_0(i)\\
x_1 & : \prd*{i,j:\Gamma_0}{q:\Gamma_1}A_1(q,x_0(i),x_0(j)).
\end{align*}
Thus we define
\begin{equation*}
\terms{A}\defeq\sm{x_0:\prd{i:\Gamma_0}A_0(i)}\prd*{i,j:\Gamma_0}{q:\Gamma_1}A_1(q,x_0(i),x_0(j)).
\end{equation*}
We also write $\Gamma\vdash x:A$ when $x$ is a term of the graph $A$ in context
$\Gamma$.
\end{defn}

\subsubsection{The interpretations of extension, weakening and substitution}
\begin{defn}
Suppose that $\Gamma\vdash A:\mfGraph$. Then we define the graph $\ctxext{\Gamma}{A}$
by
\begin{align*}
\ctxext{\Gamma}{A}_0 & \defeq \sm{i:\Gamma_0}A_0(i)\\
\ctxext{\Gamma}{A}_1(\pairr{i,x},\pairr{j,y}) & \defeq \sm{q:\Gamma_1(i,j)}A_1(q,x,y).
\end{align*}
\end{defn}

\begin{defn}
Suppose that $\Gamma\vdash A:\mfGraph$ and $\Gamma\vdash B:\mfGraph$. Then we
define the graph $\ctxwk{A}{B}$ in context $\ctxext{\Gamma}{A}$ by
\begin{align*}
(\ctxwk{A}{B})_0(\pairr{i,x}) & \defeq B_0(i)\\
(\ctxwk{A}{B})_1(\pairr{q,e},u,v) & \defeq B_1(q,u,v).
\end{align*}
\end{defn}

\begin{defn}
Suppose that $P$ is a family of graphs over $A$ in context $\Gamma$ and let
$x$ be a term of $A$. Then we define the graph $\subst{x}{P}$ 
in context $\Gamma$ by
\begin{align*}
\subst{x}{P}_0(i) & \defeq P_0(\pairr{i,x_0(i)})\\
\subst{x}{P}_1(q,u,v) & \defeq P_1(\pairr{q,x_1(q)},u,v)
\end{align*}
\end{defn}

\begin{rmk}
Note that we have the judgmental equality $\subst{x}{\ctxwk{A}{B}}\jdeq B$
for every two graphs $A$ and $B$ in context $\Gamma$ and every term $x$ of $A$.
\end{rmk}

\begin{rmk}
Using weakening we can describe the graph morphisms. Suppose that $\Delta$ and
$\Gamma$ are graphs. A graph morphism from $\Gamma$ to $\Delta$ is a term of 
the family $\ctxwk{\Gamma}{\Delta}$ of graphs over $\Gamma$. More explicitly,
a graph morphism $f$ is a pair
\begin{align*}
\pts{f} & : \pts{\Gamma}\pts{\Delta}\\
\edg{f} & : \prd*{i,j:\pts{\Gamma}} \edg{\Gamma}(i,j)\to\edg{\Delta}(\pts{f}(i),\pts{f}(j))
\end{align*}
just as expected.

We may also describe morphisms of families of graphs this way.
Let $A$ and $B$ be two families of graphs over a graph $\Gamma$. 
A morphism from $A$ to $B$ is a term of $\ctxwk{A}{B}$. More explicitly,
a morphism $f$ from $A$ to $B$ is a pair $\pairr{f_0,f_1}$
consisting of
\begin{align*}
\pts{f} & : \prd*{i:\Gamma_0} A_0(i)\to B_0(i)\\
\edg{f} & : \prd{\pairr{i,x},\pairr{j,y}:\pts{\ctxext{\Gamma}{A}}}{\pairr{q,e}:\edg{\ctxext{\Gamma}{A}}(\pairr{i,x},\pairr{j,y})} \edg{B}(q,f_0(x),f_0(y)).
\end{align*}
Here we have suppressed the notation for the projections.
\end{rmk}

\subsubsection{The interpretations of the type constructors}

\begin{defn}
The terminal graph $\unit^\mfGraph$, which we shall often denote simply 
by $\unit$, is defined by
\begin{align*}
{\unit^\mfGraph}_0 & \defeq \unit\\
{\unit^\mfGraph}_1(x,y) & \defeq \unit.
\end{align*}
\end{defn}

\begin{rmk}
We note that the function $\ctxext{\unit}{\blank}$ is an equivalence
from $\mftyp{\mfGraph}{\unit}$ to $\tfGraph$. It's inverse is the function which maps
a graph $\Gamma$ to the pair $\pairr{A_0,A_1}$ where $A_0$ is defined by
$A_0(\ttt)\defeq\Gamma_0$ and where $A_1$ is defined by $A_1(\ttt)\defeq
\Gamma_1$. 

Note that we only have an equivalence here, not a judgmental equality.
\end{rmk}

With families of graphs being available, we can give the interpretations of
dependent products, dependent sums and identity types. In the following, we
shall introduce the graph interpretations of dependent products, dependent
sums and identity types and describe the terms of the resulting graphs.\note{These definitions
should be connected to Mike's article but I don't really know how to do this}

\begin{defn}
Let $P$ be a family of graphs over $A$, where $\Gamma\vdash A:\mfGraph$. 
The dependent function graph $\mprd{A}{P}$ in context $\Gamma$ consists of
\begin{align*}
\mprd{A}{P}_0(i) & \defeq \prd{x:A_0(i)}P_0(x)\\
\mprd{A}{P}_1(q,f,g) & \defeq \prd*{x:A_0(i)}*{y:A_0(j)}{e:A_1(q,x,y)}P_1(\pairr{q,e},f(x),g(y)).
\end{align*}
\end{defn}

\begin{rmk}
A term $f:\mprd{A}{P}$ consists of
\begin{align*}
f_0 & : \prd*{i:\Gamma_0}{x:A_0(i)}P_0(x)\\
f_1 & : \prd*{i,j:\Gamma_0}{q:\Gamma_1(i,j)}*{x:A_0(i)}*{y:A_0(j)}{e:A_1(q,x,y)}P_1(\pairr{q,e},f_0(x),f_0(y))
\end{align*}
Therefore, we see that $\eqv{\terms{\mprd{A}{P}}}{\terms{P}}$. \emph{Warning:} it
is by no means the case that $\eqv{\terms{\mprd{A}{P}}}{\prd{x:\terms{A}}
\terms{\subst{x}{P}}}$ for all families $P$ over $A$. For instance, the graph
$\tilde{\emptyt}$ defined by $\tilde{\emptyt}_0\defeq\unit$ and 
$\tilde{\emptyt}_1(\ttt,\ttt)\defeq\emptyt$ has no terms and neither does
$\tilde{\emptyt}+\tilde{\emptyt}$. Nevertheless, there are two graph morphisms
from $\tilde{\emptyt}$ to $\tilde{\emptyt}+\tilde{\emptyt}$. More generally,
when $\Gamma$ is a graph such that $\Gamma_1(i,j)\jdeq\emptyt$ for all $i,j:\Gamma_0$,
then $\eqv{{\terms{\Gamma\to\Gamma'}}}{\Gamma_0\to\Gamma^\prime_0}$.
\end{rmk}

\begin{defn}
If $P$ is a family of graphs over $\Gamma$, the dependent pair graph
$\msm{A}{P}$ consists of
\begin{align*}
\msm{A}{P}_0(i) & \defeq \sm{x:A_0(i)}P_0(x)\\
\msm{A}{P}_1(q,\pairr{x,u},\pairr{y,v}) & \defeq \sm{e:A_1(q,x,y)}P_1(\pairr{q,e},u,v).
\end{align*}
\end{defn}

\begin{rmk}
A term $w:\msm{A}{P}$ consists of
\begin{align*}
w_0 & : \prd{i:\Gamma_0}\sm{x:A_0(i)}P_0(x)
\intertext{and, writing $\lam{i}\proj1(w_0(i))$ and $\lam{i}\proj2(w_0(i))$ as
$w_{00}$ and $w_{01}$ respectively,}
w_1 & : \prd*{i,j:\Gamma_0}{q:\Gamma_1(i,j)}\sm{e:A_1(q,w_{00}(i),w_{00}(j)}P_1(\pairr{q,e},w_{01}(i),w_{01}(j)).
\end{align*}
By $\choice{\infty}$ it follows that
\begin{equation*}
\eqv{\terms{\msm{A}{P}}}{\sm{x:\terms{A}}\terms{\subst{x}{P}}}.
\end{equation*}
\end{rmk}

\begin{defn}
Let $A$ be a graph in context $\Gamma$. We define the family $\idtypevar{A}$ over $\ctxwk{A}{A}$ in
context $\ctxext{\Gamma}{A}$ by
\begin{align*}
(\idtypevar{A})_0(\pairr{i,x},y) & \defeq \id{x}{y}\\
(\idtypevar{A}){}_1(\pairr{q,e},d,\alpha,\alpha') & \defeq \id{\trans{\pairr{\alpha,\alpha'}}{e}}{d}
\end{align*}
where $q:\Gamma_1(i,j)$, $e:A_1(q,x,x')$, $d:A_1(q,y,y')$, $\alpha:\id{x}{y}$
and $\alpha':\id{x'}{y'}$. The transportation along the path 
$\pairr{\alpha,\alpha'}:\id{\pairr{x,x'}}{\pairr{y,y'}}$ in $A_0(i)\times A_0(j)$
is taken with respect to the family $\lam{x}{x'}A_1(q,x,x')$.

We define the term $\refl{A}$ of the family 
$\subst{\idfunc[A]}{\idtypevar{A}}$ over $A$ in context $\Gamma$ by
\begin{align*}
(\reflf{A})_0(i) & \defeq \lam{x}\refl{x}\\
(\reflf{A})_1(q) & \defeq \lam{e}\refl{e}
\end{align*}
\end{defn}

\begin{defn}
Let $D$ be a family over $\idtypevar{A}$ in context 
$\ctxext{{\Gamma}{A}}{\ctxwk{A}{A}}$. Then we have the family
$\subst{\idfunc[A]}{D}$ over $\subst{\idfunc[A]}{\idtypevar{A}}$ in
context $\ctxext{\Gamma}{A}$ given by
\begin{align*}
\subst{\idfunc[A]}{D}_0(\pairr{i,x},\alpha) & \defeq D_0(\pairr{i,x,x},\alpha)\\
\subst{\idfunc[A]}{D}_1(\pairr{q,e},\gamma) & \defeq D_1(\pairr{q,e,e},\gamma).
\end{align*}
The family $\subst{\reflf{A}}{\subst{\idfunc[A]}{D}}$ over $A$ in
context $\Gamma$ is given by
\begin{align*}
\subst{\reflf{A}}{\subst{\idfunc[A]}{D}}_0(i,x) & \defeq D_0(i,x,x,\refl{x})\\
\subst{\reflf{A}}{\subst{\idfunc[A]}{D}}_1(q,e) & \defeq D_1(q,e,e,\refl{e}).
\end{align*}
To show that the identity graphs correctly interpret the identity elimination
rule, we must give a function
\begin{equation*}
\tfJ : \terms{\subst{\reflf{A}}{\subst{\idfunc[A]}{D}}}\to\terms{D}.
\end{equation*}
Note that a term $d$ of $\subst{\reflf{A}}{\subst{\idfunc[A]}{D}}$
consists of
\begin{align*}
d_0 & : \prd*{i:\Gamma_0}{x:A_0(i)}D_0(i,x,x,\refl{x})\\
d_1 & : \prd*{i,j:\Gamma_0}{q:\Gamma_1(i,j)}*{x:A_0(i)}*{y:A_0(j)}{e:A_1(q,x,y)}D_1(q,e,e,\refl{e})
\end{align*}
A simple argument using path induction reveals that terms of
$\subst{\reflf{A}}{\subst{\idfunc[A]}{D}}$ indeed yield terms of $D$. 
\end{defn}

\begin{rmk}
Using the identity graph $\idtypevar{A}$ we can describe the identity
graph $\id[A]{x}{y}$ in context $\Gamma$ for any two terms $x,y:A$. 
The graph $\id[A]{x}{y}$ in context $\Gamma$ consists of
\begin{align*}
(\id[A]{x}{y})_0(i) & \defeq \id{x_0(i)}{y_0(i)}\\
(\id[A]{x}{y})_1(q,\alpha,\beta) & \defeq \id{\trans{\pairr{\alpha,\beta}}{x_1(q)}}{y_1(q)}
\end{align*}
From this, we see that a term of $p:\id[A]{x}{y}$ consists of
\begin{align*}
p_0 & : \prd{i:\Gamma_0}\id{x_0(i)}{y_0(i)}\\
p_1 & : \prd*{i,j:\Gamma_0}{q:\Gamma_1(i,j)}\id{\trans{\pairr{p_0(i),p_0(j)}}{x_1(q)}}{y_1(q)}.
\end{align*}
\end{rmk}


\begin{comment}
\subsubsection{The contexts of $\mfGraph$}


The category $\psh(\cdot{\rightrightarrows}\cdot)$ of presheaves over
$\cdot{\rightrightarrows}\cdot$ is given by
\begin{equation*}
\sm{A_0,A_1:\type}(A_1\to A_0)\times(A_1\to A_0).
\end{equation*}
To see that $\psh(\cdot{\rightrightarrows}\cdot)$ is indeed equivalent to
the type $\tfGraph$ of all graphs, note that we have the equivalences
\begin{align*}
\psh(\cdot{\rightrightarrows}\cdot) & \eqvsym \sm{A_0,A_1:\type}A_1\to A_0\times A_0\\
& \eqvsym \sm{A_0:\type} A_0\times A_0\to\type\\
& \eqvsym \tfGraph.
\end{align*}

We now turn to the description of the universe $\gobjclass$ of graphs. Note
that for the category $I\defeq 0{\rightrightarrows}1$, we have
$I/0\defeq\catid{0}$ and $I/1\defeq s\rightarrow\catid{1}\leftarrow t$, where
$s$ and $t$ are the morphisms $0\to 1$. Thus we have 
\begin{align*}
\psh(I/0) & \eqvsym \type\\
\psh(I/1) & \eqvsym \sm{X,Y:\type}X\to Y\to\type.
\end{align*}
The functors $\psh(\Sigma_s)$ and $\psh(\Sigma_t)$ map a presheaf 
$\pairr{X,Y,R}$ to $X$ and $Y$, respectively. 
Thus we obtain the following graph $\gobjclass$.

\begin{defn}
The universe $\gobjclass$ of graphs is defined to be
\begin{align*}
{\gobjclass}_0 & \defeq \type\\
{\gobjclass}_1(X,Y) & \defeq X\to Y\to\type.
\end{align*}
\end{defn}

Note that the type $\terms{\gobjclass}$ of terms of $\gobjclass$ is 
exactly $\tfGraph$. 

\subsubsection{The basic type constructors for graphs}

\begin{defn}
For any graph $\Gamma$ there is a graph $\tffam{\Gamma}$ in context 
$\gobjclass$ defined by
\begin{align*}
(\tffam{\Gamma})_0 & \defeq \lam{X}X\to\Gamma_0\\
(\tffam{\Gamma})_1(X,Y,R) & \defeq \lam{f}{g}\prd*{x:X}*{y:Y}R(x,y)\to\Gamma_1(f(x),g(y)).
\end{align*}
\end{defn}

\begin{rmk}
A term $D$ of $\msm{\gobjclass}{\tffam{\Gamma}}$ consists of a term
$\pairr{\Delta_0,f_0}$ of type
\begin{equation*}
\msm{\gobjclass}{\tffam{\Gamma}}_0\jdeq \sm{\Delta_0:\type}\Delta_0\to\Gamma_0
\end{equation*}
and a term $\pairr{\Delta_1,f_1}$ of type
\begin{equation*}
\msm{\gobjclass}{\tffam{\Gamma}}_1(q)\jdeq \sm{\Delta_1}
\end{equation*}
\end{rmk}

\begin{defn}
There is a graph morphism
\begin{equation*}
\graphcharmapfunc{\Gamma} : \sm{\gobjclass}
\end{equation*}
\end{defn}
\end{comment}

\subsubsection{Contractibility and equivalences of graphs}

\begin{defn}
A graph $A$ in context $\Gamma$ is said to be \emph{contractible} if there
is a term of the graph
\begin{equation*}
\msm{A}{\mprd{\ctxwk{A}{A}}{\idtypevar{A}}}
\end{equation*}
in context $\Gamma$.
\end{defn}

\begin{lem}\label{lem:contractible-graphs}
Let $A$ be a graph in context $\Gamma$. The following are equivalent:
\begin{enumerate}
\item $A$ is a contractible graph.
\item Both $A_0(i)$ and $A_1(q,x,y)$ are always contractible.
\end{enumerate}
\end{lem}

\begin{proof}
Let $H:\msm{A}{\mprd{\ctxwk{A}{A}}{\idtypevar{A}}}$. Unfolding the definitions, we have
an element $H_0(i)$ of type
\begin{align*}
\msm{A}{\mprd{\ctxwk{A}{A}}{\idtypevar{A}}}_0(i) & \jdeq \sm{x:A_0(i)}\mprd{\ctxwk{A}{A}}{\idtypevar{A}}_0(i,x)\\
& \jdeq \sm{x:A_0(i)}\prd{y:A_0(i)}(\idtypevar{A})_0(\pairr{i,x},y)\\
& \jdeq \sm{x:A_0(i)}\prd{y:A_0(i)}\id{x}{y}
\end{align*}
for all $i:\Gamma_0$ and, writing $H_{00}(i)$ for $\proj1 H_0(i)$ and
$H_{01}(i)$ for $\lam{y}(\proj2 H_0(i))(y)$, we have $H_1(q)$ of type
\begin{align*}
& \msm{A}{\mprd{\ctxwk{A}{A}}{\idtypevar{A}}}_1(q,H_0(i),H_0(j)) \\
& \jdeq \sm{e:A_1(q,H_{00}(i),H_{00}(j))}\mprd{\ctxwk{A}{A}}{\idtypevar{A}}_1(\pairr{q,e},H_{01}(i),H_{01}(j))\\
& \jdeq \sm{e:A_1(q,H_{00}(i),H_{00}(j))}\prd*{x:A_0(i)}*{y:A_0(j)}{d:A_1(q,x,y)}(\idtypevar{A})_1(\pairr{q,e},d,H_{01}(x),H_{01}(y))\\
& \jdeq \sm{e:A_1(q,H_{00}(i),H_{00}(j))}\prd*{x:A_0(i)}*{y:A_0(j)}{d:A_1(q,x,y)}\id{\trans{\pairr{H_{01}(x),H_{01}(y)}}{e}}{d}.
\end{align*}
By $H_0$, it follows that each $A_0(i)$ is contractible. By the contractibility
of each $A_0(i)$, it follows that the type of $H_1(q)$ is equivalent to
\begin{equation*}
\sm{e:A_1(q,H_{00}(i),H_{00}(j))}\prd{d:A_1(q,H_{00}(i),H_{00}(j))}\id{e}{d}
\end{equation*}
which asserts that $A_1(q,H_{00}(i),H_{00}(j))$ is contractible. By 
the contractibility of each $A_0(i)$, is is equivalent to the assertion
that each $A_1(q,x,y)$ is contractible.
\end{proof}

\begin{rmk}
We address the question whether it is the case that a graph $A$ in context
$\Gamma$ is contractible if and only if $\terms{A}$ is contractible. As a
consequence of \autoref{lem:contractible-graphs}, it is indeed the case
that $\terms{A}$ is contractible whenever $A$ is. However, the converse
does not hold.

To see this, we first construct a counter example to the converse of the
weak function extensionality principle, which states that there is a function
of type
\begin{equation}
\iscontr(\prd{x:X}P(x))\to\prd{x:X}\iscontr(P(x))\label{eq:wfe-converse}
\end{equation}
for any type family $P:X\to\type$. In the proof of \autoref{thm:wfe-converse}, 
we will find a family
$P:X\to\type$ with the property that $\prd{x:X}P(x)$ is contractible and for
which there is a term $x:A$ with $P(x)$ not contractible. Disproving the
converse of the weak function extensionality principle suffices for our
purposes, because if $P:X\to\type$ is a counter example to \autoref{eq:wfe-converse},
then we can take $\Gamma\defeq\pairr{\Gamma_0,\Gamma_1}$ to be given by
$\Gamma_0\defeq X$ and $\Gamma_1(i,j)\defeq\emptyt$ and we take
$A\defeq\pairr{A_0,A_1}$ to be given by $A_0\defeq P$ and $A_1(q,u,v)
\defeq\emptyt$.
\end{rmk}

We define the family $\mathcal{T}:\Sn^1\to\UU$ by
We define $\mathcal{T}(\base)\defeq\mathbf{3}$. To define $\mathcal{T}(\lloop):
\id{\mathbf{3}}{\mathbf{3}}$ we apply the univalence axiom. Hence it suffices to find an
equivalence $\eqv{\mathbf{3}}{\mathbf{3}}$, for which we take the function
$e$ defined by
\begin{equation*}
e(x)\defeq\begin{cases}
0_\mathbf{3} & \text{if }x\jdeq 0_\mathbf{3}\\
2_\mathbf{3} & \text{if }x\jdeq 1_\mathbf{3}\\
1_\mathbf{3} & \text{if }x\jdeq 2_\mathbf{3}.
\end{cases}
\end{equation*}

\begin{lem}
The type $\terms{\mathcal{T}}\defeq\prd{x:\Sn^1}\mathcal{T}(x)$ is contractible.
\end{lem}

\begin{proof}
The type of sections of $\mathcal{T}$ is equivalent to $\sm{u:\mathcal{T}(\base)}\id{e(u)}{u}$.
If we have a term $\pairr{u,\alpha}$ of the latter type, it follows by induction
on $\mathbf{3}$ that $\id{\pairr{u,\alpha}}
{\pairr{0_\mathbf{3},\refl{0_\mathbf{3}}}}$ for all $\pairr{u,\alpha}:
\sm{x:\mathcal{T}(\base)}\id{e(u)}{u}$,
which shows that the type of sections of $\mathcal{T}$ is contractible.
\end{proof}

\begin{thm}\label{thm:wfe-converse}
There is a type family $P:A\to\type$ for which
\begin{equation*}
\neg\Big(\iscontr(\prd{x:A}P(x))\to\prd{x:A}\iscontr(P(x))\Big).
\end{equation*}
\end{thm}

\begin{proof}
The type family of our counter example is $\mathcal{T}$: the fiber $\mathcal{T}(\base)$ isn't contractible.
\end{proof}

\begin{defn}
A graph morphism $f:\ctxhom{\Delta}{\Gamma}$ is an equivalence of graphs when
$\graphcharmap[\Gamma]{f}$ is a contractible graph in the context $\Gamma$.
\end{defn}

\begin{lem}
Let $A$ and $B$ be graphs in a context $\Gamma$ and let $f:A\to B$. The following are equivalent:
\begin{enumerate}
\item $f[i]:A[i]\to B[i]$ is an equivalence of graphs for every term $i:\Gamma$.
\item $\ctxext{\Gamma}{f}:\ctxext{\Gamma}{A}\to\ctxext{\Gamma}{B}$ is an equivalence of graphs.
\item Both $f_0(i)$ and $f_1(q,x,y)$ are always equivalences.
\item $\terms{f}:\terms{\Delta}\to\terms{\Gamma}$ is an equivalence.
\end{enumerate}
\end{lem}

\begin{rmk}
It follows that there is an equivalence
\begin{equation*}
\ctxext{\Gamma}\msm{A}{P}\simeq\ctxext{{\Gamma}{A}}{P}
\end{equation*}
for every family $P$ of graphs over a graph $A$ in context $\Gamma$.
\end{rmk}

\subsubsection{Homotopy levels}
\begin{itemize}
\item A graph $A$ in context $\Gamma$ is of homotopy level $n$ precisely when each
$A_0(i)$ and each $A_1(q,x,y)$ are of homotopy level $n$. 
\item We can name at least three different propositions in the empty context:
\begin{enumerate}
\item $\Gamma_0\defeq\emptyt$.
\item $\Gamma_0\defeq\unit$ and $\Gamma_1(\ttt,\ttt)\defeq\emptyt$.
\item $\Gamma_0\defeq\unit$ and $\Gamma_1(\ttt,\ttt)\defeq\unit$.
\end{enumerate}
Therefore $\mfGraph$ does not satisfy the law of excluded middle.
\end{itemize}



\subsubsection{Univalence for the graph model}
In the other direction, we also obtain a graph morphism $\graphcharmap[\Gamma]{f}:
\ctxhom{\Gamma}{\gobjclass}$ for every graph morphism $f:\ctxhom{\Delta}{\Gamma}$.
In \autoref{graph-object-classifier} we will prove that the maps
$\int_\Gamma:\ctxhom{\Gamma}{\gobjclass}\to\sm{\Delta:\tfGraph}\ctxhom{\Delta}{\Gamma}$
is an equivalence with iverse $\graphcharmapfunc{\Gamma}$. 

\begin{defn}
Let $f:\Delta\to\Gamma$ be a graph morphism. We define the graph morphism
$\graphcharmap[\Gamma]{f}$ by
\begin{align*}
\graphcharmap[\Gamma]{f}_0 & \defeq \lam{i}\hfib{f_0}{i}\\
\graphcharmap[\Gamma]{f}_1(i,j) & \defeq \lam{q}\hfib{f_1(i,j)}{q}.
\end{align*}
\end{defn}


\begin{thm}\label{graph-object-classifier}
Main theorem here.
\end{thm}

\begin{comment}
To describe the object classifier for graphs, we will follow Streicher. Thus
we have to look at presheaves over $I/i$ for each object $i$ of the category
$I\defeq 0{\rightrightarrows}1$ with the morphisms named $s$ and $t$ for source
and target. The category $I/0$ is the terminal category;
the category $I/1$ looks like $\cdot{\rightarrow}\cdot{\leftarrow}\cdot$.
Therefore, we have
\begin{align*}
\type^{\op{(I/0)}} & \eqv{}{\type},\\
\type^{\op{(I/1)}} & \eqv{}{\sm{X,Y,A:\type}(A\to X)\times(A\to Y)}\\
& \eqv{}{\sm{X,Y:\type}X\to Y\to\type}.
\end{align*}
The functors $\type^\op{\Sigma_s}$ and $\type^\op{\Sigma_t}$ are given by
$\pi_1$ and $\pi_2$ respectively. This leads to our following definition
of the object classifier $\gobjclass$:

\begin{defn}
Define $\gobjclass$ to be the graph consisting of
\begin{align*}
\gobjclass_0 & \defeq  \type\\
\gobjclass_1(X,Y) & \defeq  X\to Y\to\type
\end{align*}
and define $\pointed{\gobjclass}$ by
\begin{align*}
(\pointed{\gobjclass})_0 & \defeq  \pointed{\type}\\
(\pointed{\gobjclass})_1(\pairr{X,x},\pairr{Y,y}) & \defeq  \sm{R:X\to Y\to\type}R(x,y)
\end{align*}
There is the obvious forgetful graph morphism $t:\pointed{\gobjclass}\to\gobjclass$,
given by projection on the first coordinate.

For any morphism $f:\Delta\to\Gamma$ of graphs we define a morphism
$\graphcharmap(f):\Gamma\to\gobjclass$ of graphs by
\begin{align*}
\graphcharmap(f)_0(i) & \defeq  \hfiber{f_0}{i}\\
\graphcharmap(f)_1(q,\pairr{u,\alpha},\pairr{v,\beta}) & \defeq  \sm{p:\Delta_1(u,v)}
\id{\trans{\pairr{\alpha,\beta}}{f_1(p)}}{q}
\end{align*}
where $\pairr{u,\alpha}:\graphcharmap(f)_0(i)$ and $\pairr{v,\beta}:\graphcharmap(f)_0(j)$. The
morphism $\graphcharmap(f)$ is called the \emph{characteristic map of $f$}. We obtain a function
\begin{equation*}
\graphcharmap : \big(\sm{\Delta:\tfGraph }\Delta\to\Gamma\big)\to\big(\Gamma\to\gobjclass\big)
\end{equation*}
for every graph $\Gamma$.
\end{defn}

\begin{thm}\label{thm:graph-classifier1}
The function $\graphcharmap$ is an equivalence for any graph $\Gamma$.
\end{thm}

\begin{proof}
We have to find a quasi-inverse
\begin{align*}
\Sigma : (\Gamma\to\gobjclass)\to\big(\sm{\Delta:\tfGraph}\Delta\to\Gamma\big)
\end{align*}
of $\graphcharmap$. Thus, we have to define $\Sigma_0:(\Gamma\to\gobjclass)\to\tfGraph$ and
$\Sigma_1:\prd{P:\Gamma\to\gobjclass}\Sigma_0(P)\to\Gamma$. For $P:\Gamma\to\gobjclass$ we define
\begin{align*}
\Sigma_0(P)_0 & \defeq \sm{i:\Gamma_0}P_0(i)\\
\Sigma_0(P)_1(\pairr{i,u},\pairr{j,v}) & \defeq \sm{q:\Gamma_1(i,j)}P_1(q,u,v)\\
\Sigma_1(P)_0 & \defeq \proj1\\
\Sigma_1(P)_1(\pairr{i,u},\pairr{j,v}) & \defeq \proj1.
\end{align*}
\end{proof}

\begin{thm}\label{conj:graph_classifier2}
For any graph morphism $f:\Delta\to\Gamma$, the diagram
\begin{equation*}
\begin{tikzcd}
\Delta \ar{r}{} \ar{d}[swap]{f} & \pointed{\gobjclass} \ar{d}{t} \\ 
\Gamma \ar{r}[swap]{\graphcharmap(f)} & \gobjclass 
\end{tikzcd}
\end{equation*}
is a pullback square.
\end{thm}
\note{We would like it to be a pb \emph{in} the graph model}
\end{comment}

\begingroup\color{blue}
\subsubsection{The adjunctions $\tfcolim\dashv\Delta\dashv\tflim$}
We define $\Delta:\type\to\tfGraph$ by
\begin{equation*}
\Delta(X)\defeq\pairr{X,\lam{x}{x'}\id{x}{x'}}
\end{equation*}
for $X:\type$. For $A:X\to\type$ we define $\Delta(A):\mftyp(\Delta(X))$ by
\begin{align*}
\Delta(A)_0(x) & \defeq A(x)\\
\Delta(A)_1(p,a,b) & \defeq \id{\trans{p}{a}}{b}.
\end{align*}

\begin{lem}
For any type $X$ and any graph $\Gamma$ there is an equivalence
\begin{equation*}
\eqv{(X\to\terms{\Gamma})}{\terms{\Delta(X)\to\Gamma}}.
\end{equation*}
\end{lem}

\begin{proof}
We have to find functions
\begin{align*}
\varphi & : (X\to\terms{\Gamma})\to\terms{\Delta(X)\to\Gamma}\\
\psi & : \terms{\Delta(X)\to\Gamma}\to X\to\terms{\Gamma}
\end{align*}
which are each others homotopy inverse. To define $\varphi$, let
\end{proof}
\endgroup

\subsection{The model of equifibered families of graphs}
We describe the model $\equifib$ of equifibered families of graphs. The contexts
are still the graphs. However, the types of $\equifib$ are not just the
families of graphs, but the equifibered families of graphs. This will have
far-reaching consequences for the rest of the structure of the model. As an
example, a morphism of graphs will no longer be a pair consisting of a map of
vertices and a map of edges.

\begin{defn}
Let $\Gamma$ be a graph. A type in context $\Gamma$ is a triple 
$\pairr{A_0,A_1,A_2}$ consisting of
\begin{align*}
A_0 & : \Gamma_0\to\type\\
A_1 & : \prd*{i,j:\Gamma_0}{q:\Gamma_1(i,j)}A_0(i)\to A_0(j)\\
A_2 & : \prd*{i,j:\Gamma_0}{q:\Gamma_1(i,j)}\isequiv{A_1(q)}
\end{align*}
A type in context $\Gamma$ is also called an equifibered family over $\Gamma$.
\end{defn}

\begin{defn}
Let $\Gamma$ be a graph. A term of $\Gamma$ is a term of $\Gamma_0$.
\end{defn}

\begin{defn}
Let $A$ be an equifibered family over $\Gamma$. A term $x$ of $A$ is a pair
$\pairr{x_0,x_1}$ consisting of
\begin{align*}
x_0 & : \prd{i:\Gamma_0}A_0(i)\\
x_1 & : \prd*{i,j:\Gamma_0}{q:\Gamma_1(i,j)}\id{A_1(q,x_0(i))}{x_0(j)}
\end{align*}
\end{defn}

\begin{defn}
Suppose that $A$ is an equifibered family over $\Gamma$. We define the graph
$\ctxext{\Gamma}{A}$ to consist of
\begin{align*}
\ctxext{\Gamma}{A}_0 & \defeq\sm{i:\Gamma_0}A_0(i)\\
\ctxext{\Gamma}{A}_1(\pairr{i,x},\pairr{j,y}) & \defeq \sm{q:\Gamma_1(i,j)}\id{A_1(q,x)}{y}.
\end{align*}
\end{defn}

\begin{defn}
Suppose $\Gamma$ and $\Delta$ are graphs. We define the equifibered family
$\ctxwk{\Gamma}{\Delta}$ over $\Gamma$ to consist of
\begin{align*}
\ctxwk{\Gamma}{\Delta}_0(i) & \defeq \Delta\\
\ctxwk{\Gamma}{\Delta}_1(q) & \defeq \idfunc[\Delta]\\
\ctxwk{\Gamma}{\Delta}_2(q) & \defeq \nameless.
\end{align*}
The proof $\ctxwk{\Gamma}{\Delta}_2(q)$ that $\ctxwk{\Gamma}{\Delta}_1(q)$ is an
equivalence is the canonical proof that $\idfunc[\Delta]$ is an equivalence, for
which we don't have a name.
\end{defn}

\begin{defn}
Suppose $A$ and $B$ are equifibered families over $\Gamma$. We define the equifibered
diagram $\ctxwk{A}{B}$ over $\ctxext{\Gamma}{A}$ to consist of
\begin{align*}
\ctxwk{A}{B}_0(\pairr{i,x}) & \defeq B_0(i)\\
\ctxwk{A}{B}_1(\pairr{q,\alpha}) & \defeq B_1(q)\\
\ctxwk{A}{B}_2(\pairr{q,\alpha}) & \defeq B_2(q).
\end{align*}
\end{defn}

\begin{defn}
Suppose $A$ is an equifibered family over $\Gamma$ with $x:A$ and that $P$ is an equifibered
family over $\ctxext{\Gamma}{A}$. Then we define the equifibered family $\subst{x}{P}$
over $\Gamma$ to consist of
\begin{align*}
\pts{\subst{x}{P}}(i) & \defeq \pts{P}(\pts{x}(i))\\
\edg{\subst{x}{P}}(q,u) & \defeq \edg{P}(\pairr{q,\edg{x}(q)},u)\\
\subst{x}{P}_2(q) & \defeq P_2(\pairr{q,\edg{x}(q)}).
\end{align*}
\end{defn}

\begin{defn}
Suppose that $A$ is an equifibered family over $\Gamma$ and that $P$ is an
equifibered family over $\ctxext{\Gamma}{A}$. Then we define the equifibered
family $\msm{A}{P}$ over $\Gamma$ by
\begin{align*}
\pts{\msm{A}{P}}(i) & \defeq \sm{x:\pts{A}(i)}\pts{P}(x)\\
\edg{\msm{A}{P}}(q,\pairr{x,u}) & \defeq \pairr{\edg{A}(q,x),\edg{P}(\pairr{q,\refl{\edg{A}(q,x)}},u)}\\
\msm{A}{P}_2(q,\pairr{x,u}) & \defeq \nameless.
\end{align*}
\end{defn}


\section{Internal models of type theory}
\subsection{Models of type theory without basic constructors}\label{internal-model-contexts}
\begin{defn}\label{defn:premodel}
An internal model $\mfM$ of type theory consists of the following data. 
\begin{enumerate}
\item A type $\tfctx(\mfM)$ of \emph{contexts}.
\item A function $\mftypfunc{\mfM}$ assigning to each context $\Gamma$ of $\mfM$ an internal model
$\mftyp{\mfM}{\Gamma}$ of type theory.
\begin{defn}
The type $\tfctx(\mftyp{\mfM}{\Gamma})$ is denoted by $\mftyp{\mfM}{\Gamma}$. A
term of $\mftyp{\mfM}{\Gamma}$ is called a \emph{type in context $\Gamma$}. To indicate
that $A$ is a type in context $\Gamma$ we also write $\Gamma\vdash A:\mfM$. 
When $P:\mftyp{\mftyp{\mfM}{\Gamma}}{A}$ for some type $A$ in context $\Gamma$, we
also speak of \emph{a family $P$ over $A$ in context $\Gamma$.}
\end{defn}
\item A family $\terms{\blank}:\prd*{\Gamma:\ctx(\mfM)}\mftyp{\mfM}{\Gamma}\to\type$
assigning to every type $A$ in context $\Gamma$ the type $\terms{A}$ of its
terms.
\begin{defn}
When $A$ is a type in context $\Gamma$, we define $\Gamma\vdash x:A$ 
to mean $x:\terms{A}$.
\end{defn}
\item Context extension: a morphism
\begin{equation*}
\tfext^\Gamma:\mftyp{\mfM}{\Gamma}\to\mfM
\end{equation*}
of internal models for every context $\Gamma$.
\begin{defn}
When $A$ is a type in context $\Gamma$, we also denote
the context $\tfext^\Gamma_0(A)$ of $\mfM$ by $\ctxext{\Gamma}{A}$.
We will write
$\ctxext{{\Gamma}{A}}{P}$ for $\ctxext{{\Gamma}{A}}{P}$.
\end{defn}
\item The judgmental equalities:
\begin{align*}
\mftyp{\mftyp{\mfM}{\Gamma}}{A} & \jdeq\mftyp{\mfM}{\ctxext{\Gamma}{A}}\\
\mftyp{\tfext^\Gamma}{A} & \jdeq \modelfont{id}_{\mftyp{\mfM}{\ctxext{\Gamma}{A}}}\\
\terms[{\tfext^\Gamma}]{P} & \jdeq \idfunc[\terms{P}].
\end{align*}
\begin{rmk}
In particular, we will have the judgmental equalities:
\begin{enumerate}
\item When $A$ is a type in context $\Gamma$ we have
\begin{equation*}
\mftyp{\mftyp{\mfM}{\Gamma}}{A}\jdeq\mftyp{\mfM}{\ctxext{\Gamma}{A}},
\end{equation*}
ensuring that a context in the model $\mftyp{\mftyp{\mfM}{\Gamma}}{A}$ is the same thing as a
context in the model $\mftyp{\mfM}{\ctxext{\Gamma}{A}}$.
\item If $Q$ is a family over $P$ where $P$ is a family over $A$ in context $\Gamma$, then
\begin{equation*}
\tfext^{\protect{\mftyp{\mftyp{\mfM}{\Gamma}}{A}}}(P,Q)
\jdeq
\tfext^{\protect{\mftyp{\mfM}{\ctxext{\Gamma}{A}}}}(P,Q)
\end{equation*}
ensuring that twe two possible notion of context extension are the same.
\end{enumerate}
Other judgmental equalities will be required with the ingredients that follow.
We will not list them all.
\end{rmk}
\item Weakening: a morphism
\begin{equation*}
\tfwk^A:\mftyp{\mfM}{\Gamma}\to\mftyp{\mfM}{\ctxext{\Gamma}{A}}
\end{equation*}
of internal models for every type $A$ in context $\Gamma$. When $B$ is a type
in context $\Gamma$, we denote $\tfwk^A(B)$ by $\ctxwk{A}{B}$. 
\item Substitution: a morphism
\begin{equation*}
\tfsubst^x:\mftyp{\mfM}{\ctxext{\Gamma}{A}}\to\mftyp{\mfM}{\Gamma}
\end{equation*}
of internal models for any $x:A$ in context $\Gamma$. When $P$ is a family over $A$ in context
$\Gamma$, we denote $\tfsubst^x(P)$ also by $\subst{x}{P}$. 
\item A judgmental equality
\begin{equation*}
\tfsubst^x\circ\tfwk^A\jdeq\modelfont{id}_{\mftyp{\mfM}{\Gamma}}
\end{equation*}
for any $x:A$ and $B$ in context $\Gamma$.
\item A context $\unit^\mfM:\tfctx(\mfM)$ and a term of type $\isequiv(\ctxext{\unit^\mfM}{\blank})$. We will denote
this equivalence by $e_\unit$. The context $\unit^\mfM$ is also
called the \emph{empty context}.
\begin{defn}
For any context $\Gamma$, type $\terms{\Gamma}$ is defined to mean
$\terms{e_\unit^{-1}(\Gamma)}$. 
\end{defn}
\item A section $\unit^{\blank}:\prd{\Gamma:\ctx(\mfM)}\mftyp{\mfM}{\Gamma}$ assigning
a type $\unit^\Gamma$ in context $\Gamma$ to every context $\Gamma$ and
an identification $\alpha_\unit(\Gamma):\id{\ctxext{\Gamma}{\unit^\Gamma}}{\Gamma}$
for every context $\Gamma$. We also require that there is an identification
$\id{\trans{\alpha_\unit(\Gamma)}{\ctxwk{\unit^\Gamma}{A}}}{A}$ for every
type $A$ in context $\Gamma$.
\item For any type $A$ in context $\Gamma$, a term $\idfunc[A]:\terms{\ctxwk{A}{A}}$
\end{enumerate}
\begin{flushright}
\textsl{End of \autoref{defn:premodel}.}
\end{flushright}
\end{defn}

\begin{defn}
In an internal model $\mfM$ we define
\begin{equation*}
\ctxhom{\Delta}{\Gamma}\defeq \terms{\ctxwk{e_\unit^{-1}(\Delta)}{e_\unit^{-1}(\Gamma)}}
\end{equation*}
\end{defn}

\subsection{Morphisms of internal models}
\begin{defn}\label{defn:premodel-morphism}
A morphism $f:\mfM\to \mfN$ of internal models consists of
\begin{enumerate}
\item a function $\ctx(f):\ctx(\mfM)\to\ctx(\mfN)$. The function $\ctx(f)$ is also
denoted by $f_0$.
\item a morphism $\mftyp{f}{\Gamma}:\mftyp{\mfM}{\Gamma}\to\mftyp{\mfN}{f_0(\Gamma)}$ of internal models for every
$\Gamma:\ctx(M)$.
\item the judgmental equality
\begin{equation*}
\mftyp{\mftyp{f}{\Gamma}}{A}\jdeq\mftyp{f}{\ctxext{\Gamma}{A}}
\end{equation*}
\item a function $\terms[f]{A}:\terms[\mfM]{A}\to\terms[\mfN]{\mftyp{f}{\Gamma}_0(A)}$ for
every type $A$ in context $\Gamma$. 
\item preservation of context extension: 
\begin{align*}
\alpha^f_0 & :\id{f\circ\tfext^\Gamma}{\tfext^{f_0(\Gamma)}\circ\mftyp{f}{\Gamma}}\\
\alpha^f_1 & :\id{\mftyp{f}{\Gamma}_0\circ\tfext^A}{\tfext^{\mftyp{f}{\Gamma}_0(A)}\circ\mftyp{f}{\ctxext{\Gamma}{A}}}.
\end{align*}
\item preservation of weakening: 
\begin{align*}
\beta^f_0 & :\id{\mftyp{f}{\Gamma}\circ\tfwk^\Gamma}{\tfwk^{f_0(\Gamma)}\circ f}\\
\beta^f_1 & :\id{\mftyp{f}{\ctxext{\Gamma}{A}}\circ\tfwk^A}{\tfwk^\protect{\mftyp{f}{\Gamma}_0(A)}\circ\mftyp{f}{\Gamma}}.
\end{align*}
\item preservation of substitution: 
\begin{equation*}
\gamma^f:\id{\mftyp{f}\Gamma_0\circ\tfsubst^x}{\tfsubst^\protect{\terms[f]{A}(x)}\circ\mftyp{f}{\ctxext{\Gamma}{A}}}.
\end{equation*}
\end{enumerate}
\begin{flushright}
\textsl{End of \autoref{defn:premodel-morphism}.}
\end{flushright}
\end{defn}

\begin{defn}
Suppose that $f:\mfM\to\mfN$ and $g:\mfN\to\mfN'$ are morphisms of internal models.
We define the composition $g\circ f:\mfM\to\mfN'$ to be the morphism given by
\begin{enumerate}
\item $(g\circ f)_0\defeq g_0\circ f_0$
\item $\mftypfunc{g\circ f}(\Gamma)\defeq\mftypfunc{g}(f_0(\Gamma))\circ\mftypfunc{f}(\Gamma)$
\item $\terms[g\circ f]{A}\defeq\terms[g]{\mftyp{f}{\Gamma}_0(A)}\circ\terms[f]{A}$.
\item a definition of $\alpha^{g\circ f}_0$ and $\alpha^{g\circ f}_1$.
\item a definition of $\beta^{g\circ f}_0$ and $\beta^{f\circ f}_1$.
\item a definition of $\gamma^{g\circ f}$. 
\end{enumerate}
\end{defn}

\begin{rmk}
The requirement that a morphism of models acts on terms is reminiscent of the requirement
that a functor acts on morphisms. In this fasion, the requirement that a morphism of
models preserves substitution is the counterpart of a functor preserving composition.
\end{rmk}
Context extension, weakening and substitution are required to be such morphims
of models. Thus they must be extended for this purpose.

\begin{description}
\item[Context extension] We will extend context extension to a morphism 
$\tfext(\Gamma):\mftyp{\mfM}{\Gamma}\to M$. Thus, we already have
$\tfext(\Gamma)_0\defeq\ctxext{\Gamma}{\blank}$. We also require 
\begin{align*}
\mftyp{\tfext(\Gamma)}{A} & : \mftyp{\mftyp{\mfM}{\Gamma}}{A}\to \mftyp{\mfM}{\ctxext{\Gamma}{A}}\\
\terms[\tfext(\Gamma)]{P} & : \terms[\mftyp{\mfM}{\Gamma}]{P}\to\terms[\mfM]{\mftyp{\tfext(\Gamma)}{A}_0(P)}
\end{align*}
For both of these we take the identity function.
We require furthermore that context extension preserves context extension,
weakening and substitution:
\begin{enumerate}
\item Extension preserves extension: for every family $P$ over $A$ in context
$\Gamma$ an identification $\id{\ctxext{{\Gamma}{A}}{P}}{\ctxext{\Gamma}{{A}{P}}}$.
\item Extension preserves weakening:
\item Extension preserves substitution:
Because context extension is the identity on the levels of types and terms,
the rules are easier.
\end{enumerate}
\item[Weakening] is a morphism $\tfwk:\mftyp{\mfM}{\Gamma}\to\mftyp{\mfM}{\ctxext{\Gamma}{A}}$, so we must have
\begin{align*}
\tfwk^A_0 & : \mftyp{\mfM}{\Gamma}\to\mftyp{\mfM}{\ctxext{\Gamma}{A}}
\intertext{which is denoted by $\ctxwk{A}{\blank}$,}
\mftypfunc{\tfwk^A} & : \mftyp{\mfM}{\ctxext{\Gamma}{B}}\to\mftyp{\mfM}{\ctxext{{\Gamma}{A}}{\ctxwk{A}{B}}}\\
\terms[\tfwk^A]{B} & : \terms[\mftyp{\mfM}{\Gamma}]{B}\to\terms[\mftyp{\mfM}{\ctxext{\Gamma}{A}}]{\ctxwk{A}{B}}
\end{align*}
for a term $y:B$, the term $\terms[{\tfwk^A}]{B}(y)$ is the constant map from
$A$ to $B$, assigning $y$ to every term of $A$.
\item[Substitution] is a morphism $\tfsubst^x:\mftyp{\mfM}{\ctxext{\Gamma}{A}}
\to\mftyp{\mfM}{\Gamma}$ for every $x:A$ in context $\Gamma$, so we must have
\begin{align*}
\tfsubst^x_0 & : \mftyp{\mfM}{\ctxext{\Gamma}{A}}\to\mftyp{\mfM}{\Gamma}
\intertext{which is denoted by $\subst{x}{\blank}$,}
\mftyp{\tfsubst^x}{P} & : \mftyp{\mfM}{\ctxext{{\Gamma}{A}}{P}}\to\mftyp{\mfM}{\ctxext{\Gamma}{\subst{x}{P}}}\\
\terms[\tfsubst^x]{P} & : \terms{P}\to\terms{\subst{x}{P}}
\end{align*}
The function $\terms[\tfsubst^x]{P}$ is usually denoted by $\tfev(\blank,x)$. 
\end{description}

\subsection{The basic type constructors in internal models}

\begin{defn}
\begin{enumerate}
\item A type $\mprd{A}{P}$ for every type $A$ in context $\Gamma$ and every type
$P$ in context $\ctxext{\Gamma}{A}$.
\begin{defn}
Suppose $A$ and $B$ are types in context $\Gamma$. We define
\begin{align*}
A\to B & \defeq\mprd{A}{\ctxwk{A}{B}}.
\intertext{For any two contexts $\Delta$ and $\Gamma$, we define}
\Delta\to\Gamma & \defeq e_\unit^{-1}(\Delta)\to e_\unit^{-1}(\Gamma)
\end{align*}
and furthermore we define $\ctxhom{\Delta}{\Gamma}\defeq\terms{\Delta\to\Gamma}$.
\end{defn}
\item An equivalence $\lambda:\eqv{\terms{P}}{\terms{\mprd{A}{P}}}$ for every
family $P$ over $A$ in context $\Gamma$. When $\ctxext{\Gamma}{A}\vdash 
u:P$, we call $\lambda(u)$ the \emph{$\lambda$-abstraction of $u$.}
\begin{rmk}
Note that we get an equivalence $\eqv{\terms{\unit^\Gamma\to A}}{\terms{A}}$ 
for every type $A$ in context $\Gamma$.
\end{rmk}
\begin{rmk}
Thus we see that $\eqv{\terms{A\to B}}{\terms{\ctxwk{A}{B}}}$
by $\lambda$-abstraction. 
\end{rmk}
%\item A function $\tfev:\terms{\mprd{A}{P}}\to\prd{x:\terms{A}}\terms{\subst{x}{P}}$.
\item A term $\pi^A:\ctxext{\Gamma}{A}\to \Gamma$ for every type $A$ in context
$\Gamma$.
\item A term $\iota^x:\ctxext{\Gamma}{\subst{x}{P}}\to\ctxext{{\Gamma}{A}}{P}$
for every family $P$ of types over $A$ in context $\Gamma$ and every
term $x:A$.
\item A type $A[f]$ in context $\Delta$ for every $f:\ctxhom{\Delta}{\Gamma}$ 
and every type $A$ in context $\Gamma$. {\color{blue}Could this be defined
in terms of the substitution $\subst{x}{P}$ we already have? It seems so. If this is
indeed the case we need either some identifications or we could just omit
this part of the definition.}
\item an identification $\id{B[\pi^A]}{\ctxwk{A}{B}}$ for any two types $A$
and $B$ in context $\Gamma$.
\item A type $\msm{A}{P}$ in context $\Gamma$ for every family $P$ over $A$
in context $\Gamma$, with an equivalence $\pairr{\blank,\blank}:\eqv{\sm{x:\terms{A}}\terms{\subst{x}{P}}}
{\terms{\msm{A}{P}}}$.
\item A family $\idtypevar{A}$ over $\ctxwk{A}{A}$ in context $\ctxext{\Gamma}{A}$ for
every type $A$ in context $\Gamma$ and a term of type.
\begin{rmk}
If $A$ and $B$ are types in context $\Gamma$, we may denote the type 
$\msm{A}{\ctxwk{A}{B}}$ by $A\times B$. By the end of the current definition
there will be an identification $\id{\ctxext{{\Gamma}{A}}{P}}
{\ctxext{\Gamma}{\msm{A}{P}}}$.

There is a term $\delta:\ctxhom{\ctxext{\Gamma}{A}}{\ctxext{{\Gamma}{A}}{\ctxwk{A}{A}}}$
defined by...
\end{rmk}
\item For any family $Q$ over $\idtypevar{A}$ in context $\ctxext{{\Gamma}{A}}{\ctxwk{A}{A}}$ an equivalence
$J:\eqv{\terms{Q}}{}$
\end{enumerate}
\end{defn}

\begingroup
\color{blue}
\subsubsection*{More desiderata}
Some of which hopefully follow from more elegant or general rules, but
this list is to keep them in mind:
\begin{enumerate}
\item $\id{e_\unit^{-1}(\Gamma.A)}{\msm{e_\unit^{-1}(\Gamma)}{\trans{(\eta_\Gamma)}{A}}}$, where
$\eta_\Gamma:\id{\Gamma}{e_\unit(e_\unit^{-1}(\Gamma))}$ is the unit of the
equivalence $e_\unit$.  
\item $\id{(\msm{A}{P}\to B)}{(P\to \ctxwk{A}{B})}$, expressing that $\Sigma$ is
left adjoint to weakening.
\item $\id{(\ctxwk{A}{B}\to P)}{(B\to\mprd{A}{P})}$ expressing that $\Pi$ is
right adjoint to weakening.
\item a term $\Gamma\vdash\idfunc[A]:A\to A$ for every type $A$ in context $\Gamma$.
\item a context $\UU$ for the universe.
\item an identification $\id{\pi^A\circ \pi^P\circ\iota^x}{\pi^{\subst{x}{P}}}$ for every
family $P$ of types over $A$ in context $\Gamma$ and every term $x:A$. In other
words, the diagram
\begin{equation*}
\begin{tikzcd}
\ctxext{{\Gamma}{A}}{P} \ar{r}{\pi^P} & \ctxext{\Gamma}{A} \ar{d}{\pi^A}\\
\ctxext{\Gamma}{\subst{x}{P}} \ar{u}{\iota^x} \ar{r}[swap]{\pi^{\subst{x}{P}}} & \Gamma
\end{tikzcd}
\end{equation*}
should commute.
\end{enumerate}
\endgroup

\begin{desiderata}
We should have a theorem stating that our internal model indeed interprets
the rules of type theory.
\end{desiderata}

\section{Weak $\omega$-groupoids}

\subsection{Trivial cofibrations and weak equivalences of types}
We describe a relation between types that expresses when they are weakly equivalent.
Weak equivalence is introduced because we need a weaker notion of judgmental 
equality which also makes sense when identity types are not present, since that
would allow us to state that context extension, weakening and substitution
commute with each other.

A term $f:\ctxwk{A}{B}$ is a trivial cofibration if it has the
property that for any fibration $Q$ over $B$,
to find a section of $Q$ it suffices to find a section of the fibration
$f^\ast Q$ over $A$. In our type theoretical setting, the rôle of fibrations
is played by families, the rôle of the function type $A\to B$ is played by
$\ctxwk{A}{B}$ and our version of the pullback $f^\ast Q$ is $\subst{f}{\ctxwk{A}{Q}}$.

\begin{defn}
Let $f$ be a term of $\ctxwk{A}{B}$ in context $\Gamma$.
\begin{enumerate}
\item For a family $Q$ over $B$ in context $\Gamma$ we define $f^\ast Q\jdeq\subst{f}{\ctxwk{A}{Q}}$.
\item For a term $g$ of $Q$ in context $\ctxext{\Gamma}{B}$ we define $f^\ast g\jdeq\subst{f}{\ctxwk{A}{g}}$.
\end{enumerate} 
\end{defn}
\begin{rmk}
The type $f^\ast(\ctxwk{B}{C})$ in context $\ctxext{\Gamma}{A}$ is can be viewed as the
type of functions from $A$ to $C$ which factor through $f$. Thus there should be
a function from $f^\ast(\ctxwk{B}{C})$ to $\ctxwk{A}{C}$.
\end{rmk}

\begin{rmk}
Every term $\jterm{\ctxext{\Gamma}{A}}{\ctxwk{A}{B}}{f}$ allows us to infer the following:
\begin{equation*}
\inference{\jtype{\ctxext{\Gamma}{B}}{Q}}{\jtype{\ctxext{\Gamma}{A}}{f^\ast Q}}
\qquad
\inference{\jterm{\ctxext{\Gamma}{B}}{Q}{g}}{\jterm{\ctxext{\Gamma}{A}}{f^\ast Q}{f^\ast g}}
\end{equation*}
\end{rmk}


\begin{defn}
A term $f:\ctxwk{A}{B}$ in context $\ctxext{\Gamma}{A}$ is said to be a trivial
cofibration if we can infer
\begin{equation*}
\inference{\jterm{\ctxext{\Gamma}{A}}{f^\ast Q}{t}}{\jterm{\ctxext{\Gamma}{B}}{Q}{\tilde{t}}}\qquad
\inference{\jterm{\ctxext{\Gamma}{A}}{f^\ast Q}{t}}{\jtermeq{\ctxext{\Gamma}{A}}{f^\ast Q}{f^\ast \tilde{t}}{t}}
\end{equation*}
{\color{red}This statement should be reformulated so that it only involves a single judgment...
but I don't see directly how to do that.}
\end{defn}

We have the following theorem in the type theory of \cite{TheBook}, which supports
our claim that we may indeed speak of a trivial cofibration. 

\begin{thm}
Suppose $f:A\to B$ is a function. Then $f$ is an equivalence if and only if
for every $Q:B\to\type$ and every $g:\prd{x:A}Q(f(x))$ there is a section
$h:\prd{y:B}Q(b)$ with the property that $h\circ f\htpy g$. 
\end{thm}

\begin{proof}
We can first take $Q$ to be the constant family $\lam{y}A$. Furthermore, we may
take $g\defeq\idfunc[A]$. Then we get a term of type
\begin{equation*}
\sm{h:B\to A}h\circ f\htpy \idfunc[A],
\end{equation*}
i.e.~we get a left inverse $h$ for $f$. To show that $h$ is also a right inverse
of $f$, let $Q$ be the family $\lam{y}\id{f(h(y))}{y}$. To find a section of
$Q$, which is the homotopy we aim for, it suffices to find a section of
$Q\circ f$. In other words, we have to show that $\id{f(h(f(x)))}{f(x)}$ for
every $x:A$. This follows from the fact that $h$ is a left inverse for $f$.

The reverse direction is immediate.
\end{proof}

Another approach would be to define $f:\ctxwk{A}{B}$ to be left invertible
if there is a term $\ctxext{{\Gamma}{A}}{P}$

\subsection{Some possibilities}

A weak omega groupoid is a model of type theory with $\Sigma$ and $\idtypevar{}$
in which there is a term of $\terms{A}$ for every type $A$ in context $\Gamma$.
This should work when univalence is around (so that functions can be replaced
by families).

We could also state that for every term $f:\terms{\ctxwk{A}{B}}$ and every
family $Q:\tftyp{\mfM}{\ctxext{\Gamma}{B}}$ there is a function
$\terms{\subst{f}{\ctxwk{A}{Q}}}\to\terms{Q}$ (asserting that $f$ is a trival
cofibration). This should always give that each $f$ is an equivalence and
this should be equivalent to the previous condition when univalence is around.

Another option arises when we see the identity type as an operation on a certain class of
terms. Usually, $\idtypevar{}$ is defined using the class of identity functions.
Let's take the class of all terms instead:


\bibliographystyle{hplain}
%\phantomsection\addcontentsline{toc}{section}{References}
\bibliography{refs}

\end{document}
