\documentclass{article}

%%%%%%%%%%%%%%%%%%%%%%%%%%%%%%%%%%%%%%%%%%%%%%%%%%%%%%%%%%%%%%%%%%%%%%%%%%%%%%%%
%%%% PACKAGES

\usepackage[utf8]{inputenc}
\usepackage[english]{babel}

%%%% Spicing up the document
\usepackage{mathpazo}
\usepackage[scaled=0.95]{helvet}
\usepackage{courier}
\linespread{1.05} % Palatino looks better with this
\usepackage{microtype}

\usepackage{fancyhdr} % To set headers and footers
\usepackage{enumitem,mathtools,xspace,xcolor}
\usepackage{comment}
\usepackage{ifthen}
\usepackage{pifont}
\newcommand{\cmark}{\ding{51}\xspace}
\newcommand{\xmark}{\ding{55}\xspace}

\usepackage{graphicx}
\usepackage{tikz-cd}
\usepackage{tikz}
\usetikzlibrary{decorations.pathmorphing}
\usepackage[inference]{semantic}
\usepackage{booktabs}

\usepackage[hyphens]{url} % This package has to be loaded *before* hyperref
\usepackage[pagebackref,colorlinks,citecolor=darkgreen,linkcolor=darkgreen,unicode]{hyperref}
\definecolor{darkgreen}{rgb}{0,0.45,0}

% For some reason the following can't be above hyperref...
\usepackage{amssymb,amsmath,amsthm,stmaryrd,mathrsfs,wasysym}
\usepackage{aliascnt}
\usepackage[capitalize]{cleveref}

% The braket macro shouldn't be necessary
\usepackage{braket} % used for \setof{ ... } macro

%%%%%%%%%%%%%%%%%%%%%%%%%%%%%%%%%%%%%%%%%%%%%%%%%%%%%%%%%%%%%%%%%%%%%%%%%%%%%%%%
%% To include references in TOC we should use this package rather than a hack.
\usepackage{tocbibind}
%\usepackage{etoolbox}           % get \apptocmd
%\apptocmd{\thebibliography}{\addcontentsline{toc}{section}{References}}{}{} % tell bibliography to get itself into the table of contents


\begin{comment}
%%%% Header and footers
\pagestyle{fancyplain}
\setlength{\headheight}{15pt}
\renewcommand{\chaptermark}[1]{\markboth{\textsc{Chapter \thechapter. #1}}{}}
\renewcommand{\sectionmark}[1]{\markright{\textsc{\thesection\ #1}}}
\end{comment}

% TOC depth
\setcounter{tocdepth}{3}

\lhead[\fancyplain{}{{\thepage}}]%
      {\fancyplain{}{\nouppercase{\rightmark}}}
\rhead[\fancyplain{}{\nouppercase{\leftmark}}]%
      {\fancyplain{}{\thepage}}
\cfoot{\textsc{\footnotesize [Draft of \today]}}
\lfoot[]{}
\rfoot[]{}

%%%%%%%%%%%%%%%%%%%%%%%%%%%%%%%%%%%%%%%%%%%%%%%%%%%%%%%%%%%%%%%%%%%%%%%%%%%%%%%%
%%%% We mostly use the macros of the book, to keep notations
%%%% and conventions the same. Recall that when the macros file
%%%% is updated, we need to comment the lines containing the
%%%% string `[chapter]` since our article is not a book.
%%%%
%%%% Instructions for updating the macros.tex file:
%%%% - fetch the latest macros.tex file from the HoTT/book git repository.
%%%% - comment all lines containing "[chapter]" because this is not a book.
%%%% - comment the definition of pbcorner because the xypic package is not used.
%%%%
%%%% MACROS FOR NOTATION %%%%
% Use these for any notation where there are multiple options.

%%% Notes and exercise sections
\makeatletter
\newcommand{\sectionNotes}{\phantomsection\section*{Notes}\addcontentsline{toc}{section}{Notes}\markright{\textsc{\@chapapp{} \thechapter{} Notes}}}
\newcommand{\sectionExercises}[1]{\phantomsection\section*{Exercises}\addcontentsline{toc}{section}{Exercises}\markright{\textsc{\@chapapp{} \thechapter{} Exercises}}}
\makeatother

%%% Definitional equality (used infix) %%%
\newcommand{\jdeq}{\equiv}      % An equality judgment
\let\judgeq\jdeq
%\newcommand{\defeq}{\coloneqq}  % An equality currently being defined
\newcommand{\defeq}{\vcentcolon\equiv}  % A judgmental equality currently being defined

%%% Term being defined
\newcommand{\define}[1]{\textbf{#1}}

%%% Vec (for example)

\newcommand{\Vect}{\ensuremath{\mathsf{Vec}}}
\newcommand{\Fin}{\ensuremath{\mathsf{Fin}}}
\newcommand{\fmax}{\ensuremath{\mathsf{fmax}}}
\newcommand{\seq}[1]{\langle #1\rangle}

%%% Dependent products %%%
\def\prdsym{\textstyle\prod}
%% Call the macro like \prd{x,y:A}{p:x=y} with any number of
%% arguments.  Make sure that whatever comes *after* the call doesn't
%% begin with an open-brace, or it will be parsed as another argument.
\makeatletter
% Currently the macro is configured to produce
%     {\textstyle\prod}(x:A) \; {\textstyle\prod}(y:B),\ 
% in display-math mode, and
%     \prod_{(x:A)} \prod_{y:B}
% in text-math mode.
\def\prd#1{\@ifnextchar\bgroup{\prd@parens{#1}}{\@ifnextchar\sm{\prd@parens{#1}\@eatsm}{\prd@noparens{#1}}}}
\def\prd@parens#1{\@ifnextchar\bgroup%
  {\mathchoice{\@dprd{#1}}{\@tprd{#1}}{\@tprd{#1}}{\@tprd{#1}}\prd@parens}%
  {\@ifnextchar\sm%
    {\mathchoice{\@dprd{#1}}{\@tprd{#1}}{\@tprd{#1}}{\@tprd{#1}}\@eatsm}%
    {\mathchoice{\@dprd{#1}}{\@tprd{#1}}{\@tprd{#1}}{\@tprd{#1}}}}}
\def\@eatsm\sm{\sm@parens}
\def\prd@noparens#1{\mathchoice{\@dprd@noparens{#1}}{\@tprd{#1}}{\@tprd{#1}}{\@tprd{#1}}}
% Helper macros for three styles
\def\lprd#1{\@ifnextchar\bgroup{\@lprd{#1}\lprd}{\@@lprd{#1}}}
\def\@lprd#1{\mathchoice{{\textstyle\prod}}{\prod}{\prod}{\prod}({\textstyle #1})\;}
\def\@@lprd#1{\mathchoice{{\textstyle\prod}}{\prod}{\prod}{\prod}({\textstyle #1}),\ }
\def\tprd#1{\@tprd{#1}\@ifnextchar\bgroup{\tprd}{}}
\def\@tprd#1{\mathchoice{{\textstyle\prod_{(#1)}}}{\prod_{(#1)}}{\prod_{(#1)}}{\prod_{(#1)}}}
\def\dprd#1{\@dprd{#1}\@ifnextchar\bgroup{\dprd}{}}
\def\@dprd#1{\prod_{(#1)}\,}
\def\@dprd@noparens#1{\prod_{#1}\,}

%%% Lambda abstractions.
% Each variable being abstracted over is a separate argument.  If
% there is more than one such argument, they *must* be enclosed in
% braces.  Arguments can be untyped, as in \lam{x}{y}, or typed with a
% colon, as in \lam{x:A}{y:B}. In the latter case, the colons are
% automatically noticed and (with current implementation) the space
% around the colon is reduced.  You can even give more than one variable
% the same type, as in \lam{x,y:A}.
\def\lam#1{{\lambda}\@lamarg#1:\@endlamarg\@ifnextchar\bgroup{.\,\lam}{.\,}}
\def\@lamarg#1:#2\@endlamarg{\if\relax\detokenize{#2}\relax #1\else\@lamvar{\@lameatcolon#2},#1\@endlamvar\fi}
\def\@lamvar#1,#2\@endlamvar{(#2\,{:}\,#1)}
% \def\@lamvar#1,#2{{#2}^{#1}\@ifnextchar,{.\,{\lambda}\@lamvar{#1}}{\let\@endlamvar\relax}}
\def\@lameatcolon#1:{#1}
\let\lamt\lam
% This version silently eats any typing annotation.
\def\lamu#1{{\lambda}\@lamuarg#1:\@endlamuarg\@ifnextchar\bgroup{.\,\lamu}{.\,}}
\def\@lamuarg#1:#2\@endlamuarg{#1}

%%% Dependent products written with \forall, in the same style
\def\fall#1{\forall (#1)\@ifnextchar\bgroup{.\,\fall}{.\,}}

%%% Existential quantifier %%%
\def\exis#1{\exists (#1)\@ifnextchar\bgroup{.\,\exis}{.\,}}

%%% Dependent sums %%%
\def\smsym{\textstyle\sum}
% Use in the same way as \prd
\def\sm#1{\@ifnextchar\bgroup{\sm@parens{#1}}{\@ifnextchar\prd{\sm@parens{#1}\@eatprd}{\sm@noparens{#1}}}}
\def\sm@parens#1{\@ifnextchar\bgroup%
  {\mathchoice{\@dsm{#1}}{\@tsm{#1}}{\@tsm{#1}}{\@tsm{#1}}\sm@parens}%
  {\@ifnextchar\prd%
    {\mathchoice{\@dsm{#1}}{\@tsm{#1}}{\@tsm{#1}}{\@tsm{#1}}\@eatprd}%
    {\mathchoice{\@dsm{#1}}{\@tsm{#1}}{\@tsm{#1}}{\@tsm{#1}}}}}
\def\@eatprd\prd{\prd@parens}
\def\sm@noparens#1{\mathchoice{\@dsm@noparens{#1}}{\@tsm{#1}}{\@tsm{#1}}{\@tsm{#1}}}
\def\lsm#1{\@ifnextchar\bgroup{\@lsm{#1}\lsm}{\@@lsm{#1}}}
\def\@lsm#1{\mathchoice{{\textstyle\sum}}{\sum}{\sum}{\sum}({\textstyle #1})\;}
\def\@@lsm#1{\mathchoice{{\textstyle\sum}}{\sum}{\sum}{\sum}({\textstyle #1}),\ }
\def\tsm#1{\@tsm{#1}\@ifnextchar\bgroup{\tsm}{}}
\def\@tsm#1{\mathchoice{{\textstyle\sum_{(#1)}}}{\sum_{(#1)}}{\sum_{(#1)}}{\sum_{(#1)}}}
\def\dsm#1{\@dsm{#1}\@ifnextchar\bgroup{\dsm}{}}
\def\@dsm#1{\sum_{(#1)}\,}
\def\@dsm@noparens#1{\sum_{#1}\,}

%%% W-types
\def\wtypesym{{\mathsf{W}}}
\def\wtype#1{\@ifnextchar\bgroup%
  {\mathchoice{\@twtype{#1}}{\@twtype{#1}}{\@twtype{#1}}{\@twtype{#1}}\wtype}%
  {\mathchoice{\@twtype{#1}}{\@twtype{#1}}{\@twtype{#1}}{\@twtype{#1}}}}
\def\lwtype#1{\@ifnextchar\bgroup{\@lwtype{#1}\lwtype}{\@@lwtype{#1}}}
\def\@lwtype#1{\mathchoice{{\textstyle\mathsf{W}}}{\mathsf{W}}{\mathsf{W}}{\mathsf{W}}({\textstyle #1})\;}
\def\@@lwtype#1{\mathchoice{{\textstyle\mathsf{W}}}{\mathsf{W}}{\mathsf{W}}{\mathsf{W}}({\textstyle #1}),\ }
\def\twtype#1{\@twtype{#1}\@ifnextchar\bgroup{\twtype}{}}
\def\@twtype#1{\mathchoice{{\textstyle\mathsf{W}_{(#1)}}}{\mathsf{W}_{(#1)}}{\mathsf{W}_{(#1)}}{\mathsf{W}_{(#1)}}}
\def\dwtype#1{\@dwtype{#1}\@ifnextchar\bgroup{\dwtype}{}}
\def\@dwtype#1{\mathsf{W}_{(#1)}\,}

\newcommand{\suppsym}{{\mathsf{sup}}}
\newcommand{\supp}{\ensuremath\suppsym\xspace}

\def\wtypeh#1{\@ifnextchar\bgroup%
  {\mathchoice{\@lwtypeh{#1}}{\@twtypeh{#1}}{\@twtypeh{#1}}{\@twtypeh{#1}}\wtypeh}%
  {\mathchoice{\@@lwtypeh{#1}}{\@twtypeh{#1}}{\@twtypeh{#1}}{\@twtypeh{#1}}}}
\def\lwtypeh#1{\@ifnextchar\bgroup{\@lwtypeh{#1}\lwtypeh}{\@@lwtypeh{#1}}}
\def\@lwtypeh#1{\mathchoice{{\textstyle\mathsf{W}^h}}{\mathsf{W}^h}{\mathsf{W}^h}{\mathsf{W}^h}({\textstyle #1})\;}
\def\@@lwtypeh#1{\mathchoice{{\textstyle\mathsf{W}^h}}{\mathsf{W}^h}{\mathsf{W}^h}{\mathsf{W}^h}({\textstyle #1}),\ }
\def\twtypeh#1{\@twtypeh{#1}\@ifnextchar\bgroup{\twtypeh}{}}
\def\@twtypeh#1{\mathchoice{{\textstyle\mathsf{W}^h_{(#1)}}}{\mathsf{W}^h_{(#1)}}{\mathsf{W}^h_{(#1)}}{\mathsf{W}^h_{(#1)}}}
\def\dwtypeh#1{\@dwtypeh{#1}\@ifnextchar\bgroup{\dwtypeh}{}}
\def\@dwtypeh#1{\mathsf{W}^h_{(#1)}\,}


\makeatother

% Other notations related to dependent sums
\let\setof\Set    % from package 'braket', write \setof{ x:A | P(x) }.
\newcommand{\pair}{\ensuremath{\mathsf{pair}}\xspace}
\newcommand{\tup}[2]{(#1,#2)}
\newcommand{\proj}[1]{\ensuremath{\mathsf{pr}_{#1}}\xspace}
\newcommand{\fst}{\ensuremath{\proj1}\xspace}
\newcommand{\snd}{\ensuremath{\proj2}\xspace}
\newcommand{\ac}{\ensuremath{\mathsf{ac}}\xspace} % not needed in symbol index
\newcommand{\un}{\ensuremath{\mathsf{upun}}\xspace} % not needed in symbol index, uniqueness principle for unit type

%%% recursor and induction
\newcommand{\rec}[1]{\mathsf{rec}_{#1}}
\newcommand{\ind}[1]{\mathsf{ind}_{#1}}
\newcommand{\indid}[1]{\ind{=_{#1}}} % (Martin-Lof) path induction principle for identity types
\newcommand{\indidb}[1]{\ind{=_{#1}}'} % (Paulin-Mohring) based path induction principle for identity types 

%%% the uniqueness principle for product types, formerly called surjective pairing and named \spr:
\newcommand{\uppt}{\ensuremath{\mathsf{uppt}}\xspace}

% Paths in pairs
\newcommand{\pairpath}{\ensuremath{\mathsf{pair}^{\mathord{=}}}\xspace}
% \newcommand{\projpath}[1]{\proj{#1}^{\mathord{=}}}
\newcommand{\projpath}[1]{\ensuremath{\apfunc{\proj{#1}}}\xspace}

%%% For quotients %%%
%\newcommand{\pairr}[1]{{\langle #1\rangle}}
\newcommand{\pairr}[1]{{\mathopen{}(#1)\mathclose{}}}
\newcommand{\Pairr}[1]{{\mathopen{}\left(#1\right)\mathclose{}}}

% \newcommand{\type}{\ensuremath{\mathsf{Type}}} % this command is overridden below, so it's commented out
\newcommand{\im}{\ensuremath{\mathsf{im}}} % the image

%%% 2D path operations
\newcommand{\leftwhisker}{\mathbin{{\ct}_{\ell}}}
\newcommand{\rightwhisker}{\mathbin{{\ct}_{r}}}
\newcommand{\hct}{\star}

%%% modalities %%%
\newcommand{\modal}{\ensuremath{\ocircle}}
\let\reflect\modal
\newcommand{\modaltype}{\ensuremath{\type_\modal}}
% \newcommand{\ism}[1]{\ensuremath{\mathsf{is}_{#1}}}
% \newcommand{\ismodal}{\ism{\modal}}
% \newcommand{\existsmodal}{\ensuremath{{\exists}_{\modal}}}
% \newcommand{\existsmodalunique}{\ensuremath{{\exists!}_{\modal}}}
% \newcommand{\modalfunc}{\textsf{\modal-fun}}
% \newcommand{\Ecirc}{\ensuremath{\mathsf{E}_\modal}}
% \newcommand{\Mcirc}{\ensuremath{\mathsf{M}_\modal}}
\newcommand{\mreturn}{\ensuremath{\eta}}
\let\project\mreturn
%\newcommand{\mbind}[1]{\ensuremath{\hat{#1}}}
\newcommand{\ext}{\mathsf{ext}}
%\newcommand{\mmap}[1]{\ensuremath{\bar{#1}}}
%\newcommand{\mjoin}{\ensuremath{\mreturn^{-1}}}
% Subuniverse
\renewcommand{\P}{\ensuremath{\type_{P}}\xspace}

%%% Localizations
% \newcommand{\islocal}[1]{\ensuremath{\mathsf{islocal}_{#1}}\xspace}
% \newcommand{\loc}[1]{\ensuremath{\mathcal{L}_{#1}}\xspace}

%%% Identity types %%%
\newcommand{\idsym}{{=}}
\newcommand{\id}[3][]{\ensuremath{#2 =_{#1} #3}\xspace}
\newcommand{\idtype}[3][]{\ensuremath{\mathsf{Id}_{#1}(#2,#3)}\xspace}
\newcommand{\idtypevar}[1]{\ensuremath{\mathsf{Id}_{#1}}\xspace}
% A propositional equality currently being defined
\newcommand{\defid}{\coloneqq}

%%% Dependent paths
\newcommand{\dpath}[4]{#3 =^{#1}_{#2} #4}

%%% singleton
% \newcommand{\sgl}{\ensuremath{\mathsf{sgl}}\xspace}
% \newcommand{\sctr}{\ensuremath{\mathsf{sctr}}\xspace}

%%% Reflexivity terms %%%
% \newcommand{\reflsym}{{\mathsf{refl}}}
\newcommand{\refl}[1]{\ensuremath{\mathsf{refl}_{#1}}\xspace}

%%% Path concatenation (used infix, in diagrammatic order) %%%
\newcommand{\ct}{%
  \mathchoice{\mathbin{\raisebox{0.5ex}{$\displaystyle\centerdot$}}}%
             {\mathbin{\raisebox{0.5ex}{$\centerdot$}}}%
             {\mathbin{\raisebox{0.25ex}{$\scriptstyle\,\centerdot\,$}}}%
             {\mathbin{\raisebox{0.1ex}{$\scriptscriptstyle\,\centerdot\,$}}}
}

%%% Path reversal %%%
\newcommand{\opp}[1]{\mathord{{#1}^{-1}}}
\let\rev\opp

%%% Transport (covariant) %%%
\newcommand{\trans}[2]{\ensuremath{{#1}_{*}\mathopen{}\left({#2}\right)\mathclose{}}\xspace}
\let\Trans\trans
%\newcommand{\Trans}[2]{\ensuremath{{#1}_{*}\left({#2}\right)}\xspace}
\newcommand{\transf}[1]{\ensuremath{{#1}_{*}}\xspace} % Without argument
%\newcommand{\transport}[2]{\ensuremath{\mathsf{transport}_{*} \: {#2}\xspace}}
\newcommand{\transfib}[3]{\ensuremath{\mathsf{transport}^{#1}(#2,#3)\xspace}}
\newcommand{\Transfib}[3]{\ensuremath{\mathsf{transport}^{#1}\Big(#2,\, #3\Big)\xspace}}
\newcommand{\transfibf}[1]{\ensuremath{\mathsf{transport}^{#1}\xspace}}

%%% 2D transport
\newcommand{\transtwo}[2]{\ensuremath{\mathsf{transport}^2\mathopen{}\left({#1},{#2}\right)\mathclose{}}\xspace}

%%% Constant transport
\newcommand{\transconst}[3]{\ensuremath{\mathsf{transportconst}}^{#1}_{#2}(#3)\xspace}
\newcommand{\transconstf}{\ensuremath{\mathsf{transportconst}}\xspace}

%%% Map on paths %%%
\newcommand{\mapfunc}[1]{\ensuremath{\mathsf{ap}_{#1}}\xspace} % Without argument
\newcommand{\map}[2]{\ensuremath{{#1}\mathopen{}\left({#2}\right)\mathclose{}}\xspace}
\let\Ap\map
%\newcommand{\Ap}[2]{\ensuremath{{#1}\left({#2}\right)}\xspace}
\newcommand{\mapdepfunc}[1]{\ensuremath{\mathsf{apd}_{#1}}\xspace} % Without argument
% \newcommand{\mapdep}[2]{\ensuremath{{#1}\llparenthesis{#2}\rrparenthesis}\xspace}
\newcommand{\mapdep}[2]{\ensuremath{\mapdepfunc{#1}\mathopen{}\left(#2\right)\mathclose{}}\xspace}
\let\apfunc\mapfunc
\let\ap\map
\let\apdfunc\mapdepfunc
\let\apd\mapdep

%%% 2D map on paths
\newcommand{\aptwofunc}[1]{\ensuremath{\mathsf{ap}^2_{#1}}\xspace}
\newcommand{\aptwo}[2]{\ensuremath{\aptwofunc{#1}\mathopen{}\left({#2}\right)\mathclose{}}\xspace}
\newcommand{\apdtwofunc}[1]{\ensuremath{\mathsf{apd}^2_{#1}}\xspace}
\newcommand{\apdtwo}[2]{\ensuremath{\apdtwofunc{#1}\mathopen{}\left(#2\right)\mathclose{}}\xspace}

%%% Identity functions %%%
\newcommand{\idfunc}[1][]{\ensuremath{\mathsf{id}_{#1}}\xspace}

%%% Homotopies (written infix) %%%
\newcommand{\htpy}{\sim}

%%% Other meanings of \sim
\newcommand{\bisim}{\sim}       % bisimulation
\newcommand{\eqr}{\sim}         % an equivalence relation

%%% Equivalence types %%%
\newcommand{\eqv}[2]{\ensuremath{#1 \simeq #2}\xspace}
\newcommand{\eqvspaced}[2]{\ensuremath{#1 \;\simeq\; #2}\xspace}
\newcommand{\eqvsym}{\simeq}    % infix symbol
\newcommand{\texteqv}[2]{\ensuremath{\mathsf{Equiv}(#1,#2)}\xspace}
\newcommand{\isequiv}{\ensuremath{\mathsf{isequiv}}}
\newcommand{\qinv}{\ensuremath{\mathsf{qinv}}}
\newcommand{\ishae}{\ensuremath{\mathsf{ishae}}}
\newcommand{\linv}{\ensuremath{\mathsf{linv}}}
\newcommand{\rinv}{\ensuremath{\mathsf{rinv}}}
\newcommand{\biinv}{\ensuremath{\mathsf{biinv}}}
\newcommand{\lcoh}[3]{\mathsf{lcoh}_{#1}(#2,#3)}
\newcommand{\rcoh}[3]{\mathsf{rcoh}_{#1}(#2,#3)}
\newcommand{\hfib}[2]{{\mathsf{fib}}_{#1}(#2)}

%%% Map on total spaces %%%
\newcommand{\total}[1]{\ensuremath{\mathsf{total}(#1)}}

%%% Universe types %%%
%\newcommand{\type}{\ensuremath{\mathsf{Type}}\xspace}
\newcommand{\UU}{\ensuremath{\mathcal{U}}\xspace}
\let\bbU\UU
\let\type\UU
% Universes of truncated types
\newcommand{\typele}[1]{\ensuremath{{#1}\text-\mathsf{Type}}\xspace}
\newcommand{\typeleU}[1]{\ensuremath{{#1}\text-\mathsf{Type}_\UU}\xspace}
\newcommand{\typelep}[1]{\ensuremath{{(#1)}\text-\mathsf{Type}}\xspace}
\newcommand{\typelepU}[1]{\ensuremath{{(#1)}\text-\mathsf{Type}_\UU}\xspace}
\let\ntype\typele
\let\ntypeU\typeleU
\let\ntypep\typelep
\let\ntypepU\typelepU
\renewcommand{\set}{\ensuremath{\mathsf{Set}}\xspace}
\newcommand{\setU}{\ensuremath{\mathsf{Set}_\UU}\xspace}
\newcommand{\prop}{\ensuremath{\mathsf{Prop}}\xspace}
\newcommand{\propU}{\ensuremath{\mathsf{Prop}_\UU}\xspace}
%Pointed types
\newcommand{\pointed}[1]{\ensuremath{#1_\bullet}}

%%% Ordinals and cardinals
\newcommand{\card}{\ensuremath{\mathsf{Card}}\xspace}
\newcommand{\ord}{\ensuremath{\mathsf{Ord}}\xspace}
\newcommand{\ordsl}[2]{{#1}_{/#2}}

%%% Univalence
\newcommand{\ua}{\ensuremath{\mathsf{ua}}\xspace} % the inverse of idtoeqv
\newcommand{\idtoeqv}{\ensuremath{\mathsf{idtoeqv}}\xspace}
\newcommand{\univalence}{\ensuremath{\mathsf{univalence}}\xspace} % the full axiom

%%% Truncation levels
\newcommand{\iscontr}{\ensuremath{\mathsf{isContr}}}
\newcommand{\contr}{\ensuremath{\mathsf{contr}}} % The path to the center of contraction
\newcommand{\isset}{\ensuremath{\mathsf{isSet}}}
\newcommand{\isprop}{\ensuremath{\mathsf{isProp}}}
% h-propositions
% \newcommand{\anhprop}{a mere proposition\xspace}
% \newcommand{\hprops}{mere propositions\xspace}

%%% Homotopy fibers %%%
%\newcommand{\hfiber}[2]{\ensuremath{\mathsf{hFiber}(#1,#2)}\xspace}
\let\hfiber\hfib

%%% Bracket/squash/truncation types %%%
% \newcommand{\brck}[1]{\textsf{mere}(#1)}
% \newcommand{\Brck}[1]{\textsf{mere}\Big(#1\Big)}
% \newcommand{\trunc}[2]{\tau_{#1}(#2)}
% \newcommand{\Trunc}[2]{\tau_{#1}\Big(#2\Big)}
% \newcommand{\truncf}[1]{\tau_{#1}}
%\newcommand{\trunc}[2]{\Vert #2\Vert_{#1}}
\newcommand{\trunc}[2]{\mathopen{}\left\Vert #2\right\Vert_{#1}\mathclose{}}
\newcommand{\ttrunc}[2]{\bigl\Vert #2\bigr\Vert_{#1}}
\newcommand{\Trunc}[2]{\Bigl\Vert #2\Bigr\Vert_{#1}}
\newcommand{\truncf}[1]{\Vert \blank \Vert_{#1}}
\newcommand{\tproj}[3][]{\mathopen{}\left|#3\right|_{#2}^{#1}\mathclose{}}
\newcommand{\tprojf}[2][]{|\blank|_{#2}^{#1}}
\def\pizero{\trunc0}
%\newcommand{\brck}[1]{\trunc{-1}{#1}}
%\newcommand{\Brck}[1]{\Trunc{-1}{#1}}
%\newcommand{\bproj}[1]{\tproj{-1}{#1}}
%\newcommand{\bprojf}{\tprojf{-1}}

\newcommand{\brck}[1]{\trunc{}{#1}}
\newcommand{\bbrck}[1]{\ttrunc{}{#1}}
\newcommand{\Brck}[1]{\Trunc{}{#1}}
\newcommand{\bproj}[1]{\tproj{}{#1}}
\newcommand{\bprojf}{\tprojf{}}

% Big parentheses
\newcommand{\Parens}[1]{\Bigl(#1\Bigr)}

% Projection and extension for truncations
\let\extendsmb\ext
\newcommand{\extend}[1]{\extendsmb(#1)}

%
%%% The empty type
\newcommand{\emptyt}{\ensuremath{\mathbf{0}}\xspace}

%%% The unit type
\newcommand{\unit}{\ensuremath{\mathbf{1}}\xspace}
\newcommand{\ttt}{\ensuremath{\star}\xspace}

%%% The two-element type
\newcommand{\bool}{\ensuremath{\mathbf{2}}\xspace}
\newcommand{\btrue}{{1_{\bool}}}
\newcommand{\bfalse}{{0_{\bool}}}

%%% Injections into binary sums and pushouts
\newcommand{\inlsym}{{\mathsf{inl}}}
\newcommand{\inrsym}{{\mathsf{inr}}}
\newcommand{\inl}{\ensuremath\inlsym\xspace}
\newcommand{\inr}{\ensuremath\inrsym\xspace}

%%% The segment of the interval
\newcommand{\seg}{\ensuremath{\mathsf{seg}}\xspace}

%%% Free groups
\newcommand{\freegroup}[1]{F(#1)}
\newcommand{\freegroupx}[1]{F'(#1)} % the "other" free group

%%% Glue of a pushout
\newcommand{\glue}{\mathsf{glue}}

%%% Circles and spheres
\newcommand{\Sn}{\mathbb{S}}
\newcommand{\base}{\ensuremath{\mathsf{base}}\xspace}
\newcommand{\lloop}{\ensuremath{\mathsf{loop}}\xspace}
\newcommand{\surf}{\ensuremath{\mathsf{surf}}\xspace}

%%% Suspension
\newcommand{\susp}{\Sigma}
\newcommand{\north}{\mathsf{N}}
\newcommand{\south}{\mathsf{S}}
\newcommand{\merid}{\mathsf{merid}}

%%% Blanks (shorthand for lambda abstractions)
\newcommand{\blank}{\mathord{\hspace{1pt}\text{--}\hspace{1pt}}}

%%% Nameless objects
\newcommand{\nameless}{\mathord{\hspace{1pt}\underline{\hspace{1ex}}\hspace{1pt}}}

%%% Some decorations
%\newcommand{\bbU}{\ensuremath{\mathbb{U}}\xspace}
% \newcommand{\bbB}{\ensuremath{\mathbb{B}}\xspace}
\newcommand{\bbP}{\ensuremath{\mathbb{P}}\xspace}

%%% Some categories
\newcommand{\uset}{\ensuremath{\mathcal{S}et}\xspace}
\newcommand{\ucat}{\ensuremath{{\mathcal{C}at}}\xspace}
\newcommand{\urel}{\ensuremath{\mathcal{R}el}\xspace}
\newcommand{\uhilb}{\ensuremath{\mathcal{H}ilb}\xspace}
\newcommand{\utype}{\ensuremath{\mathcal{T}\!ype}\xspace}

% Pullback corner
%\newbox\pbbox
%\setbox\pbbox=\hbox{\xy \POS(65,0)\ar@{-} (0,0) \ar@{-} (65,65)\endxy}
%\def\pb{\save[]+<3.5mm,-3.5mm>*{\copy\pbbox} \restore}

% Macros for the categories chapter
\newcommand{\inv}[1]{{#1}^{-1}}
\newcommand{\idtoiso}{\ensuremath{\mathsf{idtoiso}}\xspace}
\newcommand{\isotoid}{\ensuremath{\mathsf{isotoid}}\xspace}
\newcommand{\op}{^{\mathrm{op}}}
\newcommand{\y}{\ensuremath{\mathbf{y}}\xspace}
\newcommand{\dgr}[1]{{#1}^{\dagger}}
\newcommand{\unitaryiso}{\mathrel{\cong^\dagger}}
\newcommand{\cteqv}[2]{\ensuremath{#1 \simeq #2}\xspace}
\newcommand{\cteqvsym}{\simeq}     % Symbol for equivalence of categories

%%% Natural numbers
\newcommand{\N}{\ensuremath{\mathbb{N}}\xspace}
%\newcommand{\N}{\textbf{N}}
\let\nat\N
\newcommand{\natp}{\ensuremath{\nat'}\xspace} % alternative nat in induction chapter

\newcommand{\zerop}{\ensuremath{0'}\xspace}   % alternative zero in induction chapter
\newcommand{\suc}{\mathsf{succ}}
\newcommand{\sucp}{\ensuremath{\suc'}\xspace} % alternative suc in induction chapter
\newcommand{\add}{\mathsf{add}}
\newcommand{\ack}{\mathsf{ack}}
\newcommand{\ite}{\mathsf{iter}}
\newcommand{\assoc}{\mathsf{assoc}}
\newcommand{\dbl}{\ensuremath{\mathsf{double}}}
\newcommand{\dblp}{\ensuremath{\dbl'}\xspace} % alternative double in induction chapter


%%% Lists
\newcommand{\lst}[1]{\mathsf{List}(#1)}
\newcommand{\nil}{\mathsf{nil}}
\newcommand{\cons}{\mathsf{cons}}

%%% Vectors of given length, used in induction chapter
\newcommand{\vect}[2]{\ensuremath{\mathsf{Vec}_{#1}(#2)}\xspace}

%%% Integers
\newcommand{\Z}{\ensuremath{\mathbb{Z}}\xspace}
\newcommand{\Zsuc}{\mathsf{succ}}
\newcommand{\Zpred}{\mathsf{pred}}

%%% Rationals
\newcommand{\Q}{\ensuremath{\mathbb{Q}}\xspace}

%%% Function extensionality
\newcommand{\funext}{\mathsf{funext}}
\newcommand{\happly}{\mathsf{happly}}

%%% A naturality lemma
\newcommand{\com}[3]{\mathsf{swap}_{#1,#2}(#3)}

%%% Code/encode/decode
\newcommand{\code}{\ensuremath{\mathsf{code}}\xspace}
\newcommand{\encode}{\ensuremath{\mathsf{encode}}\xspace}
\newcommand{\decode}{\ensuremath{\mathsf{decode}}\xspace}

% Function definition with domain and codomain
\newcommand{\function}[4]{\left\{\begin{array}{rcl}#1 &
      \longrightarrow & #2 \\ #3 & \longmapsto & #4 \end{array}\right.}

%%% Cones and cocones
\newcommand{\cone}[2]{\mathsf{cone}_{#1}(#2)}
\newcommand{\cocone}[2]{\mathsf{cocone}_{#1}(#2)}
% Apply a function to a cocone
\newcommand{\composecocone}[2]{#1\circ#2}
\newcommand{\composecone}[2]{#2\circ#1}
%%% Diagrams
\newcommand{\Ddiag}{\mathscr{D}}

%%% (pointed) mapping spaces
\newcommand{\Map}{\mathsf{Map}}

%%% The interval
\newcommand{\interval}{\ensuremath{I}\xspace}
\newcommand{\izero}{\ensuremath{0_{\interval}}\xspace}
\newcommand{\ione}{\ensuremath{1_{\interval}}\xspace}

%%% Arrows
\newcommand{\epi}{\ensuremath{\twoheadrightarrow}}
\newcommand{\mono}{\ensuremath{\rightarrowtail}}

%%% Sets
\newcommand{\bin}{\ensuremath{\mathrel{\widetilde{\in}}}}

%%% Semigroup structure
\newcommand{\semigroupstrsym}{\ensuremath{\mathsf{SemigroupStr}}}
\newcommand{\semigroupstr}[1]{\ensuremath{\mathsf{SemigroupStr}}(#1)}
\newcommand{\semigroup}[0]{\ensuremath{\mathsf{Semigroup}}}

%%% Macros for the formal type theory
\newcommand{\emptyctx}{\ensuremath{\cdot}}
\newcommand{\production}{\vcentcolon\vcentcolon=}
\newcommand{\conv}{\downarrow}
\newcommand{\wfctx}[1]{#1\ \ctx}
\newcommand{\oftp}[3]{#1 \vdash #2 : #3}
\newcommand{\jdeqtp}[4]{#1 \vdash #2 \jdeq #3 : #4}
\newcommand{\judg}[2]{#1 \vdash #2}
\newcommand{\tmtp}[2]{#1 \mathord{:} #2}

% rule names
\newcommand{\form}{\textsc{form}}
\newcommand{\intro}{\textsc{intro}}
\newcommand{\elim}{\textsc{elim}}
\newcommand{\comp}{\textsc{comp}}
\newcommand{\uniq}{\textsc{uniq}}
\newcommand{\Weak}{\mathsf{Wkg}}
\newcommand{\Vble}{\mathsf{Vble}}
\newcommand{\Exch}{\mathsf{Exch}}
\newcommand{\Subst}{\mathsf{Subst}}

%%% Macros for HITs
\newcommand{\cc}{\mathsf{c}}
\newcommand{\pp}{\mathsf{p}}
\newcommand{\cct}{\widetilde{\mathsf{c}}}
\newcommand{\ppt}{\widetilde{\mathsf{p}}}
\newcommand{\Wtil}{\ensuremath{\widetilde{W}}\xspace}

%%% Macros for n-types
\newcommand{\istype}[1]{\mathsf{is}\mbox{-}{#1}\mbox{-}\mathsf{type}}
\newcommand{\nplusone}{\ensuremath{(n+1)}}
\newcommand{\nminusone}{\ensuremath{(n-1)}}
\newcommand{\fact}{\mathsf{fact}}

%%% Macros for homotopy
\newcommand{\kbar}{\overline{k}} % Used in van Kampen's theorem

%%% Macros for induction
\newcommand{\natw}{\ensuremath{\mathbf{N^w}}\xspace}
\newcommand{\zerow}{\ensuremath{0^\mathbf{w}}\xspace}
\newcommand{\sucw}{\ensuremath{\mathbf{s^w}}\xspace}
\newcommand{\nalg}{\nat\mathsf{Alg}}
\newcommand{\nhom}{\nat\mathsf{Hom}}
\newcommand{\ishinitw}{\mathsf{isHinit}_{\mathsf{W}}}
\newcommand{\ishinitn}{\mathsf{isHinit}_\nat}
\newcommand{\w}{\mathsf{W}}
\newcommand{\walg}{\w\mathsf{Alg}}
\newcommand{\whom}{\w\mathsf{Hom}}

%%% Macros for real numbers
\newcommand{\RC}{\ensuremath{\mathbb{R}_\mathsf{c}}\xspace} % Cauchy
\newcommand{\RD}{\ensuremath{\mathbb{R}_\mathsf{d}}\xspace} % Dedekind
\newcommand{\R}{\ensuremath{\mathbb{R}}\xspace}           % Either 
\newcommand{\barRD}{\ensuremath{\bar{\mathbb{R}}_\mathsf{d}}\xspace} % Dedekind completion of Dedekind

\newcommand{\close}[1]{\sim_{#1}} % Relation of closeness
\newcommand{\closesym}{\mathord\sim}
\newcommand{\rclim}{\mathsf{lim}} % HIT constructor for Cauchy reals
\newcommand{\rcrat}{\mathsf{rat}} % Embedding of rationals into Cauchy reals
\newcommand{\rceq}{\mathsf{eq}_{\RC}} % HIT path constructor
\newcommand{\CAP}{\mathcal{C}}    % The type of Cauchy approximations
\newcommand{\Qp}{\Q_{+}}
\newcommand{\apart}{\mathrel{\#}}  % apartness
\newcommand{\dcut}{\mathsf{isCut}}  % Dedekind cut
\newcommand{\cover}{\triangleleft} % inductive cover
\newcommand{\intfam}[3]{(#2, \lam{#1} #3)} % family of rational intervals

% Macros for the Cauchy reals construction
\newcommand{\bsim}{\frown}
\newcommand{\bbsim}{\smile}

\newcommand{\hapx}{\diamondsuit\approx}
\newcommand{\hapname}{\diamondsuit}
\newcommand{\hapxb}{\heartsuit\approx}
\newcommand{\hapbname}{\heartsuit}
\newcommand{\tap}[1]{\bullet\approx_{#1}\triangle}
\newcommand{\tapname}{\triangle}
\newcommand{\tapb}[1]{\bullet\approx_{#1}\square}
\newcommand{\tapbname}{\square}

%%% Macros for surreals
\newcommand{\NO}{\ensuremath{\mathsf{No}}\xspace}
\newcommand{\surr}[2]{\{\,#1\,\big|\,#2\,\}}
\newcommand{\LL}{\mathcal{L}}
\newcommand{\RR}{\mathcal{R}}
\newcommand{\noeq}{\mathsf{eq}_{\NO}} % HIT path constructor

\newcommand{\ble}{\trianglelefteqslant}
\newcommand{\blt}{\vartriangleleft}
\newcommand{\bble}{\sqsubseteq}
\newcommand{\bblt}{\sqsubset}

\newcommand{\hle}{\diamondsuit\preceq}
\newcommand{\hlt}{\diamondsuit\prec}
\newcommand{\hlname}{\diamondsuit}
\newcommand{\hleb}{\heartsuit\preceq}
\newcommand{\hltb}{\heartsuit\prec}
\newcommand{\hlbname}{\heartsuit}
% \newcommand{\tle}{(\bullet\preceq\triangle)}
% \newcommand{\tlt}{(\bullet\prec\triangle)}
\newcommand{\tle}{\triangle\preceq}
\newcommand{\tlt}{\triangle\prec}
\newcommand{\tlname}{\triangle}
% \newcommand{\tleb}{(\bullet\preceq\square)}
% \newcommand{\tltb}{(\bullet\prec\square)}
\newcommand{\tleb}{\square\preceq}
\newcommand{\tltb}{\square\prec}
\newcommand{\tlbname}{\square}

%%% Macros for set theory
\newcommand{\vset}{\mathsf{set}}  % point constructor for cummulative hierarchy V
\def\cd{\tproj0}
\newcommand{\inj}{\ensuremath{\mathsf{inj}}} % type of injections
\newcommand{\acc}{\ensuremath{\mathsf{acc}}} % accessibility

\newcommand{\atMostOne}{\mathsf{atMostOne}}

\newcommand{\power}[1]{\mathcal{P}(#1)} % power set
\newcommand{\powerp}[1]{\mathcal{P}_+(#1)} % inhabited power set

%%%% THEOREM ENVIRONMENTS %%%%

% Hyperref includes the command \autoref{...} which is like \ref{...}
% except that it automatically inserts the type of the thing you're
% referring to, e.g. it produces "Theorem 3.8" instead of just "3.8"
% (and makes the whole thing a hyperlink).  This saves a slight amount
% of typing, but more importantly it means that if you decide later on
% that 3.8 should be a Lemma or a Definition instead of a Theorem, you
% don't have to change the name in all the places you referred to it.

% The following hack improves on this by using the same counter for
% all theorem-type environments, so that after Theorem 1.1 comes
% Corollary 1.2 rather than Corollary 1.1.  This makes it much easier
% for the reader to find a particular theorem when flipping through
% the document.
\makeatletter
\def\defthm#1#2#3{%
  %% Ensure all theorem types are numbered with the same counter
  \newaliascnt{#1}{thm}
  \newtheorem{#1}[#1]{#2}
  \aliascntresetthe{#1}
  %% This command tells cleveref's \cref what to call things
  \crefname{#1}{#2}{#3}}

% Now define a bunch of theorem-type environments.
\newtheorem{thm}{Theorem}[section]
\crefname{thm}{Theorem}{Theorems}
%\defthm{prop}{Proposition}   % Probably we shouldn't use "Proposition" in this way
\defthm{cor}{Corollary}{Corollaries}
\defthm{lem}{Lemma}{Lemmas}
\defthm{axiom}{Axiom}{Axioms}
% Since definitions and theorems in type theory are synonymous, should
% we actually use the same theoremstyle for them?
\theoremstyle{definition}
\defthm{defn}{Definition}{Definitions}
\theoremstyle{remark}
\defthm{rmk}{Remark}{Remarks}
\defthm{eg}{Example}{Examples}
\defthm{egs}{Examples}{Examples}
\defthm{notes}{Notes}{Notes}
% Number exercises within chapters, with their own counter.
%\newtheorem{ex}{Exercise}[chapter]
%\crefname{ex}{Exercise}{Exercises}

% Display format for sections
\crefformat{section}{\S#2#1#3}
\Crefformat{section}{Section~#2#1#3}
\crefrangeformat{section}{\S\S#3#1#4--#5#2#6}
\Crefrangeformat{section}{Sections~#3#1#4--#5#2#6}
\crefmultiformat{section}{\S\S#2#1#3}{ and~#2#1#3}{, #2#1#3}{ and~#2#1#3}
\Crefmultiformat{section}{Sections~#2#1#3}{ and~#2#1#3}{, #2#1#3}{ and~#2#1#3}
\crefrangemultiformat{section}{\S\S#3#1#4--#5#2#6}{ and~#3#1#4--#5#2#6}{, #3#1#4--#5#2#6}{ and~#3#1#4--#5#2#6}
\Crefrangemultiformat{section}{Sections~#3#1#4--#5#2#6}{ and~#3#1#4--#5#2#6}{, #3#1#4--#5#2#6}{ and~#3#1#4--#5#2#6}

% Display format for appendices
\crefformat{appendix}{Appendix~#2#1#3}
\Crefformat{appendix}{Appendix~#2#1#3}
\crefrangeformat{appendix}{Appendices~#3#1#4--#5#2#6}
\Crefrangeformat{appendix}{Appendices~#3#1#4--#5#2#6}
\crefmultiformat{appendix}{Appendices~#2#1#3}{ and~#2#1#3}{, #2#1#3}{ and~#2#1#3}
\Crefmultiformat{appendix}{Appendices~#2#1#3}{ and~#2#1#3}{, #2#1#3}{ and~#2#1#3}
\crefrangemultiformat{appendix}{Appendices~#3#1#4--#5#2#6}{ and~#3#1#4--#5#2#6}{, #3#1#4--#5#2#6}{ and~#3#1#4--#5#2#6}
\Crefrangemultiformat{appendix}{Appendices~#3#1#4--#5#2#6}{ and~#3#1#4--#5#2#6}{, #3#1#4--#5#2#6}{ and~#3#1#4--#5#2#6}

\crefname{part}{Part}{Parts}

\crefformat{paragraph}{\S#2#1#3}
\Crefformat{paragraph}{Paragraph~#2#1#3}
\crefrangeformat{paragraph}{\S\S#3#1#4--#5#2#6}
\Crefrangeformat{paragraph}{Paragraphs~#3#1#4--#5#2#6}
\crefmultiformat{paragraph}{\S\S#2#1#3}{ and~#2#1#3}{, #2#1#3}{ and~#2#1#3}
\Crefmultiformat{paragraph}{Paragraphs~#2#1#3}{ and~#2#1#3}{, #2#1#3}{ and~#2#1#3}
\crefrangemultiformat{paragraph}{\S\S#3#1#4--#5#2#6}{ and~#3#1#4--#5#2#6}{, #3#1#4--#5#2#6}{ and~#3#1#4--#5#2#6}
\Crefrangemultiformat{paragraph}{Paragraphs~#3#1#4--#5#2#6}{ and~#3#1#4--#5#2#6}{, #3#1#4--#5#2#6}{ and~#3#1#4--#5#2#6}

% Number subsubsections
\setcounter{secnumdepth}{5}

% Display format for figures
\crefname{figure}{Figure}{Figures}

% Use cleveref instead of hyperref's \autoref
\let\autoref\cref


%%%% EQUATION NUMBERING %%%%

% The following hack uses the single theorem counter to number
% equations as well, so that we don't have both Theorem 1.1 and
% equation (1.1).
\let\c@equation\c@thm
\numberwithin{equation}{section}


%%%% ENUMERATE NUMBERING %%%%

% Number the first level of enumerates as (i), (ii), ...
\renewcommand{\theenumi}{(\roman{enumi})}
\renewcommand{\labelenumi}{\theenumi}


%%%% MARGINS %%%%

% This is a matter of personal preference, but I think the left
% margins on enumerates and itemizes are too wide.
\setitemize[1]{leftmargin=2em}
\setenumerate[1]{leftmargin=*}

% Likewise that they are too spaced out.
\setitemize[1]{itemsep=-0.2em}
\setenumerate[1]{itemsep=-0.2em}

%%% Notes %%%
\def\noteson{%
\gdef\note##1{\mbox{}\marginpar{\color{blue}\textasteriskcentered\ ##1}}}
\gdef\notesoff{\gdef\note##1{\null}}
\noteson

\newcommand{\Coq}{\textsc{Coq}\xspace}
\newcommand{\Agda}{\textsc{Agda}\xspace}
\newcommand{\NuPRL}{\textsc{NuPRL}\xspace}

%%%% CITATIONS %%%%

% \let \cite \citep

%%%% INDEX %%%%

\newcommand{\footstyle}[1]{{\hyperpage{#1}}n} % If you index something that is in a footnote
\newcommand{\defstyle}[1]{\textbf{\hyperpage{#1}}}  % Style for pageref to a definition

\newcommand{\indexdef}[1]{\index{#1|defstyle}}   % Index a definition
\newcommand{\indexfoot}[1]{\index{#1|footstyle}} % Index a term in a footnote
\newcommand{\indexsee}[2]{\index{#1|see{#2}}}    % Index "see also"


%%%% Standard phrasing or spelling of common phrases %%%%

\newcommand{\ZF}{Zermelo--Fraenkel}
\newcommand{\CZF}{Constructive \ZF{} Set Theory}

\newcommand{\LEM}[1]{\ensuremath{\mathsf{LEM}_{#1}}\xspace}
\newcommand{\choice}[1]{\ensuremath{\mathsf{AC}_{#1}}\xspace}

%%%% MISC %%%%

\newcommand{\mentalpause}{\medskip} % Use for "mental" pause, instead of \smallskip or \medskip

%% Use \symlabel instead of \label to mark a pageref that you need in the index of symbols
\newcounter{symindex}
\newcommand{\symlabel}[1]{\refstepcounter{symindex}\label{#1}}

% Local Variables:
% mode: latex
% TeX-master: "hott-online"
% End:


\newcommand{\idsymbin}{=}

%%%%%%%%%%%%%%%%%%%%%%%%%%%%%%%%%%%%%%%%%%%%%%%%%%%%%%%%%%%%%%%%%%%%%%%%%%%%%%%%
%%%% Our commands which are not part of the macros.tex file.
%%%% We should keep these commands separate, because we will
%%%% update the macros.tex following the updates of the book.

%%%% First we redefine the \id, \eqv and \ct commands so that they accept an
%%%% arbitrary number of arguments. This is useful when writing longer strings
%%%% of equalities or equivalences.

\makeatletter

\renewcommand{\id}[3][]{
  \@ifnextchar\bgroup
    {#2 \mathbin{\idsym_{#1}} \id[#1]{#3}}
    {#2 \mathbin{\idsym_{#1}} #3}
  }

\renewcommand{\eqv}[2]{
  \@ifnextchar\bgroup
    {#1 \eqvsym \eqv{#2}}
    {#1 \eqvsym #2}
  }

\newcommand{\ctsym}{%
  \mathchoice{\mathbin{\raisebox{0.5ex}{$\displaystyle\centerdot$}}}%
             {\mathbin{\raisebox{0.5ex}{$\centerdot$}}}%
             {\mathbin{\raisebox{0.25ex}{$\scriptstyle\,\centerdot\,$}}}%
             {\mathbin{\raisebox{0.1ex}{$\scriptscriptstyle\,\centerdot\,$}}}
  }

\renewcommand{\ct}[3][]{
  \@ifnextchar\bgroup
    {#2 \mathbin{\ctsym_{#1}} \ct[#1]{#3}}
    {#2 \mathbin{\ctsym_{#1}} #3}
  }

\makeatother

%%%% We always use textstyle products and sums...
%\renewcommand{\prd}{\tprd}
%\renewcommand{\sm}{\tsm}
\makeatletter
\renewcommand{\@dprd}{\@tprd}
\renewcommand{\@dsm}{\@tsm}
\renewcommand{\@dprd@noparens}{\@tprd}
\renewcommand{\@dsm@noparens}{\@tsm}

%%%% ...with a bit more spacing
\renewcommand{\@tprd}[1]{\mathchoice{{\textstyle\prod_{(#1)}\,}}{\prod_{(#1)}\,}{\prod_{(#1)}\,}{\prod_{(#1)}\,}}
\renewcommand{\@tsm}[1]{\mathchoice{{\textstyle\sum_{(#1)}\,}}{\sum_{(#1)}\,}{\sum_{(#1)}\,}{\sum_{(#1)}\,}}

%%%%%%%%%%%%%%%%%%%%%%%%%%%%%%%%%%%%%%%%%%%%%%%%%%%%%%%%%%%%%%%%%%%%%%%%%%%%%%%%
%%%% We adjust the \prd command so that implicit arguments become possible.
%%%%
%%%% First, we have the following switch. Set it to true if implicit arguments
%%%% are desired, or to false if not. Note turning off implicit arguments
%%%% might render some parts of the text harder to comprehend, since in the
%%%% text might appear $f(x)$ where we would have $f(i,x)$ without implicit
%%%% arguments.

\newcommand{\implicitargumentson}{\boolean{true}}

%%%% If one wants to use implicit arguments in the notation for product types,
%%%% a * has to be put before the argument that has to be implicit.
%%%% For example: in $\prd{x:A}*{y:B}{u:P(y)}Q(x,y,u)$, the argument y is
%%%% implicit. Any of the arguments can be made implicit this way.

%%%% First of all, we should make the command \prd search not only for a
%%%% brace, but also for a star. We introduce an auxiliary command that
%%%% determines whether the next character is a star or brace.
\newcommand{\@ifnextchar@starorbrace}[2]
%  {\@ifnextcharamong{#1}{#2}{*}{\bgroup};}
  {\@ifnextchar*{#1}{\@ifnextchar\bgroup{#1}{#2}}}
  
%%%% When encountering the \prd command, latex should determine whether it
%%%% should print implicit argument brackets or not. So the first branching
%%%% happens right here.
\renewcommand{\prd}{\@ifnextchar*{\@iprd}{\@prd}}

\newcommand{\@prd}[1]
  {\@ifnextchar@starorbrace
    {\prd@parens{#1}}
    {\@ifnextchar\sm{\prd@parens{#1}\@eatsm}{\prd@noparens{#1}}}}
\newcommand{\@prd@parens}{\@ifnextchar*{\@iprd}{\prd@parens}}
\renewcommand{\prd@parens}[1]
  {\@ifnextchar@starorbrace
    {\@theprd{#1}\@prd@parens}
    {\@ifnextchar\sm{\@theprd{#1}\@eatsm}{\@theprd{#1}}}}
\newcommand{\@theprd}[1]
  {\mathchoice{\@dprd{#1}}{\@tprd{#1}}{\@tprd{#1}}{\@tprd{#1}}}
\renewcommand{\dprd}[1]{\@dprd{#1}\@ifnextchar@starorbrace{\dprd}{}}
\renewcommand{\tprd}[1]{\@tprd{#1}\@ifnextchar@starorbrace{\tprd}{}}

%%%% Here we tell the actual symbols to be printed.
\newcommand{\@theiprd}[1]{\mathchoice{\@diprd{#1}}{\@tiprd{#1}}{\@tiprd{#1}}{\@tiprd{#1}}}
\newcommand{\@iprd}[2]{\@ifnextchar@starorbrace%
  {\@theiprd{#2}\@prd@parens}%
  {\@ifnextchar\sm%
    {\@theiprd{#2}\@eatsm}%
    {\@theiprd{#2}}}}
\def\@tiprd#1{
  \ifthenelse{\implicitargumentson}
    {\@@tiprd{#1}\@ifnextchar\bgroup{\@tiprd}{}}
    {\@tprd{#1}}}
\def\@@tiprd#1{\mathchoice{{\textstyle\prod_{\{#1\}}\,}}{\prod_{\{#1\}}\,}{\prod_{\{#1\}}\,}{\prod_{\{#1\}}\,}}
\def\@diprd{
  \ifthenelse{\implicitargumentson}
    {\@tiprd}
    {\@tprd}}
    

%%%% And finally we need to redefine \@eatprd so that implicit arguments also
%%%% works in the scope of a dependent sum.    
\def\@eatprd\prd{\@prd@parens}

\makeatother

%%%%%%%%%%%%%%%%%%%%%%%%%%%%%%%%%%%%%%%%%%%%%%%%%%%%%%%%%%%%%%%%%%%%%%%%%%%%%%%%
%%%% Redefining the quantifiers, so that some of the longer 
%%%% formulas appear one a single line without problems

%%% Dependent products written with \forall, in the same style
\makeatletter
\def\tfall#1{\forall_{(#1)}\@ifnextchar\bgroup{\,\tfall}{\,}}
\renewcommand{\fall}{\tfall}

%%% Existential quantifier %%%
\def\texis#1{\exists_{(#1)}\@ifnextchar\bgroup{\,\texis}{\,}}
\renewcommand{\exis}{\texis}

%%% Unique existence %%%
\def\uexis#1{\exists!_{(#1)}\@ifnextchar\bgroup{\,\uexis}{\,}}
\makeatother

%%%%%%%%%%%%%%%%%%%%%%%%%%%%%%%%%%%%%%%%%%%%%%%%%%%%%%%%%%%%%%%%%%%%%%%%%%%%%%%%
%%%% UNFOLD
%%%%
%%%% For each definition in the type theory we make two versions of the macro:
%%%% the macro introducing the new notation and an @unfold version of the macro
%%%% which outputs the meaning of that new notation. Thus, we can use the
%%%% following construction to write our text. When we introduce \macro, we can
%%%% write \unfold{\macro} and the output will be the result of \macro@unfold.

\makeatletter
\newcommand{\unfold}{%
  \unfoldnext}
\newcommand{\unfoldall}[1]{%
  \begingroup%
  \renewcommand{\jhom}{\jhom@unfold}%
  \renewcommand{\jhomeq}{\jhomeq@unfold}%
  \renewcommand{\jhomdefn}{\jhomdefn@unfold}%
  \renewcommand{\jfhom}{\jfhom@unfold}%
  \renewcommand{\jcomp}{\jcomp@unfold}%
  \renewcommand{\@jcomp@nested}{\@jcomp@unfold@nested}%
  \renewcommand{\@jcomp@parens}{\@jcomp@unfold@parens}%
  \renewcommand{\tmext}{\tmext@unfold}%
  \renewcommand{\@tmext@nested}{\@tmext@unfold@nested}%
  \renewcommand{\@tmext@parens}{\@tmext@unfold@parens}%
  \renewcommand{\cprojfstf}{\cprojfstf@unfold}%
  \renewcommand{\cprojfst}{\cprojfst@unfold}%
  \renewcommand{\cprojsndf}{\cprojsndf@unfold}%
  \renewcommand{\cprojsnd}{\cprojsnd@unfold}%
  \renewcommand{\jfcomp}{\jfcomp@unfold}%
%  \renewcommand{\@jfcomp@nested}{\@jfcomp@unfold@nested}%
%  \renewcommand{\@jfcomp@parens}{\@jfcomp@unfold@parens}%
  \renewcommand{\sandwich}{\sandwich@unfold}%
  \renewcommand{\finc}{\finc@unfold}%
  \renewcommand{\jvcomp}{\jvcomp@unfold}%
  \renewcommand{\subst@type@unfold}[1]{
    \@ifnextchar\cprojfstf{\@eatdo{\cprojfstf@parens}}{%
      ##1}
    }
  #1%
  \endgroup%
  }

%%%% The following command is useful when you have checked with '\@ifnextchar'
%%%% that the next character is a macro '\firstmacro' and you want to replace
%%%% it by '\secondmacro'. To establish this, simply call for
%%%% '\@ifnextchar\firstmacro{\@eatdo{\secondmacro}}{}' with the second 
%%%% argument of \@eatdo left unspecified.
\newcommand{\@eatdo}[2]{#1}

%%%% The intention of '\unfoldnext' is to unfold only the definition of the
%%%% next character, provided that it is in the list of unfoldable macros.
\newcommand{\unfoldnext}[1]{
  \@ifnextchar\jhom{\@eatdo{\jhom@unfold}}{%
  \@ifnextchar\jhomeq{\@eatdo{\jhomeq@unfold}}{%
  \@ifnextchar\jhomdefn{\@eatdo{\jhomdefn@unfold}}{%
  \@ifnextchar\jfhom{\@eatdo{\jfhom@unfold}}{%
  \@ifnextchar\jcomp{\@eatdo{\jcomp@unfold}}{%
  \@ifnextchar\@jcomp@nested{\@eatdo{\@jcomp@unfold@nested}}{%
  \@ifnextchar\@jcomp@parens{\@eatdo{\@jcomp@unfold@parens}}{%
  \@ifnextchar\tmext{\@eatdo{\tmext@unfold}}{%
  \@ifnextchar\@tmext@nested{\@eatdo{\@tmext@unfold@nested}}{%
  \@ifnextchar\@tmext@parens{\@eatdo{\@tmext@unfold@parens}}{%
  \@ifnextchar\cprojfstf{\@eatdo{\cprojfstf@unfold}}{%
  \@ifnextchar\cprojfst{\@eatdo{\cprojfst@unfold}}{%
  \@ifnextchar\cprojsndf{\@eatdo{\cprojsndf@unfold}}{%
  \@ifnextchar\cprojsnd{\@eatdo{\cprojsnd@unfold}}{%
  \@ifnextchar\jfcomp{\@eatdo{\jfcomp@unfold}}{%
%  \@ifnextchar\@jfcomp@nested{\@eatdo{\@jfcomp@unfold@nested}}{%
%  \@ifnextchar\@jfcomp@parens{\@eatdo{\@jfcomp@unfold@parens}}{%
  \@ifnextchar\sandwich{\@eatdo{\sandwich@unfold}}{%
  \@ifnextchar\finc{\@eatdo{\finc@unfold}}{%
  \@ifnextchar\jvcomp{\@eatdo{\jvcomp@unfold}}}
  #1}
\makeatother

%%%%%%%%%%%%%%%%%%%%%%%%%%%%%%%%%%%%%%%%%%%%%%%%%%%%%%%%%%%%%%%%%%%%%%%%%%%%%%%%
%%%% A PRETTY PRINTER
%%%%
%%%% We write a \pretty command that pretty prints judgments or types by
%%%% diplaying variables and omitting explicit notation for weakening.
%%%%
%%%% This command should work similar to the \unfold command
%%%%
%%%% -- UNDER CONSTRUCTION

\makeatletter
\newcommand{\vardis}[2]{\@vardis@type #2{}(\@vardis@term #1)}
\newcommand{\@vardis}{\@ifnextchar\bgroup{\@@vardis}{}}
\newcommand{\@@vardis}[1]{\@ifnextchar\bgroup{\vardis{#1}}{#1}}
\newcommand{\@vardis@term}{\@vardis}
\newcommand{\@vardis@type}{\@ifnextchar\ctxext{\@ctxext@nested}{\@ifnextchar\ctxwk{\@ctxwk@nested}{\@vardis}}}
\newcommand{\@vardis@nested}[3]{\@vardis@parens{#2}{#3}}
\newcommand{\@vardis@parens}[2]{(\vardis{#1}{#2})}
\makeatother

\makeatletter
\newcommand{\jvctx}{\jctx}
\newcommand{\jvctxeq}{\jctxeq}

\newcommand{\cctxextcombi}[2]{\@ifnextchar\bgroup{\@cctxextcombi #1}{#1:}#2}
\newcommand{\@cctxextcombi}[4]{\cctxext{{\cctxextcombi{#1}{#3}}{\@@cctxextcombi{#1}{#2}{#4}}}}
\newcommand{\@@cctxextcombi}[3]{\@ifnextchar\bgroup{\@@@ctxextcombi #2}{#2(#1):}#3(\cctxext{#1})}
\newcommand{\@@@ctxextcombi}[8] % the 5th argument is (, the 6th is \cctxext and the 8th is ).
  {\@@ctxextcombi{#7}{#1}{#3},\@@ctxextcombi{{#7}{#3}}{#2}{#4}}
\newcommand{\cctxext}[1]{\@ifnextchar\bgroup{\@cctxext}{}#1}
\newcommand{\@cctxext}[2]{\cctxext{#1},\cctxext{#2}}

\newcommand{\jvfamcombi}[3]{
  \cctxextcombi{#1}{#2} \vdash \vardis{\cctxext{#1}}{#3}
}

\newcommand{\jvfam}{\@ifnextchar*{\@jvfamAlignTrue}{\@jvfamAlignFalse}}
\newcommand{\@jvfamAlignTrue}[4]{\jfam*{#2:#3}{\vardis{#2}{#4}}}
\newcommand{\@jvfamAlignFalse}[3]{\jfam{#1:#2}{\vardis{#1}{#3}}\quad@test}

\newcommand{\jvfameq}{\@ifnextchar*{\@jvfameqAlignTrue}{\@jvfameqAlignFalse}}
\newcommand{\@jvfameqAlignTrue}[5]{\jfameq*{#2:#3}{\vardis{#2}{#4}}{\vardis{#2}{#5}}}
\newcommand{\@jvfameqAlignFalse}[4]{\jfameq{#1:#2}{\vardis{#1}{#3}}{\vardis{#1}{#4}}\quad@test}

\newcommand{\jvtype}{\@ifnextchar*{\@jvtypeAlignTrue}{\@jvtypeAlignFalse}}
\newcommand{\@jvtypeAlignTrue}[4]{\jtype*{#2:#3}{\vardis{#2}{#4}}}
\newcommand{\@jvtypeAlignFalse}[3]{\jtype{#1:#2}{\vardis{#1}{#3}}\quad@test}

\newcommand{\jvtypeeq}{\@ifnextchar*{\@jvtypeeqAlignTrue}{\@jvtypeeqAlignFalse}}
\newcommand{\@jvtypeeqAlignTrue}[5]{\jtypeeq*{#2:#3}{\vardis{#2}{#4}}{\vardis{#2}{#5}}}
\newcommand{\@jvtypeeqAlignFalse}[4]{\jtypeeq{#1:#2}{\vardis{#1}{#3}}{\vardis{#1}{#4}}\quad@test}

\newcommand{\jvterm}{\@ifnextchar*{\@jvtermAlignTrue}{\@jvtermAlignFalse}}
\newcommand{\@jvtermAlignTrue}[5]{\jterm*{#2:#3}{\vardis{#2}{#4}}{\vardis{#2}{#5}}}
\newcommand{\@jvtermAlignFalse}[4]{\jterm{#1:#2}{\vardis{#1}{#3}}{\vardis{#1}{#4}}\quad@test}

\newcommand{\jvtermeq}{\@ifnextchar*{\@jvtermeqAlignTrue}{\@jvtermeqAlignFalse}}
\newcommand{\@jvtermeqAlignTrue}[6]{\jtermeq*{#2:#3}{\vardis{#2}{#4}}{\vardis{#2}{#5}}{\vardis{#2}{#6}}}
\newcommand{\@jvtermeqAlignFalse}[5]{\jtermeq{#1:#2}{\vardis{#1}{#3}}{\vardis{#1}{#4}}{\vardis{#1}{#5}}\quad@test}
\makeatother

%%%%%%%%%%%%%%%%%%%%%%%%%%%%%%%%%%%%%%%%%%%%%%%%%%%%%%%%%%%%%%%%%%%%%%%%%%%%%%%%
%%%%

\newcommand{\famsym}{\mathcal{F}}
\newcommand{\tmsym}{\mathcal{T}}

%%%%%%%%%%%%%%%%%%%%%%%%%%%%%%%%%%%%%%%%%%%%%%%%%%%%%%%%%%%%%%%%%%%%%%%%%%%%%%%%
%%%% JUDGMENTS
%%%%
%%%% Below we define several commands for the judgments of type theory. There
%%%% are commands
%%%% * \jctx for the judgment that something is a context.
%%%% * \jctxeq for the judgment that two contexts are the same
%%%% * \jtype for the judgment that something is a type in a context
%%%% * \jtypeeq for the judgment that two types in the same context are the same
%%%% * \jterm for the judgment that something is a term of a type in a context
%%%% * \jtermeq for the judgment that two terms of the same type are the same

\makeatletter
% We first make a generic judgment command
\newcommand{\judgment}{\@ifnextchar*{\@judgmentAT}{\@judgmentAF}}
\newcommand{\@judgmentAT}[8]{\@judgment@ctx{#2} & \vdash \@judgment@rel{#3}{#4}{#5}{#6}{#7} #8}
\newcommand{\@judgmentAF}[7]{\@judgment@ctx{#1} \vdash \@judgment@rel{#2}{#3}{#4}{#5}{#6} #7\quad@test}
\newcommand{\@judgment@ctx}{\@judgment@ext}
\newcommand{\@judgment@rel}[5]{
  { \default@ctxext #1
    }
  #2 
  { \default@ctxext #3
    }
  #4
  { \default@ctxext #5
    }}
\newcommand{\@judgment@kind}[1]{~~\textit{#1}}
\newcommand{\@judgment@ext}[1]{\default@ctxext #1}

\newcommand{\quad@test}{%
  \@ifnextchar\jctx{\quad}{%
  \@ifnextchar\jctxeq{\quad}{%
  \@ifnextchar\jvctx{\quad}{%
  \@ifnextchar\jvctxeq{\quad}{%
  \@ifnextchar\jfam{\quad}{%
  \@ifnextchar\jfameq{\quad}{%
  \@ifnextchar\jvfam{\quad}{%
  \@ifnextchar\jvfameq{\quad}{%
  \@ifnextchar\jtype{\quad}{%
  \@ifnextchar\jtypeeq{\quad}{%
  \@ifnextchar\jvtype{\quad}{%
  \@ifnextchar\jvtypeeq{\quad}{%
  \@ifnextchar\jterm{\quad}{%
  \@ifnextchar\jtermeq{\quad}{%
  \@ifnextchar\jvterm{\quad}{%
  \@ifnextchar\jvtermeq{\quad}{%
  \@ifnextchar\jhom{\quad}{%
  \@ifnextchar\jhomeq{\quad}{%
  \@ifnextchar\jfhom{\quad}{%
  \@ifnextchar\jfhomeq{\quad}{%
  }}}}}}}}}}}}}}}}}}}}}

%%%% Judgments about contexts
\newcommand{\jctx@sym}{\@judgment@kind{ctx}}

\newcommand{\jctx}{\@ifnextchar*{\@jctxAlignTrue}{\@jctxAlignFalse}}
\newcommand{\@jctxAlignTrue}[2]{\judgment*{}{}{}{}{}{#2}{\jctx@sym}}
\newcommand{\@jctxAlignFalse}[1]{\judgment{}{}{}{}{}{#1}{\jctx@sym}}

\newcommand{\jctxeq}{\@ifnextchar*{\@jctxeqAlignTrue}{\@jctxeqAlignFalse}}
\newcommand{\@jctxeqAlignTrue}[3]{\judgment*{}{#2}{\jdeq}{#3}{}{}{\jctx@sym}}
\newcommand{\@jctxeqAlignFalse}[2]{\judgment{}{#1}{\jdeq}{#2}{}{}{\jctx@sym}}

\newcommand{\jctxdefn}{\@ifnextchar*{\@jctxdefnAlignTrue}{\@jctxdefnAlignFalse}}
\newcommand{\@jctxdefnAlignTrue}[3]{\judgment*{}{#2}{\defeq}{#3}{}{}{\jctx@sym}}
\newcommand{\@jctxdefnAlignFalse}[2]{\judgment{}{#1}{\defeq}{#2}{}{}{\jctx@sym}}

%%%% Judgments about families
\newcommand{\jfam@sym}{\@judgment@kind{fam}}

\newcommand{\jfam}{\@ifnextchar*{\@jfamAlignTrue}{\@jfamAlignFalse}}
\newcommand{\@jfamAlignTrue}[3]{\judgment*{#2}{}{}{}{}{#3}{\jfam@sym}}
\newcommand{\@jfamAlignFalse}[2]{\judgment{#1}{}{}{}{}{#2}{\jfam@sym}}

\newcommand{\jfameq}{\@ifnextchar*{\@jfameqAlignTrue}{\@jfameqAlignFalse}}
\newcommand{\@jfameqAlignTrue}[4]{\judgment*{#2}{#3}{\jdeq}{#4}{}{}{\jfam@sym}}
\newcommand{\@jfameqAlignFalse}[3]{\judgment{#1}{#2}{\jdeq}{#3}{}{}{\jfam@sym}}

\newcommand{\jfamdefn}{\@ifnextchar*{\@jfamdefnAlignTrue}{\@jfamdefnAlignFalse}}
\newcommand{\@jfamdefnAlignTrue}[4]{\judgment*{#2}{#3}{\defeq}{#4}{}{}{\jfam@sym}}
\newcommand{\@jfamdefnAlignFalse}[3]{\judgment{#1}{#2}{\defeq}{#3}{}{}{\jfam@sym}}
  
%%%% Judgments about types
\newcommand{\jtype@sym}{\@judgment@kind{type}}
\newcommand{\jtype}{\@ifnextchar*{\@jtypeAlignTrue}{\@jtypeAlignFalse}}
\newcommand{\@jtypeAlignTrue}[3]{\judgment*{#2}{}{}{}{}{#3}{\jtype@sym}}
\newcommand{\@jtypeAlignFalse}[2]{\judgment{#1}{}{}{}{}{#2}{\jtype@sym}}
  
\newcommand{\jtypeeq}{\@ifnextchar*{\@jtypeeqAlignTrue}{\@jtypeeqAlignFalse}}
\newcommand{\@jtypeeqAlignTrue}[4]{\judgment*{#2}{#3}{\jdeq}{#4}{}{}{\jtype@sym}}
\newcommand{\@jtypeeqAlignFalse}[3]{\judgment{#1}{#2}{\jdeq}{#3}{}{}{\jtype@sym}}

\newcommand{\jtypedefn}{\@ifnextchar*{\@jtypedefnAlignTrue}{\@jtypedefnAlignFalse}}
\newcommand{\@jtypedefnAlignTrue}[4]{\judgment*{#2}{#3}{\defeq}{#4}{}{}{\jtype@sym}}
\newcommand{\@jtypedefnAlignFalse}[3]{\judgment{#1}{#2}{\defeq}{#3}{}{}{\jtype@sym}}
  
%%%% Judgments about terms
\newcommand{\jterm}{\@ifnextchar*{\@jtermAlignTrue}{\@jtermAlignFalse}}
\newcommand{\@jtermAlignTrue}[4]{\judgment*{#2}{}{}{#4}{:}{#3}{}}
\newcommand{\@jtermAlignFalse}[3]{\judgment{#1}{}{}{#3}{:}{#2}{}}

\newcommand{\jtermeq}{\@ifnextchar*{\@jtermeqAlignTrue}{\@jtermeqAlignFalse}}
\newcommand{\@jtermeqAlignTrue}[5]{\judgment*{#2}{#4}{\jdeq}{#5}{:}{#3}{}}
\newcommand{\@jtermeqAlignFalse}[4]{\judgment{#1}{#3}{\jdeq}{#4}{:}{#2}{}}

\newcommand{\jtermdefn}{\@ifnextchar*{\@jtermdefnAlignTrue}{\@jtermdefnAlignFalse}}
\newcommand{\@jtermdefnAlignTrue}[5]{\judgment*{#2}{#4}{\defeq}{#5}{:}{#3}{}}
\newcommand{\@jtermdefnAlignFalse}[4]{\judgment{#1}{#3}{\defeq}{#4}{:}{#2}{}}
\makeatother

%%%%%%%%%%%%%%%%%%%%%%%%%%%%%%%%%%%%%%%%%%%%%%%%%%%%%%%%%%%%%%%%%%%%%%%%%%%%%%%%
%%%% THE EMPTY CONTEXT

\newcommand{\emptysym}{[\;]}
\newcommand{\emptyc}{{\emptysym}}
\newcommand{\emptyf}[1][]{{\emptysym}_{#1}}
\newcommand{\emptytm}[1][]{\typefont{\#}_{#1}}

%%%%%%%%%%%%%%%%%%%%%%%%%%%%%%%%%%%%%%%%%%%%%%%%%%%%%%%%%%%%%%%%%%%%%%%%%%%%%%%%
%%%% CONTEXT EXTENSION 
%%%%
%%%% The context extension command.
%%%%
%%%% To get a feeling of how the command works, here are a few examples.
%%%% \ctxext{A}{B} will print A.B
%%%% \ctxext{{A}{B}}{C} will print (A.B).C
%%%% \ctxext{{{A}{B}}{C}}{{D}{E}} will print ((A.B).C).(D.E)

\makeatletter
\newcommand{\ctxext}[2]{\@ctxext@ctx #1.\@ctxext@type #2}
\newcommand{\@ctxext}{\@ifnextchar\bgroup{\@@ctxext}{}}
\newcommand{\@ctxext@ctx}{%
  \@ifnextchar\ctxext{\@ctxext@nested}{%
  \@ifnextchar\ctxwk{\@ctxwk@nested}{%
  \@ifnextchar\jcomp{\@jcomp@nested}{%
  \@ifnextchar\jvcomp{\@jvcomp@nested}{%
  \@ifnextchar\jfcomp{\@jfcomp@nested}{%
  \@ctxext}}}}}}
\newcommand{\@ctxext@type}{%
  \@ifnextchar\ctxext{\@ctxext@nested}{%
  \@ifnextchar\subst{\@subst@nested}{%
  \@ifnextchar\jcomp{\@jcomp@nested}{%
  \@ifnextchar\jvcomp{\@jvcomp@nested}{%
  \@ifnextchar\jfcomp{\@jfcomp@nested}{%
  \@ctxext}}}}}}
\newcommand{\@@ctxext}[1]{\@ifnextchar\bgroup{\@ctxext@parens{#1}}{#1}}
\newcommand{\@ctxext@parens}[2]{(\ctxext{#1}{#2})}
\newcommand{\@ctxext@nested}[3]{\@ctxext@parens{#2}{#3}}

%%%% We want that some commands accept binary trees as arguments that default
%%%% into extensions. We make the following command to realize this
\newcommand{\default@ctxext}{\@ifnextchar\bgroup{\ctxext}{}}
\newcommand{\default@ctxext@parens}{\@ifnextchar\bgroup{\@ctxext@parens}{}}
\makeatother

%%%%%%%%%%%%%%%%%%%%%%%%%%%%%%%%%%%%%%%%%%%%%%%%%%%%%%%%%%%%%%%%%%%%%%%%%%%%%%%%
%%%% SUBSTITUTION

%%%% The substitution command will act the following way
%%%%
%%%% \subst{x}{P} will print P[x]
%%%% \subst{x}{{f}{Q}} will print Q[f][x]
%%%% \subst{{x}{f}}{{x}{Q}} will print Q[x][f[x]]

\makeatletter
\newcommand{\subst}[3][]{%
  \@subst@type #3{}[\@subst@term #2]^{#1}}
\newcommand{\@subst}{%
  \@ifnextchar\bgroup{\@@subst}{}}
\newcommand{\@@subst}[1]{%
  \@ifnextchar\bgroup{\subst{#1}}{#1}}
\newcommand{\@subst@term}{%
  \@subst}
\newcommand{\@subst@type}{%
  \@ifnextchar\ctxext{\@ctxext@nested}{%
  \@ifnextchar\ctxwk{\@ctxwk@nested}{%
  \@ifnextchar\jcomp{\@jcomp@nested}{%
  \@ifnextchar\tmext{\@tmext@nested}{%
  \@ifnextchar\jvcomp{\@jvcomp@nested}{%
  \@ifnextchar\jfcomp{\@jfcomp@nested}{%
%  \@ifnextchar\mfam{\@mfam@nested}{%
%  \@ifnextchar\mtm{\@mtm@nested}}
\newcommand{\subst@type@unfold}[1]{#1}
\newcommand{\@subst@nested}[3]{%
  \@subst@parens{#2}{#3}}
\newcommand{\@subst@parens}[2]{%
  (\subst{#1}{#2})}
\makeatother

%%%%%%%%%%%%%%%%%%%%%%%%%%%%%%%%%%%%%%%%%%%%%%%%%%%%%%%%%%%%%%%%%%%%%%%%%%%%%%%%
%%%% WEAKENING

%%%% The weakening command is very much like the substitution command.

\makeatletter
\newcommand{\ctxwk}[3][]{%
  \langle\@ctxwk@act #2\rangle^{#1} \@ctxwk@pass #3}
\newcommand{\@ctxwk}{%
  \@ifnextchar\bgroup{\@@ctxwk}{}}
\newcommand{\@@ctxwk}[1]{%
  \@ifnextchar\bgroup{\ctxwk{#1}}{#1}}
\newcommand{\@ctxwk@act}{%
  \@ctxwk}
\newcommand{\@ctxwk@pass}{%
  \@ifnextchar\ctxext{\@ctxext@nested}{%
  \@ifnextchar\subst{\@subst@nested}{%
  \@ifnextchar\jcomp{\@jcomp@nested}{%
  \@ifnextchar\tmext{\@tmext@nested}{%
  \@ifnextchar\jvcomp{\@jvcomp@nested}{%
  \@ifnextchar\jfcomp{\@jfcomp@nested}{%
%  \@ifnextchar\mfam{\@mfam@nested}{%
%  \@ifnextchar\mtm{\@mtm@nested}}
\newcommand{\@ctxwk@parens}[2]{%
  (\ctxwk{#1}{#2})}
\newcommand{\@ctxwk@nested}[3]{%
  \@ctxwk@parens{#2}{#3}}
\makeatother

%%%% Not sure if we're gonna need the following.
\newcommand{\ctxwkop}[2]{%
  \ctxwk{#2}{#1}}
  
%%%%%%%%%%%%%%%%%%%%%%%%%%%%%%%%%%%%%%%%%%%%%%%%%%%%%%%%%%%%%%%%%%%%%%%%%%%%%%%%
%%%% IDENTITY TERMS

\makeatletter
\newcommand{\idtm}[1]{\typefont{id}_{\default@ctxext #1}}
\makeatother

%%%%%%%%%%%%%%%%%%%%%%%%%%%%%%%%%%%%%%%%%%%%%%%%%%%%%%%%%%%%%%%%%%%%%%%%%%%%%%%%
%%%% TERM EXTENSION
%%%%
%%%% The term extension command \tmext is slightly complicated because 
%%%% \tmext@unfold should do different things depending on whether it has two
%%%% or four arguments. Thus \tmext has a full form and a short form, where
%%%% the short form has two arguments and the full form has four. 

\makeatletter

%%%% The basic term extension commands
\newcommand{\default@tmext}{\@ifnextchar\bgroup{\tmext}{}}
\newcommand{\tmext}[2]{%
  \@ifnextchar\bgroup{\tmext@full{#1}{#2}}{\tmext@short{#1}{#2}}}
\newcommand{\tmext@full}[4]{%
  \ctxext{\tmext@testleft #3}{\tmext@testright #4}}
\newcommand{\tmext@short}[2]{%
  \ctxext{\tmext@testleft #1}{\tmext@testright #2}}
\newcommand{\tmext@testleft}{%
  \@ifnextchar\bgroup{\@tmext@parens}{%
  \@ifnextchar\tmext{\@tmext@nested}{%
  \@ifnextchar\ctxwk{\@ctxwk@nested}{%
  \@ifnextchar\jcomp{\@jcomp@nested}{%
  \@ifnextchar\jvcomp{\@jvcomp@nested}{%
  \@ifnextchar\jfcomp{\@jfcomp@nested}{%
%  \default@tmext
  }}}}}}}
\newcommand{\tmext@testright}{%
  \@ifnextchar\bgroup{\@tmext@parens}{%
  \@ifnextchar\tmext{\@tmext@nested}{%
  \@ifnextchar\subst{\@subst@nested}{%
  \@ifnextchar\jcomp{\@jcomp@nested}{%
  \@ifnextchar\jvcomp{\@jvcomp@nested}{%
  \@ifnextchar\jfcomp{\@jfcomp@nested}{%
  \@ifnextchar\cprojfst{\cprojfst@nested}{%
  \@ifnextchar\cprojsnd{\cprojsnd@nested}{%
%  \default@tmext
  }}}}}}}}}
\newcommand{\@tmext@nested}[1]{%
  \@tmext@parens}
\newcommand{\@tmext@parens}[2]{%
  \@ifnextchar\bgroup
    {\tmext@full@parens{#1}{#2}}
    {(\tmext@short{#1}{#2})}}
\newcommand{\tmext@full@parens}[4]{%
  (\tmext@full{#1}{#2}{#3}{#4})}

%%%% The unfolded term extension commands
\newcommand{\tmext@unfold}[2]{%
  \@ifnextchar\bgroup{\tmext@unfold@full{#1}{#2}}{\tmext@short{#1}{#2}}}
\newcommand{\tmext@unfold@full}[4]{%  
  \subst{#4}{{#3}{\idtm{\ctxext{#1}{#2}}}}}
\newcommand{\@tmext@unfold@nested}[1]{%
  \@tmext@unfold@parens}
\newcommand{\@tmext@unfold@parens}[4]{%
  (\tmext@unfold{#1}{#2}{#3}{#4})}
\makeatother

%%%%%%%%%%%%%%%%%%%%%%%%%%%%%%%%%%%%%%%%%%%%%%%%%%%%%%%%%%%%%%%%%%%%%%%%%%%%%%%%
%%%% JUDGMENTAL MORPHISMS

\makeatletter

%%%% The judgment that f is a morphism from A to B in context \Gamma.
\newcommand{\jhomsym}[3][]{%
  ~~\textit{hom}_{#1}(\default@ctxext #2,\default@ctxext #3)}
\newcommand{\jhom}{%
  \@ifnextchar*{\@jhomAlignTrue}{\@jhomAlignFalse}}
\newcommand{\@jhomAlignTrue}[5]{%
  \judgment*{#2}{}{}{#5}{}{}{\jhomsym{#3}{#4}}}
\newcommand{\@jhomAlignFalse}[4]{%
  \judgment{#1}{}{}{#4}{}{}{\jhomsym{#2}{#3}}}
\newcommand{\jhomeq}{%
  \@ifnextchar*{\@jhomeqAlignTrue}{\@jhomeqAlignFalse}}
\newcommand{\@jhomeqAlignTrue}[6]{%
  \judgment*{#2}{#5}{\jdeq}{#6}{}{}{\jhomsym{#3}{#4}}}
\newcommand{\@jhomeqAlignFalse}[5]{%
  \judgment{#1}{#4}{\jdeq}{#5}{}{}{\jhomsym{#2}{#3}}}
\newcommand{\jhomdefn}{%
  \@ifnextchar*{\@jhomdefnAlignTrue}{\@jhomdefnAlignFalse}}
\newcommand{\@jhomdefnAlignTrue}[6]{%
  \judgment*{#2}{#5}{\defeq}{#6}{}{}{\jhomsym{#3}{#4}}}
\newcommand{\@jhomdefnAlignFalse}[5]{%
  \judgment{#1}{#4}{\defeq}{#5}{}{}{\jhomsym{#2}{#3}}}

\newcommand{\jhom@unfold}[4]{%
  \jterm
    {{#1}{#2}}
    {\ctxwk{\default@ctxext #2}{\default@ctxext@parens #3}}
    {#4}}
\newcommand{\jhomeq@unfold}[5]{%
  \jtermeq
    {{#1}{#2}}
    {\ctxwk{\default@ctxext #2}{\default@ctxext@parens #3}}
    {#4}
    {#5}}
\newcommand{\jhomdefn@unfold}[5]{%
  \jtermdefn
    {{#1}{#2}}
    {\ctxwk{\default@ctxext #2}{\default@ctxext@parens #3}}
    {#4}
    {#5}}

%%%% Composition of morphisms
\newcommand{\jcomp}[3]{%
  \jcomp@testleft #3 \circ \jcomp@testright #2}
\newcommand{\jcomp@testleft}{%
  \@ifnextchar\jcomp{\@jcomp@nested}{%
  \@ifnextchar\ctxwk{\@ctxwk@nested}{%
  \@ifnextchar\ctxext{\@ctxext@nested}{%
  \@ifnextchar\bgroup{\@jcomp@parens}{%
  \@ifnextchar\tmext{\@tmext@nested}{%
  \@ifnextchar\jvcomp{\@jvcomp@nested}{%
  \@ifnextchar\jfcomp{\@jfcomp@nested}{%
  }}}}}}}}
\newcommand{\jcomp@testright}{%
  \@ifnextchar\jcomp{\@jcomp@nested}{%
  \@ifnextchar\subst{\@subst@nested}{%
  \@ifnextchar\ctxext{\@ctxext@nested}{%
  \@ifnextchar\bgroup{\@jcomp@parens}{%
  \@ifnextchar\tmext{\@tmext@nested}{%
  \@ifnextchar\jvcomp{\@jvcomp@nested}{%
  \@ifnextchar\jfcomp{\@jfcomp@nested}{%
  }}}}}}}}
\newcommand{\@jcomp@nested}[4]{%
  \@jcomp@parens{#2}{#3}{#4}}
\newcommand{\@jcomp@parens}[3]{%
  (\jcomp{#1}{#2}{#3})}

\newcommand{\jcomp@unfold}[3]{%
  \subst
    {\jcomp@unfold@test@preside #2}
    {\ctxwk{\default@ctxext #1}{\jcomp@unfold@test@postside #3}}}
\newcommand{\jcomp@unfold@test@preside}{%
  \@ifnextchar\bgroup{\@jcomp@unfold@parens}{}}
\newcommand{\jcomp@unfold@test@postside}{%
  \@ifnextchar\bgroup{\@jcomp@unfold@parens}{%
  \@ifnextchar\subst{\@subst@nested}{%
  }}}
\newcommand{\@jcomp@unfold@nested}[4]{%
  \@jcomp@unfold@parens{#2}{#3}{#4}}
\newcommand{\@jcomp@unfold@parens}[3]{%
  (\jcomp@unfold{#1}{#2}{#3})}

%%%% Vertical composition of morphisms.
\newcommand{\jvcomp}[3]{%
  \jcomp@testleft #2 * \jcomp@testright #3}
\newcommand{\jvcomp@testleft}{%
  \@ifnextchar\jvcomp{\@jvcomp@nested}{%
  \@ifnextchar\ctxwk{\@ctxwk@nested}{%
  \@ifnextchar\ctxext{\@ctxext@nested}{%
  \@ifnextchar\bgroup{\@jvcomp@parens}{%
  \@ifnextchar\tmext{\@tmext@nested}{%
  \@ifnextchar\jcomp{\@jcomp@nested}{%
  \@ifnextchar\jfcomp{\@jfcomp@nested}{%
  }}}}}}}}
\newcommand{\jvcomp@testright}{%
  \@ifnextchar\jvcomp{\@jvcomp@nested}{%
  \@ifnextchar\subst{\@subst@nested}{%
  \@ifnextchar\ctxext{\@ctxext@nested}{%
  \@ifnextchar\bgroup{\@jvcomp@parens}{%
  \@ifnextchar\tmext{\@tmext@nested}{%
  \@ifnextchar\jcomp{\@jcomp@nested}{%
  \@ifnextchar\jfcomp{\@jfcomp@nested}{%
  }}}}}}}}
\newcommand{\@jvcomp@nested}[4]{%
  \@jvcomp@parens{#2}{#3}{#4}}
\newcommand{\@jvcomp@parens}[3]{%
  (\jvcomp{#1}{#2}{#3})}

\newcommand{\jvcomp@unfold}[3]{%
  \tmext{}{}{\ctxwk{#1}{#2}}{#3}
  }
\newcommand{\jvcomp@unfold@test@preside}{%
  \@ifnextchar\bgroup{\@jvcomp@unfold@parens}{}}
\newcommand{\jvcomp@unfold@test@postside}{%
  \@ifnextchar\bgroup{\@jvcomp@unfold@parens}{}}
\newcommand{\@jvcomp@unfold@nested}[4]{%
  \@jvcomp@unfold@parens{#2}{#3}{#4}}
\newcommand{\@jvcomp@unfold@parens}[3]{%
  (\jvcomp@unfold{#1}{#2}{#3})}

%%%% The judgment that F is a morphism from P to Q over f in context \Gamma.
\newcommand{\jfhomsym}[3]{\jhomsym[{#1}]{#2}{#3}}
\newcommand{\jfhom}{%
  \@ifnextchar*{\jfhomAlignTrue}{\jfhomAlignFalse}}
\newcommand{\jfhomAlignTrue}[8]{
  \judgment*{#2}{}{}{#8}{}{}{\jfhomsym{#5}{#6}{#7}}}
\newcommand{\jfhomAlignFalse}[7]{
  \judgment{#1}{}{}{#7}{}{}{\jfhomsym{#4}{#5}{#6}}}
\newcommand{\jfhomeq}[8]{%
  \judgment{#1}{#7}{\jdeq}{#8}{}{}{\jhomsym[{#4}]{#5}{#6}}}
\newcommand{\jfhomdefn}[8]{%
  \judgment{#1}{#7}{\defeq}{#8}{}{}{\jhomsym[{#4}]{#5}{#6}}}
\newcommand{\jfhom@unfold}[7]{%
  \jterm
    {{{#1}{#2}}{#5}}
    {\ctxwk{\default@ctxext #5}{\jcomp{#2}{#4}{#6}}}
    {#7}}
    
\newcommand{\jfcomp}[5]{%
  \jfcomp@testleft #5 \bullet \jfcomp@testright #4}
\newcommand{\jfcomp@testleft}{%
  \@ifnextchar\bgroup{\@jfcomp@parens}{%
  \@ifnextchar\jfcomp{\@jfcomp@nested}{%
  \@ifnextchar\jcomp{\@jcomp@nested}{%
  \@ifnextchar\ctxwk{\@ctxwk@nested}{%
  \@ifnextchar\tmext{\@tmext@nested}{%
  \@ifnextchar\jvcomp{\@jvcomp@nested}{%
  }}}}}}}
\newcommand{\jfcomp@testright}{%
  \@ifnextchar\bgroup{\@jfcomp@parens}{%
  \@ifnextchar\jfcomp{\@jfcomp@nested}{%
  \@ifnextchar\jcomp{\@jcomp@nested}{%
  \@ifnextchar\subst{\@subst@nested}{%
  \@ifnextchar\tmext{\@tmext@nested}{%
  \@ifnextchar\jvcomp{\@jvcomp@nested}{%
  }}}}}}}
\newcommand{\@jfcomp@nested}[1]{%
  \@jfcomp@parens}
\newcommand{\@jfcomp@parens}[5]{%
  (\jfcomp{#1}{#2}{#3}{#4}{#5})}
  
\newcommand{\jfcomp@unfold}[5]{%
  \jcomp{#3}{#4}{{#1}{#2}{#5}}}
\makeatother

%%%%%%%%%%%%%%%%%%%%%%%%%%%%%%%%%%%%%%%%%%%%%%%%%%%%%%%%%%%%%%%%%%%%%%%%%%%%%%%%
%%%% JUDGMENTAL TRIVIAL COFIBRATIONS

\newcommand{\jtcext}{\tilde}

%%%%%%%%%%%%%%%%%%%%%%%%%%%%%%%%%%%%%%%%%%%%%%%%%%%%%%%%%%%%%%%%%%%%%%%%%%%%%%%%
%%%% CONTEXT PROJECTIONS

\makeatletter
\newcommand{\cprojgenf}[3]{%
  \typefont{pr}^{%
    \@ifnextchar\bgroup{\@ctxext@parens}{%
    \@ifnextchar\ctxext{\@ctxext@nested}{%
    }}
    #2,
    \@ifnextchar\bgroup{\@ctxext@parens}{%
    \@ifnextchar\ctxext{\@ctxext@nested}{%
    }}
    #3
    }_{#1}}
\newcommand{\cprojgen}[4]{%
  \subst{#4}{\cprojgenf{#1}{#2}{#3}}}
\newcommand{\cprojgenf@nested}[1]{%
  \cprojgenf@parens}
\newcommand{\cprojgenf@parens}[3]{%
  (\cprojgenf{#1}{#2}{#3})}
\newcommand{\cprojgen@nested}[1]{%
  \cprojgen@parens}
\newcommand{\cprojgen@parens}[4]{%
  (\cprojgen{#1}{#2}{#3}{#4})}

\newcommand{\cprojfstf}[2]{%
  \cprojgenf{0}{#1}{#2}}
\newcommand{\cprojfstf@nested}[1]{%
  \cprojfstf@parens}
\newcommand{\cprojfstf@parens}[2]{%
  (\cprojfstf{#1}{#2})}
\newcommand{\cprojfstf@unfold}[2]{%
  \ctxwk{\default@ctxext #2}\idtm{\default@ctxext #1}}

\newcommand{\cprojfst}[3]{%
  \cprojgen{0}{#1}{#2}{#3}}
\newcommand{\cprojfst@nested}[1]{%
  \cprojfst@parens}
\newcommand{\cprojfst@parens}[3]{%
  (\cprojfst{#1}{#2}{#3})}
\newcommand{\cprojfst@unfold}[3]{%
  \subst{#3}{(\cprojfstf@unfold{#1}{#2})}}

\newcommand{\cprojsndf}[2]{%
  \cprojgenf{1}{#1}{#2}}
\newcommand{\cprojsndf@nested}[1]{%
  \cprojsndf@parens}
\newcommand{\cprojsndf@parens}[2]{%
  (\cprojsndf{#1}{#2})}
\newcommand{\cprojsndf@unfold}[2]{%
  \idtm{\default@ctxext #2}}

\newcommand{\cprojsnd}[3]{%
  \cprojgen{1}{#1}{#2}{#3}}
\newcommand{\cprojsnd@nested}[1]{%
  \cprojsnd@parens}
\newcommand{\cprojsnd@parens}[3]{%
  (\cprojsnd{#1}{#2}{#3})}
\newcommand{\cprojsnd@unfold}[3]{%
  \subst{#3}{\cprojsnd@unfold{#1}{#2}}}
  
%%%% The sandwich function
\newcommand{\sandwich}[3]{\typefont{sw}^{#1,#2,#3}}
\newcommand{\sandwich@unfold}[3]{\typefont{sw}^{#1,#2,#3}}
\makeatother

%%%%%%%%%%%%%%%%%%%%%%%%%%%%%%%%%%%%%%%%%%%%%%%%%%%%%%%%%%%%%%%%%%%%%%%%%%%%%%%%
%%%% FIBER INCLUSIONS

\makeatletter
\newcommand{\finc}[2]{\typefont{in}^{#2}_{#1}}
\newcommand{\finc@unfold}[2]{\tmext{}{}{\ctxwk{\subst{x}{P}}{x}}{\idtm{\subst{x}{P}}}}
\makeatother

%%%%%%%%%%%%%%%%%%%%%%%%%%%%%%%%%%%%%%%%%%%%%%%%%%%%%%%%%%%%%%%%%%%%%%%%%%%%%%%%
%%%% THE UNIT TYPE

\makeatletter
\newcommand{\unitc}[1]{%
  \unit^0_{\default@ctxext #1}}
\newcommand{\unitct}[1]{%
  \ttt^0_{\default@ctxext #1}}
\newcommand{\unitf}[2]{%
  \unit^1_{\default@ctxext #1,\default@ctxext #2}}
\newcommand{\unitft}[2]{%
  \ttt^1_{\default@ctxext #1,\default@ctxext #2}}
\makeatother

%%%%%%%%%%%%%%%%%%%%%%%%%%%%%%%%%%%%%%%%%%%%%%%%%%%%%%%%%%%%%%%%%%%%%%%%%%%%%%%%
%%%% DEPENDENT FUNCTION TYPES

\makeatletter
\newcommand{\sprd}[2]{\Pi(\default@ctxext #1,\default@ctxext #2)}
\begin{comment}
\newcommand{\@sprd@test@cod}[2]{%
  \@ifnextchar\bgroup{\@sprd@do@cod{#1}}{%
  \Pi(\@sprd@test@dom{#1}{#2} #1,
  }}
\newcommand{\@sprd@do@cod}[4]{%
  \ctxext{\@sprd{#1}{#2}}{\@sprd{#1}{#3}}
  }
\newcommand{\@sprd}[2]{
  \@ifnextchar\bgroup{\@@sprd}{%
    \Pi(}
    #1,{#2})
  }
\newcommand{\@@sprd}[5]{%
  \sprd{#1}{\sprd{#2}{#4}}
  }
\end{comment}

\newcommand{\slam}[3]{%
  \lambda^{{\default@ctxext@parens #1},{\default@ctxext@parens #2}}
  (\default@ctxext #3)
  }
\newcommand{\sev}[1]{\tfev(#1)}

\makeatother

%%%%%%%%%%%%%%%%%%%%%%%%%%%%%%%%%%%%%%%%%%%%%%%%%%%%%%%%%%%%%%%%%%%%%%%%%%%%%%%%
%%%% NON-DEPENDENT FUNCTION TYPES

\newcommand{\jfun}[2]{#1\to#2}

%%%%%%%%%%%%%%%%%%%%%%%%%%%%%%%%%%%%%%%%%%%%%%%%%%%%%%%%%%%%%%%%%%%%%%%%%%%%%%%%
%%%% THE CONSTRUCTORS OF THE TYPE THEORY OF MODELS

\makeatletter
%%%% The initial model
\newcommand{\mctx}{%
  \mathcal{C}}

%%%% The family constructor
\newcommand{\mfam}[2][]{%
  \mathcal{F}_{\default@ctxext #2}^{#1}}
\newcommand{\@mfam@nested}[1]{\@mfam@parens}
\newcommand{\@mfam@parens}[2][]{(\mfam[#1]{#2})}

%%%% The terms constructor
\newcommand{\mtm}[2][]{%
  \mathcal{T}_{\default@ctxext #2}^{#1}}
\newcommand{\@mtm@nested}[1]{\@mtm@parens}
\newcommand{\@mtm@parens}[2][]{(\mtm[#1]{#2})}

%%%% The empty type constructor
\newcommand{\tfemp}[1]{%
  \typefont{emp}_{\default@ctxext #1}}
\newcommand{\tft}[1]{%
  \typefont{t}_{\default@ctxext #1}}

%%%% The extension constructor
\newcommand{\tfext}[1]{%
  \typefont{ext}_{\default@ctxext #1}}

%%%% The substitution constructor
\newcommand{\tfsubst}[1]{%
  \typefont{subst}_{\default@ctxext #1}}
  
%%%% The weakening constructor
\newcommand{\tfwk}[1]{%
  \typefont{wk}_{\default@ctxext #1}}

%%%% The identity function constructor
\newcommand{\tfid}[1]{%
  \typefont{idtm}_{\default@ctxext #1}}
\makeatother

%%%%%%%%%%%%%%%%%%%%%%%%%%%%%%%%%%%%%%%%%%%%%%%%%%%%%%%%%%%%%%%%%%%%%%%%%%%%%%%%

%%%% Introducing logical usage of fonts.
\newcommand{\modelfont}{\mathit} % use 'mf' in command to indicate model font
\newcommand{\typefont}{\mathsf} % use 'tf' in command to indicate type font
\newcommand{\catfont}{\mathrm} % use 'cf' in command to indicate cat font

%%%%%%%%%%%%%%%%%%%%%%%%%%%%%%%%%%%%%%%%%%%%%%%%%%%%%%%%%%%%%%%%%%%%%%%%%%%%%%%%
%%%% Some macros of the book are redefined.

\renewcommand{\UU}{\typefont{U}}
\renewcommand{\isequiv}{\typefont{isEquiv}}
\renewcommand{\happly}{\typefont{hApply}}
\renewcommand{\pairr}[1]{{\mathopen{}\langle #1\rangle\mathclose{}}}
\renewcommand{\type}{\typefont{Type}}
\renewcommand{\op}[1]{{{#1}^\typefont{op}}}
\renewcommand{\susp}{\typefont{\Sigma}}

%%%%%%%%%%%%%%%%%%%%%%%%%%%%%%%%%%%%%%%%%%%%%%%%%%%%%%%%%%%%%%%%%%%%%%%%%%%%%%%%
%%%% The following is a big unorganized list of new macros that we use in the
%%%% notes. 

\newcommand{\tfW}{\typefont{W}}
\newcommand{\tfM}{\typefont{M}}
\newcommand{\mfM}{\modelfont{M}}
\newcommand{\mfN}{\modelfont{N}}
\newcommand{\tfctx}{\typefont{ctx}}
\newcommand{\mftypfunc}[1]{{\modelfont{typ}^{#1}}}
\newcommand{\mftyp}[2]{{\mftypfunc{#1}(#2)}}
\newcommand{\tftypfunc}[1]{{\typefont{typ}^{#1}}}
\newcommand{\tftyp}[2]{{\tftypfunc{#1}(#2)}}
\newcommand{\hfibfunc}[1]{\typefont{fib}_{#1}}
\newcommand{\mappingcone}[1]{\mathcal{C}_{#1}}
\newcommand{\equifib}{\typefont{equiFib}}
\newcommand{\tfcolim}{\typefont{colim}}
\newcommand{\tflim}{\typefont{lim}}
\newcommand{\tfdiag}{\typefont{diag}}
\newcommand{\tfGraph}{\typefont{Graph}}
\newcommand{\mfGraph}{\modelfont{Graph}}
\newcommand{\unitGraph}{\unit^\mfGraph}
\newcommand{\UUGraph}{\UU^\mfGraph}
\newcommand{\tfrGraph}{\typefont{rGraph}}
\newcommand{\mfrGraph}{\modelfont{rGraph}}
\newcommand{\isfunction}{\typefont{isFunction}}
\newcommand{\tfconst}{\typefont{const}}
\newcommand{\conemap}{\typefont{coneMap}}
\newcommand{\coconemap}{\typefont{coconeMap}}
\newcommand{\tflimits}{\typefont{limits}}
\newcommand{\tfcolimits}{\typefont{colimits}}
\newcommand{\islimiting}{\typefont{isLimiting}}
\newcommand{\iscolimiting}{\typefont{isColimiting}}
\newcommand{\islimit}{\typefont{isLimit}}
\newcommand{\iscolimit}{\typefont{iscolimit}}
\newcommand{\pbcone}{\typefont{cone_{pb}}}
\newcommand{\tfinj}{\typefont{inj}}
\newcommand{\tfsurj}{\typefont{surj}}
\newcommand{\tfepi}{\typefont{epi}}
\newcommand{\tftop}{\typefont{top}}
\newcommand{\sbrck}[1]{\Vert #1\Vert}
\newcommand{\strunc}[2]{\Vert #2\Vert_{#1}}
\newcommand{\gobjclass}{{\typefont{U}^\mfGraph}}
\newcommand{\gcharmap}{\typefont{fib}}
\newcommand{\diagclass}{\typefont{T}}
\newcommand{\opdiagclass}{\op{\diagclass}}
\newcommand{\equifibclass}{\diagclass^{\eqv{}{}}}
\newcommand{\universe}{\typefont{U}}
\newcommand{\catid}[1]{{\catfont{id}_{#1}}}
\newcommand{\isleftfib}{\typefont{isLeftFib}}
\newcommand{\isrightfib}{\typefont{isRightFib}}
\newcommand{\leftLiftings}{\typefont{leftLiftings}}
\newcommand{\rightLiftings}{\typefont{rightLiftings}}
\newcommand{\psh}{\typefont{Psh}}
\newcommand{\rgclass}{\typefont{\Omega^{RG}}}
\newcommand{\terms}[2][]{\lfloor #2 \rfloor^{#1}}
\newcommand{\grconstr}[2]
             {\mathchoice % max size is textstyle size.
             {{\textstyle \int_{#1}}#2}% 
             {\int_{#1}#2}%
             {\int_{#1}#2}%
             {\int_{#1}#2}}
\newcommand{\ctxhom}[3][]{\typefont{hom}_{#1}(#2,#3)}
\newcommand{\graphcharmapfunc}[1]{\gcharmap_{#1}}
\newcommand{\graphcharmap}[2][]{\graphcharmapfunc{#1}(#2)}
\newcommand{\tfexp}[1]{\typefont{exp}_{#1}}
\newcommand{\tffamfunc}{\typefont{fam}}
\newcommand{\tffam}[1]{\tffamfunc(#1)}
\newcommand{\tfev}{\typefont{ev}}
\newcommand{\tfcomp}{\typefont{comp}}
\newcommand{\isDec}[1]{\typefont{isDecidable}(#1)}
\newcommand{\smal}{\mathcal{S}}
\renewcommand{\modal}{{\ensuremath{\ocircle}}}
\newcommand{\eqrel}{\typefont{EqRel}}
\newcommand{\piw}{\ensuremath{\Pi\typefont{W}}} %% to be used in conjunction with -pretopos.
\renewcommand{\sslash}{/\!\!/}
\newcommand{\mprd}[2]{\Pi(#1,#2)}
\newcommand{\msm}[2]{\Sigma(#1,#2)}
\newcommand{\midt}[1]{\idvartype_#1}
\newcommand{\reflf}[1]{\typefont{refl}^{#1}}
\newcommand{\tfJ}{\typefont{J}}
\newcommand{\tftrans}{\typefont{trans}}

\newcommand{\tfT}{\typefont{T}}
\newcommand{\reflsym}{{\mathsf{refl}}}
\newcommand{\strans}[2]{\ensuremath{{#1}_{*}({#2})}}
\newcommand{\eqtype}[1]{\typefont{Eq}_{#1}}
\newcommand{\eqtoid}[1]{\typefont{eqtoid}(#1)}
\newcommand{\greek}{\mathrm}
\newcommand{\product}[2]{{#1}\times{#2}}
\newcommand{\pairp}[1]{(#1)}
\newcommand{\jequalizer}[3]{\{#1|#2\jdeq #3\}}
\newcommand{\jequalizerin}[2]{\iota_{#1,#2}}
\newcommand{\tounit}[1]{{!_{#1}}}
\newcommand{\trwk}{\typefont{trwk}}
\newcommand{\trext}{\typefont{trext}}

%%%%%%%%%%%%%%%%%%%%%%%%%%%%%%%%%%%%%%%%%%%%%%%%%%%%%%%%%%%%%%%%%%%%%%%%%%%%%%%%
%%%% When investigation pointed structures we use the \pt macro.

\makeatletter
\newcommand{\pt}[1][]{*_{
  \@ifnextchar\undergraph{\@undergraph@nested}
    {\@ifnextchar\underovergraph{\@underovergraph@nested}{}}#1}}
\makeatother

%%%%%%%%%%%%%%%%%%%%%%%%%%%%%%%%%%%%%%%%%%%%%%%%%%%%%%%%%%%%%%%%%%%%%%%%%%%%%%%%
%%%% OPERATIONS ON GRAPHS
%%%%
%%%% First of all, each graph has a type of vertices and a type of edges. The
%%%% type of vertices of a graph $\Gamma$ is denoted by $\pts{\Gamma}$;
%%%% and likewise for the type of edges.

\makeatletter
\newcommand{\pts}[1]{{\@graphop@nested{#1}}_{0}}
\newcommand{\edg}[1]{{\@graphop@nested{#1}}_{1}}
\newcommand{\@graphop@nested}[1]
  {\@ifnextchar\ctxext{\@ctxext@nested}
      {\@ifnextchar\undergraph{\@undergraph@nested}
         {\@ifnextchar\underovergraph{\@underovergraph@nested}{}}}
    #1}
\makeatother

%%%% The following operations of \undergraph and \underovergraph are used to
%%%% define the free category and the free groupoid of a graph, respectively

\makeatletter
\newcommand{\@undergraphtest}[2]{\@ifnextchar({#1}{#2}}
\newcommand{\undergraph}[2]{\@undergraphtest{\@undergraph@parens{#1}{#2}}{\@undergraph{#1}{#2}}}
\newcommand{\@undergraph}[2]{{#2/#1}}
\newcommand{\@undergraph@nested}[3]{\@undergraph@parens{#2}{#3}}
\newcommand{\@undergraph@parens}[2]{(\@undergraph{#1}{#2})}
\makeatother

\makeatletter
\newcommand{\underovergraph}[2]{\@underovergraphtest{\@underovergraph@parens{#1}{#2}}{\@underovergraph{#1}{#2}}}
\newcommand{\@underovergraph}[2]{{#2}\,{\parallel}\,{#1}}
\newcommand{\@underovergraphtest}{\@undergraphtest}
\newcommand{\@underovergraph@parens}[2]{(\@underovergraph{#1}{#2})}
\newcommand{\@underovergraph@nested}[3]{\@underovergraph@parens{#2}{#3}}
\makeatother

\newcommand{\graphid}[1]{\mathrm{id}_{#1}}
\newcommand{\freecat}[1]{\mathcal{C}(#1)}
\newcommand{\freegrpd}[1]{\mathcal{G}(#1)}


%%%%%%%%%%%%%%%%%%%%%%%%%%%%%%%%%%%%%%%%%%%%%%%%%%%%%%%%%%%%%%%%%%%%%%%%%%%%%%%%
%% Some tikz macros to typeset diagrams uniformly.

\tikzset{patharrow/.style={double,double equal sign distance,-,font=\scriptsize}}
\tikzset{description/.style={fill=white,inner sep=2pt}}
\tikzset{fib/.style={->>,font=\scriptsize}}

%% Used for extra wide diagrams, e.g. when the label is too large otherwise.
\tikzset{commutative diagrams/column sep/Huge/.initial=18ex}

%%%%%%%%%%%%%%%%%%%%%%%%%%%%%%%%%%%%%%%%%%%%%%%%%%%%%%%%%%%%%%%%%%%%%%%%%%%%%%%%
%%%% New theorem environment for conjectures.

\defthm{conj}{Conjecture}{Conjectures}

%%%%%%%%%%%%%%%%%%%%%%%%%%%%%%%%%%%%%%%%%%%%%%%%%%%%%%%%%%%%%%%%%%%%%%%%%%%%%%%%
%%%% The following environment for desiderata should not be there. It is better
%%%% to use the issue tracker for desiderata.

\newenvironment{desiderata}{\begingroup\color{blue}\textbf{Desiderata.}}
{\endgroup}

%%%%%%%%%%%%%%%%%%%%%%%%%%%%%%%%%%%%%%%%%%%%%%%%%%%%%%%%%%%%%%%%%%%%%%%%%%%%%%%%
%%%% The following piece of code from tex.stackexchange:
%%%%
%%%% http://tex.stackexchange.com/a/55180/14653
%%%%
%%%% We include it so that inference rules in align environments have enough
%%%% vertical space.

\newlength\minalignvsep

\makeatletter
\def\align@preamble{%
   &\hfil
    \setboxz@h{\@lign$\m@th\displaystyle{##}$}%
    \ifnum\row@>\@ne
    \ifdim\ht\z@>\ht\strutbox@
    \dimen@\ht\z@
    \advance\dimen@\minalignvsep
    \ht\strutbox\dimen@
    \fi\fi
    \strut@
    \ifmeasuring@\savefieldlength@\fi
    \set@field
    \tabskip\z@skip
   &\setboxz@h{\@lign$\m@th\displaystyle{{}##}$}%
    \ifnum\row@>\@ne
    \ifdim\ht\z@>\ht\strutbox@
    \dimen@\ht\z@
    \advance\dimen@\minalignvsep
    \ht\strutbox@\dimen@
    \fi\fi
    \strut@
    \ifmeasuring@\savefieldlength@\fi
    \set@field
    \hfil
    \tabskip\alignsep@
}
\makeatother

\minalignvsep.2em

\allowdisplaybreaks

%%%%%%%%%%%%%%%%%%%%%%%%%%%%%%%%%%%%%%%%%%%%%%%%%%%%%%%%%%%%%%%%%%%%%%%%%%%%%%%%

\setdescription[1]{itemsep=-0.2em}


%%%%%%%%%%%%%%%%%%%%%%%%%%%%%%%%%%%%%%%%%%%%%%%%%%%%%%%%%%%%%%%%%%%%%%%%%%%%%%%%
\title{Internal models of type theory and internal higher categories for the
Univalent Foundations}
\author{Egbert Rijke}
\date{\today}

\begin{document}

\maketitle

\begin{comment}
\begin{abstract}
A project is proposed where we investigate notions of internal models of type
theory and internal higher categories. We take the point of view that an
internal model is an internal higher category with extra structure interpreting
the type constructors. We propose to (1) explore possible definitions of both
notions, (2) find a zoo of examples and (3) find extensions of the underlying
theory and models thereof to include higher inductive types.
\end{abstract}
\end{comment}

\tableofcontents

\begin{comment}
\part{Introduction}
The project proposal is described in \autoref{stage1} and \autoref{stage2}. In
the appendices we give elaborations on the topics discussed in the proposal itself.
By including these we intend to clarify the proposal and present some of the 
(unfinished) work that is already done.

\section{Stage one: establishing internal higher categories and internal models}\label{stage1}
The project I propose here has its origins in the beginning of 2013, when I proved a version
of the descent theorem for homotopy colimits in type theory while I was working
with Bas Spitters to develop notions from higher category theory in the
univalent foundations. To arrive at a
notion of diagram general enough to capture all the higher inductive types described
in chapter 6 of \cite{TheBook} excluding the truncations we needed type
theoretical graphs. The graphs form a model of type theory and indeed we needed
the interpretations of several of the basic type constructors to give an efficient approach
to the descent property and its proof. Although it was not an issue to describe the graph model
and the sense in which it models the type constructors, 
not all of type theory is interpreted very well. To start with, if $A$ and $B$ 
are families of graphs over a graph $\Gamma$, then $B$ isn't also a family of 
graphs over the extended graph $\ctxext{\Gamma}{A}$ (interfering with a good 
interpretation of weakening). Since there are such issues, it is actually
not so straightforward to describe what it models!

Apart from all the type constructors like dependent function and pair types and
identity types, the graph model shows three basic operations that should come
before those type constructors. They are extension, weakening and substitution.
All three of them can be defined such that they act not only on contexts (the graphs), 
but also on families and terms. Moreover, they are all compatible with each 
other. In fact, any structure which has a notion of contexts, families and terms
and which models extension, weakening and substitution and which has identity
functions is in a weak sense a category, hypothesizing that
\begin{quote}
\emph{Category theory is dependent type theory without type constructors.}
\end{quote}
In \autoref{tt} we sketch how type theory without type constructors but with
\emph{explicit} extension, weakening and substitution might look like. This
theory is written down keeping the guiding principle in mind that type theory
without basic constructors should be a generalized algebraic theory of higher
categories.

Thus, models of this theory should be higher categories with some strictness
built in, such as the strictness of associativity of substitution. But the
theory really only talks about contexts, families and sections. The interpretation
of this as a category works as follows: The objects are going to be the contexts.
Then we may consider the terms of the weakening $\ctxwk{\Gamma}{\Delta}$ of a
context $\Delta$ by a context $\Gamma$: they are the morphisms of the category.
The composition of morphisms follows from the action of substitution and
weakening on terms. 

In the proposed approach, a model of type theory isn't a category by assumption. Rather,
a model of type theory \emph{becomes} a category because it interprets type
theory. In other words, type theory does all the work for us concerning the
categorical structure. In particular, we have neither started our notion of models
with AKS-categories as described in \cite{1categories} nor with Dybjer's notion
of internal categories with families \cite{Dybjer1996}. We should gain that we get rid of
truncatedness assumptions that are present in both AKS-categories, where the
type of objects has to be a $1$-type, and internal categories with families,
where the type of morphisms has to be a setoid. It needs not much arguing that any way to impose
conditions on internal models beyond the interpretations of the rules of
type theory stands in the way of a clear understanding of what an internal
model should be and it is a main goal of the project to design
an internal type theory without the described deficiencies. 

\begin{comment}
\subsection{Ideas in the definition of internal models}
An internal model of type theory is like a category with families, but we want
to avoid having to state higher coherences. In fact, we don't even start our
definition with a category of contexts; instead we just take a \emph{type} of contexts. 
The morphisms will come from the terms, evaluation of a function at a given
term will come from substitution. We recognize three basic ingredients to models:
first there is a type of contexts; second, for every context there is a model of types in
that context and third, for every type in a given context there is a type of its
terms. Then there are three basic attributes: context extension, weakening and
substitution. Context extension provides us with families over types as well as
with an interpretation of dependent pair types. We need weakening 
so that families can depend on the same type multiple times (the way the
identity type of a type depends two times on that type) and to be able
to talk about non-dependent function types,
the morphims of our category. Substitution will give us a way
to work with fibers of families as well as composition of functions and evaluation
of functions at terms.

Because we require a \emph{model} of types in a context, all the structure
which we require at the bottom level will be required to exist higher up as well.
Thus, the model of types in a given context $\Gamma$ will have a type of contexts
itself, which can be seen as the type of types in $\Gamma$; it will have its
own notion of types in a context, its own notion of terms, context extension,
weakening and substitution together with all the structure require for it. For
instance, when $A$ is a type in context $\Gamma$ in a model $\mfM$, then there
is the model of types in context $A$, which is the model of families over $A$. 
This model is required to be \emph{definitionally equal to} the model of types
in the context $\ctxext{\Gamma}{A}$, the context extension of $\Gamma$ and $A$.
In this way we protect ourselves from the need to dig an infinitely deep structure
of models when we want to consider examples.

To give the definition of a model we shall also need to consider certain morphisms
of internal models. Those should preserve all the structure: contexts are mapped
to contexts; for every context a morphism of models mapping the model of types
in that context to the model of types in the image of that context; there should
be a mapping of terms and context extension, weakening and substitution should be
preserved. We need to consider those morphisms because we require context extension,
weakening and substitution to be of that kind, thereby respecting each other
in all possible ways.

When we have this framework set up, we can interpret the basic type constructors
such as $\Pi$, $\Sigma$ and $\idtypevar{}$.
The higher categorical structure then comes from the
result that we have an interpretation of type theory.
\end{comment}

Of course there is no way of studying internal models or categories without
also studying the morphisms between them. In fact, we already have three of them
right from the start: the interpretations of extension, weakening and substitution
act on contexts, families and terms and preserve each of extension, weakening and
substitution. These have to be ingredients of morphisms of models as well. A
morphism of models should act on contexts because those are the contexts; it
should act on terms because that is how it acts on the morphisms; it should
preserve extension, weakening and substitution because that is exactly what
makes it functorial (in particular: preserve composition).

In fact, now that we have arrived at the idea that categorical structure should
come from the interpretation of type theory, the prettier (and necessary) way to go is to
describe a model of models of type theory. Thus, we should find a way to express
the notions of dependent model and terms thereof and explain what extension,
weakening and substitution mean. Then we should be able to derive that a morphism
of models coincides with a term of a weakened object.

The idea that morphisms of models preserve the basic type theoretical structure
can be carried further. Ideally, we would have the interpretations of the type
constructors as morphisms of models. Thus, when interpreting the operations 
$\Pi$, $\Sigma$, $\idtypevar{}$ or the universe operator 
(such as described in \cite{Palmgren1998}) we will require that
they act not only on objects, but also on families, terms and preserve extension,
weakening and substitution. And when two or more type constructors are interpreted,
they should furthermore be compatible with each other. In the case of
dependent pair types and identity types, this means function extensionality. It
should be investigated whether the compatibility of identity types and the
universe operator with each other indeed means univalence. We should note,
however, that we intend the compatibility with extension, weakening and substitution
should be strict, whereas the compatibility with the other type constructors will
not be strict. To give an exact meaning to these ideas is part of the proposal.

Among the examples of internal models we should have:
\begin{description}
\item[The setoid model] This is a classical one so it should be there. It might
      be a bit different in our case. We'll want to construct a setoid model
      of the basic type theory to interpret identity types without necessarily
      interpreting dependent function types (but that should remain possible).
\item[The graph model] This is the model that leads to the first version of the
      descent theorem. There are at least four variations of the graph model.
      The canonical graph model has the usual dependent graphs as families. We
      could also take left and right fibrations of graphs, although we loose
      interpretations of some type constructors by doing that. However, if we
      take the equifibered families as families we seem to get a model of type
      theory with the usual type constructors.
\item[Univalent unvirses] A univalent universe should be a model.
\item[Equifibered diagrams over a graph] should be the fibrations in some model.
\item[The model of all models] The objects in the model of models should be
      the models of type theory itself. There should be various flavors of the
      model of models. In the case of type theory without type constructors,
      this would be the model of all (small) higher categories. The model of
      models of type theory with several type constructors should interpret
      these type constructors as well. (This might get interesting in the case
      of type theory with a universe operator.)
\item[The model of weak $\omega$-groupoids] Since we think of models of the
      basic type theory as weak $\omega$-categories, it is not too hard to
      provide a condition on those models which enables us to talk about
      weak $\omega$-groupoids. In fact, we might have several options for such a
      condition. We investigate those; we also investigate how they relate to
      Brunerie's weak $\omega$-groupoids. Presumably, they form an internal
      model without too much fuss (because we already have an internal model
      of internal models at this stage) and we can ask whether this models
      the univalence axiom. In fact, the model of internal models might already
      have modeled the univalence axiom. 
\item[The set model] See \cite{RijkeSpitters:Sets}.
\item[Polynomial functors] The theory of polynomial functors is essential to the
      theory of $\tfW$-types, see \cite{MoerdijkPalmgren2000}. The theory of
      polynomial functors has been extensively developed by Kock \cite{Kock2011}.
      Polynomial functors are generalizations of spans and graphs are endospans,
      so a model of polynomial functors should generalize the graph model. 
\end{description}


\section{Stage two: extending the basic theory}\label{stage2}
With a firm notion of internal models of type theory and several extensions of
type theory come within reach. The first of these extensions, colimits for
diagrams over graphs, we have already started to explore. The operation 
assigning the colimit to a graph is compatible with $\Sigma$ and $\idtypevar{}$
types. The first was compatibility result was originally part of the descent
theorem. We take the point of view that the descent theorem -- which asserts that
for any graph $\Gamma$, the type of equifibered families over $\Gamma$ is
equivalent in a canonical way to the type of families over $\tfcolim(\Gamma)$ --
is just one of the aspects of many compatibility results for colimits. In this
formulation, the descent theorem tells that the colimit operation acts not only
on graphs (objects) but also on families (the equifibered diagrams over graphs)
of the model of equifibered families.
The colimit operation acts furthermore on terms and is compatible with extension,
weakening and substitution. The fact that it is compatible with identity types is a new result
which essentially expresses that the initial pointed equifibered diagram over
a pointed graph (which is just another way of considering the notion of universal
cover) describes
the identity type of the colimit with respect to the base point. The methods
in this proof actually describe all the higher identity types of the colimit
as a certain higher inductive type; in particular it gives a description of the
loop spaces of the higher inductive types appearing in \cite{TheBook}, so we
should figure out which algebraic data we can extract from this description.
In summary, this group of results related to the descent theorem may be viewed as the
key to a good description of the colimit operation.

These results need to be extended. In the
first place we could ask ourselves whether the colimit operation is compatible
with $\Pi$ and the universe operation. Perhaps more importantly, the descent
theorem and the related group of results stating that the colimit operation
is compatible with various type constructors needs to be generalized to (diagrams
over) arbitrary categories. Since we have a proposal for what a category is,
namely a model of type theory without basic constructors, we seem to have the
right tools at hand.  We envision that the colimit operation becomes a type theoretical operation
(together with compatibility results) assigning a type to a diagram over a category.
In the presence of a universe operator we might also consider to precompose
the colimit operation with the universe operator, so that it acts on all families.

The result should be a fairly general theory of colimits and one of the first
applications we must seek for is the establishment of $\im(f)$ as the colimit
of the appropriate category, given a function $f:A\to B$ where $A$ and $B$
are arbitrary types. In the case where $A$ and $B$ are sets, $\im(f)$ is
the set-colimit of $\sm{x,y:A}\id{f(x)}{f(y)}\rightrightarrows A$; in general
this diagram should be more complicated, see \S 6.1 of \cite{lurie2009higher}.
It should be possible to describe, given $f:A\to B$, a diagram $K_f$ such that
its colimit is $\im(f)$ generally. In particular we shouldn't need to take the
set-colimit in the case that $f$ is a function between sets.  When we have 
come this far we should also look for a possible definition of a
(weak) higher predicative topos; the ingredients seem to be there. We hope that
this gives us access to a description of sheaf models of type theory and we
can possibly begin exploring some internal independence proofs.


%\section{Empirical evidence of what an internal model should be}\label{eg}

\subsection{Univalent universes as internal models}

Before we give the definition we illustrate the concepts that go into it in
the case of a univalent universe $\UU$.
Regardless of the definition of an internal model, a univalent universe should
be an example of one.

The type of contexts is $\UU$ itself and for every $\Gamma:\UU$, a type in
context $\Gamma$ is simply a family $A:\Gamma\to\UU$. The type $\terms{A}$ of terms of a type $A$
in context $\Gamma$ is defined to be $\prd{i:\Gamma}A(i)$. Note that $\Gamma\to
\UU$ itself also interprets type theory where a type in context $A:\Gamma\to\UU$
is a family $P:(\sm{i:\Gamma}A(i))\to\UU$. We may denote this model by
$\mftyp{\UU}{\Gamma}$. The terms of a type $P$ in context $A$ in the model
$\Gamma\to\UU$ are the terms of $\prd{i:\Gamma}{x:A(i)}P(i,x)$.

When $A$ is a type in context
$\Gamma$ we define the context extension $\ctxext{\Gamma}{A}$ to be
$\sm{i:\Gamma}A(i)$. Note that context extension $\ctxext{\Gamma}{\blank}$
can be seen acting not only as a function from the context of the model
$\Gamma\to\UU$ to the context of the model $\UU$, but it also acts on the types 
and terms: when $A$ is a type in context $\Gamma$ we may take the identity map
from $\mftyp{\Gamma\to\UU}{A}\to\mftyp{\UU}{\ctxext{\Gamma}{A}}$ because
$\mftyp{\Gamma\to\UU}{A}$ is taken to be $\mftyp{\UU}{\ctxext{\Gamma}{A}}$;
thus, $\ctxext{\Gamma}{P}\defeq P$ for every family $P$ over $A$ in context 
$\Gamma$. Likewise, when $P$ is a type in context $A$ in the model 
$\Gamma\to\UU$ then context extension should act on the terms of $P$ via a
function $\terms{P}\to\terms{\ctxext{\Gamma}{P}}$, which we take to be
the identity map once more.

Since $\mftyp{\UU}{\Gamma}$ is a model of type theory it
has it's own notion of context extension: when $P$ is a type in context $A$ in
the model $\mftyp{\UU}{\Gamma}$ then $\ctxext{A}{P}$ is the family
$\lam{i}\sm{x:A(i)}P(i,x):\Gamma\to\UU$. Also the context extension of
$\Gamma\to\UU$ acts trivially on the types and terms. Context extension is
analoguous to the Grothendieck construction that associates the category of
elements to a presheaf and it gives us $\Sigma$-types.

When $A$ and $B$ are types in context
$\Gamma$, the weakening $\ctxwk{A}{B}$ of $B$ along $A$ is defined to be
$\lam{\pairr{i,x}}B(i):(\sm{i:\Gamma}A(i))\to\UU$. Weakening along a type $A$ 
in context $\Gamma$ also acts on types and terms. When $Q$ is a type in context
$B$ in the model $\Gamma\to\UU$, we define $\ctxwk{A}{Q}$ to be 
$\lam{\pairr{i,x}}{y}Q(i,y)$. When $g:\terms{Q}$ we define $\ctxwk{A}{g}$ to
be $\lam{\pairr{i,x}}{y}g(i,y)$.

Note that for two types $A$ and $B$ in context $\Gamma$, the terms of
$\ctxwk{A}{B}$ are the terms of $\prd{i:\Gamma}A(i)\to B(i)$, i.e.~they are
the fiberwise maps from $A$ to $B$. We shall take the type $\terms{\ctxwk{A}{B}}$
of terms of $\ctxwk{A}{B}$ to be the type of morphisms from $A$ to $B$. Also,
any context of $\UU$ may be seen as a type in the empty context $\unit$. Thus
the type of terms of a context $\Gamma$ is $\unit\to\Gamma$, which is
equivalent to $\Gamma$. A context morphism from $\Delta$ to $\Gamma$ is a term
of $\ctxwk{\Gamma}{\Delta}$, i.e.~a function from $\Gamma$ to $\Delta$. We denote
the type $\terms{\ctxwk{\Gamma}{\Delta}}$ by $\ctxhom{\Delta}{\Gamma}$. 

When $P$ is a family over
$A$ in context $\Gamma$ and $x$ is a term of $A$ we define the type $\subst{x}{P}$
in context $\Gamma$ to be $\lam{i}P(i,x(i))$. Like extension 

We will interpret the dependent product $\mprd{A}{P}$ of a family $P$ over
$A$ in context $\Gamma$ by
\begin{equation*}
\mprd{A}{P}(i)\defeq \prd{x:A(i)}P(i,x)
\end{equation*}
With this interpretation there is an equivalence 
$\lambda:\eqv{\terms{P}}{\terms{\mprd{A}{P}}}$. This is lambda-abstraction; its
inverse being evaluation. We should note however, that the rule for evaluation
we interpret here is
\begin{equation*}
\inference{\Gamma\vdash f:\mprd{A}{P}}{\ctxext{\Gamma}{A}\vdash\tfev(f):P}
\end{equation*}
which is different than the usual rule
\begin{equation*}
\inference{\Gamma\vdash f:\mprd{A}{P} \qquad \Gamma\vdash a:A}{\Gamma \vdash \tfev(f,a):\subst{a}{P}}
\end{equation*}
The reason for this is that the interpretation of the usual rule would give a
function of type $\terms{\mprd{A}{P}}\to\prd{x:\terms{A}}\terms{\subst{x}{P}}$,
but this does not describe the terms of $\mprd{A}{P}$ by any means. Moreover,
since we have implemented substitution, we obtain from $\ctxext{\Gamma}{A}\vdash\tfev(f):P$ 
and $\Gamma\vdash x:A$ a term $\Gamma\vdash\subst{x}{\tfev(f)}:\subst{x}{P}$ and
therefore we do not loose anything with this approach.

{\color{red} point to example}

\subsection{The graph model of type theory}
In this section we define the graph model of type theory, denoted by
$\mfGraph$.  In our presentation, we follow that of the definition internal
models. After we have established the graph model, we show how the graph
model can be seen as a presheaf model, namely over the category $\cdot
{\rightrightarrows}\cdot$.

\begin{defn}
A \emph{(directed) graph} $\Gamma$ is a pair $\pairr{\Gamma_0,\Gamma_1}$ 
consisting of a type $\Gamma_0$ of vertices and a family 
$\Gamma_1:\Gamma_0\to\Gamma_0\to\type$ of edges. The type $\ctx(\mfGraph)$
is defined to be the type of all graphs; we will usually denote it by
$\tfGraph$. Explicitly, we have
\begin{equation*}
\tfGraph\defeq\sm{\Gamma_0:\type}\Gamma_0\to\Gamma_0\to\type.
\end{equation*}
\end{defn}

\begin{eg}\label{ex:pb}
The underlying graph of the diagram
\begin{equation*}
\begin{tikzcd}
{} & A \ar{d}{f} \\
B \ar{r}[swap]{g} & C
\end{tikzcd}
\begin{comment}
\begin{tikzpicture}
\matrix (m) [std] { & A \\ B & C \\};
\draw[ar] (m-1-2) -- node[right] {$f$} (m-2-2);
\draw[ar] (m-2-1) -- node[below] {$g$} (m-2-2);
\end{tikzpicture}
\end{comment}
\end{equation*}
is $I\defeq\mathbf{3}$ and has $J(1,3)\defeq J(2,3)\defeq \unit $ 
and $J(x,y)\defeq\emptyt$ otherwise (it is defined using the induction 
principle of $\mathbf{3}$ and a universe).
\end{eg}

\begin{defn}
A \emph{family $A$ of graphs in the context $\Gamma$} is a pair $\pairr{A_0,A_1}$ consisting
of 
\begin{align*}
A_0 & :\Gamma_0\to\type\\
A_1 & :\prd*{i,j:\Gamma_0}\Gamma_1(i,j)\to A_0(i)\to A_0(j)\to\type.
\end{align*}
Thus, for a graph $\Gamma$, the type $\mftyp{\mfGraph}{\Gamma}$ is the type
\begin{equation*}
\sm{A_0:\Gamma_0\to\type}\prd*{i,j:\Gamma_0}\Gamma_1(i,j)\to A_0(i)\to A_0(j)\to\type.
\end{equation*}
We will also write $\Gamma\vdash A:\mfGraph$ when $A$ is a graph in context
$\Gamma$.
\end{defn}

\begin{defn}
Suppose that $\Gamma\vdash A:\mfGraph$. A term $x$ of $A$ consists of a pair
$\pairr{x_0,x_1}$ where
\begin{align*}
x_0 & : \prd{i:\Gamma_0}A_0(i)\\
x_1 & : \prd*{i,j:\Gamma_0}{q:\Gamma_1}A_1(q,x_0(i),x_0(j)).
\end{align*}
Thus we define
\begin{equation*}
\terms{A}\defeq\sm{x_0:\prd{i:\Gamma_0}A_0(i)}\prd*{i,j:\Gamma_0}{q:\Gamma_1}A_1(q,x_0(i),x_0(j)).
\end{equation*}
We also write $\Gamma\vdash x:A$ when $x$ is a term of the graph $A$ in context
$\Gamma$.
\end{defn}

\subsubsection{The interpretations of extension, weakening and substitution}
\begin{defn}
Suppose that $\Gamma\vdash A:\mfGraph$. Then we define the graph $\ctxext{\Gamma}{A}$
by
\begin{align*}
\ctxext{\Gamma}{A}_0 & \defeq \sm{i:\Gamma_0}A_0(i)\\
\ctxext{\Gamma}{A}_1(\pairr{i,x},\pairr{j,y}) & \defeq \sm{q:\Gamma_1(i,j)}A_1(q,x,y).
\end{align*}
\end{defn}

\begin{defn}
Suppose that $\Gamma\vdash A:\mfGraph$ and $\Gamma\vdash B:\mfGraph$. Then we
define the graph $\ctxwk{A}{B}$ in context $\ctxext{\Gamma}{A}$ by
\begin{align*}
(\ctxwk{A}{B})_0(\pairr{i,x}) & \defeq B_0(i)\\
(\ctxwk{A}{B})_1(\pairr{q,e},u,v) & \defeq B_1(q,u,v).
\end{align*}
\end{defn}

\begin{defn}
Suppose that $P$ is a family of graphs over $A$ in context $\Gamma$ and let
$x$ be a term of $A$. Then we define the graph $\subst{x}{P}$ 
in context $\Gamma$ by
\begin{align*}
\subst{x}{P}_0(i) & \defeq P_0(\pairr{i,x_0(i)})\\
\subst{x}{P}_1(q,u,v) & \defeq P_1(\pairr{q,x_1(q)},u,v)
\end{align*}
\end{defn}

\begin{rmk}
Note that we have the judgmental equality $\subst{x}{\ctxwk{A}{B}}\jdeq B$
for every two graphs $A$ and $B$ in context $\Gamma$ and every term $x$ of $A$.
\end{rmk}

\begin{rmk}
Using weakening we can describe the graph morphisms. Suppose that $\Delta$ and
$\Gamma$ are graphs. A graph morphism from $\Gamma$ to $\Delta$ is a term of 
the family $\ctxwk{\Gamma}{\Delta}$ of graphs over $\Gamma$. More explicitly,
a graph morphism $f$ is a pair
\begin{align*}
\pts{f} & : \pts{\Gamma}\pts{\Delta}\\
\edg{f} & : \prd*{i,j:\pts{\Gamma}} \edg{\Gamma}(i,j)\to\edg{\Delta}(\pts{f}(i),\pts{f}(j))
\end{align*}
just as expected.

We may also describe morphisms of families of graphs this way.
Let $A$ and $B$ be two families of graphs over a graph $\Gamma$. 
A morphism from $A$ to $B$ is a term of $\ctxwk{A}{B}$. More explicitly,
a morphism $f$ from $A$ to $B$ is a pair $\pairr{f_0,f_1}$
consisting of
\begin{align*}
\pts{f} & : \prd*{i:\Gamma_0} A_0(i)\to B_0(i)\\
\edg{f} & : \prd{\pairr{i,x},\pairr{j,y}:\pts{\ctxext{\Gamma}{A}}}{\pairr{q,e}:\edg{\ctxext{\Gamma}{A}}(\pairr{i,x},\pairr{j,y})} \edg{B}(q,f_0(x),f_0(y)).
\end{align*}
Here we have suppressed the notation for the projections.
\end{rmk}

\subsubsection{The interpretations of the type constructors}

\begin{defn}
The terminal graph $\unit^\mfGraph$, which we shall often denote simply 
by $\unit$, is defined by
\begin{align*}
{\unit^\mfGraph}_0 & \defeq \unit\\
{\unit^\mfGraph}_1(x,y) & \defeq \unit.
\end{align*}
\end{defn}

\begin{rmk}
We note that the function $\ctxext{\unit}{\blank}$ is an equivalence
from $\mftyp{\mfGraph}{\unit}$ to $\tfGraph$. It's inverse is the function which maps
a graph $\Gamma$ to the pair $\pairr{A_0,A_1}$ where $A_0$ is defined by
$A_0(\ttt)\defeq\Gamma_0$ and where $A_1$ is defined by $A_1(\ttt)\defeq
\Gamma_1$. 

Note that we only have an equivalence here, not a judgmental equality.
\end{rmk}

With families of graphs being available, we can give the interpretations of
dependent products, dependent sums and identity types. In the following, we
shall introduce the graph interpretations of dependent products, dependent
sums and identity types and describe the terms of the resulting graphs.\note{These definitions
should be connected to Mike's article but I don't really know how to do this}

\begin{defn}
Let $P$ be a family of graphs over $A$, where $\Gamma\vdash A:\mfGraph$. 
The dependent function graph $\mprd{A}{P}$ in context $\Gamma$ consists of
\begin{align*}
\mprd{A}{P}_0(i) & \defeq \prd{x:A_0(i)}P_0(x)\\
\mprd{A}{P}_1(q,f,g) & \defeq \prd*{x:A_0(i)}*{y:A_0(j)}{e:A_1(q,x,y)}P_1(\pairr{q,e},f(x),g(y)).
\end{align*}
\end{defn}

\begin{rmk}
A term $f:\mprd{A}{P}$ consists of
\begin{align*}
f_0 & : \prd*{i:\Gamma_0}{x:A_0(i)}P_0(x)\\
f_1 & : \prd*{i,j:\Gamma_0}{q:\Gamma_1(i,j)}*{x:A_0(i)}*{y:A_0(j)}{e:A_1(q,x,y)}P_1(\pairr{q,e},f_0(x),f_0(y))
\end{align*}
Therefore, we see that $\eqv{\terms{\mprd{A}{P}}}{\terms{P}}$. \emph{Warning:} it
is by no means the case that $\eqv{\terms{\mprd{A}{P}}}{\prd{x:\terms{A}}
\terms{\subst{x}{P}}}$ for all families $P$ over $A$. For instance, the graph
$\tilde{\emptyt}$ defined by $\tilde{\emptyt}_0\defeq\unit$ and 
$\tilde{\emptyt}_1(\ttt,\ttt)\defeq\emptyt$ has no terms and neither does
$\tilde{\emptyt}+\tilde{\emptyt}$. Nevertheless, there are two graph morphisms
from $\tilde{\emptyt}$ to $\tilde{\emptyt}+\tilde{\emptyt}$. More generally,
when $\Gamma$ is a graph such that $\Gamma_1(i,j)\jdeq\emptyt$ for all $i,j:\Gamma_0$,
then $\eqv{{\terms{\Gamma\to\Gamma'}}}{\Gamma_0\to\Gamma^\prime_0}$.
\end{rmk}

\begin{defn}
If $P$ is a family of graphs over $\Gamma$, the dependent pair graph
$\msm{A}{P}$ consists of
\begin{align*}
\msm{A}{P}_0(i) & \defeq \sm{x:A_0(i)}P_0(x)\\
\msm{A}{P}_1(q,\pairr{x,u},\pairr{y,v}) & \defeq \sm{e:A_1(q,x,y)}P_1(\pairr{q,e},u,v).
\end{align*}
\end{defn}

\begin{rmk}
A term $w:\msm{A}{P}$ consists of
\begin{align*}
w_0 & : \prd{i:\Gamma_0}\sm{x:A_0(i)}P_0(x)
\intertext{and, writing $\lam{i}\proj1(w_0(i))$ and $\lam{i}\proj2(w_0(i))$ as
$w_{00}$ and $w_{01}$ respectively,}
w_1 & : \prd*{i,j:\Gamma_0}{q:\Gamma_1(i,j)}\sm{e:A_1(q,w_{00}(i),w_{00}(j)}P_1(\pairr{q,e},w_{01}(i),w_{01}(j)).
\end{align*}
By $\choice{\infty}$ it follows that
\begin{equation*}
\eqv{\terms{\msm{A}{P}}}{\sm{x:\terms{A}}\terms{\subst{x}{P}}}.
\end{equation*}
\end{rmk}

\begin{defn}
Let $A$ be a graph in context $\Gamma$. We define the family $\idtypevar{A}$ over $\ctxwk{A}{A}$ in
context $\ctxext{\Gamma}{A}$ by
\begin{align*}
(\idtypevar{A})_0(\pairr{i,x},y) & \defeq \id{x}{y}\\
(\idtypevar{A}){}_1(\pairr{q,e},d,\alpha,\alpha') & \defeq \id{\trans{\pairr{\alpha,\alpha'}}{e}}{d}
\end{align*}
where $q:\Gamma_1(i,j)$, $e:A_1(q,x,x')$, $d:A_1(q,y,y')$, $\alpha:\id{x}{y}$
and $\alpha':\id{x'}{y'}$. The transportation along the path 
$\pairr{\alpha,\alpha'}:\id{\pairr{x,x'}}{\pairr{y,y'}}$ in $A_0(i)\times A_0(j)$
is taken with respect to the family $\lam{x}{x'}A_1(q,x,x')$.

We define the term $\refl{A}$ of the family 
$\subst{\idfunc[A]}{\idtypevar{A}}$ over $A$ in context $\Gamma$ by
\begin{align*}
(\reflf{A})_0(i) & \defeq \lam{x}\refl{x}\\
(\reflf{A})_1(q) & \defeq \lam{e}\refl{e}
\end{align*}
\end{defn}

\begin{defn}
Let $D$ be a family over $\idtypevar{A}$ in context 
$\ctxext{{\Gamma}{A}}{\ctxwk{A}{A}}$. Then we have the family
$\subst{\idfunc[A]}{D}$ over $\subst{\idfunc[A]}{\idtypevar{A}}$ in
context $\ctxext{\Gamma}{A}$ given by
\begin{align*}
\subst{\idfunc[A]}{D}_0(\pairr{i,x},\alpha) & \defeq D_0(\pairr{i,x,x},\alpha)\\
\subst{\idfunc[A]}{D}_1(\pairr{q,e},\gamma) & \defeq D_1(\pairr{q,e,e},\gamma).
\end{align*}
The family $\subst{\reflf{A}}{\subst{\idfunc[A]}{D}}$ over $A$ in
context $\Gamma$ is given by
\begin{align*}
\subst{\reflf{A}}{\subst{\idfunc[A]}{D}}_0(i,x) & \defeq D_0(i,x,x,\refl{x})\\
\subst{\reflf{A}}{\subst{\idfunc[A]}{D}}_1(q,e) & \defeq D_1(q,e,e,\refl{e}).
\end{align*}
To show that the identity graphs correctly interpret the identity elimination
rule, we must give a function
\begin{equation*}
\tfJ : \terms{\subst{\reflf{A}}{\subst{\idfunc[A]}{D}}}\to\terms{D}.
\end{equation*}
Note that a term $d$ of $\subst{\reflf{A}}{\subst{\idfunc[A]}{D}}$
consists of
\begin{align*}
d_0 & : \prd*{i:\Gamma_0}{x:A_0(i)}D_0(i,x,x,\refl{x})\\
d_1 & : \prd*{i,j:\Gamma_0}{q:\Gamma_1(i,j)}*{x:A_0(i)}*{y:A_0(j)}{e:A_1(q,x,y)}D_1(q,e,e,\refl{e})
\end{align*}
A simple argument using path induction reveals that terms of
$\subst{\reflf{A}}{\subst{\idfunc[A]}{D}}$ indeed yield terms of $D$. 
\end{defn}

\begin{rmk}
Using the identity graph $\idtypevar{A}$ we can describe the identity
graph $\id[A]{x}{y}$ in context $\Gamma$ for any two terms $x,y:A$. 
The graph $\id[A]{x}{y}$ in context $\Gamma$ consists of
\begin{align*}
(\id[A]{x}{y})_0(i) & \defeq \id{x_0(i)}{y_0(i)}\\
(\id[A]{x}{y})_1(q,\alpha,\beta) & \defeq \id{\trans{\pairr{\alpha,\beta}}{x_1(q)}}{y_1(q)}
\end{align*}
From this, we see that a term of $p:\id[A]{x}{y}$ consists of
\begin{align*}
p_0 & : \prd{i:\Gamma_0}\id{x_0(i)}{y_0(i)}\\
p_1 & : \prd*{i,j:\Gamma_0}{q:\Gamma_1(i,j)}\id{\trans{\pairr{p_0(i),p_0(j)}}{x_1(q)}}{y_1(q)}.
\end{align*}
\end{rmk}


\begin{comment}
\subsubsection{The contexts of $\mfGraph$}


The category $\psh(\cdot{\rightrightarrows}\cdot)$ of presheaves over
$\cdot{\rightrightarrows}\cdot$ is given by
\begin{equation*}
\sm{A_0,A_1:\type}(A_1\to A_0)\times(A_1\to A_0).
\end{equation*}
To see that $\psh(\cdot{\rightrightarrows}\cdot)$ is indeed equivalent to
the type $\tfGraph$ of all graphs, note that we have the equivalences
\begin{align*}
\psh(\cdot{\rightrightarrows}\cdot) & \eqvsym \sm{A_0,A_1:\type}A_1\to A_0\times A_0\\
& \eqvsym \sm{A_0:\type} A_0\times A_0\to\type\\
& \eqvsym \tfGraph.
\end{align*}

We now turn to the description of the universe $\gobjclass$ of graphs. Note
that for the category $I\defeq 0{\rightrightarrows}1$, we have
$I/0\defeq\catid{0}$ and $I/1\defeq s\rightarrow\catid{1}\leftarrow t$, where
$s$ and $t$ are the morphisms $0\to 1$. Thus we have 
\begin{align*}
\psh(I/0) & \eqvsym \type\\
\psh(I/1) & \eqvsym \sm{X,Y:\type}X\to Y\to\type.
\end{align*}
The functors $\psh(\Sigma_s)$ and $\psh(\Sigma_t)$ map a presheaf 
$\pairr{X,Y,R}$ to $X$ and $Y$, respectively. 
Thus we obtain the following graph $\gobjclass$.

\begin{defn}
The universe $\gobjclass$ of graphs is defined to be
\begin{align*}
{\gobjclass}_0 & \defeq \type\\
{\gobjclass}_1(X,Y) & \defeq X\to Y\to\type.
\end{align*}
\end{defn}

Note that the type $\terms{\gobjclass}$ of terms of $\gobjclass$ is 
exactly $\tfGraph$. 

\subsubsection{The basic type constructors for graphs}

\begin{defn}
For any graph $\Gamma$ there is a graph $\tffam{\Gamma}$ in context 
$\gobjclass$ defined by
\begin{align*}
(\tffam{\Gamma})_0 & \defeq \lam{X}X\to\Gamma_0\\
(\tffam{\Gamma})_1(X,Y,R) & \defeq \lam{f}{g}\prd*{x:X}*{y:Y}R(x,y)\to\Gamma_1(f(x),g(y)).
\end{align*}
\end{defn}

\begin{rmk}
A term $D$ of $\msm{\gobjclass}{\tffam{\Gamma}}$ consists of a term
$\pairr{\Delta_0,f_0}$ of type
\begin{equation*}
\msm{\gobjclass}{\tffam{\Gamma}}_0\jdeq \sm{\Delta_0:\type}\Delta_0\to\Gamma_0
\end{equation*}
and a term $\pairr{\Delta_1,f_1}$ of type
\begin{equation*}
\msm{\gobjclass}{\tffam{\Gamma}}_1(q)\jdeq \sm{\Delta_1}
\end{equation*}
\end{rmk}

\begin{defn}
There is a graph morphism
\begin{equation*}
\graphcharmapfunc{\Gamma} : \sm{\gobjclass}
\end{equation*}
\end{defn}
\end{comment}

\subsubsection{Contractibility and equivalences of graphs}

\begin{defn}
A graph $A$ in context $\Gamma$ is said to be \emph{contractible} if there
is a term of the graph
\begin{equation*}
\msm{A}{\mprd{\ctxwk{A}{A}}{\idtypevar{A}}}
\end{equation*}
in context $\Gamma$.
\end{defn}

\begin{lem}\label{lem:contractible-graphs}
Let $A$ be a graph in context $\Gamma$. The following are equivalent:
\begin{enumerate}
\item $A$ is a contractible graph.
\item Both $A_0(i)$ and $A_1(q,x,y)$ are always contractible.
\end{enumerate}
\end{lem}

\begin{proof}
Let $H:\msm{A}{\mprd{\ctxwk{A}{A}}{\idtypevar{A}}}$. Unfolding the definitions, we have
an element $H_0(i)$ of type
\begin{align*}
\msm{A}{\mprd{\ctxwk{A}{A}}{\idtypevar{A}}}_0(i) & \jdeq \sm{x:A_0(i)}\mprd{\ctxwk{A}{A}}{\idtypevar{A}}_0(i,x)\\
& \jdeq \sm{x:A_0(i)}\prd{y:A_0(i)}(\idtypevar{A})_0(\pairr{i,x},y)\\
& \jdeq \sm{x:A_0(i)}\prd{y:A_0(i)}\id{x}{y}
\end{align*}
for all $i:\Gamma_0$ and, writing $H_{00}(i)$ for $\proj1 H_0(i)$ and
$H_{01}(i)$ for $\lam{y}(\proj2 H_0(i))(y)$, we have $H_1(q)$ of type
\begin{align*}
& \msm{A}{\mprd{\ctxwk{A}{A}}{\idtypevar{A}}}_1(q,H_0(i),H_0(j)) \\
& \jdeq \sm{e:A_1(q,H_{00}(i),H_{00}(j))}\mprd{\ctxwk{A}{A}}{\idtypevar{A}}_1(\pairr{q,e},H_{01}(i),H_{01}(j))\\
& \jdeq \sm{e:A_1(q,H_{00}(i),H_{00}(j))}\prd*{x:A_0(i)}*{y:A_0(j)}{d:A_1(q,x,y)}(\idtypevar{A})_1(\pairr{q,e},d,H_{01}(x),H_{01}(y))\\
& \jdeq \sm{e:A_1(q,H_{00}(i),H_{00}(j))}\prd*{x:A_0(i)}*{y:A_0(j)}{d:A_1(q,x,y)}\id{\trans{\pairr{H_{01}(x),H_{01}(y)}}{e}}{d}.
\end{align*}
By $H_0$, it follows that each $A_0(i)$ is contractible. By the contractibility
of each $A_0(i)$, it follows that the type of $H_1(q)$ is equivalent to
\begin{equation*}
\sm{e:A_1(q,H_{00}(i),H_{00}(j))}\prd{d:A_1(q,H_{00}(i),H_{00}(j))}\id{e}{d}
\end{equation*}
which asserts that $A_1(q,H_{00}(i),H_{00}(j))$ is contractible. By 
the contractibility of each $A_0(i)$, is is equivalent to the assertion
that each $A_1(q,x,y)$ is contractible.
\end{proof}

\begin{rmk}
We address the question whether it is the case that a graph $A$ in context
$\Gamma$ is contractible if and only if $\terms{A}$ is contractible. As a
consequence of \autoref{lem:contractible-graphs}, it is indeed the case
that $\terms{A}$ is contractible whenever $A$ is. However, the converse
does not hold.

To see this, we first construct a counter example to the converse of the
weak function extensionality principle, which states that there is a function
of type
\begin{equation}
\iscontr(\prd{x:X}P(x))\to\prd{x:X}\iscontr(P(x))\label{eq:wfe-converse}
\end{equation}
for any type family $P:X\to\type$. In the proof of \autoref{thm:wfe-converse}, 
we will find a family
$P:X\to\type$ with the property that $\prd{x:X}P(x)$ is contractible and for
which there is a term $x:A$ with $P(x)$ not contractible. Disproving the
converse of the weak function extensionality principle suffices for our
purposes, because if $P:X\to\type$ is a counter example to \autoref{eq:wfe-converse},
then we can take $\Gamma\defeq\pairr{\Gamma_0,\Gamma_1}$ to be given by
$\Gamma_0\defeq X$ and $\Gamma_1(i,j)\defeq\emptyt$ and we take
$A\defeq\pairr{A_0,A_1}$ to be given by $A_0\defeq P$ and $A_1(q,u,v)
\defeq\emptyt$.
\end{rmk}

We define the family $\mathcal{T}:\Sn^1\to\UU$ by
We define $\mathcal{T}(\base)\defeq\mathbf{3}$. To define $\mathcal{T}(\lloop):
\id{\mathbf{3}}{\mathbf{3}}$ we apply the univalence axiom. Hence it suffices to find an
equivalence $\eqv{\mathbf{3}}{\mathbf{3}}$, for which we take the function
$e$ defined by
\begin{equation*}
e(x)\defeq\begin{cases}
0_\mathbf{3} & \text{if }x\jdeq 0_\mathbf{3}\\
2_\mathbf{3} & \text{if }x\jdeq 1_\mathbf{3}\\
1_\mathbf{3} & \text{if }x\jdeq 2_\mathbf{3}.
\end{cases}
\end{equation*}

\begin{lem}
The type $\terms{\mathcal{T}}\defeq\prd{x:\Sn^1}\mathcal{T}(x)$ is contractible.
\end{lem}

\begin{proof}
The type of sections of $\mathcal{T}$ is equivalent to $\sm{u:\mathcal{T}(\base)}\id{e(u)}{u}$.
If we have a term $\pairr{u,\alpha}$ of the latter type, it follows by induction
on $\mathbf{3}$ that $\id{\pairr{u,\alpha}}
{\pairr{0_\mathbf{3},\refl{0_\mathbf{3}}}}$ for all $\pairr{u,\alpha}:
\sm{x:\mathcal{T}(\base)}\id{e(u)}{u}$,
which shows that the type of sections of $\mathcal{T}$ is contractible.
\end{proof}

\begin{thm}\label{thm:wfe-converse}
There is a type family $P:A\to\type$ for which
\begin{equation*}
\neg\Big(\iscontr(\prd{x:A}P(x))\to\prd{x:A}\iscontr(P(x))\Big).
\end{equation*}
\end{thm}

\begin{proof}
The type family of our counter example is $\mathcal{T}$: the fiber $\mathcal{T}(\base)$ isn't contractible.
\end{proof}

\begin{defn}
A graph morphism $f:\ctxhom{\Delta}{\Gamma}$ is an equivalence of graphs when
$\graphcharmap[\Gamma]{f}$ is a contractible graph in the context $\Gamma$.
\end{defn}

\begin{lem}
Let $A$ and $B$ be graphs in a context $\Gamma$ and let $f:A\to B$. The following are equivalent:
\begin{enumerate}
\item $f[i]:A[i]\to B[i]$ is an equivalence of graphs for every term $i:\Gamma$.
\item $\ctxext{\Gamma}{f}:\ctxext{\Gamma}{A}\to\ctxext{\Gamma}{B}$ is an equivalence of graphs.
\item Both $f_0(i)$ and $f_1(q,x,y)$ are always equivalences.
\item $\terms{f}:\terms{\Delta}\to\terms{\Gamma}$ is an equivalence.
\end{enumerate}
\end{lem}

\begin{rmk}
It follows that there is an equivalence
\begin{equation*}
\ctxext{\Gamma}\msm{A}{P}\simeq\ctxext{{\Gamma}{A}}{P}
\end{equation*}
for every family $P$ of graphs over a graph $A$ in context $\Gamma$.
\end{rmk}

\subsubsection{Homotopy levels}
\begin{itemize}
\item A graph $A$ in context $\Gamma$ is of homotopy level $n$ precisely when each
$A_0(i)$ and each $A_1(q,x,y)$ are of homotopy level $n$. 
\item We can name at least three different propositions in the empty context:
\begin{enumerate}
\item $\Gamma_0\defeq\emptyt$.
\item $\Gamma_0\defeq\unit$ and $\Gamma_1(\ttt,\ttt)\defeq\emptyt$.
\item $\Gamma_0\defeq\unit$ and $\Gamma_1(\ttt,\ttt)\defeq\unit$.
\end{enumerate}
Therefore $\mfGraph$ does not satisfy the law of excluded middle.
\end{itemize}



\subsubsection{Univalence for the graph model}
In the other direction, we also obtain a graph morphism $\graphcharmap[\Gamma]{f}:
\ctxhom{\Gamma}{\gobjclass}$ for every graph morphism $f:\ctxhom{\Delta}{\Gamma}$.
In \autoref{graph-object-classifier} we will prove that the maps
$\int_\Gamma:\ctxhom{\Gamma}{\gobjclass}\to\sm{\Delta:\tfGraph}\ctxhom{\Delta}{\Gamma}$
is an equivalence with iverse $\graphcharmapfunc{\Gamma}$. 

\begin{defn}
Let $f:\Delta\to\Gamma$ be a graph morphism. We define the graph morphism
$\graphcharmap[\Gamma]{f}$ by
\begin{align*}
\graphcharmap[\Gamma]{f}_0 & \defeq \lam{i}\hfib{f_0}{i}\\
\graphcharmap[\Gamma]{f}_1(i,j) & \defeq \lam{q}\hfib{f_1(i,j)}{q}.
\end{align*}
\end{defn}


\begin{thm}\label{graph-object-classifier}
Main theorem here.
\end{thm}

\begin{comment}
To describe the object classifier for graphs, we will follow Streicher. Thus
we have to look at presheaves over $I/i$ for each object $i$ of the category
$I\defeq 0{\rightrightarrows}1$ with the morphisms named $s$ and $t$ for source
and target. The category $I/0$ is the terminal category;
the category $I/1$ looks like $\cdot{\rightarrow}\cdot{\leftarrow}\cdot$.
Therefore, we have
\begin{align*}
\type^{\op{(I/0)}} & \eqv{}{\type},\\
\type^{\op{(I/1)}} & \eqv{}{\sm{X,Y,A:\type}(A\to X)\times(A\to Y)}\\
& \eqv{}{\sm{X,Y:\type}X\to Y\to\type}.
\end{align*}
The functors $\type^\op{\Sigma_s}$ and $\type^\op{\Sigma_t}$ are given by
$\pi_1$ and $\pi_2$ respectively. This leads to our following definition
of the object classifier $\gobjclass$:

\begin{defn}
Define $\gobjclass$ to be the graph consisting of
\begin{align*}
\gobjclass_0 & \defeq  \type\\
\gobjclass_1(X,Y) & \defeq  X\to Y\to\type
\end{align*}
and define $\pointed{\gobjclass}$ by
\begin{align*}
(\pointed{\gobjclass})_0 & \defeq  \pointed{\type}\\
(\pointed{\gobjclass})_1(\pairr{X,x},\pairr{Y,y}) & \defeq  \sm{R:X\to Y\to\type}R(x,y)
\end{align*}
There is the obvious forgetful graph morphism $t:\pointed{\gobjclass}\to\gobjclass$,
given by projection on the first coordinate.

For any morphism $f:\Delta\to\Gamma$ of graphs we define a morphism
$\graphcharmap(f):\Gamma\to\gobjclass$ of graphs by
\begin{align*}
\graphcharmap(f)_0(i) & \defeq  \hfiber{f_0}{i}\\
\graphcharmap(f)_1(q,\pairr{u,\alpha},\pairr{v,\beta}) & \defeq  \sm{p:\Delta_1(u,v)}
\id{\trans{\pairr{\alpha,\beta}}{f_1(p)}}{q}
\end{align*}
where $\pairr{u,\alpha}:\graphcharmap(f)_0(i)$ and $\pairr{v,\beta}:\graphcharmap(f)_0(j)$. The
morphism $\graphcharmap(f)$ is called the \emph{characteristic map of $f$}. We obtain a function
\begin{equation*}
\graphcharmap : \big(\sm{\Delta:\tfGraph }\Delta\to\Gamma\big)\to\big(\Gamma\to\gobjclass\big)
\end{equation*}
for every graph $\Gamma$.
\end{defn}

\begin{thm}\label{thm:graph-classifier1}
The function $\graphcharmap$ is an equivalence for any graph $\Gamma$.
\end{thm}

\begin{proof}
We have to find a quasi-inverse
\begin{align*}
\Sigma : (\Gamma\to\gobjclass)\to\big(\sm{\Delta:\tfGraph}\Delta\to\Gamma\big)
\end{align*}
of $\graphcharmap$. Thus, we have to define $\Sigma_0:(\Gamma\to\gobjclass)\to\tfGraph$ and
$\Sigma_1:\prd{P:\Gamma\to\gobjclass}\Sigma_0(P)\to\Gamma$. For $P:\Gamma\to\gobjclass$ we define
\begin{align*}
\Sigma_0(P)_0 & \defeq \sm{i:\Gamma_0}P_0(i)\\
\Sigma_0(P)_1(\pairr{i,u},\pairr{j,v}) & \defeq \sm{q:\Gamma_1(i,j)}P_1(q,u,v)\\
\Sigma_1(P)_0 & \defeq \proj1\\
\Sigma_1(P)_1(\pairr{i,u},\pairr{j,v}) & \defeq \proj1.
\end{align*}
\end{proof}

\begin{thm}\label{conj:graph_classifier2}
For any graph morphism $f:\Delta\to\Gamma$, the diagram
\begin{equation*}
\begin{tikzcd}
\Delta \ar{r}{} \ar{d}[swap]{f} & \pointed{\gobjclass} \ar{d}{t} \\ 
\Gamma \ar{r}[swap]{\graphcharmap(f)} & \gobjclass 
\end{tikzcd}
\end{equation*}
is a pullback square.
\end{thm}
\note{We would like it to be a pb \emph{in} the graph model}
\end{comment}

\begingroup\color{blue}
\subsubsection{The adjunctions $\tfcolim\dashv\Delta\dashv\tflim$}
We define $\Delta:\type\to\tfGraph$ by
\begin{equation*}
\Delta(X)\defeq\pairr{X,\lam{x}{x'}\id{x}{x'}}
\end{equation*}
for $X:\type$. For $A:X\to\type$ we define $\Delta(A):\mftyp(\Delta(X))$ by
\begin{align*}
\Delta(A)_0(x) & \defeq A(x)\\
\Delta(A)_1(p,a,b) & \defeq \id{\trans{p}{a}}{b}.
\end{align*}

\begin{lem}
For any type $X$ and any graph $\Gamma$ there is an equivalence
\begin{equation*}
\eqv{(X\to\terms{\Gamma})}{\terms{\Delta(X)\to\Gamma}}.
\end{equation*}
\end{lem}

\begin{proof}
We have to find functions
\begin{align*}
\varphi & : (X\to\terms{\Gamma})\to\terms{\Delta(X)\to\Gamma}\\
\psi & : \terms{\Delta(X)\to\Gamma}\to X\to\terms{\Gamma}
\end{align*}
which are each others homotopy inverse. To define $\varphi$, let
\end{proof}
\endgroup

\subsection{The model of equifibered families of graphs}
We describe the model $\equifib$ of equifibered families of graphs. The contexts
are still the graphs. However, the types of $\equifib$ are not just the
families of graphs, but the equifibered families of graphs. This will have
far-reaching consequences for the rest of the structure of the model. As an
example, a morphism of graphs will no longer be a pair consisting of a map of
vertices and a map of edges.

\begin{defn}
Let $\Gamma$ be a graph. A type in context $\Gamma$ is a triple 
$\pairr{A_0,A_1,A_2}$ consisting of
\begin{align*}
A_0 & : \Gamma_0\to\type\\
A_1 & : \prd*{i,j:\Gamma_0}{q:\Gamma_1(i,j)}A_0(i)\to A_0(j)\\
A_2 & : \prd*{i,j:\Gamma_0}{q:\Gamma_1(i,j)}\isequiv{A_1(q)}
\end{align*}
A type in context $\Gamma$ is also called an equifibered family over $\Gamma$.
\end{defn}

\begin{defn}
Let $\Gamma$ be a graph. A term of $\Gamma$ is a term of $\Gamma_0$.
\end{defn}

\begin{defn}
Let $A$ be an equifibered family over $\Gamma$. A term $x$ of $A$ is a pair
$\pairr{x_0,x_1}$ consisting of
\begin{align*}
x_0 & : \prd{i:\Gamma_0}A_0(i)\\
x_1 & : \prd*{i,j:\Gamma_0}{q:\Gamma_1(i,j)}\id{A_1(q,x_0(i))}{x_0(j)}
\end{align*}
\end{defn}

\begin{defn}
Suppose that $A$ is an equifibered family over $\Gamma$. We define the graph
$\ctxext{\Gamma}{A}$ to consist of
\begin{align*}
\ctxext{\Gamma}{A}_0 & \defeq\sm{i:\Gamma_0}A_0(i)\\
\ctxext{\Gamma}{A}_1(\pairr{i,x},\pairr{j,y}) & \defeq \sm{q:\Gamma_1(i,j)}\id{A_1(q,x)}{y}.
\end{align*}
\end{defn}

\begin{defn}
Suppose $\Gamma$ and $\Delta$ are graphs. We define the equifibered family
$\ctxwk{\Gamma}{\Delta}$ over $\Gamma$ to consist of
\begin{align*}
\ctxwk{\Gamma}{\Delta}_0(i) & \defeq \Delta\\
\ctxwk{\Gamma}{\Delta}_1(q) & \defeq \idfunc[\Delta]\\
\ctxwk{\Gamma}{\Delta}_2(q) & \defeq \nameless.
\end{align*}
The proof $\ctxwk{\Gamma}{\Delta}_2(q)$ that $\ctxwk{\Gamma}{\Delta}_1(q)$ is an
equivalence is the canonical proof that $\idfunc[\Delta]$ is an equivalence, for
which we don't have a name.
\end{defn}

\begin{defn}
Suppose $A$ and $B$ are equifibered families over $\Gamma$. We define the equifibered
diagram $\ctxwk{A}{B}$ over $\ctxext{\Gamma}{A}$ to consist of
\begin{align*}
\ctxwk{A}{B}_0(\pairr{i,x}) & \defeq B_0(i)\\
\ctxwk{A}{B}_1(\pairr{q,\alpha}) & \defeq B_1(q)\\
\ctxwk{A}{B}_2(\pairr{q,\alpha}) & \defeq B_2(q).
\end{align*}
\end{defn}

\begin{defn}
Suppose $A$ is an equifibered family over $\Gamma$ with $x:A$ and that $P$ is an equifibered
family over $\ctxext{\Gamma}{A}$. Then we define the equifibered family $\subst{x}{P}$
over $\Gamma$ to consist of
\begin{align*}
\pts{\subst{x}{P}}(i) & \defeq \pts{P}(\pts{x}(i))\\
\edg{\subst{x}{P}}(q,u) & \defeq \edg{P}(\pairr{q,\edg{x}(q)},u)\\
\subst{x}{P}_2(q) & \defeq P_2(\pairr{q,\edg{x}(q)}).
\end{align*}
\end{defn}

\begin{defn}
Suppose that $A$ is an equifibered family over $\Gamma$ and that $P$ is an
equifibered family over $\ctxext{\Gamma}{A}$. Then we define the equifibered
family $\msm{A}{P}$ over $\Gamma$ by
\begin{align*}
\pts{\msm{A}{P}}(i) & \defeq \sm{x:\pts{A}(i)}\pts{P}(x)\\
\edg{\msm{A}{P}}(q,\pairr{x,u}) & \defeq \pairr{\edg{A}(q,x),\edg{P}(\pairr{q,\refl{\edg{A}(q,x)}},u)}\\
\msm{A}{P}_2(q,\pairr{x,u}) & \defeq \nameless.
\end{align*}
\end{defn}

\end{comment}

\section{Introduction}
In this paper we define a notion of internal model which is adapted to
the univalent setting in which we work. Thus, we only describe what it means
to be an internal model when the ambient type theory is univalent. Our
presentation is derived from the definition of~\cite{Dybjer1996}. There,
an internal model is defined to be a category with families with interpretations
for the basic type constructors $\Pi$, $\Sigma$ and $\idtypevar{}$. However,
a category in~\cite{Dybjer1996} has a set of objects and for every two objects
a setoid of morphisms. This is too restrictive for our purpose. Ideally, we
would start with an $\infty$-category of contexts and an $\infty$-presheaf
of families over it. This presents us with the problem that a notion of
$\infty$-category has not been formulated \emph{in} type theory. Therefore,
we make a deviation from the approach in~\cite{Dybjer1996}. We shall
start with a type $\tfctx$ of contexts, a family $\mftypfunc:\tfctx\to\type$
of families over it and a family $\terms{\blank}:\prd*{\Gamma:\tfctx}\mftyp{\mfM}{\Gamma}
\to\type$ sending a type $A$ in context $\Gamma$ to its type $\terms{A}$ of
terms; then we shall directly interpret $\Pi$-types, declaring
$\terms{A\to B}$ to be $\ctxhom[\Gamma]{A}{B}$. The idea is that by letting
$\Pi$ play a fundamental role in the definition of a category with families
$\mathcal{C}$,
the morphisms get their properties directly from the internal type theory
of $\mathcal{C}$.

\begin{comment}
We briefly list the data of which an internal model of type theory consits. We
consider internal models in the style of~\cite{Dybjer1996}. Thus we will
describe a model $\mfM$ as a category with families. However, since our setting is
univalent type theory we will deviate from~\cite{Dybjer1996} in the following
respects: first of all, we make use of typical ambiguity and hence omit
reference to a thing called \emph{sort}. 

We organize the definition of an internal model $\mfM$ as follows: in
\autoref{internal-model-contexts} we describe the category of contexts itself;
in \autoref{internal-model-families} we describe the families over a given context
and the related operations; in \autoref{internal-model-constructors} we
describe the basic constructors in the internal model.
\end{comment}

\subsection{Ideas in the definition}
An internal model of type theory is like a category with families, but we want
to avoid having to state higher coherences. In fact, we don't even start our
definition with a category of contexts; instead we just take a \emph{type} of contexts. 
The morphisms will come from the terms, evaluation of a function at a given
term will come from substitution. We recognize three basic ingredients to models:
first there is a type of contexts; second, for every context there is a model of types in
that context and third, for every type in a given context there is a type of its
terms. Then there are three basic attributes: context extension, weakening and
substitution. Context extension provides us with families over types as well as
with an interpretation of dependent pair types. We need weakening 
so that families can depend on the same type multiple times (the way the
identity type of a type depends two times on that type) and to be able
to talk about non-dependent function types,
the morphims of our category. Substitution will give us a way
to work with fibers of families as well as composition of functions and evaluation
of functions at terms.

Because we require a \emph{model} of types in a context, all the structure
which we require at the bottom level will be required to exist higher up as well.
Thus, the model of types in a given context $\Gamma$ will have a type of contexts
itself, which can be seen as the type of types in $\Gamma$; it will have its
own notion of types in a context, its own notion of terms, context extension,
weakening and substitution together with all the structure require for it. For
instance, when $A$ is a type in context $\Gamma$ in a model $\mfM$, then there
is the model of types in context $A$, which is the model of families over $A$. 
This model is required to be \emph{definitionally equal to} the model of types
in the context $\ctxext{\Gamma}A$, the context extension of $\Gamma$ and $A$.
In this way we protect ourselves from the need to dig an infinitely deep structure
of models when we want to consider examples.

To give the definition of a model we shall also need to consider certain morphisms
of internal models. Those should preserve all the structure: contexts are mapped
to contexts; for every context a morphism of models mapping the model of types
in that context to the model of types in the image of that context; there should
be a mapping of terms and context extension, weakening and substitution should be
preserved. We need to consider those morphisms because we require context extension,
weakening and substitution to be of that kind, thereby respecting each other
in all possible ways.

When we have this framework set up, we can interpret the basic type constructors
such as $\Pi$, $\Sigma$ and $\idtypevar{}$.
The higher categorical structure then comes from the
result that we have an interpretation of type theory.

\begingroup
\color{red}
\begin{rmk}
Currently, the definition seems to be circular. To define a model we need that
context extension, weakening and substitution be morphisms of models. A morphism
of models needs to preserve context extension, weakening and substitution.
Moreover, its definition requires the notion of composition of morphisms.

I've done it this way because it guides me to what the rules should be, trusting
that I can work my way back to give a (possibly less transparent) definition
which contains no circularities without doubt.
\end{rmk}
\endgroup


\part{Type theory}

In this part we develop type theory from the ground up. We start with a type
theory without any of the basic constructors. This is the theory of contexts
families and terms which has the basic operations of extension, weakening,
substitution and identity terms. Type theory before type constructors has not
been studied very much. Dependent product types or even universes tend to make
an early appearance in just about any presentation of type theory.
A noteable exception is the theory of categories with families of Dybjer
in \cite{Dybjer1996}, which has been elaborated on further by Dybjer and
Clairambault in \cite{DybjerClairambault2011} and in unpublished work by
Awodey \cite{Awodey2013} on natural models of type theory,
which makes the connection between categories with families and representable
transformations of presheaves. In the way type theory is presented by the
Univalent Foundations Project in \cite{TheBook}, which seems to have won the race of introducing universes
as early as possible hands down, it seems entirely unfeasible to study type
theory without type constructors. Also, the 
proof-assistant \Coq\ {\color{red}(and \Agda too?)} has universes and dependent product and pair types 
built-in, making it impossible to study type theory without type constructors in 
that environment.

Nevertheless, type theory without type constructors has received some attention
contributors to the Univalent Foundations Program recently. We name two
further investigations on this topic, other than the mentioned work by Awodey.
In \cite{Garner2014}, Garner describes the combinatorial structure
of the type operations of the weakening, substitution and projection monads
(their projections are our identity terms) and suggests lots of further research
that can be done on type theory without constructors. Also, Joyal has presented
his theory of tribes, which is a categorical explanation of type theory
without type constructors.

After we have described the E-system, which is the flavor of type theory without
type constructors presented in \autoref{tt}, we will demonstrate in \autoref{ttderived}
that the theory gives rise to a rich categorical structure. The notions and
properties we derive here will be essential for the further work on internal
models for E-systems, presented in \autoref{part:models}.

\section{The theory of contexts, families and terms}\label{tt}
In this section we give a description of dependent type theory before type
constructors. Apart from contexts, families and terms -- which provide for the
core of the language of dependent logic -- the basic ingredients
of this theory will be the operations of extension, weakening and substitution
and the identity terms. An empty context (and empty families) are also included
in the theory. The resulting theory can be seen as a manifestation of the 
structure underlying dependent type theory.

We will
formulate the theory of contexts, families and terms in such a way that contexts aren't defined
to be lists of variable declarations. The variable-free (or name-free) approach 
we take here is rather different than those appearing in 
\cite{hofmann1995extensional,TheBook} but it has appeared in the work of Coquand
and in \cite{Dybjer1996}.
The main reason we don't let variable declarations in is that we don't see them 
in the internal models either. This way we also set out to a more algebraic 
approach of type theory and higher category theory. Thirdly, we will not have to
be burdened with superficial comments about variables being bounded or not, or 
fresh or free or not occuring at all.

In our treatment, a context can be the empty context or it can be a binary
planar tree of which `the leaves are (families of) contexts'. 
The judgmental equality relation on contexts is an equivalence relation which 
expresses that binary planar
trees of contexts are judgmentally equal if their leaves are, taking only
(the isomorphism class of) 
the order of the leaves into account \emph{and not the actual shape of the three}.
The intuition behind this equivalence relation is indeed that unstructured
(i.e.~unbracketed) lists such as the lists of variable declarations which
usually appear in type theoretical syntax, may be regarded as contexts.

Besides contexts, families and terms there will also be a notion of `type in
a context'. Only a family of contexts over a context $\Gamma$ is eligible to
be a type in the context $\Gamma$ and we have the intuition that `being a type'
expresses the property of `being an atomic or irreducible family of contexts'.
We will not axiomatize that every context is either judgmentally equal to the
empty context or judgmentally equal to the extension of a context by a type,
although it is certainly worth investigating the class of contexts which fall
under this category.

We will formulate fairly strict rules governing the judgmental equalities,
expressing that extension, weakening and substitution are combatible with
each other in a judgmental manner. This does not, however, diminish the role
of isomorphisms or of homotopies could play in the theory once identity types
are added. Indeed, types could still have non-trivial identity relations and
the category of types in a certain context could genuinely display higher
categorical structure, or so we conjecture.

Much of the rules we state are just compatibility rules of extension, weakening
and substitution with each other. In a way, these rules assert that our contexts
are just structureless lists of contexts and that likewise terms are structureless
lists of terms. They are structureless in the sense that the order in which
they are formed by pairing up is irrelevant. We note that this causes complications
in the traditional way that categorical sematics of type theory is implemented,
where contexts become objects of the category which is supposed to model type
theory. The reason for this is that context extension will not satisfy all the
compatibility rules we're about to state. The first step to resolving this is taking
the types in the empty context as the objects. 

To get the overview of the compatibility rules, we list the sections
where these compatibility rules are described in the following table (in this
table, the
subsection mentioned in row $X$ and column $Y$ consideres the rules of the
operation $Y\circ X$):
\begin{center}
\begin{tabular}{r|ccc}
& extension & weakening & substitution\\
\hline
extension & \autoref{comp-ee} & \autoref{comp-ew} & \\
weakening & \autoref{comp-we} & \autoref{comp-ww} & \autoref{comp-ws}\\
substitution & \autoref{comp-se} & \autoref{comp-sw} & \autoref{comp-ss}
\end{tabular}
\end{center}

\subsection{Judgments and inference rules}\label{judgments}
The theory we describe here is a theory of contexts, families of
contexts and terms thereof. The families of contexts are by some authors called
dependent contexts, but they are handled a bit differently here because they
become the primary object of study. Dependent contexts can be types; they could
be seen as atomic or indecomposable dependent contexts.

Thus we make eight kinds of judgments: ``$\Gamma$ is a context'',
``$A$ is a family of contexts over $\Gamma$'', ``$A$ is a type in context $\Gamma$''
and ``$x$ is a term of the family $A$ of contexts over $\Gamma$''. The other four
judgments are for judgmental equality. 

\begin{align*}
\jctx*{\Gamma} 
& \jctxeq*{\Gamma}{\Gamma'}
  \\
\jfam*{\Gamma}{A} 
& \jfameq*{\Gamma}{A}{B}
  \\
\jterm*{\Gamma}{A}{x} 
& \jtermeq*{\Gamma}{A}{x}{y}.
\end{align*}

Strictly speaking, we have three different judgmental equalities in play and one
could request for a notational difference to signify that fact. For instance,
we could denote the judgmental equalities of contexts, families and terms by
$\jdeq_c$, $\jdeq_f$ and $\jdeq_t$ respectively. It will, however, always be
clear which of the three kinds of judgmental equality is meant when we assert
a judgmental equality and therefore we shall not bother to make this notational
distinction.

We note that what we call families over contexts
here could also have been named dependent contexts or telescopes, see
\cite{deBruijn1991,hofmann1995extensional}. The term family is in agreement
with the terminology scheme of \cite{TheBook}, though the reader should be
warned that the notion of familie means something slightly different there than
it does here.

\subsection{The basic rules for judgmental equality}
The rules for judgmental equality establish that it is an equivalence relation
in all three cases (contexts, types and terms).
\bgroup\small
\begin{align*}
& \inference
  { \jctx{\Gamma}
    }
  { \jctxeq{\Gamma}{\Gamma}
    } 
& & \inference
    { \jctxeq{\Gamma}{\Delta}
      }
    { \jctxeq{\Delta}{\Gamma}
      } 
& & \inference
    { \jctxeq{\Gamma}{\Delta}
      \jctxeq{\Delta}{\greek{E}}
      }
    { \jctxeq{\Gamma}{\greek{E}}
      }
    \\
& \inference
  { \jfam{\Gamma}{A}
    }
  { \jfameq{\Gamma}{A}{A}
    } 
& & \inference
    { \jfameq{\Gamma}{A}{B}
      }
    { \jfameq{\Gamma}{B}{A}
      }
& & \inference
    { \jfameq{\Gamma}{A}{B}
      \jfameq{\Gamma}{B}{C}
      }
    { \jfameq{\Gamma}{A}{C}
      }
    \\
& \inference
  { \jterm{\Gamma}{A}{x}
    }
  { \jtermeq{\Gamma}{A}{x}{x}
    }
& & \inference
    { \jtermeq{\Gamma}{A}{x}{y}
      }
    { \jtermeq{\Gamma}{A}{y}{x}
      }
& & \inference
    { \jtermeq{\Gamma}{A}{x}{y}
      \jtermeq{\Gamma}{A}{y}{z}
      }
    { \jtermeq{\Gamma}{A}{x}{z}
      }
\end{align*}
\egroup

The following convertibility rules are responsible for the strictness
of judgmental equality, which sets it apart from equivalences or identifications:

\begin{align*}
& \inference
  { \jctxeq{\Gamma}{\Delta}
    \jfam{\Gamma}{A}
    }
  { \jfam{\Delta}{A}
    }
& & \inference
    { \jctxeq{\Gamma}{\Delta}
      \jfameq{\Gamma}{A}{B}
      }
    { \jfameq{\Delta}{A}{B}
      }
    \\
& \inference
  { \jctxeq{\Gamma}{\Delta}
    \jterm{\Gamma}{A}{x}
    }
  { \jterm{\Delta}{A}{x}
    }
& & \inference
    { \jctxeq{\Gamma}{\Delta}
      \jtermeq{\Gamma}{A}{x}{y}
      }
    { \jtermeq{\Delta}{A}{x}{y}
      }
    \\
& \inference
  { \jfameq{\Gamma}{A}{B}
    \jterm{\Gamma}{A}{x}
    }
  { \jterm{\Gamma}{B}{x}
    }
& & \inference
    { \jfameq{\Gamma}{A}{B}
      \jtermeq{\Gamma}{A}{x}{y}
      }
    { \jtermeq{\Gamma}{B}{x}{y}
      }
\end{align*}

\subsection{The empty context}
In the theory of contexts, families and terms we introduce an empty context and,
for every context $\Gamma$, an empty family over $\Gamma$. 
When contexts are viewed as statements, the empty context is the statement that
asserts nothing. Likewise, in any context one can assert nothing and this is
achieved by the empty family. 
Since somebody who states nothing asserts it all right, there is a term of the empty
family over any context. Moreover, any two terms of the empty family are always
judgmentally equal.

\begin{align}
& \inference
  { }
  { \jctx{\emptyc}
    }
  \\
& \inference
  { \jctx{\Gamma}
    }
  { \jfam{\Gamma}{\emptyf[\Gamma]}
    }
  \\
& \inference
  { \jctx{\Gamma}
    }
  { \jterm{\Gamma}{\emptyf[\Gamma]}{\emptytm[\Gamma]}
    }
  \\
& \inference
  { \jterm{\Gamma}{\emptyf[\Gamma]}{x}
    }
  { \jtermeq
      {\Gamma}
      {\emptyf[\Gamma]}
      {x}
      {\emptytm[\Gamma]}
    }
\end{align}

The empty context and the empty families together with the operation of
extension will provide several shortcuts for the rest
of the theory of families and terms. This is (in part) obtained by requiring
that a context is exactly the same thing as a family of contexts over the
empty context and that judgmental equality of contexts is exactly the same
as judgmental equality of families of contexts over the empty context. 

By regarding contexts as families of contexts over the empty context, we
enable ourselves also to speak of terms of contexts. A term of a context
$\Gamma$ is a term of the family $\Gamma$ over the empty context. 

\begin{align}
& \inference
  { \jctx{\Gamma}
    }
  { \jfam{\emptyc}{\Gamma}
    } 
& & \inference
    { \jfam{\emptyc}{\Gamma}
      }
    { \jctx{\Gamma}
      }
    \\
& \inference
  { \jctxeq{\Gamma}{\Delta}
    }
  { \jfameq{\emptyc}{\Gamma}{\Delta}
    }
& & \inference
    { \jfameq{\emptyc}{\Gamma}{\Delta}
      }
    { \jctxeq{\Gamma}{\Delta}
      }
\end{align}

\subsubsection{The empty context is compatible with itself}
The empty context $\emptyc$ may be considered as a family of contexts over the empty
context. When we do this, we get $\emptyf[\emptyc]$.
\begin{equation}
\inference
  { }
  { \jfameq
      {\emptyc}
      {\emptyc}
      {\emptyf[\emptyc]}
    }
\end{equation}
In the future, we shall denote $\emptyf[\Gamma]$ by $\emptyf$. The above rule
guarantees that this will not cause confusion. Likewise, we shall denote
$\emptytm[\Gamma]$ by $\emptytm$.

\subsection{Extension}
We introduce extension which not only extends a context $\Gamma$ and a family
$A$ over it to a context $\ctxext{\Gamma}{A}$, but which also extends a family $A$
in context $\Gamma$ and a family $P$ over it to a family $\ctxext{A}{P}$ over context
$\Gamma$. We do this to ensure that all of the theory of contexts, families and
terms can be done in a context.
For instance, we could say (1) that a context in context $\Gamma$ is the same thing
as a family over $\Gamma$; (2) When $A$ is a context in this sense, a family over
$A$ is the same thing as a family $P$ over $\ctxext{\Gamma}{A}$ and 
(3) when $P$ is a family over $A$ in this sense, a term of $P$ keeps its original meaning.

\begin{align}
& \inference
  { \jfam{\Gamma}{A}
    }
  { \jctx{\ctxext{\Gamma}{A}}
    }
& & \inference
    { \jctxeq{\Gamma}{\Delta}
      \jfameq{\Gamma}{A}{B}
      }
    { \jctxeq{\ctxext{\Gamma}{A}}{\ctxext{\Delta}{B}}
      }
    \\
& \inference
  { \jfam{{\Gamma}{A}}{P}
    }
  { \jfam{\Gamma}{\ctxext{A}{P}}
    }
& & \inference
    { \jfameq{\Gamma}{A}{B} 
      \jfameq{{\Gamma}{A}}{P}{Q}
      }
    { \jfameq{\Gamma}{\ctxext{A}{P}}{\ctxext{B}{Q}}
      }
\end{align}

\subsubsection{Extension is compatible with the empty context}
The following rule asserts that extension by $\emptyc$ leaves the contexts unchanged.
\begin{align}
& \inference
  { \jctx{\Gamma}
    }
  { \jctxeq{\ctxext{\emptyc}{\Gamma}}{\Gamma}
    }
  \\
& \inference
  { \jctx{\Gamma}
    }
  { \jctxeq{\ctxext{\Gamma}{\emptyf}}{\Gamma}
    }
  \\
& \inference
  { \jfam{\Gamma}{A}
    }
  { \jfameq{\Gamma}{\ctxext{\emptyf}{A}}{A}
    }
  \\
& \inference
  { \jfam{\Gamma}{A}
    }
  { \jfameq{\Gamma}{{A}{\emptyf}}{A}
    }
\end{align}

\subsubsection{Extension is compatible with itself}\label{comp-ee}
The inference rules asserting that extension is compatible with itself assert
that contexts are unstructured lists of type declarations. This rule is
unavoidable if we want that for a family $A$ in context $\Gamma$, a family over
$A$ is the same thing as a family over $\ctxext{\Gamma}{A}$. 

\begin{align}
& \inference
  { \jfam{\Gamma}{A}
    \jfam{{\Gamma}{A}}{P}
    }
  { \jctxeq{\ctxext{{\Gamma}{A}}{P}}{\ctxext{\Gamma}{{A}{P}}}
    }
  \\
& \inference
  { \jfam{{\Gamma}{A}}{P}
    \jfam{{{\Gamma}{A}}{P}}{Q}
    }
  { \jfameq{\Gamma}{\ctxext{{A}{P}}{Q}}{\ctxext{A}{{P}{Q}}}
    }
\end{align}

\subsection{Weakening}
When $A$ is a context family over a context $\Gamma$, we wish to define a weakening
operation $\ctxwk{A}{}$. The weakening operation acts on context families $B$ 
over $\Gamma$, terms thereof, context families over $B$ and terms thereof.
It weakens those, which means that it ``adds $A$ to the context''. The context
family $\ctxwk{A}{B}$ can be seen as the constant family $B$ over $\ctxext{\Gamma}{A}$.
Likewise, when $y$ is a term of $B$, the term $\ctxwk{A}{y}$ of $\ctxwk{A}{B}$
can be seen as the constant term with value $y$.
 
 In the following inference rules we assume that $\jfam{\Gamma}{A}$ and in the
 rules asserting a judgmental equality we assume furthermore that 
 $\jfameq{\Gamma}{A}{A'}$.
\begin{align}
& \inference
  { \jfam{\Gamma}{B}
    }
  { \jfam{{\Gamma}{A}}{\ctxwk{A}{B}}
    }
& & \inference
    { \jfameq{\Gamma}{B}{B'}
      }
    { \jfameq{{\Gamma}{A}}{\ctxwk{A}{B}}{\ctxwk{A'}{B'}}
      }
    \\
& \inference
  { \jfam{{\Gamma}{B}}{Q}
    }
  { \jfam{{{\Gamma}{A}}{\ctxwk{A}{B}}}{\ctxwk{A}{Q}}
    }
& & \inference
    { \jfameq{{\Gamma}{B}}{Q}{Q'}
      }
    { \jfameq
        {{{\Gamma}{A}}{\ctxwk{A}{B}}}
        {\ctxwk{A}{Q}}
        {\ctxwk{A'}{Q'}}
      }
    \\
& \inference
  { \jterm{\Gamma}{B}{y}
    }
  { \jterm{{\Gamma}{A}}{\ctxwk{A}{B}}{\ctxwk{A}{y}}
    }
& & \inference
    { \jtermeq{\Gamma}{B}{y}{y'}
      }
    { \jtermeq
        {{\Gamma}{A}}
        {\ctxwk{A}{B}}
        {\ctxwk{A}{y}}
        {\ctxwk{A'}{y'}}
      }
    \\
& \inference
  { \jterm{{\Gamma}{B}}{Q}{g}
    }
  { \jterm{{{\Gamma}{A}}{\ctxwk{A}{B}}}{\ctxwk{A}{Q}}{\ctxwk{A}{g}}
    }
& & \inference
    { \jtermeq{{\Gamma}{B}}{Q}{g}{g'}
      }
    { \jtermeq
        {{{\Gamma}{A}}{\ctxwk{A}{B}}}
        {\ctxwk{A}{Q}}
        {\ctxwk{A}{g}}
        {\ctxwk{A'}{g'}}
      }
\end{align}

\subsubsection{Weakening is compatible with the empty context}
The following rules express that when the empty context or context family is
weakened, the result is the empty context family.
\begin{align}
& \inference
  { \jctx{\Gamma}
    }
  { \jfameq{\Gamma}{\ctxwk{\Gamma}{\emptyc}}{\emptyf}
    }
  \\
& \inference
  { \jfam{\Gamma}{A}
    }
  { \jfameq{{\Gamma}{A}}{\ctxwk{A}{\emptyf}}{\emptyf}
    }
\end{align}
Weakening by the empty family $\emptyf$ over a context $\Gamma$ leaves families, 
their terms, families over those families and
terms of those unchanged:
\begin{align}
& \inference
  { \jfam{\Gamma}{B}
    }
  { \jfameq{\Gamma}{\ctxwk{\emptyf}{B}}{B}
    }
  \\
& \inference
  { \jterm{\Gamma}{B}{y}
    }
  { \jtermeq{\Gamma}{B}{\ctxwk{\emptyf}{y}}{y}
    }
  \\
& \inference
  { \jfam{{\Gamma}{B}}{Q}
    }
  { \jfameq{{\Gamma}{B}}{\ctxwk{\emptyf}{Q}}{Q}
    }
  \\
& \inference
  { \jterm{{\Gamma}{B}}{Q}{g}
    }
  { \jtermeq{{\Gamma}{B}}{Q}{\ctxwk{\emptyf}{g}}{g}
    }
\end{align}

\subsubsection{Weakening is compatible with extension}\label{comp-we}

The following rules assert the compatibility of extension with weakening: for
every family $A$ over $\Gamma$ and every family $Q$ over $\ctxext{\Gamma}{B}$
there is a
judgmental equality $\ctxwk{A}{\ctxext{B}{Q}}\jdeq\ctxext{\ctxwk{A}{B}}
{\ctxwk{A}{Q}}$. 

When thinking of terms of $\ctxwk{A}{B}$ as morphisms of families from $A$ to
$B$, this looks already like form of type theoretic choice. It is weaker in that
it is not stated with function types, yet it is stronger in that it states a
judgmental equality between two families. When one makes the weakening operation
notationally invisible -- as is in fact the usual practice in type theory -- the
following compatibility rules become completely obvious.

In the following inference rules we assume that $\jfam{\Gamma}{A}$.
\begin{align}
& \inference
  { \jfam{{{\Gamma}{B}}{Q}}{R}
    }
  { \jfameq
      {\ctxext{{\Gamma}{A}}{\ctxwk{A}{B}}}
      {\ctxwk{A}{\ctxext{Q}{R}}}
      {\ctxext{\ctxwk{A}{Q}}{\ctxwk{A}{R}}}
    }
\end{align}

\subsubsection{Weakening is compatible with itself}\label{comp-ww}
We state judgmental equality rules expressing
that weakening is compatible with itself. These rules state that the following
diagram commutes given any two families $A$ and $B$ in context $\Gamma$:
\begin{equation*}
\begin{tikzcd}[column sep=huge]
\jfam{\Gamma}{\blank} 
  \ar{r}{C\mapsto\ctxwk{B}{C}} 
  \ar{d}[swap]{C\mapsto\ctxwk{A}{C}} 
& \jfam{{\Gamma}{B}}{\blank} 
  \ar{d}{Q\mapsto\ctxwk{A}{Q}}
  \\
\jfam{{\Gamma}{A}}{\blank} 
  \ar{r}[swap]{P\mapsto\ctxwk{{A}{B}}{P}} 
& \jfam{{{\Gamma}{A}}{\ctxwk{A}{B}}}{\blank}
\end{tikzcd}
\end{equation*}
Thus, we get the following set of inference rules:
\begin{align}
& \inference
  { \jfam{\Gamma}{A}
    \jfam{\Gamma}{B}
    \jfam{{\Gamma}{C}}{R}
    }
  { \jfameq
      {{{{\Gamma}{A}}{\ctxwk{A}{B}}}{\ctxwk{{A}{B}}{{A}{C}}}}
      {\ctxwk{A}{{B}{R}}}
      {\ctxwk{{A}{B}}{{A}{R}}}
    }
  \label{comp-ww-f}\\
& \inference
  { \jfam{\Gamma}{A}
    \jfam{\Gamma}{B}
    \jterm{{\Gamma}{C}}{R}{t}
    }
  { \jtermeq
      {{{{\Gamma}{A}}{\ctxwk{A}{B}}}{\ctxwk{{A}{B}}{{A}{C}}}}
      {\ctxwk{{A}{B}}{{A}{R}}}
      {\ctxwk{A}{{B}{t}}}
      {\ctxwk{{A}{B}}{{A}{t}}}
    }
  \label{comp-ww-t}
\end{align}

\subsubsection{Extension is compatible with weakening}\label{comp-ew}
The rules expressing that extension is compatible with weakening assert that
weakening by an extension is the same thing as weakening twice in the
appropriate way.

In the following inference rules we assume that
$\jfam{\Gamma}{A}$ and $\jfam{{\Gamma}{A}}{P}$. 
\begin{align}
& \inference
  { \jfam{{\Gamma}{B}}{Q}
    }
  { \jfameq
      {{{{\Gamma}{A}}{P}}{\ctxwk{P}{{A}{B}}}}
      {\ctxwk{\ctxext{A}{P}}{Q}}
      {\ctxwk{P}{{A}{Q}}}
    }
  \label{comp-ew-f}\\
& \inference
  { \jterm{{\Gamma}{B}}{Q}{g}
    }
  { \jtermeq
      {{{{\Gamma}{A}}{P}}{\ctxwk{P}{{A}{B}}}}
      {\ctxwk{P}{{A}{Q}}}
      {\ctxwk{\ctxext{A}{P}}{g}}
      {\ctxwk{P}{{A}{g}}}
    } 
  \label{comp-ew-t}
\end{align}

\subsection{Substitution}
Given a family $P$ over $A$ and a term $x$ of $A$, substitution gives a way to
consider the fiber $\subst{x}{P}$ of $P$ at $x$. Also, we get a way to evaluate
terms $f$ of $P$ at $x$. This will give us ways to compose functions too. In
this section, we shall first introduce the operations `substitution of a term $x$'
for families $P$ over $\ctxext{\Gamma}{A}$, terms $f$ of those, families $Q$ over
$\ctxext{{\Gamma}{A}}{P}$ and terms $g$ of those. 
Then we shall explain how substitution interacts
with itself, extension and weakening.

In the rules introducing the various substitutions we assume $\jterm{\Gamma}{A}{x}$;
in the rules introducing the definitional equalities we assume $\jtermeq{\Gamma}{A}{x}{x'}$.

\begin{align}
& \inference
  { \jfam{{\Gamma}{A}}{P}
    }
  { \jfam{\Gamma}{\subst{x}{P}}
    }
& & \inference
    { \jfameq{{\Gamma}{A}}{P}{P'}
      }
    { \jfameq{\Gamma}{\subst{x}{P}}{\subst{x'}{P'}}
      }
    \\
& \inference
  { \jfam{{{\Gamma}{A}}{P}}{Q}
    }
  { \jfam{{\Gamma}{\subst{x}{P}}}{\subst{x}{Q}}
    }
& & \inference
    { \jfameq{{{\Gamma}{A}}{P}}{Q}{Q'}
      }
    { \jfameq{{\Gamma}{\subst{x}{P}}}{\subst{x}{Q}}{\subst{x'}{Q'}}
      }
    \\
& \inference
  { \jterm{{\Gamma}{A}}{P}{f}
    }
  { \jterm{\Gamma}{\subst{x}{P}}{\subst{x}{f}}
    }
& & \inference
    { \jtermeq{{\Gamma}{A}}{P}{f}{f'}
      }
    { \jtermeq{\Gamma}{\subst{x}{P}}{\subst{x}{f}}{\subst{x'}{f'}}
      }
    \\
& \inference
  { \jterm{{{\Gamma}{A}}{P}}{Q}{g}
    }
  { \jterm{{\Gamma}{\subst{x}{P}}}{\subst{x}{Q}}{\subst{x}{g}}
    }
& & \inference
    { \jtermeq{{{\Gamma}{A}}{P}}{Q}{g}{g'}
      }
    { \jtermeq
        {{\Gamma}{\subst{x}{P}}}
        {\subst{x}{Q}}
        {\subst{x}{g}}
        {\subst{x'}{g'}}
      }
\end{align}

\subsubsection{Substitution is compatible with the empty context}
The fibers of the empty family are the empty families:
\begin{align}
& \inference
  { \jterm{\Gamma}{A}{x}
    }
  { \jfameq{\Gamma}{\subst{x}{\emptyf}}{\emptyf}
    }
  \\
& \inference
  { \jterm{\Gamma}{A}{x}
    \jfam{{\Gamma}{A}}{P}
    }
  { \jfameq
      {{\Gamma}{\subst{x}{P}}}
      {\subst{x}{\emptyf}}
      {\emptyf}
    }
\end{align}

The following rules assert that substituting by the term $\jterm{\Gamma}{\emptyf}{\emptytm}$
leaves everything unchanged.
\begin{align}
& \inference
  { \jfam{\Gamma}{A}
    }
  { \jfameq{\Gamma}{\subst{\emptytm}{A}}{A}
    }
  \\
& \inference
  { \jterm{\Gamma}{A}{x}
    }
  { \jtermeq{\Gamma}{A}{\subst{\emptytm}{x}}{x}
    }
  \\
& \inference
  { \jfam{{\Gamma}{A}}{P}
    }
  { \jfameq{{\Gamma}{A}}{\subst{\emptytm}{P}}{P}
    }
  \\
& \inference
  { \jterm{{\Gamma}{A}}{P}{f}
    }
  { \jtermeq{{\Gamma}{A}}{P}{\subst{\emptytm}{f}}{f}
    }.
\end{align}

\subsubsection{Substitution is compatible with extension}\label{comp-se}
Suppose $\jterm{\Gamma}{A}{x}$ in all of the following inference rule.
\begin{align}
& \inference
  { \jfam{{{{\Gamma}{A}}{P}}{Q}}{R}
    }
  { \jfameq
      {{\Gamma}{\subst{x}{P}}}
      {\subst{x}{\ctxext{Q}{R}}}
      {\ctxext{\subst{x}{Q}}{\subst{x}{R}}}
    }
\end{align}

\subsubsection{Substitution is compatible with weakening}\label{comp-sw}
The rules asserting the compatibility of substitution with weakening assert
that the following diagram commutes for any $\jterm{\Gamma}{A}{x}$ and any
$\jfam{{\Gamma}{A}}{P}$:
\begin{equation*}
\begin{tikzcd}[column sep=huge]
\jfam{{\Gamma}{A}}{\blank} 
  \ar{d}[swap]{Q\mapsto\subst{x}{Q}} 
  \ar{r}{Q\mapsto\ctxwk{P}{Q}} 
& \jfam{{{\Gamma}{A}}{P}}{\blank} 
    \ar{d}{R\mapsto\subst{x}{R}}
  \\ 
\jfam{\Gamma}{\blank} 
  \ar{r}[swap]{B\mapsto\ctxwk{\subst{x}{P}}{B}} 
& \jfam{{\Gamma}{\subst{x}{P}}}{\blank}
\end{tikzcd}
\end{equation*}
We plug in an extra layer of families to cover the most general case at once.
In the following inference rules we assume that $\jterm{\Gamma}{A}{x}$ and
$\jfam{{\Gamma}{A}}{P}$:
\begin{align}
& \inference
  { \jfam{{{\Gamma}{A}}{Q}}{R}
    }
  { \jfameq
      {{{\Gamma}{\subst{x}{P}}}{\subst{x}{\ctxwk{P}{Q}}}}
      {\subst{x}{\ctxwk{P}{R}}}
      {\ctxwk{\subst{x}{P}}{\subst{x}{R}}}
    }
  \label{comp-sw-f}\\
& \inference
  { \jterm{{{\Gamma}{A}}{Q}}{R}{h}
    }
  { \jtermeq
      {\ctxext{{\Gamma}{\subst{x}{P}}}{\subst{x}{\ctxwk{P}{Q}}}}
      {\subst{x}{\ctxwk{P}{R}}}
      {\subst{x}{\ctxwk{P}{h}}}
      {\ctxwk{\subst{x}{P}}{\subst{x}{h}}}
    }
  \label{comp-sw-t}
\end{align}

\subsubsection{Substitution is compatible with substitution}\label{comp-ss}

We require that substitution is compatible with itself, which is roughly the
assertion that substitution is associative. However, we cannot just state that
$\subst{x}{{f}{g}}\jdeq\subst{{x}{f}}{g}$ since the expression $\subst{{x}{f}}{g}$
is not well-formed. The term $\subst{x}{f}$ can be substituted in (terms of) families over
$\subst{x}{P}$; the term $\subst{x}{g}$ is such. Therefore, associativity of
substitution takes the form $\subst{x}{{f}{g}}\jdeq\subst{{x}{f}}{{x}{g}}$.
Note that the term $\subst{{x}{f}}{{x}{g}}$ may be written down more conveniently
as $\subst{x,\subst{x}{f}}{g}$, although we will not do that here.

In the following inference rules we assume
$\jterm{\Gamma}{A}{x}$ and $\jterm{{\Gamma}{A}}{P}{f}$.

\begin{align}
& \inference
  { \jfam{{{{\Gamma}{A}}{P}}{Q}}{R}
    }
  { \jfameq
      {{\Gamma}{\subst{x}{{f}{Q}}}}
      {\subst{x}{{f}{R}}}
      {\subst{{x}{f}}{{x}{R}}}
    }
  \label{comp-ss-f}\\
& \inference
  { \jterm{{{{\Gamma}{A}}{P}}{Q}}{R}{h}
    }
  { \jtermeq
      {{\Gamma}{\subst{x}{{f}{Q}}}}
      {\subst{x}{{f}{R}}}
      {\subst{x}{{f}{h}}}
      {\subst{{x}{f}}{{x}{h}}}
    }
  \label{comp-ss-t}
\end{align}

\subsubsection{Weakening is compatible with substitution}\label{comp-ws}
We already have rules for the compatibility of substitution with weakening, but
we still need the rules the other way around, asserting that there is a 
judgmental equality $\ctxwk{A}{\subst{y}{Q}}\jdeq\subst{\ctxwk{A}{y}}{\ctxwk{A}{Q}}$
together with all its variants.

In the following inference rules we assume that $\jfam{\Gamma}{A}$ and that
$\jterm{\Gamma}{B}{y}$.

\begin{align}
& \inference
  { \jfam{{{\Gamma}{B}}{Q}}{R}
    }
  { \jfameq
      {{{\Gamma}{A}}{\ctxwk{A}{\subst{y}{Q}}}}
      {\ctxwk{A}{\subst{y}{R}}}
      {\subst{\ctxwk{A}{y}}{\ctxwk{A}{R}}}
    }
  \label{comp-ws-f}\\
& \inference
  { \jterm{{{\Gamma}{B}}{Q}}{R}{h}
    }
  { \jtermeq
      {{{\Gamma}{A}}{\ctxwk{A}{\subst{y}{Q}}}}
      {\ctxwk{A}{\subst{y}{R}}}
      {\ctxwk{A}{\subst{y}{h}}}
      {\subst{\ctxwk{A}{y}}{\ctxwk{A}{h}}}
    }
  \label{comp-ws-t}
\end{align}

\subsection{Composition and identity terms}\label{categorical_properties}
\subsubsection{The defining property of weakening}
The judgmental equalities we're about to describe assert that substituting a term
in the weakening a thing gives you the thing back. In the case of contexts we get that each fiber
$\subst{x}{\ctxwk{A}{B}}$ is just $B$ and in the case of terms we get 
that $\ctxwk{A}{y}$ is the constant function
mapping everything to $y:B$. Thus, these rules actually establish the weakening
as the weakening. After stating the rules we will describe what it means to
compose context morphisms (terms of weakened contexts).

\begin{align}
& \inference
  { \jfam{\Gamma}{A}
    \jfam{{\Gamma}{B}}{Q}
    \jterm{\Gamma}{A}{x}
    }
  { \jfameq{{\Gamma}{B}}{\subst{x}{\ctxwk{A}{Q}}}{Q}
    }
  \label{defn-ws-3}\\
& \inference
  { \jterm{\Gamma}{A}{x}
    \jterm{{\Gamma}{B}}{Q}{g}
    }
  { \jtermeq{{\Gamma}{B}}{Q}{\subst{x}{\ctxwk{A}{g}}}{g}
    }
  \label{defn-ws-4}
\end{align}

Using the rules of the compatibility of substitution with weakening and of the
compatibility of weakening with itself, we see that we can show

\begin{lem}
The inference rule
\begin{equation*}
\inference
  { \jfam{\Gamma}{A}
    \jfam{\Gamma}{B}
    \jfam{\Gamma}{C}
    \jhom{\Gamma}{A}{B}{f}
    }
  { \jfameq
    {{\Gamma}{A}}
    {\subst{f}{\ctxwk{A}{{B}{C}}}}
    {\ctxwk{A}{C}}
    }
\end{equation*}
is valid.
\end{lem}

\begin{proof}
Let $\jfam{\Gamma}{A}$, $\jfam{\Gamma}{B}$, $\jfam{\Gamma}{C}$ and $\jhom{\Gamma}{A}{B}{f}$.
Then we have the judgmental equalities
\begin{align*}
\subst{f}{\ctxwk{A}{{B}{C}}}
& \jdeq 
  \subst{f}{\ctxwk{{A}{B}}{{A}{C}}}
  \\
& \jdeq 
  \ctxwk{A}{C}.
  \qedhere
\end{align*}
\end{proof}

It follows that for $\jterm{{\Gamma}{B}}{\ctxwk{B}{C}}{g}$ we can compose $f$
with $g$ to obtain a term of $\ctxwk{A}{C}$ in context $\ctxext{\Gamma}{A}$.
In the following definition, we work with in a slightly greater generality.

\begin{defn}
We define the judgment
\begin{equation*}
\jhom{\Gamma}{A}{B}{f},
\end{equation*}
which is pronounced as `$f$ is a morphism from $A$ to $B$ in context $\Gamma$',
to be the judgment
\begin{equation*}
\unfold{\jhom{\Gamma}{A}{B}{f}}.
\end{equation*}
Likewise, we define the judgment
\begin{equation*}
\jhomeq{\Gamma}{A}{B}{f}{f'}
\end{equation*}
to be the judgment
\begin{equation*}
\unfold{\jhomeq{\Gamma}{A}{B}{f}{f'}}.
\end{equation*}
\end{defn}

\begin{defn}
Let $\jhom{\Gamma}{A}{B}{f}$ and consider a term $\jterm{{\Gamma}{B}}{Q}{g}$.
We define
\begin{align*}
\jfamdefn*
  {{\Gamma}{A}}
  {\jcomp{A}{f}{Q}}
  {\unfold{\jcomp{A}{f}{Q}}}\\
\jtermdefn*
  {{\Gamma}{A}}
  {\jcomp{A}{f}{Q}}
  {\jcomp{A}{f}{g}}
  {\unfold{\jcomp{A}{f}{g}}}.
\end{align*}
Likewise, when we have a family $\jfam{{{\Gamma}{B}}{Q}}{R}$ and a term
$\jterm{{{\Gamma}{B}}{Q}}{R}{h}$, we define
\begin{align*}
\jfamdefn*
  {{{\Gamma}{A}}{\jcomp{A}{f}{Q}}}
  {\jcomp{A}{f}{R}}
  {\unfold{\jcomp{A}{f}{R}}}
  \\
\jtermdefn*
  {{{\Gamma}{A}}{\jcomp{A}{f}{Q}}}
  {\jcomp{A}{f}{R}}
  {\jcomp{A}{f}{h}}
  {\unfold{\jcomp{A}{f}{h}}}.
\end{align*}
\end{defn}

We have lots of compatibility properties for composition:

\begin{lem}
We have the following inference rules about the situation where something is
substituted by a composition:
\begin{align*}
& \inference
  { \jhom{\Gamma}{A}{B}{f}
    \jhom{\Gamma}{B}{C}{g}
    \jfam{{{\Gamma}{A}}{\ctxwk{A}{C}}}{R}
    }
  { \jfameq
      {{\Gamma}{A}}
      {\subst{\jcomp{A}{f}{g}}{R}}
      {\subst{f}{{\ctxwk{A}{g}}{\ctxwk{{A}{B}}{R}}}}
    }
  \\
& \inference
  { \jhom{\Gamma}{A}{B}{f}
    \jhom{\Gamma}{B}{C}{g}
    \jterm{{{\Gamma}{A}}{\ctxwk{A}{C}}}{R}{h}
    }
  { \jfameq
    {{\Gamma}{A}}
    {\subst{\jcomp{A}{f}{g}}{h}}
    {\subst{f}{{\ctxwk{A}{g}}{\ctxwk{{A}{B}}{h}}}}
    }
\end{align*}
We also have the following related inference rules, asserting that composition
is strictly associative:
\begin{align*}
& \inference
  { \jhom{\Gamma}{A}{B}{f}
    \jhom{\Gamma}{B}{C}{g}
    \jfam{{\Gamma}{C}}{R}
    }
  { \jfameq
      {{\Gamma}{A}}
      {\jcomp{A}{{A}{f}{g}}{R}}
      {\jcomp{A}{f}{{B}{g}{R}}}
    }
  \\
& \inference
    { \jhom{\Gamma}{A}{B}{f}
      \jhom{\Gamma}{B}{C}{g}
      \jterm{{\Gamma}{C}}{R}{h}
      }
    { \jtermeq
        {{\Gamma}{A}}
        {\jcomp{A}{{A}{f}{g}}{R}}
        {\jcomp{A}{{A}{f}{g}}{h}}
        {\jcomp{A}{f}{{B}{g}{h}}}
      }
\end{align*}
\end{lem}

\begin{proof}
Consider family morphisms $\jhom{\Gamma}{A}{B}{f}$ and $\jhom{\Gamma}{B}{C}{g}$
and a family $\jfam{{{\Gamma}{A}}{\ctxwk{A}{C}}}{R}$. Then we have the judgmental
equalities
\begin{align*}
\subst{\jcomp{A}{f}{g}}{R} 
& \jdeq 
  \subst{{f}{\ctxwk{A}{g}}}{R}
  \\
& \jdeq 
  \subst{{f}{\ctxwk{A}{g}}}{\subst{f}{\ctxwk{{A}{B}}{R}}}
  \\
& \jdeq 
  \subst{f}{{\ctxwk{A}{g}}{\ctxwk{{A}{B}}{R}}}
\end{align*}
The proof that 
$\subst{\jcomp{A}{f}{g}}{h}\jdeq\subst{f}{{\ctxwk{A}{g}}{\ctxwk{{A}{B}}{h}}}$
is similar.

Now suppose that $\jfam{{\Gamma}{C}}{R}$ instead. Then we have
\begin{align*}
\jcomp{A}{{A}{f}{g}}{R} 
& \jdeq 
  \subst{\jcomp{A}{f}{g}}{\ctxwk{A}{R}}
  \\
& \jdeq 
  \subst{{f}{\ctxwk{A}{g}}}{\ctxwk{A}{R}}
  \\
& \jdeq 
  \subst{f}{{\ctxwk{A}{g}}{\ctxwk{{A}{B}}{{A}{R}}}}
  \\
& \jdeq 
  \subst{f}{{\ctxwk{A}{g}}{\ctxwk{A}{{B}{R}}}}
  \\
& \jdeq 
  \subst{f}{\ctxwk{A}{\subst{g}{\ctxwk{B}{R}}}}
  \\
& \jdeq 
  \subst{f}{\ctxwk{A}{\jcomp{B}{g}{R}}}
  \\
& \jdeq 
  \jcomp{A}{f}{{B}{g}{R}}.
\end{align*}
Again, the proof is similar for terms $h$ of $R$ in context $\ctxext{\Gamma}{C}$.
\end{proof}

\begin{lem}
We have the following inference rules about the compatibility of composition with
weakening:
\begin{align*}
& \inference
  { \jhom{\Gamma}{A}{B}{f}
    \jhom{\Gamma}{B}{C}{g}
    \jfam{{\Gamma}{A}}{P}
    }
  { \jhomeq
      {\Gamma}
      {{A}{P}}
      {C}
      {\ctxwk{P}{\jcomp{A}{f}{g}}}
      {\jcomp{{A}{P}}{\ctxwk{P}{f}}{g}}
    }
  \\
& \inference
  { \jterm{\Gamma}{B}{y}
    \jhom{\Gamma}{B}{C}{g}
    }
  { \jhomeq
      {\Gamma}
      {A}
      {C}
      {\jcomp{A}{\ctxwk{A}{y}}{g}}
      {\ctxwk{A}{\subst{y}{g}}}
    }
  \\
& \inference
  { \jhom{\Gamma}{A}{B}{f}
    \jterm{\Gamma}{C}{z}
    }
  { \jhomeq
      {\Gamma}
      {A}
      {C}
      {\jcomp{A}{f}{\ctxwk{B}{z}}}
      {\ctxwk{A}{z}}
    }
\end{align*}
\end{lem}

\begin{proof}
Let $\jhom{\Gamma}{A}{B}{f}$, $\jhom{\Gamma}{B}{C}{g}$ and $\jfam{{\Gamma}{A}}{P}$.
Then we have the judgmental equalities
\begin{align*}
\ctxwk{P}{\jcomp{A}{f}{g}} 
& \jdeq 
  \ctxwk{P}{\subst{f}{\ctxwk{A}{g}}}
  \\
& \jdeq 
  \subst{\ctxwk{P}{f}}{\ctxwk{P}{{A}{g}}}
  \\
& \jdeq 
  \subst{\ctxwk{P}{f}}{\ctxwk{\ctxext{A}{P}}{g}}
  \\
& \jdeq 
  \jcomp{{A}{P}}{\ctxwk{P}{f}}{g}.
\end{align*}
Now let $\jterm{\Gamma}{B}{y}$ and $\jhom{\Gamma}{B}{C}{g}$. Then we have the
judgmental equalities
\begin{align*}
\jcomp{A}{\ctxwk{A}{y}}{g}
& \jdeq 
  \subst{\ctxwk{A}{y}}{\ctxwk{A}{g}}
  \\
& \jdeq 
  \ctxwk{A}{\subst{y}{g}}.
\end{align*}
For the third assertion, let $\jhom{\Gamma}{A}{B}{f}$ and $\jterm{\Gamma}{C}{z}$.
Then we have the judgmental equalities
\begin{align*}
\jcomp{A}{f}{\ctxwk{B}{z}} 
& \jdeq 
  \subst{f}{\ctxwk{A}{{B}{z}}}
  \\
& \jdeq 
  \subst{f}{\ctxwk{{A}{B}}{{A}{z}}}
  \\
& \jdeq 
  \ctxwk{A}{z}.
  \qedhere
\end{align*}
\end{proof}

\begin{lem}
We have the following inference rules about the compatibility of composition with
substitution:
\begin{align*}
& \inference
  { \jhom{{\Gamma}{A}}{P}{Q}{f}
    \jhom{{\Gamma}{A}}{Q}{R}{g}
    \jterm{\Gamma}{A}{x}
    }
  { \jhomeq
      {\Gamma}
      {\subst{x}{P}}
      {\subst{x}{R}}
      {\subst{x}{\jcomp{P}{f}{g}}}
      {\jcomp{\subst{x}{P}}{\subst{x}{f}}{\subst{x}{g}}}
    }
  \\
& \inference
  { \jhom{\Gamma}{A}{B}{f}
    \jhom{\Gamma}{B}{C}{g}
    \jterm{\Gamma}{A}{x}
    }
  { \jtermeq
      {\Gamma}
      {C}
      {\subst{x}{\jcomp{A}{f}{g}}}
      {\subst{{x}{f}}{g}}
    }
\end{align*}
\end{lem}

\begin{proof}
Let $\jhom{{\Gamma}{A}}{P}{Q}{f}$, $\jhom{{\Gamma}{A}}{Q}{R}{g}$ and 
$\jterm{\Gamma}{A}{x}$.
Then we have the judgmental equalities
\begin{align*}
\subst{x}{\jcomp{A}{f}{g}}
& \jdeq 
  \subst{x}{{f}{\ctxwk{P}{g}}}
  \\
& \jdeq 
  \subst{{x}{f}}{{x}{\ctxwk{P}{g}}}
  \\
& \jdeq 
  \subst{{x}{f}}{\ctxwk{\subst{x}{P}}{\subst{x}{g}}}
  \\
& \jdeq 
  \jcomp{\subst{x}{P}}{\subst{x}{f}}{\subst{x}{g}}.
\end{align*}
Now let $\jhom{\Gamma}{A}{B}{f}$, $\jhom{\Gamma}{B}{C}{g}$ and $\jterm{\Gamma}{A}{x}$.
Then we have the judgmental equalities
\begin{align*}
\subst{x}{\jcomp{A}{f}{g}}
& \jdeq 
  \subst{x}{{f}{\ctxwk{A}{g}}}
  \\
& \jdeq 
  \subst{{x}{f}}{{x}{\ctxwk{A}{g}}}
  \\
& \jdeq 
  \subst{{x}{f}}{g}.
  \qedhere
\end{align*}
\end{proof}

There is also a notion of morphism \emph{over} a morphism. We will develop this
notion because it will be needed in the theory of models later on.

\begin{defn}
Let $\jhom{\Gamma}{A}{B}{f}$ be a morphism from $A$ to $B$ in context $\Gamma$
and consider $\jfam{{\Gamma}{A}}{P}$ and $\jfam{{\Gamma}{B}}{Q}$. We define the
judgment
\begin{equation*}
\jfhom{\Gamma}{A}{B}{f}{P}{Q}{F},
\end{equation*}
which is pronounced as `$F$ is a morphism from $P$ to $Q$ over $f$ in context
$\Gamma$', to be the judgment
\begin{equation*}
\unfold{\jfhom{\Gamma}{A}{B}{f}{P}{Q}{F}}.
\end{equation*}
\end{defn}

\begin{rmk}
The judgment $\jfhom{\Gamma}{A}{B}{f}{P}{Q}{F}$ means the same thing as
\begin{equation*}
\jhom{{\Gamma}{A}}{P}{\jcomp{A}{f}{Q}}{F}.
\end{equation*}
\end{rmk}

Suppose we have morphisms $\jhom{\Gamma}{A}{B}{f}$ and $\jhom{\Gamma}{B}{C}{g}$
and that we have the morphisms $\jfhom{\Gamma}{A}{B}{f}{P}{Q}{F}$ and
$\jfhom{\Gamma}{B}{C}{g}{Q}{R}{G}$ over them. Then we have
\begin{equation*}
\jhom
  {{\Gamma}{A}}
  {\jcomp{A}{f}{Q}}
  {\jcomp{A}{f}{{B}{g}{R}}}
  {\unfold{\jcomp{A}{f}{G}}}
\end{equation*}
Because we also have $\jhom{{\Gamma}{A}}{P}{\jcomp{A}{f}{Q}}{F}$, we have the
composition
\begin{equation*}
\jhom
  {{\Gamma}{A}}
  {P}
  {\jcomp{A}{f}{{B}{g}{R}}}
  {\jcomp{P}{F}{\unfold{\jcomp{A}{f}{G}}}}.
\end{equation*}
Because of 
the judgmental equality $\jcomp{A}{f}{{B}{g}{R}}\jdeq
\jcomp{A}{{A}{f}{g}}{R}$, it follows that 
$\jcomp{P}{F}{\unfold{\jcomp{A}{f}{G}}}$ is a morphism from $P$ to $R$ over
$\jcomp{A}{f}{g}$. We make the following definition:

\begin{defn}
Let $\jhom{\Gamma}{A}{B}{f}$ and $\jhom{\Gamma}{B}{C}{g}$
be morphisms and let $\jfhom{\Gamma}{A}{B}{f}{P}{Q}{F}$ and
$\jfhom{\Gamma}{B}{C}{g}{Q}{R}{G}$ be morphisms over them. Then we define
\begin{equation*}
\jfhomdefn
  {\Gamma}
  {A}
  {C}
  {\jcomp{A}{f}{g}}
  {P}
  {R}
  {\jfcomp{A}{f}{P}{F}{G}}
  {\unfold{\jfcomp{A}{f}{P}{F}{G}}}.
\end{equation*}
\end{defn}

This composition is also judgmentally associative.

\subsubsection{identity terms}
Without a rule explicitly asserting the existence of an identity morphism we don't
get one, hence we do that here. The identity morphism is a term which introduced
in ordinary type theory via the variable rule. The variable rule is a bit more
general: it asserts that
\begin{equation*}
\jterm{\Gamma,\,x_1:A_1,\ldots,\,x_n:A_n}{A_i}{x_i}
\end{equation*}
for every $1\leq i\leq n$. Thus, it establishes the projections. In 
the current setting,
we get the projections from the identity morphisms together with weakening. We
already have weakening, so here it suffices to introduce the identity morphisms.
\begin{align}
& \inference
  { \jfam{\Gamma}{A}
    }
  { \jhom{\Gamma}{A}{A}{\idtm{A}}
    }
& & \inference
    { \jfameq{\Gamma}{A}{A'}
      }
    { \jhomeq{\Gamma}{A}{A}{\idtm{A}}{\idtm{A'}}
      }
\end{align}
Identity terms are determined by their behavior with respect to substitution combined with
weakening. The identity terms will also be subject to compatibility rules.
\begin{align}
& \inference
  { \jterm{\Gamma}{A}{x}
    }
  { \jtermeq{\Gamma}{A}{\subst{x}{\idtm{A}}}{x}
    }
  \label{idfunc-subst-defn}\\
& \inference
  { \jfam{{\Gamma}{A}}{P}
    }
  { \jfameq{{\Gamma}{A}}{\subst{\idtm{A}}{\ctxwk{A}{P}}}{P}
    }
  \label{idfunc-wk-defn}\\
& \inference
  { \jfam{{\Gamma}{A}}{P}
    }
  { \jfameq
      {{\Gamma}{A}}
      {\subst{\idtm{A}}{\ctxwk{{A}{A}}{P}}}
      {P}
    }
  \label{idfunc-wk-defn2}\\
& \inference
  { \jterm{{\Gamma}{A}}{P}{f}
    }
  { \jtermeq
      {{\Gamma}{A}}
      {P}
      {\subst{\idtm{A}}{\ctxwk{A}{f}}}
      {f}
    }
  \label{idfunc-precomp}\\
& \inference
  { \jterm{{\Gamma}{A}}{P}{f}
    }
  { \jtermeq
      {{\Gamma}{A}}
      {P}
      {\subst{\idtm{A}}{\ctxwk{{A}{A}}{f}}}
      {f}
    }
  \label{idfunc-precomp}\\
& \inference
  { \jhom{\Gamma}{A}{B}{f}
    }
  { \jhomeq{\Gamma}{A}{B}{\jcomp{A}{f}{\idtm{B}}}{f}
    }
  \label{idfunc-postcomp}
\end{align}

We won't state a compatibility rule stating that the identity term is
compatible with extension because we will be able to prove that. Instead, we
will just state the compatibility rules for the identity term combined with
weakening and with substitution.

The identity term of a weakened family is the weakened identity term:
\begin{equation}\label{idfunc-wk-comp}
\inference
  { \jfam{\Gamma}{A}
    \jfam{\Gamma}{B}
    }
  { \jhomeq
      {{\Gamma}{A}}
      {\ctxwk{A}{B}}
      {\ctxwk{A}{B}}
      {\ctxwk{A}{\idtm{B}}}
      {\idtm{\ctxwk{A}{B}}}
    }
\end{equation}

The identity term of a substituted family is the substitution of the identity term
\begin{equation}\label{idfunc-subst-comp}
\inference
  { \jterm{\Gamma}{A}{x}
    \jfam{{\Gamma}{A}}{P}
    }
  { \jhomeq
      {\Gamma}
      {\subst{x}{P}}
      {\subst{x}{P}}
      {\subst{x}{\idtm{P}}}
      {\idtm{\subst{x}{P}}}
    }
\end{equation}

Now let $\jfam{{\Gamma}{A}}{P}$ and $\jfam{{\Gamma}{A}}{Q}$ be families. A
morphism from $P$ to $Q$ over the identity term $\idtm{A}$ in context
$\Gamma$ is the same thing as a morphism from $P$ to $Q$ in context
$\ctxext{\Gamma}{A}$:

\begin{lem}\label{hom-over-id-is-hom}
We have the following valid inference rules:
\begin{align*}
& \inference
  { \jfam{{\Gamma}{A}}{P}
    \jfam{{\Gamma}{A}}{Q}
    \jfhom{\Gamma}{A}{A}{\idtm{A}}{P}{Q}{f}
    }
  { \jhom{{\Gamma}{A}}{P}{Q}{f}
    }
  \\
& \inference
  { \jfam{{\Gamma}{A}}{P}
    \jfam{{\Gamma}{A}}{Q}
    \jhom{{\Gamma}{A}}{P}{Q}{f}
    }
  { \jfhom{\Gamma}{A}{A}{\idtm{A}}{P}{Q}{f}
    }
\end{align*}
\end{lem}

\begin{proof}
If we unfold the judgments $\jhom{{\Gamma}{A}}{P}{Q}{f}$ and
$\jfhom{\Gamma}{A}{A}{\idtm{A}}{P}{Q}{f}$, we get the judgments
\begin{align*}
& \unfold{\jhom{{\Gamma}{A}}{P}{Q}{f}}
  \\
& \unfold{\jfhom{\Gamma}{A}{A}{\idtm{A}}{P}{Q}{f}}
\end{align*}
respectively. Therefore, we only need to verify that
$\ctxwk{P}{\subst{\idtm{A}}{\ctxwk{A}{Q}}}\jdeq\ctxwk{P}{Q}$, which is indeed
the case by \autoref{idfunc-wk-defn}.
\end{proof}

\subsection{The possiblity of types in the theory of contexts, families and
terms}
We have deliberately not spoken of types so far because we have taken the point
of view that a type in a context is nothing but a family in that context which
belongs to the class of types. We think of types as \emph{irreducible} families,
i.e.\ families which are neither the empty context nor the extension of two
families which are both not the empty context (in algebraic terminology: which
are both non-trivial). To allow ourselves to speak of types we introduce two
new judgments: the judgment that something is a type and the judgment that two
types are equal.
\begin{align*}
\jtype*{\Gamma}{A} 
& \jtypeeq*{\Gamma}{A}{B}
\end{align*}
But only families of contexts are eligible to be types. If $A$ is a type
in context $\Gamma$, then $A$ is also a family of contexts over $\Gamma$. 
Moreover, two types in context $\Gamma$ are judgmentally equal precisely when they are equal
as context families and if a family $B$ of contexts over $\Gamma$ is
judgmentally equal to a type $A$ in context $\Gamma$, then $B$ is a type in
context $\Gamma$. This is expressed by the following four inference rules:
\begin{align*}
& \inference
  { \jtype{\Gamma}{A}
    }
  { \jfam{\Gamma}{A}
    }
& & \inference
    { \jtypeeq{\Gamma}{A}{B}
      }
    { \jfameq{\Gamma}{A}{B}
      }
    \\
& \inference
  { \jtype{\Gamma}{A}
    \jfameq{\Gamma}{A}{B}
    }
  { \jtype{\Gamma}{B}
    }
& & \inference
    { \jtype{\Gamma}{A}
      \jfameq{\Gamma}{A}{B}
      }
    { \jtypeeq{\Gamma}{A}{B}
      }
\end{align*}
As pointed out at the beginning of this subsection, 
we do not assume that the empty context is a type, that would be like
assuming that the multiplicative unit of a ring is prime. We envision a
categorical interpretation of the theory of contexts, families and terms where 
contexts are interpreted as finite paths of arrows belonging to a predetermined
class. The empty context would be such a path of length $0$ whereas the types
would be such paths of length $1$.

We add rules asserting that a weakened type is again a type that 
substitution preserves the property of being a type:
\begin{align}
& \inference
  { \jfam{\Gamma}{B}
    \jtype{\ctxext{\Gamma}{B}}{Q}
    }
  { \jtype{\ctxext{{\Gamma}{A}}{\ctxwk{A}{B}}}{\ctxwk{A}{Q}}
    }
  \\
& \inference
  { \jterm{\Gamma}{A}{x}
    \jtype{\ctxext{{\Gamma}{A}}{P}}{Q}
    }
  { \jtype{\ctxext{\Gamma}{\subst{x}{P}}}{\subst{x}{Q}}
    }
\end{align}
With only the current rules, the possibility of making the judgment that
something is a type does not add much to the theory of contexts, families and
terms. Nevertheless, when studying models, having an interpretation for the
judgment that something is a type will allow for the possibility to study
conditions such as the one asserting that every family `factorizes' uniquely
as multiple applications of extension to types.



\section{Derived notions of the theory of contexts, families and terms}
\label{digging_deeper}

In this section we use the framework we have developed in \autoref{tt}
to derive new notions and their properties. In particular, we will
develop the notion of \emph{extension on terms} together with the projection
maps from an extension to the `base' context of family, the 
\emph{family pullback} which is a version of pullbacks for families and thirdly
the \emph{inductive morphisms} which are morphisms of type theory that allow
to find terms of families over the codomain context by pulling them back to
the domain context and finding a term there.
\emph{This section contains no new assumptions.}

In \autoref{extension-on-terms} on the extension operation on terms we will
derive all the compatibility rules that one would expect to hold for extension
on terms. The key to most of these results is the currying operation, which
could be seen as the missing feature in the table above. The extension on terms
operation depends in an essential way on the substitution operation, on the
identity terms and therefore indirectly also on the weakening operation. Thus,
we will see here all of the features of the theory we develop in
\autoref{tt} come to the 
surface.

Next, we introduce the inclusion of the fibers $\subst{x}{P}$ into the extension
$\ctxext{A}{P}$ as a morphism in context $\Gamma$. As was the case with
extension on terms and with projections, there will be a ton of compatibility
properties which we will prove about these inclusions. 

It should be kept in mind though that in the current formulation there is no
sealed deal establishing a relationship between families over a context
and any kind of morphisms -- neither with morphisms having the `base' of the
family as its codomain nor with families into a universe (universes will be
introduced in \autoref{universes}). The only thing we know here is
that a family $P$ over $\ctxext{\Gamma}{A}$ determines a context morphism
from $\ctxext{A}{P}$ to $A$ in context $\Gamma$, the projection. 
We do not see this as
a shortcoming of the theory of contexts families and terms. Rather, such a
correspondence is a feature of a theory which does incorporate universes. The
fact that we're lacking a clear connection between families and (a specified
class of) morphisms, however, does show up in our treatment of the notion we
called familie pullbacks. For instance, we can't show that a square of families
is a pullback precisely when the corresponding square of projections is a
pullback: only the backwards direction holds. 
The discrepancies continue: ordinary pullbacks do not always exist
whereas family pullbacks do but the composition of two family pullback squares
need not be a family pullback square again whereas the pasting lemma of
ordinary pullbacks holds as usually.
We feel that pointing out what we can and can't do in the current setting is
an important aspect of developing an intuition with the system and therefore
we include this subsection even though the theory of family pullbacks
might feel a bit different than the usual theory of pullbacks.

In the last subsection we give a treatment of inductive morphisms. These
morphisms appear also in the introduction of the many inductive type
constructors in \autoref{tt_constructors} and therefore a general treatment of
the subject is insightful.

\subsection{Morphisms}\label{morphisms}
Using the rules of the compatibility of substitution with weakening and of the
compatibility of weakening with itself, we see that we can show

\begin{lem}\label{lem:prehom}
The inference rule\begin{equation*}
\inference
  { \jfam{\Gamma}{A}
    \jfam{\Gamma}{B}
    \jfam{\Gamma}{C}
    \jhom{\Gamma}{A}{B}{f}
    }
  { \jfameq
    {{\Gamma}{A}}
    {\subst{f}{\ctxwk{A}{{B}{C}}}}
    {\ctxwk{A}{C}}
    }
\end{equation*}
is valid.
\end{lem}

\begin{proof}
Let $\jfam{\Gamma}{A}$, $\jfam{\Gamma}{B}$, $\jfam{\Gamma}{C}$ and $\jhom{\Gamma}{A}{B}{f}$.
Then we have the judgmental equalities
\begin{align*}
\subst{f}{\ctxwk{A}{{B}{C}}}
& \jdeq 
  \subst{f}{\ctxwk{{A}{B}}{{A}{C}}}
  \tag{by \autoref{comp-ww-f}}\\
& \jdeq 
  \ctxwk{A}{C}.
  \tag{by \autoref{cancellation-ws-t}}
\end{align*}
\end{proof}

It follows that for $\jterm{{\Gamma}{B}}{\ctxwk{B}{C}}{g}$ we can compose $f$
with $g$ to obtain a term of $\ctxwk{A}{C}$ in context $\ctxext{\Gamma}{A}$.
In the following definition, we work with in a slightly greater generality.

\begin{defn}
We define the judgment\begin{equation*}
\jhom{\Gamma}{A}{B}{f},
\end{equation*}
which is pronounced as `$f$ is a morphism from $A$ to $B$ in context $\Gamma$',
to be the judgment\begin{equation*}
\unfold{\jhom{\Gamma}{A}{B}{f}}.
\end{equation*}
Likewise, we define the judgment\begin{equation*}
\jhomeq{\Gamma}{A}{B}{f}{f'}
\end{equation*}
to be the judgment\begin{equation*}
\unfold{\jhomeq{\Gamma}{A}{B}{f}{f'}}.
\end{equation*}
\end{defn}

\begin{defn}
Let $\jhom{\Gamma}{A}{B}{f}$ and consider a family $\jfam{{\Gamma}{B}}{Q}$,
a family $\jfam{{{\Gamma}{B}}{Q}}{R}$ and a term $\jterm{{{\Gamma}{B}}{Q}}{R}{h}$.
We define\begin{align*}
\jfamdefn*
  {{\Gamma}{A}}
  {\jcomp{A}{f}{Q}}
  {\unfold{\jcomp{A}{f}{Q}}}
  \\
\jfamdefn*
  {{{\Gamma}{A}}{\jcomp{A}{f}{Q}}}
  {\jcomp{A}{f}{R}}
  {\unfold{\jcomp{A}{f}{R}}}
  \\
\jtermdefn*
  {{{\Gamma}{A}}{\jcomp{A}{f}{Q}}}
  {\jcomp{A}{f}{R}}
  {\jcomp{A}{f}{h}}
  {\unfold{\jcomp{A}{f}{h}}}.
\end{align*}
\end{defn}

\begin{lem}\label{lem:jcomp-emp}
Let $\jhom{\Gamma}{A}{B}{f}$. Then the inference rules
\begin{align*}
& \inference
  { %
    }
  { \jfameq{{\Gamma}{A}}{\jcomp{A}{f}{\emptyf}}{\emptyf}
    }
  \\
& \inference
  { \jfam{{\Gamma}{B}}{Q}
    }
  { \jfameq{{{\Gamma}{B}}{\jcomp{A}{f}{Q}}}{\jcomp{A}{f}{\emptyf}}{\emptyf}
    }
\end{align*}
are valid.
\end{lem}

\begin{proof}
For the first, note that we have the judgmental equalities
\begin{align*}
\jcomp{A}{f}{\emptyf}
& \jdeq
  \unfold{\jcomp{A}{f}{\emptyf}}
  \tag{by definition}
  \\
& \jdeq
  \subst{f}{\emptyf}
  \tag{by \autoref{comp-w0-c}}
  \\
& \jdeq
  \emptyf.
  \tag{by \autoref{comp-s0-c}}
\end{align*}
The second judgmental equality is proven similarly.
\end{proof}

\begin{rmk}
Recall that we can treat a family $\jfam{{\Gamma}{B}}{Q}$ as a family
$\jfam{{{\Gamma}{B}}{\emptyf}}{Q}$ and that $\jcomp{A}{f}{\emptyf}\jdeq
\emptyf$. Thus we can apply composition with $f$ to terms 
$\jterm{{\Gamma}{B}}{Q}{g}$. We get
\begin{equation*}
\jtermeq
  {{\Gamma}{B}}
  {\jcomp{A}{f}{Q}}
  {\jcomp{A}{f}{g}}
  {\unfold{\jcomp{A}{f}{g}}}.
\end{equation*}
In the particular situation where we take $Q$ to be a weakened family
$\ctxwk{B}{C}$, we see that we can apply composition with $f$ to morphisms
from $B$ to $C$ and we can use \autoref{lem:prehom} to see that we get
\begin{equation*}
\jhomeq{\Gamma}{A}{C}{\jcomp{A}{f}{g}}{\unfold{\jcomp{A}{f}{g}}}
\end{equation*}
for $\jhom{\Gamma}{B}{C}{g}$. 

One might argue that the notation for composition should be reserved to only
this special case, to not confuse with common intuition of composition. It is
however very convenient to see composition as an operation of the theory of
contexts, families and terms. This allows us to follow the scheme of
compatibility rules which are provable for this form of composition. 
\end{rmk}

We have lots of compatibility properties for composition. On the one hand we
have the properties that composition with a morphism $f$ is compatible with
the empty family, extension, weakening, substitution and identity terms. On
the other hand, there are compatibility properties saying what happens when
we substitute by a composition, or when we weaken or substitute a composition.
Proving all these compatibility properties is the content of the rest of this
subsection. None of it is very difficult.

\begin{lem}\label{lem:jcomp-ext}
We have the following inference rule expressing that composition with $f$ is
compatible with extension:
\begin{equation*}
\inference
  { \jfam{{{{\Gamma}{B}}{Q}}{R}}{S}
    }
  { \jfameq
      {{{\Gamma}{A}}{\jcomp{A}{f}{Q}}}
      {\jcomp{A}{f}{\ctxext{R}{S}}}
      {\ctxext{\jcomp{A}{f}{R}}{\jcomp{A}{f}{S}}}
    }
\end{equation*}
\end{lem}

\begin{proof}
Let $\jfam{{{{\Gamma}{B}}{Q}}{R}}{S}$. Then we have the judgmental equalities
\begin{align*}
\jcomp{A}{f}{\ctxext{R}{S}}
& \jdeq
  \unfold{\jcomp{A}{f}{\ctxext{R}{S}}}
  \tag{by definition}
  \\
& \jdeq
  \subst{f}{\ctxext{\ctxwk{A}{R}}{\ctxwk{A}{S}}}
  \tag{by \autoref{comp-we-f}}
  \\
& \jdeq
  \unfoldall{\ctxext{\jcomp{A}{f}{R}}{\jcomp{A}{f}{S}}}
  \tag{by \autoref{comp-se-f}}
  \\
& \jdeq
  \ctxext{\jcomp{A}{f}{R}}{\jcomp{A}{f}{S}}.
  \tag{by definition}
\end{align*}
\end{proof}

\begin{lem}\label{lem:jcomp-wk}
We have the following inference rules expressing that composition with $f$ is
compatible with weakening:
\begin{align*}
& \inference
  { \jfam{{{\Gamma}{B}}{Q}}{R}
    \jfam{{{\Gamma}{B}}{Q}}{S}
    }
  { \jfameq
      {{{{\Gamma}{A}}{\jcomp{A}{f}{Q}}}{\jcomp{A}{f}{R}}}
      {\jcomp{A}{f}{\ctxwk{R}{S}}}
      {\ctxwk{\jcomp{A}{f}{R}}{\jcomp{A}{f}{S}}}
    }
  \\
& \inference
  { \jfam{{{\Gamma}{B}}{Q}}{R}
    \jterm{{{\Gamma}{B}}{Q}}{S}{k}
    }
  { \jtermeq
      {{{{\Gamma}{A}}{\jcomp{A}{f}{Q}}}{\jcomp{A}{f}{R}}}
      {\jcomp{A}{f}{\ctxwk{R}{S}}}
      {\jcomp{A}{f}{\ctxwk{R}{k}}}
      {\ctxwk{\jcomp{A}{f}{R}}{\jcomp{A}{f}{k}}}
    }
\end{align*}
\end{lem}

\begin{proof}
Consider the families $\jfam{{{\Gamma}{B}}{Q}}{R}$ and 
$\jfam{{{\Gamma}{B}}{Q}}{S}$. Then we have the judgmental equalities
\begin{align*}
\jcomp{A}{f}{\ctxwk{R}{S}}
& \jdeq
  \unfold{\jcomp{A}{f}{\ctxwk{R}{S}}}
  \tag{by definition}
  \\
& \jdeq
  \subst{f}{\ctxwk{{A}{R}}{{A}{S}}}
  \tag{by \autoref{comp-ww-f}}
  \\
& \jdeq
  \unfoldall{\ctxwk{\jcomp{A}{f}{R}}{\jcomp{A}{f}{S}}}
  \tag{by \autoref{comp-sw-f}}
  \\
& \jdeq
  \ctxwk{\jcomp{A}{f}{R}}{\jcomp{A}{f}{S}}.
  \tag{by definition}
\end{align*}
The proof of the second property is similar.
\end{proof}

\begin{lem}\label{lem:jcomp-subst}
We have the following inference rules expressing that composition with $f$ is
compatible with substitution:
\begin{align*}
& \inference
  { \jterm{{{\Gamma}{B}}{Q}}{R}{h}
    \jfam{{{{\Gamma}{B}}{Q}}{R}}{S}
    }
  { \jfameq
      {{{\Gamma}{A}}{\jcomp{A}{f}{Q}}}
      {\jcomp{A}{f}{\subst{h}{S}}}
      {\subst{\jcomp{A}{f}{h}}{\jcomp{A}{f}{S}}}
    }
  \\
& \inference
  { \jterm{{{\Gamma}{B}}{Q}}{R}{h}
    \jterm{{{{\Gamma}{B}}{Q}}{R}}{S}{k}
    }
  { \jtermeq
      {{{\Gamma}{A}}{\jcomp{A}{f}{Q}}}
      {\jcomp{A}{f}{\subst{h}{S}}}
      {\jcomp{A}{f}{\subst{h}{k}}}
      {\subst{\jcomp{A}{f}{h}}{\jcomp{A}{f}{k}}}
    }
\end{align*}
\end{lem}

\begin{proof}
Let $\jterm{{{\Gamma}{B}}{Q}}{R}{h}$ and $\jfam{{{{\Gamma}{B}}{Q}}{R}}{S}$.
Then we have the judgmental equalities
\begin{align*}
\jcomp{A}{f}{\subst{h}{S}}
& \jdeq
  \unfold{\jcomp{A}{f}{\subst{h}{S}}}
  \tag{by definition}
  \\
& \jdeq
  \subst{f}{{\ctxwk{A}{h}}{\ctxwk{A}{S}}}
  \tag{by \autoref{comp-ws-f}}
  \\
& \jdeq
  \unfoldall{\subst{\jcomp{A}{f}{h}}{\jcomp{A}{f}{S}}}
  \tag{by \autoref{comp-ss-f}}
  \\
& \jdeq
  \subst{\jcomp{A}{f}{h}}{\jcomp{A}{f}{S}}.
  \tag{by definition}
\end{align*}
The proof of the second inference rule is similar.
\end{proof}

\begin{lem}\label{lem:jcomp-idtm}
We have the following inference rule expressing that composition with $f$ is
compatible with identity terms:
\begin{equation*}
\inference
  { \jfam{{{\Gamma}{B}}{Q}}{R}
    }
  { \jtermeq
      {{{{\Gamma}{A}}{\jcomp{A}{f}{Q}}}{\jcomp{A}{f}{R}}}
      {\ctxwk{\jcomp{A}{f}{R}}{\jcomp{A}{f}{R}}}
      {\jcomp{A}{f}{\idtm{R}}}
      {\idtm{\jcomp{A}{f}{R}}}
    }
\end{equation*}
\end{lem}

\begin{proof}
Let $\jfam{{{\Gamma}{B}}{Q}}{R}$. We have the following judgmental equalities:
\begin{align*}
\jcomp{A}{f}{\idtm{R}}
& \jdeq
  \unfold{\jcomp{A}{f}{\idtm{R}}}
  \tag{by definition}
  \\
& \jdeq
  \subst{f}{\idtm{\ctxwk{A}{R}}}
  \tag{by \autoref{comp-wi-t}}
  \\
& \jdeq
  \unfoldall{\idtm{\jcomp{A}{f}{R}}}
  \tag{by \autoref{comp-si-t}}
  \\
& \jdeq
  \idtm{\jcomp{A}{f}{R}}.
  \tag{by definition}
\end{align*}
\end{proof}

\begin{lem}\label{lem:jcomp-jcomp}
We have the following inference rules about the situation where something is
substituted by a composition:\begin{align*}
& \inference
  { \jhom{\Gamma}{A}{B}{f}
    \jhom{\Gamma}{B}{C}{g}
    \jfam{{{\Gamma}{A}}{\ctxwk{A}{C}}}{R}
    }
  { \jfameq
      {{\Gamma}{A}}
      {\subst{\jcomp{A}{f}{g}}{R}}
      {\subst{f}{{\ctxwk{A}{g}}{\ctxwk{{A}{B}}{R}}}}
    }
  \\
& \inference
  { \jhom{\Gamma}{A}{B}{f}
    \jhom{\Gamma}{B}{C}{g}
    \jterm{{{\Gamma}{A}}{\ctxwk{A}{C}}}{R}{h}
    }
  { \jfameq
    {{\Gamma}{A}}
    {\subst{\jcomp{A}{f}{g}}{h}}
    {\subst{f}{{\ctxwk{A}{g}}{\ctxwk{{A}{B}}{h}}}}
    }
\end{align*}
We also have the following related inference rules, asserting that composition
is strictly associative:\begin{align*}
& \inference
  { \jhom{\Gamma}{A}{B}{f}
    \jhom{\Gamma}{B}{C}{g}
    \jfam{{\Gamma}{C}}{R}
    }
  { \jfameq
      {{\Gamma}{A}}
      {\jcomp{A}{{A}{f}{g}}{R}}
      {\jcomp{A}{f}{{B}{g}{R}}}
    }
  \\
& \inference
    { \jhom{\Gamma}{A}{B}{f}
      \jhom{\Gamma}{B}{C}{g}
      \jterm{{\Gamma}{C}}{R}{h}
      }
    { \jtermeq
        {{\Gamma}{A}}
        {\jcomp{A}{{A}{f}{g}}{R}}
        {\jcomp{A}{{A}{f}{g}}{h}}
        {\jcomp{A}{f}{{B}{g}{h}}}
      }
\end{align*}
\end{lem}

\begin{proof}
Consider family morphisms $\jhom{\Gamma}{A}{B}{f}$ and $\jhom{\Gamma}{B}{C}{g}$
and a family $\jfam{{{\Gamma}{A}}{\ctxwk{A}{C}}}{R}$. Then we have the judgmental
equalities\begin{align*}
\subst{\jcomp{A}{f}{g}}{R} 
& \jdeq 
  \subst{{f}{\ctxwk{A}{g}}}{R}
  \tag{by definition}
  \\
& \jdeq 
  \subst{{f}{\ctxwk{A}{g}}}{\subst{f}{\ctxwk{{A}{B}}{R}}}
  \tag{by \autoref{cancellation-ws-f}}
  \\
& \jdeq 
  \subst{f}{{\ctxwk{A}{g}}{\ctxwk{{A}{B}}{R}}}.
  \tag{by \autoref{comp-ss-f}}
\end{align*}
The proof that 
$\subst{\jcomp{A}{f}{g}}{h}\jdeq\subst{f}{{\ctxwk{A}{g}}{\ctxwk{{A}{B}}{h}}}$
is similar.

Now suppose that $\jfam{{\Gamma}{C}}{R}$ instead. Then we have\begin{align*}
\jcomp{A}{{A}{f}{g}}{R} 
& \jdeq 
  \subst{\jcomp{A}{f}{g}}{\ctxwk{A}{R}}
  \tag{by definition}
  \\
& \jdeq 
  \subst{{f}{\ctxwk{A}{g}}}{\ctxwk{A}{R}}
  \tag{by definition}
  \\
& \jdeq
  \subst{{f}{\ctxwk{A}{g}}}{{f}{\ctxwk{{A}{B}}{{A}{R}}}}
  \tag{by \autoref{cancellation-ws-t}}
  \\
& \jdeq 
  \subst{f}{{\ctxwk{A}{g}}{\ctxwk{{A}{B}}{{A}{R}}}}
  \tag{by \autoref{comp-ss-f}}
  \\
& \jdeq 
  \subst{f}{{\ctxwk{A}{g}}{\ctxwk{A}{{B}{R}}}}
  \tag{by \autoref{comp-ww-f}}
  \\
& \jdeq 
  \subst{f}{\ctxwk{A}{\subst{g}{\ctxwk{B}{R}}}}
  \tag{by \autoref{comp-ws-f}}
  \\
& \jdeq 
  \subst{f}{\ctxwk{A}{\jcomp{B}{g}{R}}}
  \tag{by definition}
  \\
& \jdeq 
  \jcomp{A}{f}{{B}{g}{R}}.
  \tag{by definition}
\end{align*}
Again, the proof is similar for terms $h$ of $R$ in context $\ctxext{\Gamma}{C}$.
\end{proof}

We established already in \autoref{lem:jcomp-wk} that composition is compatible
with weakening. However, in that lemma we did not consider the possibility
of weakening a composition and neither did we consider the possibility of
composition by or with a constant morphism. We do this in the following lemma.

\begin{lem}\label{lem:jcomp-const}
The inference
\begin{align*}
& \inference
  { \jhom{\Gamma}{A}{B}{f}
    \jhom{\Gamma}{B}{C}{g}
    \jfam{{\Gamma}{A}}{P}
    }
  { \jhomeq
      {\Gamma}
      {{A}{P}}
      {C}
      {\ctxwk{P}{\jcomp{A}{f}{g}}}
      {\jcomp{{A}{P}}{\ctxwk{P}{f}}{g}}
    }
\end{align*}
tells what happens when we weaken a composition, which is a term of 
$\ctxwk{A}{C}$, by a family $P$ over $\ctxext{\Gamma}{P}$. We also have the following
inference rules expressing that compositions by or with constant morphisms are
again constant morphisms.
\begin{align*}
& \inference
  { \jterm{\Gamma}{B}{y}
    \jhom{\Gamma}{B}{C}{g}
    }
  { \jhomeq
      {\Gamma}
      {A}
      {C}
      {\jcomp{A}{\ctxwk{A}{y}}{g}}
      {\ctxwk{A}{\subst{y}{g}}}
    }
  \\
& \inference
  { \jhom{\Gamma}{A}{B}{f}
    \jterm{\Gamma}{C}{z}
    }
  { \jhomeq
      {\Gamma}
      {A}
      {C}
      {\jcomp{A}{f}{\ctxwk{B}{z}}}
      {\ctxwk{A}{z}}
    }
\end{align*}
\end{lem}

\begin{proof}
Let $\jhom{\Gamma}{A}{B}{f}$, $\jhom{\Gamma}{B}{C}{g}$ and $\jfam{{\Gamma}{A}}{P}$.
Then we have the judgmental equalities\begin{align*}
\ctxwk{P}{\jcomp{A}{f}{g}} 
& \jdeq 
  \ctxwk{P}{\subst{f}{\ctxwk{A}{g}}}
  \tag{by definition}
  \\
& \jdeq 
  \subst{\ctxwk{P}{f}}{\ctxwk{P}{{A}{g}}}
  \tag{by \autoref{comp-ws-t}}
  \\
& \jdeq 
  \subst{\ctxwk{P}{f}}{\ctxwk{\ctxext{A}{P}}{g}}
  \tag{by \autoref{comp-ew-t}}
  \\
& \jdeq 
  \jcomp{{A}{P}}{\ctxwk{P}{f}}{g}.
  \tag{by definition}
\end{align*}
Now let $\jterm{\Gamma}{B}{y}$ and $\jhom{\Gamma}{B}{C}{g}$. Then we have the
judgmental equalities\begin{align*}
\jcomp{A}{\ctxwk{A}{y}}{g}
& \jdeq 
  \subst{\ctxwk{A}{y}}{\ctxwk{A}{g}}
  \tag{by definition}
  \\
& \jdeq 
  \ctxwk{A}{\subst{y}{g}}.
  \tag{by \autoref{comp-ws-t}}
\end{align*}
For the third assertion, let $\jhom{\Gamma}{A}{B}{f}$ and $\jterm{\Gamma}{C}{z}$.
Then we have the judgmental equalities\begin{align*}
\jcomp{A}{f}{\ctxwk{B}{z}} 
& \jdeq 
  \subst{f}{\ctxwk{A}{{B}{z}}}
  \tag{by definition}
  \\
& \jdeq 
  \subst{f}{\ctxwk{{A}{B}}{{A}{z}}}
  \tag{by \autoref{comp-ww-t}}
  \\
& \jdeq 
  \ctxwk{A}{z}.
  \tag{by \autoref{cancellation-ws-t}}
\end{align*}
\end{proof}

Likewise, we can substitute a term $\jterm{\Gamma}{A}{x}$ in a composed morphism
$\jhom{\Gamma}{A}{C}{\jcomp{A}{f}{g}}$. What we get is the \emph{value}
$\subst{{x}{f}}{g}$. This is the content of one part of the following lemma.

There is a second, related case where we can substitute: when
we consider morphisms $\jhom{{\Gamma}{A}}{P}{Q}{f}$ and 
$\jhom{{\Gamma}{A}}{Q}{R}{g}$ we can restrict them to the fibers, obtaining
the inference rule
\begin{equation*}
\inference
  { \jhom{{\Gamma}{A}}{P}{Q}{f}
    \jhom{{\Gamma}{A}}{Q}{R}{g}
    \jterm{\Gamma}{A}{x}
    }
  { \jhomeq
      {\Gamma}
      {\subst{x}{P}}
      {\subst{x}{R}}
      {\subst{x}{\jcomp{P}{f}{g}}}
      {\jcomp{\subst{x}{P}}{\subst{x}{f}}{\subst{x}{g}}}
    }
\end{equation*}
In the following lemma we prove the validity of a more general version of this
inference rule.

\begin{lem}\label{lem:jcomp-fiber}
The following inference rules are valid:
\begin{align*}
& \inference
  { \jhom{{\Gamma}{A}}{P}{Q}{f}
    \jfam{{{\Gamma}{A}}{Q}}{R}
    \jterm{\Gamma}{A}{x}
    }
  { \jfameq
      {{\Gamma}{\subst{x}{P}}}
      {\subst{x}{\jcomp{P}{f}{R}}}
      {\jcomp{\subst{x}{P}}{\subst{x}{f}}{\subst{x}{R}}}
    }
  \\
& \inference
  { \jhom{{\Gamma}{A}}{P}{Q}{f}
    \jfam{{{{\Gamma}{A}}{Q}}{R}}{S}
    \jterm{\Gamma}{A}{x}
    }
  { \jfameq
      {{{\Gamma}{\subst{x}{P}}}{\subst{x}{\jcomp{P}{f}{R}}}}
      {\subst{x}{\jcomp{P}{f}{S}}}
      {\jcomp{\subst{x}{P}}{\subst{x}{f}}{\subst{x}{S}}}
    }
  \\
& \inference
  { \jhom{{\Gamma}{A}}{P}{Q}{f}
    \jterm{{{{\Gamma}{A}}{Q}}{R}}{S}{k}
    \jterm{\Gamma}{A}{x}
    }
  { \jtermeq
      {{{\Gamma}{\subst{x}{P}}}{\subst{x}{\jcomp{P}{f}{R}}}}
      {\subst{x}{\jcomp{P}{f}{S}}}
      {\subst{x}{\jcomp{P}{f}{k}}}
      {\jcomp{\subst{x}{P}}{\subst{x}{f}}{\subst{x}{k}}}
    }
\end{align*}
Also, the following inference rule computing the value at $x$ of a composed
morphism is valid:
\begin{align*}
& \inference
  { \jhom{\Gamma}{A}{B}{f}
    \jhom{\Gamma}{B}{C}{g}
    \jterm{\Gamma}{A}{x}
    }
  { \jtermeq
      {\Gamma}
      {C}
      {\subst{x}{\jcomp{A}{f}{g}}}
      {\subst{{x}{f}}{g}}
    }
\end{align*}
\end{lem}

\begin{proof}
Of the first three inference rules, we only prove the first.
Let $\jhom{{\Gamma}{A}}{P}{Q}{f}$, $\jfam{{{\Gamma}{A}}{Q}}{R}$ and 
$\jterm{\Gamma}{A}{x}$.
Then we have the judgmental equalities\begin{align*}
\subst{x}{\jcomp{A}{f}{R}}
& \jdeq 
  \subst{x}{{f}{\ctxwk{P}{R}}}
  \tag{by definition}
  \\
& \jdeq 
  \subst{{x}{f}}{{x}{\ctxwk{P}{R}}}
  \tag{by \autoref{comp-ss-f}}
  \\
& \jdeq 
  \subst{{x}{f}}{\ctxwk{\subst{x}{P}}{\subst{x}{R}}}
  \tag{by \autoref{comp-sw-f}}
  \\
& \jdeq 
  \jcomp{\subst{x}{P}}{\subst{x}{f}}{\subst{x}{R}}.
  \tag{by definition}
\end{align*}
Now let $\jhom{\Gamma}{A}{B}{f}$, $\jhom{\Gamma}{B}{C}{g}$ and $\jterm{\Gamma}{A}{x}$.
Then we have the judgmental equalities\begin{align*}
\subst{x}{\jcomp{A}{f}{g}}
& \jdeq 
  \subst{x}{{f}{\ctxwk{A}{g}}}
  \tag{by definition}
  \\
& \jdeq 
  \subst{{x}{f}}{{x}{\ctxwk{A}{g}}}
  \tag{by \autoref{comp-ss-t}}
  \\
& \jdeq 
  \subst{{x}{f}}{g}.
  \tag{by \autoref{cancellation-ws-t}}
\end{align*}
\end{proof}

\subsection{Projections and extension on terms}\label{extension-on-terms}
In this subsection we consider the notion of extension on terms, which has now
become definable inside our theory. Moreover, every compatibility rule one may
dream of is provable as well, using the compatibility rules we have introduced
earlier.

\begin{defn}
When $\jterm{\Gamma}{A}{x}$ and $\jterm{\Gamma}{\subst{x}{P}}{u}$ are terms,
we define 
\begin{equation*}
\jtermdefn
  {\Gamma}
  {\ctxext{A}{P}}
  {\tmext{A}{P}{x}{u}}
  {\unfold{\tmext{A}{P}{x}{u}}}.
\end{equation*} 
\end{defn}

Thus, the term $\tmext{A}{P}{x}{u}$ is the pairing of $x$ and $u$. Note that because
we have the judgmental equality 
$\ctxwk{P}{{A}{\ctxext{A}{P}}}\jdeq\ctxwk{\ctxext{A}{P}}{\ctxext{A}{P}}$ in the
context $\ctxext{{\Gamma}{A}}{P}$, the
pairing function could just be defined as $\idtm{\ctxext{A}{P}}$. 

When we substitute by an extended term we get an equal result as when we
substitute two consecutive times, like the way currying works.

\begin{lem}\label{comp-es}
The following inference rules are valid:
\begin{align*}
& \inference
  { \jterm{\Gamma}{A}{x}
    \jterm{\Gamma}{\subst{x}{P}}{u}
    \jfam{{{\Gamma}{A}}{P}}{Q}
    }
  { \jfameq
      {\Gamma}
      {\subst{\tmext{A}{P}{x}{u}}{Q}}
      {\subst{u}{{x}{Q}}}
    }
  \\
& \inference
  { \jterm{\Gamma}{A}{x}
    \jterm{\Gamma}{\subst{x}{P}}{u}
    \jfam{{{{\Gamma}{A}}{P}}{Q}}{R}
    }
  { \jfameq
      {{\Gamma}{\subst{u}{{x}{Q}}}}
      {\subst{\tmext{A}{P}{x}{u}}{R}}
      {\subst{u}{{x}{R}}}
    }
  \\
& \inference
  { \jterm{\Gamma}{A}{x}
    \jterm{\Gamma}{\subst{x}{P}}{u}
    \jterm{{{{\Gamma}{A}}{P}}{Q}}{R}{t}
    }
  { \jtermeq
      {{\Gamma}{\subst{u}{{x}{Q}}}}
      {\subst{u}{{x}{R}}}
      {\subst{\tmext{A}{P}{x}{u}}{h}}
      {\subst{u}{{x}{h}}}
    }
\end{align*}
\end{lem}

\begin{proof}
We prove only the first judgmental equality. All the others are similar.
Let $\jterm{\Gamma}{A}{x}$ and $\jterm{\Gamma}{\subst{x}{P}}{u}$
be terms and let $\jfam{{{\Gamma}{A}}{P}}{Q}$ be a family. Then we have
\begin{align*}
\subst
  {\tmext{A}{P}{x}{u}}
  {Q} 
& \jdeq 
  \subst
    {{u}{{x}{\idtm{\ctxext{A}{P}}}}}
    {Q}
  \tag{by definition}\\
& \jdeq 
  \subst
    {{u}{{x}{\idtm{\ctxext{A}{P}}}}}
    {{x}{\ctxwk{A}{Q}}}
  \tag{by \autoref{cancellation-ws-f}}\\
& \jdeq 
  \subst
    {{u}{{x}{\idtm{\ctxext{A}{P}}}}}
    {{u}{\ctxwk{\subst{x}{P}}{\subst{x}{\ctxwk{A}{Q}}}}}
  \tag{by \autoref{cancellation-ws-f}}\\
& \jdeq 
  \subst
    {{u}{{x}{\idtm{\ctxext{A}{P}}}}}
    {{u}{{x}{\ctxwk{P}{{A}{Q}}}}}
  \tag{by \autoref{comp-sw-f}}\\
& \jdeq 
  \subst
    {u}
    {{{x}{\idtm{\ctxext{A}{P}}}}{{x}{\ctxwk{{P}{{A}{Q}}}}}}
  \tag{by \autoref{comp-ss-f}}\\
& \jdeq 
  \subst
    {u}
    {{x}{{\idtm{\ctxext{A}{P}}}{\ctxwk{P}{{A}{Q}}}}}
  \tag{by \autoref{comp-ss-f}}\\
& \jdeq 
  \subst
    {u}
    {{x}{{\idtm{\ctxext{A}{P}}}{\ctxwk{\ctxext{A}{P}}{Q}}}}
  \tag{by \autoref{comp-ew-f}}\\
& \jdeq 
  \subst
    {u}
    {{x}{Q}}
  \tag{by \autoref{idfunc-wk-defn}}
\end{align*}
\end{proof}

We have seen above that the pairing function into $\ctxext{A}{P}$ is just the identity term on
$\ctxext{A}{P}$. To analyze the pairing functin a little further, we will also
need the projection maps from $\ctxext{A}{P}$ to $A$ and from $\ctxext{A}{P}$
to $P$. We will now define these and see that the identity term of an
extended family is the extension (or pairing) of the identity
functions on the components in the apropriate way.

To find out what the
apropriate way is, note that
\begin{align*}
\ctxwk{\ctxext{A}{P}}{\ctxext{A}{P}}
& \jdeq
  \ctxext{\ctxwk{\ctxext{A}{P}}{A}}{\ctxwk{\ctxext{A}{P}}{P}}
  \tag{by \autoref{comp-we-f}}
  \\
& \jdeq
  \ctxext{\ctxwk{P}{{A}{A}}}{\ctxwk{P}{{A}{P}}}
  \tag{by \autoref{comp-ew-f}}
\end{align*}
We have the term $\jterm{{\Gamma}{{A}{P}}}{\ctxwk{P}{A}}{\ctxwk{P}{\idtm{A}}}$.
Thus we need to find out what $\subst{\ctxwk{P}{\idtm{A}}}{\ctxwk{P}{{A}{P}}}$ is:
\begin{align*}
\subst{\ctxwk{P}{\idtm{A}}}{\ctxwk{P}{{A}{P}}} 
& \jdeq 
  \ctxwk{P}{\subst{\idtm{A}}{\ctxwk{A}{P}}}
  \tag{by \autoref{comp-ws-f}}
  \\
& \jdeq 
  \ctxwk{P}{P},
  \tag{by \autoref{idfunc-wk-defn}}
\end{align*}
where we find the term $\idtm{P}$. Therefore we define:

\begin{defn}
Let $\jfam{\Gamma}{A}$ and $\jfam{{\Gamma}{A}}{P}$ be families. We define
\begin{align*}
\jhomdefn*
  {\Gamma}
  {{A}{P}}
  {A}
  {\cprojfstf{A}{P}}
  {\unfold{\cprojfstf{A}{P}}}
  \\
\jtermdefn*
  {\ctxext{\Gamma}{{A}{P}}}
  {\ctxwk{P}{P}}
  {\cprojsndf{A}{P}}
  {\unfold{\cprojsndf{A}{P}}}
\end{align*}
\end{defn}

The constructions of the terms $\tmext{A}{P}{x}{u}$ and $\cprojfst{A}{P}{w}$ and
$\cprojsnd{A}{P}{w}$ are subject to various rules, with all of them being
consequences of earlier introduced inference rules.

\begin{lem}\label{lem:tmext-basic}
The following inference rules expressing that pairing is a strict
inverse to the combination of decompositions, are valid:
\begin{align*}
& \inference
  { \jterm{\Gamma}{\ctxext{A}{P}}{w}
    }
  { \jtermeq
      {\Gamma}
      {\ctxext{A}{P}}
      {\tmext{A}{P}{\cprojfst{A}{P}{w}}{\cprojsnd{A}{P}{w}}}
      {w}
    }
  \\
& \inference
  { \jterm{\Gamma}{A}{x}
    \jterm{\Gamma}{\subst{x}{P}}{u}
    }
  { \jtermeq
      {\Gamma}
      {A}
      {\cprojfst{A}{P}{\tmext{A}{P}{x}{u}}}
      {x}
    }
  \\
& \inference
  { \jterm{\Gamma}{A}{x}
    \jterm{\Gamma}{\subst{x}{P}}{u}
    }
  { \jtermeq
      {\Gamma}
      {\subst{x}{P}}
      {\cprojsnd{A}{P}{\tmext{A}{P}{x}{u}}}
      {u}
    }
\end{align*}
\end{lem}

\begin{proof}
To prove the first judgmental equality, note that
\begin{align*}
w 
& \jdeq 
  \subst{w}{\idtm{\ctxext{A}{P}}} 
  \tag{by \autoref{cancellation-si}}\\
& \jdeq 
  \subst
    { w}
    { { \idtm{P}}
      { { \ctxwk{P}{\idtm{A}}}
        { \idtm{\ctxwk{\ctxext{A}{P}}{\ctxext{A}{P}}}}
        }
      }
  \tag{by \autoref{idfunc-ext-comp}}\\
& \jdeq 
  \subst
    { {w}
      {\idtm{P}}
      }
    { {w}
      { { \ctxwk{P}{\idtm{A}}
          }
        { \idtm{\ctxwk{\ctxext{A}{P}}{\ctxext{A}{P}}}
          }
        }
      }
  \tag{by \autoref{comp-ss-t}}\\
& \jdeq 
  \subst
    { {w}
      {\idtm{P}}
      }
    { { {w}
        {\ctxwk{P}{\idtm{A}}}
        }
      { {w}
        {\idtm{\ctxwk{\ctxext{A}{P}}{\ctxext{A}{P}}}}
        }
      }
  \tag{by \autoref{comp-ss-t}}\\
& \jdeq 
  \subst
    { {w}
      {\idtm{P}}
      }
    { { {w}
        {\ctxwk{P}{\idtm{A}}}
        }
      { {w}
        {\ctxwk{\ctxext{A}{P}}{\idtm{\ctxext{A}{P}}}}
        }
      }
  \tag{by \autoref{comp-wi-t}}\\
& \jdeq 
  \subst
    { {w}
      {\idtm{P}}
      }
    { { {w}
        {\ctxwk{P}{\idtm{A}}}
        }
      { \idtm{\ctxext{A}{P}}
        }
      }
  \tag{by \autoref{cancellation-ws-t}}\\
& \jdeq 
  \tmext{A}{P}{\cprojfst{A}{P}{w}}{\cprojsnd{A}{P}{w}}
  \tag{by definition}
\end{align*}
To prove the second judgmental equality, let $\jterm{\Gamma}{A}{x}$ and
$\jterm{\Gamma}{\subst{x}{P}}{u}$. Then we have
\begin{align*}
\cprojfst{A}{P}{\tmext{A}{P}{x}{u}}
& \jdeq 
  \subst{\tmext{A}{P}{x}{u}}{\ctxwk{P}{\idtm{A}}}
  \tag{by definition}
  \\
& \jdeq 
  \subst{u}{{x}{\ctxwk{P}{\idtm{A}}}} 
  \tag{by \autoref{comp-es}}
  \\
& \jdeq 
  \subst{u}{\ctxwk{\subst{x}{P}}{\subst{x}{\idtm{A}}}}
  \tag{by \autoref{comp-sw-t}}
  \\
& \jdeq 
  \subst{x}{\idtm{A}}
  \tag{by \autoref{cancellation-ws-t}}
  \\
& \jdeq 
  x.
  \tag{by \autoref{cancellation-si}}
\end{align*}
To prove the third judgmental equality, note that
\begin{align*}
\cprojsnd{A}{P}{\tmext{A}{P}{x}{u}}
& \jdeq 
  \subst{\tmext{A}{P}{x}{u}}{\idtm{P}}
  \tag{by definition}
  \\
& \jdeq 
  \subst{u}{{x}{\idtm{P}}}
  \tag{by \autoref{comp-es}}
  \\
& \jdeq 
  \subst{u}{\idtm{\subst{x}{P}}}
  \tag{by \autoref{comp-si-t}}
  \\
& \jdeq 
  u.
  \tag{by \autoref{cancellation-si}}
\end{align*}
\end{proof}

In \autoref{lem:tmext-emp,lem:tmext-ext,lem:tmext-wk,lem:tmext-subst,lem:tmext-id}
we show that term extension and the projections are
compatible with the empty families, extension,
weakening, substitution and the identity terms (in that order). \autoref{lem:tmext-id}
is in fact a generalization of the above \autoref{lem:tmext-basic}.

\begin{lem}\label{lem:tmext-emp}
The following compatibility rules for extensions of the term of the empty family
are valid:
\begin{align*}
& \inference
  { \jterm{\Gamma}{A}{x}
    }
  { \jtermeq{\Gamma}{A}{\tmext{\emptytm}{x}}{x}
    }
& & \inference
  { \jterm{\Gamma}{A}{x}
    }
  { \jtermeq{\Gamma}{A}{\tmext{x}{\emptytm}}{x}
    }
  \\
& \inference
  { \jfam{\Gamma}{A}
    }
  { \jtermeq
      {{\Gamma}{A}}
      {\emptyf}
      {\cprojfstf{\emptyf}{A}}
      {\emptytm}
    }
& & \inference
  { \jfam{\Gamma}{A}
    }
  { \jhomeq
      {\Gamma}
      {A}
      {A}
      {\cprojfstf{A}{\emptyf}}
      {\idtm{A}}
    }
  \\
& \inference
  { \jfam{\Gamma}{A}
    }
  { \jhomeq
      {\Gamma}
      {A}
      {A}
      {\cprojsndf{\emptyf}{A}}
      {\idtm{A}}
    }
& & \inference
  { \jfam{\Gamma}{A}
    }
  { \jtermeq
      {{\Gamma}{A}}
      {\emptyf}
      {\cprojsndf{A}{\emptyf}}
      {\emptytm}
    }
\end{align*}
\end{lem}

\begin{proof}
These equalities are very easy to verify. We only display a proof of the first:
\begin{equation*}
\tmext{\emptytm}{x}
\jdeq \unfold{\tmext{\emptyf}{A}{\emptytm}{x}}
\jdeq \subst{x}{\idtm{{\emptyf}{A}}}
\jdeq \subst{x}{\idtm{A}}
\jdeq x.\qedhere
\end{equation*}
\end{proof}

\begin{lem}\label{lem:tmext-ext}
The following compatibility rules for two consecutive term extensions are valid:
\begin{align*}
& \inference
  { \jterm{\Gamma}{A}{x}
    \jterm{\Gamma}{\subst{x}{P}}{u}
    \jterm{\Gamma}{\subst{\tmext{A}{P}{x}{u}}{Q}}{v}
    }
  { \jtermeq
      {\Gamma}
      {\ctxext{{A}{P}}{Q}}
      {\tmext{A}{{P}{Q}}{x}{{\subst{x}{P}}{\subst{x}{Q}}{u}{v}}}
      {\tmext{{A}{P}}{Q}{{A}{P}{x}{u}}{v}}
    }
  \\
& \inference
  { \jfam{{{\Gamma}{A}}{P}}{Q}
    }
  { \jhomeq
      {\Gamma}
      {\ctxext{{A}{P}}{Q}}
      {A}
      {\jcomp{}{\cprojfstf{{A}{P}}{Q}}{\cprojfstf{A}{P}}}
      {\cprojfstf{A}{{P}{Q}}}
    }
  \\
& \inference
  { \jfam{{{\Gamma}{A}}{P}}{Q}
    }
  { \jhomeq
      {{\Gamma}{A}}
      {{P}{Q}}
      {P}
      {\jcomp{}{\cprojfstf{{A}{P}}{Q}}{\cprojsndf{A}{P}}}
      {\jcomp{}{\cprojsndf{A}{{P}{Q}}}{\cprojfstf{P}{Q}}}
    }
  \\
& \inference
    { \jfam{{{\Gamma}{A}}{P}}{Q}
      }
    { \jhomeq
        {{{\Gamma}{A}}{P}}
        {Q}
        {Q}
        {\cprojsndf{{A}{P}}{Q}}
        {\jcomp{}{\cprojsndf{A}{{P}{Q}}}{\cprojsndf{P}{Q}}}
      }
\end{align*}
\end{lem}

\begin{proof}
Consider terms $\jterm{\Gamma}{A}{x}$, $\jterm{\Gamma}{\subst{x}{P}}{u}$ and
$\jterm{\Gamma}{\subst{u}{{x}{Q}}}{v}$. Then we have
\begin{align*}
\tmext{x}{{u}{v}}
& \jdeq 
  \subst
    {\tmext{u}{v}}{{x}{\idtm{\ctxext{A}{{P}{Q}}}}}
  \\
& \jdeq 
  \subst{v}{{u}{{x}{\idtm{\ctxext{A}{{P}{Q}}}}}}
  \\
& \jdeq 
  \subst{v}{{u}{{x}{\idtm{\ctxext{{A}{P}}{Q}}}}}
  \\
& \jdeq 
  \subst{v}{{\tmext{x}{u}}{\idtm{\ctxext{{A}{P}}{Q}}}}
  \\
& \jdeq 
  \tmext{{x}{u}}{v}.
\end{align*}
To prove the judmental equality
\begin{equation*}
\jhomeq
  {\Gamma}
  {\ctxext{{A}{P}}{Q}}
  {A}
  {\jcomp{}{\cprojfstf{{A}{P}}{Q}}{\cprojfstf{A}{P}}}
  {\cprojfstf{A}{{P}{Q}}}
\end{equation*}
note that we have the judgmental equalities
\begin{align*}
\jcomp{{{A}{P}}{Q}}{\cprojfstf{{A}{P}}{Q}}{\cprojfstf{A}{P}}
& \jdeq 
  \unfoldall{\jcomp{{{A}{P}}{Q}}{\cprojfstf{{A}{P}}{Q}}{\cprojfstf{A}{P}}}
  \\
& \jdeq 
  \subst
    {\ctxwk{Q}{\idtm{{A}{P}}}}
    {\ctxwk{Q}{{\ctxext{A}{P}}{{P}{\idtm{A}}}}}
  \\
& \jdeq
  \ctxwk{Q}{\subst{\idtm{{A}{P}}}{\ctxwk{\ctxext{A}{P}}{{P}{\idtm{A}}}}}
  \\
& \jdeq
  \ctxwk{Q}{{P}{\idtm{A}}}
  \\
& \jdeq
  \ctxwk{\ctxext{P}{Q}}{\idtm{A}}
  \\
& \jdeq
  \cprojfstf{A}{{P}{Q}}
\end{align*}
To prove the judgmental equality
\begin{equation*}
\jhomeq
  {{\Gamma}{A}}
  {{P}{Q}}
  {P}
  {\jcomp{}{\cprojfstf{{A}{P}}{Q}}{\cprojsndf{A}{P}}}
  {\jcomp{}{\cprojsndf{A}{{P}{Q}}}{\cprojfstf{P}{Q}}}
\end{equation*}
note that we have the judgmental equalities
\begin{align*}
\jcomp{{{A}{P}}{Q}}{\cprojfstf{{A}{P}}{Q}}{\cprojsndf{A}{P}}
& \jdeq 
  \unfoldall{\jcomp{{{A}{P}}{Q}}{\cprojfstf{{A}{P}}{Q}}{\cprojsndf{A}{P}}}
  \\
& \jdeq
  \subst{\ctxwk{Q}{\idtm{{A}{P}}}}{\ctxwk{Q}{{\ctxext{A}{P}}{\idtm{P}}}}
  \\
& \jdeq
  \ctxwk{Q}{\subst{\idtm{{A}{P}}}{\ctxwk{\ctxext{A}{P}}{\idtm{P}}}}
  \\
& \jdeq
  \ctxwk{Q}{\idtm{P}}
  \\
& \jdeq
  \unfoldall{\jcomp{{P}{Q}}{\cprojsndf{A}{{P}{Q}}}{\cprojfstf{P}{Q}}}
  \\
& \jdeq
  \jcomp{{P}{Q}}{\cprojsndf{A}{{P}{Q}}}{\cprojfstf{P}{Q}}
\end{align*}
To prove the judgmental equality
\begin{equation*}
\jhomeq
  {{{\Gamma}{A}}{P}}
  {Q}
  {Q}
  {\cprojsndf{{A}{P}}{Q}}
  {\jcomp{}{\cprojsndf{A}{{P}{Q}}}{\cprojsndf{P}{Q}}}
\end{equation*}
note that we have the judgmental equalities
\begin{align*}
\cprojsndf{{A}{P}}{Q}
& \jdeq
  \unfoldall{\cprojsndf{{A}{P}}{Q}}
  \\
& \jdeq 
  \unfoldall{\jcomp{{P}{Q}}{\cprojsndf{A}{{P}{Q}}}{\cprojsndf{P}{Q}}}
  \\
& \jdeq
  \jcomp{}{\cprojsndf{A}{{P}{Q}}}{\cprojsndf{P}{Q}}\qedhere
\end{align*}
\begin{comment}
%%%% This was a proof of an old version of the statement
Now consider a term $\jterm{\Gamma}{\ctxext{A}{{P}{Q}}}{w}$. Then we have
\begin{align*}
w 
& \jdeq 
  \tmext
    {A}
    {{P}{Q}}
    {\cprojfst{A}{\ctxext{P}{Q}}{w}}
    {\cprojsnd{A}{\ctxext{P}{Q}}{w}}
  \\
& \jdeq 
  \tmext
    {A}
    {{P}{Q}}
    {\cprojfst{A}{\ctxext{P}{Q}}{w}}
    { {P}
      {Q}
      {\cprojfst{P}{Q}{\cprojsnd{A}{\ctxext{P}{Q}}{w}}}
      {\cprojsnd{P}{Q}{\cprojsnd{A}{\ctxext{P}{Q}}{w}}}
      }
  \\
& \jdeq 
  \tmext
    {{A}{P}}
    {Q}
    { {} % need to provide base and family, but there's no unfold.
      {}
      {\cprojfst{A}{\ctxext{P}{Q}}{w}}
      {\cprojfst{P}{Q}{\cprojsnd{A}{\ctxext{P}{Q}}{w}}}
      }
    { \cprojsnd{P}{Q}{\cprojsnd{A}{\ctxext{P}{Q}}{w}}
      }
\end{align*}
Thus we see that 
\begin{align*}
\cprojfst{\ctxext{A}{P}}{Q}{w} 
& \jdeq 
  \tmext
    {A}
    {P}
    {\cprojfst{A}{\ctxext{P}{Q}}{w}}
    {\cprojfst{P}{Q}{\cprojsnd{A}{\ctxext{P}{Q}}{w}}}
  \\ 
\cprojsnd{\ctxext{A}{P}}{Q}{w} 
& \jdeq 
  \cprojsnd{P}{Q}{\cprojsnd{A}{\ctxext{P}{Q}}{w}},
\end{align*}
proving the fourth judgmental equality, and therefore also that
\begin{align*}
\cprojfst{A}{P}{\cprojfst{\ctxext{A}{P}}{Q}{w}} 
& \jdeq 
  \cprojfst{A}{\ctxext{P}{Q}}{w}
  \\
\cprojsnd{A}{P}{\cprojfst{\ctxext{A}{P}}{Q}{w}} 
& \jdeq 
  \cprojfst{P}{Q}{\cprojsnd{A}{\ctxext{P}{Q}}{w}},
\end{align*}
proving the second and the third judgmental equalities.
\end{comment}
\end{proof}

\begin{lem}\label{lem:tmext-wk}\label{comp-we-t}
When we weaken a term $\tmext{B}{Q}{y}{v}$ of $\ctxext{B}{Q}$ in context 
$\Gamma$ by a family $A$, the term that we get is 
$\tmext{\ctxwk{A}{B}}{\ctxwk{A}{Q}}{\ctxwk{A}{y}}{\ctxwk{A}{v}}$. More
precisely, the following inference rules are valid:
\begin{align*}
& \inference
  { \jterm{{\Gamma}{B}}{Q}{g}
    \jterm{{\Gamma}{B}}{\subst{g}{R}}{t}
    }
  { \jtermeq
      {{{\Gamma}{A}}{\ctxwk{A}{B}}}
      {\ctxwk{A}{\ctxext{Q}{R}}}
      {\ctxwk{A}{\tmext{Q}{R}{g}{t}}}
      {\tmext{\ctxwk{A}{Q}}{\ctxwk{A}{R}}{\ctxwk{A}{g}}{\ctxwk{A}{t}}}
    }
  \\
& \inference
  { \jfam{{{\Gamma}{B}}{Q}}{R}
    }
  { \jhomeq
      {{{\Gamma}{A}}{\ctxwk{A}{B}}}
      {{\ctxwk{A}{Q}}{\ctxwk{A}{R}}}
      {\ctxwk{A}{Q}}
      {\cprojfstf{\ctxwk{A}{Q}}{\ctxwk{A}{R}}}
      {\ctxwk{A}{\cprojfstf{Q}{R}}}
    }
  \\
& \inference
  { \jfam{{{\Gamma}{B}}{Q}}{R}
    }
  { \jhomeq
      {{{{\Gamma}{A}}{\ctxwk{A}{B}}}{\ctxwk{A}{Q}}}
      {\ctxwk{A}{R}}
      {\ctxwk{A}{R}}
      {\cprojsndf{\ctxwk{A}{Q}}{\ctxwk{A}{R}}}
      {\ctxwk{A}{\cprojsndf{Q}{R}}}
    }
\end{align*}
\end{lem}

\begin{proof}
Consider $\jterm{{\Gamma}{B}}{Q}{g}$ and $\jterm{{\Gamma}{B}}{\subst{g}{R}}{t}$.
Then we have the judgmental equalities
\begin{align*}
\ctxwk{A}{\ctxext{Q}{R}{g}{t}}
& \jdeq 
  \ctxwk{A}{\subst{t}{{g}{\idtm{\ctxext{Q}{R}}}}}
  \\
& \jdeq 
  \subst{\ctxwk{A}{t}}{\ctxwk{A}{\subst{g}{\idtm{\ctxext{Q}{R}}}}}
  \\
& \jdeq 
  \subst{\ctxwk{A}{t}}{{\ctxwk{A}{g}}{\ctxwk{A}{\idtm{\ctxext{Q}{R}}}}}
  \\
& \jdeq 
  \subst{\ctxwk{A}{t}}{{\ctxwk{A}{g}}{\idtm{\ctxwk{A}{\ctxext{Q}{R}}}}}
  \\
& \jdeq 
  \subst{\ctxwk{A}{t}}{{\ctxwk{A}{g}}{\idtm{\ctxext{\ctxwk{A}{Q}}{\ctxwk{A}{R}}}}}
  \\
& \jdeq 
  \tmext{\ctxwk{A}{Q}}{\ctxwk{A}{R}}{\ctxwk{A}{g}}{\ctxwk{A}{t}}
\end{align*}
Next, we want to prove the judgmental equality
\begin{equation*}
\jhomeq
  {{{\Gamma}{A}}{\ctxwk{A}{B}}}
  {{\ctxwk{A}{Q}}{\ctxwk{A}{R}}}
  {\ctxwk{A}{Q}}
  {\cprojfstf{\ctxwk{A}{Q}}{\ctxwk{A}{R}}}
  {\ctxwk{A}{\cprojfstf{Q}{R}}}
\end{equation*}
Note that we have the judgmental equalities
\begin{align*}
\cprojfstf{\ctxwk{A}{Q}}{\ctxwk{A}{R}}
& \jdeq
  \unfoldall{\cprojfstf{\ctxwk{A}{Q}}{\ctxwk{A}{R}}}
  \\
& \jdeq
  \ctxwk{{A}{R}}{{A}{\idtm{Q}}}
  \\
& \jdeq
  \unfoldall{\ctxwk{A}{\cprojfstf{Q}{R}}}
  \\
& \jdeq
  \ctxwk{A}{\cprojfstf{Q}{R}}.
\end{align*}
Finally, we want to prove the judgmental equality
\begin{equation*}
\jhomeq
  {{{{\Gamma}{A}}{\ctxwk{A}{B}}}{\ctxwk{A}{Q}}}
  {\ctxwk{A}{R}}
  {\ctxwk{A}{R}}
  {\cprojsndf{\ctxwk{A}{Q}}{\ctxwk{A}{R}}}
  {\ctxwk{A}{\cprojsndf{Q}{R}}}
\end{equation*}
Note that we have the judgmental equalities
\begin{align*}
\cprojsndf{\ctxwk{A}{Q}}{\ctxwk{A}{R}}
& \jdeq
  \unfoldall{\cprojsndf{\ctxwk{A}{Q}}{\ctxwk{A}{R}}}
  \\
& \jdeq
  \unfoldall{\ctxwk{A}{\cprojsndf{Q}{R}}}
  \\
& \jdeq
  \ctxwk{A}{\cprojsndf{Q}{R}}.
  \qedhere
\end{align*}
\end{proof}

\begin{lem}\label{lem:tmext-subst}\label{comp-se-t}
When we substitute an extended term $\tmext{P}{Q}{f}{g}$ of $\ctxext{P}{Q}$ by a term
$x$ of $A$, the term that we get is $\tmext{\subst{x}{P}}{\subst{x}{Q}}{\subst{x}{f}}{\subst{x}{g}}$.
More precisely, the following inference rules are valid:
\begin{align*}
& \inference
  { \jterm{\Gamma}{A}{x}
    \jterm{{{\Gamma}{A}}{P}}{Q}{g}
    \jterm{{{\Gamma}{A}}{P}}{\subst{g}{R}}{t}
    }
  { \jtermeq
      {{\Gamma}{\subst{x}{P}}}
      {\ctxext{\subst{x}{Q}}{\subst{x}{R}}}
      {\subst{x}{\tmext{Q}{R}{g}{t}}}
      {\tmext{\subst{x}{Q}}{\subst{x}{R}}{\subst{x}{g}}{\subst{x}{t}}}
    }
  \\
& \inference
  { \jterm{\Gamma}{A}{x}
    \jfam{{{{\Gamma}{A}}{P}}{Q}}{R}
    }
  { \jhomeq
      {{\Gamma}{\subst{x}{P}}}
      {{\subst{x}{Q}}{\subst{x}{R}}}
      {\subst{x}{Q}}
      {\cprojfstf{\subst{x}{Q}}{\subst{x}{R}}}
      {\subst{x}{\cprojfstf{Q}{R}}}
    }
  \\
& \inference
  { \jterm{\Gamma}{A}{x}
    \jfam{{{{\Gamma}{A}}{P}}{Q}}{R}
    }
  { \jhomeq
      {{{\Gamma}{\subst{x}{P}}}{\subst{x}{Q}}}
      {\subst{x}{R}}
      {\subst{x}{R}}
      {\cprojsndf{\subst{x}{Q}}{\subst{x}{R}}}
      {\subst{x}{\cprojsndf{Q}{R}}}
    }
\end{align*}
\end{lem}

\begin{proof}
Consider $\jterm{{\Gamma}{B}}{Q}{g}$ and $\jterm{{\Gamma}{B}}{\subst{g}{R}}{t}$.
Then we have the judgmental equalities
\begin{align*}
\subst{x}{\tmext{Q}{R}{g}{t}}
& \jdeq 
  \subst{x}{{t}{{g}{\idtm{\ctxext{Q}{R}}}}}
  \\
& \jdeq 
  \subst{{x}{t}}{{x}{{g}{\idtm{\ctxext{Q}{R}}}}}
  \\
& \jdeq 
  \subst{{x}{t}}{{{x}{g}}{{x}{\idtm{\ctxext{Q}{R}}}}}
  \\
& \jdeq 
  \subst{{x}{t}}{{{x}{g}}{\idtm{\subst{x}{\ctxext{Q}{R}}}}}
  \\
& \jdeq 
  \subst{{x}{t}}{{{x}{g}}{\idtm{\ctxext{\subst{x}{Q}}{\subst{x}{R}}}}}
  \\
& \jdeq 
  \tmext{\subst{x}{Q}}{\subst{x}{R}}{\subst{x}{g}}{\subst{x}{t}}.
\end{align*}
Next, we want to prove the judgmental equality
\begin{equation*}
\jhomeq
  {{\Gamma}{\subst{x}{P}}}
  {{\subst{x}{Q}}{\subst{x}{R}}}
  {\subst{x}{Q}}
  {\cprojfstf{\subst{x}{Q}}{\subst{x}{R}}}
  {\subst{x}{\cprojfstf{Q}{R}}}
\end{equation*}
Note that we have the judgmental equalities
\begin{align*}
\cprojfstf{\subst{x}{Q}}{\subst{x}{R}}
& \jdeq
  \unfoldall{\cprojfstf{\subst{x}{Q}}{\subst{x}{R}}}
  \\
& \jdeq
  \ctxwk{\subst{x}{R}}{\subst{x}{\idtm{Q}}}
  \\
& \jdeq
  \unfoldall{\subst{x}{\cprojfstf{Q}{R}}}
  \\
& \jdeq
  \subst{x}{\cprojfstf{Q}{R}}.
\end{align*}
And finally we want to prove the judgmental equality
\begin{equation*}
\jhomeq
  {{{\Gamma}{\subst{x}{P}}}{\subst{x}{Q}}}
  {\subst{x}{R}}
  {\subst{x}{R}}
  {\cprojsndf{\subst{x}{Q}}{\subst{x}{R}}}
  {\subst{x}{\cprojsndf{Q}{R}}}
\end{equation*}
Note that we have the judgmental equalities
\begin{align*}
\cprojsndf{\subst{x}{Q}}{\subst{x}{R}}
& \jdeq
  \unfoldall{\cprojsndf{\subst{x}{Q}}{\subst{x}{R}}}
  \\
& \jdeq
  \unfoldall{\subst{x}{\cprojsndf{Q}{R}}}
  \\
& \jdeq
  \subst{x}{\cprojsndf{Q}{R}}.
  \qedhere
\end{align*}
\end{proof}

We find the following inference rule, which expresses that the identity term
is compatible with extension:

\begin{lem}\label{lem:tmext-id}\label{comp-ie}
For any $\jfam{\Gamma}{A}$ and $\jfam{{\Gamma}{A}}{P}$ we have
\begin{equation}\label{idfunc-ext-comp}
\inference
  { \jfam{\Gamma}{A}
    \jfam{{\Gamma}{A}}{P}
    }
  { \jhomeq
      {\Gamma}
      {{A}{P}}{{A}{P}}
      {\idtm{\ctxext{A}{P}}}
      { \tmext
          {\ctxwk{\ctxext{A}{P}}{A}}
          {\ctxwk{\ctxext{A}{P}}{P}}
          {\cprojfstf{A}{P}}
          {\cprojsndf{A}{P}}
        }
    }
\end{equation}
\end{lem}

\begin{proof}
Consider the families $\jfam{\Gamma}{A}$ and $\jfam{{\Gamma}{A}}{P}$. Then
we have the judgmental equalities
\begin{align*}
\tmext
  {\ctxwk{\ctxext{A}{P}}{A}}
  {\ctxwk{\ctxext{A}{P}}{P}}
  {\cprojfstf{A}{P}}
  {\cprojsndf{A}{P}}
& \jdeq 
  \unfold
  { \tmext
      {\ctxwk{\ctxext{A}{P}}{A}}
      {\ctxwk{\ctxext{A}{P}}{P}}
      {\cprojfstf{A}{P}}
      {\cprojsndf{A}{P}}
    }
  \\
& \jdeq 
  \subst
    { \idtm{P}
      }
    { {\ctxwk{P}{\idtm{A}}}
      {\idtm{\ctxwk{\ctxext{A}{P}}{\ctxext{A}{P}}}}
      }
  \\
& \jdeq 
  \subst
    { \idtm{P}
      }
    { {\ctxwk{P}{\idtm{A}}}
      {\ctxwk{\ctxext{A}{P}}{\idtm{\ctxext{A}{P}}}}
      }
  \\
& \jdeq 
  \subst
    { \idtm{P}
      }
    { {\ctxwk{P}{\idtm{A}}}
      {\ctxwk{P}{{A}{\idtm{\ctxext{A}{P}}}}}
      }
  \\
& \jdeq 
  \subst
    { \idtm{P}
      }
    {\ctxwk
      {P}
      { \subst
        {\idtm{A}}
        {\ctxwk{A}{\idtm{\ctxext{A}{P}}}}
        }
      }
  \\
& \jdeq 
  \subst
    { \idtm{A}
      }
    { \ctxwk{A}{\idtm{\ctxext{A}{P}}}
      }
  \\
& \jdeq 
  \idtm{\ctxext{A}{P}}.
  \qedhere
\end{align*}
\end{proof}

\subsection{Family morphisms}
In this section we are going to develop dependent morphisms and analyze various
sorts of composition. The notion of family morphism is the notion of `a 
morphism over a context morphism'. When we have established all the associativity
and interchange laws for all the compositions that come with the theory of
family morphisms, we will also be able to consider \emph{family diagrams}. Those
are graphical displays of situations in the theory of contexts, families and
terms, but unlike diagrams in category theory they can contain families (which
are strictly speaking not morphisms) and terms (which can be seen as morphisms,
but not in the way we're about to display them).

For the following lemma, recall that the judgment $\jhom{\Gamma}{A}{{B}{Q}}{f}$
unfolds as
\begin{equation*}
\unfold{\jhom{\Gamma}{A}{{B}{Q}}{f}}
\end{equation*}
and that we have the judgmental equality 
$ \jfameq
    {{\Gamma}{A}}
    {\ctxwk{A}{\ctxext{B}{Q}}}
    {\ctxext{\ctxwk{A}{B}}{\ctxwk{A}{Q}}}.
  $
Therefore, each morphism into an extended family can itself be described as
an extended term. The following lemma explains how this goes.

\begin{lem}
Let $\jhom{\Gamma}{A}{{B}{Q}}{f}$ be a morphism from $A$ to $\ctxext{B}{Q}$
in a context $\Gamma$. Then we have
\begin{equation*}
\jhomeq
  {\Gamma}
  {A}
  {{B}{Q}}
  {f}
  {\tmext{\jcomp{}{f}{\cprojfstf{B}{Q}}}{\jcomp{}{f}{\cprojsndf{B}{Q}}}}.
\end{equation*}
Alternatively, when $\jhom{\Gamma}{A}{B}{f_0}$ and 
$\jterm{{\Gamma}{A}}{\jcomp{}{f_0}{Q}}{f_1}$ we obtain a morphism
$\jhom{\Gamma}{A}{{B}{Q}}{\tmext{f_0}{f_1}}$ with the property that
\begin{align*}
\jhomeq*{\Gamma}{A}{B}{\jcomp{}{\tmext{f_0}{f_1}}{\cprojfstf{B}{Q}}}{f_0}\\
\jtermeq*{{\Gamma}{A}}{\jcomp{}{f_0}{Q}}{\jcomp{}{\tmext{f_0}{f_1}}{\cprojsndf{B}{Q}}}{f_1}.
\end{align*}
\end{lem}

\begin{proof}
Let $\jhom{\Gamma}{A}{{B}{Q}}{f}$ be a morphism in context $\Gamma$. Then we
have the judgmental equalities
\begin{align*}
\cprojfst{\ctxwk{A}{B}}{\ctxwk{A}{Q}}{f}
& \jdeq
  \subst{f}{\ctxwk{A}{\cprojfstf{B}{Q}}}
  \\
& \jdeq
  \jcomp{}{f}{\cprojfstf{B}{Q}}
  \\
\cprojsnd{\ctxwk{A}{B}}{\ctxwk{A}{Q}}{f}
& \jdeq
  \subst{f}{\ctxwk{A}{\cprojsndf{B}{Q}}}
  \\
& \jdeq
  \jcomp{}{f}{\cprojsndf{B}{Q}}
\end{align*}
The alternative formulation of the statement is a direct corollary.
\end{proof}

We also have the following lemma about the compatibility of pairing and composition:

\begin{lem}\label{lem:tmext-jcomp}
The following inference rule is valid
\begin{align*}
& \inference
  { \jhom{\Gamma}{A}{B}{f}
    \jhom{\Gamma}{B}{C}{g}
    \jfam{{\Gamma}{C}}{R}
    \jterm{{\Gamma}{B}}{\subst{g}{\ctxwk{B}{R}}}{w}
    }
  { \jhomeq
      {\Gamma}
      {A}
      {{C}{R}}
      {\jcomp{A}{f}{\tmext{\ctxwk{B}{C}}{\ctxwk{B}{R}}{g}{w}}}
      {\tmext{\ctxwk{A}{C}}{\ctxwk{A}{R}}{\jcomp{A}{f}{g}}{\jcomp{A}{f}{w}}}
    }
\end{align*}
\end{lem}

\begin{proof}
Let $\jhom{\Gamma}{A}{B}{f}$, $\jhom{\Gamma}{B}{C}{g}$, $\jfam{{\Gamma}{C}}{R}$
and $\jterm{{\Gamma}{B}}{\subst{g}{\ctxwk{B}{R}}}{w}$. Then we have the
judgmental equalities
\begin{align*}
\jcomp{A}{f}{\tmext{\ctxwk{B}{C}}{\ctxwk{B}{R}}{g}{w}}
& \jdeq 
  \subst{f}{\ctxwk{A}{\tmext{\ctxwk{B}{C}}{\ctxwk{B}{R}}{g}{w}}}
  \\
& \jdeq 
  \subst
    {f}
    {\tmext{\ctxwk{A}{{B}{C}}}{\ctxwk{A}{{B}{R}}}{\ctxwk{A}{g}}{\ctxwk{A}{w}}}
  \\
& \jdeq 
  \tmext
    {\ctxwk{A}{C}}
    {\ctxwk{A}{R}}
    {\subst{f}{\ctxwk{A}{g}}}
    {\subst{f}{\ctxwk{A}{w}}}
  \\
& \jdeq 
  \tmext{\ctxwk{A}{C}}{\ctxwk{A}{R}}{\jcomp{A}{f}{g}}{\jcomp{A}{f}{w}}.
  \qedhere
\end{align*}
\end{proof}

There is a notion of morphism \emph{over} a morphism. We will develop this
notion because it will be needed in the theory of models later on.

\begin{defn}
Let $\jhom{\Gamma}{A}{B}{f}$ be a morphism from $A$ to $B$ in context $\Gamma$
and consider $\jfam{{\Gamma}{A}}{P}$ and $\jfam{{\Gamma}{B}}{Q}$. We define the
judgment\begin{equation*}
\jfhom{\Gamma}{A}{B}{f}{P}{Q}{F},
\end{equation*}
which is pronounced as `$F$ is a morphism from $P$ to $Q$ over $f$ in context
$\Gamma$', to be the judgment\begin{equation*}
\unfold{\jfhom{\Gamma}{A}{B}{f}{P}{Q}{F}}.
\end{equation*}
\end{defn}

\begin{rmk}
The judgment $\jfhom{\Gamma}{A}{B}{f}{P}{Q}{F}$ means the same thing as
\begin{equation*}
\jhom{{\Gamma}{A}}{P}{\jcomp{A}{f}{Q}}{F}.
\end{equation*}
Thus we see that a morphism from $P$ to $Q$ over the identity term $\idtm{A}$ in
context $\Gamma$ is the same thing as a morphism from $P$ to $Q$ in context
$\ctxext{\Gamma}{A}$, i.e.~the following inference rules are valid:
\begin{align*}
& \inference
  { \jfam{{\Gamma}{A}}{P}
    \jfam{{\Gamma}{A}}{Q}
    \jfhom{\Gamma}{A}{A}{\idtm{A}}{P}{Q}{f}
    }
  { \jhom{{\Gamma}{A}}{P}{Q}{f}
    }
  \\
& \inference
  { \jfam{{\Gamma}{A}}{P}
    \jfam{{\Gamma}{A}}{Q}
    \jhom{{\Gamma}{A}}{P}{Q}{f}
    }
  { \jfhom{\Gamma}{A}{A}{\idtm{A}}{P}{Q}{f}
    }
\end{align*}
To see this, we only have to note that
$\ctxwk{P}{\subst{\idtm{A}}{\ctxwk{A}{Q}}}\jdeq\ctxwk{P}{Q}$, which
holds by \autoref{idfunc-wk-defn}.
\end{rmk}

Suppose we have morphisms $\jhom{\Gamma}{A}{B}{f}$ and $\jhom{\Gamma}{B}{C}{g}$
and that we have the morphisms $\jfhom{\Gamma}{A}{B}{f}{P}{Q}{F}$ and
$\jfhom{\Gamma}{B}{C}{g}{Q}{R}{G}$ over them. Then we have\begin{equation*}
\jhom
  {{\Gamma}{A}}
  {\jcomp{A}{f}{Q}}
  {\jcomp{A}{f}{{B}{g}{R}}}
  {\jcomp{A}{f}{G}}
\end{equation*}
Because we also have $\jhom{{\Gamma}{A}}{P}{\jcomp{A}{f}{Q}}{F}$, we have the
composition\begin{equation*}
\jhom
  {{\Gamma}{A}}
  {P}
  {\jcomp{A}{f}{{B}{g}{R}}}
  {\jcomp{P}{F}{\jcomp{A}{f}{G}}}.
\end{equation*}
Because of 
the judgmental equality $\jcomp{A}{f}{{B}{g}{R}}\jdeq
\jcomp{A}{{A}{f}{g}}{R}$, it follows that 
$\jcomp{P}{F}{\jcomp{A}{f}{G}}$ is a morphism from $P$ to $R$ over
$\jcomp{A}{f}{g}$. This could be considered as the composition of $G$ with $F$.
In the following definition, we formulate this more generally.

\begin{defn}
\emph{Horizontal composition of morphisms over morphisms} is defined by
\begin{align*}
& \inference
  { \jfhom{\Gamma}{A}{B}{f}{P}{Q}{F}
    \jfam{{{\Gamma}{B}}{Q}}{R}
    }
  { \jfamdefn
      {{{\Gamma}{A}}{P}}
      {\jfcomp{A}{f}{P}{F}{R}}
      {\unfold{\jfcomp{A}{f}{P}{F}{R}}}
    }
  \\
& \inference
  { \jfhom{\Gamma}{A}{B}{f}{P}{Q}{F}
    \jfam{{{{\Gamma}{B}}{Q}}{R}}{S}
    }
  { \jfamdefn
      {{{{\Gamma}{A}}{P}}{\jfcomp{A}{f}{P}{F}{R}}}
      {\jfcomp{A}{f}{P}{F}{S}}
      {\unfold{\jfcomp{A}{f}{P}{F}{S}}}
    }
  \\
& \inference
  { \jfhom{\Gamma}{A}{B}{f}{P}{Q}{F}
    \jterm{{{{\Gamma}{B}}{Q}}{R}}{S}{k}
    }
  { \jtermdefn
      {{{{\Gamma}{A}}{P}}{\jfcomp{A}{f}{P}{F}{R}}}
      {\jfcomp{A}{f}{P}{F}{S}}
      {\jfcomp{A}{f}{P}{F}{k}}
      {\unfold{\jfcomp{A}{f}{P}{F}{S}}}
    }
\end{align*}
\end{defn}

Since horizontal composition is just consecutive composition, we will get the
compatibility with the empty context, extension, weakening, substitution and
the identity terms for free. We will state these compatibility properities in
\autoref{lem:jfcomp-emp,lem:jfcomp-ext,lem:jfcomp-wk,lem:jfcomp-subst,lem:jfcomp-idtm},
but the proofs are left to the reader.

\begin{lem}\label{lem:jfcomp-emp}
Horizontal composition is compatible with the empty context:
\begin{align*}
& \inference
  { \jfhom{\Gamma}{A}{B}{f}{P}{Q}{F}
    }
  { \jfameq
      {{{\Gamma}{A}}{P}}
      {\jfcomp{A}{f}{P}{F}{\emptyf}}
      {\emptyf}
    }
  \\
& \inference
  { \jfhom{\Gamma}{A}{B}{f}{P}{Q}{F}
    \jfam{{{\Gamma}{B}}{Q}}{R}
    }
  { \jfameq
      {{{{\Gamma}{A}}{P}}{\jfcomp{A}{f}{P}{F}{R}}}
      {\jfcomp{A}{f}{P}{F}{\emptyf}}
      {\emptyf}
    }
\end{align*}
\end{lem}

\begin{lem}\label{lem:jfcomp-ext}
Horizontal composition is compatible with extension:
\begin{align*}
& \inference
  { \jfhom{\Gamma}{A}{B}{f}{P}{Q}{F}
    \jfam{{{{\Gamma}{B}}{Q}}{R}}{S}
    }
  { \jfameq
      {{{\Gamma}{A}}{P}}
      {\jfcomp{A}{f}{P}{F}{\ctxext{R}{S}}}
      {\ctxext{\jfcomp{A}{f}{P}{F}{R}}{\jfcomp{A}{f}{P}{F}{S}}}
    }
  \\
& \inference
  { \jfhom{\Gamma}{A}{B}{f}{P}{Q}{F}
    \jfam{{{{{\Gamma}{B}}{Q}}{R}}{S}}{T}
    }
  { \jfameq
      {{{{\Gamma}{A}}{P}}{\jfcomp{A}{f}{P}{F}{R}}}
      {\jfcomp{A}{f}{P}{F}{\ctxext{S}{T}}}
      {\ctxext{\jfcomp{A}{f}{P}{F}{S}}{\jfcomp{A}{f}{P}{F}{T}}}
    }
  \\
& \inference
  { \jfhom{\Gamma}{A}{B}{f}{P}{Q}{F}
    \jterm{{{{\Gamma}{B}}{Q}}{R}}{{S}{T}}{\tmext{k}{l}}
    }
  { \jtermdefn
      {{{{\Gamma}{A}}{P}}{\jfcomp{A}{f}{P}{F}{R}}}
      {\ctxext{\jfcomp{A}{f}{P}{F}{S}}{\jfcomp{A}{f}{P}{F}{T}}}
      {\jfcomp{A}{f}{P}{F}{\tmext{k}{l}}}
      {\tmext{\jfcomp{A}{f}{P}{F}{k}}{\jfcomp{A}{f}{P}{F}{l}}}
    }
\end{align*}
\end{lem}

\begin{lem}\label{lem:jfcomp-wk}
Horizontal composition is compatible with weakening:
\begin{align*}
& \inference
  { \jfhom{\Gamma}{A}{B}{f}{P}{Q}{F}
    \jfam{{{{\Gamma}{B}}{Q}}{R}}{S}
    \jfam{{{{\Gamma}{B}}{Q}}{R}}{T}
    }
  { \jfameq
      {{{{{\Gamma}{A}}{P}}{\jfcomp{A}{f}{P}{F}{R}}}{\jfcomp{A}{f}{P}{F}{S}}}
      {\jfcomp{A}{f}{P}{F}{\ctxwk{S}{T}}}
      {\ctxwk{\jfcomp{A}{f}{P}{F}{S}}{\jfcomp{A}{f}{P}{F}{T}}}
    }
  \\
& \inference
  { \jfhom{\Gamma}{A}{B}{f}{P}{Q}{F}
    \jfam{{{{\Gamma}{B}}{Q}}{R}}{S}
    \jfam{{{{{\Gamma}{B}}{Q}}{R}}{T}}{U}
    }
  { \jfameq
      {{{{{{\Gamma}{A}}{P}}
        {\jfcomp{A}{f}{P}{F}{R}}}
        {\jfcomp{A}{f}{P}{F}{S}}}
        {\ctxwk{\jfcomp{A}{f}{P}{F}{S}}{\jfcomp{A}{f}{P}{F}{T}}}}
      {\jfcomp{A}{f}{P}{F}{\ctxwk{S}{U}}}
      {\ctxwk{\jfcomp{A}{f}{P}{F}{S}}{\jfcomp{A}{f}{P}{F}{U}}}
    }
  \\
& \inference
  { \jfhom{\Gamma}{A}{B}{f}{P}{Q}{F}
    \jfam{{{{\Gamma}{B}}{Q}}{R}}{S}
    \jterm{{{{{\Gamma}{B}}{Q}}{R}}{T}}{U}{m}
    }
  { \jtermeq
      {{{{{{\Gamma}{A}}{P}}
        {\jfcomp{A}{f}{P}{F}{R}}}
        {\jfcomp{A}{f}{P}{F}{S}}}
        {\ctxwk{\jfcomp{A}{f}{P}{F}{S}}{\jfcomp{A}{f}{P}{F}{T}}}}
      {\jfcomp{A}{f}{P}{F}{\ctxwk{S}{U}}}
      {\jfcomp{A}{f}{P}{F}{\ctxwk{S}{m}}}
      {\ctxwk{\jfcomp{A}{f}{P}{F}{S}}{\jfcomp{A}{f}{P}{F}{m}}}
    }
\end{align*}
\end{lem}

\begin{lem}\label{lem:jfcomp-subst}
Horizontal composition is compatible with substitution:
\begin{align*}
& \inference
  { \jfhom{\Gamma}{A}{B}{f}{P}{Q}{F}
    \jterm{{{{\Gamma}{B}}{Q}}{R}}{S}{k}
    \jfam{{{{{\Gamma}{B}}{Q}}{R}}{S}}{T}
    }
  { \jfameq
      {{{{\Gamma}{A}}{P}}{\jfcomp{A}{f}{P}{F}{R}}}
      {\jfcomp{A}{f}{P}{F}{\subst{k}{T}}}
      {\subst{\jfcomp{A}{f}{P}{F}{k}}{\jfcomp{A}{f}{P}{F}{T}}}
    }
  \\
& \inference
  { \jfhom{\Gamma}{A}{B}{f}{P}{Q}{F}
    \jterm{{{{\Gamma}{B}}{Q}}{R}}{S}{k}
    \jfam{{{{{{\Gamma}{B}}{Q}}{R}}{S}}{T}}{U}
    }
  { \jfameq
      {{{{{\Gamma}{A}}{P}}{\jfcomp{A}{f}{P}{F}{R}}}{\jfcomp{A}{f}{P}{F}{\subst{k}{T}}}}
      {\jfcomp{A}{f}{P}{F}{\subst{k}{U}}}
      {\subst{\jfcomp{A}{f}{P}{F}{k}}{\jfcomp{A}{f}{P}{F}{U}}}
    }
  \\
& \inference
  { \jfhom{\Gamma}{A}{B}{f}{P}{Q}{F}
    \jterm{{{{\Gamma}{B}}{Q}}{R}}{S}{k}
    \jterm{{{{{{\Gamma}{B}}{Q}}{R}}{S}}{T}}{U}{m}
    }
  { \jfameq
      {{{{{\Gamma}{A}}{P}}{\jfcomp{A}{f}{P}{F}{R}}}{\jfcomp{A}{f}{P}{F}{\subst{k}{T}}}}
      {\jfcomp{A}{f}{P}{F}{\subst{k}{U}}}
      {\jfcomp{A}{f}{P}{F}{\subst{k}{m}}}
      {\subst{\jfcomp{A}{f}{P}{F}{k}}{\jfcomp{A}{f}{P}{F}{m}}}
    }
\end{align*}
\end{lem}

\begin{lem}\label{lem:jfcomp-idtm}
Horizontal composition is compatible with identity terms:
\begin{equation*}
\inference
  { \jfhom{\Gamma}{A}{B}{f}{P}{Q}{F}
    \jfam{{{{\Gamma}{B}}{Q}}{R}}{S}
    }
  { \jtermeq
      {{{{{\Gamma}{A}}{P}}{\jfcomp{A}{f}{P}{F}{R}}}{\jfcomp{A}{f}{P}{F}{S}}}
      {\ctxwk{{\jfcomp{A}{f}{P}{F}{S}}}{\jfcomp{A}{f}{P}{F}{S}}}
      {\jfcomp{A}{f}{P}{F}{\idtm{S}}}
      {\idtm{\jfcomp{A}{f}{P}{F}{S}}}
    }
\end{equation*}
\end{lem}

There is also a notion of vertical composition, although this is not an
operation the way horizontal composition is. Nevertheless, to analyze the
properties of horizontal composition it is useful to
consider vertical composition.

\begin{defn}
Let $\jhom{\Gamma}{A}{B}{f}$ be a morphism from $A$ to $B$ in context $\Gamma$
and let $\jfhom{\Gamma}{A}{B}{f}{P}{Q}{F}$ be a morphism over $f$ in context 
$\Gamma$. We define
\begin{equation*}
\jhomdefn{\Gamma}{{A}{P}}{{B}{Q}}{\jvcomp{P}{f}{F}}{\unfold{\jvcomp{P}{f}{F}}}
\end{equation*}
\end{defn}

\begin{lem}\label{lem:composition-threesome}
We have the following composition threesome:
\begin{align*}
& \inference
  { \jfhom{\Gamma}{A}{B}{f}{P}{Q}{F}
    \jfam{{{\Gamma}{B}}{Q}}{R}
    }
  { \jfameq
      {{{\Gamma}{A}}{P}}
      {\jfcomp{A}{f}{P}{F}{R}}
      {\jcomp{{A}{P}}{\jvcomp{P}{f}{F}}{R}}
    }
  \\
& \inference
  { \jfhom{\Gamma}{A}{B}{f}{P}{Q}{F}
    \jfam{{{{\Gamma}{B}}{Q}}{R}}{S}
    }
  { \jfameq
      {{{{\Gamma}{A}}{P}}{\jfcomp{A}{f}{P}{F}{R}}}
      {\jfcomp{A}{f}{P}{F}{S}}
      {\jcomp{{A}{P}}{\jvcomp{P}{f}{F}}{S}}
    }
  \\
& \inference
  { \jfhom{\Gamma}{A}{B}{f}{P}{Q}{F}
    \jterm{{{{\Gamma}{B}}{Q}}{R}}{S}{k}
    }
  { \jtermeq
      {{{{\Gamma}{A}}{P}}{\jfcomp{A}{f}{P}{F}{R}}}
      {\jfcomp{A}{f}{P}{F}{S}}
      {\jfcomp{A}{f}{P}{F}{k}}
      {\jcomp{{A}{P}}{\jvcomp{P}{f}{F}}{k}}
    }
\end{align*}
\end{lem}

\begin{proof}
Let $F$ be a morphism from $P$ to $Q$ over $f$ in context $\Gamma$ and let
$R$ be a family over $\ctxext{{\Gamma}{B}}{Q}$. Then we have
\begin{align*}
\jfcomp{A}{f}{P}{F}{R}
& \jdeq
  \unfold{\jfcomp{A}{f}{P}{F}{R}}
  \tag{by definition}
  \\
& \jdeq
  \subst{F}{{\ctxwk{P}{f}}{\ctxwk{P}{{A}{R}}}}
  \tag{by \autoref{comp-ws-f}}
  \\
& \jdeq
  \subst{F}{{\ctxwk{P}{f}}{\ctxwk{\ctxext{A}{P}}{R}}}
  \tag{by \autoref{comp-ew-f}}
  \\
& \jdeq
  \subst{\tmext{\ctxwk{P}{f}}{F}}{\ctxwk{\ctxext{A}{P}}{R}}
  \tag{by \autoref{comp-es}}
  \\
& \jdeq
  \jcomp{{A}{P}}{\jvcomp{P}{f}{F}}{R}.
  \tag{by definition}
\end{align*}
\end{proof}

\begin{lem}\label{lem:composition-interchange}
We have the following interchange law for composition:
\begin{equation*}
\inference
  { \jfhom{\Gamma}{A}{B}{f}{P}{Q}{F}
    \jfhom{\Gamma}{B}{C}{g}{Q}{R}{G}
    }
  { \jhomeq
      {\Gamma}
      {{A}{P}}
      {{C}{R}}
      {\jcomp{{A}{P}}{\jvcomp{P}{f}{F}}{\jvcomp{Q}{g}{G}}}
      {\jvcomp{P}{\jcomp{A}{f}{g}}{\jfcomp{A}{f}{P}{F}{G}}}
    }
\end{equation*}
\end{lem}

\begin{proof}
Consider $\jfhom{\Gamma}{A}{B}{f}{P}{Q}{F}$ and 
$\jfhom{\Gamma}{B}{C}{g}{Q}{R}{G}$. Then we have
\begin{align*}
\jcomp{{A}{P}}{\jvcomp{P}{f}{F}}{\jvcomp{Q}{g}{G}}
& \jdeq
  \jcomp{P}{F}{\jcomp{A}{f}{\tmext{}{}{\ctxwk{Q}{g}}{G}}}
  \tag{by \autoref{lem:composition-threesome}}
  \\
& \jdeq
  \tmext{}{}
    {\jcomp{P}{F}{\jcomp{A}{f}{\ctxwk{Q}{g}}}}
    {\jcomp{P}{F}{\jcomp{A}{f}{G}}}
  \tag{by \autoref{lem:tmext-jcomp}}
  \\
& \jdeq
  \tmext{}{}
    {\jcomp{P}{F}{\jcomp{A}{f}{\ctxwk{Q}{g}}}}
    {\jfcomp{A}{f}{P}{F}{G}}
  \tag{by definition}
  \\
& \jdeq
  \tmext{}{}
    {\jcomp{P}{F}{\ctxwk{\jcomp{A}{f}{Q}}{\jcomp{A}{f}{g}}}}
    {\jfcomp{A}{f}{P}{F}{G}}
  \tag{by \autoref{lem:jcomp-wk}}
  \\
& \jdeq
  \tmext{}{}
    {\subst{F}{\ctxwk{P}{\ctxwk{\jcomp{A}{f}{Q}}{\jcomp{A}{f}{g}}}}}
    {\jfcomp{A}{f}{P}{F}{G}}
  \tag{by definition}
  \\
& \jdeq
  \tmext{}{}
    {\subst{F}{\ctxwk{{P}{\jcomp{A}{f}{Q}}}{{P}{\jcomp{A}{f}{g}}}}}
    {\jfcomp{A}{f}{P}{F}{G}}
  \tag{by \autoref{comp-ww-t}}
  \\
& \jdeq
  \tmext{}{}
    {\ctxwk{P}{\jcomp{A}{f}{g}}}
    {\jfcomp{A}{f}{P}{F}{G}}
  \tag{by \autoref{cancellation-ws-t}}
  \\
& \jdeq
  \jvcomp{P}{\jcomp{A}{f}{g}}{\jfcomp{A}{f}{P}{F}{G}}
  \tag{by definition}
\end{align*}
\end{proof}

\begin{lem}
Horizontal composition of morphisms over morphisms is associative, i.e.~the
following inference rules are valid:
\begin{align*}
& \inference
  { \jfhom{\Gamma}{A}{B}{f}{P}{Q}{F}
    \jfhom{\Gamma}{B}{C}{g}{Q}{R}{G}
    \jfam{{{\Gamma}{C}}{R}}{S}
    }
  { \jfameq
      {{{\Gamma}{A}}{P}}
      {\jfcomp{A}{f}{P}{F}{\jfcomp{B}{g}{Q}{G}{S}}}
      {\jfcomp{A}{\jcomp{A}{f}{g}}{P}{\jfcomp{A}{f}{P}{F}{G}}{S}}
    }
  \\
& \inference
  { \jfhom{\Gamma}{A}{B}{f}{P}{Q}{F}
    \jfhom{\Gamma}{B}{C}{g}{Q}{R}{G}
    \jfam{{{{\Gamma}{C}}{R}}{S}}{T}
    }
  { \jfameq
      {{{{\Gamma}{A}}{P}}{\jfcomp{A}{f}{P}{F}{\jfcomp{B}{g}{Q}{G}{S}}}}
      {\jfcomp{A}{f}{P}{F}{\jfcomp{B}{g}{Q}{G}{T}}}
      {\jfcomp{A}{\jcomp{A}{f}{g}}{P}{\jfcomp{A}{f}{P}{F}{G}}{T}}
    }
  \\
& \inference
  { \jfhom{\Gamma}{A}{B}{f}{P}{Q}{F}
    \jfhom{\Gamma}{B}{C}{g}{Q}{R}{G}
    \jterm{{{{\Gamma}{C}}{R}}{S}}{T}{l}
    }
  { \jtermeq
      {{{{\Gamma}{A}}{P}}{\jfcomp{A}{f}{P}{F}{\jfcomp{B}{g}{Q}{G}{S}}}}
      {\jfcomp{A}{f}{P}{F}{\jfcomp{B}{g}{Q}{G}{T}}}
      {\jfcomp{A}{f}{P}{F}{\jfcomp{B}{g}{Q}{G}{l}}}
      {\jfcomp{A}{\jcomp{A}{f}{g}}{P}{\jfcomp{A}{f}{P}{F}{G}}{l}}
    }
\end{align*}
\end{lem}

\begin{proof}
We only prove the first inference rule. Let
$\jfhom{\Gamma}{A}{B}{f}{P}{Q}{F}$, $\jfhom{\Gamma}{B}{C}{g}{Q}{R}{G}$ and
$\jfam{{{\Gamma}{C}}{R}}{S}$. Then we have the judgmental equalities
\begin{align*}
\jfcomp{A}{f}{P}{F}{\jfcomp{B}{g}{Q}{G}{S}}
& \jdeq
  \jcomp{{A}{P}}{\jvcomp{P}{f}{F}}{\jcomp{{B}{Q}}{\jvcomp{Q}{g}{G}}{S}}
  \tag{by \autoref{lem:composition-threesome}}
  \\
& \jdeq
  \jcomp{{A}{P}}{\jcomp{{A}{P}}{\jvcomp{P}{f}{F}}{\jvcomp{Q}{g}{G}}}{S}
  \tag{by \autoref{lem:jcomp-jcomp}}
  \\
& \jdeq
  \jcomp{{A}{P}}{\jvcomp{P}{\jcomp{A}{f}{g}}{\jfcomp{A}{f}{P}{F}{G}}}{S}
  \tag{by \autoref{lem:composition-interchange}}
  \\
& \jdeq
  \jfcomp{A}{\jcomp{A}{f}{g}}{P}{\jfcomp{A}{f}{P}{F}{G}}{S}.
  \tag{by \autoref{lem:composition-threesome}}
\end{align*}
\end{proof}

\begin{lem}
The following inference rules explain how vertically and horizontally composed 
morphisms are evaluated:
\begin{align*}
& \inference
  { \jfhom{\Gamma}{A}{B}{f}{P}{Q}{F}
    \jterm{\Gamma}{A}{x}
    \jterm{\Gamma}{\subst{x}{P}}{u}
    }
  { \jtermeq
      {\Gamma}
      {{B}{Q}}
      {\subst{\tmext{}{}{x}{u}}{\jvcomp{P}{f}{F}}}
      {\tmext{}{}{\subst{x}{f}}{\subst{u}{{x}{F}}}}
    }
\end{align*}
\end{lem}

\begin{proof}
Let $\jfhom{\Gamma}{A}{B}{f}{P}{Q}{F}$, $\jterm{\Gamma}{A}{x}$ and
$\jterm{\Gamma}{\subst{x}{P}}{u}$. Then we have the judgmental equalities
\begin{align*}
\subst{\tmext{}{}{x}{u}}{\jvcomp{P}{f}{F}}
& \jdeq
  \subst{\tmext{}{}{x}{u}}{\tmext{}{}{\ctxwk{P}{f}}{F}}
  \tag{by definition}
  \\
& \jdeq
  \tmext{\subst{\tmext{x}{u}}{\ctxwk{P}{f}}}{\subst{\tmext{x}{u}}{F}}
  \tag{by \autoref{lem:tmext-subst}}
  \\
& \jdeq
  \tmext{\subst{u}{{x}{\ctxwk{P}{f}}}}{\subst{u}{{x}{F}}}
  \tag{by \autoref{comp-es}}
  \\
& \jdeq
  \tmext{\subst{u}{\ctxwk{\subst{x}{P}}{\subst{x}{f}}}}{\subst{u}{{x}{F}}}
  \tag{by \autoref{comp-sw-t}}
  \\
& \jdeq
  \tmext{\subst{x}{f}}{\subst{u}{{x}{F}}}.
  \tag{by \autoref{cancellation-ws-t}}
\end{align*}
\end{proof}

\label{pullback}
Now that we have introduced the notions of morphisms and composition,
we can develop a diagramatic style of of displaying type dependencies
combined with morphisms. We give an informal, metatheoretical definition of
such diagrams by indicating what the various components mean. The definition
is informal because we will only use such diagrams occasionally to provide a
graphical indication of the situation in which we're working. In particular,
we will not shy away from using natural numbers and trust that the reader can
figure out what we mean.

\begin{defn}
A diagram is said to be a \emph{dependency diagram in context $\Gamma$}
if it is built up according to the following steps:
\begin{itemize}
\item The arrows appearing in a dependency diagram are either ordinary, like the
arrow%
$\begin{tikzcd}[ampersand replacement = \&]
X \ar{r} \& Y,
\end{tikzcd}$
or double-headed, like
$\begin{tikzcd}[ampersand replacement = \&]
X \ar[fib]{r} \& Y.
\end{tikzcd}$
\item An ordinary arrow 
\begin{equation*}
\begin{tikzcd}
A \ar{r}{f} & B
\end{tikzcd}
\end{equation*}
between two families $A$ and $B$ of contexts over $\Gamma$ indicates that
$f$ is a morphism from $A$ to $B$ in context $\Gamma$, i.e.~that we have the
judgment $\jhom{\Gamma}{A}{B}{f}$.
\item The set of double-headed arrows must form a forest and the root of
each maximal tree of double-headed arrows is a family of contexts over $\Gamma$.
In particular, if an object is not the domain of a double-headed arrow it must
be a family of contexts over $\Gamma$.
\item A sequence of double-headed 
arrows
\begin{equation*}
\begin{tikzcd}
P_{n} \ar[fib]{r} & \cdots \ar[fib]{r} & P_1 \ar[fib]{r} & A
\end{tikzcd}
\end{equation*}
indicates that $P_1$ is a family of contexts over $\ctxext{\Gamma}{A}$, that
$P_2$ is a family of contexts over $\ctxext{{\Gamma}{A}}{P_1}$, etcetera.
\item There can be two kinds of ladders of double-headed arrows:
\begin{equation*}
\begin{tikzcd}
P_{n} \ar{r}{F_{n}} \ar[fib]{d} & Q_{n} \ar[fib]{d}\\
\vdots \ar[fib]{d} & \vdots \ar[fib]{d}\\
P_1 \ar{r}{F_1} \ar[fib]{d} & Q_1 \ar[fib]{d}\\
A \ar{r}{f} & B
\end{tikzcd}
\qquad
\begin{tikzcd}[column sep = tiny]
P_{n+m} \ar{rr}{F_{n+m}} \ar[fib]{d} & & Q_{n+m} \ar[fib]{d}\\
\vdots \ar[fib]{d} & & \vdots \ar[fib]{d}\\
P_{n+1} \ar{rr}{F_{n+1}} \ar[fib]{dr} & & Q_{n+1} \ar[fib]{dl}\\
& P_n \ar[fib]{d}\\
& \vdots \ar[fib]{d}\\
& P_1 \ar[fib]{d}\\
& A
\end{tikzcd}
\end{equation*}
The ladder on the left 
indicates that $F_1$ is a morphism from $P_1$ to $Q_1$ \emph{over} $f$,
i.e.~that the judgment $\jfhom{\Gamma}{A}{B}{f}{P_1}{Q_1}{F_1}$ holds, that
$F_2$ is a morphism from $P_2$ to $Q_2$ over
the morphism $\tmext{\ctxwk{P_1}{f}}{F_1}$ from $\ctxext{A}{P_1}$ to
$\ctxext{B}{Q_1}$, etcetera.

The ladder on the right indicates that $F_{n+1}$ is a morphism from $P_{n+1}$ to
$Q_{n+1}$ in the appropriate context, that $F_{n+2}$ is a morphism from
$P_{n+2}$ to $Q_{n+2}$ over $F_{n+1}$, etcetera.
 
Note that the object(s) at the bottom of a ladder are always families of contexts
over $\Gamma$, so that the typing of the various ingredients makes sense.
\end{itemize}
Such a diagram is said to be commutative if the subdiagram consisting of only
the normal headed arrows is commutative in the usual sense (using judgmental
equality). Note that the ladders are inherently commutative.
\end{defn}

The most basic illustrative example of a commutative dependency diagram is
the diagram
\begin{equation*}
\begin{tikzcd}
P \ar[fib]{d} \ar{r}{F} & Q \ar[fib]{d} \\
A \ar{r}{f} & B
\end{tikzcd}
\end{equation*}
indicating a morphism $F$ from $P$ to $Q$ over the morphism $f$ from $A$ to
$B$ in a context $\Gamma$.

We can just copy the usual categorical definition of a pullback square to our
current situation, but we have to require that each arrow in the pullback square
is an ordinary arrow. When families (i.e. double-headed arrows) are involved
in the diagram, we make the following definition of a family pullback:

\begin{defn}
We say that a commutative dependency diagram of the form
\begin{equation*}
\begin{tikzcd}
P \ar[fib]{d} \ar{r}{F} & Q \ar[fib]{d} \\
A \ar{r}{f} & B
\end{tikzcd}
\end{equation*}
is a \emph{family pullback} if the following inference rules are valid:
\begin{align*}
& \inference
  { \jfam{{\Gamma}{A}}{P'}
    \jfhom{\Gamma}{A}{B}{f}{P'}{Q}{F'}
    }
  { \jhom{{\Gamma}{A}}{P'}{P}{u}
    }
  \\
& \inference
  { \jfam{{\Gamma}{A}}{P'}
    \jfhom{\Gamma}{A}{B}{f}{P'}{Q}{F'}
    }
  { \jfhomeq{\Gamma}{A}{B}{f}{P'}{Q}{\jcomp{}{u}{F}}{F'}
    }
  \\
& \inference
  { \jhom{{\Gamma}{A}}{P'}{P}{v}
    \jfhomeq{\Gamma}{A}{B}{f}{P'}{Q}{\jcomp{}{v}{F}}{F'}
    }
  { \jhomeq{{\Gamma}{A}}{P'}{P}{v}{u}
    }
\end{align*}
\end{defn}

The following lemma explains that when a square involving families is a
family pullback square whenever the corresponding square involving projections is a
pullback square. There is no proof in the the opposite direction.

\begin{lem}
A square
\begin{equation}\label{eq:fpb_to_pb_eqv_fpb}
\begin{tikzcd}
P \ar[fib]{d} \ar{r}{F} & Q \ar[fib]{d} \\
A \ar{r}{f} & B
\end{tikzcd}
\end{equation}
is a family pullback square whenever the square
\begin{equation}\label{eq:fpb_to_pb_eqv_pb}
\begin{tikzcd}[column sep = large]
\ctxext{A}{P} \ar{d}[swap]{\cprojfstf{A}{P}} \ar{r}{\tmext{\ctxwk{P}{f}}{F}} & \ctxext{B}{Q} \ar{d}{\cprojfstf{B}{Q}} \\
A \ar{r}{f} & B
\end{tikzcd}
\end{equation}
is a pullback square.
\end{lem}

The family pullback of a family along any morphism always exists. It is simply given
by the precomposition of the family with the morphism. Note that this fact does
not carry over to arbitrary pullbacks.

\begin{lem}
The diagram
\begin{equation*}
\begin{tikzcd}
\jcomp{}{f}{Q} \ar[fib]{d} \ar{r}{\idtm{\jcomp{}{f}{Q}}} & Q \ar[fib]{d} \\
A \ar{r}{f} & B
\end{tikzcd}
\end{equation*}
is a family pullback diagram.
\end{lem}

\begin{proof}
The proof is a triviality because $\jhom{{\Gamma}{A}}{P'}{\jcomp{}{f}{Q}}{F'}$
is the same judgment as $\jfhom{\Gamma}{A}{B}{f}{P'}{Q}{F'}$ and
$\jcomp{}{\idtm{\jcomp{}{f}{Q}}}{F'}\jdeq F'$.
\end{proof}

For arbitrary pullbacks we have the pasting lemma as usual, but for family
pullbacks we can only derive one of the parts of the pasting lemma.

\begin{lem}
Suppose we have the diagram
\begin{equation*}
\begin{tikzcd}
P \ar{r}{F} \ar[fib]{d} & Q \ar{r}{G} \ar[fib]{d} & R \ar[fib]{d}\\
A \ar{r}{f} & B \ar{r}{g} & C
\end{tikzcd}
\end{equation*}
where the square on the right and the outer rectangle are family pullback 
diagrams. Then the square on the left is a family pullback diagram.
\end{lem}

\begin{proof}
Let $\jfam{{\Gamma}{A}}{P'}$ be a family and let $\jfhom{\Gamma}{A}{B}{f}
{P'}{Q}{F}$ be a morphism over $f$.
\begin{itemize}
\item Then we compose $F'$ with $G$ to obtain a morphism over $\jcomp{}{f}{g}$.
\item Then we get $\jhom{{\Gamma}{A}}{P'}{P}{u}$ with a uniqueness property.
      The property that $\jcomp{}{u}{F}\jdeq F'$ follows from the assumption
      that the right square is a pullback.
\item Now assume that we have another such $v$. Compose it with $F$ and $G$.
      By the assumed properties this is the same as $u$ composed with $F$ and
      $G$. By the pullback condition we now get $u\jdeq v$. 
\end{itemize}
\end{proof}

\begin{lem}
For any $\jterm{\Gamma}{A}{x}$ and any $\jfam{{\Gamma}{A}}{P}$, the square
\begin{equation*}
\begin{tikzcd}
\subst{x}{P} \ar{d} \ar{r}{\finc{x}{P}} & \ctxext{A}{P} \ar{d}{\cprojfstf{A}{P}}\\
\emptyf \ar{r}{\ctxwk{\emptyf}{x}} & A
\end{tikzcd}
\end{equation*}
is a pullback square.
\end{lem}

In the following lemma we assert that pulling back a family $Q$ 
over $\ctxext{{\Gamma}{A}}{P}$ along a fiber
inclusion $\finc{x}{P}$ gives the family $\subst{x}{Q}$ over $\ctxext{\Gamma}{\subst{x}{P}}$. 

\begin{lem}
We have the following inference rule:
\begin{equation*}
\inference
  { \jfam{{{\Gamma}{A}}{P}}{Q}
    \jterm{\Gamma}{A}{x}
    }
  { \jfameq
      {{\Gamma}{\subst{x}{P}}}
      {\jcomp{}{\finc{x}{P}}{Q}}
      {\subst{x}{Q}}
    }
\end{equation*}
\end{lem}

\begin{proof}
We have the judgmental equalities:
\begin{align*}
\jcomp{\subst{x}{P}}{\finc{x}{P}}{Q}
& \jdeq
  \subst{\tmext{\ctxwk{\subst{x}{P}}{x}}{\idtm{\subst{x}{P}}}}{\ctxwk{\subst{x}{P}}{Q}}
  \\
& \jdeq
  \subst{\idtm{\subst{x}{P}}}{{\ctxwk{\subst{x}{P}}{x}}{\ctxwk{\subst{x}{P}}{Q}}}
  \\
& \jdeq
  \subst{\idtm{\subst{x}{P}}}{\ctxwk{\subst{x}{P}}{\subst{x}{Q}}}
  \\
& \jdeq
  \subst{x}{Q}.
\end{align*}
\end{proof}

\begin{lem}
The following inference rule is valid:
\begin{equation*}
\inference
  { \jfam{{\Gamma}{A}}{P}
    \jfam{{\Gamma}{A}}{Q}
    }
  { \jfameq
      {{\Gamma}{{A}{P}}}
      {\jcomp{{A}{P}}{\cprojfstf{A}{P}}{Q}}
      {\ctxwk{P}{Q}}
    }
\end{equation*}
\end{lem}

\begin{proof}
Let $\jfam{{\Gamma}{A}}{P}$ and $\jfam{{\Gamma}{A}}{Q}$ be
families. Then we have
\begin{align*}
\jcomp{{A}{P}}{\cprojfstf{A}{P}}{Q}
& \jdeq
  \unfoldall{\jcomp{{A}{P}}{\cprojfstf{A}{P}}{Q}}
  \tag{by definition}\\
& \jdeq 
  \subst{\ctxwk{P}{\idtm{A}}}{\ctxwk{P}{{A}{Q}}} 
  \tag{by \autoref{comp-ww-f}}\\
& \jdeq 
  \ctxwk{P}{\subst{\idtm{A}}{\ctxwk{A}{Q}}} 
  \tag{by \autoref{comp-ws-f}}\\
& \jdeq 
  \ctxwk{P}{Q} 
  \tag{by \autoref{idfunc-wk-defn}}
\end{align*}
\end{proof}


\subsection{Fiber inclusions}
We will use the insights of \autoref{extension-on-terms} to define and study
\emph{fiber inclusions}. The fiber inclusion of the \emph{fiber}
$\subst{x}{P}$ into the extension $\ctxext{A}{P}$ is a morphism
$\jhom{\Gamma}{\subst{x}{P}}{{A}{P}}{\finc{x}{P}}$, for any family
$\jfam{{\Gamma}{A}}{P}$ and any term $\jterm{\Gamma}{A}{x}$. Then we will determine
the ways in which it is compatible with the other operators. Note that in this
subsection we will focus on the compatibility properties; the fact that
the fiber inclusions also appear in a pullback diagram will be established in
\autoref{pullback}. 

\begin{defn}
Let $\jterm{\Gamma}{A}{x}$ be a term and let $\jfam{{\Gamma}{A}}{P}$ be a
family. Then we define the \emph{fiber inclusion} of $\subst{x}{P}$ into
$\ctxext{A}{P}$ in context $\Gamma$ to be the morphism
\begin{equation*}
\jhomdefn{\Gamma}{\subst{x}{P}}{{A}{P}}{\finc{x}{P}}{\unfoldnext{\finc{x}{P}}}.
\end{equation*}
\end{defn}

We can immediately show that composing with a fiber inclusion is substitution.
Thus, the following lemma asserts that we have a family pullback square
\begin{equation*}
\begin{tikzcd}
\subst{x}{Q} \ar[fib]{d} \ar{r} & Q \ar[fib]{d} \\
\subst{x}{P} \ar{r}[swap]{\finc{x}{P}} & \ctxext{A}{P}
\end{tikzcd}
\end{equation*}
in context $\Gamma$.
The upper arrow is just the identity morphism over $\finc{x}{P}$.

\begin{lem}\label{lem:finc-precomp}
Composition with $\finc{x}{P}$ is (the action on families of)
substitution with $x$:
\begin{align*}
& \inference
  { \jterm{\Gamma}{A}{x}
    \jfam{{{\Gamma}{A}}{P}}{Q}
    }
  { \jfameq
      {{\Gamma}{\subst{x}{P}}}
      {\jcomp{\subst{x}{P}}{\finc{x}{P}}{Q}}
      {\subst{x}{Q}}
    }
  \\
& \inference
  { \jterm{\Gamma}{A}{x}
    \jfam{{{{\Gamma}{A}}{P}}{Q}}{R}
    }
  { \jfameq
      {{{\Gamma}{\subst{x}{P}}}{\subst{x}{Q}}}
      {\jcomp{\subst{x}{P}}{\finc{x}{P}}{R}}
      {\subst{x}{R}}
    }
  \\
& \inference
  { \jterm{\Gamma}{A}{x}
    \jterm{{{{\Gamma}{A}}{P}}{Q}}{R}{h}
    }
  { \jtermeq
      {{{\Gamma}{\subst{x}{P}}}{\subst{x}{Q}}}
      {\subst{x}{R}}
      {\jcomp{\subst{x}{P}}{\finc{x}{P}}{h}}
      {\subst{x}{h}}
    }
\end{align*}
\end{lem}

\begin{proof}
We only prove the validity of the first inference rule. Let $\jterm{\Gamma}{A}{x}$
be a term and let $\jfam{{{\Gamma}{A}}{P}}{Q}$ be a family. Then we have the
judgmental equalities
\begin{align*}
\jcomp{\subst{x}{P}}{\finc{x}{P}}{Q}
& \jdeq
  \unfold{\jcomp{\subst{x}{P}}{\unfold{\finc{x}{P}}}{Q}}
  \tag{by definition}
  \\
& \jdeq
  \subst
    {\idtm{\subst{x}{P}}}
    {{\ctxwk{\subst{x}{P}}{x}}{\ctxwk{\subst{x}{P}}{Q}}}
  \tag{by \autoref{comp-es}}
  \\
& \jdeq
  \subst
    {\idtm{\subst{x}{P}}}
    {\ctxwk{\subst{x}{P}}{\subst{x}{Q}}}
  \tag{by \autoref{comp-ws-f}}
  \\
& \jdeq
  \subst{x}{Q}.
  \tag{by \autoref{idfunc-precomp}}
\end{align*}
\end{proof}

We can give a second characterization of fiber inclusions:

\begin{lem}\label{lem:finc-char2}
We have the following inference rule
\begin{equation*}
\inference
  { \jterm{\Gamma}{A}{x}
    \jfam{{\Gamma}{A}}{P}
    }
  { \jhomeq{\Gamma}{\subst{x}{P}}{{A}{P}}{\finc{x}{P}}{\subst{x}{\idtm{{A}{P}}}}
    } 
\end{equation*}
\end{lem}

\begin{proof}
The proof is the following calculation:
\begin{align*}
\finc{x}{P}
& \jdeq 
  \tmext{\ctxwk{\subst{x}{P}}{x}}{\idtm{\subst{x}{P}}}
  \tag{by definition}
  \\
& \jdeq
  \subst
    { \idtm{\subst{x}{P}}
      }
    { { \ctxwk{\subst{x}{P}}{x}
        }
      { \idtm{\ctxext{\ctxwk{\subst{x}{P}}{A}}{\ctxwk{\subst{x}{P}}{P}}}
        }
      }
  \tag{by definition}
  \\
& \jdeq
  \subst
    { \idtm{\subst{x}{P}}
      }
    { { \ctxwk{\subst{x}{P}}{x}
        }
      { \idtm{\ctxwk{\subst{x}{P}}{\ctxext{A}{P}}}
        }
      }
  \tag{by \autoref{comp-we-f}}
  \\
& \jdeq
  \subst
    { \idtm{\subst{x}{P}}
      }
    { { \ctxwk{\subst{x}{P}}{x}
        }
      { \ctxwk{\subst{x}{P}}{\idtm{\ctxext{A}{P}}}
        }
      }
  \tag{by \autoref{comp-wi-t}}
  \\
& \jdeq
  \subst
    { \idtm{\subst{x}{P}}
      }
    { \ctxwk
        { \subst{x}{P}
          }
        { \subst{x}{\idtm{\ctxext{A}{P}}}
          }
      }
  \tag{by \autoref{comp-ws-t}}
  \\
& \jdeq
  \subst{x}{\idtm{\ctxext{A}{P}}}.
  \tag{by \autoref{idfunc-precomp}}
\end{align*}
\end{proof}

\begin{cor}
The following inference rule is valid:
\begin{equation*}
\inference
  { \jterm{\Gamma}{A}{x}
    \jterm{\Gamma}{\subst{x}{P}}{u}
    }
  { \jtermeq{\Gamma}{{A}{P}}{\subst{u}{\finc{x}{P}}}{\tmext{x}{u}}
    }
\end{equation*}
\end{cor}

We have the following lemmas expressing the compatibility of the fiber
inclusions with the empty context, extension, weakening and substitution. 

\begin{lem}
The fiber inclusions are compatible with the empty families; i.e.~the following
inference rules are valid
\begin{align*}
& \inference
  { \jterm{\Gamma}{A}{x}
    }
  { \jtermeq
      {\Gamma}
      {A}
      {\finc{x}{\emptyf}}
      {x}
    }
  \\
& \inference
  { \jfam{\Gamma}{A}
    }
  { \jhomeq
      {\Gamma}
      {A}
      {A}
      {\finc{\emptytm}{A}}
      {\idtm{A}}
    }
\end{align*}
\end{lem}

\begin{proof}
Let $\jterm{\Gamma}{A}{x}$. We have the judgmental equalities
\begin{align*}
\finc{x}{\emptyf}
& \jdeq
  \unfold{\finc{x}{\emptyf}}
  \tag{by definition}
  \\
& \jdeq
  \tmext{\ctxwk{\emptyf}{x}}{\idtm{\emptyf}}
  \tag{by \autoref{comp-s0-c}}
  \\
& \jdeq
  \tmext{x}{\idtm{\emptyf}}
  \tag{by \autoref{comp-0w-t}}
  \\
& \jdeq
  \tmext{x}{\emptytm}
  \tag{by \autoref{comp-00-t}}
  \\
& \jdeq
  x
  \tag{by \autoref{lem:tmext-emp}}
\end{align*}
to prove the first judgmental equality. For the second, we have
\begin{align*}
\finc{\emptytm}{A}
& \jdeq
  \unfold{\finc{\emptytm}{A}}
  \tag{by definition}
  \\
& \jdeq
  \tmext{\emptytm}{\idtm{\subst{\emptytm}{A}}}
  \tag{by \autoref{comp-00-t}}
  \\
& \jdeq
  \idtm{\subst{\emptytm}{A}}
  \tag{by \autoref{lem:tmext-emp}}
  \\
& \jdeq
  \idtm{A}
  \tag{by \autoref{comp-0s-f}}.
\end{align*}
\end{proof}

\begin{lem}
The fiber inclusions are compatible with extension; i.e.~the following inference
rule is valid
\begin{equation*}
\inference
  { \jterm{\Gamma}{A}{x}
    \jterm{\Gamma}{\subst{x}{P}}{u}
    \jfam{{{\Gamma}{A}}{P}}{Q}
    }
  { \jhomeq
      {\Gamma}
      {\subst{\tmext{x}{u}}{Q}}
      {{{A}{P}}{Q}}
      {\finc{\tmext{x}{u}}{Q}}
      {\jcomp{}{\finc{u}{\subst{x}{Q}}}{\finc{x}{\ctxext{P}{Q}}}}
    }
\end{equation*}
\end{lem}

\begin{proof}
Let $\jterm{\Gamma}{A}{x}$, $\jterm{\Gamma}{\subst{x}{P}}{u}$ and
$\jfam{{{\Gamma}{A}}{P}}{Q}$. Then we have the judgmental equalities
\begin{align*}
\finc{\tmext{x}{u}}{Q}
& \jdeq
  \subst{\tmext{}{}{x}{u}}{\idtm{{{A}{P}}{Q}}}
  \tag{by \autoref{lem:finc-char2}}
  \\
& \jdeq
  \subst{u}{{x}{\idtm{{{A}{P}}{Q}}}}
  \tag{by \autoref{comp-es}}
  \\
& \jdeq
  \subst{u}{{x}{\idtm{{A}{{P}{Q}}}}}
  \tag{by \autoref{comp-ee-c}}
  \\
& \jdeq
  \subst{u}{\finc{x}{\ctxext{P}{Q}}}
  \tag{by \autoref{lem:finc-char2}}
  \\
& \jdeq
  \jcomp{}{\finc{u}{\subst{x}{Q}}}{\finc{x}{\ctxext{P}{Q}}}
  \tag{by \autoref{lem:finc-precomp}}
\end{align*}
\end{proof}

\begin{lem}
The fiber inclusions are compatible with weakening; i.e.~the following inference
rule is valid
\begin{equation*}
\inference
  { \jfam{\Gamma}{A}
    \jterm{\Gamma}{B}{y}
    \jfam{{\Gamma}{B}}{Q}
    }
  { \jhomeq
      {{\Gamma}{A}}
      {\subst{\ctxwk{A}{y}}{\ctxwk{A}{Q}}}
      {{\ctxwk{A}{B}}{\ctxwk{A}{Q}}}
      {\finc{\ctxwk{A}{y}}{\ctxwk{A}{Q}}}
      {\ctxwk{A}{\finc{y}{Q}}}
    }
\end{equation*}
\end{lem}

\begin{proof}
Let $\jterm{\Gamma}{B}{y}$ and $\jfam{{\Gamma}{B}}{Q}$. Then we have the
judgmental equalities
\begin{align*}
\finc{\ctxwk{A}{y}}{\ctxwk{A}{Q}}
& \jdeq
  \subst{\ctxwk{A}{y}}{\idtm{\ctxext{\ctxwk{A}{B}}{\ctxwk{A}{Q}}}}
  \tag{by \autoref{lem:finc-char2}}
  \\
& \jdeq
  \subst{\ctxwk{A}{y}}{\idtm{\ctxwk{A}{\ctxext{B}{Q}}}}
  \tag{by \autoref{comp-we-f}}
  \\
& \jdeq
  \subst{\ctxwk{A}{y}}{\ctxwk{A}{\idtm{\ctxext{B}{Q}}}}
  \tag{by \autoref{comp-wi-t}}
  \\
& \jdeq
  \ctxwk{A}{\subst{y}{\idtm{{B}{Q}}}}
  \tag{by \autoref{comp-ws-t}}
  \\
& \jdeq
  \ctxwk{A}{\finc{y}{Q}}.
  \tag{by \autoref{lem:finc-char2}}
\end{align*}
\end{proof}

\begin{lem}
The fiber inclusions are compatible with substitution; i.e.~the following
inference rule is valid
\begin{equation*}
\inference
  { \jterm{\Gamma}{A}{x}
    \jfam{{{\Gamma}{A}}{P}}{Q}
    \jterm{{\Gamma}{A}}{P}{f}
    }
  { \jhomeq
      {\Gamma}
      {\subst{{x}{f}}{{x}{Q}}}
      {{\subst{x}{P}}{\subst{x}{Q}}}
      {\finc{\subst{x}{f}}{\subst{x}{Q}}}
      {\subst{x}{\finc{f}{Q}}}
    }
\end{equation*}
\end{lem}

\begin{proof}
Let $\jterm{\Gamma}{A}{x}$, $\jterm{{\Gamma}{A}}{P}{f}$ and 
$\jfam{{{\Gamma}{A}}{P}}{Q}$. Then we have the judgmental equalities
\begin{align*}
\finc{\subst{x}{f}}{\subst{x}{Q}}
& \jdeq
  \subst{x}{{f}{\idtm{{P}{Q}}}}
  \tag{by \autoref{lem:finc-char2}}
  \\
& \jdeq
  \subst{{x}{f}}{{x}{\idtm{{P}{Q}}}}
  \tag{by \autoref{comp-ss-t}}
  \\
& \jdeq
  \subst{{x}{f}}{\idtm{\subst{x}{\ctxext{P}{Q}}}}
  \tag{by \autoref{comp-si-t}}
  \\
& \jdeq
  \subst{{x}{f}}{\idtm{\ctxext{\subst{x}{P}}{\subst{x}{Q}}}}
  \tag{by \autoref{comp-se-f}}
  \\
& \jdeq
  \finc{\subst{x}{f}}{\subst{x}{Q}}.
  \tag{by \autoref{lem:finc-char2}}
\end{align*}
\end{proof}

\begin{lem}
The fiber inclusions are compatible with identity terms; i.e.~the following
inference rule is valid
\begin{equation*}
\inference
  { \jfam{{\Gamma}{A}}{P}
    }
  { \jhomeq
      {{\Gamma}{A}}
      {P}
      {\ctxwk{A}{\ctxext{A}{P}}}
      {\finc{\idtm{A}}{\ctxwk{A}{P}}}
      {\idtm{{A}{P}}}
    }
\end{equation*}
\end{lem}

\begin{proof}
Let $\jfam{\Gamma}{A}$. Then we have the judgmental equalities
\begin{align*}
\finc{\idtm{A}}{\ctxwk{A}{P}}
& \jdeq
  \subst{\idtm{A}}{\idtm{\ctxext{\ctxwk{A}{A}}{\ctxwk{A}{P}}}}
  \tag{by \autoref{lem:finc-char2}}
  \\
& \jdeq
  \subst{\idtm{A}}{\idtm{\ctxwk{A}{\ctxext{A}{P}}}}
  \tag{by \autoref{comp-we-f}}
  \\
& \jdeq
  \subst{\idtm{A}}{\ctxwk{A}{\idtm{{A}{P}}}}
  \tag{by \autoref{comp-wi-t}}
  \\
& \jdeq
  \idtm{{A}{P}}
  \tag{by \autoref{idfunc-precomp}}
\end{align*}
\end{proof}

\begin{comment}
\subsection{Another special case of projections}
In this subsection we investigate the special case of a projection which
appears as a morphism from $\ctxext{{A}{P}}{\ctxwk{P}{Q}}$ to $\ctxext{A}{Q}$
in context $\Gamma$, where we assume to have the families 
$\jfam{{\Gamma}{A}}{P}$ and $\jfam{{\Gamma}{A}}{Q}$. 

Note that we have the judgmental
equalities
\begin{align*}
\ctxwk{\ctxext{{A}{P}}{\ctxwk{P}{\mfam{A}}}}{\ctxext{A}{\mfam{A}}}
& \jdeq 
  \ctxext
    {\ctxwk{\ctxext{{A}{P}}{\ctxwk{P}{\mfam{A}}}}{A}}
    {\ctxwk{\ctxext{{A}{P}}{\ctxwk{P}{\mfam{A}}}}{\mfam{A}}}
  \\
& \jdeq
  \ctxext
    {\ctxwk{{P}{\mfam{A}}}{{\ctxext{A}{P}}{A}}}
    {\ctxwk{\ctxext{{A}{P}}{\ctxwk{P}{\mfam{A}}}}{\mfam{A}}}
\end{align*}
Note that we have the term $\ctxwk{{P}{\mfam{A}}}{\cprojfstf{A}{P}}$ of the
family $\ctxwk{{P}{\mfam{A}}}{{\ctxext{A}{P}}{A}}$. Therefore, we need to
find a term of type $\subst{\ctxwk{{P}{\mfam{A}}}{\cprojfstf{A}{P}}}
{\ctxwk{\ctxext{{A}{P}}{\ctxwk{P}{\mfam{A}}}}{\mfam{A}}}$. Note that we have
the judgmental equalities:
\begin{align*}
\subst
  {\ctxwk{{P}{\mfam{A}}}{\cprojfstf{A}{P}}}
  {\ctxwk{\ctxext{{A}{P}}{\ctxwk{P}{\mfam{A}}}}{\mfam{A}}}
& \jdeq
  \subst
    {\ctxwk{{P}{\mfam{A}}}{\cprojfstf{A}{P}}}
    {\ctxwk{{P}{\mfam{A}}}{{\ctxext{A}{P}}{\mfam{A}}}}
  \\
& \jdeq
  \ctxwk
    { {P}{\mfam{A}}
      }
    { \subst
        {\cprojfstf{A}{P}}
        {\ctxwk{\ctxext{A}{P}}{\mfam{A}}}
      }
  \\
& \jdeq
  \ctxwk
    { {P}{\mfam{A}}
      }
    { {P}{\mfam{A}}
      }
  \\
& \jdeq
  \ctxwk{P}{{\mfam{A}}{\mfam{A}}}
\end{align*}
We find the term $\ctxwk{P}{\idtm{\mfam{A}}}$ here. Thus we can now define
$\bar{\typefont{pr}}$ by:
\begin{equation}\label{barproj}
\jhomdefn
  {\Gamma}
  {{{A}{P}}{\mfam{A}}}
  {{A}{\mfam{A}}}
  {\bar{\typefont{pr}}}
  {\tmext{\ctxwk{{P}{\mfam{A}}}{\cprojfstf{A}{P}}}{\ctxwk{P}{\idtm{\mfam{A}}}}}
\end{equation}

\begin{lem}
We have the judgmental equality
\begin{equation*}
\jfameq
  {{{{\Gamma}{A}}{P}}{\ctxwk{P}{\mfam{A}}}}
  {\jcomp{}{\bar{\typefont{pr}}}{Q}}
  {\ctxwk{P}{Q}}
\end{equation*}
for any family $Q$ of contexts over $\ctxext{{\Gamma}{A}}{\mfam{A}}$ 
\end{lem}
\end{comment}

\subsection{Extension algebras}
In this subsection our goal is to define the notion of extension algebras,
which are internal versions of the extension operation of the theory of
contexts, families and terms. In this article, their use will be mainly in
universes. The theory of extension algebras requires the full power (i.e.~all
of the ingredients) of the theory of contexts, families and terms in its
formulation and it is (perhaps surprisingly) quite involved to formulate it.

Let $P$ be a family over an extended context $\ctxext{\Gamma}{A}$. We could
mimic extension by requiring to have terms
\begin{align*}
\jhom*{\Gamma}{{A}{P}}{A}{\epsilon_0}\\
\jhom*{{\Gamma}{A}}{{P}{\jcomp{}{\epsilon_0}{P}}}{P}{\epsilon_1}.
\end{align*}

\begin{rmk}
Instead of looking at $\epsilon_1$ as a context morphism from $\ctxext{P}{\jcomp{}{\epsilon_0}{P}}$
to $P$, one could also look at $\epsilon_1$ as a morphism \emph{over $\cprojfstf{A}{P}$},
as indicated in the following diagram:
\begin{equation*}
\begin{tikzcd}
P
  \ar[fib]{d}
& \jcomp{}{\epsilon_0}{P}
  \ar[fib]{d}
  \ar{l}[swap]{\epsilon_1}
  \ar{r}
& P
  \ar[fib]{d}
  \\
A
& \ctxext{A}{P}
  \ar{l}{\cprojfstf{A}{P}}
  \ar{r}[swap]{\epsilon_0}
& A
\end{tikzcd}
\end{equation*}
This makes it clear that $\epsilon_1$ takes a family over an extended context as an
argument. The extended context consists of a `base part' and a `family part'. 
The output of $\epsilon_1$ is a new (extended) family over that base part. Forgetting 
the family part is what the projection takes care of.
\end{rmk}

Extension also satisfies the properties explained in \autoref{comp-ee}, so we
must find the two judgmental equalities for $\epsilon_0$ and $\epsilon_1$ mimicing those. 
The first of these judgmental equalities is easy to give: it says that the
following diagram commutes:
\begin{equation}\label{eq:ealg-eq1}
\begin{tikzcd}[column sep=huge]
\ctxext{A}{{P}{\jcomp{}{\epsilon_0}{P}}} 
  \ar{d}[swap]{\jvcomp{}{\epsilon_0}{\idtm{\jcomp{}{\epsilon_0}{P}}}
    } 
  \ar{r}{\jvcomp{}{\idtm{A}}{\epsilon_1}
    } 
  & \ctxext{A}{P} \ar{d}{\epsilon_0}\\
\ctxext{A}{P} \ar{r}[swap]{\epsilon_0} & A
\end{tikzcd}
\end{equation}
To get a feel for this judgmental equality we include the following lemma.

\begin{lem}
Let $A$, $P$, $\epsilon_0$ and $\epsilon_1$ be as above, satisfying \autoref{eq:ealg-eq1} and
let $x_0:A$,
$x_1:\subst{x_0}{P}$ and $x_2:\subst{x_1}{{x_0}{\jcomp{}{\epsilon_0}{P}}}$.
Then we have the judgmental equality
\begin{equation*}
\subst{{x_2}{{x_1}{\epsilon_1}}}{{x_0}{\epsilon_0}}
\jdeq
\subst{x_2}{{{x_1}{{x_0}{\epsilon_0}}}{\epsilon_0}}.
\end{equation*}
\end{lem}

\begin{proof}
The proof is a simple computation:
\begin{align*}
\subst{{x_2}{{x_1}{\epsilon_1}}}{{x_0}{\epsilon_0}}
& \jdeq
  \subst{\tmext{x_0}{\subst{x_2}{{x_1}{\epsilon_1}}}}{\epsilon_0}
  \\
& \jdeq 
  \subst{{x_2}{{x_1}{{x_0}{\jvcomp{}{\idtm{A}}{\epsilon_1}}}}}{\epsilon_0}
  \\
& \jdeq
  \subst{x_2}{{x_1}{{x_0}{\jcomp{}{\jvcomp{}{\idtm{A}}{\epsilon_1}}{\epsilon_0}}}}
  \\
& \jdeq
  \subst{x_2}{{x_1}{{x_0}{\jcomp{}{\jvcomp{}{\epsilon_0}{\idtm{\jcomp{}{\epsilon_0}{P}}}}{\epsilon_0}}}}
  \\
& \jdeq 
  \subst{{x_2}{{x_1}{{x_0}{\jvcomp{}{\epsilon_0}{\idtm{\jcomp{}{\epsilon_0}{P}}}}}}}{\epsilon_0}
  \\
& \jdeq
  \subst{\tmext{\subst{x_1}{{x_0}{\epsilon_0}}}{x_2}}{\epsilon_0}
  \\
& \jdeq
  \subst{x_2}{{{x_1}{{x_0}{\epsilon_0}}}{\epsilon_0}}.\qedhere
\end{align*}
\end{proof}

The second of the judgmental equalities is harder to describe, however. We need
to consider `higher' families, i.e.~families over families over families, and
thus we need to look at the family $\jcomp{}{\epsilon_0}{\jcomp{}{\epsilon_0}{P}}$ and find the
two dotted morphisms in the diagram
\begin{equation*}
\begin{tikzcd}
\ctxext{P}{{\jcomp{}{\epsilon_0}{P}}{\jcomp{}{\epsilon_0}{\jcomp{}{\epsilon_0}{P}}}}
  \ar[densely dotted]{d}
  \ar[densely dotted]{r}
& \ctxext{P}{\jcomp{}{\epsilon_0}{P}} \ar{d}{\epsilon_1}\\
\ctxext{P}{\jcomp{}{\epsilon_0}{P}} \ar{r}{\epsilon_1} & P
\end{tikzcd}
\end{equation*}
The first is easy to find. Note that we have the judgmental equality
\begin{equation*}
\ctxext{\jcomp{}{\epsilon_0}{P}}{\jcomp{}{\epsilon_0}{\jcomp{}{\epsilon_0}{P}}}
  \jdeq
  \jcomp{}{\epsilon_0}{\ctxext{P}{\jcomp{}{\epsilon_0}{P}}}
\end{equation*}
and therefore we may just take the morphism
\begin{equation*}
\jhom
  {{\Gamma}{A}}
  {{P}{{\jcomp{}{\epsilon_0}{P}}{\jcomp{}{\epsilon_0}{\jcomp{}{\epsilon_0}{P}}}}}
  {{P}{\jcomp{}{\epsilon_0}{P}}}
  {\jvcomp{}{\idtm{P}}{\jcomp{}{\epsilon_0}{\epsilon_1}}}.
\end{equation*}
For the other morphism we need to look at the family
$\ctxext{P}{{\jcomp{}{\epsilon_0}{P}}{\jcomp{}{\epsilon_0}{\jcomp{}{\epsilon_0}{P}}}}$ differently. We do
that in the following lemma.

\begin{lem}
Suppose we have $A$, $P$, $\epsilon_0$ and $\epsilon_1$ satisfying \autoref{eq:ealg-eq1}. Then
we have the judgmental equality
\begin{equation*}
\jcomp{}{\epsilon_0}{\jcomp{}{\epsilon_0}{P}}
  \jdeq
  \jcomp{}{\epsilon_1}{\jcomp{}{\epsilon_0}{P}}.
\end{equation*}
\end{lem}

\begin{proof}
The proof is a rather long computation. It was found by starting with the
application of \autoref{eq:ealg-eq1} and making the computation in both ways
from there. The judgmental equality below where \autoref{eq:ealg-eq1} is applied,
is marked with an $*$. The computation goes as follows:
\begin{align*}
\jcomp{}{\epsilon_0}{\jcomp{}{\epsilon_0}{P}}
& \jdeq
  \subst{\epsilon_0}
    {\ctxwk{\ctxext{A}{P}}{\jcomp{}{\epsilon_0}{P}}}
  \\
& \jdeq
  \subst
    {\idtm{\jcomp{}{\epsilon_0}{P}}}
    { \ctxwk
        {\jcomp{}{\epsilon_0}{P}}{\subst{\epsilon_0}
        {\ctxwk{\ctxext{A}{P}}{\jcomp{}{\epsilon_0}{P}}}}}
  \\
& \jdeq
  \subst
    {\idtm{\jcomp{}{\epsilon_0}{P}}}
    { {\ctxwk{\jcomp{}{\epsilon_0}{P}}{\epsilon_0}}
      {\ctxwk{\jcomp{}{\epsilon_0}{P}}{{\ctxext{A}{P}}{\jcomp{}{\epsilon_0}{P}}}}}
  \\
& \jdeq
  \subst
    {\idtm{\jcomp{}{\epsilon_0}{P}}}
    { {\ctxwk{\jcomp{}{\epsilon_0}{P}}{\epsilon_0}}
      {\ctxwk{\ctxext{{A}{P}}{\jcomp{}{\epsilon_0}{P}}}{\jcomp{}{\epsilon_0}{P}}}}
  \\
& \jdeq
  \subst
    {\idtm{\jcomp{}{\epsilon_0}{P}}}
    { {\ctxwk{\jcomp{}{\epsilon_0}{P}}{\epsilon_0}}
      {\ctxwk{\ctxext{A}{{P}{\jcomp{}{\epsilon_0}{P}}}}{\jcomp{}{\epsilon_0}{P}}}}
  \\
& \jdeq
  \subst
    {\unfold{\jvcomp{\jcomp{}{\epsilon_0}{P}}{\epsilon_0}{\idtm{\jcomp{}{\epsilon_0}{P}}}}}
    {\ctxwk{\ctxext{A}{{P}{\jcomp{}{\epsilon_0}{P}}}}{\jcomp{}{\epsilon_0}{P}}}
  \\
& \jdeq
  \subst
    {\jvcomp{}{\epsilon_0}{\idtm{\jcomp{}{\epsilon_0}{P}}}}
    {\ctxwk{\ctxext{A}{{P}{\jcomp{}{\epsilon_0}{P}}}}{\jcomp{}{\epsilon_0}{P}}}
  \\
& \jdeq
  \jcomp{}{\jvcomp{}{\epsilon_0}{\idtm{\jcomp{}{\epsilon_0}{P}}}}{\jcomp{}{\epsilon_0}{P}}
  \\
& \jdeq
  \jcomp{}{\jcomp{}{\jvcomp{}{\epsilon_0}{\idtm{\jcomp{}{\epsilon_0}{P}}}}{\epsilon_0}}{P}
  \\
& \stackrel{*}{\jdeq}
  \jcomp{}{
    \jcomp{}{\jvcomp{}{\idtm{A}}{\epsilon_1}}{\epsilon_0}}{P}
  \\
& \jdeq
  \jcomp{}{\jvcomp{}{\idtm{A}}{\epsilon_1}}{
    \jcomp{}{\epsilon_0}{P}}
  \\
& \jdeq
  \subst
    {\tmext{\ctxwk{\ctxext{P}{\jcomp{}{\epsilon_0}{P}}}{\idtm{A}}}{\epsilon_1}}
    {\ctxwk{\ctxext{A}{{P}{\jcomp{}{\epsilon_0}{P}}}}{\jcomp{}{\epsilon_0}{P}}}
  \\
& \jdeq
  \subst
    {\epsilon_1}
    { {\ctxwk{\ctxext{P}{\jcomp{}{\epsilon_0}{P}}}{\idtm{A}}}
      {\ctxwk{\ctxext{A}{{P}{\jcomp{}{\epsilon_0}{P}}}}{\jcomp{}{\epsilon_0}{P}}}
      }
  \\
& \jdeq
  \subst
    {\epsilon_1}
    { {\cprojfstf{A}{\ctxext{P}{\jcomp{}{\epsilon_0}{P}}}}
      {\ctxwk{\ctxext{A}{{P}{\jcomp{}{\epsilon_0}{P}}}}{\jcomp{}{\epsilon_0}{P}}}
      }
  \\
& \jdeq
  \subst
    {\epsilon_1}
    {\jcomp{}{\cprojfstf{A}{\ctxext{P}{\jcomp{}{\epsilon_0}{P}}}}{\jcomp{}{\epsilon_0}{P}}}
  \\
& \jdeq
  \subst
    {\epsilon_1}
    {\ctxwk{\ctxext{P}{\jcomp{}{\epsilon_0}{P}}}{\jcomp{}{\epsilon_0}{P}}}
  \\
& \jdeq
  \jcomp{}{\epsilon_1}{\jcomp{}{\epsilon_0}{P}}.
  \qedhere
\end{align*}
\end{proof}

Now we see that we can use the morphism
\begin{equation*}
\jhom
  {{\Gamma}{A}}
  {{P}{{\jcomp{}{\epsilon_0}{P}}{\jcomp{}{\epsilon_1}{{}{\epsilon_0}{P}}}}}
  {{P}{\jcomp{}{\epsilon_0}{P}}}
  {\jvcomp{}{\epsilon_1}{\idtm{\jcomp{}{\epsilon_1}{{}{\epsilon_0}{P}}}}}.
\end{equation*}
Thus, the second judgmental equality we will need is that the diagram
\begin{equation}\label{eq:ealg-eq2}
\begin{tikzcd}[column sep=huge]
\ctxext{P}{{\jcomp{}{\epsilon_0}{P}}{\jcomp{}{\epsilon_0}{{}{\epsilon_0}{P}}}} 
  \ar{r}{\jvcomp{}{\idtm{P}}{\jcomp{}{\epsilon_0}{\epsilon_1}}}
  \ar{d}[swap]{
    \jvcomp{}{\epsilon_1}{\idtm{\jcomp{}{\epsilon_1}{{}{\epsilon_0}{P}}}}
%    \tmext
%      {\ctxwk{\jcomp{}{\bar{e}}{{}{e}{P}}}{f}}
%      {\idtm{\jcomp{}{\bar{e}}{{}{e}{P}}}}
    }
& \ctxext{P}{\jcomp{}{\epsilon_0}{P}} \ar{d}{\epsilon_1}\\
\ctxext{P}{\jcomp{}{\epsilon_0}{P}} \ar{r}[swap]{\epsilon_1} & P
\end{tikzcd}
\end{equation}
commutes judgmentally. Now we can confidently formulate the definition of
extension algebras.

\begin{defn}
An \emph{extension algebra in context $\Gamma$} is a quadruple $(A,P,\epsilon_0,\epsilon_1)$
consisting of a family $A$ over context $\Gamma$, a family $P$ over the context
$\ctxext{\Gamma}{A}$, a morphism $\epsilon_0$ from $\ctxext{A}{P}$ to $A$ in context
$\Gamma$ and a morphism $\epsilon_1$ from $\ctxext{P}{\jcomp{}{\epsilon_0}{P}}$ to $P$ in context
$\ctxext{\Gamma}{A}$, satisfying the judgmental equalities of
\autoref{eq:ealg-eq1,eq:ealg-eq2}.
\end{defn}

We also give a bit of intuition to the requirement of \autoref{eq:ealg-eq2} by
means of the following lemma.

\begin{lem}
Let $(A,P,\epsilon_0,\epsilon_1)$ be an extension algebra in context $\Gamma$ and let
$y_0:P$, $y_1:\subst{y_0}{\jcomp{}{\epsilon_0}{P}}$ and 
$y_2:\subst{y_1}{{y_0}{\jcomp{}{\epsilon_0}{\jcomp{}{\epsilon_0}{P}}}}$. Then we have the
judgmental equality
\begin{equation*}
\subst{{y_2}{{y_1}{\jcomp{}{\epsilon_0}{\epsilon_1}}}}{{y_0}{\epsilon_1}}
  \jdeq
  \subst{y_2}{{{y_1}{{y_0}{\epsilon_1}}}{\epsilon_1}}
\end{equation*}
\end{lem}

\begin{proof}
The proof is a straightforward calculation:
\begin{align*}
\subst{{y_2}{{y_1}{\jcomp{}{\epsilon_0}{\epsilon_1}}}}{{y_0}{\epsilon_1}}
& \jdeq
  \subst{\tmext{y_0}{\subst{y_2}{{y_1}{\jcomp{}{\epsilon_0}{\epsilon_1}}}}}{\epsilon_1}
  \\
& \jdeq
  \subst{{y_2}{{y_1}{{y_0}{\jvcomp{}{\idtm{P}}{\jcomp{}{\epsilon_0}{\epsilon_1}}}}}}{\epsilon_1}
  \\
& \jdeq
  \subst
    {y_2}
    { {y_1}
      { {y_0}
        {\jcomp{}{\jvcomp{}{\idtm{P}}{\jcomp{}{\epsilon_0}{\epsilon_1}}}{\epsilon_1}}
        }
      }
  \\
& \jdeq
  \subst
    {y_2}
    { {y_1}
      { {y_0}
        {\jcomp{}{\jvcomp{}{\epsilon_1}{\idtm{\jcomp{}{\epsilon_1}{{}{\epsilon_0}{P}}}}}{\epsilon_1}}
        }
      }
  \\
& \jdeq
  \subst{{y_2}{{y_1}{{y_0}{\jvcomp{}{\epsilon_1}{\idtm{\jcomp{}{\epsilon_1}{{}{\epsilon_0}{P}}}}}}}}{\epsilon_1}
  \\
& \jdeq
  \subst{\tmext{\subst{y_1}{{y_0}{\epsilon_1}}}{y_2}}{\epsilon_1}
  \\
& \jdeq
  \subst{y_2}{{{y_1}{{y_0}{\epsilon_1}}}{\epsilon_1}}.\qedhere
\end{align*}
\end{proof}

\begin{comment}
Extension algebras don't come in isolation. There are also extension algebra
families and extension algebra terms. We now aim to define these and to
establish various constructions of new extension algebras out of old ones:
the empty extension algebra, and extensions, weakenings and substitutions
of extension algebras and of course the identity term as an extension algebra
term. We start with extension algebra families.

\begin{defn}
Consider an extension algebra $\mathcal{A}\defeq(A,P,e,f)$. 
An extension algebra family over $\mathcal{A}$ is likewise a quadruple
$\mathcal{B}\defeq(B,Q,g,h)$. Here we have a family $\jfam{{\Gamma}{A}}{B}$, a
family $\jfam{{{{\Gamma}{A}}{P}}{\ctxwk{P}{B}}}{Q}$ and
\begin{align*}
\jhom*{{{\Gamma}{A}}{P}}{\ctxext{\ctxwk{P}{B}}{Q}}{\jcomp{}{\epsilon_0}{B}}{g}\\
\jhom*{{{{\Gamma}{A}}{P}}{\ctxwk{P}{B}}}{\ctxext{Q}{\jcomp{}{g}{Q}}}{Q}{h}.
\end{align*}
The quadruple $(\jcomp{}{\epsilon_0}{B},Q,g,h)$ is required to be an extension algebra
in context $\ctxext{{\Gamma}{A}}{P}$.
\end{defn}

\begin{defn}
Suppose $\mathcal{A}$ is an extension algebra and $\mathcal{B}$ is an extension
algebra family over $\mathcal{A}$. A term of $\mathcal{B}$ is a pair $(x,y)$
consisting of
\begin{align*}
\jterm*{{\Gamma}{A}}{B}{x}\\
\jterm*{{{\Gamma}{A}}{P}}{\subst{\jcomp{}{\epsilon_0}{x}}{Q}}{y}
\end{align*}
such that the diagrams
\begin{equation*}
\begin{tikzcd}
\ctxext{\jcomp{}{\epsilon_0}{B}}{Q} 
  \ar{r}{g} 
  \ar[shift right=.7ex,fib]{d}
& B 
  \ar[shift right=.7ex,fib]{d} 
  \\
\ctxext{A}{P} 
  \ar[shift right=.7ex,dotted]{u}[swap]{\tmext{\jcomp{}{\epsilon_0}{x}}{y}}
  \ar{r}{e}
& A
  \ar[shift right=.7ex,dotted]{u}[swap]{x}
\end{tikzcd}
\end{equation*}
and
\begin{equation*}
\begin{tikzcd}
\jcomp{}{f}{\ctxext{Q}{\jcomp{}{g}{Q}}}
  \ar{r}{\jcomp{}{f}{h}}
  \ar[shift right=.7ex,fib]{d}
& Q
  \ar[shift right=.7ex,fib]{d}
  \\
\jcomp{}{f}{\jcomp{}{\epsilon_0}{B}}
  \ar{r}{\idtm{\jcomp{}{f}{\jcomp{}{\epsilon_0}{B}}}}
  \ar[shift right=.7ex,fib]{d}
  \ar[shift right=.7ex,dotted,mapsto]{u}[swap]{\jcomp{}{f}{y}}
& \jcomp{}{\epsilon_0}{B}
  \ar[shift right=.7ex,fib]{d}
  \ar[shift right=.7ex,dotted,mapsto]{u}[swap]{y}
  \\
\ctxext{P}{\jcomp{}{\epsilon_0}{P}}
  \ar{r}[swap]{f}
  \ar[shift right=.7ex,dotted]{u}[swap]{\jcomp{}{f}{\jcomp{}{\epsilon_0}{x}}}
& P
  \ar[shift right=.7ex,dotted]{u}[swap]{\jcomp{}{\epsilon_0}{x}}
\end{tikzcd}
\end{equation*}
commute.
\end{defn}

\begin{defn}
Suppose $\mathcal{A}$ and $\mathcal{B}$ are extension algebras in context
$\Gamma$. We define the extension algebra $\ctxwk{\mathcal{A}}{\mathcal{B}}$
to be the quadruple
\begin{equation*}
(\ctxwk{A}{B},\ctxwk{\ctxext{A}{P}}{Q},\ctxwk{\ctxext{A}{P}}{g},\ctxwk{\ctxext{A}{P}}{h}).
\end{equation*}
Note that $\ctxwk{\ctxext{A}{P}}{Q}$ is a family over $\ctxwk{\ctxext{A}{P}}{B}$,
whereas it should be a family over $\jcomp{}{\epsilon_0}{\ctxwk{A}{B}}$. These are the
same by \autoref{lem:prehom}.
\end{defn}

\begin{rmk}
Before we continue, let us explore what it means to be an extension algebra
term of the extension algebra $\ctxwk{\mathcal{A}}{\mathcal{B}}$. Such an
extension algebra term $(x,y)$ would consist of
\begin{align*}
\jterm*{{\Gamma}{A}}{\ctxwk{A}{B}}{x}\\
\jterm*{{{\Gamma}{A}}{P}}{\subst{\jcomp{}{\epsilon_0}{x}}{\ctxwk{\ctxext{A}{P}}{Q}}}{y}.
\end{align*}
Thus, $x$ is a context morphism from $A$ to $B$ and $y$ is nothing but a term
of $\jcomp{}{\jcomp{}{\epsilon_0}{x}}{Q}$. For $x$, we see that the diagram
\begin{equation*}
\begin{tikzcd}
\ctxext{B}{Q} 
  \ar{r}{g} 
& B 
  \\
\ctxext{A}{P} 
  \ar{u}{\jvcomp{}{x}{y}}
  \ar{r}{e}
& A
  \ar{u}[swap]{x}
\end{tikzcd}
\end{equation*}
commutes.
\end{rmk}
\end{comment}

\subsection{Extension-empty algebras}
\begin{defn}
Let $P$ be a family over the extended context $\ctxext{\Gamma}{A}$, let
$\jterm{\Gamma}{A}{\phi_0}$ and $\jterm{{\Gamma}{A}}{P}{\phi_1}$ be terms. Then the
quadruple $(A,P,\phi_0,\phi_1)$ is said to be an \emph{empty algebra in context $\Gamma$}
if the following judgmental equalities hold:
\begin{align}
\jfameq*{\Gamma}{\subst{\phi_0}{P}}{A}
  \label{empalg-eq1}
  \\
\jtermeq*{\Gamma}{A}{\subst{\phi_0}{\phi_1}}{\phi_0}.
  \label{empalg-eq2}
\end{align}
\end{defn}

Thus, the empty algebras are the kind of algebras that require that families
are compatible with contexts, just as our motivation in \autoref{empty}. We
now combine the notion of extension algebras and empty algebras.

An extension-empty algebra in context $\Gamma$ is going to be a sextuple
$(A,P,\epsilon_0,\epsilon_1,\phi_0,\phi_1)$ for which 
the quadruple $(A,P,\epsilon_0,\epsilon_1)$ is an extension algebra in context 
$\Gamma$, the quadruple $(A,P,\phi_0,\phi_1)$ is an empty algebra in context
$\Gamma$, satisfying additional judgmental equalities expressing the 
compatibility of $\epsilon_0$ and $\epsilon_1$ with $\phi_0$ and $\phi_1$.
There will be four such judgmental equalities.

We can immediately state the first two:
\begin{align}
\jtermeq*{{\Gamma}{A}}{\ctxwk{A}{A}}{\subst{\phi_0}{\epsilon_0}}{\idtm{A}}
  \label{extempalg-eq1}
  \\
\jtermeq*{{\Gamma}{A}}{\ctxwk{A}{A}}{\subst{\phi_1}{\epsilon_0}}{\idtm{A}}
  \label{extempalg-eq2}
\end{align}
To see what $\subst{\phi_0}{\epsilon_1}$ can be, we must know its type first.
It is a morphism from $\subst{\phi_0}{\ctxext{P}{\jcomp{}{\epsilon_0}{P}}}$ to
$\subst{\phi_0}{P}$. We already know that $\subst{\phi_0}{P}\jdeq A$ by
\autoref{empalg-eq1} and to compute $\subst{\phi_0}{\jcomp{}{\epsilon_0}{P}}$
we use the following lemma.

\begin{lem}\label{lem:empalg-mor}
Consider an empty algebra $(A,P,\phi_0,\phi_1)$ in context $\Gamma$
and a morphism $\jhom{\Gamma}{{A}{P}}{B}{f}$.
Then $\subst{\phi_i}{f}$ is a morphism from $A$ to $B$ in context $\Gamma$ and
the following inference rules are valid for $i$ being $0$ or $1$:
\begin{align*}
& \inference
  { \jfam{{\Gamma}{A}}{Q}
    }
  { \jfameq
      {{\Gamma}{A}}
      {\subst{\phi_i}{\jcomp{}{f}{Q}}}
      {\jcomp{}{\subst{\phi_i}{f}}{Q}}
    }
  \\
& \inference
  { \jfam{{{\Gamma}{A}}{Q}}{R}
    }
  { \jfameq
      {{{\Gamma}{A}}{\jcomp{}{\subst{\phi_i}{f}}{Q}}}
      {\subst{\phi_i}{\jcomp{}{f}{R}}}
      {\jcomp{}{\subst{\phi_i}{f}}{R}}
    }
  \\
& \inference
  { \jterm{{{\Gamma}{A}}{Q}}{R}{h}
    }
  { \jtermeq
      {{{\Gamma}{A}}{\jcomp{}{\subst{\phi_i}{f}}{Q}}}
      {\jcomp{}{\subst{\phi_i}{f}}{R}}
      {\subst{\phi_i}{\jcomp{}{f}{h}}}
      {\jcomp{}{\subst{\phi_i}{f}}{h}}
    }
\end{align*}
\end{lem}

\begin{proof}
We only prove the first inference rule in both cases.
Let $Q$ be a family over $\ctxext{\Gamma}{A}$. In the case $i=0$
 we have the judgmental equalities
\begin{align*}
\subst{\phi_0}{\jcomp{}{f}{Q}}
& \jdeq
  \subst{\phi_0}{{f}{\ctxwk{\ctxext{A}{P}}{Q}}}
  \tag{by definition}
  \\
& \jdeq
  \subst{{\phi_0}{f}}{{\phi_0}{\ctxwk{\ctxext{A}{P}}{Q}}}
  \tag{by \autoref{comp-ss-f}}
  \\
& \jdeq
  \subst{{\phi_0}{f}}{{\phi_0}{\ctxwk{P}{{A}{Q}}}}
  \tag{by \autoref{comp-ew-f}}
  \\
& \jdeq
  \subst{{\phi_0}{f}}{\ctxwk{\subst{\phi_0}{P}}{\subst{\phi_0}{\ctxwk{A}{Q}}}}
  \tag{by \autoref{comp-sw-f}}
  \\
& \jdeq
  \subst{{\phi_0}{f}}{\ctxwk{\subst{\phi_0}{P}}{Q}}
  \tag{by \autoref{cancellation-ws-f}}
  \\
& \jdeq
  \subst{{\phi_0}{f}}{\ctxwk{A}{Q}}
  \tag{by \autoref{empalg-eq1}}
  \\
& \jdeq
  \jcomp{}{\subst{\phi_0}{f}}{Q}.
  \tag{by definition}
\end{align*}
In the case $i=1$ we have the judgmental equalities
\begin{align*}
\subst{\phi_1}{\jcomp{}{f}{Q}}
& \jdeq
  \subst{\phi_1}{{f}{\ctxwk{\ctxext{A}{P}}{Q}}}
  \tag{by definition}
  \\
& \jdeq
  \subst{{\phi_1}{f}}{{\phi_1}{\ctxwk{\ctxext{A}{P}}{Q}}}
  \tag{by \autoref{comp-ss-f}}
  \\
& \jdeq
  \subst{{\phi_1}{f}}{{\phi_1}{\ctxwk{P}{{A}{Q}}}}
  \tag{by \autoref{comp-ew-f}}
  \\
& \jdeq
  \subst{{\phi_1}{f}}{\ctxwk{A}{Q}}
  \tag{by \autoref{cancellation-ws-f}}
  \\
& \jdeq
  \jcomp{}{\subst{\phi_1}{f}}{Q}.
  \tag{by definition}
\end{align*}
\end{proof}

As an immediate corollary, if we assume the judgmental equalities
\autoref{extempalg-eq1,extempalg-eq2} we get that 
\begin{equation}\label{cor:empalg-mor}
\jfameq{{\Gamma}{A}}{\subst{\phi_i}{\jcomp{}{\epsilon_0}{P}}}{P}
\end{equation}
and hence that $\subst{\phi_0}{\epsilon_1}$ is a 
morphism from $\ctxext{A}{P}$ to $A$. Thus, we can require
\begin{equation}\label{extempalg-eq3}
\jhomeq{\Gamma}{{A}{P}}{A}{\subst{\phi_0}{\epsilon_1}}{\epsilon_0}.
\end{equation}
For the final judgmental equality we need to explain the term
\begin{equation*}
\jterm
  {{{\Gamma}{A}}{\subst{\phi_1}{\jcomp{}{\epsilon_0}{P}}}}
  {\subst{\phi_1}{\ctxwk{\ctxext{P}{\jcomp{}{\epsilon_0}{P}}}{P}}}
  {\subst{\phi_1}{\epsilon_1}}.
\end{equation*}
We have already established that $\subst{\phi_1}{\jcomp{}{\epsilon_0}{P}}\jdeq
P$. We also see that 
\begin{align*}
\subst{\phi_1}{\ctxwk{\ctxext{P}{\jcomp{}{\epsilon_0}{P}}}{P}}
& \jdeq
  \subst{\phi_1}{\ctxwk{\jcomp{}{\epsilon_0}{P}}{{P}{P}}}
  \tag{by \autoref{comp-ew-f}}
  \\
& \jdeq
  \ctxwk{\subst{\phi_1}{\jcomp{}{\epsilon_0}{P}}}{\subst{\phi_1}{\ctxwk{P}{P}}}
  \tag{by \autoref{comp-sw-f}}
  \\
& \jdeq
  \ctxwk{P}{\subst{\phi_1}{\ctxwk{P}{P}}}
  \tag{by \autoref{cor:empalg-mor}}
  \\
& \jdeq
  \ctxwk{P}{P}
  \tag{by \autoref{cancellation-ws-f}}
\end{align*}
and we will therefore require that
\begin{equation}\label{extempalg-eq4}
\jtermeq{{{\Gamma}{A}}{P}}{\ctxwk{P}{P}}{\subst{\phi_1}{\epsilon_1}}{\idtm{P}}.
\end{equation}
We bring all this together in the definition of extension-empty algebras:

\begin{defn}
An \emph{extension-empty algebra in context $\Gamma$} 
is a sextuple $(A,P,\epsilon_0,\epsilon_1,\phi_0,\phi_1)$ for which 
the quadruple $(A,P,\epsilon_0,\epsilon_1)$ is an extension algebra in context 
$\Gamma$, the quadruple $(A,P,\phi_0,\phi_1)$ is an empty algebra in context
$\Gamma$, satisfying the judgmental equalities 
\autoref{extempalg-eq1,extempalg-eq2,extempalg-eq3,extempalg-eq4}.
\end{defn}

\subsection{Extension-weakening algebras}
The notion of (extension-)weakening algebra is dependent on the notion of extension algebra.
Although it is strictly speaking not dependent on the notion of empty-algebra,
we shall only formulate a weakening algebras in the setting of an
extension-empty algebras. When one wants to have a weakening operation which
also acts on the level of contexts in an extension algebra without empty
context, extra work has to be done to introduce these separately.

An extension-weakening algebra in context $\Gamma$ will be an octuple
\begin{equation*}
(A,P,\epsilon_0,\epsilon_1,\phi_0,\phi_1,\omega_0,\omega_1)
\end{equation*}
where $(A,P,\epsilon_0,\epsilon_1,\phi_0,\phi_1)$ is an extension-empty algebra
in context $\Gamma$ and where
\begin{align*}
\jhom*
  {{{\Gamma}{A}}{P}}
  {\ctxwk{P}{P}}
  {\jcomp{}{\epsilon_0}{P}}
  {\omega_0}
  \\
\jfhom*
  {{{\Gamma}{A}}{P}}
  {\ctxwk{P}{P}}
  {\jcomp{}{\epsilon_0}{P}}
  {\omega_0}
  {\ctxwk{P}{\jcomp{}{\epsilon_0}{P}}}
  {\jcomp{}{\epsilon_0}{\jcomp{}{\epsilon_0}{P}}}
  {\omega_1}
\end{align*}
satisfying nine additional judgmental equalities expressing that $\omega_0$
and $\omega_1$ are compatible with $\epsilon_1$, $\phi_1$
and with each other. Our current goal is to figure out what these are. 

Before we go into that, we develop a bit of intuition by explaining how
$\omega_0$ and $\omega_1$ act when applied to a family $x_1:\subst{x_0}{P}$
over $x_0:A$. Note that
\begin{equation*}
\jterm{{\Gamma}{\subst{x_0}{P}}}{\subst{{x_1}{{x_0}{\epsilon_0}}}{P}}{\subst{x_1}{{x_0}{\omega_0}}},
\end{equation*}
so $\subst{x_1}{{x_0}{\omega_0}}$ takes families over $x_0$ to families over
the extended $\subst{x_1}{{x_0}{\epsilon_0}}$.

To compute the type of $\subst{x_2}{{x_1}{{x_0}{\omega_1}}}$ for families
$x_1,x_2:\subst{x_0}{P}$ over $x_0:A$ we have to do a
bit more work. Note that
\begin{align*}
& \subst{x_2}{{x_1}{{x_0}{\jcomp{}{\omega_0}{\jcomp{}{\epsilon_0}{\jcomp{}{\epsilon_0}{P}}}}}}
  \\
& \jdeq
  \subst{x_2}{{x_1}{{x_0}{\unfold{\jcomp{\ctxwk{P}{P}}{\omega_0}{\jcomp{}{\epsilon_0}{\jcomp{}{\epsilon_0}{P}}}}}}}
  \tag{by definition}
  \\
& \jdeq
  \subst
    { {x_2}
      { {x_1}
        { {x_0}
          {\omega_0}
          }
        }
      }
    { {x_2}
      { {x_1}
        { {x_0}
          {\ctxwk{{P}{P}}{\jcomp{}{\epsilon_0}{\jcomp{}{\epsilon_0}{P}}}}
          }
        }
      }
  \tag{by \autoref{comp-ss-f}}
  \\
& \jdeq
  \subst
    { {x_2}
      { {x_1}
        { {x_0}
          {\omega_0}
          }
        }
      }
    { {x_2}
      { \ctxwk
          {\subst{x_1}{\ctxwk{\subst{x_0}{P}}{\subst{x_0}{P}}}}
          {\subst{x_1}{{x_0}{\jcomp{}{\epsilon_0}{\jcomp{}{\epsilon_0}{P}}}}}
        }
      }
  \tag{by \autoref{comp-sw-f}}
  \\
& \jdeq
  \subst
    { {x_2}
      { {x_1}
        { {x_0}
          {\omega_0}
          }
        }
      }
    { {x_1}
      { {x_0}
        {\jcomp{}{\epsilon_0}{\jcomp{}{\epsilon_0}{P}}}
        }
      }
  \tag{by \autoref{cancellation-ws-f}}
  \\
& \jdeq
  \subst
    { {{x_2}{{x_1}{{x_0}{\omega_0}}}}
      {{{x_1}{{x_0}{\epsilon_0}}}{\epsilon_0}}
      }
    { P
      }.
  \tag{by \autoref{comp-ss-f}}
\end{align*}
Therefore, we see that
\begin{equation*}
\jterm
  {{\Gamma}{\subst{{x_2}{{x_0}{\epsilon_0}}}{P}}}
  { \subst
      { {{x_2}{{x_1}{{x_0}{\omega_0}}}}
        {{{x_1}{{x_0}{\epsilon_0}}}{\epsilon_0}}
        }
      { P
        }
    }
  { \subst{x_2}{{x_1}{{x_0}{\omega_1}}}
    }
\end{equation*}
Thus, the term $\subst{x_2}{{x_1}{{x_0}{\omega_1}}}$ takes a family over the
extended context $\subst{x_2}{{x_0}{\epsilon_0}}$ to a family over the
weakening $\subst{x_2}{{x_1}{{x_0}{\omega_0}}}$, which is itself a family
over $\subst{{x_1}{{x_0}{\epsilon_0}}}{\epsilon_0}$. In other words, $\omega_1$
is precisely the internalization of the action on families of weakening, as
intended.
\subsubsection{The compatibility of weakening with extension}

\subsubsection{The compatibility of weakening with the empty context and family}
The first two judgmental equalities expressing that $\omega_0$ and $\omega_1$
are compatible with $\phi_1$ are easy to state:
\begin{align}
\jhomeq*{{\Gamma}{A}}{P}{P}{\subst{\phi_1}{\omega_0}}{\idtm{P}}\\
\jhomeq*{{\Gamma}{A}}{P}{P}{\subst{\phi_1}{\omega_1}}{\idtm{P}}
\end{align}

\subsubsection{The compatibility of weakening with itself}

\subsubsection{The definition of weakening algebras}

\subsection{Extension-substitution algebras}

\subsection{Pre-universes}
Pre-universes are internal versions of the theory of contexts, families and
terms. They interpret extension, the empty context, weakening, substitution
and identity terms all at once in a compatible way. Besides the compatibility
properties there will be judgmental equalities analoguous to the cancellation
properties of \autoref{cancellation-ws,cancellation-i}. Pre-universes are to
the theory of contexts, families and terms what internal categories to a
category.


%\section{Pretty type theory}
In this section we shall discuss conventions to algorithmically `pretty-print' judgments of
the theory of contexts, families and terms. These will include variable
declarations in contexts and omission of explicit notation of weakening. This
will render various notationally different expressions identical and thus
we shall have to show that when this happens, the expressions under consideration
were already judgmentally equal. Pretty type theory could be seen as a
semi-formal version of the theory of contexts, families and terms and it
will facilitate translation to the informal language developed in \cite{TheBook}.

\begin{comment}
\subsection{The basic judgments}
The basic judgments of pretty type theory are the same as for structural type
theory. There are judgments for: ``$\Gamma$ is a context'',
``$A(i)$ over $i:\Gamma$ is a family over $\Gamma$'', ``$A(i)$ over $i:\Gamma$ 
is a type in context $\Gamma$''
and ``$x(i)$ is a term of $A(i)$ above $i:\Gamma$''. The other four
judgments are for judgmental equality. 

\begin{align*}
\jvctx*{\Gamma} & \jvctxeq*{\Gamma}{\Gamma'}\\
\jvfam*{i}{\Gamma}{A} & \jvfameq*{i}{\Gamma}{A}{B}\\
\jvtype*{i}{\Gamma}{A} & \jvtypeeq*{i}{\Gamma}{A}{B}\\
\jvterm*{i}{\Gamma}{A}{x} & \jvtermeq*{i}{\Gamma}{A}{x}{y}.
\end{align*}

We have the following basic inference rules that relate types and families:

\begin{small}
\begin{align*}
& \inference
  {\jvtype{i}{\Gamma}{A}}
  {\jvfam{i}{\Gamma}{A}}
& & \inference
    {\jvtypeeq{i}{\Gamma}{A}{B}}
    {\jvfameq{i}{\Gamma}{A}{B}}\\
& \inference
  {\jvtype{i}{\Gamma}{A}
   \jvfameq{i}{\Gamma}{A}{B}}
  {\jvtype{i}{\Gamma}{B}}
& & \inference
    {\jvtype{i}{\Gamma}{A}
     \jvfameq{i}{\Gamma}{A}{B}}
    {\jvtypeeq{i}{\Gamma}{A}{B}}
\end{align*}
\end{small}

\subsection{The basic rules for judgmental equality}
The rules for judgmental equality establish that it is an equivalence relation.
\bgroup\small
\begin{align*}
& \inference
  {\jvctx{\Gamma}}
  {\jvctxeq{\Gamma}{\Gamma}} 
& & \inference
    {\jvctxeq{\Gamma}{\Delta}}
    {\jvctxeq{\Delta}{\Gamma}} 
& & \inference
    {\jvctxeq{\Gamma}{\Delta}
     \jvctxeq{\Delta}{\greek{E}}}
    {\jvctxeq{\Gamma}{\greek{E}}}\\
& \inference
  {\jvfam{i}{\Gamma}{A}}
  {\jvfameq{i}{\Gamma}{A}{A}} 
& & \inference
    {\jvfameq{i}{\Gamma}{A}{B}}
    {\jvfameq{i}{\Gamma}{B}{A}}
& & \inference
    {\jvfameq{i}{\Gamma}{A}{B}
     \jvfameq{i}{\Gamma}{B}{C}}
    {\jvfameq{i}{\Gamma}{A}{C}}\\
& \inference
  {\jvterm{i}{\Gamma}{A}{x}}
  {\jvtermeq{i}{\Gamma}{A}{x}{x}}
& & \inference
    {\jvtermeq{i}{\Gamma}{A}{x}{y}}
    {\jvtermeq{i}{\Gamma}{A}{y}{x}}
& & \inference
    {\jvtermeq{i}{\Gamma}{A}{x}{y}
     \jvtermeq{i}{\Gamma}{A}{y}{z}}
    {\jvtermeq{i}{\Gamma}{A}{x}{z}}
\end{align*}
\egroup

The following convertibility rules are responsible for the strictness
of judgmental equality, which sets it apart from equivalences or identifications:

\begin{align*}
& \inference
  {\jvctxeq{\Gamma}{\Delta}
   \jvfam{i}{\Gamma}{A}}
  {\jvfam{i}{\Delta}{A}}
& & \inference
    {\jvctxeq{\Gamma}{\Delta}
     \jvfameq{i}{\Gamma}{A}{B}}
    {\jvfameq{i}{\Delta}{A}{B}}\\
& \inference
  {\jvctxeq{\Gamma}{\Delta}
   \jvterm{i}{\Gamma}{A}{x}}
  {\jvterm{i}{\Delta}{A}{x}}
& & \inference
    {\jvctxeq{\Gamma}{\Delta}
     \jvtermeq{i}{\Gamma}{A}{x}{y}}
    {\jvtermeq{i}{\Delta}{A}{x}{y}}\\
& \inference
  {\jvfameq{i}{\Gamma}{A}{B}
   \jvterm{i}{\Gamma}{A}{x}}
  {\jvterm{i}{\Gamma}{B}{x}}
& & \inference
    {\jvfameq{i}{\Gamma}{A}{B}
     \jvtermeq{i}{\Gamma}{A}{x}{y}}
    {\jvtermeq{i}{\Gamma}{B}{x}{y}}
\end{align*}

\subsection{The empty context}
The empty context looks a bit strange when we explicitly denote the terms. But
we will not do so anymore after this subsection.

\begin{align}
& \inference
  {}
  {\jctx{\emptyc}}\\
& \inference
  {\jctx{\Gamma}}
  {\jvfam{i}{\Gamma}{\emptyf[\Gamma]}}\\
& \inference
  {\jctx{\Gamma}}
  {\jvterm{i}{\Gamma}{\emptyf[\Gamma]}{\emptytm[\Gamma]}}\\
& \inference
  {\jvterm{i}{\Gamma}{\emptyf[\Gamma]}{x}}
  {\jvtermeq{i}{\Gamma}{\emptyf[\Gamma]}{x}{\emptytm[\Gamma]}}
\end{align}

Moreover, if $\Gamma$ is a context family over the
empty context, then $\Gamma$ is a context and every context is a context
family over the empty context. Note that this allows us to speak
of terms of contexts too.

\begin{align}
& \inference
  {\jctx{\Gamma}}
  {\jvfam{\nameless}{\emptyc}{\Gamma}} 
& & \inference
    {\jvfam{\nameless}{\emptyc}{\Gamma}}
    {\jctx{\Gamma}}\\
& \inference
  {\jctxeq{\Gamma}{\Delta}}
  {\jvfameq{\nameless}{\emptyc}{\Gamma}{\Delta}}
& & \inference
    {\jvfameq{\nameless}{\emptyc}{\Gamma}{\Delta}}
    {\jctxeq{\Gamma}{\Delta}}
\end{align}

\subsubsection{The empty context is compatible with itslef}
The empty context $\emptyc$ may be considered as a family of contexts over the empty
context. When we do this, we get $\emptyf[\emptyc]$.
\begin{equation}
\inference
  {}
  {\jvfameq{\nameless}{\emptyc}{\emptyc}{\emptyf[\emptyc]}}
\end{equation}
In the future, we shall denote $\emptyf[\Gamma]$ by $\emptyf$. The above rule
guarantees that this will not cause confusion. Likewise, we shall denote
$\emptytm[\Gamma]$ by $\emptytm$.

\subsection{Extension}
We introduce extension which not only extends a context $\Gamma$ and a family
$A$ over it to a context $\ctxext{\Gamma}{A}$, but which also extends a family $A$
in context $\Gamma$ and a family $P$ over it to a family $\ctxext{A}{P}$ over context
$\Gamma$. We do this to ensure that all of type theory can be done in a context.
For instance, we could say (1) that a context in context $\Gamma$ is the same thing
as a family over $\Gamma$; (2) When $A$ is a context in this sense, a family over
$A$ is the same thing as a family $P$ over $\ctxext{\Gamma}{A}$ and 
(3) when $P$ is a family over $A$ in this sense, a term of $P$ keeps its original meaning.

\begin{align}
& \inference
  {\jvfam{i}{\Gamma}{A}}
  {\jvfamcombi{{i}{x}}{{\Gamma}{A}}{P}}
& & \inference
    {\jctxeq{\Gamma}{\Delta}
     \jfameq{\Gamma}{A}{B}}
    {\jctxeq{\ctxext{\Gamma}{A}}{\ctxext{\Delta}{B}}}\\
& \inference
  {\jfam{{\Gamma}{A}}{P}}
  {\jfam{\Gamma}{\ctxext{A}{P}}}
& & \inference
    {\jfameq{\Gamma}{A}{B}
     \jfameq{{\Gamma}{A}}{P}{Q}}
    {\jfameq{\Gamma}{\ctxext{A}{P}}{\ctxext{B}{Q}}}
\end{align}

\subsubsection{Extension is compatible with the empty context}
The following rule asserts that extension by $\emptyc$ leaves the contexts unchanged.
\begin{align}
& \inference
  {\jctx{\Gamma}}
  {\jctxeq{\ctxext{\emptyc}{\Gamma}}{\Gamma}}\\
& \inference
  {\jctx{\Gamma}}
  {\jctxeq{\ctxext{\Gamma}{\emptyf}}{\Gamma}}\\
& \inference
  {\jfam{\Gamma}{A}}
  {\jfameq{\Gamma}{\ctxext{\emptyf}{A}}{A}}
\end{align}

\subsubsection{Extension is compatible with itself}
The inference rules asserting that extension is compatible with itself assert
that contexts are unstructured lists of type declarations. This rule is
unavoidable if we want that for a family $A$ in context $\Gamma$, a family over
$A$ is the same thing as a family over $\ctxext{\Gamma}{A}$. 

\begin{align}
& \inference
  {\jfam{\Gamma}{A}
   \jfam{{\Gamma}{A}}{P}}
  {\jctxeq{\ctxext{{\Gamma}{A}}{P}}{\ctxext{\Gamma}{{A}{P}}}}\\
& \inference
  {\jfam{{\Gamma}{A}}{P}
   \jfam{{{\Gamma}{A}}{P}}{Q}}
  {\jfameq{\Gamma}{\ctxext{{A}{P}}{Q}}{\ctxext{A}{{P}{Q}}}}
\end{align}
\end{comment}


\part{Internal Models}\label{part:models}

\section{Internalizing the theory of contexts families and terms}
One of the guiding ideas behind the design of the theory of contexts, families
and terms was that it would have to be possible to consider internal versions
of the theory. In this section we aim for this internalization. We stress that
we shall not make any further assumptions in this section, and thus that
it is \emph{by default} possible to consider internal models of the theory
of contexts, families and terms in itself. In particular, we do not assume that
there are universes; this is the subject of a later section in this part.

It would be interesting to write out a weak version of pre-universes, internal
to Martin-L\"of type theory with the function extensionality principle. 
To do this, the empty context needs to
be replaced by a contractible type, extension by dependent pair types,
judgmental equalities of terms by identifications and judgmental equalities
of families by equivalences of types. We conjecture that it is possible to
carry this out (in particular to make sure that all the constructions 
type-check). This could serve as a starting point for investigating internal
models without truncatedness assumptions and for investigating internal higher
categories. 
Moreover, one could then extend the notion of `weak' pre-universes
with the requirement that every internal morphism is weakly anodyne. This
could give an internal theory of weak higher groupoids.

\subsection{Extension algebras}\label{sec:extension-algebras}
In this subsection our goal is to define the notion of extension algebras,
which are internal versions of the extension operation of the theory of
contexts, families and terms. In this article, their use will be mainly in
universes. The theory of extension algebras requires the full power (i.e.~all
of the ingredients) of the theory of contexts, families and terms in its
formulation and it is (perhaps surprisingly) quite involved to formulate it.
The definition of extension algebras is given in \autoref{defn:extension-algebras}.

Let $P$ be a family over an extended context $\ctxext{\Gamma}{A}$. We could
mimic extension by requiring to have terms
\begin{align*}
\jhom*{\Gamma}{{A}{P}}{A}{e_0}\\
\jhom*{{\Gamma}{A}}{{P}{\jcomp{}{e_0}{P}}}{P}{e_1}.
\end{align*}

\begin{rmk}
Instead of looking at $e_1$ as a context morphism from $\ctxext{P}{\jcomp{}{e_0}{P}}$
to $P$, one could also look at $e_1$ as a morphism \emph{over $\cprojfstf{A}{P}$},
as indicated in the following diagram:
\begin{equation*}
\begin{tikzcd}
P
  \ar[fib]{d}
& \jcomp{}{e_0}{P}
  \ar[fib]{d}
  \ar{l}[swap]{e_1}
  \ar{r}
& P
  \ar[fib]{d}
  \\
A
& \ctxext{A}{P}
  \ar{l}{\cprojfstf{A}{P}}
  \ar{r}[swap]{e_0}
& A
\end{tikzcd}
\end{equation*}
This makes it clear that $e_1$ takes a family over an extended context as an
argument. The extended context consists of a `base part' and a `family part'. 
The output of $e_1$ is a new (extended) family over that base part. Forgetting 
the family part is what the projection takes care of.
\end{rmk}

Extension also satisfies the properties explained in \autoref{comp-ee}, so we
must find the two judgmental equalities for $e_0$ and $e_1$ mimicing those. 
The first of these judgmental equalities is easy to give: it says that the
following diagram commutes:
\begin{equation}\label{eq:extalg-eq1}
\begin{tikzcd}[column sep=huge]
\ctxext{A}{{P}{\jcomp{}{e_0}{P}}} 
  \ar{d}[swap]{\jvcomp{}{e_0}{\idtm{\jcomp{}{e_0}{P}}}
    } 
  \ar{r}{\jvcomp{}{\idtm{A}}{e_1}
    } 
  & \ctxext{A}{P} \ar{d}{e_0}\\
\ctxext{A}{P} \ar{r}[swap]{e_0} & A
\end{tikzcd}
\end{equation}

The second of the judgmental equalities is harder to describe, however. We need
to consider `higher' families, i.e.~families over families over families, and
thus we need to look at the family $\jcomp{}{e_0}{\jcomp{}{e_0}{P}}$ and find the
two dotted morphisms in the diagram
\begin{equation*}
\begin{tikzcd}
\ctxext{P}{{\jcomp{}{e_0}{P}}{\jcomp{}{e_0}{\jcomp{}{e_0}{P}}}}
  \ar[densely dotted]{d}
  \ar[densely dotted]{r}
& \ctxext{P}{\jcomp{}{e_0}{P}} \ar{d}{e_1}\\
\ctxext{P}{\jcomp{}{e_0}{P}} \ar{r}{e_1} & P
\end{tikzcd}
\end{equation*}
The first is easy to find. Note that we have the judgmental equality
\begin{equation*}
\ctxext{\jcomp{}{e_0}{P}}{\jcomp{}{e_0}{\jcomp{}{e_0}{P}}}
  \jdeq
  \jcomp{}{e_0}{\ctxext{P}{\jcomp{}{e_0}{P}}}
\end{equation*}
and therefore we may just take the morphism
\begin{equation*}
\jhom
  {{\Gamma}{A}}
  {{P}{{\jcomp{}{e_0}{P}}{\jcomp{}{e_0}{\jcomp{}{e_0}{P}}}}}
  {{P}{\jcomp{}{e_0}{P}}}
  {\jvcomp{}{\idtm{P}}{\jcomp{}{e_0}{e_1}}}.
\end{equation*}
For the other morphism we need to look at the family
$\ctxext{P}{{\jcomp{}{e_0}{P}}{\jcomp{}{e_0}{\jcomp{}{e_0}{P}}}}$ differently. We do
that in the following lemma.

\begin{lem}\label{lem:extalg-twins}
Suppose we have $A$, $P$, $e_0$ and $e_1$ satisfying \autoref{eq:extalg-eq1}. Then
the inference rules
\begin{align*}
& \inference
  { \jfam{{\Gamma}{A}}{Q}
    }
  { \jfameq
      {{{{\Gamma}{A}}{P}}{\jcomp{}{e_0}{P}}}
      {\jcomp{}{e_0}{\jcomp{}{e_0}{Q}}}
      {\jcomp{}{e_1}{\jcomp{}{e_0}{Q}}}
    }
  \\
& \inference
  { \jfam{{{\Gamma}{A}}{Q}}{R}
    }
  { \jfameq
      {{{{{\Gamma}{A}}{P}}{\jcomp{}{e_0}{P}}}
        {\jcomp{}{e_0}{\jcomp{}{e_0}{Q}}}}
      {\jcomp{}{e_0}{\jcomp{}{e_0}{R}}}
      {\jcomp{}{e_1}{\jcomp{}{e_0}{R}}}
    }
  \\
& \inference
  { \jterm{{{\Gamma}{A}}{Q}}{R}{h}
    }
  { \jtermeq
      {{{{{\Gamma}{A}}{P}}{\jcomp{}{e_0}{P}}}
        {\jcomp{}{e_0}{\jcomp{}{e_0}{Q}}}}
      {\jcomp{}{e_0}{\jcomp{}{e_0}{R}}}
      {\jcomp{}{e_0}{\jcomp{}{e_0}{h}}}
      {\jcomp{}{e_1}{\jcomp{}{e_0}{h}}}
    }
\end{align*}
are valid.
\end{lem}

\begin{proof}
We only prove the first inference rule. Let $\jfam{{\Gamma}{A}}{Q}$ be a family.
Then we have the judgmental equalities
\begin{align*}
\jcomp{}{e_0}{\jcomp{}{e_0}{Q}}
& \jdeq
  \jcomp{}{\idtm{\jcomp{}{e_0}{P}}}{\jcomp{}{e_0}{\jcomp{}{e_0}{Q}}}
  \tag{by \autoref{precomp-idtm-f}}
  \\
& \jdeq
  \jcomp{}{\jvcomp{}{e_0}{\idtm{\jcomp{}{e_0}{P}}}}{\jcomp{}{e_0}{Q}}
  \tag{by \autoref{lem:composition-threesome}}
  \\
& \jdeq
  \jcomp{}{\jcomp{}{\jvcomp{}{e_0}{\idtm{\jcomp{}{e_0}{P}}}}{e_0}}{Q}
  \tag{by \autoref{lem:jcomp-jcomp}}
  \\
& \jdeq
  \jcomp{}{\jcomp{}{\jvcomp{}{\idtm{A}}{e_1}}{e_0}}{Q}
  \tag{by \autoref{eq:extalg-eq1}}
  \\
& \jdeq
  \jcomp{}{\jvcomp{}{\idtm{A}}{e_1}}{\jcomp{}{e_0}{Q}}
  \tag{by \autoref{lem:jcomp-jcomp}}
  \\
& \jdeq
  \jcomp{}{e_1}{\jcomp{}{\idtm{A}}{\jcomp{}{e_0}{Q}}}
  \tag{by \autoref{lem:composition-threesome}}
  \\
& \jdeq
  \jcomp{}{e_1}{\jcomp{}{e_0}{Q}}.
  \tag{by \autoref{precomp-idtm-f}}
\end{align*}
\end{proof}

Now we see that we can use the morphism
\begin{equation*}
\jhom
  {{\Gamma}{A}}
  {{P}{{\jcomp{}{e_0}{P}}{\jcomp{}{e_1}{{}{e_0}{P}}}}}
  {{P}{\jcomp{}{e_0}{P}}}
  {\jvcomp{}{e_1}{\idtm{\jcomp{}{e_1}{{}{e_0}{P}}}}}.
\end{equation*}
Thus, the second judgmental equality we will need is that the diagram
\begin{equation}\label{eq:extalg-eq2}
\begin{tikzcd}[column sep=huge]
\ctxext{P}{{\jcomp{}{e_0}{P}}{\jcomp{}{e_0}{{}{e_0}{P}}}} 
  \ar{r}{\jvcomp{}{\idtm{P}}{\jcomp{}{e_0}{e_1}}}
  \ar{d}[swap]{
    \jvcomp{}{e_1}{\idtm{\jcomp{}{e_1}{{}{e_0}{P}}}}
    }
& \ctxext{P}{\jcomp{}{e_0}{P}} \ar{d}{e_1}\\
\ctxext{P}{\jcomp{}{e_0}{P}} \ar{r}[swap]{e_1} & P
\end{tikzcd}
\end{equation}
commutes judgmentally. Now we can confidently formulate the definition of
extension algebras.

\begin{defn}\label{defn:extension-algebras}
We define the judgment
\begin{equation*}
\jextalg{\Gamma}{A}
\end{equation*}
asserting that $\extalg{A}$ is an \emph{extension algebra in context $\Gamma$}
to be the conjunction of the following seven judmgents:
\begin{align*}
\jextalgctx*{\Gamma}{A}
  \\
\jextalgfam*{\Gamma}{A}
  \\
\jextalgtm*{\Gamma}{A}
  \\
\jextalgctxext*{\Gamma}{A}
  \\
\jextalgfamext*{\Gamma}{A}
  \\
\jhomeq*
  { \Gamma}
  { {\cftalgc{\cftalg{A}}}
    { {\cftalgf{\cftalg{A}}}
      {\jcomp{}{\cftctxext[\cftalg{A}]}{\cftalgf{\cftalg{A}}}}
      }
    }
  { \cftalgc{\cftalg{A}}}
  { \jcomp{}
      { \jvcomp{}
          {\cftctxext[\cftalg{A}]}
          {\idtm{\jcomp{}{\cftctxext[\cftalg{A}]}{\cftalgf{\cftalg{A}}}}}
        }
      { \cftctxext[\cftalg{A}]}
    }
  { \jcomp{}
      { \jvcomp{}
          {\idtm{\cftalgc{\cftalg{A}}}}
          {\cftfamext[\cftalg{A}]}
        }
      { \cftctxext[\cftalg{A}]}
    }
  \\
\jhomeq*
  { {\Gamma}{\cftalgc{\cftalg{A}}}}
  { { \cftalgf{\cftalg{A}}}
    { { \jcomp{}{\cftctxext[\cftalg{A}]}{\cftalgf{\cftalg{A}}}}
      { \jcomp{}
          {\cftctxext[\cftalg{A}]}
          {{}{\cftctxext[\cftalg{A}]}{\cftalgf{\cftalg{A}}}}
        }
      }
    }
  { \cftalgf{\cftalg{A}}}
  { \jcomp{}
      { \cftfamext[\cftalg{A}]}
      { \jvcomp{}
          {\cftfamext[\cftalg{A}]}
          {\idtm{\jcomp{}{\cftfamext[\cftalg{A}]}{{}{\cftctxext[\cftalg{A}]}{\cftalgf{\cftalg{A}}}}}}}
    }
  { \jcomp{}
      { \cftfamext[\cftalg{A}]}
      { \jvcomp{}
          {\idtm{\cftalgf{\cftalg{A}}}}
          {\jcomp{}{\cftctxext[\cftalg{A}]}{\cftfamext[\cftalg{A}]}}
        }
    }
\end{align*}
In other words, an extension algebra $\extalg{A}$ in context $\Gamma$
is a quintuple $\unfold{\extalg{A}}$ 
satisfying the judgmental equalities displayed in the diagrams in
\autoref{eq:extalg-eq1,eq:extalg-eq2}. Extension algebras are judgmentally
equal if they are componentwise judgmentally equal.

When $\extalg{A}$ is an extension algebra in context $\Gamma$, we also refer to
$\cftctxext[\cftalg{A}]$ as the \emph{context extension of $\extalg{A}$} and to
$\cftfamext[\cftalg{A}]$ as the \emph{family extension of $\extalg{A}$}.
\end{defn}

The following lemma both provides intuition behind the judgmental equalities
we have required for context and family extension and it proves that context 
and family extension in fact satisfy the compatibility rules stated in 
\autoref{comp-ee}. 

\begin{lem}
Let $\extalg{A}$ be an extension algebra in context $\Gamma$.
Then $\cftctxext[\extalg{A}]$ and $\cftfamext[\extalg{A}]$ 
correctly interpret the rules for context
extension and family extension. 

More precisely, for $x_0:\cftalgc{\cftalg{A}}$, 
$x_1:\subst{x_0}{\cftalgf{\cftalg{A}}}$ and 
$x_2:\subst{{x_1}{{x_0}{\cftctxext[\extalg{A}]}}}{\cftalgf{\cftalg{A}}}$ 
we have the judgmental equality
\begin{align*}
\subst{{x_2}{{x_1}{\cftfamext[\extalg{A}]}}}{{x_0}{\cftctxext[\extalg{A}]}}
& \jdeq
  \subst
    {x_2}
    {{{x_1}{{x_0}{\cftctxext[\extalg{A}]}}}{\cftctxext[\extalg{A}]}}.
  \intertext{%
and for $y_0:\cftalgf{\extalg{A}}$, 
$y_1:\subst{y_0}{\jcomp{}{\cftfamext[\extalg{A}]}{\cftalgf{\extalg{A}}}}$ and 
$y_2:
  \subst
    {y_1}
    { {y_0}
      { \jcomp
          {}
          {\cftfamext[\extalg{A}]}
          {\jcomp{}{\cftfamext[\extalg{A}]}{\cftalgf{\extalg{A}}}}
        }
      }
$
we have the judgmental equality}
\subst
  {{y_2}{{y_1}{\jcomp{}{\cftfamext[\extalg{A}]}{\cftfamext[\extalg{A}]}}}}
  {{y_0}{\cftfamext[\extalg{A}]}}
& \jdeq
  \subst{y_2}{{{y_1}{{y_0}{\cftfamext[\extalg{A}]}}}{\cftfamext[\extalg{A}]}}.
\end{align*}
\end{lem}

\begin{proof}
Both proofs are simple calculations. For the first judgmental equality we have
\begin{align*}
\subst{{x_2}{{x_1}{\cftfamext[\cftalg{A}]}}}{{x_0}{\cftctxext[\cftalg{A}]}}
& \jdeq
  \subst{\tmext{x_0}{\subst{x_2}{{x_1}{\cftfamext[\cftalg{A}]}}}}{\cftctxext[\cftalg{A}]}
  \\
& \jdeq 
  \subst{{x_2}{{x_1}{{x_0}{\jvcomp{}{\idtm{\cftalgc{\cftalg{A}}}}{\cftfamext[\cftalg{A}]}}}}}{\cftctxext[\cftalg{A}]}
  \\
& \jdeq
  \subst{x_2}{{x_1}{{x_0}{\jcomp{}{\jvcomp{}{\idtm{\cftalgc{\cftalg{A}}}}{\cftfamext[\cftalg{A}]}}{\cftctxext[\cftalg{A}]}}}}
  \\
& \jdeq
  \subst{x_2}{{x_1}{{x_0}{\jcomp{}{\jvcomp{}{\cftctxext[\cftalg{A}]}{\idtm{\jcomp{}{\cftctxext[\cftalg{A}]}{\cftalgf{\cftalg{A}}}}}}{\cftctxext[\cftalg{A}]}}}}
  \\
& \jdeq 
  \subst{{x_2}{{x_1}{{x_0}{\jvcomp{}{\cftctxext[\cftalg{A}]}{\idtm{\jcomp{}{\cftctxext[\cftalg{A}]}{\cftalgf{\cftalg{A}}}}}}}}}{\cftctxext[\cftalg{A}]}
  \\
& \jdeq
  \subst{\tmext{\subst{x_1}{{x_0}{\cftctxext[\cftalg{A}]}}}{x_2}}{\cftctxext[\cftalg{A}]}
  \\
& \jdeq
  \subst{x_2}{{{x_1}{{x_0}{\cftctxext[\cftalg{A}]}}}{\cftctxext[\cftalg{A}]}}.
\end{align*}
and to prove the second judgmental equality we calculate
\begin{align*}
\subst{{y_2}{{y_1}{\jcomp{}{\cftctxext[\cftalg{A}]}{\cftfamext[\cftalg{A}]}}}}{{y_0}{\cftfamext[\cftalg{A}]}}
& \jdeq
  \subst{\tmext{y_0}{\subst{y_2}{{y_1}{\jcomp{}{\cftctxext[\cftalg{A}]}{\cftfamext[\cftalg{A}]}}}}}{\cftfamext[\cftalg{A}]}
  \\
& \jdeq
  \subst{{y_2}{{y_1}{{y_0}{\jvcomp{}{\idtm{P}}{\jcomp{}{\cftctxext[\cftalg{A}]}{\cftfamext[\cftalg{A}]}}}}}}{\cftfamext[\cftalg{A}]}
  \\
& \jdeq
  \subst
    {y_2}
    { {y_1}
      { {y_0}
        {\jcomp{}{\jvcomp{}{\idtm{P}}{\jcomp{}{\cftctxext[\cftalg{A}]}{\cftfamext[\cftalg{A}]}}}{\cftfamext[\cftalg{A}]}}
        }
      }
  \\
& \jdeq
  \subst
    {y_2}
    { {y_1}
      { {y_0}
        {\jcomp{}{\jvcomp{}{\cftfamext[\cftalg{A}]}{\idtm{\jcomp{}{\cftfamext[\cftalg{A}]}{{}{\cftctxext[\cftalg{A}]}{\cftalgf{\cftalg{A}}}}}}}{\cftfamext[\cftalg{A}]}}
        }
      }
  \\
& \jdeq
  \subst{{y_2}{{y_1}{{y_0}{\jvcomp{}{\cftfamext[\cftalg{A}]}{\idtm{\jcomp{}{\cftfamext[\cftalg{A}]}{{}{\cftctxext[\cftalg{A}]}{\cftalgf{\cftalg{A}}}}}}}}}}{\cftfamext[\cftalg{A}]}
  \\
& \jdeq
  \subst{\tmext{\subst{y_1}{{y_0}{\cftfamext[\cftalg{A}]}}}{y_2}}{\cftfamext[\cftalg{A}]}
  \\
& \jdeq
  \subst{y_2}{{{y_1}{{y_0}{\cftfamext[\cftalg{A}]}}}{\cftfamext[\cftalg{A}]}}.\qedhere
\end{align*}
\end{proof}

There is a trivial class of examples of extension algebras we can give right
away. More examples will be introduced by universes, later on.

\begin{eg}
Let $A$ be a family in context $\Gamma$. Then the quadruple
\begin{equation*}
(A,\emptyf,\emptyf,\idtm{A},\emptytm)
\end{equation*}
is an extension algebra in context $\Gamma$, as is the quadruple
\begin{equation*}
(\emptyf,A,\emptyf,\emptytm,\idtm{A}).
\end{equation*}
Also, the quadruple
\begin{equation*}
(A,\ctxwk{A}{A},\emptyf,\cprojfstf{A}{\ctxwk{A}{A}},\cprojfstf{\ctxwk{A}}{\ctxwk{A}{{A}{A}}})
\end{equation*}
is an extension algebra in context $\Gamma$.
\end{eg}

As a consequence of the following theorem, every extension algebra gives rise
to infinitely many extension algebras by constructing the extension algebra
of families of $\cftalg{A}$ for each extension algebra $\cftalg{A}$. Notice
that this also explains that the family $\jcomp{}{\cftctxext}{\cftalgf{\cftalg{A}}}$
over $\ctxext{{\Gamma}{\cftalgc{\cftalg{A}}}}{\cftalgf{\cftalg{A}}}$ is the family
of families over families. Likewise, higher families are obtained by pulling
back more times along $\cftctxext[\cftalg{A}]$.

\begin{thm}\label{thm:extalg-fam}
Suppose that $\extalg{A}$ is an extension algebra in context
$\Gamma$. Then 
\begin{equation*}
\cftalg{F_A}
\defeq
( \cftalgf{\cftalg{A}},
  \jcomp{}{\cftctxext[\cftalg{A}]}{\cftalgf{\cftalg{A}}},
  \jcomp{}{\cftctxext[\cftalg{A}]}{\cftalgt{\cftalg{A}}},
  \cftfamext,
  \jcomp{}{\cftctxext[\cftalg{A}]}{\cftfamext})
\end{equation*}
is an extension algebra in context $\ctxext{\Gamma}{\cftalgc{\cftalg{A}}}$.
\end{thm}

\begin{proof}
In this proof, we shall use the short-hand notations $\cftctxext$ and $\cftfamext$
only to refer to $\cftctxext[\cftalg{A}]$ and $\cftfamext[\cftalg{A}]$, respectively,
and not to $\cftctxext[\cftalg{F_A}]$ or $\cftfamext[\cftalg{F_A}]$.

We first need to verify that the domain of the morphism 
$\jcomp{}{\cftctxext}{\cftfamext}$ is indeed
$\ctxext{\jcomp{}{\cftctxext}{\cftalgf{\cftalg{A}}}}{\jcomp{}{\cftfamext}{{}{\cftctxext}{\cftalgf{\cftalg{A}}}}}$. 
This follows from the judgmental equality
$\jcomp{}{\cftfamext}{{}{\cftctxext}{\cftalgf{\cftalg{A}}}}\jdeq
\jcomp{}{\cftctxext}{{}{\cftctxext}{\cftalgf{\cftalg{A}}}}$, which we have proved in
\autoref{lem:extalg-twins}. Notice how the diagram in \autoref{eq:extalg-eq2} is
of exactly the right sort, so the quadruple
$(\cftalgf{\cftalg{A}},\jcomp{}{\cftctxext}{\cftalgf{\cftalg{A}}},\cftfamext,\jcomp{}{\cftctxext}{\cftfamext})$
satisfies its version of \autoref{eq:extalg-eq1}. It is left to verify that the diagram
\begin{small}
\begin{equation*}
\begin{tikzcd}[column sep=huge]
\ctxext
  { \jcomp{}{\cftctxext}{\cftalgf{\cftalg{A}}}
    }
  { { \jcomp{}{\cftfamext}{%
        \jcomp{}{\cftctxext}{\cftalgf{\cftalg{A}}}
        }
      }
    { \jcomp{}{\cftfamext}{%
        \jcomp{}{\cftfamext}{%
          \jcomp{}{\cftctxext}{\cftalgf{\cftalg{A}}}
          }
        }
      }
    } 
  \ar{r}
    { \jvcomp{}{\idtm{\jcomp{}{\cftctxext}{\cftalgf{\cftalg{A}}}}}{%
        \jcomp{}{\cftfamext}{%
          \jcomp{}{\cftctxext}{\cftfamext}}}}
  \ar{d}[swap]{
    \jvcomp{}{\jcomp{}{\cftctxext}{\cftfamext}}{%
      \idtm{
        \jcomp{}{\jcomp{}{\cftctxext}{\cftfamext}}{%
          \jcomp{}{\cftfamext}{%
            \jcomp{}{\cftctxext}{\cftalgf{\cftalg{A}}}
            }
          }
        }
      }
    }
& \ctxext
    {\jcomp{}{\cftctxext}{\cftalgf{\cftalg{A}}}}
    {\jcomp{}{\cftfamext}{\jcomp{}{\cftctxext}{\cftalgf{\cftalg{A}}}}} 
  \ar{d}{\jcomp{}{\cftctxext}{\cftfamext}}
  \\
\ctxext
  {\jcomp{}{\cftctxext}{\cftalgf{\cftalg{A}}}}
  {\jcomp{}{\cftfamext}{\jcomp{}{\cftctxext}{\cftalgf{\cftalg{A}}}}} 
  \ar{r}[swap]{\jcomp{}{\cftctxext}{\cftfamext}} 
& \jcomp{}{\cftctxext}{\cftalgf{\cftalg{A}}}
\end{tikzcd}
\end{equation*}
\end{small}%
commutes judgmentally; this diagram is the version of \autoref{eq:extalg-eq2}
for the quadruple
$(\cftalgf{\cftalg{A}},\jcomp{}{\cftctxext}{\cftalgf{\cftalg{A}}},\cftfamext,\jcomp{}{\cftctxext}{\cftfamext})$. Note
that this follows from \autoref{eq:extalg-eq2} provided that we can show that
\begin{align}
\jvcomp{}{\idtm{\jcomp{}{\cftctxext}{\cftalgf{\cftalg{A}}}}}{%
  \jcomp{}{\cftfamext}{%
    \jcomp{}{\cftctxext}{\cftfamext}}}
& \jdeq
  \jcomp{}{\cftctxext}{%
    \jvcomp{}{\idtm{\cftalgf{\cftalg{A}}}}{\jcomp{}{\cftctxext}{\cftfamext}}
    }
  \label{eq:extalg-infty1}
  \\
\jvcomp{}{\jcomp{}{\cftctxext}{\cftfamext}}{%
  \idtm{
    \jcomp{}{\jcomp{}{\cftctxext}{\cftfamext}}{%
      \jcomp{}{\cftfamext}{%
        \jcomp{}{\cftctxext}{\cftalgf{\cftalg{A}}}
        }
      }
    }
  }
& \jdeq
\jcomp{}{\cftctxext}{%
  \jvcomp{}{\cftfamext}{%
    \idtm{
      \jcomp{}{\cftfamext}{%
        \jcomp{}{\cftctxext}{\cftalgf{\cftalg{A}}}
        }
      }
    }
  }
  \label{eq:extalg-infty2}
\end{align}
Note that \autoref{eq:extalg-infty1} follows if we can show that
\begin{equation*}
\jcomp{}{\cftfamext}{\jcomp{}{\cftctxext}{\cftfamext}}
  \jdeq
  \jcomp{}{\cftctxext}{\jcomp{}{\cftctxext}{\cftfamext}}.
\end{equation*}
This is a special case of \autoref{lem:extalg-twins}. The second judgmental
equality, \autoref{eq:extalg-infty2}, is trivial.
\end{proof}

\begin{thm}\label{thm:extalg-wk}
Let $\extalg{Q}$ be an extension algebra in context $\ctxext{\Gamma}{B}$ and let
$\jfam{\Gamma}{A}$ be a family of contexts. Then the quintuple
\begin{equation*}
\ctxwk{A}{\extalg{Q}}
  \defeq
  ( \ctxwk{A}{\cftalgc{\cftalg{Q}}},
    \ctxwk{A}{\cftalgf{\cftalg{Q}}},
    \ctxwk{A}{\cftalgt{\cftalg{Q}}},
    \ctxwk{A}{\cftctxext[\cftalg{Q}]},
    \ctxwk{A}{\cftfamext[\cftalg{Q}]})
\end{equation*}
is an extension algebra in context $\ctxext{{\Gamma}{A}}{\ctxwk{A}{B}}$.
\end{thm}

\begin{proof}
The proof follows from the fact that weakening by $A$ is compatible with all
the involved operations.
\end{proof}

\begin{thm}\label{thm:extalg-subst}
Let $\extalg{Q}$ be an extension algebra in context $\ctxext{{\Gamma}{A}}{P}$
and let $\jterm{\Gamma}{A}{x}$ be a term. Then the quintuple
\begin{equation*}
\subst{x}{\cftalg{Q}}
  \defeq
  ( \subst{x}{\cftalgc{\cftalg{Q}}},
    \subst{x}{\cftalgf{\cftalg{Q}}},
    \subst{x}{\cftalgt{\cftalg{Q}}},
    \subst{x}{\cftctxext[\cftalg{Q}]},
    \subst{x}{\cftfamext[\cftalg{Q}]})
\end{equation*}
is an extension algebra in context $\ctxext{\Gamma}{\subst{x}{P}}$.
\end{thm}

\begin{proof}
The proof follows from the fact that substitution with $x$ is compatible with
all the involved operations.
\end{proof}

\begin{cor}
Let $\extalg{Q}$ be an extension algebra in context $\ctxext{\Gamma}{B}$
and let $\jhom{\Gamma}{A}{B}{f}$. Then the quintuple
\begin{equation*}
\jcomp{A}{f}{\extalg{Q}}
  \defeq
  ( \jcomp{A}{f}{\cftalgc{\cftalg{Q}}},
    \jcomp{A}{f}{\cftalgf{\cftalg{Q}}},
    \jcomp{A}{f}{\cftalgt{\cftalg{Q}}},
    \jcomp{A}{f}{\cftctxext[\cftalg{Q}]},
    \jcomp{A}{f}{\cftfamext[\cftalg{Q}]})
\end{equation*}
is an extension algebra in context $\ctxext{\Gamma}{A}$.
\end{cor}

\begin{cor}
Let $\extalg{A}$ be an extension algebra in context $\Gamma$. Then
$\extalg{F_{F_A}}$ and $\jcomp{}{\cftctxext[\extalg{A}]}{\extalg{F_A}}$ are
judgmentally equal extension algebras in context $\ctxext{{\Gamma}{\cftalgc{\cftalg{A}}}}{\cftalgf{\cftalg{A}}}$.
\end{cor}

%%%%%%%%%%%%%%%%%%%%%%%%%%%%%%%%%%%%%%%%%%%%%%%%%%%%%%%%%%%%%%%%%%%%%%%%%%%%%%%%
\subsection{Extension homomorphisms}

\begin{defn}
An \emph{extension homomorphism $\cfthom{f}$ from $\extalg{A}$ to
$\extalg{B}$ in context $\Gamma$} is a triple $\unfold{\cfthom{f}}$ consisting of
\begin{align*}
\jhom*
  {\Gamma}
  {\cftalgc{\cftalg{A}}}
  {\cftalgc{\cftalg{B}}}
  {\cfthomc{\cfthom{f}}}
  \\
\jfhom*
  {\Gamma}
  {\cftalgc{\cftalg{A}}}
  {\cftalgc{\cftalg{B}}}
  {\cfthomc{\cfthom{f}}}
  {\cftalgf{\cftalg{A}}}
  {\cftalgf{\cftalg{B}}}
  {\cfthomf{\cfthom{f}}}
  \\
\jfhom*
  {\Gamma}
  {{\cftalgc{\cftalg{A}}}{\cftalgf{\cftalg{A}}}}
  {{\cftalgc{\cftalg{B}}}{\cftalgf{\cftalg{B}}}}
  {\jvcomp{}{\cfthomc{\cfthom{f}}}{\cfthomf{\cfthom{f}}}}
  {\cftalgt{\cftalg{A}}}
  {\cftalgt{\cftalg{B}}}
  {\cfthomt{\cfthom{f}}}
\end{align*}
for which the diagrams
\begin{equation}\label{eq:exthom1}
\begin{tikzcd}
\ctxext{\cftalgc{\cftalg{A}}}{\cftalgf{\cftalg{A}}}
  \ar{r}{\jvcomp{}{\cfthomc{\cfthom{f}}}{\cfthomf{\cfthom{f}}}}
  \ar{d}[swap]{\cftctxext[\cftalg{A}]}
& \ctxext{\cftalgc{\cftalg{B}}}{\cftalgf{\cftalg{B}}}
  \ar{d}{\cftctxext[\cftalg{B}]}
  \\
\cftalgc{\cftalg{A}}
  \ar{r}[swap]{\cfthomc{\cfthom{f}}}
& \cftalgc{\cftalg{A}}
\end{tikzcd}
\end{equation}
and
\begin{equation}\label{eq:exthom2}
\begin{tikzcd}[column sep=huge]
\ctxext{\cftalgf{\cftalg{A}}}{\jcomp{}{\cftctxext[\cftalg{A}]}{\cftalgf{\cftalg{A}}}}
  \ar{r}{\jvcomp{}{\cfthomf{\cfthom{f}}}{\jcomp{}{\cftctxext[\cftalg{A}]}{\cfthomf{\cfthom{f}}}}}
  \ar{d}[swap]{\cftfamext[\cftalg{A}]}
& \jcomp{}{\cfthomc{\cfthom{f}}}{\ctxext{\cftalgf{\cftalg{B}}}{\jcomp{}{\cftctxext[\cftalg{B}]}{\cftalgf{\cftalg{B}}}}}
  \ar{d}{\jcomp{}{\cfthomc{\cfthom{f}}}{\cftfamext[\cftalg{B}]}}
  \\
\cftalgf{\cftalg{A}}
  \ar{r}[swap]{\cfthomf{\cfthom{f}}}
& \jcomp{}{\cfthomc{\cfthom{f}}}{\cftalgf{\cftalg{B}}}
\end{tikzcd}
\end{equation}
commute judgmentally.
\end{defn}

\begin{rmk}
To see that the upper morphism in the diagram of \autoref{eq:exthom2} has
indeed the indicated codomain provided that the diagram of \autoref{eq:exthom1}
commutes judgmentally, note that we have the judgmental equalities
\begin{align*}
\jcomp{}{\cfthomf{\cfthom{f}}}{\jcomp{}{\cfthomc{\cfthom{f}}}{\jcomp{}{\cftctxext[\cftalg{B}]}{\cftalgf{\cftalg{B}}}}}
& \jdeq 
  \jcomp{}{\jvcomp{}{\cfthomc{\cfthom{f}}}{\cfthomf{\cfthom{f}}}}{\jcomp{}{\cftctxext[\cftalg{B}]}{\cftalgf{\cftalg{B}}}}
  \tag{by \autoref{lem:composition-threesome}}
  \\
& \jdeq
  \jcomp{}{\jcomp{}{\jvcomp{}{\cfthomc{\cfthom{f}}}{\cfthomf{\cfthom{f}}}}{\cftctxext[\cftalg{B}]}}{\cftalgf{\cftalg{B}}}
  \tag{by \autoref{lem:jcomp-jcomp}}
  \\
& \jdeq
  \jcomp{}{\jcomp{}{\cftctxext[\cftalg{A}]}{\cfthomc{\cfthom{f}}}}{\cftalgf{\cftalg{B}}}
  \tag{by \autoref{eq:exthom1}}
  \\
& \jdeq
  \jcomp{}{\cftctxext[\cftalg{A}]}{\jcomp{}{\cfthomc{\cfthom{f}}}{\cftalgf{\cftalg{B}}}}.
  \tag{by \autoref{lem:jcomp-jcomp}}
\end{align*}
and we indeed have the morphism $\jcomp{}{\cftctxext[\cftalg{A}]}{\cfthomf{\cfthom{f}}}$ from 
$\jcomp{}{\cftctxext[\cftalg{A}]}{\cftalgf{\cftalg{A}}}$ to $\jcomp{}{\cftctxext[\cftalg{A}]}{\jcomp{}{\cfthomc{\cfthom{f}}}{\cftalgf{\cftalg{B}}}}$.
\end{rmk}

\begin{thm}
Let $\cfthom{f}$ be an extension homomorphism from $\cftalg{A}$ to $\cftalg{B}$
in context $\Gamma$. Then
\begin{equation*}
\boldsymbol{\mathcal{F}}_\cfthom{f}
  \defeq
  ( \cfthomf{\cfthom{f}},
    \jcomp{}{\cftctxext[\cftalg{A}]}{\cfthomf{\cfthom{f}}},
    \jcomp{}{\cftctxext[\cftalg{A}]}{\cfthomt{\cfthom{f}}})
\end{equation*}
is an extension homomorphism from $\cftalg{F_A}$ to $\jcomp{}{\cfthomc{\cfthom{f}}}{\cftalg{F_B}}$
in context $\ctxext{\Gamma}{\cftalgc{\cftalg{A}}}$. 
\end{thm}

\begin{proof}
We have to verify that the diagrams in \autoref{eq:exthom1,eq:exthom2}
commute for $\cfthom{F_f}$. Unfolding the ingredients of \autoref{eq:exthom1}
for $\cfthom{F_f}$ gives us quite directly \autoref{eq:exthom2}. The
diagram in \autoref{eq:exthom2} for $\cfthom{F_f}$ is judgmentally equal
to the pullback of everything in the diagram in \autoref{eq:exthom2} for
$\cfthom{f}$ by $\epsilon_0$, and therefore it commutes too.
\end{proof}

\begin{defn}
Let $\cfthom{f}$ and $\cfthom{g}$ be extension homomorphisms from
$\extalg{A}$ to $\extalg{B}$ and from $\extalg{B}$ to $\extalg{C}$, respectively.
We define the composition
\begin{equation*}
\cfthomcomp{\cfthom{f}}{\cfthom{g}}
  \defeq
  \unfold{\cfthomcomp{\cfthom{f}}{\cfthom{g}}}.
\end{equation*}
In other words, extension homomorphisms are composed by taking the horizontal 
rectangles in the diagram
\begin{equation*}
\begin{tikzcd}
\extalgt{\extalg{A}} 
  \ar[fib]{d}
  \ar{r}{\cfthomt{\cfthom{f}}}
& \extalgt{\extalg{B}} 
  \ar[fib]{d}
  \ar{r}{\cfthomt{\cfthom{g}}}
& \extalgt{\extalg{C}}
  \ar[fib]{d}
  \\
\extalgf{\extalg{A}} 
  \ar[fib]{d}
  \ar{r}{\cfthomf{\cfthom{f}}}
& \extalgf{\extalg{B}} 
  \ar[fib]{d}
  \ar{r}{\cfthomf{\cfthom{g}}}
& \extalgf{\extalg{C}}
  \ar[fib]{d}
  \\
\extalgc{\extalg{A}}
  \ar{r}{\cfthomc{\cfthom{f}}}
& \extalgc{\extalg{B}}
  \ar{r}{\cfthomc{\cfthom{g}}}
& \extalgc{\extalg{C}}
\end{tikzcd}
\end{equation*}
\end{defn}

\begin{rmk}
It follows from \autoref{lem:jcomp-jcomp,lem:jfcomp-jfcomp} that composition
of extension homomorphisms is associative.
\end{rmk}

\begin{thm}
Let $\cfthom{f}$ and $\cfthom{g}$ be extension homomorphisms from
$\extalg{A}$ to $\extalg{B}$ and from $\extalg{B}$ to $\extalg{C}$, respectively.
Then $\cfthomcomp{\cfthom{f}}{\cfthom{g}}$ is an extension homomorphism from
$\extalg{A}$ to $\extalg{C}$. 
\end{thm}

\begin{proof}
Both judgmental equalities are applications of the interchange law for composition,
\autoref{lem:composition-interchange}. 
\end{proof}

The following theorem explains how context extension can be seen as an extension
homomorphism. Note that in combination with \autoref{thm:extalg-fam}, this also
explains how family extension can be seen as an extension homomorphism.

\begin{thm}
Let $\extalg{A}$ be an extension algebra in context $\Gamma$. Then
\begin{align*}
& ( \ctxext{\extalgc{\extalg{A}}}{\extalgf{\extalg{A}}},
    \jcomp{}{\cftctxext}{\extalgf{\extalg{A}}},
    \jvcomp{}{\ctxwk{\extalgf{\extalg{A}}}{\idtm{\extalgc{\extalg{A}}}}}{\cftfamext},
    \jcomp{}{\cftctxext}{\cftfamext}
    )
\intertext{and}
& ( \ctxext{\extalgc{\extalg{A}}}{\extalgf{\extalg{A}}},
    \jcomp{}{\cftctxext}{\extalgf{\extalg{A}}},
    \jvcomp{}{\cftctxext}{\idtm{\jcomp{}{\cftctxext}{\extalgf{\extalg{A}}}}},
    \jcomp{}{\cftctxext}{\cftfamext}
    )
\end{align*}
are extension algebras in context $\Gamma$ and
\begin{equation*}
\boldsymbol{\cftctxext}\defeq ( \cftctxext,
  \idtm{\jcomp{}{\cftctxext}{\extalgf{\extalg{A}}}}
  )
\end{equation*}
is an extension homomorphism from both of them to $\extalg{A}$. 
\end{thm}

\begin{comment}
\begin{rmk}
I suspect that if we copy this theory of extension algebras to Martin-L\"of
type theory, with the judgmental equalities replaced by identifications, with
dependent pair types rather than those strict extensions, etcetera, then
the type of $f_1$ for which these two diagrams commute is a mere proposition.

With this notion of morphism, a term of an extension
algebra $(A,P,\epsilon_0,\epsilon_1)$ is a pair $(x_0,x_1)$ such that
$\subst{x_1}{{x_0}{\epsilon_0}}\jdeq x_0$.
\end{rmk}
\end{comment}

\begin{comment}
Extension algebras don't come in isolation. There are also extension algebra
families and extension algebra terms. We now aim to define these and to
establish various constructions of new extension algebras out of old ones:
the empty extension algebra, and extensions, weakenings and substitutions
of extension algebras and of course the identity term as an extension algebra
term. We start with extension algebra families.

\begin{defn}
Consider an extension algebra $\mathcal{A}\defeq(A,P,e,f)$. 
An extension algebra family over $\mathcal{A}$ is likewise a quadruple
$\mathcal{B}\defeq(B,Q,g,h)$. Here we have a family $\jfam{{\Gamma}{A}}{B}$, a
family $\jfam{{{{\Gamma}{A}}{P}}{\ctxwk{P}{B}}}{Q}$ and
\begin{align*}
\jhom*{{{\Gamma}{A}}{P}}{\ctxext{\ctxwk{P}{B}}{Q}}{\jcomp{}{\epsilon_0}{B}}{g}\\
\jhom*{{{{\Gamma}{A}}{P}}{\ctxwk{P}{B}}}{\ctxext{Q}{\jcomp{}{g}{Q}}}{Q}{h}.
\end{align*}
The quadruple $(\jcomp{}{\epsilon_0}{B},Q,g,h)$ is required to be an extension algebra
in context $\ctxext{{\Gamma}{A}}{P}$.
\end{defn}

\begin{defn}
Suppose $\mathcal{A}$ is an extension algebra and $\mathcal{B}$ is an extension
algebra family over $\mathcal{A}$. A term of $\mathcal{B}$ is a pair $(x,y)$
consisting of
\begin{align*}
\jterm*{{\Gamma}{A}}{B}{x}\\
\jterm*{{{\Gamma}{A}}{P}}{\subst{\jcomp{}{\epsilon_0}{x}}{Q}}{y}
\end{align*}
such that the diagrams
\begin{equation*}
\begin{tikzcd}
\ctxext{\jcomp{}{\epsilon_0}{B}}{Q} 
  \ar{r}{g} 
  \ar[shift right=.7ex,fib]{d}
& B 
  \ar[shift right=.7ex,fib]{d} 
  \\
\ctxext{A}{P} 
  \ar[shift right=.7ex,dotted]{u}[swap]{\tmext{\jcomp{}{\epsilon_0}{x}}{y}}
  \ar{r}{e}
& A
  \ar[shift right=.7ex,dotted]{u}[swap]{x}
\end{tikzcd}
\end{equation*}
and
\begin{equation*}
\begin{tikzcd}
\jcomp{}{f}{\ctxext{Q}{\jcomp{}{g}{Q}}}
  \ar{r}{\jcomp{}{f}{h}}
  \ar[shift right=.7ex,fib]{d}
& Q
  \ar[shift right=.7ex,fib]{d}
  \\
\jcomp{}{f}{\jcomp{}{\epsilon_0}{B}}
  \ar{r}{\idtm{\jcomp{}{f}{\jcomp{}{\epsilon_0}{B}}}}
  \ar[shift right=.7ex,fib]{d}
  \ar[shift right=.7ex,dotted,mapsto]{u}[swap]{\jcomp{}{f}{y}}
& \jcomp{}{\epsilon_0}{B}
  \ar[shift right=.7ex,fib]{d}
  \ar[shift right=.7ex,dotted,mapsto]{u}[swap]{y}
  \\
\ctxext{P}{\jcomp{}{\epsilon_0}{P}}
  \ar{r}[swap]{f}
  \ar[shift right=.7ex,dotted]{u}[swap]{\jcomp{}{f}{\jcomp{}{\epsilon_0}{x}}}
& P
  \ar[shift right=.7ex,dotted]{u}[swap]{\jcomp{}{\epsilon_0}{x}}
\end{tikzcd}
\end{equation*}
commute.
\end{defn}

\begin{defn}
Suppose $\mathcal{A}$ and $\mathcal{B}$ are extension algebras in context
$\Gamma$. We define the extension algebra $\ctxwk{\mathcal{A}}{\mathcal{B}}$
to be the quadruple
\begin{equation*}
(\ctxwk{A}{B},\ctxwk{\ctxext{A}{P}}{Q},\ctxwk{\ctxext{A}{P}}{g},\ctxwk{\ctxext{A}{P}}{h}).
\end{equation*}
Note that $\ctxwk{\ctxext{A}{P}}{Q}$ is a family over $\ctxwk{\ctxext{A}{P}}{B}$,
whereas it should be a family over $\jcomp{}{\epsilon_0}{\ctxwk{A}{B}}$. These are the
same by \autoref{lem:prehom}.
\end{defn}

\begin{rmk}
Before we continue, let us explore what it means to be an extension algebra
term of the extension algebra $\ctxwk{\mathcal{A}}{\mathcal{B}}$. Such an
extension algebra term $(x,y)$ would consist of
\begin{align*}
\jterm*{{\Gamma}{A}}{\ctxwk{A}{B}}{x}\\
\jterm*{{{\Gamma}{A}}{P}}{\subst{\jcomp{}{\epsilon_0}{x}}{\ctxwk{\ctxext{A}{P}}{Q}}}{y}.
\end{align*}
Thus, $x$ is a context morphism from $A$ to $B$ and $y$ is nothing but a term
of $\jcomp{}{\jcomp{}{\epsilon_0}{x}}{Q}$. For $x$, we see that the diagram
\begin{equation*}
\begin{tikzcd}
\ctxext{B}{Q} 
  \ar{r}{g} 
& B 
  \\
\ctxext{A}{P} 
  \ar{u}{\jvcomp{}{x}{y}}
  \ar{r}{e}
& A
  \ar{u}[swap]{x}
\end{tikzcd}
\end{equation*}
commutes.
\end{rmk}
\end{comment}

%%%%%%%%%%%%%%%%%%%%%%%%%%%%%%%%%%%%%%%%%%%%%%%%%%%%%%%%%%%%%%%%%%%%%%%%%%%%%%%%
\subsection{CFT-algebras}
The notion of CFT-algebras that we will
study in this subsection will also have an empty object $(\cftempc{\cftalg{A}},
\cftempf{\cftalg{A}})$ and guarantiees that families over the empty object
are just the terms of the underlying family $\cftalgc{\cftalg{A}}$ of contexts
of the term-algebra $\cftalg{A}$. We will begin with an auxilary lemma which
will help us dealing with the internal empty contexts:

\begin{lem}\label{lem:empalg-mor}
Consider a family $\jfam{\Gamma}{A}$ with a term $\jterm{\Gamma}{A}{x}$ and 
a family $\jfam{{\Gamma}{A}}{P}$ % with a term $\jterm{{\Gamma}{A}}{P}{y}$
satisfying the judgmental equality
\begin{align}
\jfameq*{\Gamma}{\subst{x}{P}}{A}
  \label{eq1:empalg-mor}
%  \\
%\jtermeq*{\Gamma}{A}{\subst{x}{y}}{x}.
%  \label{eq2:empalg-mor}
\end{align}
Consider also a morphism
\begin{equation*}
\jhom{\Gamma}{{A}{P}}{B}{f}.
\end{equation*}
Then $\subst{\xi}{f}$ is a morphism from $A$ to $B$ 
in context $\Gamma$ and the following inference rules are valid for $\xi$ being 
$x$ or $y$:
\begin{align*}
& \inference
  { \jfam{{\Gamma}{B}}{Q}
    }
  { \jfameq
      {{\Gamma}{A}}
      {\subst{\xi}{\jcomp{}{f}{Q}}}
      {\jcomp{}{\subst{\xi}{f}}{Q}}
    }
  \\
& \inference
  { \jfam{{{\Gamma}{B}}{Q}}{R}
    }
  { \jfameq
      {{{\Gamma}{A}}{\jcomp{}{\subst{\xi}{f}}{Q}}}
      {\subst{\xi}{\jcomp{}{f}{R}}}
      {\jcomp{}{\subst{\xi}{f}}{R}}
    }
  \\
& \inference
  { \jterm{{{\Gamma}{B}}{Q}}{R}{h}
    }
  { \jtermeq
      {{{\Gamma}{A}}{\jcomp{}{\subst{\xi}{f}}{Q}}}
      {\jcomp{}{\subst{\xi}{f}}{R}}
      {\subst{\xi}{\jcomp{}{f}{h}}}
      {\jcomp{}{\subst{\xi}{f}}{h}}
    }
\end{align*}
\end{lem}

\begin{proof}
We only prove the first inference rule in both cases.
Let $Q$ be a family over $\ctxext{\Gamma}{\cftalgc{\cftalg{A}}}$. In the case 
where $\xi$ is $x$ we have the judgmental equalities
\begin{align*}
\subst{x}{\jcomp{}{f}{Q}}
& \jdeq
  \subst
    {x}
    {{f}{\ctxwk{\ctxext{\cftalgc{\cftalg{A}}}{\cftalgf{\cftalg{A}}}}{Q}}}
  \tag{by definition}
  \\
& \jdeq
  \subst
    { {x}{f}
      }
    { {x}
      {\ctxwk{\ctxext{\cftalgc{\cftalg{A}}}{\cftalgf{\cftalg{A}}}}{Q}}
      }
  \tag{by \autoref{comp-ss-f}}
  \\
& \jdeq
  \subst
    { {x}{f}
      }
    { {x}
      {\ctxwk{\cftalgf{\cftalg{A}}}{{\cftalgc{\cftalg{A}}}{Q}}}
      }
  \tag{by \autoref{comp-ew-f}}
  \\
& \jdeq
  \subst
    { {x}{f}
      }
    { \ctxwk
        {\subst{x}{\cftalgf{\cftalg{A}}}}
        {\subst{x}{\ctxwk{\cftalgc{\cftalg{A}}}{Q}}}
      }
  \tag{by \autoref{comp-sw-f}}
  \\
& \jdeq
  \subst
    { {x}{f}
      }
    { \ctxwk
        {\subst{x}{\cftalgf{\cftalg{A}}}}
        {Q}
      }
  \tag{by \autoref{cancellation-ws-f}}
  \\
& \jdeq
  \subst{{x}{f}}{\ctxwk{\cftalgc{\cftalg{A}}}{Q}}
  \tag{by \autoref{eq1:empalg-mor}}
  \\
& \jdeq
  \jcomp{}{\subst{x}{f}}{Q}.
  \tag{by definition}
\end{align*}
In the case where $\xi$ is $y$ we have the judgmental equalities
\begin{align*}
\subst{y}{\jcomp{}{f}{Q}}
& \jdeq
  \subst
    { y
      }
    { {f}
      {\ctxwk{\ctxext{\cftalgc{\cftalg{A}}}{\cftalgf{\cftalg{A}}}}{Q}}
      }
  \tag{by definition}
  \\
& \jdeq
  \subst
    { {y}{f}
      }
    { {y}
      {\ctxwk{\ctxext{\cftalgc{\cftalg{A}}}{\cftalgf{\cftalg{A}}}}{Q}}
      }
  \tag{by \autoref{comp-ss-f}}
  \\
& \jdeq
  \subst
    { {y}{f}
      }
    { {y}
      {\ctxwk{\cftalgf{\cftalg{A}}}{{\cftalgc{\cftalg{A}}}{Q}}}
      }
  \tag{by \autoref{comp-ew-f}}
  \\
& \jdeq
  \subst
    { {y}{f}
      }
    { \ctxwk{\cftalgc{\cftalg{A}}}{Q}
      }
  \tag{by \autoref{cancellation-ws-f}}
  \\
& \jdeq
  \jcomp{}{\subst{y}{f}}{Q}.
  \tag{by definition}
\end{align*}
\end{proof}

\begin{defn}
A \emph{CFT-algebra $\cftalg{A}$ in context $\Gamma$} is a septuple
\begin{equation*}
\unfold{\cftalg{A}}
\end{equation*}
where the quintuple $\unfold{\extalg{A}}$ is an extension
algebra in context $\Gamma$, and where
\begin{align*}
\jterm*{\Gamma}{\cftalgc{\cftalg{A}}}{\cftempc{\cftalg{A}}}
  \\
\jterm*{{\Gamma}{\cftalgc{\cftalg{A}}}}{\cftalgf{\cftalg{A}}}{\cftempf{\cftalg{A}}}
\end{align*}
which satisfy the following judgmental equalities
\begin{enumerate}
\item Families over the empty context in $\cftalg{A}$ are contexts of $\cftalg{A}$:
\begin{align}
\jfameq*
  {\Gamma}
  {\subst{\cftempc{\cftalg{A}}}{\cftalgf{\cftalg{A}}}}
  {\cftalgc{\cftalg{A}}}
  \label{empalg-eq1}
\end{align}
\item The empty family over the empty context of $\cftalg{A}$ is the empty context of $\cftalg{A}$:
\begin{align}
\jtermeq*
  {\Gamma}
  {\cftalgc{\cftalg{A}}}
  {\subst{\cftempc{\cftalg{A}}}{\cftempf{\cftalg{A}}}}
  {\cftempc{\cftalg{A}}}.
  \label{empalg-eq2}
\end{align}
\item Context extension of $\cftalg{A}$ is compatible with the empty context and
family of $\cftalg{A}$:
\begin{align}
\jtermeq*
  {{\Gamma}{\cftalgc{\cftalg{A}}}}
  {\ctxwk{\cftalgc{\cftalg{A}}}{\cftalgc{\cftalg{A}}}}
  {\subst{\cftempc{\cftalg{A}}}{\cftctxext[\cftalg{A}]}}
  {\idtm{\cftalgc{\cftalg{A}}}}
  \label{cftalg-eq1}
  \\
\jtermeq*
  {{\Gamma}{\cftalgc{\cftalg{A}}}}
  {\ctxwk{\cftalgc{\cftalg{A}}}{\cftalgc{\cftalg{A}}}}
  {\subst{\cftempf{\cftalg{A}}}{\cftctxext[\cftalg{A}]}}
  {\idtm{\cftalgc{\cftalg{A}}}}
  \label{cftalg-eq2}
\end{align}
\item Family extension of $\cftalg{A}$ is compatible with the empty family and the empty family
over families of $\cftalg{A}$:
\begin{align}
\jtermeq*
  {{{\Gamma}{\cftalgc{\cftalg{A}}}}{\cftalgf{\cftalg{A}}}}
  {\ctxwk{\cftalgf{\cftalg{A}}}{\cftalgf{\cftalg{A}}}}
  {\subst{\cftempf{\cftalg{A}}}{\cftfamext}}
  {\idtm{\cftalgf{\cftalg{A}}}}
  \label{cftalg-eq3}
  \\
\jtermeq*
  {{{\Gamma}{\cftalgc{\cftalg{A}}}}{\cftalgf{\cftalg{A}}}}
  {\ctxwk{\cftalgf{\cftalg{A}}}{\cftalgf{\cftalg{A}}}}
  {\subst{\jcomp{}{\cftctxext}{\cftempf{\cftalg{A}}}}{\cftfamext}}
  {\idtm{\cftalgf{\cftalg{A}}}}
  \label{cftalg-eq4}
\end{align}
\item Family extension over the empty context of $\cftalg{A}$ is context extension
of $\cftalg{A}$:
\begin{align}
\jhomeq*
  {\Gamma}
  {{\cftalgc{\cftalg{A}}}{\cftalgf{\cftalg{A}}}}
  {\cftalgc{\cftalg{A}}}
  {\subst{\cftempc{\cftalg{A}}}{\cftfamext}}
  {\cftctxext}
  \label{cftalg-eq5}
\end{align}
\end{enumerate}
\end{defn}

\begin{rmk}\label{rmk:cftalg-defn}
We need to verify that the judgmental equalities
\autoref{cftalg-eq1,cftalg-eq2,cftalg-eq3,cftalg-eq4,cftalg-eq5}
are indeed well-typed (i.e.~compare two terms of the same type).
\begin{enumerate}
\item \autoref{lem:empalg-mor} implies that the morphisms 
$\subst{\cftempc{\cftalg{A}}}{\cftctxext}$ and
$\subst{\cftempf{\cftalg{A}}}{\cftctxext}$ both go from $\cftalgc{\cftalg{A}}$ 
to $\cftalgc{\cftalg{A}}$.
\item \label{rmkenum:famfamempf-jdeq-fam}
Now we have the judgmental equalities
\begin{align*}
\subst{\cftempf{\cftalg{A}}}{\jcomp{}{\cftctxext}{\cftalgf{\cftalg{A}}}}
& \jdeq
  \jcomp{}{\subst{\cftempf{\cftalg{A}}}{\cftctxext}}{\cftalgf{\cftalg{A}}}
  \tag{by \autoref{lem:empalg-mor}}
  \\
& \jdeq
  \jcomp{}{\idtm{\cftalgc{\cftalg{A}}}}{\cftalgf{\cftalg{A}}}
  \tag{by \autoref{cftalg-eq1}}
  \\
& \jdeq
  \cftalgf{\cftalg{A}}
  \tag{by \autoref{precomp-idtm-c}}
\end{align*}
Therefore we can apply \autoref{lem:empalg-mor} with the triple
$(\cftalgf{\cftalg{A}},\jcomp{}{\cftctxext}{\cftalgf{\cftalg{A}}},
\cftempf{\cftalg{A}})$ and the morphism $\cftfamext$ to see that both 
$\subst{\cftempf{\cftalg{A}}}{\cftfamext}$ and
$\subst{\jcomp{}{\cftctxext}{\cftempf{\cftalg{A}}}}{\cftfamext}$
are morphisms from $\cftalgf{\cftalg{A}}$ to $\cftalgf{\cftalg{A}}$.
\item The morphism $\subst{\cftempc{\cftalg{A}}}{\cftfamext}$ goes from
$ \subst
    {\cftempc{\cftalg{A}}}
    {\ctxext{\cftalgf{\cftalg{A}}}{\jcomp{}{\cftctxext}{\cftalgf{\cftalg{A}}}}}
  $
to $\subst{\cftempc{\cftalg{A}}}{\cftalgf{\cftalg{A}}}$. For the domain we
have the judgmental equalities
\begin{align*}
  \subst
    {\cftempc{\cftalg{A}}}
    {\ctxext{\cftalgf{\cftalg{A}}}{\jcomp{}{\cftctxext}{\cftalgf{\cftalg{A}}}}}
& \jdeq
  \ctxext
    {\subst{\cftempc{\cftalg{A}}}{\cftalgf{\cftalg{A}}}}
    {\subst{\cftempc{\cftalg{A}}}{\jcomp{}{\cftctxext}{\cftalgf{\cftalg{A}}}}}
  \tag{by \autoref{comp-se-c}}
  \\
& \jdeq
  \ctxext
    {\cftalgc{\cftalg{A}}}
    {\subst{\cftempc{\cftalg{A}}}{\jcomp{}{\cftctxext}{\cftalgf{\cftalg{A}}}}}
  \tag{by \autoref{empalg-eq1}}
  \\
& \jdeq
  \ctxext
    {\cftalgc{\cftalg{A}}}
    {\cftalgf{\cftalg{A}}}
  \tag{by \autoref{rmkenum:famfamempf-jdeq-fam} above}
\end{align*}
The codomain of $\subst{\cftempc{\cftalg{A}}}{\cftfamext}$
is judgmentally equal to $\cftalgc{\cftalg{A}}$ by
\autoref{empalg-eq1}.
\end{enumerate}
\end{rmk}

\begin{comment}
\begin{defn}
A quintuple $(\cftalgc{\cftalg{A}},\cftalgf{\cftalg{A}},\cftalgt{\cftalg{A}},
\cftempc{\cftalg{A}},\cftempf{\cftalg{A}})$ consisting of
\begin{enumerate}
\item A family $\cftalgc{\cftalg{A}}$ in context $\Gamma$,
\item A family $\cftalgf{\cftalg{A}}$ in context $\ctxext{\Gamma}{\cftalgc{\cftalg{A}}}$,
\item A family $\cftalgt{\cftalg{A}}$ in context $\ctxext{{\Gamma}{A}}{\cftalgf{\cftalg{A}}}$,
\item A term $\cftempc{\cftalg{A}}$ of $\cftalgc{\cftalg{A}}$ in context $\Gamma$,
\item A term $\cftempf{\cftalg{A}}$ of $\cftalgf{\cftalg{A}}$ in context $\ctxext{\Gamma}{\cftalgc{\cftalg{A}}}$,
\end{enumerate}
is said to be an \emph{empty-object algebra in context $\Gamma$} if additionally
the following judgmental equalities hold:
\begin{align}
\jfameq*{\Gamma}{\subst{\cftempc{\cftalg{A}}}{\cftalgf{\cftalg{A}}}}{\cftalgc{\cftalg{A}}}
  \label{empalg-eq1}
  \\
\jtermeq*{\Gamma}{\cftalgc{\cftalg{A}}}{\subst{\cftempc{\cftalg{A}}}{\cftempf{\cftalg{A}}}}{\cftempc{\cftalg{A}}}.
  \label{empalg-eq2}
%  \\
%\jfameq*{{\Gamma}{\cftalgc{\cftalg{A}}}}{\subst{\cftempf{\cftalg{A}}}{\cftalgt{\cftalg{A}}}}{\emptyf}
\end{align}
When $(\cftalgc{\cftalg{A}},\cftalgf{\cftalg{A}},\cftalgt{\cftalg{A}},
\cftempc{\cftalg{A}},\cftempf{\cftalg{A}})$ is an empty-object algebra
in context $\Gamma$, we will
also refer to the pair $(\cftempc{\cftalg{A}},\cftempf{\cftalg{A}})$ as the
\emph{empty object of $\cftalg{A}$}.
\end{defn}

Below, we give the definition of empty-object homomorphisms. They have, like
extension homomorphisms, an action on contexts and an action on families and
in addition they have an action on terms. Thus, we can picture a term
homomorphism as follows:
\begin{equation*}
\begin{tikzcd}
\cftalgt{\cftalg{A}}
  \ar[fib]{d}
  \ar{r}{\cfthomt{\cfthom{f}}}
& \cftalgt{\cftalg{B}}
  \ar[fib]{d}
  \\
\cftalgf{\cftalg{A}}
  \ar[fib]{d}
  \ar{r}{\cfthomf{\cfthom{f}}}
& \cftalgf{\cftalg{B}}
  \ar[fib]{d}
  \\
\cftalgc{\cftalg{A}}
  \ar{r}[swap]{\cfthomc{\cfthom{f}}}
& \cftalgc{\cftalg{B}}
\end{tikzcd}
\end{equation*}

\begin{defn}
Let $\cftalg{A}$ and $\cftalg{B}$ be term algebras in context $\Gamma$.
An \emph{empty-object homomorphism $\cfthom{f}$ from $\cftalg{A}$ to $\cftalg{B}$
in context $\Gamma$} is a triple 
$(\cfthomc{\cfthom{f}},\cfthomf{\cfthom{f}},\cfthomt{\cfthom{f}})$ consisting of
\begin{align*}
\jhom*
  {\Gamma}
  {\cftalgc{\cftalg{A}}}
  {\cftalgc{\cftalg{B}}}
  {\cfthomc{\cfthom{f}}}\\
\jfhom*
  {\Gamma}
  {\cftalgc{\cftalg{A}}}
  {\cftalgc{\cftalg{B}}}
  {\cfthomc{\cfthom{f}}}
  {\cftalgf{\cftalg{A}}}
  {\cftalgf{\cftalg{B}}}
  {\cfthomf{\cfthom{f}}}\\
\jfhom*
  {\Gamma}
  {{\cftalgc{\cftalg{A}}}{\cftalgf{\cftalg{A}}}}
  {{\cftalgc{\cftalg{B}}}{\cftalgf{\cftalg{B}}}}
  {\jvcomp{\cftalgf{\cftalg{A}}}{\cfthomc{\cfthom{f}}}{\cfthomf{\cfthom{f}}}}
  {\cftalgt{\cftalg{A}}}
  {\cftalgt{\cftalg{B}}}
  {\cfthomt{\cfthom{f}}}
\end{align*}
satisfying the judgmental equalities
\begin{align*}
\jtermeq*
  {\Gamma}
  {\cftalgc{\cftalg{B}}}
  {\subst{\cftempc{\cftalg{A}}}{\cfthomc{\cfthom{f}}}}
  {\cftempc{\cftalg{B}}}
  \\
\jhomeq*
  {\Gamma}
  {\cftalgc{\cftalg{A}}}
  {\cftalgc{\cftalg{B}}}
  {\subst{\cftempc{\cftalg{A}}}{\cfthomf{\cfthom{f}}}}
  {\cfthomc{\cfthom{f}}}
  \\
\jtermeq*
  {{\Gamma}{\cftalgc{\cftalg{A}}}}
  {\jcomp{\cftalgc{\cftalg{A}}}{\cfthomc{\cfthom{f}}}{\cftalgf{\cftalg{B}}}}
  {\subst{\cftempf{\cftalg{A}}}{\cfthomf{\cfthom{f}}}}
  {\jcomp{\cftalgc{\cftalg{A}}}{\cfthomc{\cfthom{f}}}{\cftempf{\cftalg{B}}}}.
\end{align*}
\end{defn}

A term-algebra $\cftalg{A}$ in context $\Gamma$ is going to be a septuple
\begin{equation*}
(\cftalgc{\cftalg{A}},\cftalgf{\cftalg{A}},\cftalgt{\cftalg{A}},
\cftctxext[\cftalg{A}],\cftfamext[\cftalg{A}],\cftempc{\cftalg{A}},
\cftempf{\cftalg{A}})
\end{equation*}
for which 
the quadruple $\unfold{\extalg{A}}$ is an extension algebra in context 
$\Gamma$, the quintuple $\unfold{\empalg{A}}$ is an empty-object algebra in context
$\Gamma$, satisfying additional judgmental equalities expressing the 
compatibility of $\cftctxext[\cftalg{A}]$ and $\cftfamext[\cftalg{A}]$ with 
$\cftempc{\cftalg{A}}$ and $\cftempf{\cftalg{A}}$.
There will be five such judgmental equalities. Four of them will express that
extension acts as the identity term when applied to the empty object in one of
its arguments. The other expresses that family extension restricted to
families above the empty object is the same as context extension. To be able
to state these, we need to know that $\unfold{\empalg{A}}$ is an empty-object
algebra. In the following, we verify step-by-step that we can indeed make the
suggested requirements.
\begin{enumerate}
\item The type of $\subst{\cftempc{\cftalg{A}}}{\cftctxext[\cftalg{A}]}$ is
$\subst
  { \cftempc{\cftalg{A}}
    }
  { \ctxwk
      {\ctxext{\cftalgc{\cftalg{A}}}{\cftalgf{\cftalg{A}}}}
      {\cftalgc{\cftalg{A}}}
    }$.
We have the judgmental equalities
\begin{align*}
\subst
  { \cftempc{\cftalg{A}}
    }
  { \ctxwk
      {\ctxext{\cftalgc{\cftalg{A}}}{\cftalgf{\cftalg{A}}}}
      {\cftalgc{\cftalg{A}}}
    }
& \jdeq
  \subst
    { \cftempc{\cftalg{A}}
      }
    { \ctxwk
        { \cftalgf{\cftalg{A}}
          }
        { {\cftalgc{\cftalg{A}}}
          {\cftalgc{\cftalg{A}}}
          }
      }
  \tag{by \autoref{comp-ew-f}}
  \\
& \jdeq
  \ctxwk
    { \subst{\cftempc{\cftalg{A}}}{\cftalgf{\cftalg{A}}}
      }
    { \subst{\cftempc{\cftalg{A}}}{\ctxwk{\cftalgc{\cftalg{A}}}{\cftalgc{\cftalg{A}}}}
      }
  \tag{by \autoref{comp-sw-f}}
  \\
& \jdeq
  \ctxwk
    { \subst{\cftempc{\cftalg{A}}}{\cftalgf{\cftalg{A}}}
      }
    { A
      }
  \tag{by \autoref{cancellation-ws-f}}
  \\
& \jdeq
  \ctxwk{A}{A}
  \tag{by \autoref{empalg-eq2}}
\end{align*}
Thus, we require that
\begin{equation}
\jtermeq{{\Gamma}{\cftalgc{\cftalg{A}}}}{\ctxwk{\cftalgc{\cftalg{A}}}{\cftalgc{\cftalg{A}}}}{\subst{\cftempc{\cftalg{A}}}{\cftctxext[\cftalg{A}]}}{\idtm{\cftalgc{\cftalg{A}}}}
  \label{cftalg-eq1}
\end{equation}
\end{enumerate}

We can immediately state the first two:
\begin{align}
\jtermeq*{{\Gamma}{\cftalgc{\cftalg{A}}}}{\ctxwk{\cftalgc{\cftalg{A}}}{\cftalgc{\cftalg{A}}}}{\subst{\cftempc{\cftalg{A}}}{\cftctxext[\cftalg{A}]}}{\idtm{\cftalgc{\cftalg{A}}}}
  \label{cftalg-eq1}
  \\
\jtermeq*{{\Gamma}{\cftalgc{\cftalg{A}}}}{\ctxwk{\cftalgc{\cftalg{A}}}{\cftalgc{\cftalg{A}}}}{\subst{\cftempf{\cftalg{A}}}{\cftctxext[\cftalg{A}]}}{\idtm{\cftalgc{\cftalg{A}}}}
  \label{cftalg-eq2}
\end{align}
To see what $\subst{\phi_0}{\epsilon_1}$ can be, we must know its type first.
It is a morphism from $\subst{\phi_0}{\ctxext{P}{\jcomp{}{\epsilon_0}{P}}}$ to
$\subst{\phi_0}{P}$. We already know that $\subst{\phi_0}{P}\jdeq A$ by
\autoref{empalg-eq1} and to compute $\subst{\phi_0}{\jcomp{}{\epsilon_0}{P}}$
we use \autoref{lem:empalg-mor}. Note that in combination with 
\autoref{cftalg-eq1} and \autoref{cftalg-eq2} we get the judgmental
equalities
\begin{equation}\label{cor:empalg-mor}
\jfameq{{\Gamma}{A}}{\subst{\phi_i}{\jcomp{}{\epsilon_0}{P}}}{P}
\end{equation}
for $i$ being $\mathsf{c}$ or $\mathsf{f}$. Consequently, 
$\subst{\phi_0}{\epsilon_1}$ is a morphism from $\ctxext{A}{P}$ to $A$, so we 
can require that
\begin{equation}\label{cftalg-eq5}
\jhomeq{\Gamma}{{A}{P}}{A}{\subst{\phi_0}{\epsilon_1}}{\epsilon_0}.
\end{equation}
This judgmental equality expresses that family extension restricted to the empty object
$\phi_0$ is context extension. We also have two judgmental
equalities expressing that $\epsilon_1$ restricted to $\phi_1$ and
$\jcomp{}{\epsilon_0}{\phi_1}$ are both the identity term on $P$. Note that
the term $\jcomp{}{\epsilon_0}{\phi_1}$ serves as the empty family of families,
being a term of $\jcomp{}{\epsilon_0}{P}$.

Before we can require that $\subst{\phi_1}{\epsilon_1}\jdeq\idtm{P}$, we need
to compute the family
\begin{equation*}
\jfam
  {{{\Gamma}{A}}{\subst{\phi_1}{\jcomp{}{\epsilon_0}{P}}}}
  {\subst{\phi_1}{\ctxwk{\ctxext{P}{\jcomp{}{\epsilon_0}{P}}}{P}}}.
\end{equation*}
of which $\subst{\phi_1}{\epsilon_1}$ is a term.
We have already established that $\subst{\phi_1}{\jcomp{}{\epsilon_0}{P}}\jdeq
P$. We also see that 
\begin{align*}
\subst{\phi_1}{\ctxwk{\ctxext{P}{\jcomp{}{\epsilon_0}{P}}}{P}}
& \jdeq
  \subst{\phi_1}{\ctxwk{\jcomp{}{\epsilon_0}{P}}{{P}{P}}}
  \tag{by \autoref{comp-ew-f}}
  \\
& \jdeq
  \ctxwk{\subst{\phi_1}{\jcomp{}{\epsilon_0}{P}}}{\subst{\phi_1}{\ctxwk{P}{P}}}
  \tag{by \autoref{comp-sw-f}}
  \\
& \jdeq
  \ctxwk{P}{\subst{\phi_1}{\ctxwk{P}{P}}}
  \tag{by \autoref{cor:empalg-mor}}
  \\
& \jdeq
  \ctxwk{P}{P},
  \tag{by \autoref{cancellation-ws-f}}
\end{align*}
so we see that it indeed makes sense to require that
\begin{equation}\label{cftalg-eq3}
\jtermeq{{{\Gamma}{A}}{P}}{\ctxwk{P}{P}}{\subst{\phi_1}{\epsilon_1}}{\idtm{P}}.
\end{equation}
For the last judgmental equality, we need to compute the family
\begin{equation*}
\jfam
  {{{\Gamma}{A}}{\subst{\phi_1}{\jcomp{}{\epsilon_0}{P}}}}
  {\subst{\jcomp{}{\epsilon_0}{\phi_1}}{\ctxwk{\ctxext{P}{\jcomp{}{\epsilon_0}{P}}}{P}}}.
\end{equation*}
This is easy:
\begin{align*}
\subst{\jcomp{}{\epsilon_0}{\phi_1}}{\ctxwk{\ctxext{P}{\jcomp{}{\epsilon_0}{P}}}{P}}
& \jdeq
  \subst{\jcomp{}{\epsilon_0}{\phi_1}}{\ctxwk{\jcomp{}{\epsilon_0}{P}}{{P}{P}}}
  \\
& \jdeq
  \ctxwk{P}{P}.
\end{align*}
Thus we are allowed to require that
\begin{equation}\label{cftalg-eq4}
\subst{\jcomp{}{\epsilon_0}{\phi_1}}{\epsilon_1}.
\end{equation}

We bring all this together in the definition of extension-empty algebras:

\begin{defn}
An \emph{extension-empty algebra in context $\Gamma$} 
is a sextuple $(A,P,\epsilon_0,\epsilon_1,\phi_0,\phi_1)$ for which 
the quadruple $(A,P,\epsilon_0,\epsilon_1)$ is an extension algebra in context 
$\Gamma$, the quadruple $(A,P,\phi_0,\phi_1)$ is an empty algebra in context
$\Gamma$, satisfying the judgmental equalities 
\autoref{cftalg-eq1,cftalg-eq2,cftalg-eq5,cftalg-eq3,cftalg-eq4}.
\end{defn}
\end{comment}

We can extend the results of 
\autoref{thm:extalg-fam,thm:extalg-wk,thm:extalg-subst} to CFT-algebras.

\begin{thm}\label{thm:cftalg-fam}
Let $\cftalg{A}$ be a CFT-algebra
in context $\Gamma$. Then the septuple
\begin{equation*}
\cftalg{F_A}
  \defeq
  ( \cftalgf{\cftalg{A}},
    \jcomp{}{\cftctxext}{\cftalgf{\cftalg{A}}},
    \jcomp{}{\cftctxext}{\cftalgt{\cftalg{A}}},
    \cftfamext,
    \jcomp{}{\cftctxext}{\cftfamext},
    \cftempf{\cftalg{A}},
    \jcomp{}{\cftctxext}{\cftempf{\cftalg{A}}})
\end{equation*}
is a CFT-algebra in context $\ctxext{\Gamma}{\cftalgc{\cftalg{A}}}$. 
\end{thm}

\begin{proof}
\begin{enumerate}
\item The judgmental equality
\begin{equation*}
\jfameq{{\Gamma}{\cftalgc{\cftalg{A}}}}{\subst{\cftempf{\cftalg{A}}}{\jcomp{}{\cftctxext}{\cftalgf{\cftalg{A}}}}}{\cftalgf{\cftalg{A}}}
\end{equation*}
was verified in \autoref{rmkenum:famfamempf-jdeq-fam} of \autoref{rmk:cftalg-defn}.
\item Now we verify the judgmental equality
\begin{equation*}
\jtermeq
  {{\Gamma}{\cftalgc{\cftalg{A}}}}
  {\cftalgf{\cftalg{A}}}
  {\subst{\cftempf{\cftalg{A}}}{\jcomp{}{\cftctxext}{\cftempf{\cftalg{A}}}}}
  {\cftempf{\cftalg{A}}}.
\end{equation*}
Note that we can apply \autoref{lem:empalg-mor}, so we get the judgmental
equalities
\begin{align*}
\subst{\cftempf{\cftalg{A}}}{\jcomp{}{\cftctxext}{\cftempf{\cftalg{A}}}}
& \jdeq
  \jcomp{}{\subst{\cftempf{\cftalg{A}}}{\cftctxext}}{\cftempf{\cftalg{A}}}
  \tag{by \autoref{lem:empalg-mor}}
  \\
& \jdeq
  \jcomp{}{\idtm{\cftalgf{\cftalg{A}}}}{\cftempf{\cftalg{A}}}
  \tag{by \autoref{cftalg-eq3}}
  \\
& \jdeq
  \cftempf{\cftalg{A}}.
  \tag{by \autoref{precomp-idtm-t}}
\end{align*}
\item The judgmental equalities
\begin{align*}
\subst{\cftempf{\cftalg{A}}}{\cftfamext}
& \jdeq 
  \idtm{\cftalgf{\cftalg{A}}}
  \\
\subst{\jcomp{}{\cftctxext}{\cftempf{\cftalg{A}}}}{\cftfamext}
& \jdeq 
  \idtm{\cftalgf{\cftalg{A}}}
\end{align*}
are given by assumption.
\item The judgmental equalities
\begin{align*}
\subst{\jcomp{}{\cftctxext}{\cftempf{\cftalg{A}}}}{\jcomp{}{\cftctxext}{\cftfamext}}
& \jdeq
  \idtm{\jcomp{}{\cftctxext}{\cftalgf{\cftalg{A}}}}
  \\
\subst{\jcomp{}{\cftfamext}{\jcomp{}{\cftctxext}{\cftempf{\cftalg{A}}}}}{\jcomp{}{\cftctxext}{\cftfamext}}
& \jdeq
  \idtm{\jcomp{}{\cftctxext}{\cftalgf{\cftalg{A}}}}
\end{align*}
follow from the judgmental equalities
\begin{align*}
\subst{\jcomp{}{\cftctxext}{\cftempf{\cftalg{A}}}}{\jcomp{}{\cftctxext}{\cftfamext}}
& \jdeq
  \jcomp{}{\cftctxext}{\subst{\cftempf{\cftalg{A}}}{\cftfamext}}
  \tag{by \autoref{lem:jcomp-subst}}
  \\
& \jdeq
  \jcomp{}{\cftctxext}{\idtm{\cftalgf{\cftalg{A}}}}
  \tag{by \autoref{cftalg-eq3}}
  \\
& \jdeq
  \idtm{\jcomp{}{\cftctxext}{\cftalgf{\cftalg{A}}}}
  \tag{by \autoref{comp-wi-t}}
\end{align*}
and
\begin{align*}
\subst
  {\jcomp{}{\cftfamext}{\jcomp{}{\cftctxext}{\cftempf{\cftalg{A}}}}}
  {\jcomp{}{\cftctxext}{\cftfamext}}
& \jdeq
  \subst
    {\jcomp{}{\cftctxext}{\jcomp{}{\cftctxext}{\cftempf{\cftalg{A}}}}}
    {\jcomp{}{\cftctxext}{\cftfamext}}
  \tag{by \autoref{lem:extalg-twins}}
  \\
& \jdeq
  \jcomp
    {}
    {\cftctxext}
    {\subst{\jcomp{}{\cftctxext}{\cftempf{\cftalg{A}}}}{\cftfamext}}
  \tag{by \autoref{comp-sw-t}}
  \\
& \jdeq
  \jcomp
    {}
    {\cftctxext}
    {\idtm{\cftalgf{\cftalg{A}}}}
  \tag{by \autoref{cftalg-eq4}}
  \\
& \jdeq
  \idtm{\jcomp{}{\cftctxext}{\cftalgf{\cftalg{A}}}}.
  \tag{by \autoref{comp-wi-t}}
\end{align*}
\item Finally, the judgmental equality
\begin{equation*}
\subst{\cftempf{\cftalg{A}}}{\jcomp{}{\cftctxext}{\cftfamext}}
  \jdeq
  \cftfamext
\end{equation*}
is verified as follows:
\begin{align*}
\subst{\cftempf{\cftalg{A}}}{\jcomp{}{\cftctxext}{\cftfamext}}
& \jdeq
  \jcomp{}{\subst{\cftempf{\cftalg{A}}}{\cftctxext}}{\cftfamext}
  \tag{by \autoref{lem:empalg-mor}}
  \\
& \jdeq
  \jcomp{}{\idtm{\cftalgc{\cftalg{A}}}}{\cftfamext}
  \tag{by \autoref{cftalg-eq1}}
  \\
& \jdeq
  \cftfamext.
  \tag{by \autoref{precomp-idtm-t}}
\end{align*}
\end{enumerate}
\end{proof}

\subsection{CFT-homomorphisms}
\begin{defn}
A homomorphism of term algebras $\cfthom{f}$ from $\cftalg{A}$ to $\cftalg{B}$
consists of a triple $(\cfthomc{\cfthom{f}},\cfthomf{\cfthom{f}},\cfthomt{\cfthom{f}})$
such that $(\cfthomc{\cfthom{f}},\cfthomf{\cfthom{f}})$ is an extension homomorphism
\end{defn}

\subsection{Weakening algebras}
Weakening algebras will be term-algebras with certain added structure.
Although strictly speaking one could formulate a notion of weakening algebra
which only depends on extension algebras and which omits both empty objects
and terms, we shall not do so here.

Let $\cftalg{A}$ be a term-algebra in context $\Gamma$. Weakening will be a
CFT-homomorphism
\begin{equation}\label{cftwk}
\jhom
  {{{\Gamma}{\cftalgc{\cftalg{A}}}}{\cftalgf{\cftalg{A}}}}
  {\ctxwk{\cftalgf{\cftalg{A}}}{\cftalg{F_A}}}
  {\cftalg{F_{F_A}}}
  {\cftwk{\cftalg{A}}}
\end{equation}
satisfying judgmental equalities which express abstracted analogues of
the rules in sections \autoref{comp-ww,comp-ew,comp-0w}. The
definition of weakening algebras can be found in \autoref{sec:cftwkalg-defn}.

We begin with some trivial remarks to familiarize ourselves with the situation.
The morphism $\cftwk{\cftalg{A}}$ of \autoref{cftwk} is a triple
$(\cftwkc{\cftalg{A}},\cftwkf{\cftalg{A}},\cftwkt{\cftalg{A}})$ consisting of
the morphisms displayed in the following diagram:
\begin{equation*}
\begin{tikzcd}
\ctxwk{\cftalgf{\cftalg{A}}}{\jcomp{}{\cftctxext}{\cftalgt{\cftalg{A}}}}
  \ar[fib]{d}
  \ar{r}{\cftwkt{\cftalg{A}}}
& \jcomp{}{\cftctxext}{\jcomp{}{\cftctxext}{\cftalgt{\cftalg{A}}}}
  \ar[fib]{d}
  \\
\ctxwk{\cftalgf{\cftalg{A}}}{\jcomp{}{\cftctxext}{\cftalgf{\cftalg{A}}}}
  \ar[fib]{d}
  \ar{r}{\cftwkf{\cftalg{A}}}
& \jcomp{}{\cftctxext}{\jcomp{}{\cftctxext}{\cftalgf{\cftalg{A}}}}
  \ar[fib]{d}
  \\
\ctxwk{\cftalgf{\cftalg{A}}}{\cftalgf{\cftalg{A}}}
  \ar{r}[swap]{\cftwkc{\cftalg{A}}}
& \jcomp{}{\cftctxext}{\cftalgf{\cftalg{A}}}
\end{tikzcd}
\end{equation*}

\begin{rmk}
Let $\cftalg{A}$ be a term-algebra and let $\cftwk{\cftalg{A}}$ be as in
\autoref{cftwk}. Then we have
\begin{equation*}
\jterm
  {\Gamma}
  {\subst{{a}{{\gamma}{\cftctxext}}}{\cftalgf{\cftalg{A}}}}
  {\subst{b}{{a}{{\gamma}{\cftwkc{\cftalg{A}}}}}}
\end{equation*}
for $\gamma:A$ and $a,b:\subst{\gamma}{\cftalgf{\cftalg{A}}}$. We have
\begin{equation*}
\jterm
  {\Gamma}
  {\subst{{{b}{{a}{{\gamma}{\cftwkc{\cftalg{A}}}}}}{{{a}{{\gamma}{\cftctxext}}}{\cftctxext}}}{\cftalgf{\cftalg{A}}}}
  {\subst{q}{{b}{{a}{{\gamma}{\cftwkf{\cftalg{A}}}}}}}
\end{equation*}
for $\gamma:A$, $a,b:\subst{\gamma}{\cftalgf{\cftalg{A}}}$ and
$q:\subst{{b}{{\gamma}{\cftctxext}}}{\cftalgf{\cftalg{A}}}$. Finally, we have
\begin{equation*}
\jterm
  {\Gamma}
  {\subst{{q}{{b}{{a}{{\gamma}{\cftwkf{\cftalg{A}}}}}}}{{{{b}{{a}{{\gamma}{\cftwkc{\cftalg{A}}}}}}{{{a}{{\gamma}{\cftctxext}}}{\cftctxext}}}{\cftalgt{\cftalg{A}}}}}
  {\subst{g}{{q}{{b}{{a}{{\gamma}{\cftwkt{\cftalg{A}}}}}}}}
\end{equation*}
for $\gamma:A$, $a,b:\subst{\gamma}{\cftalgf{\cftalg{A}}}$,
$q:\subst{{b}{{\gamma}{\cftctxext}}}{\cftalgf{\cftalg{A}}}$ and
$g:\subst{q}{{{b}{{\gamma}{\cftctxext}}}{\cftalgt{\cftalg{A}}}}$.
These observations show that the weakening morphism $\cftwk{\cftalg{A}}$ is
indeed an abstraction of the introduction rules for the weakening operation.
\end{rmk}

\subsubsection{Weakening preserves itself}
The following diagram displays the compatibility of weakening with itself:
\begin{equation*}\label{diag:cftwk-cftwk}
\begin{tikzcd}[column sep=huge]
\ctxwk{\cftalgf{\cftalg{A}}}{{\cftalgf{\cftalg{F_A}}}{\cftalg{F_{F_A}}}}
  \ar{d}
    [swap]{ \ctxwk
        {{\cftalgf{\cftalg{A}}}{\cftalgf{\cftalg{F_A}}}}
        {\boldsymbol{\mathcal{F}}_{\cftwk{\cftalg{A}}}}
      }
  \ar{r}{\ctxwk{\cftalgf{\cftalg{A}}}{\jcomp{}{\cftctxext}{\cftwk{\cftalg{A}}}}}
& \ctxwk{\cftalgf{\cftalg{A}}}{\cftalg{F_{F_{F_A}}}}
  \ar{d}{\jcomp{}{\cftctxext}{\boldsymbol{\mathcal{F}}_{\cftwk{\cftalg{A}}}}}
  \\
\ctxwk{{\cftalgf{\cftalg{A}}}{\cftalgf{\cftalg{F_A}}}}{\jcomp{}{\cftwkc{A}}{\cftalg{F_{F_{F_A}}}}}
  \ar{r}[swap]{\jcomp{}{\cftwkc{A}}{\jcomp{}{\cftctxext}{\cftwk{A}}}}
& \jcomp{}{\cftwkc{A}}{\cftalg{F_{F_{F_{F_A}}}}}
\end{tikzcd}
\end{equation*}
This is a complicated diagram and it is not immediately clear that all of it
makes sense. Therefore we examine its ingredients in more detail.

The morphisms on the top and on the right are fine as they are displayed

\begin{lem}
Let $\cftalg{A}$ be a term-algebra in context $\Gamma$ and let
\begin{equation*}
\jhom
  {{{\Gamma}{\cftalgc{\cftalg{A}}}}{\cftalgf{\cftalg{A}}}}
  {\ctxwk{\cftalgf{\cftalg{A}}}{\cftalg{F_A}}}
  {\jcomp{}{\cftctxext}{\cftalg{F_A}}}
  {\cftwk{\cftalg{A}}}
\end{equation*}
be a homomorphism of term-algebras satisfying \autoref{diag:cftwk-cftwk}. Then
we have the judgmental equality
\begin{equation*}
\jtermeq
  {}
  {}
  {}
  {}
\end{equation*}
for $\gamma:A$, $x:\cftalgf{\cftalg{A}}$, $y_0:\cftalgf{\cftalg{A}}$ and
$y_1:\cftalgf{\cftalg{F_A}}$.
\end{lem}

\subsubsection{Currying for weakening}
\begin{equation*}
\jhomeq
  { {{{\Gamma}{\cftalgc{\cftalg{A}}}}{\cftalgf{\cftalg{A}}}}
    {\cftalgf{\cftalg{F_A}}}
    }
  { \ctxwk
      {\cftalgf{\cftalg{F_A}}}
      {\cftalg{F_{F_A}}}
    }
  { \cftalg{F_{F_{F_A}}}
    }
  { \jcomp{}{\cftfamext}{\cftwk{\cftalg{A}}}
    }
  { \jcomp{}{\cftwk{\cftalg{A}}}{\jcomp{}{\cftctxext}{\cftwk{\cftalg{A}}}}
    }
\end{equation*}

\subsubsection{Weakening by the empty family}
\begin{equation*}
\jhomeq
  { {\Gamma}{\cftalgc{\cftalg{A}}}
    }
  { \cftalg{F_A}
    }
  { \cftalg{F_A}
    }
  { \subst{\cftempf{\cftalg{A}}}{\cftwk{\cftalg{A}}}
    }
  { id_{\cftalg{F_A}}
    }
\end{equation*}

\subsubsection{The definition of weakening algebras}\label{sec:cftwkalg-defn}

\subsubsection{Derivable properties of weakening algebras}
We should theoremize the following statements:
\begin{enumerate}
\item If $\cftalg{A}$ is a weakening algebra, then so is $\cftalg{F_A}$ if we take
$\jcomp{}{\cftctxext}{\cftwk{\cftalg{A}}}$ as its weakening.
\end{enumerate}

\begin{comment}
Because the pair $(\omega_0,\omega_1)$ is required to be an extension homomorphism,
the diagrams
\begin{equation}\label{wkalg-exteq1}
\begin{tikzcd}[column sep=large]
\ctxwk{P}{\ctxext{P}{\jcomp{}{\epsilon_0}{P}}}
  \ar{r}{\jvcomp{}{\omega_0}{\omega_1}}
  \ar{d}[swap]{\ctxwk{P}{\epsilon_1}}
& \ctxext
    {\jcomp{}{\epsilon_0}{P}}
    {\jcomp{}{\epsilon_0}{\jcomp{}{\epsilon_0}{P}}}
  \ar{d}{\jcomp{}{\epsilon_0}{\epsilon_1}}
  \\
\ctxwk{P}{P}
  \ar{r}[swap]{\omega_0}
& \jcomp{}{\epsilon_0}{P}
\end{tikzcd}
\end{equation}
in context $\ctxext{{\Gamma}{A}}{P}$, and
\begin{equation}\label{wkalg-exteq2}
\begin{tikzcd}[column sep=huge]
\ctxwk
  {P}
  { \ctxext
      {\jcomp{}{\epsilon_0}{P}}
      {\jcomp{}{\epsilon_0}{\jcomp{}{\epsilon_0}{P}}}
    }
  \ar{r}{\jvcomp{}{\omega_1}{\jcomp{}{\ctxwk{P}{\epsilon_1}}{\omega_1}}}
  \ar{d}[swap]{\ctxwk{P}{\jcomp{}{\epsilon_0}{\epsilon_1}}}
& \jcomp{}{\omega_0}{%
    \jcomp{}{\epsilon_0}{%
      \jcomp{}{\epsilon_0}{%
        \ctxext{P}{\jcomp{}{\epsilon_0}{P}}
        }
      }
    }
  \ar{d}{ \jcomp{}{\omega_0}{%
            \jcomp{}{\epsilon_0}{%
              \jcomp{}{\epsilon_0}{\epsilon_1}
              }
            }
          }
  \\
\ctxwk{P}{\jcomp{}{\epsilon_0}{P}}
  \ar{r}[swap]{\omega_1}
& \jcomp{}{\omega_0}{%
    \jcomp{}{\epsilon_0}{%
      \jcomp{}{\epsilon_0}{P}
      }
    }
\end{tikzcd}
\end{equation}
in context $\ctxext{{{\Gamma}{A}}{P}}{\ctxwk{P}{P}}$, commute judgmentally.
To see that the morphism $\jvcomp{}{\omega_1}{\jcomp{}{\ctxwk{P}{\epsilon_0}}{\omega_1}}$
indeed has the suggested codomain, we have the following lemma:

\begin{lem}
Let $(A,P,\epsilon_0,\epsilon_1)$ be an extension algebra and let
\begin{align*}
\jhom*
  {{{\Gamma}{A}}{P}}
  {\ctxwk{P}{P}}
  {\jcomp{}{\epsilon_0}{P}}
  {\omega_0}
  \\
\jfhom*
  {{{\Gamma}{A}}{P}}
  {\ctxwk{P}{P}}
  {\jcomp{}{\epsilon_0}{P}}
  {\omega_0}
  {\ctxwk{P}{\jcomp{}{\epsilon_0}{P}}}
  {\jcomp{}{\epsilon_0}{\jcomp{}{\epsilon_0}{P}}}
  {\omega_1}
\end{align*}
be morphisms satisfying \autoref{wkalg-exteq1}. Then the inference rules
\begin{align*}
& \inference
  { \jfam{{\Gamma}{A}}{Q}
    }
  { \jfameq
      {\blank}
      { \jcomp{}{\omega_1}{%
          \jcomp{}{\omega_0}{%
            \jcomp{}{\epsilon_0}{%
              \jcomp{}{\epsilon_0}{%
                \jcomp{}{\epsilon_0}{Q}
                }
              }
            }
          } 
        }
      { \jcomp{}{\ctxwk{P}{\epsilon_0}}{%
          \jcomp{}{\omega_0}{%
            \jcomp{}{\epsilon_0}{%
              \jcomp{}{\epsilon_0}{%
                Q
                }
              }
            }
          }
        }
      }
\end{align*}
are valid.
\end{lem}

\begin{proof}
We have the judgmental equalities
\begin{align*}
& \jcomp{}{\omega_1}{%
          \jcomp{}{\omega_0}{%
            \jcomp{}{\epsilon_0}{%
              \jcomp{}{\epsilon_0}{%
                \jcomp{}{\epsilon_0}{Q}
                }
              }
            }
          } 
  \\
& \jdeq
  \jcomp{}{\jvcomp{}{\omega_0}{\omega_1}}{%
            \jcomp{}{\epsilon_0}{%
              \jcomp{}{\epsilon_0}{%
                \jcomp{}{\epsilon_0}{Q}
                }
              }
            }
  \tag{by \autoref{lem:composition-threesome}}
  \\
& \jdeq
  \jcomp{}{\jvcomp{}{\omega_0}{\omega_1}}{%
            \jcomp{}{\epsilon_0}{%
              \jcomp{}{\epsilon_1}{%
                \jcomp{}{\epsilon_0}{Q}
                }
              }
            }
  \tag{by \autoref{lem:extalg-twins}}
  \\
& \jdeq
  \jcomp{}{\jvcomp{}{\omega_0}{\omega_1}}{%
            \jcomp{}{\jcomp{}{\epsilon_0}{\epsilon_1}}{%
              \jcomp{}{\epsilon_0}{%
                \jcomp{}{\epsilon_0}{Q}
                }
              }
            }
   \tag{by \autoref{lem:jcomp-higherjcomp}}
   \\
& \jdeq
  \jcomp{}{\jcomp{}{%
             \jvcomp{}{\omega_0}{\omega_1}}{%
               \jcomp{}{\epsilon_0}{\epsilon_1}}}{%
              \jcomp{}{\epsilon_0}{%
                \jcomp{}{\epsilon_0}{Q}
                }
              }
  \tag{by \autoref{lem:jcomp-jcomp}}
  \\
& \jdeq
  \jcomp{}{\jcomp{}{\ctxwk{P}{\epsilon_0}}{\omega_0}}{%
              \jcomp{}{\epsilon_0}{%
                \jcomp{}{\epsilon_0}{Q}
                }
              }
  \tag{by \autoref{wkalg-exteq1}}
  \\
& \jdeq
  \jcomp{}{\ctxwk{P}{\epsilon_0}}{%
    \jcomp{}{\omega_0}{%
      \jcomp{}{\epsilon_0}{%
        \jcomp{}{\epsilon_0}{%
          Q
          }
        }
      }
    }
  \tag{by \autoref{lem:jcomp-jcomp}}
\end{align*}
\end{proof}

\subsubsection{The compatibility of weakening with the empty context and family}
The first two judgmental equalities expressing that $\omega_0$ and $\omega_1$
are compatible with $\phi_1$ are easy to state:
\begin{align}
\jhomeq*{{\Gamma}{A}}{P}{P}{\subst{\phi_1}{\omega_0}}{\idtm{P}}\\
\jhomeq*{{\Gamma}{A}}{P}{P}{\subst{\phi_1}{\omega_1}}{\idtm{P}}
\end{align}

\subsubsection{The compatibility of weakening with itself}
To express the compatibility of weakening with itself, we must fill in the
following diagram:
\begin{equation*}
\begin{tikzcd}
\ctxwk{P}{{P}{P}}
  \ar{r}{\ctxwk{P}{\omega_0}}
  \ar{d}[swap]{\ctxwk{{P}{P}}{\omega_0}}
& \ctxwk{P}{\jcomp{}{\epsilon_0}{P}}
  \ar{d}{\omega_1}
  \\
\ctxwk{{P}{P}}{\jcomp{}{\epsilon_0}{P}}
  \ar[densely dotted]{r}
& \jcomp{}{\omega_0}{\jcomp{}{\epsilon_0}{\jcomp{}{\epsilon_0}{P}}}
\end{tikzcd}
\end{equation*}
\end{comment}

\subsection{Projection algebras}
Projection algebras will be weakening algebras with additional terms implementing
the identity terms. We call these algebras projection algebras because the
identity terms can only be formulated in weakening algebras and together with
weakening, the identity terms provide for all the projections. In a weakening
algebra $\cftalg{A}$ in context $\Gamma$, identity terms are implemented by a term
\begin{equation*}
\jterm
  { { {\Gamma}
      {\cftalgc{\cftalg{A}}}
      }
    { \cftalgf{\cftalg{A}}
      }
    }
  {\subst{{\idtm{\cftalgf{\cftalg{A}}}}{\cftwkc{\cftalg{A}}}}{\cftalgt{\cftalg{F_A}}}}
  {\cftidtm{\cftalg{A}}}
\end{equation*}
We will require additionally that identity terms are compatible with weakening.
Thus we will formulate an abstraction of the rule stated in \autoref{comp-wi}.

\begin{equation}
\jtermeq
  { \xi}
  { \xi}
  { \subst
      { \jcomp
          { }
          { \cftctxext}
          { \cftidtm{\cftalg{A}}
            }
        }
      { { { \idtm
              { \jcomp
                  { }
                  { \cftctxext}
                  { \cftalgf{\cftalg{A}}
                    }
                }
            }
            { \jcomp
                { }
                { \cftctxext
                  }
                { \cftwkc{\cftalg{A}}
                  }
              }
          }
        { \jcomp
            { }
            { \jcomp{}{\cftctxext}{\cftfamext}}
            { \cftwkt{\cftalg{A}}}
          }
        }
    }
  { \xi}
\end{equation}

\subsection{Substitution algebras}
Substitution algebras will be CFT-algebras with an additional substitution 
operation. Given a CFT-algebra $\cftalg{A}$ in context $\Gamma$, substitution 
will be a CFT-homomorphism
\begin{equation*}
\jhom
  {{{{\Gamma}{\cftalgc{\cftalg{A}}}}{\cftalgf{\cftalg{A}}}}{\cftalgt{\cftalg{A}}}}
  {\ctxwk{\cftalgt{\cftalg{A}}}{\cftalg{F_{F_A}}}}
  {\cftalg{F_A}}
  {\cftsubst{\cftalg{A}}}
\end{equation*}
We will require additionally that substitution preserves itself. Thus, we will
formulate an abstraction of the rules stated in \autoref{comp-ss}.

\subsection{E-algebras}
E-algebras will be CFT-algebras with a weakening operation, identity terms and
a substitution operation making it a projection algebra and a substitution 
algebra at the same time. Additionally the weakening operation, identity terms
and the substitution operation are all required to be compatible with each
other. Thus, we need to require the abstractions of the rules stated in
\autoref{comp-sw,comp-ws,cancellation-ws,cancellation-i}.

\subsection{Pre-universes}
Pre-universes are internal versions of the theory of contexts, families and
terms. They interpret extension, the empty context, weakening, substitution
and identity terms all at once in a compatible way. Besides the compatibility
properties there will be judgmental equalities analoguous to the cancellation
properties of \autoref{cancellation-ws,cancellation-i}. Pre-universes are to
the theory of contexts, families and terms what internal categories to a
category.


%\section{The type theory of models of type theory}
In this section we pursue the idea of what a general model of type theory is by
axiomatizing what you can do with them. We have the following ideas:
\begin{itemize}
\item There are dependent models and sections thereof. Particular instances
  of sections: extension, weakening and substitution (and something for identity
  function?). And like the original
  extension, weakening and substitution, they're going to be compatible with each
  other. Therefore we state our theory of models as an extension of type theory
  without constructors.
\item Type theory without constructors is a model of itself, the canonical model $\mctx$.
\item for any (family of) model(s) $A$, there is the family model $\mfam{A}$ which
  is a family of models over $A$.
\item The terms of $\mctx$ should be precisely contexts. Terms of $\subst{\Gamma}{\mfam{\mctx}}$
  should be families over $\Gamma$.
\item if a model and a family of models over it are given, there is an extended model.
  If we extend $A$ by $\mfam{A}$, we get the Sierpinski model of $A$.
\item Likewise, models can be weakened and substituted and there are identity
  functions.
\item So the theory of models is going to be an extension of this theory with this
  data. The theory we are about to describe can be seen as an elementary theory
  of the category (with families (and terms)) of categories (with families (and terms)).
\item We would also like to remark explicitly that the valid judgments of the original type
  theory become contexts, families or terms, depending on what kind of judgments
  it was. Then, valid inferences of the original type theory become valid
  judgments here. In some sense, the theory we present here is therefore a second
  order theory over the basic theory of types.
\item Martin Escardo and Mike Shulman have been promoting the use of 
  inductive-inductive definitions for internal models. What we're doing here looks
  different, because we're writing down a type theory for internal models, but it
  might not be that different at all. The type theory can be seen as the theory
  of (dependent) algebras over the inductive-inductively defined model and the
  terms of this type theory are the (dependent) algebra homomorphisms. The
  asserted initial object of our type theory is the inductive-inductively defined
  type. 
\end{itemize}

\subsection{The basic ingredients of the type theory of models}
We first introduce the basic ingredients of our abstract theory of models.
\begin{align*}
& \inference
  { }
  { \jctx{\mctx}
    }
  \tag{the canonical model}\\
& \inference
  { \jfam{\mctx}{P}
    }
  { \jterm{\mctx}{P}{i}
    }
  \tag{initiality of $\mctx$}\\
& \inference
  { \jfam{\Gamma}{A}
    }
  { \jfam{{\Gamma}{A}}{\mfam{A}}
    }
  \tag{families}\\
& \inference
  { \jctx{\Gamma}
    }
  { \jfam{{\Gamma}{\mfam{\Gamma}}}{\mtm{\Gamma}}
    }
  \tag{terms}
\intertext{%
  The following two rules essentially make a start with saying that every term is functorial
  in the apropriate sense (there will be more rules contributing to this vies):}
& \inference
  { \jterm{{\Gamma}{A}}{P}{f}
    }
  { \jhom
      {{\Gamma}{A}}
      {\mfam{A}}
      {\subst{f}{\mfam{P}}}
      {\mfam{f}}
    }
  \\
& \inference
  { \jterm{{\Gamma}{A}}{P}{f}
    }
  { \jfhom
      {{\Gamma}{A}}
      {\mfam{A}}
      {\subst{f}{\mfam{P}}}
      {\mfam{f}}
      {\mtm{A}}
      {\subst{f}{\mtm{P}}}
      {\mtm{f}}
    }
\intertext{%
  When the situation requires clarity --  for instance when extensions
  are involved -- we will write $\mfam[\Gamma]{f}$ and $\mtm[\Gamma]{f}$ to make
  what we regard as the context explicit.}
\intertext{%
The empty context of a model $A$ is simply a term of $A$:}
& \inference
  { \jfam{\Gamma}{A}
    }
  { \jterm{\Gamma}{A}{\tfemp{A}}
    }
  \tag{empty context}\\
& \inference
  { \jfam{\Gamma}{A}
    }
  { \jterm{{{\Gamma}{A}}{\mfam{A}}}{\subst{\tfemp{\mfam{A}}}{\mtm{A}}}{\tft{A}}
    }
\intertext{%
  Extension is going to be a morphism from $\ctxext{A}{\mfam{A}}$ to
  $A$ in context $\Gamma$:}
& \inference
  { \jfam{\Gamma}{A}
    }
  { \jhom{\Gamma}{{A}{\mfam{A}}}{A}{\tfext{A}}
    }
  \tag{extension}
\intertext{%
  The action of weakening on families
  is a morphism from $\ctxwk{\mfam{A}}{\mfam{A}}$ to
  $\mfam{\mfam{A}}$ in context $\ctxext{{\Gamma}{A}}{\mfam{A}}$:}
& \inference
  { \jfam{\Gamma}{A}
    }
  { \jhom
      {{{\Gamma}{A}}{\mfam{A}}}
      {\ctxwk{\mfam{A}}{\mfam{A}}}
      {\mfam{\mfam{A}}}
      {\tfwk{A}^0}
    }
  \tag{weakening}
\intertext{%
  The action of weakening on terms
  is a morphism from $\ctxwk{\mfam{A}}{\mtm{A}}$ to $\mtm{\mfam{\mfam{A}}}$ over
  $\tfwk{A}^0$ in context $\ctxext{{\Gamma}{A}}{\mfam{A}}$:}
& \inference
  { \jfam{\Gamma}{A}
    }
  { \jfhom
      {{{\Gamma}{A}}{\mfam{A}}}
      {\ctxwk{\mfam{A}}{\mfam{A}}}
      {\mfam{\mfam{A}}}
      {\tfwk{A}^0}
      {\ctxwk{\mfam{A}}{\mtm{A}}}
      {\mtm{\mfam{\mfam{A}}}}
      {\tfwk{A}^1}
    }
  \tag{weakening}
\intertext{%
  The action of substitution by a term on families
  is going to be a morphism from $\ctxwk{\mtm{A}}{\mfam{\mfam{A}}}$ to 
  $\ctxwk{\mtm{A}}{{\mfam{A}}{\mfam{A}}}$
  in context $\ctxext{{{\Gamma}{A}}{\mfam{A}}}{\mtm{A}}$:}
& \inference
  { \jfam{\Gamma}{A}
    }
  { \jhom
      {{{{\Gamma}{A}}{\mfam{A}}}{\mtm{A}}}
      {\ctxwk{\mtm{A}}{\mfam{\mfam{A}}}}
      {\ctxwk{\mtm{A}}{{\mfam{A}}{\mfam{A}}}}
      {\tfsubst{A}^0}
    }
  \tag{substitution}
\intertext{%
  Likewise, the action of substitution by a term on terms of those
  families is going to be a morphism from $\ctxwk{\mtm{A}}{\mtm{\mfam{\mfam{A}}}}$
  to $\ctxwk{\mtm{A}}{{\mfam{A}}{\mtm{A}}}$ over $\tfsubst{A}^0$ in context
  $\ctxext{{{\Gamma}{A}}{\mfam{A}}}{\mtm{A}}$:}
& \inference
  { \jfam{\Gamma}{A}
    }
  { \jfhom
      {{{{\Gamma}{A}}{\mfam{A}}}{\mtm{A}}}
      {\ctxwk{\mtm{A}}{\mfam{\mfam{A}}}}
      {\ctxwk{\mtm{A}}{{\mfam{A}}{\mfam{A}}}}
      {\tfsubst{A}^0}
      {\ctxwk{\mtm{A}}{\mtm{\mfam{\mfam{A}}}}}
      {\ctxwk{\mtm{A}}{{\mfam{A}}{\mtm{A}}}}
      {\tfsubst{A}^1}
    }
  \tag{substitution}
\intertext{%
  The identity terms are coded by}
& \inference
  { \jfam{\Gamma}{A}
    }
  { \jterm
      {{{\Gamma}{A}}{\mfam{A}}}
      {\subst{{\idfunc[\mfam{A}]}{\tfwk{A}^0}}{\mtm{\mfam{\mfam{A}}}}}
      {\tfid{A}}
    }
  \tag{identity terms}
\end{align*}
We will call the terms that we introduced here the model constructors.

\subsection{The compatibility rules}
We need to do several things in this section:
\begin{itemize}
\item postulate that ordinary extension, weakening, substitution and identity
functions are compatible with the model constructors.
\item postulate that each of the model constructors is compatible with ordinary
extension, weakening, substitution and identity terms. Actually, we want
that every term of this type theory is compatible with those. I.e.~every term
is a functor/morphism of models. Also the sections.
\item postulate that the model constructors are compatible with each other,
so that they come to model ordinary extension, weakening, substitution and
identity terms respectively.
\item Note that the rules for compatibility with extension are going to explain which
model $\ctxext{A}{P}$ is by telling what the families, the terms, extension,
weakening, substitution and identity terms are. Likewise the rules for
compatibility with weakening and substitution do this for their respective
cases.
\item The easiest set of compatibility rules comes with weakening. The compatibility
rules that deal with substitution are likely going to have to do with the
Yoneda lemma, which we should be able to implement at some point.
\item The model $\subst{\tfemp{A}}{\mtm{A}}$ is going to be initial in the category of
all models over $A$. That means: whenever $P$ is a family over $\subst{\tfemp{A}}
{\mtm{A}}$ there will be a term of $P$. This means in particular that
$\subst{\tfemp{\emptyf}}{\mtm{\emptyf}}$ is going to be $\mctx$.
\end{itemize}

\subsubsection{Families of weakenings}
\begin{equation*}
\inference
  { \jfam{\Gamma}{A}
    \jfam{\Gamma}{B}
    }
  { \jfameq
      {{{\Gamma}{A}}{\ctxwk{A}{B}}}
      {\mfam{\ctxwk{A}{B}}}
      {\ctxwk{A}{\mfam{B}}}
    }.
\end{equation*}

\subsubsection{Terms of weakenings}
\begin{equation*}
\inference
  { \jfam{\Gamma}{A}
    \jfam{\Gamma}{B}
    }
  { \jfameq
    {{{{\Gamma}{A}}{\ctxwk{A}{B}}}{\mfam{\ctxwk{A}{B}}}}
    {\mtm{\ctxwk{A}{B}}}
    {\ctxwk{A}{\mtm{B}}}
    }
\end{equation*}

\subsubsection{Families over families are families over extensions}
\begin{equation*}
\inference
  { \jfam{\Gamma}{A}
    }
  { \jfameq
      {{{\Gamma}{A}}{\mfam{A}}}
      {\mfam{\mfam{A}}}
      {\jcomp{{A}{\mfam{A}}}{\tfext{A}}{\mfam{A}}}
    }
\end{equation*}
\emph{(Note: this rule might be a consequence of an explanation of what $\mfam{{A}{P}}$
is in general, but at the moment I don't see how to do this)}

\subsubsection{Terms of families over families are terms of families over extensions}
\begin{equation*}
\inference
  { \jfam{\Gamma}{A}
    }
  { \jfameq
      {{{{\Gamma}{A}}{\mfam{A}}}{\mfam{\mfam{A}}}}
      {\mtm{\mfam{A}}}
      {\jcomp{}{\tfext{A}}{\mtm{A}}}
      }
\end{equation*}

\subsubsection{Extension acts as the identity on families of families}
The family of which $\tfext{A}$ is a term, is unfolded as
\begin{equation*}
\unfold{\jhom{\Gamma}{{A}{\mfam{A}}}{A}{\tfext{A}}}
\end{equation*}
Because we have the judgmental equality $\ctxext{\Gamma}{{A}{\mfam{A}}}
\jdeq \ctxext{{\Gamma}{A}}{\mfam{A}}$, there is a term
\begin{equation*}
\jhom
  {{{\Gamma}{A}}{\mfam{A}}}
  {\mfam{\mfam{A}}}
  {\subst{\tfext{A}}{\mfam{\ctxwk{\ctxext{A}{\mfam{A}}}{A}}}}
  {\mfam[\ctxext{\Gamma}{A}]{\tfext{A}}}
\end{equation*}
By the rule asserting that families of families are families of extensions, we
have $\mfam{\mfam{A}}\jdeq\jcomp{}{\tfext{A}}{\mfam{A}}$. By the rule that
families over weakenings are weakenings of families, we have the judgmental
equalities
\begin{equation*}
\subst{\tfext{A}}{\mfam{\ctxwk{\ctxext{A}{\mfam{A}}}{A}}}
\jdeq 
  \unfold{\jcomp{{A}{P}}{\tfext{A}}{\mfam{A}}}
\jdeq 
  \jcomp{{A}{P}}{\tfext{A}}{\mfam{A}}
\end{equation*}
Therefore, we can compare the term $\mfam[\ctxext{\Gamma}{A}]{\tfext{A}}$
to the identity term $\idfunc[\jcomp{}{\tfext{A}}{\mfam{A}}]$. 
The compatibility
rule for the action of extension on families of families asserts that
\begin{equation*}
\inference
  { \jfam{\Gamma}{A}
    }
  { \jhomeq
      {{{\Gamma}{A}}{\mfam{A}}}
      {\jcomp{}{\tfext{A}}{\mfam{A}}}
      {\jcomp{}{\tfext{A}}{\mfam{A}}}
      {\mfam[\ctxext{\Gamma}{A}]{\tfext{A}}}
      {\idfunc[\jcomp{}{\tfext{A}}{\mfam{A}}]}
    }
\end{equation*}
\emph{(Note: this rule might be a consequence of an explanation of what $\mfam{{A}{P}}$
is in general, but at the moment I don't see how to do this)}

\subsubsection{Extension acts as the identity on terms of families of families}
We now investigate the nature of the morphism.
\begin{equation*}
\jfhom
  {{{\Gamma}{A}}{\mfam{A}}}
  {\mfam{\mfam{A}}}
  {\subst{\tfext{A}}{\mfam{\ctxwk{\ctxext{A}{\mfam{A}}}{A}}}}
  {\mfam[\ctxext{\Gamma}{A}]{\tfext{A}}}
  {\mtm{\mfam{A}}}
  {\subst{\tfext{A}}{\mtm{\ctxwk{\ctxext{A}{\mfam{A}}}{A}}}}
  {\mtm[\ctxext{\Gamma}{A}]{\tfext{A}}}
\end{equation*}
Note that we have the judgmental equality 
$\mtm{\mfam{A}}\jdeq\jcomp{}{\tfext{A}}{\mtm{A}}$. Likewise, we have the
judgmental equality $\subst{\tfext{A}}{\mtm{\ctxwk{\ctxext{A}{\mfam{A}}}{A}}}
\jdeq\jcomp{}{\tfext{A}}{\mtm{A}}$. Thirdly, we have the judgmental equality
$\mfam[\ctxext{\Gamma}{A}]{\tfext{A}}\jdeq\idfunc[\jcomp{}{\tfext{A}}{\mfam{A}}]$.
We may combine these three facts with \autoref{hom-over-id-is-hom} to see that
\begin{equation*}
\jhom
  {{{{\Gamma}{A}}{\mfam{A}}}{\jcomp{}{\tfext{A}}}{\mfam{A}}}
  {\jcomp{}{\tfext{A}}{\mtm{A}}}
  {\jcomp{}{\tfext{A}}{\mtm{A}}}
  {\mtm[\ctxext{\Gamma}{A}]{\tfext{A}}}
\end{equation*}
Now we see that we can require that $\mtm[\ctxext{\Gamma}{A}]{\tfext{A}}$ is
the identity term on $\jcomp{}{\tfext{A}}{\mtm{A}}$:
\begin{equation*}
\inference
  { \jfam{\Gamma}{A}
    }
  { \jhomeq
    {{{{\Gamma}{A}}{\mfam{A}}}{\jcomp{}{\tfext{A}}}{\mfam{A}}}
    {\jcomp{}{\tfext{A}}{\mtm{A}}}
    {\jcomp{}{\tfext{A}}{\mtm{A}}}
    {\mtm[\ctxext{\Gamma}{A}]{\tfext{A}}}
    {\idfunc[\jcomp{}{\tfext{A}}{\mtm{A}}]}
    }
\end{equation*}

\subsubsection{Weakening followed by substitution is the identity}
The weakening morphism $\tfwk{A}^0$ gives a weakened morphism
\begin{equation*}
\jhom
  {{{{\Gamma}{A}}{\mfam{A}}}{\mtm{A}}}
  {\ctxwk{\mtm{A}}{{\mfam{A}}{\mfam{A}}}}
  {\ctxwk{\mtm{A}}{\mfam{\mfam{A}}}}
  {\ctxwk{\mtm{A}}{\tfwk{A}^0}}
\end{equation*}
which can be composed with the morphism $\tfsubst{A}^0$. We require that this
composition is the identity term on $\ctxwk{\mtm{A}}{{\mfam{A}}{\mfam{A}}}$:
\begin{equation*}
\inference
  { \jfam{\Gamma}{A}
    }
  { \jhomeq
      {{{{\Gamma}{A}}{\mfam{A}}}{\mtm{A}}}
      {\ctxwk{\mtm{A}}{{\mfam{A}}{\mfam{A}}}}
      {\ctxwk{\mtm{A}}{{\mfam{A}}{\mfam{A}}}}
      {\jcomp{}{\ctxwk{\mtm{A}}{\tfwk{A}^0}}{\tfsubst{A}^0}}
      {\idfunc[\ctxwk{\mtm{A}}{{\mfam{A}}{\mfam{A}}}]}
    }
\end{equation*}
Likewise, the weakening morphism $\tfwk{A}^1$ gives a weakened morphism
\begin{equation*}
\jfhom
  {{{{\Gamma}{A}}{\mfam{A}}}{\mtm{A}}}
  {\ctxwk{\mtm{A}}{{\mfam{A}}{\mfam{A}}}}
  {\ctxwk{\mtm{A}}{\mfam{\mfam{A}}}}
  {\ctxwk{\mtm{A}}{\tfwk{A}^0}}
  {\ctxwk{\mtm{A}}{\ctxwk{\mfam{A}}{\mtm{A}}}}
  {\ctxwk{\mtm{A}}{\mtm{\mfam{\mfam{A}}}}}
  {\ctxwk{\mtm{A}}{\tfwk{A}^1}}
\end{equation*}
which can be composed with the the morphism $\tfsubst{A}^1$. We require that
this composition is the identity term on 
$\ctxwk{\mtm{A}}{\ctxwk{\mfam{A}}{\mtm{A}}}$. Note that this composition gives
a morphism over $\jcomp{}{\ctxwk{\mtm{A}}{\tfwk{A}^0}}{\tfsubst{A}^0}$, which
is the identity term on $\ctxwk{\mtm{A}}{{\mfam{A}}{\mfam{A}}}$; so in fact
it is an ordinary morphism in the context
$\ctxext{{{{\Gamma}{A}}{\mfam{A}}}{\mtm{A}}}{\ctxwk{\mtm{A}}{{\mfam{A}}{\mfam{A}}}}$.
Thus the inference rule we require to become valid is:
\begin{equation*}
\inference
  { \jfam{\Gamma}{A}
    }
  { \jhomeq
      {{{{{\Gamma}{A}}{\mfam{A}}}{\mtm{A}}}{\ctxwk{\mtm{A}}{{\mfam{A}}{\mfam{A}}}}}
      {\ctxwk{\mtm{A}}{\ctxwk{\mfam{A}}{\mtm{A}}}}
      {\ctxwk{\mtm{A}}{\ctxwk{\mfam{A}}{\mtm{A}}}}
      {\jcomp{}{\ctxwk{\mtm{A}}{\tfwk{A}^1}}{\tfsubst{A}^1}}
      {\idfunc[\ctxwk{\mtm{A}}{\ctxwk{\mfam{A}}{\mtm{A}}}]}
    }
\end{equation*}

\subsubsection{Weakening by a family is precomposing with extension}
We seem to need the following rule:
\begin{equation*}
\inference
  { \jfam{{\Gamma}{A}}{P}
    }
  { \jfameq
      {{{\Gamma}{A}}{\mfam{A}}}
      {\ctxwk{\mfam{A}}{P}}
      {\jcomp{}{\tfext{A}}{P}}
    }
\end{equation*}
I don't know very well how to motivate requiring this rule. Maybe it's provable?

\subsubsection{Families of extensions}

\subsubsection{Terms of families of extensions}

\subsubsection{Terms of extensions}

\subsubsection{Families over the empty context}
\begin{equation*}
\inference
  { \jfam{\Gamma}{A}
    }
  { \jfameq{\Gamma}{\subst{\tfemp{A}}{\mfam{A}}}{A}
    }
\end{equation*}

\subsubsection{Terms of the empty family}
\begin{equation*}
\inference
  { \jfam{\Gamma}{A}
    }
  { \jfameq{{\Gamma}{A}}{\subst{\tfemp{\mfam{A}}}{\mtm{A}}}{\emptyf[A]}
    }
\end{equation*}

\subsubsection{The empty family over the empty family}
\begin{equation*}
\inference
  { \jfam{\Gamma}{A}
    }
  { \jtermeq{\Gamma}{A}{\subst{\tfemp{A}}{\tfemp{\mfam{A}}}}{\tfemp{A}}
    }
\end{equation*}

\subsubsection{Extension by the empty family}
To understand the following inference rule, recall that for $\jfam{\Gamma}{A}$
we have the judgmental equalities
\begin{equation*}
\subst{\tfemp{\mfam{A}}}{\ctxwk{\ctxext{A}{\mfam{A}}}{A}}
\jdeq
  \subst{\tfemp{\mfam{A}}}{\ctxwk{\mfam{A}}{{A}{A}}}
\jdeq
  \ctxwk{A}{A}.
\end{equation*}
Therefore, we can postulate:
\begin{equation*}
\inference
  { \jfam{\Gamma}{A}
    }
  { \jhomeq{\Gamma}{A}{A}{\subst{\tfemp{\mfam{A}}}{\tfext{A}}}{\idfunc[A]}
    }
\end{equation*}

\subsubsection{Extensions of the empty context}
To understand the following inference rule, recall that for $\jfam{\Gamma}{A}$
we have the judgmental equalities
\begin{align*}
\subst{\tfemp{A}}{\ctxwk{\ctxext{A}{\mfam{A}}}{A}}
& \jdeq 
  \subst{\tfemp{A}}{\ctxwk{\mfam{A}}{{A}{A}}}
  \\
& \jdeq 
  \ctxwk{\subst{\tfemp{A}}{\mfam{A}}}{\subst{\tfemp{A}}{\ctxwk{A}{A}}}
  \\
& \jdeq 
  \ctxwk{A}{\subst{\tfemp{A}}{\ctxwk{A}{A}}}
  \\
& \jdeq 
  \ctxwk{A}{A}
\end{align*}
Therefore, we can postulate
\begin{equation*}
\inference
  { \jfam{\Gamma}{A}
    }
  { \jhomeq{\Gamma}{A}{A}{\subst{\tfemp{A}}{\tfext{A}}}{\idfunc[A]}
    }
\end{equation*}

\subsubsection{Weakening by the empty family}

\subsubsection{Weakenings of the empty family}

\subsubsection{Substitutions of the empty family}

\subsubsection{Substitution by the term of the empty family}

\subsection{Compatibility properties of arbitrary terms}
In this subsection we will state the inference rules that assert that every
term $\jterm{{\Gamma}{A}}{P}{f}$ acts functorially. That means, every term
preserves the empty context, extension, weakening, substitution and the
identity terms. Moreover, there will be several inference rules
involving the behavior of $\mfam{f}$ and $\mtm{f}$.

\subsubsection{The action on families of families of a term is the action on
families over extensions}
Let $\jterm{{\Gamma}{A}}{P}{f}$. Then we have the term
\begin{equation*}
\jterm{{{\Gamma}{A}}{\mfam{A}}}{\ctxwk{\mfam{A}}{\subst{f}{\mfam{P}}}}{\mfam{f}}
\end{equation*}
and we have the term
\begin{equation*}
\jterm
  { {{{\Gamma}{A}}{\mfam{A}}}{\mfam{\mfam{A}}}
    }
  { \ctxwk
      {\mfam{\mfam{A}}}
      {\subst{\mfam{f}}{\mfam{\ctxwk{\mfam{A}}{\subst{f}{\mfam{P}}}}}}
    }
  { \mfam{\mfam{f}}
    }.
\end{equation*}
We have the judgmental equality $\mfam{\mfam{A}}\jdeq
\jcomp{}{\tfext{A}}{\mfam{A}}$; in this subsubsection we wish to establish a
similar judgmental equality explaining what $\mfam{\mfam{f}}$ is. Note that
we have the judgmental equalities
\begin{align*}
\ctxwk
  {\mfam{\mfam{A}}}
  {\subst{\mfam{f}}{\mfam{\ctxwk{\mfam{A}}{\subst{f}{\mfam{P}}}}}}
& \jdeq
  \ctxwk
    {\jcomp{}{\tfext{A}}{\mfam{A}}}
    {\subst{\mfam{f}}{\mfam{\ctxwk{\mfam{A}}{\subst{f}{\mfam{P}}}}}}
  \\
& \jdeq
  \ctxwk
    {\unfold{\jcomp{\ctxext{A}{\mfam{A}}}{\tfext{A}}{\mfam{A}}}}
    {\subst{\mfam{f}}{\mfam{\ctxwk{\mfam{A}}{\subst{f}{\mfam{P}}}}}}
  \\
& \jdeq
  \ctxwk
    {\subst{\tfext{A}}{\ctxwk{\mfam{A}}{{A}{\mfam{A}}}}}
    {\subst{\mfam{f}}{\mfam{\ctxwk{\mfam{A}}{\subst{f}{\mfam{P}}}}}}
  \\
& \jdeq
  \ctxwk
    { \subst{\tfext{A}}{\ctxwk{\mfam{A}}{{A}{\mfam{A}}}}
      }
    { \subst
        { \tfext{A}
          }
        { \ctxwk
            {A}
            {\subst{\mfam{f}}{\mfam{\ctxwk{\mfam{A}}{\subst{f}{\mfam{P}}}}}}}
      }
  \\
& \jdeq
  \subst
    { \tfext{A}
      }
    { \ctxwk
        { {\mfam{A}}{{A}{\mfam{A}}}
          }
        { {A}
          {\subst{\mfam{f}}{\mfam{\ctxwk{\mfam{A}}{\subst{f}{\mfam{P}}}}}}
          }
      }
  \\
& \jdeq \unfold{\jcomp{{A}{\mfam{A}}}{\tfext{A}}{\mfam{\ctxwk{\mfam{A}}{\subst{f}{\mfam{P}}}}}}
\end{align*}


\subsubsection{Every term is compatible with the empty context}
\begin{equation*}
\inference
  { \jterm{{\Gamma}{A}}{P}{f}
    }
  { \jtermeq
      {\Gamma}
      {\subst{\tfemp{A}}{P}}
      {\subst{\tfemp{A}}{f}}
      {\tfemp{\subst{\tfemp{A}}{P}}}
    }
\end{equation*}

\subsubsection{The action on families of the action on families of a term is the action
on contexts of that term}
\emph{In this subsubsection we attempted to establish a rule asserting that
$\subst{\tfemp{A}}{\mfam{f}}\jdeq f$ for any $\jterm{{\Gamma}{A}}{P}{f}$. This
is however not true.}

Consider a term $\jterm{{\Gamma}{A}}{P}{f}$. Then we have the term
\begin{equation*}
\jterm
  {{\Gamma}{\subst{\tfemp{A}}{\mfam{A}}}}
  {\subst{\tfemp{A}}{\ctxwk{\mfam{A}}{\subst{f}{\mfam{P}}}}}
  {\subst{\tfemp{A}}{\mfam{f}}}
\end{equation*}
We have required that $\subst{\tfemp{A}}{\mfam{A}}\jdeq A$. Also, we have the
judgmental equalities
\begin{align*}
\subst{\tfemp{A}}{\ctxwk{\mfam{A}}{\subst{f}{\mfam{P}}}}
& \jdeq
  \ctxwk{\subst{\tfemp{A}}{\mfam{A}}}{\subst{\tfemp{A}}{{f}{\mfam{P}}}}
  \\
& \jdeq
  \ctxwk{A}{\subst{{\tfemp{A}}{f}}{{\tfemp{A}}{\mfam{P}}}}
  \\
& \jdeq
  \ctxwk{A}{\subst{\tfemp{\subst{\tfemp{A}}{P}}}{{\tfemp{A}}{\mfam{P}}}}
  \\
& \jdeq
  \ctxwk{A}{\subst{{\tfemp{A}}{\tfemp{P}}}{{\tfemp{A}}{\mfam{P}}}}
  \\
& \jdeq
  \ctxwk{A}{\subst{\tfemp{A}}{{\tfemp{P}}{\mfam{P}}}}
  \\
& \jdeq
  \ctxwk{A}{\subst{\tfemp{A}}{P}}
\end{align*}
If we require also the following inference rule, we would get $\mfam{A}\jdeq
\ctxwk{A}{A}$.
\begin{equation*}
\inference
  { \jfam{{\Gamma}{A}}{P}
    }
  { \jfameq{{\Gamma}{A}}{P}{\ctxwk{A}{\subst{\tfemp{A}}{P}}}
    }
\end{equation*}
This rule would say that a family $P$ of models over $A$ is determined by the
model $\subst{\tfemp{A}}{P}$ and that the only possible families of models
are the constant families.

\subsubsection{Every term is compatible with extension}
Let $\jterm{{\Gamma}{A}}{P}{f}$. The rule we're about to explain is that $f$
commutes with extension.

We have the morphism
$\jhom{{\Gamma}{A}}{\mfam{A}}{\subst{f}{\mfam{P}}}{\mfam{f}}$. Let's unfold
to remind ourselves what this means:
\begin{equation*}
\unfold{\jhom{{\Gamma}{A}}{\mfam{A}}{\subst{f}{\mfam{P}}}{\mfam{f}}}
\end{equation*}
Now note that we have the judgmental equality
\begin{equation*}
  \ctxwk{\mfam{A}}{\subst{f}{\mfam{P}}}
  \jdeq
  \subst{\ctxwk{\mfam{A}}{f}}{\ctxwk{\mfam{A}}{\mfam{P}}}
\end{equation*}
and we see that we get the term
$ \tmext
    {\ctxwk{\mfam{A}}{P}}
    {\ctxwk{\mfam{A}}{\mfam{P}}}
    {\ctxwk{\mfam{A}}{f}}
    {\mfam{f}}
  $
of the family $\ctxwk{\mfam{A}}{\ctxext{P}{\mfam{P}}}$. We can substitute this
term in the morphism
\begin{equation*}
\jhom
  {{{\Gamma}{A}}{\mfam{A}}}
  {\ctxwk{\mfam{A}}{\ctxext{P}{\mfam{P}}}}
  {\ctxwk{\mfam{A}}{P}}
  {\ctxwk{\mfam{A}}{\tfext{P}}}
\end{equation*}
to obtain the term 
$ \jterm
    {{{\Gamma}{A}}{\mfam{A}}}
    {\ctxwk{\mfam{A}}{P}}
    { \subst
        { \tmext
            {\ctxwk{\mfam{A}}{P}}
            {\ctxwk{\mfam{A}}{\mfam{P}}}
            {\ctxwk{\mfam{A}}{f}}
            {\mfam{f}}
          }
        { \ctxwk{\mfam{A}}{\tfext{P}}
          }
      }.
  $
We can also compose $\tfext{A}$ with $f$ to obtain the term
\begin{equation*}
\jterm{{{\Gamma}{A}}{\mfam{f}}}{\jcomp{}{\tfext{A}}{P}}{\jcomp{}{\tfext{A}}{f}}
\end{equation*}
Note that we have (?) the judgmental equality $\jcomp{}{\tfext{A}}{P}
\jdeq\ctxwk{\mfam{A}}{P}$ and therefore
we can require:
\begin{equation*}
\inference
  { \jterm{{\Gamma}{A}}{P}{f}
    }
  { \jtermeq
       {{{\Gamma}{A}}{\mfam{A}}}
       {\ctxwk{\mfam{A}}{P}}
       { \subst
           { \tmext
               {\ctxwk{\mfam{A}}{P}}
               {\ctxwk{\mfam{A}}{\mfam{P}}}
               {\ctxwk{\mfam{A}}{f}}
               {\mfam{f}}
             }
           { \ctxwk{\mfam{A}}{\tfext{P}}
             }
         }
       {\jcomp{}{\tfext{A}}{f}}
    }
\end{equation*}

\subsubsection{Every term is compatible with weakening}
Let $\jterm{{\Gamma}{A}}{P}{f}$ be a term. We will establish an inference rule
asserting that $f$ commutes with weakening. 

\subsubsection{Every term is compatible with substitution}

\subsubsection{Every term preserves identity}


\section{Pretty type theory}
In this section we will do three things. First we explain basic type theory with
variable names, which we simply call \emph{Pretty Type Theory}. 
Then we will show how every formula in basic type theory without
variable names (shall we call it \emph{Structural Type Theory}?) 
can be interpreted in pretty type theory. Finally, we will show how every formula
in pretty type theory can be interpreted in structural type theory.

Pretty type theory is pretty as the ugly little duckling that grew up. At first, it is actually uglier than
structural type theory because one has to keep track of a variable nobody is
actually interested in. The situation improves when we introduce the operators
of extension and weakening, because from that point onwards one can draw notational
advantages from having the variable around.

Pretty type theory should be such that if you write down two contexts, families
or terms in exactly the same way, then they are the same, and there should be
notational shortcuts for extension, weakening and substitution which make
this interesting.

\subsection{The basic judgments}
The basic judgments of pretty type theory are the same as for structural type
theory. There are judgments for: ``$\Gamma$ is a context'',
``$A(i)$ over $i:\Gamma$ is a family over $\Gamma$'', ``$A(i)$ over $i:\Gamma$ 
is a type in context $\Gamma$''
and ``$x(i)$ is a term of $A(i)$ above $i:\Gamma$''. The other four
judgments are for judgmental equality. 

\begin{align*}
\jvctx*{\Gamma} & \jvctxeq*{\Gamma}{\Gamma'}\\
\jvfam*{i}{\Gamma}{A} & \jvfameq*{i}{\Gamma}{A}{B}\\
\jvtype*{i}{\Gamma}{A} & \jvtypeeq*{i}{\Gamma}{A}{B}\\
\jvterm*{i}{\Gamma}{A}{x} & \jvtermeq*{i}{\Gamma}{A}{x}{y}.
\end{align*}

We have the following basic inference rules that relate types and families:

\begin{small}
\begin{align*}
& \inference
  {\jvtype{i}{\Gamma}{A}}
  {\jvfam{i}{\Gamma}{A}}
& & \inference
    {\jvtypeeq{i}{\Gamma}{A}{B}}
    {\jvfameq{i}{\Gamma}{A}{B}}\\
& \inference
  {\jvtype{i}{\Gamma}{A}
   \jvfameq{i}{\Gamma}{A}{B}}
  {\jvtype{i}{\Gamma}{B}}
& & \inference
    {\jvtype{i}{\Gamma}{A}
     \jvfameq{i}{\Gamma}{A}{B}}
    {\jvtypeeq{i}{\Gamma}{A}{B}}
\end{align*}
\end{small}

\subsection{The basic rules for judgmental equality}
The rules for judgmental equality establish that it is an equivalence relation.
\bgroup\small
\begin{align*}
& \inference
  {\jvctx{\Gamma}}
  {\jvctxeq{\Gamma}{\Gamma}} 
& & \inference
    {\jvctxeq{\Gamma}{\Delta}}
    {\jvctxeq{\Delta}{\Gamma}} 
& & \inference
    {\jvctxeq{\Gamma}{\Delta}
     \jvctxeq{\Delta}{\greek{E}}}
    {\jvctxeq{\Gamma}{\greek{E}}}\\
& \inference
  {\jvfam{i}{\Gamma}{A}}
  {\jvfameq{i}{\Gamma}{A}{A}} 
& & \inference
    {\jvfameq{i}{\Gamma}{A}{B}}
    {\jvfameq{i}{\Gamma}{B}{A}}
& & \inference
    {\jvfameq{i}{\Gamma}{A}{B}
     \jvfameq{i}{\Gamma}{B}{C}}
    {\jvfameq{i}{\Gamma}{A}{C}}\\
& \inference
  {\jvterm{i}{\Gamma}{A}{x}}
  {\jvtermeq{i}{\Gamma}{A}{x}{x}}
& & \inference
    {\jvtermeq{i}{\Gamma}{A}{x}{y}}
    {\jvtermeq{i}{\Gamma}{A}{y}{x}}
& & \inference
    {\jvtermeq{i}{\Gamma}{A}{x}{y}
     \jvtermeq{i}{\Gamma}{A}{y}{z}}
    {\jvtermeq{i}{\Gamma}{A}{x}{z}}
\end{align*}
\egroup

The following convertibility rules are responsible for the strictness
of judgmental equality, which sets it apart from equivalences or identifications:

\begin{align*}
& \inference
  {\jvctxeq{\Gamma}{\Delta}
   \jvfam{i}{\Gamma}{A}}
  {\jvfam{i}{\Delta}{A}}
& & \inference
    {\jvctxeq{\Gamma}{\Delta}
     \jvfameq{i}{\Gamma}{A}{B}}
    {\jvfameq{i}{\Delta}{A}{B}}\\
& \inference
  {\jvctxeq{\Gamma}{\Delta}
   \jvterm{i}{\Gamma}{A}{x}}
  {\jvterm{i}{\Delta}{A}{x}}
& & \inference
    {\jvctxeq{\Gamma}{\Delta}
     \jvtermeq{i}{\Gamma}{A}{x}{y}}
    {\jvtermeq{i}{\Delta}{A}{x}{y}}\\
& \inference
  {\jvfameq{i}{\Gamma}{A}{B}
   \jvterm{i}{\Gamma}{A}{x}}
  {\jvterm{i}{\Gamma}{B}{x}}
& & \inference
    {\jvfameq{i}{\Gamma}{A}{B}
     \jvtermeq{i}{\Gamma}{A}{x}{y}}
    {\jvtermeq{i}{\Gamma}{B}{x}{y}}
\end{align*}

\subsection{The empty context}
The empty context looks a bit strange when we explicitly denote the terms. But
we will not do so anymore after this subsection.

\begin{align}
& \inference
  {}
  {\jctx{\emptyc}}\\
& \inference
  {\jctx{\Gamma}}
  {\jvfam{i}{\Gamma}{\emptyf[\Gamma]}}\\
& \inference
  {\jctx{\Gamma}}
  {\jvterm{i}{\Gamma}{\emptyf[\Gamma]}{\emptytm[\Gamma]}}\\
& \inference
  {\jvterm{i}{\Gamma}{\emptyf[\Gamma]}{x}}
  {\jvtermeq{i}{\Gamma}{\emptyf[\Gamma]}{x}{\emptytm[\Gamma]}}
\end{align}

Moreover, if $\Gamma$ is a context family over the
empty context, then $\Gamma$ is a context and every context is a context
family over the empty context. Note that this allows us to speak
of terms of contexts too.

\begin{align}
& \inference
  {\jctx{\Gamma}}
  {\jvfam{\nameless}{\emptyc}{\Gamma}} 
& & \inference
    {\jvfam{\nameless}{\emptyc}{\Gamma}}
    {\jctx{\Gamma}}\\
& \inference
  {\jctxeq{\Gamma}{\Delta}}
  {\jvfameq{\nameless}{\emptyc}{\Gamma}{\Delta}}
& & \inference
    {\jvfameq{\nameless}{\emptyc}{\Gamma}{\Delta}}
    {\jctxeq{\Gamma}{\Delta}}
\end{align}

\subsubsection{The empty context is compatible with itslef}
The empty context $\emptyc$ may be considered as a family of contexts over the empty
context. When we do this, we get $\emptyf[\emptyc]$.
\begin{equation}
\inference
  {}
  {\jvfameq{\nameless}{\emptyc}{\emptyc}{\emptyf[\emptyc]}}
\end{equation}
In the future, we shall denote $\emptyf[\Gamma]$ by $\emptyf$. The above rule
guarantees that this will not cause confusion. Likewise, we shall denote
$\emptytm[\Gamma]$ by $\emptytm$.

\subsection{Extension}
We introduce extension which not only extends a context $\Gamma$ and a family
$A$ over it to a context $\ctxext{\Gamma}{A}$, but which also extends a family $A$
in context $\Gamma$ and a family $P$ over it to a family $\ctxext{A}{P}$ over context
$\Gamma$. We do this to ensure that all of type theory can be done in a context.
For instance, we could say (1) that a context in context $\Gamma$ is the same thing
as a family over $\Gamma$; (2) When $A$ is a context in this sense, a family over
$A$ is the same thing as a family $P$ over $\ctxext{\Gamma}{A}$ and 
(3) when $P$ is a family over $A$ in this sense, a term of $P$ keeps its original meaning.

\begin{align}
& \inference
  {\jvfam{i}{\Gamma}{A}}
  {\jvfamcombi{{i}{x}}{{\Gamma}{A}}{P}}
& & \inference
    {\jctxeq{\Gamma}{\Delta}
     \jfameq{\Gamma}{A}{B}}
    {\jctxeq{\ctxext{\Gamma}{A}}{\ctxext{\Delta}{B}}}\\
& \inference
  {\jfam{{\Gamma}{A}}{P}}
  {\jfam{\Gamma}{\ctxext{A}{P}}}
& & \inference
    {\jfameq{\Gamma}{A}{B}
     \jfameq{{\Gamma}{A}}{P}{Q}}
    {\jfameq{\Gamma}{\ctxext{A}{P}}{\ctxext{B}{Q}}}
\end{align}

\subsubsection{Extension is compatible with the empty context}
The following rule asserts that extension by $\emptyc$ leaves the contexts unchanged.
\begin{align}
& \inference
  {\jctx{\Gamma}}
  {\jctxeq{\ctxext{\emptyc}{\Gamma}}{\Gamma}}\\
& \inference
  {\jctx{\Gamma}}
  {\jctxeq{\ctxext{\Gamma}{\emptyf}}{\Gamma}}\\
& \inference
  {\jfam{\Gamma}{A}}
  {\jfameq{\Gamma}{\ctxext{\emptyf}{A}}{A}}
\end{align}

\subsubsection{Extension is compatible with itself}
The inference rules asserting that extension is compatible with itself assert
that contexts are unstructured lists of type declarations. This rule is
unavoidable if we want that for a family $A$ in context $\Gamma$, a family over
$A$ is the same thing as a family over $\ctxext{\Gamma}{A}$. 

\begin{align}
& \inference
  {\jfam{\Gamma}{A}
   \jfam{{\Gamma}{A}}{P}}
  {\jctxeq{\ctxext{{\Gamma}{A}}{P}}{\ctxext{\Gamma}{{A}{P}}}}\\
& \inference
  {\jfam{{\Gamma}{A}}{P}
   \jfam{{{\Gamma}{A}}{P}}{Q}}
  {\jfameq{\Gamma}{\ctxext{{A}{P}}{Q}}{\ctxext{A}{{P}{Q}}}}
\end{align}


\section{Chalmers type theory}
In this section we will present the type theory that Coquand and Dybjer are using.
It is a weak type theory (I think), with not so many operations and judgmental equalities.
We will show how every formula of Chalmers type theory can be interpreted in
structural type theory.


\part{Type constructors}

\section{The dependent function constructor}
As we have done with most everything so far, we will take the point of view that
the dependent function constructor is an operation with an action on contexts,
on families and on terms. When $A$ is a family of contexts over $\Gamma$,
$\mprd{A}{\blank}$ takes things in context $\ctxext{\Gamma}{A}$ to the context
$\Gamma$. It's action on terms restricted to the empty context is usually 
denoted by $\lambda$, so we shall denote the entire action by $\lambda$. As
usual, $\lambda$ is reversed by evaluation. 

The dependent function type constructor not only acts on families $P$ of
contexts over $\ctxext{\Gamma}{A}$, but it also sends families $Q$ of contexts
over $\ctxext{{\Gamma}{A}}{P}$ to families $\mprd[\famsym]{A}{Q}$ over $\ctxext{\Gamma}
{\mprd{A}{P}}$. Moreover, we will also have a version of $\lambda$-abstraction
for terms of such families $Q$.

\begin{align*}
& \inference
    { \jfam{{\Gamma}{A}}{P}
      }
    { \jfam{\Gamma}{\mprd{A}{P}}
      }
& & \inference
      { \jfameq{\Gamma}{A}{A'}
        \jfameq{{\Gamma}{A}}{P}{P'}
        }
      { \jfameq{\Gamma}{\mprd{A}{P}}{\mprd{A'}{P'}}
        }
  \\
& \inference
    { \jfam{{{\Gamma}{A}}{P}}{Q}
      }
    { \jfam{{\Gamma}{\mprd{A}{P}}}{\mprd[\famsym]{A}{Q}}
      }
& & \inference
      { \jfameq{\Gamma}{A}{A'}
        \jfameq{{{\Gamma}{A}}{P}}{Q}{Q'}
        }
      { \jfameq{{\Gamma}{\mprd{A}{P}}}{\mprd[\famsym]{A}{Q}}{\mprd[\famsym]{A'}{Q'}}
        }
  \\
& \inference
    { \jterm{{{\Gamma}{A}}{P}}{Q}{g}
      }
    { \jterm{{\Gamma}{\mprd{A}{P}}}{\mprd[\famsym]{A}{Q}}{\slam{A}{Q}{g}}
      }
& & \inference
      { \jfameq{\Gamma}{A}{A'}
        \jtermeq{{{\Gamma}{A}}{P}}{Q}{g}{g'}
        }
      { \jtermeq{{\Gamma}{\mprd{A}{P}}}{\mprd[\famsym]{A}{Q}}{\slam{A}{Q}{g}}{\slam{A'}{Q}{g'}}
        }
  \\
& \inference
    { \jterm{{\Gamma}{\mprd{A}{P}}}{\mprd[\famsym]{A}{Q}}{g}
      }
    { \jterm{{{\Gamma}{A}}{P}}{Q}{\sev{A}{Q}{g}}
      }
& & \inference
      { \jfameq{\Gamma}{A}{A'}
        \jtermeq{{\Gamma}{\mprd{A}{P}}}{\mprd[\famsym]{A}{Q}}{g}{g'}
        }
      { \jtermeq{{{\Gamma}{A}}{P}}{Q}{\sev{A}{Q}{g}}{\sev{A'}{Q}{g'}}
        }
\end{align*}

\begin{defn}
Let $A$ and $B$ be families of contexts over $\Gamma$. Then we define
\begin{equation*}
\jfamdefn{\Gamma}{\jfun{A}{B}}{\sprd{A}{\ctxwk{A}{B}}}.
\end{equation*}
\end{defn}

\subsection{The compatibility rules for the dependent product constructor}
In this subsection we lay out the compatibility rules which we will require
for the dependent product constructor.

\subsubsection{Dependent products are compatible with the empty context}
The empty context can appear in the domain and in the codomain of the dependent
function type constructor. We have the following inference rules explaining
what happens when the empty context appears in the domain:
\begin{align}
& \inference
    { \jfam{\Gamma}{P}
      }
    { \jfameq{\Gamma}{\mprd{\emptyf}{P}}{P}
      }
  \\
& \inference
    { \jfam{{\Gamma}{P}}{Q}
      }
    { \jfameq{{\Gamma}{P}}{\mprd[\famsym]{\emptyf}{Q}}{Q}
      }
  \\
& \inference
    { \jterm{{\Gamma}{P}}{Q}{g}
      }
    { \jtermeq{{\Gamma}{P}}{Q}{\slam{\emptyf}{Q}{g}}{g}
      }
  \\
& \inference
    { \jterm{{\Gamma}{P}}{Q}{g}
      }
    { \jtermeq{{\Gamma}{P}}{Q}{\sev{\emptyf}{Q}{g}}{g}
      }
\end{align}
We have the following infernece rules explaining what happens when the empty
context appears in the codomain:
\begin{align}
& \inference
    { \jfam{\Gamma}{A}
      }
    { \jfameq{\Gamma}{\mprd{A}{\emptyf}}{\emptyf}
      }
  \\
& \inference
    { \jfam{{\Gamma}{A}}{P}
      }
    { \jfameq{{\Gamma}{\mprd{A}{P}}}{\mprd[\famsym]{A}{\emptyf}}{\emptyf}
      }
\end{align}
Now we see that we can compare $\mprd{A}{P}$ with $\mprd[\famsym]{A}{P}$ by
seeing $P$ as a family over $\ctxext{{\Gamma}{A}}{\emptyf}$. We will require
the following inference rule to be valid:
\begin{equation}
\inference
  { \jfam{{\Gamma}{A}}{P}
    }
  { \jfameq{\Gamma}{\mprd{A}{P}}{\mprd[\famsym]{A}{P}}
    }
\end{equation}
This rule will allow us to omit the annotation indicating the action on families,
which we will do from now on. We also note that because the empty family over
$\ctxext{\Gamma}{A}$ is mapped to the empty family over $\Gamma$ by the
dependent function constructor, we obtain the important special cases of
lambda abstraction and evaluation that
\begin{align*}
& \inference
  { \jterm{{\Gamma}{A}}{P}{f}
    }
  { \jterm{\Gamma}{\mprd{A}{P}}{\slam{A}{P}{f}}
    }
  \\
& \inference
  { \jterm{\Gamma}{\mprd{A}{P}}{f}
    }
  { \jterm{{\Gamma}{A}}{P}{\sev{A}{P}{f}}
    }
\end{align*}
Thus, we retrieve what is originally meant by lambda abstraction.

\subsubsection{Dependent products are compatible with extension}
\begin{align}
& \inference
  { \jfam{{{{\Gamma}{A}}{B}}{P}}{Q}
    }
  { \jfameq
      {\Gamma}
      {\sprd{{A}{B}}{{P}{Q}}}
      {\ctxext{\sprd{A}{\sprd{B}{P}}}{\sprd{A}{\sprd{B}{Q}}}}
    }
  \\
& \inference
  { \jterm{{{\Gamma}{A}}{B}}{P}{f}
    }
  { \jtermeq
      {\Gamma}
      {\sprd{A}{\sprd{B}{P}}}
      {\slam{{A}{B}}{P}{f}}
      {\slam{A}{\sprd{B}{P}}{\slam{B}{P}{f}}}
    }
\end{align}

\subsubsection{Weakening is compatible with the dependent product constructor}

\begin{align*}
& \inference
  { \jfam{\Gamma}{A}
    \jfam{{{\Gamma}{B}}{Q}}{R}
    }
  { \jfameq
      {{{\Gamma}{A}}{\ctxwk{A}{B}}}
      {\ctxwk{A}{\mprd{Q}{R}}}
      {\mprd{\ctxwk{A}{Q}}{\ctxwk{A}{R}}}
    }
  \\
& \inference
  { \jfam{\Gamma}{A}
    \jfam{{{{\Gamma}{B}}{Q}}{R}}{S}
    }
  { \jfameq
      {{{{\Gamma}{A}}{\ctxwk{A}{B}}}{\ctxwk{A}{\mprd{Q}{R}}}}
      {\ctxwk{A}{\mprd{Q}{S}}}
      {\mprd{\ctxwk{A}{Q}}{\ctxwk{A}{S}}}
    }
  \\
& \inference
  { \jfam{\Gamma}{A}
    \jterm{{{{\Gamma}{B}}{Q}}{R}}{S}{k}
    }
  { \jtermeq
      {{{{\Gamma}{A}}{\ctxwk{A}{B}}}{\ctxwk{A}{\mprd{Q}{R}}}}
      {\ctxwk{A}{\mprd{Q}{S}}}
      {\ctxwk{A}{\slam{Q}{S}{k}}}
      {\slam{\ctxwk{A}{Q}}{\ctxwk{A}{S}}{\ctxwk{A}{k}}}
    }
  \\
& \inference
  { \jfam{\Gamma}{A}
    \jterm{{{\Gamma}{B}}{\mprd{Q}{R}}}{\mprd{Q}{S}}{k}
    }
  { \jtermeq
      {{{{{\Gamma}{A}}{\ctxwk{A}{B}}}{\ctxwk{A}{Q}}}{\ctxwk{A}{R}}}
      {\ctxwk{A}{S}}
      {\ctxwk{A}{\sev{Q}{S}{k}}}
      {\sev{\ctxwk{A}{Q}}{\ctxwk{A}{S}}{\ctxwk{A}{k}}}
    }
\end{align*}

\subsection{Dependent products are taken fiberwise}
The rules explaining that substitution is compatible with the dependent product
constructor assert that dependent products are taken fiberwise (as is usual).

\begin{align*}
& \inference
  { \jterm{\Gamma}{A}{x}
    \jfam{{{{\Gamma}{A}}{P}}{Q}}{R}
    }
  { \jfameq
      {{\Gamma}{\subst{x}{P}}}
      {\subst{x}{\mprd{Q}{R}}}
      {\mprd{\subst{x}{Q}}{\subst{x}{R}}}
    }
  \\
& \inference
  { \jterm{\Gamma}{A}{x}
    \jfam{{{{{\Gamma}{A}}{P}}{Q}}{R}}{S}
    }
  { \jfameq
      {{{\Gamma}{\subst{x}{P}}}{\mprd{\subst{x}{Q}}{\subst{x}{R}}}}
      {\subst{x}{\mprd{Q}{S}}}
      {\mprd{\subst{x}{Q}}{\subst{x}{S}}}
    }
  \\
& \inference
  { \jterm{\Gamma}{A}{x}
    \jterm{{{{{\Gamma}{A}}{P}}{Q}}{R}}{S}{k}
    }
  { \jtermeq
      {{{\Gamma}{\subst{x}{P}}}{\mprd{\subst{x}{Q}}{\subst{x}{R}}}}
      {\subst{x}{\mprd{Q}{S}}}
      {\subst{x}{\slam{Q}{S}{k}}}
      {\slam{\subst{x}{Q}}{\subst{x}{S}}{\subst{x}{k}}}
    }
  \\
& \inference
  { \jterm{\Gamma}{A}{x}
    \jterm{{{{\Gamma}{A}}{P}}{\mprd{Q}{R}}}{\mprd{Q}{S}}{k}
    }
  { \jtermeq
      {{{{\Gamma}{\subst{x}{P}}}{\subst{x}{Q}}}{\subst{x}{R}}}
      {\subst{x}{S}}
      {\subst{x}{\sev{Q}{S}{k}}}
      {\sev{\subst{x}{Q}}{\subst{x}{S}}{\subst{x}{k}}}
    } 
\end{align*}

\subsubsection{Dependent products are compatible with the identity functions}?


\section{Universes}\label{sec:universes}

\subsection{Pre-universes}
We would like to investigate the universe operator, which assigns to a family
$A$ over $\Gamma$ and a family $P$ over $\ctxext{\Gamma}{A}$ a universe
$\UU(A,P)$, which is a family over $\Gamma$, and a decoding $\tilde{\UU}(A,P)$,
which is a family over $\ctxext{\Gamma}{\UU(A,P)}$. However, we find it useful
to first look at a pre-universe operator which extends the pair $(A,P)$ to a
quadruple $(V_0(A,P),\tilde{V}_0(A,P),V_1(A,P),\tilde{V}_1(A,P))$ consisting of
a family $V_0(A,P)$ in context $\Gamma$, a family $\tilde{V}_0(A,P)$ over
$\ctxext{\Gamma}{V_0(A,P)}$, a family $V_1(A,P)$ over 
$\ctxext{\Gamma}{V_0(A,P)}$ and a family $\tilde{V}_1(A,P)$ over.... and only
later blur the distinction between $\subst{x}{V_1(A,P)}$ and family morphisms
from $\subst{x}{\tilde{V}_0(A,P)}$ to $V_0(A,P)$ for terms $x:V_0(A,P)$. The
pair $(V_0(A,P),V_1(A,P))$ can be seen as a version of $(A,P)$ which is
expanded to incorporate extension, the empty context, weakening, substitution
and identity terms.

Our initial setup for pre-universes consists just of the two inference rules
calling for $V_0$ and $V_1$:
\begin{align*}
& \inference
  { \jfam{{\Gamma}{A}}{P}
    }
  { \jfam{\Gamma}{V_0(A,P)}
    }
  \\
& \inference
  { \jfam{{\Gamma}{A}}{P}
    }
  { \jfam{{\Gamma}{V_0(A,P)}}{V_1(A,P)}
    }
\end{align*}

\subsection{Universe operators}
\label{universes}
Now we investigate universes. When $A$ is a family over context $\Gamma$ and
$P$ is a family over $\ctxext{\Gamma}{A}$, the universe at $(A,P)$ consists,
among other things, of a family $\UU_0(A,P)$ over context $\Gamma$ and a family
$\UU_1(A,P)$

The type operator $\UU$ has as input a pair $(A,P)$ consisting of a family
$A$ in context $\Gamma$ and a family $P$ in context $\ctxext{\Gamma}{A}$. Its
output is a pair $(U(A,P),\tilde{U}(A,P))$ consisting of a family
$U(A,P)$ in context $\Gamma$ and a family $\tilde{U}(A,P)$ over it for which
there are terms representing the empty context, the operations of extension,
weakening and substitution and the identity terms. Moreover there is a morphism
$i$ from $A$ to $U(A,P)$ with the property that 
$\jcomp{}{i}{\tilde{U}(A,P)}\jdeq P$.


% \part{Inductive constructions}

\section{Inductive constructions}

\subsection{Strong inductive morphisms}
In this subsection we investigate the notion of inductive morphisms. An 
inductive morphism $f$ from $A$ to $B$ in context $\Gamma$ is a morphism which
induces an operation which is judgmentally the inverse of composition
with $f$. We explore a notion of inductive morphism which is much stronger
than the usual notion: it pushes families over $A$ to families over $B$,
families over families over $A$ to families over families over $B$ and terms
thereof to terms of the output families over families in a manner compatible
with the empty family, extension, weakening, substitution and identity terms.
As a result, inductive morphisms will be stable under extension, weakening,
substitution and the identity term is an inductive morphism.

Inductive morphisms are useful to for inductive types which are defined using
only one (ordinary morphism as) constructor, like the unit type and dependent 
pair types.
They can't be used to define the empty type, disjoint sums,
the natural numbers or identity types. 

{\color{red} Maybe
we can solve this by writing down a type theory of inductive constructions.}

\begin{defn}
Let $\jhom{\gamma}{A}{B}{f}$ be a context morphism. We say that $f$ is an
\emph{inductive morphism} if there is an operation $\tfindf{f}$ given by
\begin{align*}
& \inference
  { \jfam{{\Gamma}{A}}{P}
    }
  { \jfam{{\Gamma}{B}}{\tfind{f}{P}}
    }
  \\
& \inference
  { \jfam{{{\Gamma}{A}}{P}}{Q}
    }
  { \jfam{{{\Gamma}{B}}{\tfind{f}{P}}}{\tfind[\famsym]{f}{Q}}
    }
  \\
& \inference
  { \jterm{{{\Gamma}{A}}{P}}{Q}{g}
    }
  { \jterm{{{\Gamma}{B}}{\tfind{f}{P}}}{\tfind[\famsym]{f}{Q}}{\tfind[\tmsym]{f}{g}}
    }
\end{align*}
for which the inference rules in the following list are valid:
\begin{enumerate}
\item The operation $\tfindf{f}$ is compatible with the empty context:
\begin{align*}
& \inference
  { 
    }
  { \jfameq{{\Gamma}{B}}{\tfind{f}{\emptyf}}{\emptyf}
    }
  \\
& \inference
  { \jfam{{\Gamma}{A}}{P}
    }
  { \jfameq{{{\Gamma}{B}}{\tfind{f}{P}}}{\tfind[\famsym]{f}{\emptyf}}{\emptyf}
    }
\end{align*}
\item The action on families $\tfindf[\famsym]{f}$ of $\tfindf{f}$ is compatible
with the action on contexts:
\begin{equation*}
\inference
  { \jfam{{\Gamma}{A}}{P}
    }
  { \jfameq{{\Gamma}{B}}{\tfind[\famsym]{f}{P}}{\tfind{f}{P}}
    }
\end{equation*}
Because of this inference rule we shall henceforth omit the annotations
$\famsym$ and $\tmsym$ from the operation $\tfindf{f}$ as usual.
\item The operation $\tfindf{f}$ is compatible with extension:
\begin{align*}
& \inference
  { \jfam{{{\Gamma}{A}}{P}}{Q}
    }
  { \jfameq
      {{\Gamma}{B}}
      {\tfind{f}{\ctxext{P}{Q}}}
      {\ctxext{\tfind{f}{P}}{\tfind{f}{Q}}}
    }
  \\
& \inference
  { \jfam{{{{\Gamma}{A}}{P}}{Q}}{R}
    }
  { \jfameq
      {{{\Gamma}{B}}{\tfind{f}{P}}}
      {\tfind{f}{\ctxext{Q}{R}}}
      {\ctxext{\tfind{f}{Q}}{\tfind{f}{R}}}
    }
\end{align*}
\item The operation $\tfindf{f}$ is compatible with weakening:
\begin{align*}
& \inference
  { \jfam{{\Gamma}{A}}{P}
    \jfam{{\Gamma}{A}}{Q}
    }
  { \jfameq
      {{{\Gamma}{B}}{\tfind{f}{P}}}
      {\tfind{f}{\ctxwk{P}{Q}}}
      {\ctxwk{\tfind{f}{P}}{\tfind{f}{Q}}}
    }
  \\
& \inference
  { \jfam{{\Gamma}{A}}{P}
    \jfam{{{\Gamma}{A}}{Q}}{R}
    }
  { \jfameq
      {{{{\Gamma}{B}}{\tfind{f}{P}}}{\ctxwk{\tfind{f}{P}}{\tfind{f}{Q}}}}
      {\tfind{f}{\ctxwk{P}{R}}}
      {\ctxwk{\tfind{f}{P}}{\tfind{f}{R}}}
    }
  \\
& \inference
  { \jfam{{\Gamma}{A}}{P}
    \jterm{{{\Gamma}{A}}{Q}}{R}{h}
    }
  { \jtermeq
      {{{{\Gamma}{B}}{\tfind{f}{P}}}{\ctxwk{\tfind{f}{P}}{\tfind{f}{Q}}}}
      {\ctxwk{\tfind{f}{P}}{\tfind{f}{R}}}
      {\tfind{f}{\ctxwk{P}{h}}}
      {\ctxwk{\tfind{f}{P}}{\tfind{f}{h}}}
    }
\end{align*}
\item We will also require that the operation $\tfindf{f}$ is compatible with
weakening by $A$:
\begin{align*}
& \inference
  { \jfam{\Gamma}{X}
    }
  { \jfameq
      {{\Gamma}{B}}
      {\tfind{f}{\ctxwk{A}{X}}}
      {\ctxwk{B}{X}}
    }
  \\
& \inference
  { \jfam{{\Gamma}{X}}{Y}
    }
  { \jfameq
      {{{\Gamma}{B}}{\ctxwk{B}{X}}}
      {\tfind{f}{\ctxwk{A}{Y}}}
      {\ctxwk{B}{Y}}
    }
  \\
& \inference
  { \jterm{{\Gamma}{X}}{Y}{y}
    }
  { \jtermeq
      {{{\Gamma}{B}}{\ctxwk{B}{X}}}
      {\ctxwk{B}{Y}}
      {\tfind{f}{\ctxwk{A}{y}}}
      {\ctxwk{B}{y}}
    }
\end{align*}
These rules assert that constant families and terms are mapped to constant
families and terms.
\item The operation $\tfindf{f}$ is compatible with substitution:
\begin{align*}
& \inference
  { \jterm{{{\Gamma}{A}}{P}}{Q}{g}
    \jfam{{{{\Gamma}{A}}{P}}{Q}}{R}
    }
  { \jfameq
      {{{\Gamma}{B}}{\tfind{f}{P}}}
      {\tfind{f}{\subst{g}{R}}}
      {\subst{\tfind{f}{g}}{\tfind{f}{R}}}
    }
  \\
& \inference
  { \jterm{{{\Gamma}{A}}{P}}{Q}{g}
    \jfam{{{{{\Gamma}{A}}{P}}{Q}}{R}}{S}
    }
  { \jfameq
      {{{{\Gamma}{B}}{\tfind{f}{P}}}{\subst{\tfind{f}{g}}{\tfind{f}{R}}}}
      {\tfind{f}{\subst{g}{S}}}
      {\subst{\tfind{f}{g}}{\tfind{f}{S}}}
    }
  \\
& \inference
  { \jterm{{{\Gamma}{A}}{P}}{Q}{g}
    \jterm{{{{{\Gamma}{A}}{P}}{Q}}{R}}{S}{k}
    }
  { \jtermeq
      {{{{\Gamma}{B}}{\tfind{f}{P}}}{\subst{\tfind{f}{g}}{\tfind{f}{R}}}}
      {\subst{\tfind{f}{g}}{\tfind{f}{S}}}
      {\tfind{f}{\subst{g}{k}}}
      {\subst{\tfind{f}{g}}{\tfind{f}{k}}}
    }
\end{align*}
\item The operation $\tfindf{f}$ is compatible with the identity terms:
\begin{equation*}
\inference
  { \jfam{{{\Gamma}{A}}{P}}{Q}
    }
  { \jtermeq
      {{{{\Gamma}{B}}{\tfind{f}{P}}}{\tfind{f}{Q}}}
      {\ctxwk{\tfind{f}{Q}}{\tfind{f}{Q}}}
      {\tfind{f}{\idtm{Q}}}
      {\idtm{\tfind{f}{Q}}}
    }
\end{equation*}
\item We will also require that $\tfindf{f}$ is compatible with $f$ itself:
\begin{equation*}
\inference
  {
    }
  { \jtermeq
      {{\Gamma}{B}}
      {\ctxwk{B}{B}}
      {\tfind{f}{f}}
      {\idtm{B}}
    }
\end{equation*}
\item Finally, we require that $\tfindf{f}$ is the right inverse of composition
with $f$:
\begin{align*}
& \inference
  { \jfam{{\Gamma}{A}}{P}
    }
  { \jfameq
      {{\Gamma}{A}}
      {\jcomp{}{f}{\tfind{f}{P}}}
      {P}
    }
  \\
& \inference
  { \jfam{{{\Gamma}{A}}{P}}{Q}
    }
  { \jfameq
      {{{\Gamma}{A}}{P}}
      {\jcomp{}{f}{\tfind{f}{Q}}}
      {Q}
    }
  \\
& \inference
  { \jterm{{{\Gamma}{A}}{P}}{Q}{g}
    }
  { \jtermeq
      {{{\Gamma}{A}}{P}}
      {Q}
      {\jcomp{}{f}{\tfind{f}{g}}}
      {g}
    }
\end{align*}
\end{enumerate}
\end{defn}

\begin{rmk}
The rules expressing that $\tfindf{f}$ is a right inverse to composition with
$f$ are usually called the `computation rules' of the induction principle.

Recall that we had announced that $\tfindf{f}$ would be an actual inverse of
composition with $f$, while we only have stated explicitly that $\tfindf{f}$
is a right inverse. We get the fact that it is also a left inverse from the
other compatibility rules. For example: given $\jfam{{\Gamma}{B}}{Q}$ we get
\begin{equation*}
\tfind{f}{\jcomp{}{f}{Q}}
  \jdeq
  \tfind{f}{\subst{f}{\ctxwk{A}{Q}}}
  \jdeq
  \subst{\tfind{f}{f}}{\tfind{f}{\ctxwk{A}{Q}}}
  \jdeq
  \subst{\idtm{B}}{\ctxwk{B}{Q}}
  \jdeq
  Q.
\end{equation*}
We thus recover the usual sort of induction principle. When $Q$ is a family
over $\ctxext{\Gamma}{B}$, all we have to do to find a term of $Q$ is to find
a term $g$ of $\jcomp{}{f}{Q}$. The result of applying $\tfindf{f}$ to $g$
will be a term of $Q$.
\end{rmk}

\begin{rmk}
These stronger rules also seem to imply that not every equivalence is going
to be an inductive morphism (when we add all the type constructors to the
theory). For instance, the interval is equivalent to the unit type. If the
unit type is defined via an inductive morphism $\emptyc\to\unit$ we get that
every family over $\unit$ is definitionally a constant type because
every family over $\emptyc$ is a weakening by the empty family. If the
equivalence from $\unit$ to the interval were inductive, this would in turn
imply that every type family over the interval is constant. However, this is
not the case because we have the family which has the unit type as a fiber
above one endpoint and the interval above the other.
\end{rmk}

\begin{lem}
The identity term
$\jhom{\Gamma}{A}{A}{\idtm{A}}$ is an inductive morphism
for each family $A$ of contexts over $\Gamma$
\end{lem}

\begin{proof}
Composition with the identity morphism is an identity operation.
\end{proof}

\begin{itemize}
\item Extensions of inductive morphisms are inductive
\item Weakenings of inductive morphisms are inductive
\item Substitutions of inductive morphisms are inductive
\end{itemize}


%\section{Type constructors in structural type theory}
In this section we will introduce the usual type constructors to structural
type theory. We do so with the point of view that each type constructor is a
(class of) operations on type theory that should be compatible with extension,
weakening, substitution and identity morphisms. Likewise, each induction
principle of an inductively defined type constructor is going to be such an
operator, and it is therefore also required to be compatible with extension,
weakening, substitution and identity functions.

We will first state all the rules of the individual type constructors, where
only the $\wtypesym$-type constructor depends on the presence of dependent
function types (or universes too?), and hence comes with the notion that $\wtypesym$
is compatible with dependent product types and vice versa. All the other
compatibility properties are stated in \autoref{compatibility-of-type-constructors}.

\subsubsection{Issues to keep in mind}
\begin{enumerate}
\item The rule asserting that unit types are compatible with context extension makes
no sense from the point of view that contexts are lists of types and types
are contexts of length one; in other words, that types are indecomposable (i.e.~non-extended)
families. Remidies:
\begin{enumerate}
\item This point of view is wrong and should be abandoned.
\item We have, as Vladimir proposed, two kinds of judgmental equalities. One
      could be used for the very strict equalities, the other could be used
      to state the compatibility rules with. In other words, the other equality
      is a compatibility relation. If we do that, we shouldn't require that
      if a family is compatible with a type, then the family is a type. The
      compatibility relation could relate things by uniqueness up to unique
      isomorphism.
\item Don't state such compatibility rules for the type constructors.
\item Don't even give the unit type a dependent action.
\end{enumerate}
\end{enumerate}

\subsection{The unit type in structural type theory}
In this section we explore what we get if we pose compatibility conditions on
an inductively defined unit type. We will assume that not only the unit type
is compatible with extension, weakening, substitution and identity functions,
but also its induction principle should be compatible with those.

\begin{align*}
& \inference
  { \jfam{\Gamma}{A}
    }
  { \jtype{{\Gamma}{A}}{\unitc{A}}
    }
& & \inference
    { \jfameq{\Gamma}{A}{A'}
      }
    { \jtypeeq
        {{\Gamma}{A}}
        {\unitc{A}}
        {\unitc{A'}}
      }
  \\
& \inference
    { \jfam{\Gamma}{A}
      }
    { \jterm{{\Gamma}{A}}{\unitc{A}}{\unitct{A}}
      }
& & \inference
    { \jfameq{\Gamma}{A}{A'}
      }
    { \jtermeq{{\Gamma}{A}}{\unitc{A}}{\unitct{A}}{\unitct{A'}}
      }
  \\
& \inference
  { \jfam{{\Gamma}{A}}{P}
    }
  { \jtype{{{{\Gamma}{A}}{P}}{\ctxwk{P}{\unitc{A}}}}{\unitf{A}{P}}
    }
& & \inference
    { \jfameq{{\Gamma}{A}}{P}{P'}
      }
    { \jtypeeq
        {{{{\Gamma}{A}}{P}}{\ctxwk{P}{\unitc{A}}}}
        {\unitf{A}{P}}
        {\unitf{A}{P'}}
      }
  \\
& \inference
  { \jfam{{\Gamma}{A}}{P}
    }
  { \jterm
      {{{{\Gamma}{A}}{P}}{\ctxwk{P}{\unitc{A}}}}
      {\unitf{A}{P}}
      {\unitft{A}{P}}
    }
& & \inference
    { \jfameq{{\Gamma}{A}}{P}{P'}
      }
    { \jtermeq
        {{{{\Gamma}{A}}{P}}{\ctxwk{P}{\unitc{A}}}}
        {\unitf{A}{P}}
        {\unitft{A}{P}}
        {\unitft{A}{P'}}
      }
\end{align*}

We impose the following compatibility rules for the unit type:

\begin{align*}
& \inference
  { \jfam{{\Gamma}{A}}{P}
    }
  { \jtypeeq
      {{{\Gamma}{A}}{P}}
      {\ctxext{\ctxwk{P}{\unitc{A}}}{\unitf{A}{P}}}
      {\unitc{{\Gamma}{A}}}
    }\\
& \inference
  { \jfam{\Gamma}{A}
    \jfam{\Gamma}{B}
    }
  { \jtypeeq{{{\Gamma}{A}}{\ctxwk{A}{B}}}{}{}
    }  
\end{align*}

\subsection{The empty type}

\subsection{The natural numbers}

\subsection{Dependent pair types}

\subsection{Identity types}

\subsection{Dependent function types}
The dependent function type constructor bundles up the terms of a family $P$
of contexts over $\ctxext{\Gamma}{A}$ and produces a family $\mprd{A}{P}$ of
contexts over $\Gamma$. Its terms are copies of those of $P$. The operation
$\mprd{A}{\blank}$ also acts on terms of families -- this is called
$\lambda$-abstraction.

The dependent function type constructor not only acts on families $P$ of
contexts over $\ctxext{\Gamma}{A}$, but it also sends families $Q$ of contexts
over $\ctxext{{\Gamma}{A}}{P}$ to families $\mprd{A}{Q}$ over $\ctxext{\Gamma}
{\mprd{A}{P}}$. Moreover, we will also have a version of $\lambda$-abstraction
for terms of such families $Q$.

\begin{align*}
& \inference
    { \jfam{{\Gamma}{A}}{P}
      }
    { \jfam{\Gamma}{\mprd{A}{P}}
      }
& & \inference
      { \jfameq{\Gamma}{A}{A'}
        \jfameq{{\Gamma}{A}}{P}{P'}
        }
      { \jfameq{\Gamma}{\mprd{A}{P}}{\mprd{A'}{P'}}
        }
  \\
& \inference
    { \jterm{{\Gamma}{A}}{P}{f}
      }
    { \jterm{\Gamma}{\mprd{A}{P}}{\slam{A}{P}{f}}
      }
& & \inference
      { \jtermeq{{\Gamma}{A}}{P}{f}{f'}
        }
      { \jtermeq{\Gamma}{\mprd{A}{P}}{\slam{A}{P}{f}}{\slam{A}{P}{f'}}
        }
  \\
& \inference
    { \jterm{\Gamma}{\mprd{A}{P}}{g}
      }
    { \jterm{{\Gamma}{A}}{P}{\sev{g}}
      }
& & \inference
      { \jtermeq{\Gamma}{\mprd{A}{P}}{g}{g'}
        }
      { \jtermeq{{\Gamma}{A}}{P}{\sev{g}}{\sev{g'}}
        }
\end{align*}

\begin{defn}
Let $A$ and $B$ be families of contexts over $\Gamma$. Then we define
\begin{equation*}
\jfamdefn{\Gamma}{\jfun{A}{B}}{\sprd{A}{\ctxwk{A}{B}}}.
\end{equation*}
\end{defn}

\subsubsection{Dependent products are compatible with the empty context}
The empty context can appear in the domain and in the codomain of the dependent
function type constructor. We have the following inference rules explaining
what happens when the empty context appears in the domain:
\begin{align}
& \inference
    { \jfam{\Gamma}{P}
      }
    { \jfameq{\Gamma}{\mprd{\emptyf}{P}}{P}
      }
  \\
& \inference
    { \jterm{\Gamma}{P}{f}
      }
    { \jtermeq{\Gamma}{P}{\slam{A}{P}{f}}{f}
      }
  \\
& \inference
    { \jterm{\Gamma}{P}{g}
      }
    { \jtermeq{\Gamma}{P}{\sev{g}}{g}
      }
\end{align}
We have the following infernece rules explaining what happens when the empty
context appears in the codomain:
\begin{align}
& \inference
    { \jfam{\Gamma}{A}
      }
    { \jfameq{\Gamma}{\mprd{A}{\emptyf}}{\emptyf}
      }
\end{align}

\subsubsection{Dependent products are compatible with extension}
\begin{align}
& \inference
  { \jfam{{{{\Gamma}{A}}{B}}{P}}{Q}
    }
  { \jfameq
      {\Gamma}
      {\sprd{{A}{B}}{{P}{Q}}}
      {\ctxext{\sprd{A}{\sprd{B}{P}}}{\sprd{A}{\sprd{B}{Q}}}}
    }
  \\
& \inference
  { \jterm{{{\Gamma}{A}}{B}}{P}{f}
    }
  { \jtermeq
      {\Gamma}
      {\sprd{A}{\sprd{B}{P}}}
      {\slam{{A}{B}}{P}{f}}
      {\slam{A}{\sprd{B}{P}}{\slam{B}{P}{f}}}
    }
\end{align}
\subsubsection{Dependent products are compatible with weakening}
\subsubsection{Dependent products are compatible with substitution}?
\subsubsection{Dependent products are compatible with the identity functions}?

\subsection{$\wtypesym$-types}

\subsection{Universes}

\subsection{The compatibility of the type constructors with each other}
\label{compatibility-of-type-constructors}


%\section{Identity types}
In this appendix we introduce identity types in for the type theory without type
constructors of \autoref{tt}. Since we do not use variable declarations in our
contexts, the notational appearance is somewhat different. The purpose here is
not only to demonstrate how identity types should be introduced, but also to
show that the new presentation of type theory is workable.

\subsection{The rules for identity types}
We will use the symbol $\reflsym$ slightly different than the book does. For us,
$\refl{A}$ is a term of $\subst{\idfunc[A]}{\idtypevar{A}}$ in context
$\ctxext{\Gamma}{A}$ and we will write $\subst{x}{\refl{A}}$ for the reflexivity
path at $x$ (which would have been denoted by $\refl{x}$ in the book).

A notable difference in the formulation of identity types is that in our current
setting we must state the elimination rule not only for families $P$ in context
$\ctxext{{\Gamma}{\ctxwk{\Gamma}{\Gamma}}}{\idtypevar{\Gamma}}$,
but also for families $Q$ in context 
$\ctxext{{{\Gamma}{\ctxwk{\Gamma}{\Gamma}}}{\idtypevar{\Gamma}}}{P}$. The reason
is that all operations have to be closed under slicing: everything may happen
in a context. Secretly, a reason is that we don't have dependent function types.
We wouldn't even be able to find the transport maps if we didn't state the
identity elimination in an extended context.

\begin{align}
& \inference{\jctx{\Gamma}}{\jtype{\ctxext{\Gamma}{\ctxwk{\Gamma}{\Gamma}}}{\idtypevar{\Gamma}}}\\
& \inference{\jctx{\Gamma}}{\jterm{\Gamma}{\subst{\idfunc[\Gamma]}{\idtypevar{\Gamma}}}{\refl{\Gamma}}}\\
& \inference{\jtype{\ctxext{{\Gamma}{\ctxwk{\Gamma}{\Gamma}}}{\idtypevar{\Gamma}}}{P}
           \qquad
           \jterm{\Gamma}{\subst{\refl{\Gamma}}{{\idfunc[\Gamma]}{P}}}{d}}
           {\jterm{\ctxext{{\Gamma}{\ctxwk{\Gamma}{\Gamma}}}{\idtypevar{\Gamma}}}{P}{\tfJ(d)}}\\
& \inference{\jtype{\ctxext{{\Gamma}{\ctxwk{\Gamma}{\Gamma}}}{\idtypevar{\Gamma}}}{P}\qquad
\jterm{\Gamma}{\subst{\refl{\Gamma}}{{\idfunc[\Gamma]}{P}}}{d}}
{\jtermeq
  {\ctxext{{\Gamma}{\ctxwk{\Gamma}{\Gamma}}}{\idtypevar{\Gamma}}}
  {\subst{\refl{\Gamma}}{{\idfunc[\Gamma]}{P}}}
  {\subst{\refl{\Gamma}}{{\idfunc[\Gamma]}{\tfJ(d)}}}
  {d}}\\
& \inference{\jtype{\ctxext{{{\Gamma}{\ctxwk{\Gamma}{\Gamma}}}{\idtypevar{\Gamma}}}{P}}{Q}
           \qquad
           \jterm{\ctxext{\Gamma}{\subst{\refl{\Gamma}}{{\idfunc[\Gamma]}{P}}}}{\subst{\refl{\Gamma}}{{\idfunc[\Gamma]}{Q}}}{d}}
           {\jterm{\ctxext{{{\Gamma}{\ctxwk{\Gamma}{\Gamma}}}{\idtypevar{\Gamma}}}{P}}{Q}{\tfJ(d)}}\\
& \inference
  {\jtype
    {\ctxext{{{\Gamma}{\ctxwk{\Gamma}{\Gamma}}}{\idtypevar{\Gamma}}}{P}}
    {Q}
  \qquad
  \jterm
    {\ctxext{\Gamma}{\subst{\refl{\Gamma}}{{\idfunc[\Gamma]}{P}}}}
    {\subst{\refl{\Gamma}}{{\idfunc[\Gamma]}{Q}}}
    {d}}
  {\jtermeq
    {\ctxext{{{\Gamma}{\ctxwk{\Gamma}{\Gamma}}}{\idtypevar{\Gamma}}}{P}}
    {\subst{\refl{\Gamma}}{{\idfunc[\Gamma]}{Q}}}
    {\subst{\refl{\Gamma}}{{\idfunc[\Gamma]}{\tfJ(d)}}}
    {d}}
\end{align}

Likewise, we introduce identity types in a context. The rules are slightly simpler.

\begin{align}
& \inference{\jtype{\Gamma}{A}}{\jtype{\ctxext{{\Gamma}{A}}{\ctxwk{A}{A}}}{\idtypevar{A}}}\\
& \inference{\jtype{\Gamma}{A}}{\jterm{\ctxext{\Gamma}{A}}{\subst{\idfunc[A]}{\idtypevar{A}}}{\refl{A}}}\\
& \inference{\jtype{\ctxext{{{\Gamma}{A}}{\ctxwk{A}{A}}}{\idtypevar{A}}}{P}\qquad
\jterm{\ctxext{\Gamma}{A}}{\subst{\refl{A}}{{\idfunc[A]}{P}}}{d}}
{\jterm{\ctxext{{{\Gamma}{A}}{\ctxwk{A}{A}}}{\idtypevar{A}}}{P}{\tfJ(d)}}\\
& \inference{\jtype{\ctxext{{{\Gamma}{A}}{\ctxwk{A}{A}}}{\idtypevar{A}}}{P}\qquad
\jterm{\ctxext{\Gamma}{A}}{\subst{\refl{A}}{{\idfunc[A]}{P}}}{d}}
{\jtermeq
  {\ctxext{{{\Gamma}{A}}{\ctxwk{A}{A}}}{\idtypevar{A}}}
  {\subst{\refl{A}}{{\idfunc[A]}{P}}}
  {\subst{\refl{A}}{{\idfunc[A]}{\tfJ(d)}}}
  {d}}
\end{align}

Suppose we have terms $\jterm{\Gamma}{A}{x}$ and $\jterm{\Gamma}{A}{y}$. Then
we may define $\id[A]{x}{y}\defeq\subst{y}{{x}{\idtypevar{A}}}$. A term
$\jterm{\Gamma}{\id[A]{x}{y}}{p}$ is called an identification of $x$ and $y$.

\subsubsection{Basic properties of identity types}
In this subsubsection we prove some basic properties of identity types, just to
know whether we got the type theory right.

Suppose we have a family $\jtype{\ctxext{\Gamma}{A}}{P}$. Then we can consider
the family $\jtype{\ctxext{{\Gamma}{A}}{\ctxwk{A}{A}}}{\ctxwk{A}{P}}$, which has the role of the family $\jtype{\Gamma,\,x,y:A}{P(y)}$
of ordinary Martin-L\"of type theory. We may also consider the family
$\jtype{\ctxext{{\Gamma}{A}}{\ctxwk{A}{A}}}{\ctxwk{{A}{A}}{P}}$; this one has the
role of the family $\jtype{\Gamma,\,x,y:A}{P(x)}$. Those are families in the
same context, so we have
\begin{equation*}
\jtype{\ctxext{{{\Gamma}{A}}{\ctxwk{A}{A}}}{\ctxwk{{A}{A}}{P}}}{\ctxwk{\ctxwk{{A}{A}}{P}}{{A}{P}}}
\end{equation*}

\begin{lem}
There is a term
\begin{equation*}
\jterm{\ctxext{{{{\Gamma}{A}}{\ctxwk{A}{A}}}{\idtypevar{A}}}{\ctxwk{\idtypevar{A}}{{{A}{A}}{P}}}}{\ctxwk{\idtypevar{A}}{{{{A}{A}}{P}}{{A}{P}}}}{\transfibf{P}}
\end{equation*}
\end{lem}

\begin{proof}
By identity elimination it suffices to find a term
\begin{equation*}
\jterm{\ctxext{\Gamma}{A}}{\subst{\refl{A}}{{\idfunc[A]}{\ctxwk{\idtypevar{A}}{{{{A}{A}}{P}}{{A}{P}}}}}}{t}
\end{equation*}
By the judgmental equality $\jtypeeq{\Gamma}{\subst{{x}{f}}{{x}{Q}}}{\subst{x}{{f}{Q}}}$
it follows that we have the judgmental equalities
\begin{align*}
& \subst{\refl{A}}{{\idfunc[A]}{\ctxwk{\idtypevar{A}}{{{{A}{A}}{P}}{{A}{P}}}}}\\
& \qquad \jdeq \subst{\idfunc[A]}{{\ctxwk{{A}{A}}{\refl{A}}}{\ctxwk{\idtypevar{A}}{{{{A}{A}}{P}}{{A}{P}}}}}\\
& \qquad \jdeq \subst{\idfunc[A]}{\ctxwk{{{A}{A}}{P}}{{A}{P}}}\\
& \qquad \jdeq \ctxwk{\subst{\idfunc[A]}{\ctxwk{{A}{A}}{P}}}{\subst{\idfunc[A]}{\ctxwk{A}{P}}}\\
& \qquad \jdeq \ctxwk{P}{P}
\end{align*}
in context $\ctxext{{\Gamma}{A}}{P}$. We have the term $\jterm{\ctxext{{\Gamma}{A}}{P}}{\ctxwk{P}{P}}{\idfunc[P]}$.
\end{proof}

\begin{comment}
Suppose we have a term $\jterm{\ctxext{\Gamma}{A}}{P}{f}$. 

Using identity types, we can assert that a function $\jhom{\Gamma}{\Delta}{f}$ has
a left inverse $\jhom{\Delta}{\Gamma}{g}$ by asserting that there is an identification
\begin{equation*}
..
\end{equation*}
\end{comment}

\begin{comment}
\subsubsection{Compatibility of identity types with extension}

\subsubsection{Compatibility of identity types with weakening}
Suppose $A$ and $B$ are types in context $\Gamma$. Then we can consider the types
\begin{align*}
\jtype*{\ctxext{{{\Gamma}{A}}{\ctxwk{A}{B}}}{\ctxwk{A}{{B}{B}}}}{\ctxwk{A}{\idtypevar{B}}}\\
\jtype*{\ctxext{{{\Gamma}{A}}{\ctxwk{A}{B}}}{\ctxwk{{A}{B}}{{A}{B}}}}{\idtypevar{\ctxwk{A}{B}}}
\end{align*}
Note that we have the judgmental equality
\begin{equation*}
\jtypeeq
  {\ctxext{{\Gamma}{A}}{\ctxwk{A}{B}}}
  {\ctxwk{A}{{B}{B}}}
  {\ctxwk{{A}{B}}{{A}{B}}}
\end{equation*}
so $\ctxwk{A}{\idtypevar{B}}$ and $\idtypevar{\ctxwk{A}{B}}$ are types in the
same context. 

\begin{lem}
There is a term of type
\begin{equation*}
\jterm
  {\ctxext{{{{\Gamma}{A}}{\ctxwk{A}{B}}}{\ctxwk{{A}{B}}{{A}{B}}}}{\idtypevar{\ctxwk{A}{B}}}}
  {\ctxwk{\idtypevar{\ctxwk{A}{B}}}{\ctxwk{A}{\idtypevar{B}}}}
  {\typefont{idwktowkid}}
\end{equation*}
\end{lem}

\begin{proof}
By the identity elimination rule it suffices to find a term
\begin{equation*}
\jterm
  {\ctxext{{\Gamma}{A}}{\ctxwk{A}{B}}}
  {\subst{\refl{\ctxwk{A}{B}}}{{\idfunc[\ctxwk{A}{B}]}{\ctxwk{\idtypevar{\ctxwk{A}{B}}}{\ctxwk{A}{\idtypevar{B}}}}}}
  {i}
\end{equation*}
We may simplify the type $\subst{\refl{\ctxwk{A}{B}}}{{\idfunc[\ctxwk{A}{B}]}{\ctxwk{\idtypevar{\ctxwk{A}{B}}}{\ctxwk{A}{\idtypevar{B}}}}}$ as follows:
\begin{align*}
& \subst{\refl{\ctxwk{A}{B}}}{{\idfunc[\ctxwk{A}{B}]}{\ctxwk{\idtypevar{\ctxwk{A}{B}}}{\ctxwk{A}{\idtypevar{B}}}}}\\
& \qquad\jdeq \subst{\idfunc[\ctxwk{A}{B}]}{{\ctxwk{..}{\refl{\ctxwk{A}{B}}}}{\ctxwk{\idtypevar{\ctxwk{A}{B}}}{\ctxwk{A}{\idtypevar{B}}}}}
\end{align*}
\end{proof}

\subsubsection{Compatibility of identity types with substitution}
\end{comment}

\subsection{Generalized identity types}
From the way identity types are treated and denoted, we might take the point of
view that the identity types are instances of an operation which acts on the
identity functions. In the following we introduce a new type constructor
$\eqtype{x}$ for a given term $\jterm{\Gamma}{A}{x}$ which behaves very similarly
to identity types. In the case of $\jterm{\Gamma}{\ctxwk{\Gamma}{\Gamma}}{\idfunc[\Gamma]}$
the type $\eqtype{\idfunc[\Gamma]}$ is just the type $\idtypevar{\Gamma}$. In
the case of a context morphism $\jterm{\Gamma}{\ctxwk{\Gamma}{\Delta}}{f}$, the
type $\eqtype{f}$ can be thought of as the relation $\id{f(x)}{y}$ for terms $x:\Gamma$
and $y:\Delta$. We prove that $\eqtype{f}$ is indeed equivalent to the type
$\subst{f}{\ctxwk{\Gamma}{\idtypevar{\Delta}}}$ when both $\eqtype{f}$ and identity
types are present. Since the relation $\id{f(x)}{y}$ is inductively generated
by the reflexivity term if and only if $f$ is an equivalence, asserting the existence
of $\eqtype{f}$ for all context morphisms $\jterm{\Gamma}{\ctxwk{\Gamma}{\Delta}}{f}$
puts a groupoid condition on the type theory.

\begin{infarray}{c}
\inference{\jterm{\Gamma}{A}{x}}{\jtype{\ctxext{\Gamma}{A}}{\eqtype{x}}}\\
\inference{\jterm{\Gamma}{A}{x}}{\jterm{\Gamma}{\subst{x}{\eqtype{x}}}{\refl{x}}}\\
\inference{\jtype{\ctxext{{\Gamma}{A}}{\eqtype{x}}}{P}\qquad\jterm{\Gamma}{\subst{\refl{x}}{{x}{P}}}{d}}
          {\jterm{\ctxext{{\Gamma}{A}}{\eqtype{x}}}{P}{\ind{\eqtype{x}}(d)}}\\
\inference{\jtype{\ctxext{{\Gamma}{A}}{\eqtype{x}}}{P}\qquad\jterm{\Gamma}{\subst{\refl{x}}{{x}{P}}}{d}}
          {\jtermeq{\ctxext{{\Gamma}{A}}{\eqtype{x}}}{\subst{\refl{x}}{{x}{P}}}{\subst{\refl{x}}{{x}{\ind{\eqtype{x}}(d)}}}{d}}
\end{infarray}

We get something that looks like Paulin-Mohring equality. But we should be able to use it to
show that every $\jterm{\Gamma}{\ctxwk{\Gamma}{\Delta}}{f}$ is a trivial cofibration.

For the following conjecture, note that if we have $\jterm{\Gamma}{\ctxwk{\Gamma}{\Delta}}{f}$
Then we may consider the type $\jtype{\ctxext{{\Gamma}{\ctxwk{\Gamma}{\Delta}}}{\ctxwk{\Gamma}{{\Delta}{\Delta}}}}
{\ctxwk{\Gamma}{\idtypevar{\Delta}}}$ and we may substitute $f$ to obtain the type
\begin{equation*}
\jtype{\ctxext{\Gamma}{\subst{f}{\ctxwk{\Gamma}{{\Delta}{\Delta}}}}}
{\subst{f}{\ctxwk{\Gamma}{\idtypevar{\Delta}}}}
\end{equation*}
Note that we have the judgmental equality
\begin{equation*}
\jtypeeq{\Gamma}{\subst{f}{\ctxwk{\Gamma}{{\Delta}{\Delta}}}}{\ctxwk{\Gamma}{\Delta}}
\end{equation*}
so we obtain the type
\begin{equation*}
\jtype{\ctxext{\Gamma}{\ctxwk{\Gamma}{\Delta}}}
{\subst{f}{\ctxwk{\Gamma}{\idtypevar{\Delta}}}}
\end{equation*}
We also have $\jtype{\ctxext{\Gamma}{\ctxwk{\Gamma}{\Delta}}}{\eqtype{f}}$, so we
may ask for a term of type $\ctxwk{\eqtype{f}}{\subst{f}{\ctxwk{\Gamma}{\idtypevar{\Delta}}}}$
in context $\ctxext{\Gamma}{\ctxwk{\Gamma}{\Delta}}$ and we can ask ourselves the question
whether this is a trivial cofibration. This is the content of the following conjecture.

\begin{conj}
There is a term
\begin{equation*}
\jterm{\ctxext{{\Gamma}{\ctxwk{\Gamma}{\Delta}}}{\eqtype{f}}}{\ctxwk{\eqtype{f}}{\subst{f}{\ctxwk{\Gamma}{\idtypevar{\Delta}}}}}{\eqtoid{f}}
\end{equation*}
Moreover, $\eqtoid{f}$ is a trivial cofibration.
\end{conj}

\begin{proof}
We use the induction principle of $\eqtype{f}$ to construct $\eqtoid{f}$. Thus, we
have to find a term
\begin{equation*}
\jterm
  {\Gamma}
  {\subst{\refl{f}}{{f}{\ctxwk{\eqtype{f}}{\subst{f}{\ctxwk{\Gamma}{\idtypevar{\Delta}}}}}}}
  {\subst{\refl{f}}{{f}{\eqtoid{f}}}}
\end{equation*}
Note that we have the judgmental equalities
\begin{align*}
\subst{\refl{f}}{{f}{\ctxwk{\eqtype{f}}{\subst{f}{\ctxwk{\Gamma}{\idtypevar{\Delta}}}}}}
& \jdeq \subst{f}{{\ctxwk{{\Gamma}{\Delta}}{\refl{f}}}{\ctxwk{\eqtype{f}}{\subst{f}{\ctxwk{\Gamma}{\idtypevar{\Delta}}}}}} \\
& \jdeq \subst{f}{\ctxwk{\Gamma}{\subst{f}{\ctxwk{\Gamma}{\idtypevar{\Delta}}}}}
\end{align*}
where we find the term $\subst{f}{\ctxwk{\Gamma}{\subst{f}{\ctxwk{\Gamma}{\refl{\Delta}}}}}$.
\end{proof}

\begin{conj}
Suppose we have a type $\eqtype{x}$ for every term $\jterm{\Gamma}{A}{x}$. Then
every term $\jterm{\Gamma}{\ctxwk{\Gamma}{\Delta}}{f}$ is a trivial cofibration.
\end{conj}

\begin{proof}
Suppose that $Q$ is a type in context $\Delta$. We want to show that
\begin{equation*}
\inference{\jterm{\Gamma}{\subst{f}{\ctxwk{\Gamma}{Q}}}{g}}{\jterm{\Delta}{Q}{\tilde{g}}}
\end{equation*}
Let $g$ be a term of $\subst{f}{\ctxwk{\Gamma}{Q}}$ in context $\Gamma$. We get
the term
\begin{equation*}
\jterm{\Gamma}{\subst{\refl{f}}{{f}{\ctxwk{\eqtype{f}}{{\Gamma}{Q}}}}}{\subst{\refl{f}}{\ctxwk{\eqtype{f}}{g}}}
\end{equation*}
which is judgmentally equal to $g$; the types

For any term $\jterm{\Gamma}{\subst{f}{\ctxwk{\Gamma}{Q}}}{g}$ we get a term 
$\jterm{\Gamma}{\subst{\refl{x}}{{x}{P}}}{d}$.

$\jterm{\ctxext{{\Gamma}{\ctxwk{\Gamma}{\Delta}}}{\eqtype{f}}}{\ctxwk{\Gamma}{Q}}{\ind{\eqtype{f}}(d)}$.
\end{proof}

However, we could take other classes of terms, such as the projections
$\proj1:\ctxwk{\ctxext{\Gamma}{A}}{\Gamma}$. Let's see how that goes:

\begin{infarray}{c}
\inference{\jtype{\Gamma}{A}}{\jtype{\ctxext{{\Gamma}{A}}{\ctxwk{\ctxext{\Gamma}{A}}{\Gamma}}}{\eqtype{\proj1^A}}}\\
\inference{\jtype{\Gamma}{A}}{\jterm{\ctxext{\Gamma}{A}}{\subst{\proj1^A}{\eqtype{\proj1^A}}}{\refl{\proj1^A}}}
\end{infarray}



%\section{Introducing the type constructors}
We will now describe the rules for the type constructors that we don't assume
in our type theory by default. The first of these are the dependent function
types. We also discuss $\tfW$-types, suspensions, truncations. More general higher inductive
types will have to wait until we have introduced models, for the models of the
basic type theory because we will use them as index categories of the diagrams.

\subsection{The universal property of dependent pair types}
This section shouldn't be about extension anymore.

Using weakening and substitution we are able to state the universal property
for extension. It looks a bit more involved, since we cannot directly refer
to the variables in the contexts. On the other hand, we can now plainly see
were there were secretly weakenings going on.

We begin with stating the universal property of the extension $\ctxext{\Gamma}{A}$.
In these rules we assume we have $\jtype{\Gamma}{A}$ in the hypotheses.
\begin{align}
& \inference{}
{\jterm{\ctxext{\Gamma}{A}}{\ctxwk{A}{{\Gamma}{\ctxext{\Gamma}{A}}}}{\pair_A}}\\
& \inference{
  \jtype{\ctxext{\Gamma}{A}}{P}
  \qquad
  \jterm{\ctxext{\Gamma}{A}}{\subst{\pair_A}{\ctxwk{A}{{\Gamma}{P}}}}{f}}
  {\jterm{\ctxext{\Gamma}{A}}{P}{\ind{\tfext_\Gamma(A)}(f)}}\\
& \inference{
  \jtype{\ctxext{\Gamma}{A}}{P}
  \qquad
  \jterm{\ctxext{\Gamma}{A}}{\subst{\pair_A}{\ctxwk{A}{{\Gamma}{P}}}}{f}}
  {\jtermeq{\ctxext{\Gamma}{A}}{\subst{\pair_A}{\ctxwk{A}{{\Gamma}{P}}}}{\subst{\pair_A}{\ctxwk{A}{{\Gamma}{\ind{\tfext_\Gamma(A)}(f)}}}}{f}}
\end{align}

Under the hypothesis that $\jtypeeq{\Gamma}{A}{A'}$
we will also have the rules

\begin{align}
& \inference{}{\jtermeq{\ctxext{\Gamma}{A}}{\ctxwk{A}{{\Gamma}{\ctxext{\Gamma}{A}}}}{\pair_A}{\pair_{A'}}}\\
& \inference{\jtype{\ctxext{\Gamma}{A}}{P}\qquad\jterm{\ctxext{\Gamma}{A}}{\subst{\pair_A}{\ctxwk{A}{{\Gamma}{P}}}}{f}}
{\jtermeq{\ctxext{\Gamma}{A}}{P}{\ind{\tfext_\Gamma(A)}(f)}{\ind{\tfext_\Gamma(A')}(f)}}
\end{align}

Note that we don't need the notion of terms for contexts to state the universal
property of context extension (which is a good thing, for we don't assume to have them).

Next, we give the universal property of the extension $\ctxext{A}{P}$ in context
$\Gamma$.
In all of the following inference rules we assume that $\jtype{\ctxext{\Gamma}{A}}{P}$
is among the hypothesis. The induction principle for extension consists of three
inference rules:

\begin{align}
& \inference{}
{\jterm{\ctxext{{\Gamma}{A}}{P}}{\ctxwk{P}{{A}{\ctxext{A}{P}}}}{\pair_P}}\\
& \inference{
  \jtype{\ctxext{\Gamma}{{A}{P}}}{Q}
  \qquad
  \jterm{\ctxext{{\Gamma}{A}}{P}}{\subst{\pair_P}{\ctxwk{P}{{A}{Q}}}}{f}}
  {\jterm{\ctxext{\Gamma}{{A}{P}}}{Q}{\ind{\tfext_A(P)}(f)}}\\
& \inference{
  \jtype{\ctxext{\Gamma}{{A}{P}}}{Q}
  \qquad
  \jterm{\ctxext{{\Gamma}{A}}{P}}{\subst{\pair_P}{\ctxwk{P}{{A}{Q}}}}{f}}
  {\jtermeq{\ctxext{{\Gamma}{A}}{P}}{\subst{\pair_P}{\ctxwk{P}{{A}{Q}}}}{\subst{\pair_P}{\ctxwk{P}{{A}{\ind{\tfext_A(P)}(f)}}}}{f}}
\end{align}

As with context extension, we shall require two more inference rules stating that
$\pair_P$ and $\ind{\tfext_A(P)}$ are invariant under judgmental equality.

\subsection{The unit type}
Since we don't have a notion of terms of a context, we just say that the context
$\unit$ is the terminal context.

\begin{align}
& \inference{}{\jctx{\unit}}\\
& \inference{\jctx{\Gamma}}{\jhom{\Gamma}{\unit}{\tounit{\Gamma}}}\\
& \inference{\jctx{\Gamma}\qquad\jhom{\Gamma}{\unit}{f}}{\jhomeq{\Gamma}{\unit}{f}{\tounit{\Gamma}}}
\end{align}

Note that we don't have to require that $\jhomeq{\Gamma}{\unit}{\tounit{\Gamma}}{\tounit{\Delta}}$
whenever we have a judgmental equality $\jctxeq{\Gamma}{\Delta}$, since this already follows from the third rule.

When the context $\unit$ is present, we may use the expression $\jtype{}{\Gamma}$
as a shorthand for the judgment $\jtype{\unit}{\ctxwk{\unit}{\Gamma}}$. Likewise,
we may use the expression $\jtermc{\Gamma}{i}$ as a shorthand
for the judgment $\jhom{\unit}{\Gamma}{i}$. If we have a type $A$ in context
$\Gamma$, we may use the expression $\jtype{}{\subst{i}{A}}$ to stand for
the judgment $\jtype{\unit}{\subst{i}{\ctxwk{\unit}{A}}}$. We see that in every
respect, contexts are types in the empty context.

We have created a strictly terminal object $\unit$. This is not necessary when
we're working in a context. We introduce the unit type $\unit_\Gamma$ in context
$\Gamma$ in the familiar type theoretical way.

\begin{align}
& \inference{\jctx{\Gamma}}{\jtype{\Gamma}{\unit_\Gamma}}\\
& \inference{\jctx{\Gamma}}{\jterm{\Gamma}{\unit_\Gamma}{\ttt_\Gamma}}\\
& \inference{\jtype{\ctxext{\Gamma}{\unit_\Gamma}}{P}\qquad\jterm{\Gamma}{\subst{\ttt_\Gamma}{P}}{u}}
          {\jterm{\ctxext{\Gamma}{\unit_\Gamma}}{P}{\ind{\unit_\Gamma}(u)}}\\
& \inference{\jtype{\ctxext{\Gamma}{\unit_\Gamma}}{P}\qquad\jterm{\Gamma}{\subst{\ttt_\Gamma}{P}}{u}}
          {\jtermeq{\Gamma}{\subst{\ttt_\Gamma}{P}}{\subst{\ttt_\Gamma}{\ind{\unit_\Gamma}(u)}}{u}}
\end{align}

\subsection{Subterminal types}
The subterminal types we're about to present are strict, so they're more like \verb+Prop+
in \Coq. We can define subterminal types in two ways: the first equalizes all elements
of the subject type; the second is universal with the property that for every constant
map factors through it.

\subsubsection{Equalizing all terms}

\subsubsection{Factorizing constant maps}
Let $\jhom{\Gamma}{\Delta}{f}$. We can weaken $f$ by $\Gamma$ to obtain a term
$\jterm{\ctxext{\Gamma}{\ctxwk{\Gamma}{\Gamma}}}{\ctxwk{\Gamma}{{\Gamma}{\Delta}}}{\ctxwk{\Gamma}{f}}$.
This term is `like the function $\lam{x}{y}f(y)$'. Likewise, we can weaken $f$
by $\ctxwk{{\Gamma}{\Gamma}}$ to obtain a term
$\jterm{\ctxext{\Gamma}{\ctxwk{\Gamma}{\Gamma}}}{\ctxwk{\Gamma}{{\Gamma}{\Delta}}}{\ctxwk{{\Gamma}{\Gamma}}{f}}$,
which is `like the function $\lam{x}{y}f(x)$'. Since we have
the judgmental equality 
$\jtypeeq{\ctxext{\Gamma}{\ctxwk{\Gamma}{\Gamma}}}{\ctxwk{{\Gamma}{\Gamma}}{{\Gamma}{\Delta}}}{\ctxwk{\Gamma}{{\Gamma}{\Delta}}}$ we can consider the judgmental equality between $\ctxwk{\Gamma}{f}$
and $\ctxwk{{\Gamma}{\Gamma}}{f}$, which is exactly what we'll do in the definition
of judgmentally constant.

\begin{defn}
A term $\jhom{\Gamma}{\Delta}{f}$ is said to be \emph{judgmentally constant} if
the judgment
\begin{equation*}
\jtermeq{\ctxext{\Gamma}{\ctxwk{\Gamma}{\Gamma}}}{\ctxwk{{\Gamma}{\Gamma}}{{\Gamma}{\Delta}}}{\ctxwk{\Gamma}{f}}{\ctxwk{{\Gamma}{\Gamma}}{f}}
\end{equation*}
can be derived.
\end{defn}

\begin{defn}
\begin{align}
& \inference{\jctx{\Gamma}}{\jctx{\tau\Gamma}}\\
& \inference{\jctx{\Gamma}}{\jhom{\Gamma}{\tau\Gamma}{t}}\\
& \inference{\jctx{\Gamma}}{\jtermeq{\ctxext{\Gamma}{\ctxwk{\Gamma}{\Gamma}}}{\ctxwk{{\Gamma}{\Gamma}}{{\Gamma}{\Delta}}}{\ctxwk{\Gamma}{t}}{\ctxwk{{\Gamma}{\Gamma}}{t}}}\\
& \inference{\jhom{\Gamma}{\Delta}{f}\qquad\jtermeq{\ctxext{\Gamma}{\ctxwk{\Gamma}{\Gamma}}}{\ctxwk{{\Gamma}{\Gamma}}{{\Gamma}{\Delta}}}{\ctxwk{\Gamma}{f}}{\ctxwk{{\Gamma}{\Gamma}}{f}}}{\jhom{\tau\Gamma}{\Delta}{\tilde{f}}}\\
& \inference{\jhom{\Gamma}{\Delta}{f}\qquad\jtermeq{\ctxext{\Gamma}{\ctxwk{\Gamma}{\Gamma}}}{\ctxwk{{\Gamma}{\Gamma}}{{\Gamma}{\Delta}}}{\ctxwk{\Gamma}{f}}{\ctxwk{{\Gamma}{\Gamma}}{f}}}
{\jtermeq{\Gamma}{\ctxwk{\Gamma}{\Delta}}{\jcomp{\Gamma}{t}{\tilde{f}}}{f}}
\end{align}
\end{defn}

\begin{lem}
Any term $\jhom{\unit}{\Gamma}{i}$ is judgmentally constant.
\end{lem}

\begin{proof}

\end{proof}

\subsection{Product types}

\subsubsection{Products}
\begin{align}
& \inference{\jctx{\Gamma}\qquad\jctx{\Delta}}{\jctx{\product{\Gamma}{\Delta}}}\\
& \inference{\jctx{\Gamma}\qquad\jctx{\Delta}}{\jhom{{\Gamma}{\Delta}}{\product{\Gamma}{\Delta}}{\pair}}
\end{align}

\subsubsection{Strict products}
As with the unit type, we may use the categorical description of the product
for our type theoretical definition. If we do that, we get strict products.

\begin{align}
& \inference{\jctx{\Gamma}\qquad\jctx{\Delta}}{\jctx{\product{\Gamma}{\Delta}}}\\
& \inference{\jctx{\Gamma}\qquad\jctx{\Delta}}{\jhom{\product{\Gamma}{\Delta}}{\Gamma}{\proj1}}\\
& \inference{\jctx{\Gamma}\qquad\jctx{\Delta}}{\jhom{\product{\Gamma}{\Delta}}{\Delta}{\proj2}}\\
& \inference{\jhom{\greek{E}}{\Gamma}{f}\qquad\jhom{\greek{E}}{\Delta}{g}}
          {\jhom{\greek{E}}{\product{\Gamma}{\Delta}}{\pairp{f,g}}}\\
& \inference{\jhom{\greek{E}}{\Gamma}{f}\qquad\jhom{\greek{E}}{\Delta}{g}}
          {\jhomeq{\greek{E}}{\Gamma}{\jcomp{\greek{E}}{\pairp{f,g}}{\proj1}}{f}}\\
& \inference{\jhom{\greek{E}}{\Gamma}{f}\qquad\jhom{\greek{E}}{\Delta}{g}}
          {\jhomeq{\greek{E}}{\Delta}{\jcomp{\greek{E}}{\pairp{f,g}}{\proj2}}{g}}
\end{align}

\begin{equation}
\inference{\jhom{\greek{E}}{\product{\Gamma}{\Delta}}{h}
           \qquad
           \jhomeq{\greek{E}}{\Gamma}{\jcomp{E}{h}{\proj1}}{f}
           \qquad
           \jhomeq{\greek{E}}{\Delta}{\jcomp{E}{h}{\proj2}}{g}}
          {\jhomeq{\greek{E}}{\product{\Gamma}{\Delta}}{h}{\pairp{f,g}}}
\end{equation}

\subsection{Equalizer types}
Now that we have introduced a terminal object and products the categorical way,
we may just continue and present (strict) equalizers and pullbacks too, just to
see where we get. These notions are probably just not very useful in a univalent
setting until we got a good computational interpretation.

\begin{align}
& \inference{\jhom{\Gamma}{\Delta}{f}\qquad\jhom{\Gamma}{\Delta}{g}}{\jctx{\jequalizer{\Gamma}{f}{g}}}\\
& \inference{\jhom{\Gamma}{\Delta}{f}\qquad\jhom{\Gamma}{\Delta}{g}}{\jhom{\jequalizer{\Gamma}{f}{g}}{\Gamma}{\jequalizerin{f}{g}}}\\
& \inference{\jhom{\greek{E}}{\Gamma}{h}\qquad\jhomeq{\greek{E}}{\Delta}{\jcomp{\greek{E}}{h}{f}}{\jcomp{\greek{E}}{h}{g}}}{\jhom{\greek{E}}{\jequalizer{\Gamma}{f}{g}}{\jequalizer{h}{f}{g}}}\\
& \inference{
  {\begin{array}{l}
    \jhom{\greek{E}}{\Gamma}{h}\\
    \jhomeq{\greek{E}}{\Delta}{\jcomp{\greek{E}}{h}{f}}{\jcomp{\greek{E}}{h}{g}}
  \end{array}}
  \qquad
  {\begin{array}{l}
    \jhom{\greek{E}}{\jequalizer{\Gamma}{f}{g}}{k}\\
    \jhomeq{\greek{E}}{\Gamma}{\jcomp{\greek{E}}{k}{\jequalizerin{f}{g}}}{h}
  \end{array}}
}
  {\jhom{\greek{E}}{\jequalizer{\Gamma}{f}{g}}{\jequalizer{h}{f}{g}}}
\end{align}

\subsection{Pullback types}

\subsection{Dependent function types}
\begin{align}
& \inference{\jtype{\ctxext{\Gamma}{A}}{P}}{\jtype{\Gamma}{\mprd{A}{P}}}\\
& \inference{\jterm{\ctxext{\Gamma}{A}}{P}{f}}{\jterm{\Gamma}{\mprd{A}{P}}{\lambda(f)}}\\
& \inference{\jterm{\Gamma}{\mprd{A}{P}}{g}}{\jterm{\ctxext{\Gamma}{A}}{P}{\tfev(g)}}\\
& \inference{\jterm{\ctxext{\Gamma}{A}}{P}{f}}{\jtermeq{\Gamma}{A}{\tfev(\lambda(f))}{f}}\\
& \inference{\jtermeq{\ctxext{\Gamma}{A}}{P}{f}{f'}}{\jtermeq{\Gamma}{\mprd{A}{P}}{\lambda(f)}{\lambda(f')}}\\
& \inference{\jtermeq{\Gamma}{\mprd{A}{P}}{g}{g'}}{\jtermeq{\ctxext{\Gamma}{A}}{P}{\tfev(g)}{\tfev(g')}}\\
& \inference{\jtypeeq{\ctxext{\Gamma}{A}}{P}{P'}}{\jtypeeq{\Gamma}{\mprd{A}{P}}{\mprd{A}{P'}}}\\
\end{align}

With these rules we will not get the weak $\eta$-rule when identity types are present.
So it might be better to state that $\lambda$ is a trivial cofibration.

\subsection{Subterminal types}
Let $\jhom{\Gamma}{\Delta}{f}$. Then we may also consider the type $\ctxwk{{\Gamma}{\Gamma}}{{\Gamma}{\Delta}}$
in context $\ctxwk{\Gamma}{\Gamma}$ and we have the terms
\begin{align*}
& \jhom{\ctxext{\Gamma}{\ctxwk{\Gamma}{\Gamma}}}{\ctxwk{\Gamma}{A}}{\ctxwk{\Gamma}{f}}\\
& \jhom{\ctxext{\Gamma}{\ctxwk{\Gamma}{\Gamma}}}{\ctxwk{\Gamma}{A}}{\ctxwk{{\Gamma}{\Gamma}}{f}}
\end{align*}

\subsection{The empty type}

\subsection{Coproduct types}

\subsection{The natural numbers}

\section{A categorical explanation of the variable-free type theory}

The categorical interpretation of type theory is a category $\catfont{C}$ of
contexts, for each object $\Gamma$ of $\catfont{C}$ a class $\catfont{type}(\Gamma)$
of morphisms into $\Gamma$ and for each $A\in\catfont{type}(\Gamma)$ a category
$\catfont{term}(A)$ of sections of $A$. 

\begin{itemize}
\item The context extension $\ctxext{\Gamma}{A}$ is the domain of $A$.
\item There is a choice of $\Gamma\times\Delta$ for each $\Gamma$ and $\Delta$
      for which the first projection $\pi_1:\Gamma\times\Delta\to\Gamma$ is in $\catfont{type}(\Gamma)$;
      this is the weakening $\ctxwk{\Gamma}{\Delta}$.
\item For each $A\in\catfont{type}(\Gamma)$ and $x:A$ there is a choice of
      $\pi_1:\Gamma\times_A P\to\Gamma$ which is in $\catfont{type}(\Gamma)$.
\end{itemize}

\begin{comment}
\section{Binary trees}
We prove a little meta theorem about the type system we have so far. It is not
really of importance to the development of the theory. 

Let's say
that a binary tree of contexts is a pair $\pair{T,f}$ consisting of a binary 
tree $T$ together with a function $f$ assigning to each leaf a context. The set
of all binary trees of contexts is denoted by $B$. Such
binary trees may be defined inductively: $(\unit,f):B$ for any $f:\unit\to ctx$
and given any $(T_1,f_1)$ and $(T_2,f_2)$ in $B$ we have $(T_1,f_1)*(T_2,f_2)$
in $B$.

We simultaneously define
the following two functions:
\begin{align*}
\trext & : B\to ctx\\
\trwk_0 & : \prd{X,Y:B} typ(\trext(X))\\
\trwk_1 & : \prd{X:B}{\Gamma~ctx} typ(\Gamma)\to typ(\ctxext{\trext(X)}{\trwk_0(X,\Gamma)})
\end{align*}
Both these functions are defined by induction on binary trees. We set
\begin{align*}
\trext((\unit,f)) & \defeq f(\unit)\\
\trext((T_1,f_1)*(T_2,f_2)) & \defeq \ctxext{\trext((T_1,f_1))}
{\trwk_0((T_1,f_1),\trext((T_2,f_2)))}
\end{align*}
and for any context $\Gamma$
\begin{align*}
\trwk_0((\unit,f),(\unit,g)) & \defeq \ctxwk{f(\unit)}{g(\unit)}\\
\trwk_0((T_1,f_1)*(T_2,f_2),\Gamma) 
  & \defeq \trwk_1((T_1,f_1),\trwk_0((T_2,f_2)),\Gamma))
\end{align*}
and for any type $A$ in context $\Gamma$
\begin{align*}
\trwk_1((\unit,f),A) & \defeq \ctxwk{f(\unit)}{A}\\
\trwk_1((T_1,f_1)*(T_2,f_2),A) & \defeq
  \trwk_1((T_1,f_1),\trwk_1((T_2,f_2)),A))
\end{align*}

\begin{lem}
$\trwk_0((T,f),\Gamma)$ is a type in context $\trext((T,f))$ and
$\trwk_1((T,f),A)$ is a type in context $\ctxext{\trext((T,f))}{\trwk_0((T,f),\Gamma)}$
for any type $A$ in context $\Gamma$.
\end{lem}

\begin{proof}
It is immediate that $\trwk_0((\unit,f),\Gamma)$ is a type in context $\trext((\unit,f))$
and that $\trwk_1((\unit,f),A)$ is a type in context $\ctxext{\trext((\unit,f))}
{\trwk_0((\unit,f),\Gamma)}$ for any type $A$ in context $\Gamma$.

Suppose that $(T_1,f_1)$ and $(T_2,f_2)$ are binary trees of contexts such that
$\trwk_0((T_i,f_i),\Gamma)$ is a type in context
$\trext((T_i,f_i))$ for $i\jdeq 1$ and $i\jdeq 2$, and such that
$\trwk_1((T_i,f_1),A)$ is a type in context $\ctxext{\trext((T_i,f_i))}{\trwk_0((T_i,f_i),\Gamma)}$
for $i\jdeq 1$ and $i\jdeq 2$. Then
\begin{equation*}
\trwk_0((T_1,f_1)*(T_2,f_2),\Gamma) 
\jdeq \trwk_1((T_1,f_1),\trwk_0((T_2,f_2),\Gamma))
\end{equation*}
is a type in context
\begin{equation*}
\ctxext{\trext((T_1,f_1))}{\trwk_0((T_1,f_1),\trext((T_2,f_2)))}
\jdeq \trext((T_1,f_1)*(T_2,f_2)).
\end{equation*}
Also,
\begin{equation*}
\trwk_1((T_1,f_1)*(T_2,f_2),A) \jdeq \trwk_1((T_1,f_1),\trwk_1((T_2,f_2),A))
\end{equation*}
is a type in context
\begin{equation*}
\ctxext{\trext((T_1,f_1))}{\trwk_0((T_1,f_1),\ctxext{\trext((T_2,f_2))}{\trwk_0((T_2,f_2),\Gamma))}}
\end{equation*}
\end{proof}
\end{comment}



%\section{Internal models of type theory}
\subsection{Models of type theory without basic constructors}\label{internal-model-contexts}
\begin{defn}\label{defn:premodel}
An internal model $\mfM$ of type theory consists of the following data. 
\begin{enumerate}
\item A type $\tfctx(\mfM)$ of \emph{contexts}.
\item A function $\mftypfunc{\mfM}$ assigning to each context $\Gamma$ of $\mfM$ an internal model
$\mftyp{\mfM}{\Gamma}$ of type theory.
\begin{defn}
The type $\tfctx(\mftyp{\mfM}{\Gamma})$ is denoted by $\mftyp{\mfM}{\Gamma}$. A
term of $\mftyp{\mfM}{\Gamma}$ is called a \emph{type in context $\Gamma$}. To indicate
that $A$ is a type in context $\Gamma$ we also write $\Gamma\vdash A:\mfM$. 
When $P:\mftyp{\mftyp{\mfM}{\Gamma}}{A}$ for some type $A$ in context $\Gamma$, we
also speak of \emph{a family $P$ over $A$ in context $\Gamma$.}
\end{defn}
\item A family $\terms{\blank}:\prd*{\Gamma:\ctx(\mfM)}\mftyp{\mfM}{\Gamma}\to\type$
assigning to every type $A$ in context $\Gamma$ the type $\terms{A}$ of its
terms.
\begin{defn}
When $A$ is a type in context $\Gamma$, we define $\Gamma\vdash x:A$ 
to mean $x:\terms{A}$.
\end{defn}
\item Context extension: a morphism
\begin{equation*}
\tfext^\Gamma:\mftyp{\mfM}{\Gamma}\to\mfM
\end{equation*}
of internal models for every context $\Gamma$.
\begin{defn}
When $A$ is a type in context $\Gamma$, we also denote
the context $\tfext^\Gamma_0(A)$ of $\mfM$ by $\ctxext{\Gamma}{A}$.
We will write
$\ctxext{{\Gamma}{A}}{P}$ for $\ctxext{{\Gamma}{A}}{P}$.
\end{defn}
\item The judgmental equalities:
\begin{align*}
\mftyp{\mftyp{\mfM}{\Gamma}}{A} & \jdeq\mftyp{\mfM}{\ctxext{\Gamma}{A}}\\
\mftyp{\tfext^\Gamma}{A} & \jdeq \modelfont{id}_{\mftyp{\mfM}{\ctxext{\Gamma}{A}}}\\
\terms[{\tfext^\Gamma}]{P} & \jdeq \idfunc[\terms{P}].
\end{align*}
\begin{rmk}
In particular, we will have the judgmental equalities:
\begin{enumerate}
\item When $A$ is a type in context $\Gamma$ we have
\begin{equation*}
\mftyp{\mftyp{\mfM}{\Gamma}}{A}\jdeq\mftyp{\mfM}{\ctxext{\Gamma}{A}},
\end{equation*}
ensuring that a context in the model $\mftyp{\mftyp{\mfM}{\Gamma}}{A}$ is the same thing as a
context in the model $\mftyp{\mfM}{\ctxext{\Gamma}{A}}$.
\item If $Q$ is a family over $P$ where $P$ is a family over $A$ in context $\Gamma$, then
\begin{equation*}
\tfext^{\protect{\mftyp{\mftyp{\mfM}{\Gamma}}{A}}}(P,Q)
\jdeq
\tfext^{\protect{\mftyp{\mfM}{\ctxext{\Gamma}{A}}}}(P,Q)
\end{equation*}
ensuring that twe two possible notion of context extension are the same.
\end{enumerate}
Other judgmental equalities will be required with the ingredients that follow.
We will not list them all.
\end{rmk}
\item Weakening: a morphism
\begin{equation*}
\tfwk^A:\mftyp{\mfM}{\Gamma}\to\mftyp{\mfM}{\ctxext{\Gamma}{A}}
\end{equation*}
of internal models for every type $A$ in context $\Gamma$. When $B$ is a type
in context $\Gamma$, we denote $\tfwk^A(B)$ by $\ctxwk{A}{B}$. 
\item Substitution: a morphism
\begin{equation*}
\tfsubst^x:\mftyp{\mfM}{\ctxext{\Gamma}{A}}\to\mftyp{\mfM}{\Gamma}
\end{equation*}
of internal models for any $x:A$ in context $\Gamma$. When $P$ is a family over $A$ in context
$\Gamma$, we denote $\tfsubst^x(P)$ also by $\subst{x}{P}$. 
\item A judgmental equality
\begin{equation*}
\tfsubst^x\circ\tfwk^A\jdeq\modelfont{id}_{\mftyp{\mfM}{\Gamma}}
\end{equation*}
for any $x:A$ and $B$ in context $\Gamma$.
\item A context $\unit^\mfM:\tfctx(\mfM)$ and a term of type $\isequiv(\ctxext{\unit^\mfM}{\blank})$. We will denote
this equivalence by $e_\unit$. The context $\unit^\mfM$ is also
called the \emph{empty context}.
\begin{defn}
For any context $\Gamma$, type $\terms{\Gamma}$ is defined to mean
$\terms{e_\unit^{-1}(\Gamma)}$. 
\end{defn}
\item A section $\unit^{\blank}:\prd{\Gamma:\ctx(\mfM)}\mftyp{\mfM}{\Gamma}$ assigning
a type $\unit^\Gamma$ in context $\Gamma$ to every context $\Gamma$ and
an identification $\alpha_\unit(\Gamma):\id{\ctxext{\Gamma}{\unit^\Gamma}}{\Gamma}$
for every context $\Gamma$. We also require that there is an identification
$\id{\trans{\alpha_\unit(\Gamma)}{\ctxwk{\unit^\Gamma}{A}}}{A}$ for every
type $A$ in context $\Gamma$.
\item For any type $A$ in context $\Gamma$, a term $\idfunc[A]:\terms{\ctxwk{A}{A}}$
\end{enumerate}
\begin{flushright}
\textsl{End of \autoref{defn:premodel}.}
\end{flushright}
\end{defn}

\begin{defn}
In an internal model $\mfM$ we define
\begin{equation*}
\ctxhom{\Delta}{\Gamma}\defeq \terms{\ctxwk{e_\unit^{-1}(\Delta)}{e_\unit^{-1}(\Gamma)}}
\end{equation*}
\end{defn}

\subsection{Morphisms of internal models}
\begin{defn}\label{defn:premodel-morphism}
A morphism $f:\mfM\to \mfN$ of internal models consists of
\begin{enumerate}
\item a function $\ctx(f):\ctx(\mfM)\to\ctx(\mfN)$. The function $\ctx(f)$ is also
denoted by $f_0$.
\item a morphism $\mftyp{f}{\Gamma}:\mftyp{\mfM}{\Gamma}\to\mftyp{\mfN}{f_0(\Gamma)}$ of internal models for every
$\Gamma:\ctx(M)$.
\item the judgmental equality
\begin{equation*}
\mftyp{\mftyp{f}{\Gamma}}{A}\jdeq\mftyp{f}{\ctxext{\Gamma}{A}}
\end{equation*}
\item a function $\terms[f]{A}:\terms[\mfM]{A}\to\terms[\mfN]{\mftyp{f}{\Gamma}_0(A)}$ for
every type $A$ in context $\Gamma$. 
\item preservation of context extension: 
\begin{align*}
\alpha^f_0 & :\id{f\circ\tfext^\Gamma}{\tfext^{f_0(\Gamma)}\circ\mftyp{f}{\Gamma}}\\
\alpha^f_1 & :\id{\mftyp{f}{\Gamma}_0\circ\tfext^A}{\tfext^{\mftyp{f}{\Gamma}_0(A)}\circ\mftyp{f}{\ctxext{\Gamma}{A}}}.
\end{align*}
\item preservation of weakening: 
\begin{align*}
\beta^f_0 & :\id{\mftyp{f}{\Gamma}\circ\tfwk^\Gamma}{\tfwk^{f_0(\Gamma)}\circ f}\\
\beta^f_1 & :\id{\mftyp{f}{\ctxext{\Gamma}{A}}\circ\tfwk^A}{\tfwk^\protect{\mftyp{f}{\Gamma}_0(A)}\circ\mftyp{f}{\Gamma}}.
\end{align*}
\item preservation of substitution: 
\begin{equation*}
\gamma^f:\id{\mftyp{f}\Gamma_0\circ\tfsubst^x}{\tfsubst^\protect{\terms[f]{A}(x)}\circ\mftyp{f}{\ctxext{\Gamma}{A}}}.
\end{equation*}
\end{enumerate}
\begin{flushright}
\textsl{End of \autoref{defn:premodel-morphism}.}
\end{flushright}
\end{defn}

\begin{defn}
Suppose that $f:\mfM\to\mfN$ and $g:\mfN\to\mfN'$ are morphisms of internal models.
We define the composition $g\circ f:\mfM\to\mfN'$ to be the morphism given by
\begin{enumerate}
\item $(g\circ f)_0\defeq g_0\circ f_0$
\item $\mftypfunc{g\circ f}(\Gamma)\defeq\mftypfunc{g}(f_0(\Gamma))\circ\mftypfunc{f}(\Gamma)$
\item $\terms[g\circ f]{A}\defeq\terms[g]{\mftyp{f}{\Gamma}_0(A)}\circ\terms[f]{A}$.
\item a definition of $\alpha^{g\circ f}_0$ and $\alpha^{g\circ f}_1$.
\item a definition of $\beta^{g\circ f}_0$ and $\beta^{f\circ f}_1$.
\item a definition of $\gamma^{g\circ f}$. 
\end{enumerate}
\end{defn}

\begin{rmk}
The requirement that a morphism of models acts on terms is reminiscent of the requirement
that a functor acts on morphisms. In this fasion, the requirement that a morphism of
models preserves substitution is the counterpart of a functor preserving composition.
\end{rmk}
Context extension, weakening and substitution are required to be such morphims
of models. Thus they must be extended for this purpose.

\begin{description}
\item[Context extension] We will extend context extension to a morphism 
$\tfext(\Gamma):\mftyp{\mfM}{\Gamma}\to M$. Thus, we already have
$\tfext(\Gamma)_0\defeq\ctxext{\Gamma}{\blank}$. We also require 
\begin{align*}
\mftyp{\tfext(\Gamma)}{A} & : \mftyp{\mftyp{\mfM}{\Gamma}}{A}\to \mftyp{\mfM}{\ctxext{\Gamma}{A}}\\
\terms[\tfext(\Gamma)]{P} & : \terms[\mftyp{\mfM}{\Gamma}]{P}\to\terms[\mfM]{\mftyp{\tfext(\Gamma)}{A}_0(P)}
\end{align*}
For both of these we take the identity function.
We require furthermore that context extension preserves context extension,
weakening and substitution:
\begin{enumerate}
\item Extension preserves extension: for every family $P$ over $A$ in context
$\Gamma$ an identification $\id{\ctxext{{\Gamma}{A}}{P}}{\ctxext{\Gamma}{{A}{P}}}$.
\item Extension preserves weakening:
\item Extension preserves substitution:
Because context extension is the identity on the levels of types and terms,
the rules are easier.
\end{enumerate}
\item[Weakening] is a morphism $\tfwk:\mftyp{\mfM}{\Gamma}\to\mftyp{\mfM}{\ctxext{\Gamma}{A}}$, so we must have
\begin{align*}
\tfwk^A_0 & : \mftyp{\mfM}{\Gamma}\to\mftyp{\mfM}{\ctxext{\Gamma}{A}}
\intertext{which is denoted by $\ctxwk{A}{\blank}$,}
\mftypfunc{\tfwk^A} & : \mftyp{\mfM}{\ctxext{\Gamma}{B}}\to\mftyp{\mfM}{\ctxext{{\Gamma}{A}}{\ctxwk{A}{B}}}\\
\terms[\tfwk^A]{B} & : \terms[\mftyp{\mfM}{\Gamma}]{B}\to\terms[\mftyp{\mfM}{\ctxext{\Gamma}{A}}]{\ctxwk{A}{B}}
\end{align*}
for a term $y:B$, the term $\terms[{\tfwk^A}]{B}(y)$ is the constant map from
$A$ to $B$, assigning $y$ to every term of $A$.
\item[Substitution] is a morphism $\tfsubst^x:\mftyp{\mfM}{\ctxext{\Gamma}{A}}
\to\mftyp{\mfM}{\Gamma}$ for every $x:A$ in context $\Gamma$, so we must have
\begin{align*}
\tfsubst^x_0 & : \mftyp{\mfM}{\ctxext{\Gamma}{A}}\to\mftyp{\mfM}{\Gamma}
\intertext{which is denoted by $\subst{x}{\blank}$,}
\mftyp{\tfsubst^x}{P} & : \mftyp{\mfM}{\ctxext{{\Gamma}{A}}{P}}\to\mftyp{\mfM}{\ctxext{\Gamma}{\subst{x}{P}}}\\
\terms[\tfsubst^x]{P} & : \terms{P}\to\terms{\subst{x}{P}}
\end{align*}
The function $\terms[\tfsubst^x]{P}$ is usually denoted by $\tfev(\blank,x)$. 
\end{description}

\subsection{The basic type constructors in internal models}

\begin{defn}
\begin{enumerate}
\item A type $\mprd{A}{P}$ for every type $A$ in context $\Gamma$ and every type
$P$ in context $\ctxext{\Gamma}{A}$.
\begin{defn}
Suppose $A$ and $B$ are types in context $\Gamma$. We define
\begin{align*}
A\to B & \defeq\mprd{A}{\ctxwk{A}{B}}.
\intertext{For any two contexts $\Delta$ and $\Gamma$, we define}
\Delta\to\Gamma & \defeq e_\unit^{-1}(\Delta)\to e_\unit^{-1}(\Gamma)
\end{align*}
and furthermore we define $\ctxhom{\Delta}{\Gamma}\defeq\terms{\Delta\to\Gamma}$.
\end{defn}
\item An equivalence $\lambda:\eqv{\terms{P}}{\terms{\mprd{A}{P}}}$ for every
family $P$ over $A$ in context $\Gamma$. When $\ctxext{\Gamma}{A}\vdash 
u:P$, we call $\lambda(u)$ the \emph{$\lambda$-abstraction of $u$.}
\begin{rmk}
Note that we get an equivalence $\eqv{\terms{\unit^\Gamma\to A}}{\terms{A}}$ 
for every type $A$ in context $\Gamma$.
\end{rmk}
\begin{rmk}
Thus we see that $\eqv{\terms{A\to B}}{\terms{\ctxwk{A}{B}}}$
by $\lambda$-abstraction. 
\end{rmk}
%\item A function $\tfev:\terms{\mprd{A}{P}}\to\prd{x:\terms{A}}\terms{\subst{x}{P}}$.
\item A term $\pi^A:\ctxext{\Gamma}{A}\to \Gamma$ for every type $A$ in context
$\Gamma$.
\item A term $\iota^x:\ctxext{\Gamma}{\subst{x}{P}}\to\ctxext{{\Gamma}{A}}{P}$
for every family $P$ of types over $A$ in context $\Gamma$ and every
term $x:A$.
\item A type $A[f]$ in context $\Delta$ for every $f:\ctxhom{\Delta}{\Gamma}$ 
and every type $A$ in context $\Gamma$. {\color{blue}Could this be defined
in terms of the substitution $\subst{x}{P}$ we already have? It seems so. If this is
indeed the case we need either some identifications or we could just omit
this part of the definition.}
\item an identification $\id{B[\pi^A]}{\ctxwk{A}{B}}$ for any two types $A$
and $B$ in context $\Gamma$.
\item A type $\msm{A}{P}$ in context $\Gamma$ for every family $P$ over $A$
in context $\Gamma$, with an equivalence $\pairr{\blank,\blank}:\eqv{\sm{x:\terms{A}}\terms{\subst{x}{P}}}
{\terms{\msm{A}{P}}}$.
\item A family $\idtypevar{A}$ over $\ctxwk{A}{A}$ in context $\ctxext{\Gamma}{A}$ for
every type $A$ in context $\Gamma$ and a term of type.
\begin{rmk}
If $A$ and $B$ are types in context $\Gamma$, we may denote the type 
$\msm{A}{\ctxwk{A}{B}}$ by $A\times B$. By the end of the current definition
there will be an identification $\id{\ctxext{{\Gamma}{A}}{P}}
{\ctxext{\Gamma}{\msm{A}{P}}}$.

There is a term $\delta:\ctxhom{\ctxext{\Gamma}{A}}{\ctxext{{\Gamma}{A}}{\ctxwk{A}{A}}}$
defined by...
\end{rmk}
\item For any family $Q$ over $\idtypevar{A}$ in context $\ctxext{{\Gamma}{A}}{\ctxwk{A}{A}}$ an equivalence
$J:\eqv{\terms{Q}}{}$
\end{enumerate}
\end{defn}

\begingroup
\color{blue}
\subsubsection*{More desiderata}
Some of which hopefully follow from more elegant or general rules, but
this list is to keep them in mind:
\begin{enumerate}
\item $\id{e_\unit^{-1}(\Gamma.A)}{\msm{e_\unit^{-1}(\Gamma)}{\trans{(\eta_\Gamma)}{A}}}$, where
$\eta_\Gamma:\id{\Gamma}{e_\unit(e_\unit^{-1}(\Gamma))}$ is the unit of the
equivalence $e_\unit$.  
\item $\id{(\msm{A}{P}\to B)}{(P\to \ctxwk{A}{B})}$, expressing that $\Sigma$ is
left adjoint to weakening.
\item $\id{(\ctxwk{A}{B}\to P)}{(B\to\mprd{A}{P})}$ expressing that $\Pi$ is
right adjoint to weakening.
\item a term $\Gamma\vdash\idfunc[A]:A\to A$ for every type $A$ in context $\Gamma$.
\item a context $\UU$ for the universe.
\item an identification $\id{\pi^A\circ \pi^P\circ\iota^x}{\pi^{\subst{x}{P}}}$ for every
family $P$ of types over $A$ in context $\Gamma$ and every term $x:A$. In other
words, the diagram
\begin{equation*}
\begin{tikzcd}
\ctxext{{\Gamma}{A}}{P} \ar{r}{\pi^P} & \ctxext{\Gamma}{A} \ar{d}{\pi^A}\\
\ctxext{\Gamma}{\subst{x}{P}} \ar{u}{\iota^x} \ar{r}[swap]{\pi^{\subst{x}{P}}} & \Gamma
\end{tikzcd}
\end{equation*}
should commute.
\end{enumerate}
\endgroup

\begin{desiderata}
We should have a theorem stating that our internal model indeed interprets
the rules of type theory.
\end{desiderata}

\section{Weak $\omega$-groupoids}

\subsection{Trivial cofibrations and weak equivalences of types}
We describe a relation between types that expresses when they are weakly equivalent.
Weak equivalence is introduced because we need a weaker notion of judgmental 
equality which also makes sense when identity types are not present, since that
would allow us to state that context extension, weakening and substitution
commute with each other.

A term $f:\ctxwk{A}{B}$ is a trivial cofibration if it has the
property that for any fibration $Q$ over $B$,
to find a section of $Q$ it suffices to find a section of the fibration
$f^\ast Q$ over $A$. In our type theoretical setting, the rôle of fibrations
is played by families, the rôle of the function type $A\to B$ is played by
$\ctxwk{A}{B}$ and our version of the pullback $f^\ast Q$ is $\subst{f}{\ctxwk{A}{Q}}$.

\begin{defn}
Let $f$ be a term of $\ctxwk{A}{B}$ in context $\Gamma$.
\begin{enumerate}
\item For a family $Q$ over $B$ in context $\Gamma$ we define $f^\ast Q\jdeq\subst{f}{\ctxwk{A}{Q}}$.
\item For a term $g$ of $Q$ in context $\ctxext{\Gamma}{B}$ we define $f^\ast g\jdeq\subst{f}{\ctxwk{A}{g}}$.
\end{enumerate} 
\end{defn}
\begin{rmk}
The type $f^\ast(\ctxwk{B}{C})$ in context $\ctxext{\Gamma}{A}$ is can be viewed as the
type of functions from $A$ to $C$ which factor through $f$. Thus there should be
a function from $f^\ast(\ctxwk{B}{C})$ to $\ctxwk{A}{C}$.
\end{rmk}

\begin{rmk}
Every term $\jterm{\ctxext{\Gamma}{A}}{\ctxwk{A}{B}}{f}$ allows us to infer the following:
\begin{equation*}
\inference{\jtype{\ctxext{\Gamma}{B}}{Q}}{\jtype{\ctxext{\Gamma}{A}}{f^\ast Q}}
\qquad
\inference{\jterm{\ctxext{\Gamma}{B}}{Q}{g}}{\jterm{\ctxext{\Gamma}{A}}{f^\ast Q}{f^\ast g}}
\end{equation*}
\end{rmk}


\begin{defn}
A term $f:\ctxwk{A}{B}$ in context $\ctxext{\Gamma}{A}$ is said to be a trivial
cofibration if we can infer
\begin{equation*}
\inference{\jterm{\ctxext{\Gamma}{A}}{f^\ast Q}{t}}{\jterm{\ctxext{\Gamma}{B}}{Q}{\tilde{t}}}\qquad
\inference{\jterm{\ctxext{\Gamma}{A}}{f^\ast Q}{t}}{\jtermeq{\ctxext{\Gamma}{A}}{f^\ast Q}{f^\ast \tilde{t}}{t}}
\end{equation*}
{\color{red}This statement should be reformulated so that it only involves a single judgment...
but I don't see directly how to do that.}
\end{defn}

We have the following theorem in the type theory of \cite{TheBook}, which supports
our claim that we may indeed speak of a trivial cofibration. 

\begin{thm}
Suppose $f:A\to B$ is a function. Then $f$ is an equivalence if and only if
for every $Q:B\to\type$ and every $g:\prd{x:A}Q(f(x))$ there is a section
$h:\prd{y:B}Q(b)$ with the property that $h\circ f\htpy g$. 
\end{thm}

\begin{proof}
We can first take $Q$ to be the constant family $\lam{y}A$. Furthermore, we may
take $g\defeq\idfunc[A]$. Then we get a term of type
\begin{equation*}
\sm{h:B\to A}h\circ f\htpy \idfunc[A],
\end{equation*}
i.e.~we get a left inverse $h$ for $f$. To show that $h$ is also a right inverse
of $f$, let $Q$ be the family $\lam{y}\id{f(h(y))}{y}$. To find a section of
$Q$, which is the homotopy we aim for, it suffices to find a section of
$Q\circ f$. In other words, we have to show that $\id{f(h(f(x)))}{f(x)}$ for
every $x:A$. This follows from the fact that $h$ is a left inverse for $f$.

The reverse direction is immediate.
\end{proof}

Another approach would be to define $f:\ctxwk{A}{B}$ to be left invertible
if there is a term $\ctxext{{\Gamma}{A}}{P}$

\subsection{Some possibilities}

A weak omega groupoid is a model of type theory with $\Sigma$ and $\idtypevar{}$
in which there is a term of $\terms{A}$ for every type $A$ in context $\Gamma$.
This should work when univalence is around (so that functions can be replaced
by families).

We could also state that for every term $f:\terms{\ctxwk{A}{B}}$ and every
family $Q:\tftyp{\mfM}{\ctxext{\Gamma}{B}}$ there is a function
$\terms{\subst{f}{\ctxwk{A}{Q}}}\to\terms{Q}$ (asserting that $f$ is a trival
cofibration). This should always give that each $f$ is an equivalence and
this should be equivalent to the previous condition when univalence is around.

Another option arises when we see the identity type as an operation on a certain class of
terms. Usually, $\idtypevar{}$ is defined using the class of identity functions.
Let's take the class of all terms instead:


\part{Categorical semantics}

\section{E-objects in categories with finite limits}
In this section we assume that $\cat{C}$ is a finitely complete category and
whenever we write a pullback, we assume that it is chosen. Recall that for
any morphism $f:A\to B$ in a category $\cat{C}$ with chosen pullbacks, there
is a functor
\begin{equation*}
f^\ast : \cat{C}/B\to\cat{C}/A.
\end{equation*}
As usual, when $g:X\to B$ is a morphism, we will write $f^\ast(X)$ for the
domain of $f^\ast(g)$. When there is more than one morphism $X\to B$ involved,
as will be the case below, we will write $\pullback{A}{X}{f}{g}$. The projections
will be written as $\pullbackpr{1}{f}{g}$ and $\pullbackpr{2}{f}{g}$. So in this notation, a
typical pullback diagram has the following form:
\begin{equation*}
\begin{tikzcd}[column sep=large]
\pullback{A}{X}{f}{g}
  \ar{r}{\pullbackpr{1}{f}{g}}
  \ar{d}[swap]{\pullbackpr{2}{f}{g}}
  &
A \ar{d}{f}
  \\
X \ar{r}[swap]{g}
  &
B
\end{tikzcd}
\end{equation*}
Also, when we have a commutative diagram of the form
\begin{equation*}
\begin{tikzcd}
A \ar{r}{f}
  \ar{d}{a}
  &
X \ar{d}
  & 
B \ar{l}[swap]{g}
  \ar{d}{b}
  \\
A'
  \ar{r}[swap]{f'}
  &
X'
  &
B'
  \ar{l}{g'}
\end{tikzcd}
\end{equation*}
we will denote the unique map from $\pullback{A}{B}{f}{g}$ to $\pullback{A'}{B'}{f'}{g'}$
such that the diagram
\begin{equation*}
\begin{tikzcd}
  {}
  & 
\pullback{A'}{B'}{f'}{g'}
  \ar{dd}
  \ar{rr}
  &
  &
B'
  \ar{dd}{g'}
  \\
\pullback{A}{B}{f}{g}
  \ar{dd}
  \ar[crossing over]{rr}
  \ar[dotted]{ur}{\pullback{a}{b}{f'}{g'}}
  &
  &
B \ar{ur}{b}
  \\
  {}
  &
A'
  \ar{rr}
  &
  &
X'
  \\
A \ar{rr}[swap]{f}
  \ar{ur}{a}
  &
  &
X \ar[crossing over,leftarrow]{uu}[near end,swap]{g}
  \ar{ur}
\end{tikzcd}
\end{equation*}
commutes, by $\pullback{a}{b}{f'}{g'}$. In the current work, we shall
write $A\times B$ for the pullback of $A\rightarrow 1\leftarrow B$, and
$\pi_1$ and $\pi_2$ for its projections (thus, no separate choice of
cartesian products is made).


\subsection{Extension objects}
\begin{defn}
A \emph{pre-extension object $\stesys$ in $\cat{C}$} consists of a \emph{fundamental structure}, which is a diagram of the form
\begin{equation*}
\begin{tikzcd}
\stesyst
  \ar{d}[swap]{\ebd}
  \\
\stesysf
  \ar{d}[swap]{\eft}
  \\
\stesysc
\end{tikzcd}
\end{equation*}
in $\cat{C}$ together with the \emph{context extension} and \emph{family extension} operations
\begin{align*}
\ectxext &:\stesysf\to \stesysc\\
\efamext & :\stesysff\to \stesysf,
\end{align*}
respectively, such that the diagram
\begin{equation*}
\begin{tikzcd}
\stesysf_2 
  \ar{r}{\efamext} 
  \ar{d}[swap]{\eft[1]} 
  & 
\stesysf 
  \ar{d}{\eft}
  \\
\stesysf
  \ar{r}[swap]{\eft} 
  & 
\stesysc
\end{tikzcd}
\end{equation*}
commutes.
\end{defn}

\begin{defn}
We introduce the following notation:
\begin{align*}
\stesysf_2 
  & := \stesysff
  \\
\eft[1] 
  & := \pullbackpr{1}{\ectxext}{\eft} : \stesysf_2\to\stesysf
  \\
\stesysf_3 & := \pullback{\stesysf_2}{\stesysf_2}{\efamext}{\eft[1]}
  \\
\eft[2]
  & := \pullbackpr{1}{\efamext}{\eft[1]} : \stesysf_3\to\stesysf_2.
\end{align*}
Then it follows that the outer square in the diagram
\begin{equation*}
\begin{tikzcd}[column sep=large]
\stesysf_3
  \ar[dotted]{dr}{\eext{2}}
  \ar{rr}{\pullback{\pullbackpr{2}{\ectxext}{\eft}}{\pullbackpr{2}{\ectxext}{\eft}}{\ectxext}{\eft}}
  \ar{dd}[swap]{\eft[2]}
  & 
  &
\stesysf_2
  \ar{d}{\efamext}
  \\
  &
\stesysf_2
  \ar{d}[swap]{\eft[1]}
  \ar{r}{\pullbackpr{2}{\ectxext}{\eft}}
  &
\stesysf
  \ar{d}{\eft}
  \\
\stesysf_2
  \ar{r}[swap]{\eft[1]}
  &
\stesysf
  \ar{r}[swap]{\ectxext}
  &
\stesysc
\end{tikzcd}
\end{equation*}
commutes. We define $\eext{2}$ to be the unique morphism rendering the above diagram
commutative. Now we may continue to define
\begin{align*}
\stesysf_4 
  & := 
\pullback{\stesysf_3}{\stesysf_3}{\eext{2}}{\eft[2]}
  \\
\eft[3] 
  & := 
\pullbackpr{1}{\eext{2}}{\eft[2]}.
\end{align*}
Then we see that the outer square of the diagram
\begin{equation*}
\begin{tikzcd}[column sep=large]
\stesysf_4
  \ar[dotted]{dr}{\eext{3}}
  \ar{rr}{\pullback{\pullbackpr{2}{\efamext}{\eft[1]}}{\pullbackpr{2}{\efamext}{\eft[1]}}{\efamext}{\eft[1]}}
  \ar{dd}[swap]{\eft[3]}
  & 
  &
\stesysf_3
  \ar{d}{\eext{2}}
  \\
  &
\stesysf_3
  \ar{d}[swap]{\eft[2]}
  \ar{r}{\pullbackpr{2}{\efamext}{\eft[1]}}
  &
\stesysf_2
  \ar{d}{\eft[1]}
  \\
\stesysf_3
  \ar{r}[swap]{\eft[2]}
  &
\stesysf_2
  \ar{r}[swap]{\efamext}
  &
\stesysf
\end{tikzcd}
\end{equation*}
commutes,
so we may define $\eext{3}$ to be the unique map which renders the diagram. It
is straightforward to continue this process by induction, but we shall need not
go any further in this article.
\end{defn}

\begin{defn} An extension object is a pre-extension object $\stesys$ for which 
the diagrams
\begin{equation*}
\begin{tikzcd}
\stesysf_2 
  \ar{d}[swap]{\pullbackpr{2}{\ectxext}{\eft}} 
  \ar{r}{\efamext} 
  & 
\stesysf 
  \ar{d}{\ectxext}
  \\
\stesysf 
  \ar{r}[swap]{\ectxext} 
  & 
\stesysc
\end{tikzcd}
\qquad
\begin{tikzcd}
\stesysf_3
  \ar{d}[swap]{\pullbackpr{2}{\efamext}{\eft[1]}}
  \ar{r}{\eext{2}}
  & 
\stesysf_2 
  \ar{d}{\efamext} 
  \\
\stesysf_2 
  \ar{r}[swap]{\efamext} 
  &
\stesysf
\end{tikzcd}
\end{equation*}
commute.
\end{defn}

\begin{comment}
\begin{lem}
There exists an isomorphism $\alpha$ such that the triangle
\begin{equation*}
\begin{tikzcd}[column sep=tiny]
\pullback{\stesysf}{\stesysf_2}{\ectxext}{\eft\circ\eft[1]}
  \ar[dotted]{rr}{\alpha}
  \ar{dr}[swap]{\pullback{\catid{\stesysf}}{\efamext}{\ectxext}{\eft}}
  &
  &
\stesysf_3
  \ar{dl}{\eext{2}}
  \\
& \stesysf_2
\end{tikzcd}
\end{equation*}
commutes
\end{lem}

\begin{proof}
There is a unique morphism $\alpha:
\pullback{\stesysf}{\stesysf_2}{\ectxext}{\eft\circ\eft[1]}\to\stesysf_3$
rendering the diagram
\begin{equation*}
\begin{tikzcd}[column sep=large]
\pullback{\stesysf}{\stesysf_2}{\ectxext}{\eft\circ\eft[1]}
  \ar[bend left=10,yshift=.5ex]{drrr}{\pullbackpr{2}{\ectxext}{\eft}\circ\pullbackpr{2}{\ectxext}{\eft\circ\eft[1]}}
  \ar[bend right=10]{ddr}[swap]{\pullback{\catid{\stesysf}}{\eft[1]}{\ectxext}{\eft}}
  \ar[dotted]{dr}{\alpha}
  \\
& \stesysf_3
  \ar{r}{\pullbackpr{2}{\efamext}{\eft[1]}}
  \ar{d}{\eft[2]}
  &
\stesysf_2
  \ar{d}[swap]{\eft[1]}
  \ar{r}[swap]{\pullbackpr{2}{\ectxext}{\eft}}
  &
\stesysf
  \ar{d}{\eft}
  \\
{} & \stesysf_2
  \ar{r}[swap]{\efamext}
  &
\stesysf
  \ar{r}[swap]{\ectxext}
  &
\stesysc
\end{tikzcd}
\end{equation*}
\end{proof}
\end{comment}

\begin{defn}
Suppose $\stesys$ is a pre-extension object of $\cat{C}$. Then we define the pre-extension object
$\famesys{\stesys}$ to consist of the fundamental structure
\begin{equation*}
\begin{tikzcd}
\stesyst_2
  \ar{d}{\ebd[1]}
  \\
\stesysf_2
  \ar{d}{\eft[1]}
  \\
\stesysf
\end{tikzcd}
\end{equation*}
where
\begin{align*}
\stesyst_2 
  & := \pullback{\stesysf}{\stesyst}{\ectxext}{\eft\circ\ebd}
  \\
\ebd[1]
  & := \ectxext^\ast(\ebd),
\end{align*}
with the extension operations
\begin{align*}
\efamext 
  & 
  : \stesysf_2\to\stesysf\\
\eext{2} & : \stesysf_3\to\stesysf_2.
\end{align*}
\end{defn}

In \autoref{famextobj} below we shall show that $\famesys{\stesys}$ is an
extension algebra whenever $\stesys$ is an extension algebra. We shall need
a handful of lemmas to give the proof.

\begin{defn}
We define
\begin{align*}
\beta_1 
  & := 
\pullbackpr{2}{\ectxext}{\eft}
  & &
  : \stesysf_2\to\stesysf
  \\
\beta_2
  & :=
\pullback{\beta_1}{\beta_1}{\ectxext}{\eft}
  & &
  : \stesysf_3\to\stesysf_2
  \\
\beta_3
  & :=
\pullback{\beta_2}{\beta_2}{\efamext}{\eft[1]}
  & &
  : \stesysf_4\to\stesysf_3.
\end{align*}
\end{defn}

\begin{lem}
Let $\stesys$ be a pre-extension object. Then the square
\begin{equation*}
\begin{tikzcd}[column sep=10em]
\stesysf_4
  \ar{r}{\pullback{\pullbackpr{2}{\efamext}{\eft[1]}}{\pullbackpr{2}{\efamext}{\eft[1]}}{\efamext}{\eft[1]}}
  \ar{d}[swap]{\beta_3}
  &
\stesysf_3
  \ar{d}{\beta_2}
  \\
\stesysf_3
  \ar{r}[swap]{\beta_2}
  &
\stesysf_2
\end{tikzcd}
\end{equation*}
commutes.
\end{lem}

\begin{proof}
Left to the reader.
\end{proof}

\begin{comment}
\begin{proof}
It is straightforward to verify the equalities
\begin{align*}
\pullbackpr{1}{\ectxext}{\eft}\circ\beta_2\circ
  (\pullback{\pullbackpr{2}{\efamext}{\eft[1]}}{\pullbackpr{2}{\efamext}{\eft[1]}}{\efamext}{\eft[1]})
  & =
\pullbackpr{1}{\ectxext}{\eft}\circ\beta_2\circ\beta_3
  \\
\pullbackpr{2}{\ectxext}{\eft}\circ\beta_2\circ
  (\pullback{\pullbackpr{2}{\efamext}{\eft[1]}}{\pullbackpr{2}{\efamext}{\eft[1]}}{\efamext}{\eft[1]})
  & =
\pullbackpr{2}{\ectxext}{\eft}\circ\beta_2\circ\beta_3.\qedhere
\end{align*}
\end{proof}
\end{comment}

Note that the square
\begin{equation*}
\begin{tikzcd}
\stesysf_3
  \ar{r}{\beta_2}
  \ar{d}[swap]{\eext{2}}
  &
\stesysf_2
  \ar{d}{\efamext}
  \\
\stesysf_2
  \ar{r}[swap]{\beta_1}
  &
\stesysf
\end{tikzcd}
\end{equation*}
commutes by definition. We have a similar result relating $\eext{3}$ and
$\eext{2}$.

\begin{lem}
Let $\stesys$ be a pre-extension object. Then the square
\begin{equation*}
\begin{tikzcd}
\stesysf_4
  \ar{r}{\beta_3}
  \ar{d}[swap]{\eext{3}}
  &
\stesysf_3
  \ar{d}{\eext{2}}
  \\
\stesysf_3
  \ar{r}[swap]{\beta_2}
  &
\stesysf_2
\end{tikzcd}
\end{equation*}
commutes.
\end{lem}

\begin{proof}
Left to the reader.
\end{proof}
\begin{comment}
\begin{proof}
It is straightforward to verify the equalities
\begin{align*}
\pullbackpr{1}{\ectxext}{\eft}\circ\beta_2\circ\eext{3}
  & = \beta_1\circ\eft[2]\circ\eft[3]
  \\
\pullbackpr{1}{\ectxext}{\eft}\circ\eext{2}\circ\beta_3
  & = \beta_1\circ\eft[2]\circ\eft[3].
\end{align*}
Thus, it remains to verify that
\begin{equation*}
\pullbackpr{2}{\ectxext}{\eft}\circ\beta_2\circ\eext{3}
  = \pullbackpr{2}{\ectxext}{\eft}\circ\eext{2}\circ\beta_3.
\end{equation*}
It is straightforward to see that the diagram
\begin{equation*}
\begin{tikzcd}[column sep=large]
\stesysf_4
  \ar{dd}[swap]{\pullback{\pullbackpr{2}{\efamext}{\eft[1]}}{\pullbackpr{2}{\efamext}{\eft[1]}}{\efamext}{\eft[1]}}
  \ar{r}{\eext{3}}
  &
\stesysf_3
  \ar{r}{\beta_2}
  \ar{d}[swap]{\pullbackpr{2}{\efamext}{\eft[1]}}
  &
\stesysf_2
  \ar{dd}{\pullbackpr{2}{\ectxext}{\eft}}
  \\
  {} &
\stesysf_2
  \ar{dr}{\beta_1}
  \\
\stesysf_3
  \ar{ur}{\eext{2}}
  \ar{r}[swap]{\beta_2}
  &
\stesysf_2
  \ar{r}[swap]{\efamext}
  &
\stesysf
\end{tikzcd}
\end{equation*}
commutes. It is likewise straightforward to see that the diagram
\begin{equation*}
\begin{tikzcd}
\stesysf_4
  \ar{r}{\beta_3}
  \ar{d}[swap]{\pullback{\pullbackpr{2}{\efamext}{\eft[1]}}{\pullbackpr{2}{\efamext}{\eft[1]}}{\efamext}{\eft[1]}}
  &
\stesysf_3
  \ar{r}{\eext{2}}
  \ar{d}[swap]{\beta_2}
  &
\stesysf_2
  \ar{d}{\pullbackpr{2}{\ectxext}{\eft}}
  \\
\stesysf_3
  \ar{r}[swap]{\beta_2}
  &
\stesysf_2
  \ar{r}[swap]{\efamext}
  &
\stesysf
\end{tikzcd}
\end{equation*}
commutes, completing our goal.
\end{proof}
\end{comment}

\begin{thm}[Local extension structure]\label{famextobj}
If $\stesys$ is an extension object, then so is $\famesys{\stesys}$.
\end{thm}

\begin{proof}
Note that the diagram
\begin{equation*}
\begin{tikzcd}
\stesysf_3
  \ar{d}[swap]{\pullbackpr{2}{\efamext}{\eft[1]}}
  \ar{r}{\eext{2}}
  & 
\stesysf_2 
  \ar{d}{\efamext} 
  \\
\stesysf_2 
  \ar{r}[swap]{\efamext} 
  &
\stesysf
\end{tikzcd}
\end{equation*}
commutes by assumption. For the second condition, we have to show that the
diagram
\begin{equation*}
\begin{tikzcd}
\stesysf_4
  \ar{d}[swap]{\pullbackpr{2}{\eext{2}}{\eft[2]}}
  \ar{r}{\eext{3}}
  & 
\stesysf_3
  \ar{d}{\eext{2}} 
  \\
\stesysf_3
  \ar{r}[swap]{\eext{2}} 
  &
\stesysf_2
\end{tikzcd}
\end{equation*}
Since this is a question about two maps into a pullback, it suffices to verify
that
\begin{align*}
\pullbackpr{1}{\ectxext}{\eft}\circ\eext{2}\circ\eext{3}
  & =
\pullbackpr{1}{\ectxext}{\eft}\circ\eext{2}\circ\pullbackpr{2}{\eext{2}}{\eft[2]}
  \\
\pullbackpr{2}{\ectxext}{\eft}\circ\eext{2}\circ\eext{3}
  & =
\pullbackpr{2}{\ectxext}{\eft}\circ\eext{2}\circ\pullbackpr{2}{\eext{2}}{\eft[2]}.
\end{align*}
For the first equality, it is fairly straightforward to show that both the
equalities
\begin{equation*}
\pullbackpr{1}{\ectxext}{\eft}\circ\eext{2}\circ\eext{3}
  =
\eft[1]\circ\eft[2]\circ\eft[3]
\end{equation*}
and
\begin{equation*}
\pullbackpr{1}{\ectxext}{\eft}\circ\eext{2}\circ\pullbackpr{2}{\eext{2}}{\eft[2]}
  =
\eft[1]\circ\eft[2]\circ\eft[3].
\end{equation*}
hold. For the second subgoal (which is more tricky). Notice first that the
diagram
\begin{equation*}
\begin{tikzcd}
\stesysf_4
  \ar{r}{\eext{3}}
  \ar{d}[swap]{\beta_3}
  &
\stesysf_3
  \ar{r}{\eext{2}}
  \ar{d}[swap]{\beta_2}
  &
\stesysf_2
  \ar{d}{\pullbackpr{2}{\ectxext}{\eft}}
  \\
\stesysf_3
  \ar{r}[swap]{\eext{2}}
  &
\stesysf_2
  \ar{r}[swap]{\efamext}
  &
\stesysf
\end{tikzcd}
\end{equation*}
commutes. We also have the commutative diagram
\begin{equation*}
\begin{tikzcd}[column sep=large]
\stesysf_4
  \ar{r}{\pullbackpr{2}{\eext{2}}{\eft[2]}}
  \ar{d}[swap]{\beta_3}
  &
\stesysf_3
  \ar{r}{\eext{2}}
  \ar{d}[swap]{\beta_2}
  &
\stesysf_2
  \ar{d}{\pullbackpr{2}{\ectxext}{\eft}}
  \\
\stesysf_3
  \ar{r}{\pullbackpr{2}{\efamext}{\eft[1]}}
  \ar{dr}[swap]{\eext{2}}
  &
\stesysf_2
  \ar{r}{\efamext}
  &
\stesysf
  \\
  {} &
\stesysf_2
  \ar{ur}[swap]{\efamext}
\end{tikzcd}
\end{equation*}
completing the proof.
\end{proof}

\subsection{(Pre-)extension homomorphisms}\label{subsection:e_extension_homomorphisms}
In this subsection we start with the study of pre-extension homomorphisms, which
will include the extension homomorphisms since they will be the pre-extension
homomorphisms of which both the domain and codomain are extension objects.
Our main examples of extension homomorphisms will be the operations of weakening
and substitution. There are some basic examples of pre-extension homomorphisms
that will be useful too, which get introduced in the this section and in
\autoref{subsection:change_of_base}. In this section, we will mainly be
interested in pre-extension homomorphisms between local pre-extension objects.
We will end this section by proving that a retract of an extension object is
always an extension object.

\begin{defn}
Let $\stesys$ and $\stesys'$ be pre-extension algebras. A \emph{pre-extension 
homomorphism $f$ from $\stesys$ to $\stesys'$} is a triple $(f_0,f_1,f^t)$ 
consisting of morphisms
\begin{equation*}
\begin{tikzcd}
\stesyst 
  \ar{r}{f^t}
  \ar{d}[swap]{\ebd}
  &
\stesyst'
  \ar{d}{\ebd'}
  \\
\stesysf 
  \ar{r}{f_1}
  \ar{d}[swap]{\eft}
  &
\stesysf'
  \ar{d}{\eft'}
  \\
\stesysc 
  \ar{r}[swap]{f_0}
  &
\stesysc'
\end{tikzcd}
\end{equation*}
such that the indicated squares commute, for which furthermore the squares
\begin{equation*}
\begin{tikzcd}
\stesysf \ar{r}{f_1}
  \ar{d}[swap]{\ectxext}
  &
\stesysf'
  \ar{d}{\ectxext'}
  \\
\stesysc
  \ar{r}[swap]{f_0}
  &
\stesysc'
\end{tikzcd}
\end{equation*}
and
\begin{equation*}
\begin{tikzcd}[column sep=large]
\stesysf\times_{\ectxext,\eft} \stesysf
  \ar{r}{f_1\times_{\ectxext',\eft'} f_1}
  \ar{d}[swap]{\efamext}
  &
\stesysf'\times_{\ectxext',\eft'} \stesysf'
  \ar{d}{\efamext'}
  \\
\stesysf
  \ar{r}[swap]{f_1}
  &
\stesysf'
\end{tikzcd}
\end{equation*}
Composition and the identity homomorphism are defined in the expected way. We
define furthermore
\begin{align*}
f_2 & := \pullback{f_1}{f_1}{\ectxext'}{\eft'}
  \\
f_3 & := \pullback{f_2}{f_2}{\efamext'}{\eft[1]'}.
\end{align*}
\end{defn}

\begin{defn}
A pre-extension homomorphism between extension objects is called an extension
homomorphism.
\end{defn}

\begin{lem}
Let $\stesys$ be an extension object. Then
\begin{equation*}
\begin{tikzcd}[column sep=large]
\stesyst_2
  \ar{r}{\pullbackpr{2}{\ectxext}{\eft\circ\ebd}}
  \ar{d}[swap]{\ebd[1]}
  &
\stesyst
  \ar{d}{\ebd}
  \\
\stesysf_2
  \ar{r}{\pullbackpr{2}{\ectxext}{\eft}}
  \ar{d}[swap]{\eft[1]}
  &
\stesysf
  \ar{d}{\eft}
  \\
\stesysf
  \ar{r}[swap]{\ectxext}
  &
\stesysc
\end{tikzcd}
\end{equation*}
assembles an extension homomorphism $\mathbf{e}_0:\famesys{\stesys}\to\stesys$.
Likewise, we have an extension homomorphism
$\mathbf{e}_1:\famesys{\famesys{\stesys}}\to\famesys{\stesys}$. Thus, a
pre-extension object is an extension object if and only if $\mathbf{e}_0$
and $\mathbf{e}_1$ are pre-extension homomorphisms.
\end{lem}

\begin{proof}
Immediate from the conditions of being an extension object.
\end{proof}

\begin{defn}
Let $\stesys$ be a pre-extension object. Then
\begin{equation*}
\begin{tikzcd}
\stesyst_3
  \ar{r}{\beta^t}
  \ar{d}[swap]{\ebd[2]}
  &
\stesyst_2
  \ar{d}{\ebd[1]}
  \\
\stesysf_3
  \ar{r}{\beta_2}
  \ar{d}[swap]{\eft[2]}
  &
\stesysf_2
  \ar{d}{\eft[1]}
  \\
\stesysf_2
  \ar{r}[swap]{\beta_1}
  &
\stesysf
\end{tikzcd}
\qquad
\text{and}
\qquad
\begin{tikzcd}
\stesyst_4
  \ar{r}{\beta^t_2}
  \ar{d}[swap]{\ebd[3]}
  &
\stesyst_3
  \ar{d}{\ebd[2]}
  \\
\stesysf_4
  \ar{r}{\beta_3}
  \ar{d}[swap]{\eft[3]}
  &
\stesysf_3
  \ar{d}{\eft[2]}
  \\
\stesysf_3
  \ar{r}[swap]{\beta_2}
  &
\stesysf_2
\end{tikzcd}
\end{equation*}
assemble pre-extension homomorphisms 
\(
\boldsymbol{\beta}
  :
\famesys{\famesys{\stesys}}
  \to
\famesys{\stesys}
\) 
and
\(
\boldsymbol{\beta}_\mathbf{2}
  :
\famesys{\famesys{\famesys{\stesys}}}
  \to
\famesys{\famesys{\stesys}}
\).
\end{defn}

\begin{defn}\label{famehom}
Suppose that $f:\stesys'\to\stesys$ is a pre-extension homomorphism. Then we
define $\famehom{f}:\famesys{\stesys'}\to\famesys{\stesys}$ to consist of
\begin{equation*}
\begin{tikzcd}
\stesyst_2'
  \ar{r}{f^t_2}
  \ar{d}[swap]{\ebd[1]'}
  &
\stesyst_2
  \ar{d}{\ebd[1]}
  \\
\stesysf_2'
  \ar{r}{f_2}
  \ar{d}[swap]{\eft[1]'}
  &
\stesysf_2
  \ar{d}{\eft[1]}
  \\
\stesysf'
  \ar{r}[swap]{f_1}
  &
\stesysf
\end{tikzcd}
\end{equation*}
where we define
\begin{equation*}
f^t_2 := \pullback{f_1}{f^t}{\ectxext}{\eft\circ\ebd}.
\end{equation*}
\end{defn}

\begin{lem}
The triple $\famehom{f}$ defined in \autoref{famehom} is a pre-extension homomorphism.
\end{lem}

\begin{proof}
Note that the square
\begin{equation*}
\begin{tikzcd}
\stesysf_2'
  \ar{r}{f_2}
  \ar{d}[swap]{\efamext'}
  &
\stesysf_2
  \ar{d}{\efamext}
  \\
\stesysf'
  \ar{r}[swap]{f_1}
  &
\stesysf
\end{tikzcd}
\end{equation*}
commutes by assumption. Thus, it remains to show that the square
\begin{equation*}
\begin{tikzcd}
\stesysf_3'
  \ar{r}{f_3}
  \ar{d}[swap]{\eext{2}'}
  &
\stesysf_3
  \ar{d}{\eext{2}}
  \\
\stesysf_2'
  \ar{r}[swap]{f_2}
  &
\stesysf_2
\end{tikzcd}
\end{equation*}
commutes. It is equivalent to show that the equalities
\begin{align*}
\pullbackpr{1}{\ectxext}{\eft}\circ f_2\circ\eext{2}'
  & =
\pullbackpr{1}{\ectxext}{\eft}\circ \eext{2}\circ f_3
  \\
\pullbackpr{2}{\ectxext}{\eft}\circ f_2\circ\eext{2}'
  & =
\pullbackpr{2}{\ectxext}{\eft}\circ \eext{2}\circ f_3
\end{align*}
both hold. For the first, it is straightforward to verify that the diagram
\begin{equation*}
\begin{tikzcd}[column sep=large]
{} &
\stesysf_2'
  \ar{r}{f_2}
  \ar{dr}[near end]{\pullbackpr{1}{\ectxext'}{\eft'}}
  &
\stesysf_2
  \ar{dr}{\pullbackpr{1}{\ectxext}{\eft}}
  \\
\stesysf_3'
  \ar{ur}{\eext{2}'}
  \ar{r}[swap]{\beta_2'}
  \ar{ddr}[swap]{f_3}
  &
\stesysf_2'
  \ar{r}{\efamext'}
  \ar{dr}[swap]{f_2}
  &
\stesysf'
  \ar{r}{f_1}
  &
\stesysf
  \\
{} & {} &
\stesysf_2
  \ar{ur}[near start]{\efamext}
  \\
{} &
\stesysf_3
  \ar{r}[swap]{\eext{2}}
  \ar{ur}{\beta_2}
  &
\stesysf_2
  \ar{uur}[swap]{\pullbackpr{1}{\ectxext}{\eft}}
\end{tikzcd}
\end{equation*}
commutes. For the second, note that the diagram
\begin{equation*}
\begin{tikzcd}[column sep=large]
{} &
\stesysf_2'
  \ar{r}{f_2}
  \ar{dr}{\beta_1'}
  &
\stesysf_2
  \ar{ddr}{\pullbackpr{2}{\ectxext}{\eft}}
  \\
{} & {} &
\stesysf'
  \ar{dr}[swap,near start]{f_1}
  \\
\stesysf_3'
  \ar{uur}{\eext{2}'}
  \ar{r}{\beta_2'}
  \ar{dr}[swap]{f_3}
  &
\stesysf_2'
  \ar{r}{f_2}
  \ar{ur}{\efamext'}
  &
\stesysf_2
  \ar{r}[swap]{\efamext}
  &
\stesysf
  \\
{} &
\stesysf_3
  \ar{r}[swap]{\eext{2}}
  \ar{ur}[near start]{\beta_2}
  &
\stesysf_2
  \ar{ur}[swap]{\pullbackpr{2}{\ectxext}{\eft}}
\end{tikzcd}
\end{equation*}
commutes.
\end{proof}

\begin{lem}[Stability under retracts]\label{esys-retract}
Suppose $f:\stesys\to\stesys'$ is a pre-extension homomorphism between
pre-extension objects. If there is a pre-extension homomorphism $g:\stesys'\to
\stesys$ such that $g\circ f=\catid{\stesys}$ and $\stesys'$ is an extension
algebra, then $\stesys$ is an extension algebra.
\end{lem}

Before we start with the proof, note that we have the equalities
$g_2\circ f_2=\catid{\stesysf_2}$ and $g_3\circ f_3=\catid{\stesysf_3}$
under the hypotheses of the lemma.

\begin{proof}
Our first subgoal is to show that the square
\begin{equation*}
\begin{tikzcd}
\stesysf_2 
  \ar{r}{\efamext} 
  \ar{d}[swap]{\pullbackpr{2}{\ectxext}{\eft}} 
  & 
\stesysf 
  \ar{d}{\ectxext}
  \\
\stesysf
  \ar{r}[swap]{\ectxext} 
  & 
\stesysc
\end{tikzcd}
\end{equation*}
commutes. Note that in the diagram
\begin{equation*}
\begin{tikzcd}
  {}
  & 
\stesysf
  \ar{dd}[near start]{\ectxext}
  \ar{rr}{f_1}
  &
  &
\stesysf'
  \ar{dd}[near start]{\ectxext'}
  \ar{rr}{g_1}
  &
  &
\stesysf
  \ar{dd}{\ectxext}
  \\
\stesysf_2
  \ar{dd}[swap]{\pullbackpr{2}{\ectxext}{\eft}}
  \ar[crossing over]{rr}[swap,near start]{f_2}
  \ar{ur}{\efamext}
  &
  &
\stesysf_2'
  \ar{ur}[near start]{\efamext'}
  \ar[crossing over]{rr}[swap,near start]{g_2}
  &
  &
\stesysf_2
  \ar{ur}[swap,near start]{\efamext}
  \\
  {}
  &
\stesysc
  \ar{rr}[near start]{f_0}
  &
  &
\stesysc'
  \ar{rr}[near start]{g_0}
  &
  &
\stesysc
  \\
\stesysf 
  \ar{rr}[swap]{f_1}
  \ar{ur}{\ectxext}
  &
  &
\stesysf' 
  \ar[crossing over,leftarrow]{uu}[near end,swap]{\pullbackpr{2}{\ectxext'}{\eft'}}
  \ar{ur}[swap,near end]{\ectxext'}
  \ar{rr}[swap]{g_1}
  &
  &
\stesysf
  \ar[crossing over,leftarrow]{uu}[near end,swap]{\pullbackpr{2}{\ectxext}{\eft}}
  \ar{ur}[swap]{\ectxext}
\end{tikzcd}
\end{equation*}
all the faces minus the far left and far right face commute. Using that $g$
is a section of $f$, we can read off that also the far left face commutes,
completing our first subgoal.
 
For the second subgoal, note that also $\famehom{g}\circ\famehom{f}=
\catid{\famesys{\stesys}}$ and that $\famesys{\stesys'}$ is an extension object.
Thus we can apply what we have proven so far to conclude that the square
\begin{equation*}
\begin{tikzcd}
\stesysf_3 
  \ar{r}{\eext{2}} 
  \ar{d}[swap]{\pullbackpr{2}{\efamext}{\eft[1]}} 
  & 
\stesysf_2 
  \ar{d}{\efamext}
  \\
\stesysf_2
  \ar{r}[swap]{\efamext} 
  & 
\stesysf
\end{tikzcd}
\end{equation*}
commutes.
\end{proof}

\subsection{The change of base of (pre-)extension objects}
\label{subsection:change_of_base}
An important construction of (pre-)extension objects is the change of base. It
allows us to consider `parametrized homomorphisms', such as weakening and
substitution.

\begin{defn}
Suppose $f:\stesys\to\stesys'$ is a pre-extension homomorphism. We say that
a diagram
\begin{equation*}
\begin{tikzcd}
\stesys
  \ar{r}{f}
  \ar{d}[swap]{p}
  &
\stesys'
  \ar{d}{p'}
  \\
X \ar{r}[swap]{g}
  &
Y
\end{tikzcd}
\end{equation*}
commutes if the diagram
\begin{equation*}
\begin{tikzcd}
\stesysc
  \ar{r}{f_0}
  \ar{d}[swap]{p}
  &
\stesysc'
  \ar{d}{p'}
  \\
X \ar{r}[swap]{g}
  &
Y
\end{tikzcd}
\end{equation*}
commutes.
\end{defn}

The first goal in this subsection is to define for every (pre-)extension object 
$\stesys$ and every $p:\stesysc\rightarrow X\leftarrow Y:g$, a (pre-)extension
object $\cobesys{Y}{\stesys}{g}{p}$ with a homomorphism $\pullbackpr{2}{g}{p}:
\cobesys{Y}{\stesys}{g}{p}\to\stesys$ and a morphism $\pullbackpr{1}{g}{p}:
\pullback{Y}{\stesysc}{g}{p}\to Y$ such that for every diagram
\begin{equation*}
\begin{tikzcd}[column sep=large]
\stesys'
  \ar[bend right=10]{ddr}[swap]{p'}
  \ar[bend left=10]{rrd}{f}
  \ar[dotted]{dr}[near end]{[p',f]}
  \\
  {}&
\cobesys{Y}{\stesys}{g}{p}
  \ar{d}{\pullbackpr{1}{g}{p}}
  \ar{r}[swap]{\pullbackpr{2}{g}{p}}
  &
\stesys
  \ar{d}{p}
  \\
  {}&
Y \ar{r}[swap]{g}
  &
X
\end{tikzcd}
\end{equation*}
of which the outer square commutes, the (pre-)extension homomorphism $[p',f]$ exists
and is unique with the property that it renders the diagram commutative. We will
give the definition of $\cobesys{Y}{\stesys}{g}{p}$ in \autoref{cobesys}. After
proving that the change of base of a pre-extension algebra is indeed a
pre-extension algebra (\autoref{cobesys-preext}) and that the change of base
of an extension algebra is an extension algebra (\autoref{cobesys-ext}), we
will demonstrate the above unique existence in \autoref{cobesys-existence,%
cobesys-pullback}.

The second goal in this subsection is to follow the same procedure for
$\famesys{\famesys{\stesys}}$ to show that it is equivalent to
$\cobesys{\stesysf}{\famesys{\stesys}}{\ectxext}{\eft}$. We will do this by
verifying directly that it has the universal property of the change of base
described above, because we will use the ingredients in our definition of
weakening and substitution objects.

\begin{defn}[Change of base]\label{cobesys}
Suppose $\stesys$ is a pre-extension object in $\cat{C}$ and that 
$p:\stesysc\rightarrow X\leftarrow Y:g$.
Then we define the pre-extension object $\cobesys{Y}{\stesys}{g}{p}$ to consist of
\begin{equation*}
\begin{tikzcd}
\cobesys{Y}{\stesyst}{g}{p\circ\eft\circ\ebd}
  \ar{r}
  \ar{d}[swap]{g^\ast(\ebd)}
  &
\stesyst
  \ar{d}{\ebd}
  \\
\cobesys{Y}{\stesysf}{g}{p\circ\eft}
  \ar{r}
  \ar{d}[swap]{g^\ast(\eft)}
  &
\stesysf
  \ar{d}{\eft}
  \\
\cobesys{Y}{\stesysc}{g}{p}
  \ar{r}
  \ar{d}[swap]{\pullbackpr{1}{g}{p}}
  &
\stesysc
  \ar{d}{p}
  \\
Y \ar{r}[swap]{g}
  &
X
\end{tikzcd}
\end{equation*} 
and the operations
\begin{align*}
\cobesys{Y}{\ectxext}{g}{p} 
  & : \pullback{Y}{\stesysf}{g}{p\circ\eft}\to \pullback{Y}{\stesysc}{g}{p}\\
\cobesys{Y}{\efamext}{g}{p} 
  & : \pullback
    {\pullback{Y}{\stesysf}{g}{p\circ\eft}}
    {\pullback{Y}{\stesysf}{g}{p\circ\eft}}
    {\cobesys{Y}{\ectxext}{g}{p}}
    {g^\ast(\eft)}
  \to 
  \pullback{Y}{\stesysf}{g}{p\circ\eft}.
\end{align*}
defined by
\begin{equation*}
\cobesys{Y}{\ectxext}{g}{p} := \pullback{\catid{Y}}{\ectxext}{g}{p}
\end{equation*}
and where $\cobesys{Y}{\efamext}{g}{p}$ is defined by rendering the diagram
\begin{equation*}
\begin{tikzcd}[column sep=large]
(\cobesys{Y}{\stesysf}{g}{p\circ\eft})_2
  \ar{rr}{\pullback{\pullbackpr{2}{g}{p\circ\eft}}{\pullbackpr{2}{g}{p\circ\eft}}{\ectxext}{\eft}}
  \ar{dd}[swap]{\pullbackpr{1}{\cobesys{Y}{\ectxext}{g}{p}}{g^\ast(\eft)}}
  \ar[dotted]{dr}[swap]{\cobesys{Y}{\efamext}{g}{p}}
  &
  &
\stesysf_2
  \ar{d}{\efamext}
  \\
  {}&
\cobesys{Y}{\stesysf}{g}{p\circ\eft}
  \ar{r}{\pullbackpr{2}{g}{p\circ\eft}}
  \ar{d}[swap]{\pullbackpr{1}{g}{p\circ\eft}}
  &
\stesysf
  \ar{d}{p\circ\eft}
  \\
\cobesys{Y}{\stesysf}{g}{p\circ\eft}
  \ar{r}[swap]{\pullbackpr{1}{g}{p\circ\eft}}
  &
Y \ar{r}[swap]{g}
  &
X
\end{tikzcd}
\end{equation*} 
commutative. 
The process of obtaining the pre-extension object $\cobesys{Y}{\stesys}{g}{p}$ out of $\stesys$
and $g:Y\to X$ is also called the \emph{change of base}.
\end{defn}

\begin{lem}\label{cobesys-preext}
Any change of base of a pre-extension object is a pre-extension object.
\end{lem}

\begin{proof}
Let $\stesys$ be an extension algebra and consider $p:\stesysc\rightarrow X\leftarrow Y:g$.
We need to verify that the square
\begin{equation*}
\begin{tikzcd}[column sep=large]
(\pullback{Y}{\stesysf}{g}{p\circ\eft})_2
  \ar{r}{\cobesys{Y}{\efamext}{g}{p}} 
  \ar{d}[swap]{\pullbackpr{1}{\cobesys{Y}{\ectxext}{g}{p}}{g^\ast(\eft)}} 
  & 
\pullback{Y}{\stesysf}{g}{p\circ\eft}
  \ar{d}{g^\ast(\eft)}
  \\
\pullback{Y}{\stesysf}{g}{p\circ\eft}
  \ar{r}[swap]{g^\ast(\eft)} 
  & 
\pullback{Y}{\stesysc}{g}{p}
\end{tikzcd}
\end{equation*}
commutes. It is fairly obvious that
\begin{equation*}
\pullbackpr{1}{g}{p}\circ g^\ast(\eft)\circ (\cobesys{Y}{\efamext}{g}{p})
  =
\pullbackpr{1}{g}{p\circ\eft}\circ \pullbackpr{1}{\cobesys{Y}{\ectxext}{g}{p}}{g^\ast(\eft)}
\end{equation*}
and that the diagram
\begin{equation*}
\begin{tikzcd}
  {}&
  {}&
\pullback{Y}{\stesysf}{g}{p\circ\eft}
  \ar{rr}{g^\ast(\eft)}
  \ar{dr}[swap]{\pullbackpr{2}{g}{p\circ\eft}}
  &
  {}&
\pullback{Y}{\stesysc}{g}{p}
  \ar{ddr}{\pullbackpr{2}{g}{p}}
  \\
  {}&
  {}&
  {}&
\stesysf
  \ar{drr}[swap]{\eft}
  \\
(\pullback{y}{\stesysf}{g}{p\circ\eft})_2
  \ar{uurr}{\cobesys{Y}{\efamext}{g}{p}}
  \ar{rr}[swap,yshift=-.5ex]{\pullback{\pullbackpr{2}{g}{p\circ\eft}}{\pullbackpr{2}{g}{p\circ\eft}}{\ectxext}{\eft}}
  \ar{ddrr}[swap]{\pullbackpr{1}{\cobesys{Y}{\ectxext}{g}{p}}{g^\ast(\eft)}}
  &
  {}&
\stesysf_2
  \ar{ur}{\efamext}
  \ar{dr}[swap]{\eft[1]}
  &
  {}&
  {}&
\stesysc
  \\
  {}&
  {}&
  {}&
\stesysf
  \ar{urr}{\eft}
  \\
  {}&
  {}&
\pullback{Y}{\stesysf}{g}{p\circ\eft}
  \ar{rr}[swap]{g^\ast(\eft)}
  \ar{ur}{\pullbackpr{2}{g}{p\circ\eft}}
  &
  {}&
\pullback{Y}{\stesysc}{g}{p}
  \ar{uur}[swap]{\pullbackpr{2}{g}{p}}
\end{tikzcd}
\end{equation*}
commutes.
\end{proof}

\begin{thm}\label{cobesys-ext}
The change of base of an extension algebra is an extension algebra.
\end{thm}

\begin{proof}
Our first subgoal is to verify that the square
\begin{equation*}
\begin{tikzcd}[column sep=large]
(\pullback{Y}{\stesysf}{g}{p\circ\eft})_2
  \ar{r}{\cobesys{Y}{\efamext}{g}{p}} 
  \ar{d}[swap]{\pullbackpr{2}{\cobesys{Y}{\ectxext}{g}{p}}{g^\ast(\eft)}} 
  & 
\pullback{Y}{\stesysf}{g}{p\circ\eft}
  \ar{d}{\cobesys{Y}{\ectxext}{g}{p}}
  \\
\pullback{Y}{\stesysf}{g}{p\circ\eft}
  \ar{r}[swap]{\cobesys{Y}{\ectxext}{g}{p}} 
  & 
\pullback{Y}{\stesysc}{g}{p}
\end{tikzcd}
\end{equation*}
\end{proof}

The following construction is useful for defining extension homomorphisms into
`higher' extension objects

\begin{defn}\label{cobesys-existence}
Consider a commutative diagram
\begin{equation*}
\begin{tikzcd}
\stesys'
  \ar{r}{f}
  \ar{d}[swap]{p'}
  &
\stesys
  \ar{d}{p}
  \\
Y \ar{r}[swap]{g}
  &
X
\end{tikzcd}
\end{equation*}
Then we construct $[p,f]:\stesys'\to\cobesys{Y}{\stesys}{g}{p}$
\begin{itemize}
\item by defining $[p,f]_0:\stesysc'\to\pullback{Y}{\stesysc}{g}{p}$ be the uniqe
morphism rendering the diagram
\begin{equation*}
\begin{tikzcd}[column sep=large]
\stesysc'
  \ar[bend right=10]{ddr}[swap]{p'}
  \ar[bend left=10]{rrd}{f_0}
  \ar{dr}[near end]{[p,f]_0}
  \\
  {}&
\pullback{Y}{\stesysc}{g}{p}
  \ar{r}[swap]{\pullbackpr{2}{g}{p}}
  \ar{d}{\pullbackpr{1}{g}{p}}
  &
\stesysc
  \ar{d}{p}
  \\
  {}&
Y \ar{r}[swap]{g}
  &
X
\end{tikzcd}
\end{equation*}
commutative.
\item by defining $[p,f]_1:\stesysf'\to\pullback{Y}{\stesysf}{g}{p\circ\eft}$ be the uniqe
morphism rendering the diagram
\begin{equation*}
\begin{tikzcd}[column sep=huge]
\stesysf'
  \ar[bend right=10]{ddr}[swap]{p'\circ\eft'}
  \ar[bend left=10]{rrd}{f_1}
  \ar{dr}[near end]{[p,f]_1}
  \\
  {}&
\pullback{Y}{\stesysf}{g}{p\circ\eft}
  \ar{r}[swap]{\pullbackpr{2}{g}{p\circ\eft}}
  \ar{d}{\pullbackpr{1}{g}{p\circ\eft}}
  &
\stesysc
  \ar{d}{p\circ\eft}
  \\
  {}&
Y \ar{r}[swap]{g}
  &
X
\end{tikzcd}
\end{equation*}
commutative.
\item by defining $[p,f]^t:\stesyst'\to\pullback{Y}{\stesyst}{g}{p\circ\eft\circ\ebd}$ be the uniqe
morphism rendering the diagram
\begin{equation*}
\begin{tikzcd}[column sep=huge]
\stesyst'
  \ar[bend right=10]{ddr}[swap]{p'\circ\eft'\circ\ebd'}
  \ar[bend left=10]{rrd}{f^t}
  \ar{dr}[near end]{[p,f]^t}
  \\
  {}&
\pullback{Y}{\stesysf}{g}{p\circ\eft\circ\ebd}
  \ar{r}[swap]{\pullbackpr{2}{g}{p\circ\eft\circ\ebd}}
  \ar{d}{\pullbackpr{1}{g}{p\circ\eft\circ\ebd}}
  &
\stesysc
  \ar{d}{p\circ\eft\circ\ebd}
  \\
  {}&
Y \ar{r}[swap]{g}
  &
X
\end{tikzcd}
\end{equation*}
commutative.
\end{itemize}
\end{defn}

\begin{thm}\label{cobesys-pullback}
For every diagram
\begin{equation*}
\begin{tikzcd}[column sep=large]
\stesys'
  \ar[bend right=10]{ddr}[swap]{p'}
  \ar[bend left=10]{rrd}{f}
  \ar[dotted]{dr}[near end]{[p',f]}
  \\
  {}&
\cobesys{Y}{\stesys}{g}{p}
  \ar{d}{\pullbackpr{1}{g}{p}}
  \ar{r}[swap]{\pullbackpr{2}{g}{p}}
  &
\stesys
  \ar{d}{p}
  \\
  {}&
Y \ar{r}[swap]{g}
  &
X
\end{tikzcd}
\end{equation*}
of which the outer square commutes, the pre-extension homomorphism $[p',f]$
is unique with the property that it renders the whole diagram commutative.
\end{thm}

\begin{defn}\label{famfamstesys_into}
Consider a commutative square
\begin{equation*}
\begin{tikzcd}
\stesys'
  \ar{r}{f}
  \ar{d}[swap]{p}
  &
\famesys\stesys
  \ar{d}{\eft}
  \\
\stesysf \ar{r}[swap]{\ectxext}
  &
\stesysc
\end{tikzcd}
\end{equation*}
Then we construct
\begin{equation*}
|[p,f]|:\stesys'\to\famesys{\famesys{\stesys}}
\end{equation*}
as follows:
\begin{itemize}
\item let $|[p,f]|_0:\stesysc'\to\stesysf_2$ be the unique morphism rendering
the diagram
\begin{equation*}
\begin{tikzcd}[column sep=large]
\stesysc' 
  \ar[bend left=10]{rrd}{f_0}
  \ar[swap,bend right=10]{ddr}{p}
  \ar[dotted]{dr}[near end]{|[p,f]|_0}
  \\
  {}&
\stesysf_2
  \ar{r}[swap]{\pullbackpr{2}{\ectxext}{\eft}}
  \ar{d}{\eft[1]}
  &
\stesysf
  \ar{d}{\eft}
  \\
  {}&
\stesysf
  \ar{r}[swap]{\ectxext}
  &
\stesysc
\end{tikzcd}
\end{equation*}
commutative.
\item Let $|[p,f]|_1:\stesysf'\to\stesysf_3$ be the unique morphism rendering
the diagram
\begin{equation*}
\begin{tikzcd}[column sep=large]
\stesysf'
  \ar[bend left=10]{drr}{f_1}
  \ar[swap]{dd}{\eft'}
  \ar[dotted]{dr}[near end]{|[p,f]|_1}
  \\
  {}&
\stesysf_3
  \ar{r}[swap]{\pullbackpr{2}{\efamext}{\eft[1]}}
  \ar{d}{\eft[1]}
  &
\stesysf_2
  \ar{d}{\eft[1]}
  \\
\stesysc'
  \ar{r}[swap]{|[p,f]|_0}
  &
\stesysf_2
  \ar{r}[swap]{\efamext}
  &
\stesysf
\end{tikzcd}
\end{equation*}
commutative.
\item Let $|[p,f]|^t:\stesyst'\to\stesyst_3$ be the unique morphism rendering
the diagram
\begin{equation*}
\begin{tikzcd}[column sep=huge]
\stesyst'
  \ar[bend left=10]{drr}{f^t}
  \ar[swap]{dd}{\eft'\circ\ebd'}
  \ar[dotted]{dr}[near end]{|[p,f]|^t}
  \\
  {}&
\stesyst_3
  \ar{r}[swap]{\pullbackpr{2}{\efamext}{\eft[1]\circ\ebd[1]}}
  \ar{d}[swap]{\pullbackpr{1}{\efamext}{\eft[1]\circ\ebd[1]}}
  &
\stesyst_2
  \ar{d}{\eft[1]\circ\ebd[1]}
  \\
\stesysc'
  \ar{r}[swap]{|[p,f]|_0}
  &
\stesysf_2
  \ar{r}[swap]{\efamext}
  &
\stesysf
\end{tikzcd}
\end{equation*}
commutative.
\end{itemize}
\end{defn}

\begin{lem}
Under the hypotheses of \autoref{famfamstesys_into}, $|[p,f]|$ is a pre-extension
homomorphism. Moreover, it is the unique pre-extension homomorphism for which
the diagram
\begin{equation*}
\begin{tikzcd}[column sep=large]
\stesys' 
  \ar[bend left=10]{rrd}{f}
  \ar[swap,bend right=10]{ddr}{p}
  \ar[dotted]{dr}[near end]{|[p,f]|}
  \\
  {}&
\famesys{\famesys{\stesys}}
  \ar{r}[swap]{\pullbackpr{2}{\ectxext}{\eft}}
  \ar{d}{\eft[1]}
  &
\famesys{\stesys}
  \ar{d}{\eft}
  \\
  {}&
\stesysf
  \ar{r}[swap]{\ectxext}
  &
\stesysc
\end{tikzcd}
\end{equation*}
commutes.
\end{lem}

\begin{lem}
Suppose $f:\stesys\to \stesys'$ is a pre-extension homomorphism and consider a morphism
$p:\stesys'\to X$ and $g:Y\to X$. Then the change of base 
$g^\ast(f):\cobesys{Y}{\stesys}{g}{p\circ f_0}\to
\cobesys{Y}{\stesys'}{g}{p}$ is a pre-extension morphism.
\end{lem}

\begin{lem}
Let $\stesys$ be a pre-extension algebra and consider $p:\stesysc\rightarrow X\leftarrow Y:g$.
Then there is an isomorphism
\begin{equation*}
\varphi:\famesys{\cobesys{Y}{\stesys}{g}{p}}
  \simeq
\cobesys{Y}{\famesys{\stesys}}{g}{p\circ\eft}
\end{equation*}
uniquely determined by
\end{lem}

\begin{proof}
This follows from the pasting lemma for pullbacks.
\end{proof}

\subsection{Weakening objects}
\begin{defn}
Let $\stesys$ be an extension object in $\cat{C}$. A pre-weakening operation
on $\stesys$ is an extension homomorphism 
$ \mathbf{w}(\stesys)
    :
  \cobesys{\stesysf}{\famesys{\stesys}}{\eft}{\eft}
    \to
  \famesys{\famesys{\stesys}}$
for which the diagram
\begin{equation*}
\begin{tikzcd}[column sep=large]
\cobesys{\stesysf}{\famesys{\stesys}}{\eft}{\eft}
  \ar{r}{\mathbf{w}(\stesys)}
  \ar{dr}[swap]{\pullbackpr{1}{\eft}{\eft}}
  &
\famesys{\famesys{\stesys}}
  \ar{d}{\eft[1]}
  \\
& \stesysf
\end{tikzcd}
\end{equation*}
commutes.
\end{defn}

\begin{defn}
Let $\stesys$ be an extension object with pre-weakening operation
$\mathbf{w}(\stesys)$. Then $\famesys{\stesys}$ has the pre-weakening operation
$\mathbf{w}(\famesys{\stesys})$ which is uniquely determined by rendering the
diagram
\begin{equation*}
\begin{tikzcd}[column sep=large]
\cobesys{\stesysf_2}{\famesys{\famesys{\stesys}}}{\eft[1]}{\eft[1]}
  \ar{rr}{%
      \pullback{\beta_1}{\boldsymbol{\beta}}{\eft}{\eft})
    }
  \ar[bend right]{ddr}[swap]{\pullbackpr{1}{\eft[1]}{\eft[1]}}
  \ar[dotted]{dr}{\mathbf{w}(\famesys{\stesys})}
  &
  {}&
\cobesys{\stesysf}{\famesys{\stesys}}{\eft}{\eft}
  \ar{r}{\mathbf{w}(\stesys)}
  &
\famesys{\famesys{\stesys}}
  \ar{d}{\boldsymbol{\beta}}
  \\
  {}&
\famesys{\famesys{\famesys{\stesys}}}
  \ar{r}{\boldsymbol{\beta}_\mathbf{2}}
  \ar{d}[swap]{\eft[2]}
  &
\famesys{\famesys{\stesys}}
  \ar{d}{\eft[1]}
  \ar{r}{\boldsymbol{\beta}}
  &
\famesys{\stesys}
  \ar{d}{\eft}
  \\
  {}&
\stesysf_2
  \ar{r}[swap]{\efamext}
  &
\stesysf
  \ar{r}[swap]{\ectxext}
  &
\stesysc
\end{tikzcd}
\end{equation*}
commutative.
\end{defn}

\begin{defn}
A pre-weakening object $\stesys$ in $\cat{C}$ is an extension object $\stesys$ 
with a pre-weakening operation 
$ \mathbf{w}(\stesys)
    :
  \cobesys{\stesysf}{\famesys{\stesys}}{\eft}{\eft}
    \to
  \famesys{\famesys{\stesys}}$
for which the diagram
\begin{equation*}
\begin{tikzcd}[column sep=15em]
\cobesys{\stesysf_2}{\famesys{\stesys}}{\eft\circ\eft[1]}{\eft}
  \ar[bend right=10]{dr}[swap]%
    { [ \pullbackpr{1}{\eft\circ\eft[1]}{\eft},%
        \mathbf{w}(\stesys)%
          \circ%
        (\pullback{\efamext}{\catid{\famesys{\stesys}}}{\eft}{\eft})%
        ]%
      }
  \ar{r}{
    [ \pullbackpr{1}{\eft\circ\eft[1]}{\eft},%
      \mathbf{w}(\stesys)%
        \circ%
      (\pullback{\eft[1]}{\catid{\famesys{\stesys}}}{\eft}{\eft})%
      ]}%
  &
\cobesys{\stesysf_2}{\famesys{\famesys{\stesys}}}{\eft[1]}{\eft[1]}
  \ar{d}{\mathbf{w}(\famesys{\stesys})}
  \\
  {}&
\famesys{\famesys{\famesys{\stesys}}}
\end{tikzcd}
\end{equation*}
commutes. This condition is called \emph{Currying for weakening}.
\end{defn}

\begin{lem}
If $\stesys$ is a pre-weakening algebra, then so is $\famesys{\stesys}$. 
\end{lem}

\begin{proof}
We only have to verify that $\mathbf{w}(\famesys{\stesys})$ is a pre-extension
homomorphism, the requirement on weakenings is satisfied by construction.
The bottom square of the diagram
\begin{equation*}
\begin{tikzcd}[column sep=large]
\pullback{\stesysf_2}{\stesyst_3}{\eft[1]}{\eft[1]\circ\eft[2]\circ\ebd[2]}
  \ar{r}{w(\famesys{\stesys})^t}
  \ar{d}[swap]{\eft[1]^\ast(\ebd[2])}
  &
\stesyst_4
  \ar{d}{\ebd[3]}
  \\
\pullback{\stesysf_2}{\stesysf_3}{\eft[1]}{\eft[1]\circ\eft[2]}
  \ar{r}{w(\famesys{\stesys})_1}
  \ar{d}[swap]{\eft[1]^\ast(\eft[2])}
  &
\stesysf_4
  \ar{d}{\eft[3]}
  \\
\pullback{\stesysf_2}{\stesysf_2}{\eft[1]}{\eft[1]}
  \ar{r}[swap]{w(\famesys{\stesys})_0}
  &
\stesysf_3
\end{tikzcd}
\end{equation*}
commutes by construction. To show that the top square commtues, post-compose
with the pullback projections.
\end{proof}

\begin{defn}
Let $\stesys$ be a pre-weakening algebra and consider $p:\stesysc\rightarrow X\leftarrow Y:p$.
Then we define
\begin{equation*}
\mathbf{w}(\cobesys{Y}{\stesys}{g}{p}):
  \cobesys{(\pullback{Y}{\stesysf}{g}{p\circ\eft})}{\famesys{\cobesys{Y}{\stesys}{g}{p}}}{g^\ast(\eft)}{g^\ast(\eft)}\to\famesys{\famesys{\cobesys{Y}{\stesys}{g}{p}}}
\end{equation*}
\begin{itemize}
\item by defining $w(\cobesys{Y}{\stesys}{g}{p})_0$ to be the unique morphism
rendering the diagram
\begin{equation*}
\begin{tikzcd}[column sep=large]
\pullback
  {\pullback{Y}{\stesysf}{g}{p\circ\eft}}
  {\pullback{Y}{\stesysf}{g}{p\circ\eft}}
  {g^\ast(\eft)}
  {g^\ast(\eft)}
  \ar{rr}
  \ar{ddd}[swap]{\pullbackpr{1}{g^\ast(\eft)}{g^\ast(\eft)}}
  &
  {}&
\pullback
  {Y}
  {\pullback{\stesysf}{\stesysf}{\eft}{\eft}}
  {g}
  {p\circ\eft\circ\pullbackpr{1}{\eft}{\eft}}
  \ar{d}{\pullback{\catid{Y}}{w(\stesys)_0}{g}{p\circ\eft}}
  \\
  {}&
  {}&
\pullback{Y}{\stesysf_2}{g}{p\circ\eft\circ\eft[1]}
  \ar{d}{\pullbackpr{2}{\ectxext}{\eft}}
  \\
  {}&
(\pullback{Y}{\stesysf}{g}{p\circ\eft})_2
  \ar{r}{\pullbackpr{2}{\cobesys{Y}{\ectxext}{g}{p}}{g^\ast(\eft)}}
  \ar{d}[swap]{\pullbackpr{1}{\cobesys{Y}{\ectxext}{g}{p}}{g^\ast(\eft)}}
  &
\pullback{Y}{\stesysf}{g}{p\circ\eft}
  \ar{d}{g^\ast(\eft)}
  \\
\pullback{Y}{\stesysf}{g}{p\circ\eft}
  \ar[equals]{r}
  &
\pullback{Y}{\stesysf}{g}{p\circ\eft}
  \ar{r}[swap]{\cobesys{Y}{\ectxext}{g}{p}}
  &
\pullback{Y}{\stesysc}{g}{p}
\end{tikzcd}
\end{equation*}
commutative.
\end{itemize}
\end{defn}

\begin{defn}
A pre-weakening morphism between preweakening objects $\stesys$ and $\stesys'$ is an
extension homomorphism $f:\stesys\to \stesys'$ such that additionally the diagram
\begin{equation*}
\begin{tikzcd}[column sep=large]
\cobesys{\stesysf}{\famesys{\stesys}}{\eft}{\eft}
  \ar{d}[swap]{\mathbf{w}(\stesys)}
  \ar{r}{\pullback{f_1}{\famehom{f}}{\eft'}{\eft'}}
  &
\cobesys{\stesysf'}{\famesys{\stesys'}}{\eft'}{\eft'}
  \ar{d}{\mathbf{w}(\stesys')}
  \\
\famesys{\famesys{\stesys}}
  \ar{r}[swap]{\famehom{\famehom{f}}}
  &
\famesys{\famesys{\stesys'}}
\end{tikzcd}
\end{equation*}
commutes.
\end{defn}

\begin{defn}
A weakening object is a pre-weakening object $\stesys$ with the property that
$\mathbf{w}(\stesys)$ is a pre-weakening morphism,
\end{defn}

\subsection{Projection objects}
\begin{defn}
A pre-projection object is a weakening object $\stesys$ for which there is a term
$\mathbf{i}:\stesysf\to \stesyst_2$ such that the diagram
\begin{equation*}
\begin{tikzcd}[column sep=large]
\stesysf \ar{r}{\mathbf{i}} \ar{d}[swap]{\Delta_{\eft}} & \stesyst_2 \ar{d}{\ebd[1]}\\
\pullback{\stesysf}{\stesysf}{\eft}{\eft} \ar{r}[swap]{w(\stesys)_0} & \stesysf_2
\end{tikzcd}
\end{equation*}
commutes. In this diagram $\Delta_{\eft}:\stesysf\to \pullback{\stesysf}{\stesysf}{\eft}{\eft}$ is the diagonal.
\end{defn}

\begin{defn}
A pre-projection homomorphism from $\stesys$ to $\stesys'$ is a weakening homomorphism
$f:\stesys\to \stesys'$ such that the square
\begin{equation*}
\begin{tikzcd}[column sep=large]
\stesyst_2
  \ar{r}{f^t_1}
  &
\stesyst_2'
  \\
\stesysf \ar{r}[swap]{f_1}
  \ar{u}{\mathbf{i}}
  &
\stesysf'
  \ar{u}[swap]{\mathbf{i}'}
\end{tikzcd}
\end{equation*}
commutes
\end{defn}

\begin{lem}
The change of base of a pre-projection object is again a pre-projection object.
\end{lem}

\begin{lem}
If $CFT$ is a pre-projection object, then so is $\mathbf{F}_{CFT}$, where
$\mathbf{F}_{\mathbf{i}}$ is defined to be $F\times_{e_0,c}\mathbf{i}$ is
a pre-projection algebra.
\end{lem}

\begin{defn}
A projection algebra is a pre-projection algebra for which weakening is a
pre-projection homomorphism.
\end{defn}

\begin{cor}
The change of base of a projection object is again a projection object.
\end{cor}

\begin{cor}
If $CFT$ is a projection object, then so is $\mathbf{F}_{CFT}$, where
$\mathbf{F}_{\mathbf{i}}$ is defined to be $F\times_{e_0,c}\mathbf{i}$ is
a projection algebra.
\end{cor}

\subsection{Substitution objects}

\begin{defn}
A pre-substitution object is an extension object for which the is an
extension homomorphism
\begin{equation*}
\mathbf{s}(\stesys):\cobesys{\stesyst}{\famesys{\famesys{\stesys}}}{\ebd}{\eft[1]}\to \famesys{\stesys}
\end{equation*}
such that the square
\begin{equation*}
\begin{tikzcd}[column sep=large]
\pullback{\stesyst}{\stesysf_2}{\ebd}{\eft[1]}
  \ar{r}{s(\stesys)_0}
  \ar{d}[swap]{\ebd\circ\pullbackpr{1}{\ebd}{\eft[1]}}
  &
\stesysf 
  \ar{d}{\eft}
  \\
\stesysf 
  \ar{r}[swap]{\eft}
  &
\stesysc
\end{tikzcd}
\end{equation*}
commutes.
\end{defn}

\begin{defn}
A pre-substitution homomorphism is an extension homomorphism $f:CFT\to CFT'$
such that the square
\begin{equation*}
\begin{tikzcd}
T\times\mathbf{F}_{\mathbf{F}_{CFT}}
  \ar{r}{f^t\times\mathbf{F}_{\mathbf{F}_f}}
  \ar{d}[swap]{\mathbf{s}}
  &
T'\times\mathbf{F}_{\mathbf{F}_{CFT'}}
  \ar{d}{\mathbf{s}'}
  \\
\mathbf{F}_{CFT}
  \ar{r}[swap]{\mathbf{F}_f}
  &
\mathbf{F}_{CFT'}
\end{tikzcd}
\end{equation*}
\end{defn}

\begin{lem}
The change of base of a pre-substitution object is again a pre-substitution object.
\end{lem}

\begin{lem}
If $CFT$ is a pre-substitution object, then so is $\mathbf{F}_{CFT}$ with
$\mathbf{F}_{\mathbf{s}}=...$.
\end{lem}

\begin{defn}
A substitution object is a pre-substitution object for which substitution is
a pre-substitution homomorphism.
\end{defn}

\begin{cor}
The change of base of a substitution object is again a substitution object.
\end{cor}

\begin{cor}
If $CFT$ is a substitution object, then so is $\mathbf{F}_{CFT}$.
\end{cor}

\subsection{E-objects}
\begin{defn}
An E-object is an extension object with the structure of a projection object,
the structure of a substitution object and which has an empty context and families,
such that additionally:
\begin{enumerate}
\item substitution is a projection homomorphism
\item weakening is a substitution homomorphism
\item both weakening and substitution are empty-CF homomorphisms.
\item 
\end{enumerate}
\end{defn}


\bibliographystyle{plain}
%\phantomsection\addcontentsline{toc}{section}{References}
\bibliography{refs}

\end{document}
