\documentclass{article}

%%%%%%%%%%%%%%%%%%%%%%%%%%%%%%%%%%%%%%%%%%%%%%%%%%%%%%%%%%%%%%%%%%%%%%%%%%%%%%%%
%%%% PACKAGES

\usepackage[utf8]{inputenc}
\usepackage[english]{babel}

%%%% Spicing up the document
\usepackage{mathpazo}
\usepackage[scaled=0.95]{helvet}
\usepackage{courier}
\linespread{1.05} % Palatino looks better with this
\usepackage{microtype}

\usepackage{fancyhdr} % To set headers and footers
\usepackage{enumitem,mathtools,xspace,xcolor}
\usepackage{comment}
\usepackage{ifthen}
\usepackage{pifont}
\newcommand{\cmark}{\ding{51}\xspace}
\newcommand{\xmark}{\ding{55}\xspace}

\usepackage{graphicx}
\usepackage{tikz-cd}
\usepackage{tikz}
\usetikzlibrary{decorations.pathmorphing}
\usepackage[inference]{semantic}
\usepackage{booktabs}

\usepackage[hyphens]{url} % This package has to be loaded *before* hyperref
\usepackage[pagebackref,colorlinks,citecolor=darkgreen,linkcolor=darkgreen,unicode]{hyperref}
\definecolor{darkgreen}{rgb}{0,0.45,0}

% For some reason the following can't be above hyperref...
\usepackage{amssymb,amsmath,amsthm,stmaryrd,mathrsfs,wasysym}
\usepackage{aliascnt}
\usepackage[capitalize]{cleveref}

% The braket macro shouldn't be necessary
\usepackage{braket} % used for \setof{ ... } macro

%%%%%%%%%%%%%%%%%%%%%%%%%%%%%%%%%%%%%%%%%%%%%%%%%%%%%%%%%%%%%%%%%%%%%%%%%%%%%%%%
%% To include references in TOC we should use this package rather than a hack.
\usepackage{tocbibind}
%\usepackage{etoolbox}           % get \apptocmd
%\apptocmd{\thebibliography}{\addcontentsline{toc}{section}{References}}{}{} % tell bibliography to get itself into the table of contents


\begin{comment}
%%%% Header and footers
\pagestyle{fancyplain}
\setlength{\headheight}{15pt}
\renewcommand{\chaptermark}[1]{\markboth{\textsc{Chapter \thechapter. #1}}{}}
\renewcommand{\sectionmark}[1]{\markright{\textsc{\thesection\ #1}}}
\end{comment}

% TOC depth
\setcounter{tocdepth}{3}

\lhead[\fancyplain{}{{\thepage}}]%
      {\fancyplain{}{\nouppercase{\rightmark}}}
\rhead[\fancyplain{}{\nouppercase{\leftmark}}]%
      {\fancyplain{}{\thepage}}
\cfoot{\textsc{\footnotesize [Draft of \today]}}
\lfoot[]{}
\rfoot[]{}

%%%%%%%%%%%%%%%%%%%%%%%%%%%%%%%%%%%%%%%%%%%%%%%%%%%%%%%%%%%%%%%%%%%%%%%%%%%%%%%%
%%%% We mostly use the macros of the book, to keep notations
%%%% and conventions the same. Recall that when the macros file
%%%% is updated, we need to comment the lines containing the
%%%% string `[chapter]` since our article is not a book.
%%%%
%%%% Instructions for updating the macros.tex file:
%%%% - fetch the latest macros.tex file from the HoTT/book git repository.
%%%% - comment all lines containing "[chapter]" because this is not a book.
%%%% - comment the definition of pbcorner because the xypic package is not used.
%%%%
\input{macros}

\newcommand{\idsymbin}{=}

%%%%%%%%%%%%%%%%%%%%%%%%%%%%%%%%%%%%%%%%%%%%%%%%%%%%%%%%%%%%%%%%%%%%%%%%%%%%%%%%
%%%% Our commands which are not part of the macros.tex file.
%%%% We should keep these commands separate, because we will
%%%% update the macros.tex following the updates of the book.

%%%% First we redefine the \id, \eqv and \ct commands so that they accept an
%%%% arbitrary number of arguments. This is useful when writing longer strings
%%%% of equalities or equivalences.

\makeatletter

\renewcommand{\id}[3][]{
  \@ifnextchar\bgroup
    {#2 \mathbin{\idsym_{#1}} \id[#1]{#3}}
    {#2 \mathbin{\idsym_{#1}} #3}
  }

\renewcommand{\eqv}[2]{
  \@ifnextchar\bgroup
    {#1 \eqvsym \eqv{#2}}
    {#1 \eqvsym #2}
  }

\newcommand{\ctsym}{%
  \mathchoice{\mathbin{\raisebox{0.5ex}{$\displaystyle\centerdot$}}}%
             {\mathbin{\raisebox{0.5ex}{$\centerdot$}}}%
             {\mathbin{\raisebox{0.25ex}{$\scriptstyle\,\centerdot\,$}}}%
             {\mathbin{\raisebox{0.1ex}{$\scriptscriptstyle\,\centerdot\,$}}}
  }

\renewcommand{\ct}[3][]{
  \@ifnextchar\bgroup
    {#2 \mathbin{\ctsym_{#1}} \ct[#1]{#3}}
    {#2 \mathbin{\ctsym_{#1}} #3}
  }

\makeatother

%%%% We always use textstyle products and sums...
%\renewcommand{\prd}{\tprd}
%\renewcommand{\sm}{\tsm}
\makeatletter
\renewcommand{\@dprd}{\@tprd}
\renewcommand{\@dsm}{\@tsm}
\renewcommand{\@dprd@noparens}{\@tprd}
\renewcommand{\@dsm@noparens}{\@tsm}

%%%% ...with a bit more spacing
\renewcommand{\@tprd}[1]{\mathchoice{{\textstyle\prod_{(#1)}\,}}{\prod_{(#1)}\,}{\prod_{(#1)}\,}{\prod_{(#1)}\,}}
\renewcommand{\@tsm}[1]{\mathchoice{{\textstyle\sum_{(#1)}\,}}{\sum_{(#1)}\,}{\sum_{(#1)}\,}{\sum_{(#1)}\,}}

%%%%%%%%%%%%%%%%%%%%%%%%%%%%%%%%%%%%%%%%%%%%%%%%%%%%%%%%%%%%%%%%%%%%%%%%%%%%%%%%
%%%% We adjust the \prd command so that implicit arguments become possible.
%%%%
%%%% First, we have the following switch. Set it to true if implicit arguments
%%%% are desired, or to false if not. Note turning off implicit arguments
%%%% might render some parts of the text harder to comprehend, since in the
%%%% text might appear $f(x)$ where we would have $f(i,x)$ without implicit
%%%% arguments.

\newcommand{\implicitargumentson}{\boolean{true}}

%%%% If one wants to use implicit arguments in the notation for product types,
%%%% a * has to be put before the argument that has to be implicit.
%%%% For example: in $\prd{x:A}*{y:B}{u:P(y)}Q(x,y,u)$, the argument y is
%%%% implicit. Any of the arguments can be made implicit this way.

%%%% First of all, we should make the command \prd search not only for a
%%%% brace, but also for a star. We introduce an auxiliary command that
%%%% determines whether the next character is a star or brace.
\newcommand{\@ifnextchar@starorbrace}[2]
%  {\@ifnextcharamong{#1}{#2}{*}{\bgroup};}
  {\@ifnextchar*{#1}{\@ifnextchar\bgroup{#1}{#2}}}
  
%%%% When encountering the \prd command, latex should determine whether it
%%%% should print implicit argument brackets or not. So the first branching
%%%% happens right here.
\renewcommand{\prd}{\@ifnextchar*{\@iprd}{\@prd}}

\newcommand{\@prd}[1]
  {\@ifnextchar@starorbrace
    {\prd@parens{#1}}
    {\@ifnextchar\sm{\prd@parens{#1}\@eatsm}{\prd@noparens{#1}}}}
\newcommand{\@prd@parens}{\@ifnextchar*{\@iprd}{\prd@parens}}
\renewcommand{\prd@parens}[1]
  {\@ifnextchar@starorbrace
    {\@theprd{#1}\@prd@parens}
    {\@ifnextchar\sm{\@theprd{#1}\@eatsm}{\@theprd{#1}}}}
\newcommand{\@theprd}[1]
  {\mathchoice{\@dprd{#1}}{\@tprd{#1}}{\@tprd{#1}}{\@tprd{#1}}}
\renewcommand{\dprd}[1]{\@dprd{#1}\@ifnextchar@starorbrace{\dprd}{}}
\renewcommand{\tprd}[1]{\@tprd{#1}\@ifnextchar@starorbrace{\tprd}{}}

%%%% Here we tell the actual symbols to be printed.
\newcommand{\@theiprd}[1]{\mathchoice{\@diprd{#1}}{\@tiprd{#1}}{\@tiprd{#1}}{\@tiprd{#1}}}
\newcommand{\@iprd}[2]{\@ifnextchar@starorbrace%
  {\@theiprd{#2}\@prd@parens}%
  {\@ifnextchar\sm%
    {\@theiprd{#2}\@eatsm}%
    {\@theiprd{#2}}}}
\def\@tiprd#1{
  \ifthenelse{\implicitargumentson}
    {\@@tiprd{#1}\@ifnextchar\bgroup{\@tiprd}{}}
    {\@tprd{#1}}}
\def\@@tiprd#1{\mathchoice{{\textstyle\prod_{\{#1\}}\,}}{\prod_{\{#1\}}\,}{\prod_{\{#1\}}\,}{\prod_{\{#1\}}\,}}
\def\@diprd{
  \ifthenelse{\implicitargumentson}
    {\@tiprd}
    {\@tprd}}
    

%%%% And finally we need to redefine \@eatprd so that implicit arguments also
%%%% works in the scope of a dependent sum.    
\def\@eatprd\prd{\@prd@parens}

\makeatother

%%%%%%%%%%%%%%%%%%%%%%%%%%%%%%%%%%%%%%%%%%%%%%%%%%%%%%%%%%%%%%%%%%%%%%%%%%%%%%%%
%%%% Redefining the quantifiers, so that some of the longer 
%%%% formulas appear one a single line without problems

%%% Dependent products written with \forall, in the same style
\makeatletter
\def\tfall#1{\forall_{(#1)}\@ifnextchar\bgroup{\,\tfall}{\,}}
\renewcommand{\fall}{\tfall}

%%% Existential quantifier %%%
\def\texis#1{\exists_{(#1)}\@ifnextchar\bgroup{\,\texis}{\,}}
\renewcommand{\exis}{\texis}

%%% Unique existence %%%
\def\uexis#1{\exists!_{(#1)}\@ifnextchar\bgroup{\,\uexis}{\,}}
\makeatother

%%%%%%%%%%%%%%%%%%%%%%%%%%%%%%%%%%%%%%%%%%%%%%%%%%%%%%%%%%%%%%%%%%%%%%%%%%%%%%%%
%%%% UNFOLD
%%%%
%%%% For each definition in the type theory we make two versions of the macro:
%%%% the macro introducing the new notation and an @unfold version of the macro
%%%% which outputs the meaning of that new notation. Thus, we can use the
%%%% following construction to write our text. When we introduce \macro, we can
%%%% write \unfold{\macro} and the output will be the result of \macro@unfold.

\makeatletter
\newcommand{\unfold}{%
  \unfoldnext}
\newcommand{\unfoldall}[1]{%
  \begingroup%
  \renewcommand{\jhom}{\jhom@unfold}%
  \renewcommand{\jhomeq}{\jhomeq@unfold}%
  \renewcommand{\jhomdefn}{\jhomdefn@unfold}%
  \renewcommand{\jfhom}{\jfhom@unfold}%
  \renewcommand{\jcomp}{\jcomp@unfold}%
  \renewcommand{\@jcomp@nested}{\@jcomp@unfold@nested}%
  \renewcommand{\@jcomp@parens}{\@jcomp@unfold@parens}%
  \renewcommand{\tmext}{\tmext@unfold}%
  \renewcommand{\@tmext@nested}{\@tmext@unfold@nested}%
  \renewcommand{\@tmext@parens}{\@tmext@unfold@parens}%
  \renewcommand{\cprojfstf}{\cprojfstf@unfold}%
  \renewcommand{\cprojfst}{\cprojfst@unfold}%
  \renewcommand{\cprojsndf}{\cprojsndf@unfold}%
  \renewcommand{\cprojsnd}{\cprojsnd@unfold}%
  \renewcommand{\jfcomp}{\jfcomp@unfold}%
%  \renewcommand{\@jfcomp@nested}{\@jfcomp@unfold@nested}%
%  \renewcommand{\@jfcomp@parens}{\@jfcomp@unfold@parens}%
  \renewcommand{\sandwich}{\sandwich@unfold}%
  \renewcommand{\finc}{\finc@unfold}%
  \renewcommand{\jvcomp}{\jvcomp@unfold}%
  \renewcommand{\subst@type@unfold}[1]{
    \@ifnextchar\cprojfstf{\@eatdo{\cprojfstf@parens}}{%
      ##1}
    }
  #1%
  \endgroup%
  }

%%%% The following command is useful when you have checked with '\@ifnextchar'
%%%% that the next character is a macro '\firstmacro' and you want to replace
%%%% it by '\secondmacro'. To establish this, simply call for
%%%% '\@ifnextchar\firstmacro{\@eatdo{\secondmacro}}{}' with the second 
%%%% argument of \@eatdo left unspecified.
\newcommand{\@eatdo}[2]{#1}

%%%% The intention of '\unfoldnext' is to unfold only the definition of the
%%%% next character, provided that it is in the list of unfoldable macros.
\newcommand{\unfoldnext}[1]{
  \@ifnextchar\jhom{\@eatdo{\jhom@unfold}}{%
  \@ifnextchar\jhomeq{\@eatdo{\jhomeq@unfold}}{%
  \@ifnextchar\jhomdefn{\@eatdo{\jhomdefn@unfold}}{%
  \@ifnextchar\jfhom{\@eatdo{\jfhom@unfold}}{%
  \@ifnextchar\jcomp{\@eatdo{\jcomp@unfold}}{%
  \@ifnextchar\@jcomp@nested{\@eatdo{\@jcomp@unfold@nested}}{%
  \@ifnextchar\@jcomp@parens{\@eatdo{\@jcomp@unfold@parens}}{%
  \@ifnextchar\tmext{\@eatdo{\tmext@unfold}}{%
  \@ifnextchar\@tmext@nested{\@eatdo{\@tmext@unfold@nested}}{%
  \@ifnextchar\@tmext@parens{\@eatdo{\@tmext@unfold@parens}}{%
  \@ifnextchar\cprojfstf{\@eatdo{\cprojfstf@unfold}}{%
  \@ifnextchar\cprojfst{\@eatdo{\cprojfst@unfold}}{%
  \@ifnextchar\cprojsndf{\@eatdo{\cprojsndf@unfold}}{%
  \@ifnextchar\cprojsnd{\@eatdo{\cprojsnd@unfold}}{%
  \@ifnextchar\jfcomp{\@eatdo{\jfcomp@unfold}}{%
%  \@ifnextchar\@jfcomp@nested{\@eatdo{\@jfcomp@unfold@nested}}{%
%  \@ifnextchar\@jfcomp@parens{\@eatdo{\@jfcomp@unfold@parens}}{%
  \@ifnextchar\sandwich{\@eatdo{\sandwich@unfold}}{%
  \@ifnextchar\finc{\@eatdo{\finc@unfold}}{%
  \@ifnextchar\jvcomp{\@eatdo{\jvcomp@unfold}}}
  #1}
\makeatother

%%%%%%%%%%%%%%%%%%%%%%%%%%%%%%%%%%%%%%%%%%%%%%%%%%%%%%%%%%%%%%%%%%%%%%%%%%%%%%%%
%%%% A PRETTY PRINTER
%%%%
%%%% We write a \pretty command that pretty prints judgments or types by
%%%% diplaying variables and omitting explicit notation for weakening.
%%%%
%%%% This command should work similar to the \unfold command
%%%%
%%%% -- UNDER CONSTRUCTION

\makeatletter
\newcommand{\vardis}[2]{\@vardis@type #2{}(\@vardis@term #1)}
\newcommand{\@vardis}{\@ifnextchar\bgroup{\@@vardis}{}}
\newcommand{\@@vardis}[1]{\@ifnextchar\bgroup{\vardis{#1}}{#1}}
\newcommand{\@vardis@term}{\@vardis}
\newcommand{\@vardis@type}{\@ifnextchar\ctxext{\@ctxext@nested}{\@ifnextchar\ctxwk{\@ctxwk@nested}{\@vardis}}}
\newcommand{\@vardis@nested}[3]{\@vardis@parens{#2}{#3}}
\newcommand{\@vardis@parens}[2]{(\vardis{#1}{#2})}
\makeatother

\makeatletter
\newcommand{\jvctx}{\jctx}
\newcommand{\jvctxeq}{\jctxeq}

\newcommand{\cctxextcombi}[2]{\@ifnextchar\bgroup{\@cctxextcombi #1}{#1:}#2}
\newcommand{\@cctxextcombi}[4]{\cctxext{{\cctxextcombi{#1}{#3}}{\@@cctxextcombi{#1}{#2}{#4}}}}
\newcommand{\@@cctxextcombi}[3]{\@ifnextchar\bgroup{\@@@ctxextcombi #2}{#2(#1):}#3(\cctxext{#1})}
\newcommand{\@@@ctxextcombi}[8] % the 5th argument is (, the 6th is \cctxext and the 8th is ).
  {\@@ctxextcombi{#7}{#1}{#3},\@@ctxextcombi{{#7}{#3}}{#2}{#4}}
\newcommand{\cctxext}[1]{\@ifnextchar\bgroup{\@cctxext}{}#1}
\newcommand{\@cctxext}[2]{\cctxext{#1},\cctxext{#2}}

\newcommand{\jvfamcombi}[3]{
  \cctxextcombi{#1}{#2} \vdash \vardis{\cctxext{#1}}{#3}
}

\newcommand{\jvfam}{\@ifnextchar*{\@jvfamAlignTrue}{\@jvfamAlignFalse}}
\newcommand{\@jvfamAlignTrue}[4]{\jfam*{#2:#3}{\vardis{#2}{#4}}}
\newcommand{\@jvfamAlignFalse}[3]{\jfam{#1:#2}{\vardis{#1}{#3}}\quad@test}

\newcommand{\jvfameq}{\@ifnextchar*{\@jvfameqAlignTrue}{\@jvfameqAlignFalse}}
\newcommand{\@jvfameqAlignTrue}[5]{\jfameq*{#2:#3}{\vardis{#2}{#4}}{\vardis{#2}{#5}}}
\newcommand{\@jvfameqAlignFalse}[4]{\jfameq{#1:#2}{\vardis{#1}{#3}}{\vardis{#1}{#4}}\quad@test}

\newcommand{\jvtype}{\@ifnextchar*{\@jvtypeAlignTrue}{\@jvtypeAlignFalse}}
\newcommand{\@jvtypeAlignTrue}[4]{\jtype*{#2:#3}{\vardis{#2}{#4}}}
\newcommand{\@jvtypeAlignFalse}[3]{\jtype{#1:#2}{\vardis{#1}{#3}}\quad@test}

\newcommand{\jvtypeeq}{\@ifnextchar*{\@jvtypeeqAlignTrue}{\@jvtypeeqAlignFalse}}
\newcommand{\@jvtypeeqAlignTrue}[5]{\jtypeeq*{#2:#3}{\vardis{#2}{#4}}{\vardis{#2}{#5}}}
\newcommand{\@jvtypeeqAlignFalse}[4]{\jtypeeq{#1:#2}{\vardis{#1}{#3}}{\vardis{#1}{#4}}\quad@test}

\newcommand{\jvterm}{\@ifnextchar*{\@jvtermAlignTrue}{\@jvtermAlignFalse}}
\newcommand{\@jvtermAlignTrue}[5]{\jterm*{#2:#3}{\vardis{#2}{#4}}{\vardis{#2}{#5}}}
\newcommand{\@jvtermAlignFalse}[4]{\jterm{#1:#2}{\vardis{#1}{#3}}{\vardis{#1}{#4}}\quad@test}

\newcommand{\jvtermeq}{\@ifnextchar*{\@jvtermeqAlignTrue}{\@jvtermeqAlignFalse}}
\newcommand{\@jvtermeqAlignTrue}[6]{\jtermeq*{#2:#3}{\vardis{#2}{#4}}{\vardis{#2}{#5}}{\vardis{#2}{#6}}}
\newcommand{\@jvtermeqAlignFalse}[5]{\jtermeq{#1:#2}{\vardis{#1}{#3}}{\vardis{#1}{#4}}{\vardis{#1}{#5}}\quad@test}
\makeatother

%%%%%%%%%%%%%%%%%%%%%%%%%%%%%%%%%%%%%%%%%%%%%%%%%%%%%%%%%%%%%%%%%%%%%%%%%%%%%%%%
%%%%

\newcommand{\famsym}{\mathcal{F}}
\newcommand{\tmsym}{\mathcal{T}}

%%%%%%%%%%%%%%%%%%%%%%%%%%%%%%%%%%%%%%%%%%%%%%%%%%%%%%%%%%%%%%%%%%%%%%%%%%%%%%%%
%%%% JUDGMENTS
%%%%
%%%% Below we define several commands for the judgments of type theory. There
%%%% are commands
%%%% * \jctx for the judgment that something is a context.
%%%% * \jctxeq for the judgment that two contexts are the same
%%%% * \jtype for the judgment that something is a type in a context
%%%% * \jtypeeq for the judgment that two types in the same context are the same
%%%% * \jterm for the judgment that something is a term of a type in a context
%%%% * \jtermeq for the judgment that two terms of the same type are the same

\makeatletter
% We first make a generic judgment command
\newcommand{\judgment}{\@ifnextchar*{\@judgmentAT}{\@judgmentAF}}
\newcommand{\@judgmentAT}[8]{\@judgment@ctx{#2} & \vdash \@judgment@rel{#3}{#4}{#5}{#6}{#7} #8}
\newcommand{\@judgmentAF}[7]{\@judgment@ctx{#1} \vdash \@judgment@rel{#2}{#3}{#4}{#5}{#6} #7\quad@test}
\newcommand{\@judgment@ctx}{\@judgment@ext}
\newcommand{\@judgment@rel}[5]{
  { \default@ctxext #1
    }
  #2 
  { \default@ctxext #3
    }
  #4
  { \default@ctxext #5
    }}
\newcommand{\@judgment@kind}[1]{~~\textit{#1}}
\newcommand{\@judgment@ext}[1]{\default@ctxext #1}

\newcommand{\quad@test}{%
  \@ifnextchar\jctx{\quad}{%
  \@ifnextchar\jctxeq{\quad}{%
  \@ifnextchar\jvctx{\quad}{%
  \@ifnextchar\jvctxeq{\quad}{%
  \@ifnextchar\jfam{\quad}{%
  \@ifnextchar\jfameq{\quad}{%
  \@ifnextchar\jvfam{\quad}{%
  \@ifnextchar\jvfameq{\quad}{%
  \@ifnextchar\jtype{\quad}{%
  \@ifnextchar\jtypeeq{\quad}{%
  \@ifnextchar\jvtype{\quad}{%
  \@ifnextchar\jvtypeeq{\quad}{%
  \@ifnextchar\jterm{\quad}{%
  \@ifnextchar\jtermeq{\quad}{%
  \@ifnextchar\jvterm{\quad}{%
  \@ifnextchar\jvtermeq{\quad}{%
  \@ifnextchar\jhom{\quad}{%
  \@ifnextchar\jhomeq{\quad}{%
  \@ifnextchar\jfhom{\quad}{%
  \@ifnextchar\jfhomeq{\quad}{%
  }}}}}}}}}}}}}}}}}}}}}

%%%% Judgments about contexts
\newcommand{\jctx@sym}{\@judgment@kind{ctx}}

\newcommand{\jctx}{\@ifnextchar*{\@jctxAlignTrue}{\@jctxAlignFalse}}
\newcommand{\@jctxAlignTrue}[2]{\judgment*{}{}{}{}{}{#2}{\jctx@sym}}
\newcommand{\@jctxAlignFalse}[1]{\judgment{}{}{}{}{}{#1}{\jctx@sym}}

\newcommand{\jctxeq}{\@ifnextchar*{\@jctxeqAlignTrue}{\@jctxeqAlignFalse}}
\newcommand{\@jctxeqAlignTrue}[3]{\judgment*{}{#2}{\jdeq}{#3}{}{}{\jctx@sym}}
\newcommand{\@jctxeqAlignFalse}[2]{\judgment{}{#1}{\jdeq}{#2}{}{}{\jctx@sym}}

\newcommand{\jctxdefn}{\@ifnextchar*{\@jctxdefnAlignTrue}{\@jctxdefnAlignFalse}}
\newcommand{\@jctxdefnAlignTrue}[3]{\judgment*{}{#2}{\defeq}{#3}{}{}{\jctx@sym}}
\newcommand{\@jctxdefnAlignFalse}[2]{\judgment{}{#1}{\defeq}{#2}{}{}{\jctx@sym}}

%%%% Judgments about families
\newcommand{\jfam@sym}{\@judgment@kind{fam}}

\newcommand{\jfam}{\@ifnextchar*{\@jfamAlignTrue}{\@jfamAlignFalse}}
\newcommand{\@jfamAlignTrue}[3]{\judgment*{#2}{}{}{}{}{#3}{\jfam@sym}}
\newcommand{\@jfamAlignFalse}[2]{\judgment{#1}{}{}{}{}{#2}{\jfam@sym}}

\newcommand{\jfameq}{\@ifnextchar*{\@jfameqAlignTrue}{\@jfameqAlignFalse}}
\newcommand{\@jfameqAlignTrue}[4]{\judgment*{#2}{#3}{\jdeq}{#4}{}{}{\jfam@sym}}
\newcommand{\@jfameqAlignFalse}[3]{\judgment{#1}{#2}{\jdeq}{#3}{}{}{\jfam@sym}}

\newcommand{\jfamdefn}{\@ifnextchar*{\@jfamdefnAlignTrue}{\@jfamdefnAlignFalse}}
\newcommand{\@jfamdefnAlignTrue}[4]{\judgment*{#2}{#3}{\defeq}{#4}{}{}{\jfam@sym}}
\newcommand{\@jfamdefnAlignFalse}[3]{\judgment{#1}{#2}{\defeq}{#3}{}{}{\jfam@sym}}
  
%%%% Judgments about types
\newcommand{\jtype@sym}{\@judgment@kind{type}}
\newcommand{\jtype}{\@ifnextchar*{\@jtypeAlignTrue}{\@jtypeAlignFalse}}
\newcommand{\@jtypeAlignTrue}[3]{\judgment*{#2}{}{}{}{}{#3}{\jtype@sym}}
\newcommand{\@jtypeAlignFalse}[2]{\judgment{#1}{}{}{}{}{#2}{\jtype@sym}}
  
\newcommand{\jtypeeq}{\@ifnextchar*{\@jtypeeqAlignTrue}{\@jtypeeqAlignFalse}}
\newcommand{\@jtypeeqAlignTrue}[4]{\judgment*{#2}{#3}{\jdeq}{#4}{}{}{\jtype@sym}}
\newcommand{\@jtypeeqAlignFalse}[3]{\judgment{#1}{#2}{\jdeq}{#3}{}{}{\jtype@sym}}

\newcommand{\jtypedefn}{\@ifnextchar*{\@jtypedefnAlignTrue}{\@jtypedefnAlignFalse}}
\newcommand{\@jtypedefnAlignTrue}[4]{\judgment*{#2}{#3}{\defeq}{#4}{}{}{\jtype@sym}}
\newcommand{\@jtypedefnAlignFalse}[3]{\judgment{#1}{#2}{\defeq}{#3}{}{}{\jtype@sym}}
  
%%%% Judgments about terms
\newcommand{\jterm}{\@ifnextchar*{\@jtermAlignTrue}{\@jtermAlignFalse}}
\newcommand{\@jtermAlignTrue}[4]{\judgment*{#2}{}{}{#4}{:}{#3}{}}
\newcommand{\@jtermAlignFalse}[3]{\judgment{#1}{}{}{#3}{:}{#2}{}}

\newcommand{\jtermeq}{\@ifnextchar*{\@jtermeqAlignTrue}{\@jtermeqAlignFalse}}
\newcommand{\@jtermeqAlignTrue}[5]{\judgment*{#2}{#4}{\jdeq}{#5}{:}{#3}{}}
\newcommand{\@jtermeqAlignFalse}[4]{\judgment{#1}{#3}{\jdeq}{#4}{:}{#2}{}}

\newcommand{\jtermdefn}{\@ifnextchar*{\@jtermdefnAlignTrue}{\@jtermdefnAlignFalse}}
\newcommand{\@jtermdefnAlignTrue}[5]{\judgment*{#2}{#4}{\defeq}{#5}{:}{#3}{}}
\newcommand{\@jtermdefnAlignFalse}[4]{\judgment{#1}{#3}{\defeq}{#4}{:}{#2}{}}
\makeatother

%%%%%%%%%%%%%%%%%%%%%%%%%%%%%%%%%%%%%%%%%%%%%%%%%%%%%%%%%%%%%%%%%%%%%%%%%%%%%%%%
%%%% THE EMPTY CONTEXT

\newcommand{\emptysym}{[\;]}
\newcommand{\emptyc}{{\emptysym}}
\newcommand{\emptyf}[1][]{{\emptysym}_{#1}}
\newcommand{\emptytm}[1][]{\typefont{\#}_{#1}}

%%%%%%%%%%%%%%%%%%%%%%%%%%%%%%%%%%%%%%%%%%%%%%%%%%%%%%%%%%%%%%%%%%%%%%%%%%%%%%%%
%%%% CONTEXT EXTENSION 
%%%%
%%%% The context extension command.
%%%%
%%%% To get a feeling of how the command works, here are a few examples.
%%%% \ctxext{A}{B} will print A.B
%%%% \ctxext{{A}{B}}{C} will print (A.B).C
%%%% \ctxext{{{A}{B}}{C}}{{D}{E}} will print ((A.B).C).(D.E)

\makeatletter
\newcommand{\ctxext}[2]{\@ctxext@ctx #1.\@ctxext@type #2}
\newcommand{\@ctxext}{\@ifnextchar\bgroup{\@@ctxext}{}}
\newcommand{\@ctxext@ctx}{%
  \@ifnextchar\ctxext{\@ctxext@nested}{%
  \@ifnextchar\ctxwk{\@ctxwk@nested}{%
  \@ifnextchar\jcomp{\@jcomp@nested}{%
  \@ifnextchar\jvcomp{\@jvcomp@nested}{%
  \@ifnextchar\jfcomp{\@jfcomp@nested}{%
  \@ctxext}}}}}}
\newcommand{\@ctxext@type}{%
  \@ifnextchar\ctxext{\@ctxext@nested}{%
  \@ifnextchar\subst{\@subst@nested}{%
  \@ifnextchar\jcomp{\@jcomp@nested}{%
  \@ifnextchar\jvcomp{\@jvcomp@nested}{%
  \@ifnextchar\jfcomp{\@jfcomp@nested}{%
  \@ctxext}}}}}}
\newcommand{\@@ctxext}[1]{\@ifnextchar\bgroup{\@ctxext@parens{#1}}{#1}}
\newcommand{\@ctxext@parens}[2]{(\ctxext{#1}{#2})}
\newcommand{\@ctxext@nested}[3]{\@ctxext@parens{#2}{#3}}

%%%% We want that some commands accept binary trees as arguments that default
%%%% into extensions. We make the following command to realize this
\newcommand{\default@ctxext}{\@ifnextchar\bgroup{\ctxext}{}}
\newcommand{\default@ctxext@parens}{\@ifnextchar\bgroup{\@ctxext@parens}{}}
\makeatother

%%%%%%%%%%%%%%%%%%%%%%%%%%%%%%%%%%%%%%%%%%%%%%%%%%%%%%%%%%%%%%%%%%%%%%%%%%%%%%%%
%%%% SUBSTITUTION

%%%% The substitution command will act the following way
%%%%
%%%% \subst{x}{P} will print P[x]
%%%% \subst{x}{{f}{Q}} will print Q[f][x]
%%%% \subst{{x}{f}}{{x}{Q}} will print Q[x][f[x]]

\makeatletter
\newcommand{\subst}[3][]{%
  \@subst@type #3{}[\@subst@term #2]^{#1}}
\newcommand{\@subst}{%
  \@ifnextchar\bgroup{\@@subst}{}}
\newcommand{\@@subst}[1]{%
  \@ifnextchar\bgroup{\subst{#1}}{#1}}
\newcommand{\@subst@term}{%
  \@subst}
\newcommand{\@subst@type}{%
  \@ifnextchar\ctxext{\@ctxext@nested}{%
  \@ifnextchar\ctxwk{\@ctxwk@nested}{%
  \@ifnextchar\jcomp{\@jcomp@nested}{%
  \@ifnextchar\tmext{\@tmext@nested}{%
  \@ifnextchar\jvcomp{\@jvcomp@nested}{%
  \@ifnextchar\jfcomp{\@jfcomp@nested}{%
%  \@ifnextchar\mfam{\@mfam@nested}{%
%  \@ifnextchar\mtm{\@mtm@nested}}
\newcommand{\subst@type@unfold}[1]{#1}
\newcommand{\@subst@nested}[3]{%
  \@subst@parens{#2}{#3}}
\newcommand{\@subst@parens}[2]{%
  (\subst{#1}{#2})}
\makeatother

%%%%%%%%%%%%%%%%%%%%%%%%%%%%%%%%%%%%%%%%%%%%%%%%%%%%%%%%%%%%%%%%%%%%%%%%%%%%%%%%
%%%% WEAKENING

%%%% The weakening command is very much like the substitution command.

\makeatletter
\newcommand{\ctxwk}[3][]{%
  \langle\@ctxwk@act #2\rangle^{#1} \@ctxwk@pass #3}
\newcommand{\@ctxwk}{%
  \@ifnextchar\bgroup{\@@ctxwk}{}}
\newcommand{\@@ctxwk}[1]{%
  \@ifnextchar\bgroup{\ctxwk{#1}}{#1}}
\newcommand{\@ctxwk@act}{%
  \@ctxwk}
\newcommand{\@ctxwk@pass}{%
  \@ifnextchar\ctxext{\@ctxext@nested}{%
  \@ifnextchar\subst{\@subst@nested}{%
  \@ifnextchar\jcomp{\@jcomp@nested}{%
  \@ifnextchar\tmext{\@tmext@nested}{%
  \@ifnextchar\jvcomp{\@jvcomp@nested}{%
  \@ifnextchar\jfcomp{\@jfcomp@nested}{%
%  \@ifnextchar\mfam{\@mfam@nested}{%
%  \@ifnextchar\mtm{\@mtm@nested}}
\newcommand{\@ctxwk@parens}[2]{%
  (\ctxwk{#1}{#2})}
\newcommand{\@ctxwk@nested}[3]{%
  \@ctxwk@parens{#2}{#3}}
\makeatother

%%%% Not sure if we're gonna need the following.
\newcommand{\ctxwkop}[2]{%
  \ctxwk{#2}{#1}}
  
%%%%%%%%%%%%%%%%%%%%%%%%%%%%%%%%%%%%%%%%%%%%%%%%%%%%%%%%%%%%%%%%%%%%%%%%%%%%%%%%
%%%% IDENTITY TERMS

\makeatletter
\newcommand{\idtm}[1]{\typefont{id}_{\default@ctxext #1}}
\makeatother

%%%%%%%%%%%%%%%%%%%%%%%%%%%%%%%%%%%%%%%%%%%%%%%%%%%%%%%%%%%%%%%%%%%%%%%%%%%%%%%%
%%%% TERM EXTENSION
%%%%
%%%% The term extension command \tmext is slightly complicated because 
%%%% \tmext@unfold should do different things depending on whether it has two
%%%% or four arguments. Thus \tmext has a full form and a short form, where
%%%% the short form has two arguments and the full form has four. 

\makeatletter

%%%% The basic term extension commands
\newcommand{\default@tmext}{\@ifnextchar\bgroup{\tmext}{}}
\newcommand{\tmext}[2]{%
  \@ifnextchar\bgroup{\tmext@full{#1}{#2}}{\tmext@short{#1}{#2}}}
\newcommand{\tmext@full}[4]{%
  \ctxext{\tmext@testleft #3}{\tmext@testright #4}}
\newcommand{\tmext@short}[2]{%
  \ctxext{\tmext@testleft #1}{\tmext@testright #2}}
\newcommand{\tmext@testleft}{%
  \@ifnextchar\bgroup{\@tmext@parens}{%
  \@ifnextchar\tmext{\@tmext@nested}{%
  \@ifnextchar\ctxwk{\@ctxwk@nested}{%
  \@ifnextchar\jcomp{\@jcomp@nested}{%
  \@ifnextchar\jvcomp{\@jvcomp@nested}{%
  \@ifnextchar\jfcomp{\@jfcomp@nested}{%
%  \default@tmext
  }}}}}}}
\newcommand{\tmext@testright}{%
  \@ifnextchar\bgroup{\@tmext@parens}{%
  \@ifnextchar\tmext{\@tmext@nested}{%
  \@ifnextchar\subst{\@subst@nested}{%
  \@ifnextchar\jcomp{\@jcomp@nested}{%
  \@ifnextchar\jvcomp{\@jvcomp@nested}{%
  \@ifnextchar\jfcomp{\@jfcomp@nested}{%
  \@ifnextchar\cprojfst{\cprojfst@nested}{%
  \@ifnextchar\cprojsnd{\cprojsnd@nested}{%
%  \default@tmext
  }}}}}}}}}
\newcommand{\@tmext@nested}[1]{%
  \@tmext@parens}
\newcommand{\@tmext@parens}[2]{%
  \@ifnextchar\bgroup
    {\tmext@full@parens{#1}{#2}}
    {(\tmext@short{#1}{#2})}}
\newcommand{\tmext@full@parens}[4]{%
  (\tmext@full{#1}{#2}{#3}{#4})}

%%%% The unfolded term extension commands
\newcommand{\tmext@unfold}[2]{%
  \@ifnextchar\bgroup{\tmext@unfold@full{#1}{#2}}{\tmext@short{#1}{#2}}}
\newcommand{\tmext@unfold@full}[4]{%  
  \subst{#4}{{#3}{\idtm{\ctxext{#1}{#2}}}}}
\newcommand{\@tmext@unfold@nested}[1]{%
  \@tmext@unfold@parens}
\newcommand{\@tmext@unfold@parens}[4]{%
  (\tmext@unfold{#1}{#2}{#3}{#4})}
\makeatother

%%%%%%%%%%%%%%%%%%%%%%%%%%%%%%%%%%%%%%%%%%%%%%%%%%%%%%%%%%%%%%%%%%%%%%%%%%%%%%%%
%%%% JUDGMENTAL MORPHISMS

\makeatletter

%%%% The judgment that f is a morphism from A to B in context \Gamma.
\newcommand{\jhomsym}[3][]{%
  ~~\textit{hom}_{#1}(\default@ctxext #2,\default@ctxext #3)}
\newcommand{\jhom}{%
  \@ifnextchar*{\@jhomAlignTrue}{\@jhomAlignFalse}}
\newcommand{\@jhomAlignTrue}[5]{%
  \judgment*{#2}{}{}{#5}{}{}{\jhomsym{#3}{#4}}}
\newcommand{\@jhomAlignFalse}[4]{%
  \judgment{#1}{}{}{#4}{}{}{\jhomsym{#2}{#3}}}
\newcommand{\jhomeq}{%
  \@ifnextchar*{\@jhomeqAlignTrue}{\@jhomeqAlignFalse}}
\newcommand{\@jhomeqAlignTrue}[6]{%
  \judgment*{#2}{#5}{\jdeq}{#6}{}{}{\jhomsym{#3}{#4}}}
\newcommand{\@jhomeqAlignFalse}[5]{%
  \judgment{#1}{#4}{\jdeq}{#5}{}{}{\jhomsym{#2}{#3}}}
\newcommand{\jhomdefn}{%
  \@ifnextchar*{\@jhomdefnAlignTrue}{\@jhomdefnAlignFalse}}
\newcommand{\@jhomdefnAlignTrue}[6]{%
  \judgment*{#2}{#5}{\defeq}{#6}{}{}{\jhomsym{#3}{#4}}}
\newcommand{\@jhomdefnAlignFalse}[5]{%
  \judgment{#1}{#4}{\defeq}{#5}{}{}{\jhomsym{#2}{#3}}}

\newcommand{\jhom@unfold}[4]{%
  \jterm
    {{#1}{#2}}
    {\ctxwk{\default@ctxext #2}{\default@ctxext@parens #3}}
    {#4}}
\newcommand{\jhomeq@unfold}[5]{%
  \jtermeq
    {{#1}{#2}}
    {\ctxwk{\default@ctxext #2}{\default@ctxext@parens #3}}
    {#4}
    {#5}}
\newcommand{\jhomdefn@unfold}[5]{%
  \jtermdefn
    {{#1}{#2}}
    {\ctxwk{\default@ctxext #2}{\default@ctxext@parens #3}}
    {#4}
    {#5}}

%%%% Composition of morphisms
\newcommand{\jcomp}[3]{%
  \jcomp@testleft #3 \circ \jcomp@testright #2}
\newcommand{\jcomp@testleft}{%
  \@ifnextchar\jcomp{\@jcomp@nested}{%
  \@ifnextchar\ctxwk{\@ctxwk@nested}{%
  \@ifnextchar\ctxext{\@ctxext@nested}{%
  \@ifnextchar\bgroup{\@jcomp@parens}{%
  \@ifnextchar\tmext{\@tmext@nested}{%
  \@ifnextchar\jvcomp{\@jvcomp@nested}{%
  \@ifnextchar\jfcomp{\@jfcomp@nested}{%
  }}}}}}}}
\newcommand{\jcomp@testright}{%
  \@ifnextchar\jcomp{\@jcomp@nested}{%
  \@ifnextchar\subst{\@subst@nested}{%
  \@ifnextchar\ctxext{\@ctxext@nested}{%
  \@ifnextchar\bgroup{\@jcomp@parens}{%
  \@ifnextchar\tmext{\@tmext@nested}{%
  \@ifnextchar\jvcomp{\@jvcomp@nested}{%
  \@ifnextchar\jfcomp{\@jfcomp@nested}{%
  }}}}}}}}
\newcommand{\@jcomp@nested}[4]{%
  \@jcomp@parens{#2}{#3}{#4}}
\newcommand{\@jcomp@parens}[3]{%
  (\jcomp{#1}{#2}{#3})}

\newcommand{\jcomp@unfold}[3]{%
  \subst
    {\jcomp@unfold@test@preside #2}
    {\ctxwk{\default@ctxext #1}{\jcomp@unfold@test@postside #3}}}
\newcommand{\jcomp@unfold@test@preside}{%
  \@ifnextchar\bgroup{\@jcomp@unfold@parens}{}}
\newcommand{\jcomp@unfold@test@postside}{%
  \@ifnextchar\bgroup{\@jcomp@unfold@parens}{}}
\newcommand{\@jcomp@unfold@nested}[4]{%
  \@jcomp@unfold@parens{#2}{#3}{#4}}
\newcommand{\@jcomp@unfold@parens}[3]{%
  (\jcomp@unfold{#1}{#2}{#3})}

%%%% Vertical composition of morphisms.
\newcommand{\jvcomp}[3]{%
  \jcomp@testleft #2 * \jcomp@testright #3}
\newcommand{\jvcomp@testleft}{%
  \@ifnextchar\jvcomp{\@jvcomp@nested}{%
  \@ifnextchar\ctxwk{\@ctxwk@nested}{%
  \@ifnextchar\ctxext{\@ctxext@nested}{%
  \@ifnextchar\bgroup{\@jvcomp@parens}{%
  \@ifnextchar\tmext{\@tmext@nested}{%
  \@ifnextchar\jcomp{\@jcomp@nested}{%
  \@ifnextchar\jfcomp{\@jfcomp@nested}{%
  }}}}}}}}
\newcommand{\jvcomp@testright}{%
  \@ifnextchar\jvcomp{\@jvcomp@nested}{%
  \@ifnextchar\subst{\@subst@nested}{%
  \@ifnextchar\ctxext{\@ctxext@nested}{%
  \@ifnextchar\bgroup{\@jvcomp@parens}{%
  \@ifnextchar\tmext{\@tmext@nested}{%
  \@ifnextchar\jcomp{\@jcomp@nested}{%
  \@ifnextchar\jfcomp{\@jfcomp@nested}{%
  }}}}}}}}
\newcommand{\@jvcomp@nested}[4]{%
  \@jvcomp@parens{#2}{#3}{#4}}
\newcommand{\@jvcomp@parens}[3]{%
  (\jvcomp{#1}{#2}{#3})}

\newcommand{\jvcomp@unfold}[3]{%
  \tmext{}{}{\ctxwk{#1}{#2}}{#3}
  }
\newcommand{\jvcomp@unfold@test@preside}{%
  \@ifnextchar\bgroup{\@jvcomp@unfold@parens}{}}
\newcommand{\jvcomp@unfold@test@postside}{%
  \@ifnextchar\bgroup{\@jvcomp@unfold@parens}{}}
\newcommand{\@jvcomp@unfold@nested}[4]{%
  \@jvcomp@unfold@parens{#2}{#3}{#4}}
\newcommand{\@jvcomp@unfold@parens}[3]{%
  (\jvcomp@unfold{#1}{#2}{#3})}

%%%% The judgment that F is a morphism from P to Q over f in context \Gamma.
\newcommand{\jfhomsym}[3]{\jhomsym[{#1}]{#2}{#3}}
\newcommand{\jfhom}{%
  \@ifnextchar*{\jfhomAlignTrue}{\jfhomAlignFalse}}
\newcommand{\jfhomAlignTrue}[8]{
  \judgment*{#2}{}{}{#8}{}{}{\jfhomsym{#5}{#6}{#7}}}
\newcommand{\jfhomAlignFalse}[7]{
  \judgment{#1}{}{}{#7}{}{}{\jfhomsym{#4}{#5}{#6}}}
\newcommand{\jfhomeq}[8]{%
  \judgment{#1}{#7}{\jdeq}{#8}{}{}{\jhomsym[{#4}]{#5}{#6}}}
\newcommand{\jfhomdefn}[8]{%
  \judgment{#1}{#7}{\defeq}{#8}{}{}{\jhomsym[{#4}]{#5}{#6}}}
\newcommand{\jfhom@unfold}[7]{%
  \jterm
    {{{#1}{#2}}{#5}}
    {\ctxwk{\default@ctxext #5}{\jcomp{#2}{#4}{#6}}}
    {#7}}
    
\newcommand{\jfcomp}[5]{%
  \jfcomp@testleft #5 \bullet \jfcomp@testright #4}
\newcommand{\jfcomp@testleft}{%
  \@ifnextchar\bgroup{\@jfcomp@parens}{%
  \@ifnextchar\jfcomp{\@jfcomp@nested}{%
  \@ifnextchar\jcomp{\@jcomp@nested}{%
  \@ifnextchar\ctxwk{\@ctxwk@nested}{%
  \@ifnextchar\tmext{\@tmext@nested}{%
  \@ifnextchar\jvcomp{\@jvcomp@nested}{%
  }}}}}}}
\newcommand{\jfcomp@testright}{%
  \@ifnextchar\bgroup{\@jfcomp@parens}{%
  \@ifnextchar\jfcomp{\@jfcomp@nested}{%
  \@ifnextchar\jcomp{\@jcomp@nested}{%
  \@ifnextchar\subst{\@subst@nested}{%
  \@ifnextchar\tmext{\@tmext@nested}{%
  \@ifnextchar\jvcomp{\@jvcomp@nested}{%
  }}}}}}}
\newcommand{\@jfcomp@nested}[1]{%
  \@jfcomp@parens}
\newcommand{\@jfcomp@parens}[5]{%
  (\jfcomp{#1}{#2}{#3}{#4}{#5})}
  
\newcommand{\jfcomp@unfold}[5]{%
  \jcomp{#3}{#4}{{#1}{#2}{#5}}}
\makeatother

%%%%%%%%%%%%%%%%%%%%%%%%%%%%%%%%%%%%%%%%%%%%%%%%%%%%%%%%%%%%%%%%%%%%%%%%%%%%%%%%
%%%% JUDGMENTAL TRIVIAL COFIBRATIONS

\newcommand{\jtcext}{\tilde}

%%%%%%%%%%%%%%%%%%%%%%%%%%%%%%%%%%%%%%%%%%%%%%%%%%%%%%%%%%%%%%%%%%%%%%%%%%%%%%%%
%%%% CONTEXT PROJECTIONS

\makeatletter
\newcommand{\cprojgenf}[3]{%
  \typefont{pr}^{%
    \@ifnextchar\bgroup{\@ctxext@parens}{%
    \@ifnextchar\ctxext{\@ctxext@nested}{%
    }}
    #2,
    \@ifnextchar\bgroup{\@ctxext@parens}{%
    \@ifnextchar\ctxext{\@ctxext@nested}{%
    }}
    #3
    }_{#1}}
\newcommand{\cprojgen}[4]{%
  \subst{#4}{\cprojgenf{#1}{#2}{#3}}}
\newcommand{\cprojgenf@nested}[1]{%
  \cprojgenf@parens}
\newcommand{\cprojgenf@parens}[3]{%
  (\cprojgenf{#1}{#2}{#3})}
\newcommand{\cprojgen@nested}[1]{%
  \cprojgen@parens}
\newcommand{\cprojgen@parens}[4]{%
  (\cprojgen{#1}{#2}{#3}{#4})}

\newcommand{\cprojfstf}[2]{%
  \cprojgenf{0}{#1}{#2}}
\newcommand{\cprojfstf@nested}[1]{%
  \cprojfstf@parens}
\newcommand{\cprojfstf@parens}[2]{%
  (\cprojfstf{#1}{#2})}
\newcommand{\cprojfstf@unfold}[2]{%
  \ctxwk{\default@ctxext #2}\idtm{\default@ctxext #1}}

\newcommand{\cprojfst}[3]{%
  \cprojgen{0}{#1}{#2}{#3}}
\newcommand{\cprojfst@nested}[1]{%
  \cprojfst@parens}
\newcommand{\cprojfst@parens}[3]{%
  (\cprojfst{#1}{#2}{#3})}
\newcommand{\cprojfst@unfold}[3]{%
  \subst{#3}{(\cprojfstf@unfold{#1}{#2})}}

\newcommand{\cprojsndf}[2]{%
  \cprojgenf{1}{#1}{#2}}
\newcommand{\cprojsndf@nested}[1]{%
  \cprojsndf@parens}
\newcommand{\cprojsndf@parens}[2]{%
  (\cprojsndf{#1}{#2})}
\newcommand{\cprojsndf@unfold}[2]{%
  \idtm{\default@ctxext #2}}

\newcommand{\cprojsnd}[3]{%
  \cprojgen{1}{#1}{#2}{#3}}
\newcommand{\cprojsnd@nested}[1]{%
  \cprojsnd@parens}
\newcommand{\cprojsnd@parens}[3]{%
  (\cprojsnd{#1}{#2}{#3})}
\newcommand{\cprojsnd@unfold}[3]{%
  \subst{#3}{\cprojsnd@unfold{#1}{#2}}}
  
%%%% The sandwich function
\newcommand{\sandwich}[3]{\typefont{sw}^{#1,#2,#3}}
\newcommand{\sandwich@unfold}[3]{\typefont{sw}^{#1,#2,#3}}
\makeatother

%%%%%%%%%%%%%%%%%%%%%%%%%%%%%%%%%%%%%%%%%%%%%%%%%%%%%%%%%%%%%%%%%%%%%%%%%%%%%%%%
%%%% FIBER INCLUSIONS

\makeatletter
\newcommand{\finc}[2]{\typefont{in}^{#2}_{#1}}
\newcommand{\finc@unfold}[2]{\tmext{}{}{\ctxwk{\subst{x}{P}}{x}}{\idtm{\subst{x}{P}}}}
\makeatother

%%%%%%%%%%%%%%%%%%%%%%%%%%%%%%%%%%%%%%%%%%%%%%%%%%%%%%%%%%%%%%%%%%%%%%%%%%%%%%%%
%%%% THE UNIT TYPE

\makeatletter
\newcommand{\unitc}[1]{%
  \unit^0_{\default@ctxext #1}}
\newcommand{\unitct}[1]{%
  \ttt^0_{\default@ctxext #1}}
\newcommand{\unitf}[2]{%
  \unit^1_{\default@ctxext #1,\default@ctxext #2}}
\newcommand{\unitft}[2]{%
  \ttt^1_{\default@ctxext #1,\default@ctxext #2}}
\makeatother

%%%%%%%%%%%%%%%%%%%%%%%%%%%%%%%%%%%%%%%%%%%%%%%%%%%%%%%%%%%%%%%%%%%%%%%%%%%%%%%%
%%%% DEPENDENT FUNCTION TYPES

\makeatletter
\newcommand{\sprd}[2]{\Pi(\default@ctxext #1,\default@ctxext #2)}
\begin{comment}
\newcommand{\@sprd@test@cod}[2]{%
  \@ifnextchar\bgroup{\@sprd@do@cod{#1}}{%
  \Pi(\@sprd@test@dom{#1}{#2} #1,
  }}
\newcommand{\@sprd@do@cod}[4]{%
  \ctxext{\@sprd{#1}{#2}}{\@sprd{#1}{#3}}
  }
\newcommand{\@sprd}[2]{
  \@ifnextchar\bgroup{\@@sprd}{%
    \Pi(}
    #1,{#2})
  }
\newcommand{\@@sprd}[5]{%
  \sprd{#1}{\sprd{#2}{#4}}
  }
\end{comment}

\newcommand{\slam}[3]{%
  \lambda^{{\default@ctxext@parens #1},{\default@ctxext@parens #2}}
  (\default@ctxext #3)
  }
\newcommand{\sev}[1]{\tfev(#1)}

\makeatother

%%%%%%%%%%%%%%%%%%%%%%%%%%%%%%%%%%%%%%%%%%%%%%%%%%%%%%%%%%%%%%%%%%%%%%%%%%%%%%%%
%%%% NON-DEPENDENT FUNCTION TYPES

\newcommand{\jfun}[2]{#1\to#2}

%%%%%%%%%%%%%%%%%%%%%%%%%%%%%%%%%%%%%%%%%%%%%%%%%%%%%%%%%%%%%%%%%%%%%%%%%%%%%%%%
%%%% THE CONSTRUCTORS OF THE TYPE THEORY OF MODELS

\makeatletter
%%%% The initial model
\newcommand{\mctx}{%
  \mathcal{C}}

%%%% The family constructor
\newcommand{\mfam}[2][]{%
  \mathcal{F}_{\default@ctxext #2}^{#1}}
\newcommand{\@mfam@nested}[1]{\@mfam@parens}
\newcommand{\@mfam@parens}[2][]{(\mfam[#1]{#2})}

%%%% The terms constructor
\newcommand{\mtm}[2][]{%
  \mathcal{T}_{\default@ctxext #2}^{#1}}
\newcommand{\@mtm@nested}[1]{\@mtm@parens}
\newcommand{\@mtm@parens}[2][]{(\mtm[#1]{#2})}

%%%% The empty type constructor
\newcommand{\tfemp}[1]{%
  \typefont{emp}_{\default@ctxext #1}}
\newcommand{\tft}[1]{%
  \typefont{t}_{\default@ctxext #1}}

%%%% The extension constructor
\newcommand{\tfext}[1]{%
  \typefont{ext}_{\default@ctxext #1}}

%%%% The substitution constructor
\newcommand{\tfsubst}[1]{%
  \typefont{subst}_{\default@ctxext #1}}
  
%%%% The weakening constructor
\newcommand{\tfwk}[1]{%
  \typefont{wk}_{\default@ctxext #1}}

%%%% The identity function constructor
\newcommand{\tfid}[1]{%
  \typefont{idtm}_{\default@ctxext #1}}
\makeatother

%%%%%%%%%%%%%%%%%%%%%%%%%%%%%%%%%%%%%%%%%%%%%%%%%%%%%%%%%%%%%%%%%%%%%%%%%%%%%%%%

%%%% Introducing logical usage of fonts.
\newcommand{\modelfont}{\mathit} % use 'mf' in command to indicate model font
\newcommand{\typefont}{\mathsf} % use 'tf' in command to indicate type font
\newcommand{\catfont}{\mathrm} % use 'cf' in command to indicate cat font

%%%%%%%%%%%%%%%%%%%%%%%%%%%%%%%%%%%%%%%%%%%%%%%%%%%%%%%%%%%%%%%%%%%%%%%%%%%%%%%%
%%%% Some macros of the book are redefined.

\renewcommand{\UU}{\typefont{U}}
\renewcommand{\isequiv}{\typefont{isEquiv}}
\renewcommand{\happly}{\typefont{hApply}}
\renewcommand{\pairr}[1]{{\mathopen{}\langle #1\rangle\mathclose{}}}
\renewcommand{\type}{\typefont{Type}}
\renewcommand{\op}[1]{{{#1}^\typefont{op}}}
\renewcommand{\susp}{\typefont{\Sigma}}

%%%%%%%%%%%%%%%%%%%%%%%%%%%%%%%%%%%%%%%%%%%%%%%%%%%%%%%%%%%%%%%%%%%%%%%%%%%%%%%%
%%%% The following is a big unorganized list of new macros that we use in the
%%%% notes. 

\newcommand{\tfW}{\typefont{W}}
\newcommand{\tfM}{\typefont{M}}
\newcommand{\mfM}{\modelfont{M}}
\newcommand{\mfN}{\modelfont{N}}
\newcommand{\tfctx}{\typefont{ctx}}
\newcommand{\mftypfunc}[1]{{\modelfont{typ}^{#1}}}
\newcommand{\mftyp}[2]{{\mftypfunc{#1}(#2)}}
\newcommand{\tftypfunc}[1]{{\typefont{typ}^{#1}}}
\newcommand{\tftyp}[2]{{\tftypfunc{#1}(#2)}}
\newcommand{\hfibfunc}[1]{\typefont{fib}_{#1}}
\newcommand{\mappingcone}[1]{\mathcal{C}_{#1}}
\newcommand{\equifib}{\typefont{equiFib}}
\newcommand{\tfcolim}{\typefont{colim}}
\newcommand{\tflim}{\typefont{lim}}
\newcommand{\tfdiag}{\typefont{diag}}
\newcommand{\tfGraph}{\typefont{Graph}}
\newcommand{\mfGraph}{\modelfont{Graph}}
\newcommand{\unitGraph}{\unit^\mfGraph}
\newcommand{\UUGraph}{\UU^\mfGraph}
\newcommand{\tfrGraph}{\typefont{rGraph}}
\newcommand{\mfrGraph}{\modelfont{rGraph}}
\newcommand{\isfunction}{\typefont{isFunction}}
\newcommand{\tfconst}{\typefont{const}}
\newcommand{\conemap}{\typefont{coneMap}}
\newcommand{\coconemap}{\typefont{coconeMap}}
\newcommand{\tflimits}{\typefont{limits}}
\newcommand{\tfcolimits}{\typefont{colimits}}
\newcommand{\islimiting}{\typefont{isLimiting}}
\newcommand{\iscolimiting}{\typefont{isColimiting}}
\newcommand{\islimit}{\typefont{isLimit}}
\newcommand{\iscolimit}{\typefont{iscolimit}}
\newcommand{\pbcone}{\typefont{cone_{pb}}}
\newcommand{\tfinj}{\typefont{inj}}
\newcommand{\tfsurj}{\typefont{surj}}
\newcommand{\tfepi}{\typefont{epi}}
\newcommand{\tftop}{\typefont{top}}
\newcommand{\sbrck}[1]{\Vert #1\Vert}
\newcommand{\strunc}[2]{\Vert #2\Vert_{#1}}
\newcommand{\gobjclass}{{\typefont{U}^\mfGraph}}
\newcommand{\gcharmap}{\typefont{fib}}
\newcommand{\diagclass}{\typefont{T}}
\newcommand{\opdiagclass}{\op{\diagclass}}
\newcommand{\equifibclass}{\diagclass^{\eqv{}{}}}
\newcommand{\universe}{\typefont{U}}
\newcommand{\catid}[1]{{\catfont{id}_{#1}}}
\newcommand{\isleftfib}{\typefont{isLeftFib}}
\newcommand{\isrightfib}{\typefont{isRightFib}}
\newcommand{\leftLiftings}{\typefont{leftLiftings}}
\newcommand{\rightLiftings}{\typefont{rightLiftings}}
\newcommand{\psh}{\typefont{Psh}}
\newcommand{\rgclass}{\typefont{\Omega^{RG}}}
\newcommand{\terms}[2][]{\lfloor #2 \rfloor^{#1}}
\newcommand{\grconstr}[2]
             {\mathchoice % max size is textstyle size.
             {{\textstyle \int_{#1}}#2}% 
             {\int_{#1}#2}%
             {\int_{#1}#2}%
             {\int_{#1}#2}}
\newcommand{\ctxhom}[3][]{\typefont{hom}_{#1}(#2,#3)}
\newcommand{\graphcharmapfunc}[1]{\gcharmap_{#1}}
\newcommand{\graphcharmap}[2][]{\graphcharmapfunc{#1}(#2)}
\newcommand{\tfexp}[1]{\typefont{exp}_{#1}}
\newcommand{\tffamfunc}{\typefont{fam}}
\newcommand{\tffam}[1]{\tffamfunc(#1)}
\newcommand{\tfev}{\typefont{ev}}
\newcommand{\tfcomp}{\typefont{comp}}
\newcommand{\isDec}[1]{\typefont{isDecidable}(#1)}
\newcommand{\smal}{\mathcal{S}}
\renewcommand{\modal}{{\ensuremath{\ocircle}}}
\newcommand{\eqrel}{\typefont{EqRel}}
\newcommand{\piw}{\ensuremath{\Pi\typefont{W}}} %% to be used in conjunction with -pretopos.
\renewcommand{\sslash}{/\!\!/}
\newcommand{\mprd}[2]{\Pi(#1,#2)}
\newcommand{\msm}[2]{\Sigma(#1,#2)}
\newcommand{\midt}[1]{\idvartype_#1}
\newcommand{\reflf}[1]{\typefont{refl}^{#1}}
\newcommand{\tfJ}{\typefont{J}}
\newcommand{\tftrans}{\typefont{trans}}

\newcommand{\tfT}{\typefont{T}}
\newcommand{\reflsym}{{\mathsf{refl}}}
\newcommand{\strans}[2]{\ensuremath{{#1}_{*}({#2})}}
\newcommand{\eqtype}[1]{\typefont{Eq}_{#1}}
\newcommand{\eqtoid}[1]{\typefont{eqtoid}(#1)}
\newcommand{\greek}{\mathrm}
\newcommand{\product}[2]{{#1}\times{#2}}
\newcommand{\pairp}[1]{(#1)}
\newcommand{\jequalizer}[3]{\{#1|#2\jdeq #3\}}
\newcommand{\jequalizerin}[2]{\iota_{#1,#2}}
\newcommand{\tounit}[1]{{!_{#1}}}
\newcommand{\trwk}{\typefont{trwk}}
\newcommand{\trext}{\typefont{trext}}

%%%%%%%%%%%%%%%%%%%%%%%%%%%%%%%%%%%%%%%%%%%%%%%%%%%%%%%%%%%%%%%%%%%%%%%%%%%%%%%%
%%%% When investigation pointed structures we use the \pt macro.

\makeatletter
\newcommand{\pt}[1][]{*_{
  \@ifnextchar\undergraph{\@undergraph@nested}
    {\@ifnextchar\underovergraph{\@underovergraph@nested}{}}#1}}
\makeatother

%%%%%%%%%%%%%%%%%%%%%%%%%%%%%%%%%%%%%%%%%%%%%%%%%%%%%%%%%%%%%%%%%%%%%%%%%%%%%%%%
%%%% OPERATIONS ON GRAPHS
%%%%
%%%% First of all, each graph has a type of vertices and a type of edges. The
%%%% type of vertices of a graph $\Gamma$ is denoted by $\pts{\Gamma}$;
%%%% and likewise for the type of edges.

\makeatletter
\newcommand{\pts}[1]{{\@graphop@nested{#1}}_{0}}
\newcommand{\edg}[1]{{\@graphop@nested{#1}}_{1}}
\newcommand{\@graphop@nested}[1]
  {\@ifnextchar\ctxext{\@ctxext@nested}
      {\@ifnextchar\undergraph{\@undergraph@nested}
         {\@ifnextchar\underovergraph{\@underovergraph@nested}{}}}
    #1}
\makeatother

%%%% The following operations of \undergraph and \underovergraph are used to
%%%% define the free category and the free groupoid of a graph, respectively

\makeatletter
\newcommand{\@undergraphtest}[2]{\@ifnextchar({#1}{#2}}
\newcommand{\undergraph}[2]{\@undergraphtest{\@undergraph@parens{#1}{#2}}{\@undergraph{#1}{#2}}}
\newcommand{\@undergraph}[2]{{#2/#1}}
\newcommand{\@undergraph@nested}[3]{\@undergraph@parens{#2}{#3}}
\newcommand{\@undergraph@parens}[2]{(\@undergraph{#1}{#2})}
\makeatother

\makeatletter
\newcommand{\underovergraph}[2]{\@underovergraphtest{\@underovergraph@parens{#1}{#2}}{\@underovergraph{#1}{#2}}}
\newcommand{\@underovergraph}[2]{{#2}\,{\parallel}\,{#1}}
\newcommand{\@underovergraphtest}{\@undergraphtest}
\newcommand{\@underovergraph@parens}[2]{(\@underovergraph{#1}{#2})}
\newcommand{\@underovergraph@nested}[3]{\@underovergraph@parens{#2}{#3}}
\makeatother

\newcommand{\graphid}[1]{\mathrm{id}_{#1}}
\newcommand{\freecat}[1]{\mathcal{C}(#1)}
\newcommand{\freegrpd}[1]{\mathcal{G}(#1)}


%%%%%%%%%%%%%%%%%%%%%%%%%%%%%%%%%%%%%%%%%%%%%%%%%%%%%%%%%%%%%%%%%%%%%%%%%%%%%%%%
%% Some tikz macros to typeset diagrams uniformly.

\tikzset{patharrow/.style={double,double equal sign distance,-,font=\scriptsize}}
\tikzset{description/.style={fill=white,inner sep=2pt}}
\tikzset{fib/.style={->>,font=\scriptsize}}

%% Used for extra wide diagrams, e.g. when the label is too large otherwise.
\tikzset{commutative diagrams/column sep/Huge/.initial=18ex}

%%%%%%%%%%%%%%%%%%%%%%%%%%%%%%%%%%%%%%%%%%%%%%%%%%%%%%%%%%%%%%%%%%%%%%%%%%%%%%%%
%%%% New theorem environment for conjectures.

\defthm{conj}{Conjecture}{Conjectures}

%%%%%%%%%%%%%%%%%%%%%%%%%%%%%%%%%%%%%%%%%%%%%%%%%%%%%%%%%%%%%%%%%%%%%%%%%%%%%%%%
%%%% The following environment for desiderata should not be there. It is better
%%%% to use the issue tracker for desiderata.

\newenvironment{desiderata}{\begingroup\color{blue}\textbf{Desiderata.}}
{\endgroup}

%%%%%%%%%%%%%%%%%%%%%%%%%%%%%%%%%%%%%%%%%%%%%%%%%%%%%%%%%%%%%%%%%%%%%%%%%%%%%%%%
%%%% The following piece of code from tex.stackexchange:
%%%%
%%%% http://tex.stackexchange.com/a/55180/14653
%%%%
%%%% We include it so that inference rules in align environments have enough
%%%% vertical space.

\newlength\minalignvsep

\makeatletter
\def\align@preamble{%
   &\hfil
    \setboxz@h{\@lign$\m@th\displaystyle{##}$}%
    \ifnum\row@>\@ne
    \ifdim\ht\z@>\ht\strutbox@
    \dimen@\ht\z@
    \advance\dimen@\minalignvsep
    \ht\strutbox\dimen@
    \fi\fi
    \strut@
    \ifmeasuring@\savefieldlength@\fi
    \set@field
    \tabskip\z@skip
   &\setboxz@h{\@lign$\m@th\displaystyle{{}##}$}%
    \ifnum\row@>\@ne
    \ifdim\ht\z@>\ht\strutbox@
    \dimen@\ht\z@
    \advance\dimen@\minalignvsep
    \ht\strutbox@\dimen@
    \fi\fi
    \strut@
    \ifmeasuring@\savefieldlength@\fi
    \set@field
    \hfil
    \tabskip\alignsep@
}
\makeatother

\minalignvsep.2em

\allowdisplaybreaks

%%%%%%%%%%%%%%%%%%%%%%%%%%%%%%%%%%%%%%%%%%%%%%%%%%%%%%%%%%%%%%%%%%%%%%%%%%%%%%%%

\setdescription[1]{itemsep=-0.2em}


%%%%%%%%%%%%%%%%%%%%%%%%%%%%%%%%%%%%%%%%%%%%%%%%%%%%%%%%%%%%%%%%%%%%%%%%%%%%%%%%
\title{Internal models of type theory and internal higher categories for the
Univalent Foundations}
\author{Egbert Rijke}
\date{\today}

\begin{document}

\maketitle

\begin{abstract}
A project is proposed where we investigate notions of internal models of type
theory and internal higher categories. We take the point of view that an
internal model is an internal higher category with extra structure interpreting
the type constructors. We propose to (1) explore possible definitions of both
notions, (2) find a zoo of examples and (3) find extensions of the underlying
theory and models thereof to include higher inductive types.
\end{abstract}

\tableofcontents

\part{Introduction}
The project proposal is described in \autoref{stage1} and \autoref{stage2}. In
the appendices we give elaborations on the topics discussed in the proposal itself.
By including these we intend to clarify the proposal and present some of the 
(unfinished) work that is already done.

\section{Stage one: establishing internal higher categories and internal models}\label{stage1}
The project I propose here has its origins in the beginning of 2013, when I proved a version
of the descent theorem for homotopy colimits in type theory while I was working
with Bas Spitters to develop notions from higher category theory in the
univalent foundations. To arrive at a
notion of diagram general enough to capture all the higher inductive types described
in chapter 6 of \cite{TheBook} excluding the truncations we needed type
theoretical graphs. The graphs form a model of type theory and indeed we needed
the interpretations of several of the basic type constructors to give an efficient approach
to the descent property and its proof. Although it was not an issue to describe the graph model
and the sense in which it models the type constructors, 
not all of type theory is interpreted very well. To start with, if $A$ and $B$ 
are families of graphs over a graph $\Gamma$, then $B$ isn't also a family of 
graphs over the extended graph $\ctxext{\Gamma}{A}$ (interfering with a good 
interpretation of weakening). Since there are such issues, it is actually
not so straightforward to describe what it models!

Apart from all the type constructors like dependent function and pair types and
identity types, the graph model shows three basic operations that should come
before those type constructors. They are extension, weakening and substitution.
All three of them can be defined such that they act not only on contexts (the graphs), 
but also on families and terms. Moreover, they are all compatible with each 
other. In fact, any structure which has a notion of contexts, families and terms
and which models extension, weakening and substitution and which has identity
functions is in a weak sense a category, hypothesizing that
\begin{quote}
\emph{Category theory is dependent type theory without type constructors.}
\end{quote}
In \autoref{tt} we sketch how type theory without type constructors but with
\emph{explicit} extension, weakening and substitution might look like. This
theory is written down keeping the guiding principle in mind that type theory
without basic constructors should be a generalized algebraic theory of higher
categories.

Thus, models of this theory should be higher categories with some strictness
built in, such as the strictness of associativity of substitution. But the
theory really only talks about contexts, families and sections. The interpretation
of this as a category works as follows: The objects are going to be the contexts.
Then we may consider the terms of the weakening $\ctxwk{\Gamma}{\Delta}$ of a
context $\Delta$ by a context $\Gamma$: they are the morphisms of the category.
The composition of morphisms follows from the action of substitution and
weakening on terms. 

In the proposed approach, a model of type theory isn't a category by assumption. Rather,
a model of type theory \emph{becomes} a category because it interprets type
theory. In other words, type theory does all the work for us concerning the
categorical structure. In particular, we have neither started our notion of models
with AKS-categories as described in \cite{1categories} nor with Dybjer's notion
of internal categories with families. We should gain that we get rid of
truncatedness assumptions that are present in both AKS-categories, where the
type of objects has to be a $1$-type, and internal categories with families,
where the type of morphisms has to be a setoid. It needs not much arguing that any way to impose
conditions on internal models beyond the interpretations of the rules of
type theory stands in the way of a clear understanding of what an internal
model should be and it is a main goal of the project to design
an internal type theory without the described deficiencies. 

\begin{comment}
\subsection{Ideas in the definition of internal models}
An internal model of type theory is like a category with families, but we want
to avoid having to state higher coherences. In fact, we don't even start our
definition with a category of contexts; instead we just take a \emph{type} of contexts. 
The morphisms will come from the terms, evaluation of a function at a given
term will come from substitution. We recognize three basic ingredients to models:
first there is a type of contexts; second, for every context there is a model of types in
that context and third, for every type in a given context there is a type of its
terms. Then there are three basic attributes: context extension, weakening and
substitution. Context extension provides us with families over types as well as
with an interpretation of dependent pair types. We need weakening 
so that families can depend on the same type multiple times (the way the
identity type of a type depends two times on that type) and to be able
to talk about non-dependent function types,
the morphims of our category. Substitution will give us a way
to work with fibers of families as well as composition of functions and evaluation
of functions at terms.

Because we require a \emph{model} of types in a context, all the structure
which we require at the bottom level will be required to exist higher up as well.
Thus, the model of types in a given context $\Gamma$ will have a type of contexts
itself, which can be seen as the type of types in $\Gamma$; it will have its
own notion of types in a context, its own notion of terms, context extension,
weakening and substitution together with all the structure require for it. For
instance, when $A$ is a type in context $\Gamma$ in a model $\mfM$, then there
is the model of types in context $A$, which is the model of families over $A$. 
This model is required to be \emph{definitionally equal to} the model of types
in the context $\ctxext{\Gamma}{A}$, the context extension of $\Gamma$ and $A$.
In this way we protect ourselves from the need to dig an infinitely deep structure
of models when we want to consider examples.

To give the definition of a model we shall also need to consider certain morphisms
of internal models. Those should preserve all the structure: contexts are mapped
to contexts; for every context a morphism of models mapping the model of types
in that context to the model of types in the image of that context; there should
be a mapping of terms and context extension, weakening and substitution should be
preserved. We need to consider those morphisms because we require context extension,
weakening and substitution to be of that kind, thereby respecting each other
in all possible ways.

When we have this framework set up, we can interpret the basic type constructors
such as $\Pi$, $\Sigma$ and $\idtypevar{}$.
The higher categorical structure then comes from the
result that we have an interpretation of type theory.
\end{comment}

Of course there is no way of studying internal models or categories without
also studying the morphisms between them. In fact, we already have three of them
right from the start: the interpretations of extension, weakening and substitution
act on contexts, families and terms and preserve each of extension, weakening and
substitution. These have to be ingredients of morphisms of models as well. A
morphism of models should act on contexts because those are the contexts; it
should act on terms because that is how it acts on the morphisms; it should
preserve extension, weakening and substitution because that is exactly what
makes it functorial (in particular: preserve composition).

In fact, now that we have arrived at the idea that categorical structure should
come from the interpretation of type theory, the prettier (and necessary) way to go is to
describe a model of models of type theory. Thus, we should find a way to express
the notions of dependent model and terms thereof and explain what extension,
weakening and substitution mean. Then we should be able to derive that a morphism
of models coincides with a term of a weakened object.

The idea that morphisms of models preserve the basic type theoretical structure
can be carried further. Ideally, we would have the interpretations of the type
constructors as morphisms of models. Thus, when interpreting the operations 
$\Pi$, $\Sigma$, $\idtypevar{}$ or the universe operator 
(such as described in \cite{Palmgren1998}) we will require that
they act not only on objects, but also on families, terms and preserve extension,
weakening and substitution. And when two or more type constructors are interpreted,
they should furthermore be compatible with each other. In the case of
dependent pair types and identity types, this means function extensionality. It
should be investigated whether the compatibility of identity types and the
universe operator with each other indeed means univalence. We should note,
however, that we intend the compatibility with extension, weakening and substitution
should be strict, whereas the compatibility with the other type constructors will
not be strict. To give an exact meaning to these ideas is part of the proposal.

Among the examples of internal models we should have:
\begin{description}
\item[The setoid model] This is a classical one so it should be there. It might
      be a bit different in our case. We'll want to construct a setoid model
      of the basic type theory to interpret identity types without necessarily
      interpreting dependent function types (but that should remain possible).
\item[The graph model] This is the model that leads to the first version of the
      descent theorem.
\item[Univalent unvirses] A univalent universe should be a model.
\item[Equifibered diagrams over a graph] should be the fibrations in some model.
\item[The model of all models] The objects in the model of models should be
      the models of type theory itself. There should be various flavors of the
      model of models. In the case of type theory without type constructors,
      this would be the model of all (small) higher categories. The model of
      models of type theory with several type constructors should interpret
      these type constructors as well. (This might get interesting in the case
      of type theory with a universe operator.)
\item[The model of weak $\omega$-groupoids] Since we think of models of the
      basic type theory as weak $\omega$-categories, it is not too hard to
      provide a condition on those models which enables us to talk about
      weak $\omega$-groupoids. In fact, we might have several options for such a
      condition. We investigate those; we also investigate how they relate to
      Brunerie's weak $\omega$-groupoids. Presumably, they form an internal
      model without too much fuss (because we already have an internal model
      of internal models at this stage) and we can ask whether this models
      the univalence axiom. In fact, the model of internal models might already
      have modeled the univalence axiom. 
\item[The set model] See \cite{RijkeSpitters:Sets}.
\item[Polynomial functors] The theory of polynomial functors is essential to the
      theory of $\tfW$-types, see 
\end{description}


\section{Stage two: extending the basic theory}\label{stage2}
With a firm notion of internal models of type theory and several extensions of
type theory come within reach. The first of these extensions, colimits for
diagrams over graphs, we have already started to explore. The operation 
assigning the colimit to a graph is compatible with $\Sigma$ and $\idtypevar{}$
types. The first was compatibility result was originally part of the descent
theorem. We take the point of view that the descent theorem -- which asserts that
for any graph $\Gamma$, the type of equifibered families over $\Gamma$ is
equivalent in a canonical way to the type of families over $\tfcolim(\Gamma)$ --
is just one of the aspects of many compatibility results for colimits. In this
formulation, the descent theorem tells that the colimit operation acts not only
on graphs (objects) but also on families (the equifibered diagrams over graphs)
of the model of equifibered families.
The colimit operation acts furthermore on terms and is compatible with extension,
weakening and substitution. The fact that it is compatible with identity types is a new result
which essentially expresses that the initial pointed equifibered diagram over
a pointed graph (which is just another way of considering the notion of universal
cover) describes
the identity type of the colimit with respect to the base point. The methods
in this proof actually describe all the higher identity types of the colimit
as a certain higher inductive type; in particular it gives a description of the
loop spaces of the higher inductive types appearing in \cite{TheBook}, so we
should figure out which algebraic data we can extract from this description.
In summary, this group of results related to the descent theorem may be viewed as the
key to a good description of the colimit operation.

These results need to be extended. In the
first place we could ask ourselves whether the colimit operation is compatible
with $\Pi$ and the universe operation. Perhaps more importantly, the descent
theorem and the related group of results stating that the colimit operation
is compatible with various type constructors needs to be generalized to (diagrams
over) arbitrary categories. Since we have a proposal for what a category is,
namely a model of type theory without basic constructors, we seem to have the
right tools at hand.  We envision that the colimit operation becomes a type theoretical operation
(together with compatibility results) assigning a type to a diagram over a category.
In the presence of a universe operator we might also consider to precompose
the colimit operation with the universe operator, so that it acts on all families.

The result should be a fairly general theory of colimits and one of the first
applications we must seek for is the establishment of $\im(f)$ as the colimit
of the appropriate category, given a function $f:A\to B$ where $A$ and $B$
are arbitrary types. In the case where $A$ and $B$ are sets, $\im(f)$ is
the set-colimit of $\sm{x,y:A}\id{f(x)}{f(y)}\rightrightarrows A$; in general
this diagram should be more complicated, see \S 6.1 of \cite{lurie2009higher}.
It should be possible to describe, given $f:A\to B$, a diagram $K_f$ such that
its colimit is $\im(f)$ generally. In particular we shouldn't need to take the
set-colimit in the case that $f$ is a function between sets.  When we have 
come this far we should also look for a possible definition of a
(weak) higher predicative topos; the ingredients seem to be there. We hope that
this gives us access to a description of sheaf models of type theory and we
can possibly begin exploring some internal independence proofs.


%\section{Empirical evidence of what an internal model should be}

\subsection{Univalent universes as internal models}

Before we give the definition we illustrate the concepts that go into it in
the case of a univalent universe $\UU$.
Regardless of the definition of an internal model, a univalent universe should
be an example of one.

The type of contexts is $\UU$ itself and for every $\Gamma:\UU$, a type in
context $\Gamma$ is simply a family $A:\Gamma\to\UU$. The type $\terms{A}$ of terms of a type $A$
in context $\Gamma$ is defined to be $\prd{i:\Gamma}A(i)$. Note that $\Gamma\to
\UU$ itself also interprets type theory where a type in context $A:\Gamma\to\UU$
is a family $P:(\sm{i:\Gamma}A(i))\to\UU$. We may denote this model by
$\mftyp{\UU}{\Gamma}$. The terms of a type $P$ in context $A$ in the model
$\Gamma\to\UU$ are the terms of $\prd{i:\Gamma}{x:A(i)}P(i,x)$.

When $A$ is a type in context
$\Gamma$ we define the context extension $\ctxext{\Gamma}A$ to be
$\sm{i:\Gamma}A(i)$. Note that context extension $\ctxext{\Gamma}\blank$
can be seen acting not only as a function from the context of the model
$\Gamma\to\UU$ to the context of the model $\UU$, but it also acts on the types 
and terms: when $A$ is a type in context $\Gamma$ we may take the identity map
from $\mftyp{\Gamma\to\UU}{A}\to\mftyp{\UU}{\ctxext{\Gamma}A}$ because
$\mftyp{\Gamma\to\UU}{A}$ is taken to be $\mftyp{\UU}{\ctxext{\Gamma}A}$;
thus, $\ctxext{\Gamma}P\defeq P$ for every family $P$ over $A$ in context 
$\Gamma$. Likewise, when $P$ is a type in context $A$ in the model 
$\Gamma\to\UU$ then context extension should act on the terms of $P$ via a
function $\terms{P}\to\terms{\ctxext{\Gamma}P}$, which we take to be
the identity map once more.

Since $\mftyp{\UU}{\Gamma}$ is a model of type theory it
has it's own notion of context extension: when $P$ is a type in context $A$ in
the model $\mftyp{\UU}{\Gamma}$ then $\ctxext{A}P$ is the family
$\lam{i}\sm{x:A(i)}P(i,x):\Gamma\to\UU$. Also the context extension of
$\Gamma\to\UU$ acts trivially on the types and terms. Context extension is
analoguous to the Grothendieck construction that associates the category of
elements to a presheaf and it gives us $\Sigma$-types.

When $A$ and $B$ are types in context
$\Gamma$, the weakening $\ctxwk{A}{B}$ of $B$ along $A$ is defined to be
$\lam{\pairr{i,x}}B(i):(\sm{i:\Gamma}A(i))\to\UU$. Weakening along a type $A$ 
in context $\Gamma$ also acts on types and terms. When $Q$ is a type in context
$B$ in the model $\Gamma\to\UU$, we define $\ctxwk{A}{Q}$ to be 
$\lam{\pairr{i,x}}{y}Q(i,y)$. When $g:\terms{Q}$ we define $\ctxwk{A}{g}$ to
be $\lam{\pairr{i,x}}{y}g(i,y)$.

Note that for two types $A$ and $B$ in context $\Gamma$, the terms of
$\ctxwk{A}{B}$ are the terms of $\prd{i:\Gamma}A(i)\to B(i)$, i.e.~they are
the fiberwise maps from $A$ to $B$. We shall take the type $\terms{\ctxwk{A}{B}}$
of terms of $\ctxwk{A}{B}$ to be the type of morphisms from $A$ to $B$. Also,
any context of $\UU$ may be seen as a type in the empty context $\unit$. Thus
the type of terms of a context $\Gamma$ is $\unit\to\Gamma$, which is
equivalent to $\Gamma$. A context morphism from $\Delta$ to $\Gamma$ is a term
of $\ctxwk{\Gamma}{\Delta}$, i.e.~a function from $\Gamma$ to $\Delta$. We denote
the type $\terms{\ctxwk{\Gamma}{\Delta}}$ by $\ctxhom{\Delta}{\Gamma}$. 

When $P$ is a family over
$A$ in context $\Gamma$ and $x$ is a term of $A$ we define the type $\subst{x}{P}$
in context $\Gamma$ to be $\lam{i}P(i,x(i))$. Like extension 

We will interpret the dependent product $\mprd{A}{P}$ of a family $P$ over
$A$ in context $\Gamma$ by
\begin{equation*}
\mprd{A}{P}(i)\defeq \prd{x:A(i)}P(i,x)
\end{equation*}
With this interpretation there is an equivalence 
$\lambda:\eqv{\terms{P}}{\terms{\mprd{A}{P}}}$. This is lambda-abstraction; its
inverse being evaluation. We should note however, that the rule for evaluation
we interpret here is
\begin{equation*}
\inference{\Gamma\vdash f:\mprd{A}{P}}{\ctxext{\Gamma}A\vdash\tfev(f):P}
\end{equation*}
which is different than the usual rule
\begin{equation*}
\inference{\Gamma\vdash f:\mprd{A}{P} \qquad \Gamma\vdash a:A}{\Gamma \vdash \tfev(f,a):\subst{a}{P}}
\end{equation*}
The reason for this is that the interpretation of the usual rule would give a
function of type $\terms{\mprd{A}{P}}\to\prd{x:\terms{A}}\terms{\subst{x}{P}}$,
but this does not describe the terms of $\mprd{A}{P}$ by any means. Moreover,
since we have implemented substitution, we obtain from $\ctxext{\Gamma}A\vdash\tfev(f):P$ 
and $\Gamma\vdash x:A$ a term $\Gamma\vdash\subst{x}{\tfev(f)}:\subst{x}{P}$ and
therefore we do not loose anything with this approach.

{\color{red} point to example}

\subsection{The Sierpinsky model}.
In the Sierpinsky model, contexts are interpreted by dependent types $A\to\type$.

\subsection{The graph model of type theory}
In this section we define the graph model of type theory, denoted by
$\mfGraph$.  In our presentation, we follow that of the definition internal
models. After we have established the graph model, we show how the graph
model can be seen as a presheaf model, namely over the category $\cdot
{\rightrightarrows}\cdot$.

\begin{defn}
A \emph{(directed) graph} $\Gamma$ is a pair $\pairr{\Gamma_0,\Gamma_1}$ 
consisting of a type $\Gamma_0$ of vertices and a family 
$\Gamma_1:\Gamma_0\to\Gamma_0\to\type$ of edges. The type $\ctx(\mfGraph)$
is defined to be the type of all graphs; we will usually denote it by
$\tfGraph$. Explicitly, we have
\begin{equation*}
\tfGraph\defeq\sm{\Gamma_0:\type}\Gamma_0\to\Gamma_0\to\type.
\end{equation*}
\end{defn}

\begin{eg}\label{ex:pb}
The underlying graph of the diagram
\begin{equation*}
\begin{tikzcd}
{} & A \ar{d}{f} \\
B \ar{r}[swap]{g} & C
\end{tikzcd}
\begin{comment}
\begin{tikzpicture}
\matrix (m) [std] { & A \\ B & C \\};
\draw[ar] (m-1-2) -- node[right] {$f$} (m-2-2);
\draw[ar] (m-2-1) -- node[below] {$g$} (m-2-2);
\end{tikzpicture}
\end{comment}
\end{equation*}
is $I\defeq\mathbf{3}$ and has $J(1,3)\defeq J(2,3)\defeq \unit $ 
and $J(x,y)\defeq\emptyt$ otherwise (it is defined using the induction 
principle of $\mathbf{3}$ and a universe).
\end{eg}


\begin{defn}
A \emph{graph $A$ in the context $\Gamma$} is a pair $\pairr{A_0,A_1}$ consisting
of 
\begin{align*}
A_0 & :\Gamma_0\to\type\\
A_1 & :\prd*{i,j:\Gamma_0}\Gamma_1(i,j)\to A_0(i)\to A_0(j)\to\type.
\end{align*}
Thus, for a graph $\Gamma$, the type $\mftyp{\mfGraph}{\Gamma}$ is the type
\begin{equation*}
\sm{A_0:\Gamma_0\to\type}\prd*{i,j:\Gamma_0}\Gamma_1(i,j)\to A_0(i)\to A_0(j)\to\type.
\end{equation*}
We will also write $\Gamma\vdash A:\mfGraph$ when $A$ is a graph in context
$\Gamma$.
\end{defn}

\begin{defn}
Suppose that $\Gamma\vdash A:\mfGraph$. Then we define the graph $\ctxext{\Gamma}A$
by
\begin{align*}
\ctxext{\Gamma}A_0 & \defeq \sm{i:\Gamma_0}A_0(i)\\
\ctxext{\Gamma}A_1(\pairr{i,x},\pairr{j,y}) & \defeq \sm{q:\Gamma_1(i,j)}A_1(q,x,y).
\end{align*}
\end{defn}

\begin{defn}
Suppose that $\Gamma\vdash A:\mfGraph$ and $\Gamma\vdash B:\mfGraph$. Then we
define the graph $\ctxwk{A}{B}$ in context $\ctxext{\Gamma}A$ by
\begin{align*}
(\ctxwk{A}{B})_0(\pairr{i,x}) & \defeq B_0(i)\\
(\ctxwk{A}{B})_1(\pairr{q,e},u,v) & \defeq B_1(q,u,v).
\end{align*}
\end{defn}

\begin{defn}
Suppose that $\Gamma\vdash A:\mfGraph$. A term $x$ of $A$ consists of a pair
$\pairr{x_0,x_1}$ where
\begin{align*}
x_0 & : \prd{i:\Gamma_0}A_0(i)\\
x_1 & : \prd*{i,j:\Gamma_0}{q:\Gamma_1}A_1(q,x_0(i),x_0(j)).
\end{align*}
Thus we define
\begin{equation*}
\terms{A}\defeq\sm{x_0:\prd{i:\Gamma_0}A_0(i)}\prd*{i,j:\Gamma_0}{q:\Gamma_1}A_1(q,x_0(i),x_0(j)).
\end{equation*}
We also write $\Gamma\vdash x:A$ when $x$ is a term of the graph $A$ in context
$\Gamma$.
\end{defn}

\begin{defn}
Suppose that $P$ is a family of graphs over $A$ in context $\Gamma$ and let
$x$ be a term of $A$. Then we define the graph $\subst{x}{P}$ 
in context $\Gamma$ by
\begin{align*}
\subst{x}{P}_0(i) & \defeq P_0(\pairr{i,x_0(i)})\\
\subst{x}{P}_1(q,u,v) & \defeq P_1(\pairr{q,x_1(q)},u,v)
\end{align*}
\end{defn}

\begin{rmk}
Note that we have the judgmental equality $\subst{x}{\ctxwk{A}{B}}\jdeq B$
for every two graphs $A$ and $B$ in context $\Gamma$ and every term $x$ of $A$.
\end{rmk}

\begin{defn}
The terminal graph $\unit^\mfGraph$, which we shall often denote simply 
by $\unit$, is defined by
\begin{align*}
{\unit^\mfGraph}_0 & \defeq \unit\\
{\unit^\mfGraph}_1(x,y) & \defeq \unit.
\end{align*}
\end{defn}

\begin{rmk}
We note that the function $\ctxext{\unit}\blank$ is an equivalence
from $\mftyp{\mfGraph}{\unit}$ to $\tfGraph$. It's inverse is the function which maps
a graph $\Gamma$ to the pair $\pairr{A_0,A_1}$ where $A_0$ is defined by
$A_0(\ttt)\defeq\Gamma_0$ and where $A_1$ is defined by $A_1(\ttt)\defeq
\Gamma_1$. 

Note that we only have an equivalence here, not a judgmental equality.
\end{rmk}

\subsubsection{The contexts of $\mfGraph$}


The category $\psh(\cdot{\rightrightarrows}\cdot)$ of presheaves over
$\cdot{\rightrightarrows}\cdot$ is given by
\begin{equation*}
\sm{A_0,A_1:\type}(A_1\to A_0)\times(A_1\to A_0).
\end{equation*}
To see that $\psh(\cdot{\rightrightarrows}\cdot)$ is indeed equivalent to
the type $\tfGraph$ of all graphs, note that we have the equivalences
\begin{align*}
\psh(\cdot{\rightrightarrows}\cdot) & \eqvsym \sm{A_0,A_1:\type}A_1\to A_0\times A_0\\
& \eqvsym \sm{A_0:\type} A_0\times A_0\to\type\\
& \eqvsym \tfGraph.
\end{align*}

\begin{defn}
Suppose that $\Delta\defeq \pairr{\Delta_0,\Delta_1}$ and $\Gamma\defeq \pairr{\Gamma_0,
\Gamma_1}$ are graphs. A morphism $f$ from $\Delta$ to $\Gamma$ is a pair
$\pairr{f_0,f_1}$ consisting of
\begin{align*}
f_0 & : \Delta_0\to\Gamma_0\\
f_1 & : \prd*{u,v:\Delta_0}\Delta_1(u,v)\to\Gamma_1(f_0(u),f_0(v))
\end{align*}
We write $\ctxhom{\Delta}{\Gamma}$ for the type of morphisms of graphs.
\end{defn}

The terminal graph $\unit^\mfGraph$ plays the role of the empty context.

\begin{defn}
The terminal graph $\unit^\mfGraph$, which we shall often denote simply 
by $\unit$, is defined by
\begin{align*}
{\unit^\mfGraph}_0 & \defeq \unit\\
{\unit^\mfGraph}_1(x,y) & \defeq \unit.
\end{align*}
\end{defn}

It is immediate that the type of graph morphisms $\Gamma\to\unit^\mfGraph$ is
contractible for every graph $\Gamma$. The terms of a graph $\Gamma$ are the
global elements of $\Gamma$, i.e.~they are the morphisms of type 
$\ctxhom{\unit^\mfGraph}{\Gamma}$.

\begin{defn}
Let $\Gamma$ be a graph. The type $\terms[\mfGraph]{\Gamma}$ of terms of $\Gamma$ is
defined to be
\begin{equation*}
\sm{x_0:\Gamma_0}\Gamma_1(x_0,x_0).
\end{equation*}
\end{defn}

We now turn to the description of the universe $\gobjclass$ of graphs. Note
that for the category $I\defeq 0{\rightrightarrows}1$, we have
$I/0\defeq\catid{0}$ and $I/1\defeq s\rightarrow\catid{1}\leftarrow t$, where
$s$ and $t$ are the morphisms $0\to 1$. Thus we have 
\begin{align*}
\psh(I/0) & \eqvsym \type\\
\psh(I/1) & \eqvsym \sm{X,Y:\type}X\to Y\to\type.
\end{align*}
The functors $\psh(\Sigma_s)$ and $\psh(\Sigma_t)$ map a presheaf 
$\pairr{X,Y,R}$ to $X$ and $Y$, respectively. 
Thus we obtain the following graph $\gobjclass$.

\begin{defn}
The universe $\gobjclass$ of graphs is defined to be
\begin{align*}
{\gobjclass}_0 & \defeq \type\\
{\gobjclass}_1(X,Y) & \defeq X\to Y\to\type.
\end{align*}
\end{defn}

Note that the type $\terms{\gobjclass}$ of terms of $\gobjclass$ is 
exactly $\tfGraph$. 

\subsubsection{Families of graphs}
Using $\gobjclass$, we obtain a description of what
it means to be a graph in a context $\Gamma$, where $\Gamma$ is itself a
graph. The notion of ``a graph in a context $\Gamma$'' is the interpretation
of ``a type in a context $\Gamma$''.

\begin{defn}
A \emph{graph $A$ in the context $\Gamma$} is a pair $\pairr{A_0,A_1}$ consisting
of 
\begin{align*}
A_0 & :\Gamma_0\to\type\\
A_1 & :\prd*{i,j:\Gamma_0}\Gamma_1(i,j)\to A_0(i)\to A_0(j)\to\type.
\end{align*}
We will also write $\Gamma\vdash A:\mfGraph$ when $A$ is a graph in context
$\Gamma$.
\end{defn}

\begin{rmk}
Note that every graph can be seen as a graph in context $\unit^\mfGraph$.
\end{rmk}

\begin{defn}
Let $A$ be a graph in context $\Gamma$ and let $f:\ctxhom{\Delta}{\Gamma}$
be a graph morphism. We define the graph $A[f]$ in context $\Delta$ by
\begin{align*}
A[f]_0(u) & \defeq A_0(f_0(u))\\
A[f]_1(p,x,y) & \defeq A_1(f_1(p),x,y).
\end{align*}
\end{defn}

\begin{defn}
Let $\Gamma\vdash A:\mfGraph$ and $\Gamma\vdash B:\mfGraph$ be two graphs in
the context $\Gamma$. A morphism $f$ from $A$ to $B$ is a pair $\pairr{f_0,f_1}$
consisting of
\begin{align*}
f_0 & : \prd*{i:\Gamma_0} A_0(i)\to B_0(i)\\
f_1 & : \prd*{i,j:\Gamma_0}{q:\Gamma_0}*{x:A_0(i)}*{y:A_0{j}}A_1(q,x,y)\to B_1(q,f_0(x),f_0(y)).
\end{align*}
We write $\ctxhom[\Gamma]{A}{B}$ for the type of morphisms between graphs in
context $\Gamma$.
\end{defn}

\begin{defn}
The unit graph $\unit^\Gamma$ in context $\Gamma$ consists of
\begin{align*}
(\unit^\Gamma)_0 & \defeq \lam{i}\unit\\
(\unit^\Gamma)_1(q) & \defeq \lam{x}{y}\unit.
\end{align*}
\end{defn}

As was the case with graphs, terms of graphs in a context are described by
morphisms from $\unit$. Explicating this in more detail, we get

\begin{defn}
Let $A$ be a graph in context $\Gamma$. A term $x$ of $A$ is a pair
$\pairr{x_0,x_1}$ consisting of
\begin{align*}
x_0 & : \prd{i:\Gamma_0}A_0(i)\\
x_1 & : \prd*{i,j:\Gamma_0}{q:\Gamma_1(i,j)}A_1(q,x_0(i),x_0(j)).
\end{align*}
We define $\terms[\mfGraph]{A}$ to be the type of terms of $A$.
\end{defn}

\begin{rmk}
Note that the type of terms of a graph $\Gamma$ is equivalent to the type
of terms of the graph $\Gamma$ in context $\unit^\mfGraph$.
\end{rmk}

\begin{defn}
Suppose that $\Gamma\vdash A:\mfGraph$. Then we define the graph $\ctxext{\Gamma}A$
by
\begin{align*}
(\ctxext{\Gamma}A)_0 & \defeq \sm{i:\Gamma_0}A_0(i)\\
(\ctxext{\Gamma}A)_1(\pairr{i,x},\pairr{j,y}) & \defeq \sm{q:\Gamma_1(i,j)}A_1(q,x,y).
\end{align*}
There is a morphism $\pi^A:\ctxhom{\ctxext{\Gamma}A,\Gamma}$ given by
\begin{align*}
\pi^A_0 & \defeq \lam{\pairr{i,x}}i && : (\sm{i:\Gamma_0}A_0(i))\to \Gamma_0\\
\pi^A_1(\pairr{i,x},\pairr{j,y}) & \defeq \lam{\pairr{q,e}}q && : (\sm{q:\Gamma_1(i,j)}A_1(q,x,y))\to\Gamma_1(i,j).
\end{align*}
\end{defn}

\begin{rmk}
The construction of $\ctxext{\Gamma}A$ is similar to the Grothendieck
construction of the category of elements for a presheaf.
\end{rmk}

\begin{defn}
Let $\Gamma\vdash A:\mfGraph$. A family $P$ of graphs over $A$ is by
definition a graph $P$ in the context $\ctxext{\Gamma}A$. More explicitly, a family
$P$ of graphs over $A$ in context $\Gamma$ consists of
\begin{align*}
P_0 & : \prd*{i:\Gamma_0}A_0(i)\to\type\\
P_1 & : \prd*{i,j:\Gamma_0}{q:\Gamma_1(i,j)}*{x:A_0(i)}*{y:A_0(j)} A_1(q,x,y)\to P_0(x)\to P_0(y)\to\type
%P_1 & : \prd{\pairr{i,x},\pairr{j,y}:(\ctxext{\Gamma}A)_0}(\ctxext{\Gamma}A)_1(\pairr{i,x},\pairr{j,y})\to P_0(i,x)\to P_0(j,y)\to\type.
\end{align*}
\end{defn}

\begin{defn}
When $P$ is a family of graphs over $A$ in context $\Gamma$ and $x:A$, we
get a graph $\subst{x}{P}$ in context $\Gamma$ given by
\begin{align*}
\subst{x}{P}_0(i) & \defeq P_0(x_0(i))\\
\subst{x}{P}_1(q,u,v) & \defeq P_1(\pairr{q,x_1(q)},u,v)
\end{align*}
\end{defn}

\subsubsection{The basic type constructors for graphs}
With families of graphs being available, we can give the interpretations of
dependent products, dependent sums and identity types. In the following, we
shall introduce the graph interpretations of dependent products, dependent
sums and identity types and describe the terms of the resulting graphs.\note{These definitions
should be connected to Mike's article but I don't really know how to do this}

\begin{defn}
Let $P$ be a family of graphs over $A$, where $\Gamma\vdash A:\mfGraph$. 
The dependent function graph $\mprd{A}{P}$ in context $\Gamma$ consists of
\begin{align*}
\mprd{A}{P}_0(i) & \defeq \prd{x:A_0(i)}P_0(x)\\
\mprd{A}{P}_1(q,f,g) & \defeq \prd*{x:A_0(i)}*{y:A_0(j)}{e:A_1(q,x,y)}P_1(\pairr{q,e},f(x),g(y)).
\end{align*}
\end{defn}

\begin{rmk}
A term $f:\mprd{A}{P}$ consists of
\begin{align*}
f_0 & : \prd*{i:\Gamma_0}{x:A_0(i)}P_0(x)\\
f_1 & : \prd*{i,j:\Gamma_0}{q:\Gamma_1(i,j)}*{x:A_0(i)}*{y:A_0(j)}{e:A_1(q,x,y)}P_1(\pairr{q,e},f_0(x),f_0(y))
\end{align*}
Therefore, we see that $\eqv{\terms{\mprd{A}{P}}}{\terms{P}}$. \emph{Warning:} it
is by no means the case that $\eqv{\terms{\mprd{A}{P}}}{\prd{x:\terms{A}}
\terms{\subst{x}{P}}}$ for all families $P$ over $A$. For instance, the graph
$\tilde{\emptyt}$ defined by $\tilde{\emptyt}_0\defeq\unit$ and 
$\tilde{\emptyt}_1(\ttt,\ttt)\defeq\emptyt$ has no terms and neither does
$\tilde{\emptyt}+\tilde{\emptyt}$. Nevertheless, there are two graph morphisms
from $\tilde{\emptyt}$ to $\tilde{\emptyt}+\tilde{\emptyt}$. More generally,
when $\Gamma$ is a graph such that $\Gamma_1(i,j)\jdeq\emptyt$ for all $i,j:\Gamma_0$,
then $\eqv{{\terms{\Gamma\to\Gamma'}}}{\Gamma_0\to\Gamma^\prime_0}$.
\end{rmk}

\begin{defn}
If $P$ is a family of graphs over $\Gamma$, the dependent pair graph
$\msm{A}{P}$ consists of
\begin{align*}
\msm{A}{P}_0(i) & \defeq \sm{x:A_0(i)}P_0(x)\\
\msm{A}{P}_1(q,\pairr{x,u},\pairr{y,v}) & \defeq \sm{e:A_1(q,x,y)}P_1(\pairr{q,e},u,v).
\end{align*}
\end{defn}

\begin{rmk}
A term $w:\msm{A}{P}$ consists of
\begin{align*}
w_0 & : \prd{i:\Gamma_0}\sm{x:A_0(i)}P_0(x)
\intertext{and, writing $\lam{i}\proj1(w_0(i))$ and $\lam{i}\proj2(w_0(i))$ as
$w_{00}$ and $w_{01}$ respectively,}
w_1 & : \prd*{i,j:\Gamma_0}{q:\Gamma_1(i,j)}\sm{e:A_1(q,w_{00}(i),w_{00}(j)}P_1(\pairr{q,e},w_{01}(i),w_{01}(j)).
\end{align*}
By $\choice{\infty}$ it follows that
\begin{equation*}
\eqv{\terms{\msm{A}{P}}}{\sm{x:\terms{A}}\terms{\subst{x}{P}}}.
\end{equation*}
\end{rmk}

\begin{rmk}
For any two graphs $A$ and $B$ in a context $\Gamma$, we have the graph
$B[\pi^A]$ in the context $\ctxext{\Gamma}A$, which we may just write
as $B$. Therefore, we can
consider the graphs $\mprd{A}{B}$ and $\msm{A}{B}$. As usually those shall be
denoted  by $A\to B$ and $A\times B$, respectively. 

A term $f$ of $A\to B$ consists of
\begin{align*}
f_0 & : \prd*{i:\Gamma_0}A_0(i)\to B_0(i)\\
f_1 & : \prd*{i,j:\Gamma_0}{q:\Gamma_1(i,j)}*{x:A_0(i)}*{y:A_0(j)}A_1(q,x,y)\to B_1(q,f_0(x),f_0(y)),
\end{align*}
so we see that $\terms{A\to B}\jdeq\ctxhom[\Gamma]{A}{B}\jdeq\terms{B[\pi^A]}$. 

A term $\pairr{x,y}$ of $A\times B$ consists of
\begin{align*}
\pairr{x,y}_0 & : \prd*{i:\Gamma_0} A_0(i)\times B_0(i)\\
\pairr{x,y}_1 & : \prd*{i,j:\Gamma_0}{q:\Gamma_1(i,j)} A_1(q,x_0(i),x_0(j))\times B_1(q,y_0(i),y_0(j))
\end{align*}
so we see that $\eqv{\terms{A\times B}}{\terms{A}\times\terms{B}}$. 

Note that we also obtain
a \emph{graph} $\Delta\to\Gamma$ this way, for every two graphs $\Delta$ and
$\Gamma$ seen as graphs in the context $\unit^\mfGraph$. 
The type $\terms{\Delta\to\Gamma}$ of terms of $\Delta\to\Gamma$ is just
$\ctxhom{\Delta}{\Gamma}$.
\end{rmk}

\begin{defn}
For any graph $\Gamma$ there is a graph $\tffam{\Gamma}$ in context 
$\gobjclass$ defined by
\begin{align*}
(\tffam{\Gamma})_0 & \defeq \lam{X}X\to\Gamma_0\\
(\tffam{\Gamma})_1(X,Y,R) & \defeq \lam{f}{g}\prd*{x:X}*{y:Y}R(x,y)\to\Gamma_1(f(x),g(y)).
\end{align*}
\end{defn}

\begin{rmk}
A term $D$ of $\msm{\gobjclass}{\tffam{\Gamma}}$ consists of a term
$\pairr{\Delta_0,f_0}$ of type
\begin{equation*}
\msm{\gobjclass}{\tffam{\Gamma}}_0\jdeq \sm{\Delta_0:\type}\Delta_0\to\Gamma_0
\end{equation*}
and a term $\pairr{\Delta_1,f_1}$ of type
\begin{equation*}
\msm{\gobjclass}{\tffam{\Gamma}}_1(q)\jdeq \sm{\Delta_1}
\end{equation*}
\end{rmk}

\begin{defn}
There is a graph morphism
\begin{equation*}
\graphcharmapfunc{\Gamma} : \sm{\gobjclass}
\end{equation*}
\end{defn}

\begin{defn}
Let $A$ be a graph in context $\Gamma$. We define the family $\idtypevar{A}$ over $\ctxwk{A}{A}$ in
context $\ctxext{\Gamma}A$ by
\begin{align*}
(\idtypevar{A})_0(\pairr{i,x},y) & \defeq \id{x}{y}\\
(\idtypevar{A}){}_1(\pairr{q,e},d,\alpha,\alpha') & \defeq \id{\trans{\pairr{\alpha,\alpha'}}{e}}{d}
\end{align*}
where $q:\Gamma_1(i,j)$, $e:A_1(q,x,x')$, $d:A_1(q,y,y')$, $\alpha:\id{x}{y}$
and $\alpha':\id{x'}{y'}$. The transportation along the path 
$\pairr{\alpha,\alpha'}:\id{\pairr{x,x'}}{\pairr{y,y'}}$ in $A_0(i)\times A_0(j)$
is taken with respect to the family $\lam{x}{x'}A_1(q,x,x')$.

We define the term $\refl{A}$ of the family 
$\subst{\idfunc[A]}{\idtypevar{A}}$ over $A$ in context $\Gamma$ by
\begin{align*}
(\reflf{A})_0(i) & \defeq \lam{x}\refl{x}\\
(\reflf{A})_1(q) & \defeq \lam{e}\refl{e}
\end{align*}
\end{defn}

\begin{defn}
Let $D$ be a family over $\idtypevar{A}$ in context 
$\ctxext({\Gamma}{A})\ctxwk{A}{A}$. Then we have the family
$\subst{\idfunc[A]}{D}$ over $\subst{\idfunc[A]}{\idtypevar{A}}$ in
context $\ctxext{\Gamma}A$ given by
\begin{align*}
\subst{\idfunc[A]}{D}_0(\pairr{i,x},\alpha) & \defeq D_0(\pairr{i,x,x},\alpha)\\
\subst{\idfunc[A]}{D}_1(\pairr{q,e},\gamma) & \defeq D_1(\pairr{q,e,e},\gamma).
\end{align*}
The family $\subst{\reflf{A}}{\subst{\idfunc[A]}{D}}$ over $A$ in
context $\Gamma$ is given by
\begin{align*}
\subst{\reflf{A}}{\subst{\idfunc[A]}{D}}_0(i,x) & \defeq D_0(i,x,x,\refl{x})\\
\subst{\reflf{A}}{\subst{\idfunc[A]}{D}}_1(q,e) & \defeq D_1(q,e,e,\refl{e}).
\end{align*}
To show that the identity graphs correctly interpret the identity elimination
rule, we must give a function
\begin{equation*}
\tfJ : \terms{\subst{\reflf{A}}{\subst{\idfunc[A]}{D}}}\to\terms{D}.
\end{equation*}
Note that a term $d$ of $\subst{\reflf{A}}{\subst{\idfunc[A]}{D}}$
consists of
\begin{align*}
d_0 & : \prd*{i:\Gamma_0}{x:A_0(i)}D_0(i,x,x,\refl{x})\\
d_1 & : \prd*{i,j:\Gamma_0}{q:\Gamma_1(i,j)}*{x:A_0(i)}*{y:A_0(j)}{e:A_1(q,x,y)}D_1(q,e,e,\refl{e})
\end{align*}
A simple argument using path induction reveals that terms of
$\subst{\reflf{A}}{\subst{\idfunc[A]}{D}}$ indeed yield terms of $D$. 
\end{defn}

\begin{rmk}
Using the identity graph $\idtypevar{A}$ we can describe the identity
graph $\id[A]{x}{y}$ in context $\Gamma$ for any two terms $x,y:A$. 
The graph $\id[A]{x}{y}$ in context $\Gamma$ consists of
\begin{align*}
(\id[A]{x}{y})_0(i) & \defeq \id{x_0(i)}{y_0(i)}\\
(\id[A]{x}{y})_1(q,\alpha,\beta) & \defeq \id{\trans{\pairr{\alpha,\beta}}{x_1(q)}}{y_1(q)}
\end{align*}
From this, we see that a term of $p:\id[A]{x}{y}$ consists of
\begin{align*}
p_0 & : \prd{i:\Gamma_0}\id{x_0(i)}{y_0(i)}\\
p_1 & : \prd*{i,j:\Gamma_0}{q:\Gamma_1(i,j)}\id{\trans{\pairr{p_0(i),p_0(j)}}{x_1(q)}}{y_1(q)}.
\end{align*}
\end{rmk}

\subsubsection{Properties of graphs}

\subsubsection{Contractibility and equivalences of graphs}

\begin{defn}
A graph $A$ in context $\Gamma$ is said to be \emph{contractible} if there
is a term of the graph
\begin{equation*}
\msm{A}{\mprd{\ctxwk{A}{A}}{\idtypevar{A}}}
\end{equation*}
in context $\Gamma$.
\end{defn}

\begin{lem}\label{lem:contractible-graphs}
Let $A$ be a graph in context $\Gamma$. The following are equivalent:
\begin{enumerate}
\item $A$ is a contractible graph.
\item Both $A_0(i)$ and $A_1(q,x,y)$ are always contractible.
\end{enumerate}
\end{lem}

\begin{proof}
Let $H:\msm{A}{\mprd{\ctxwk{A}{A}}{\idtypevar{A}}}$. Unfolding the definitions, we have
an element $H_0(i)$ of type
\begin{align*}
\msm{A}{\mprd{\ctxwk{A}{A}}{\idtypevar{A}}}_0(i) & \jdeq \sm{x:A_0(i)}\mprd{\ctxwk{A}{A}}{\idtypevar{A}}_0(i,x)\\
& \jdeq \sm{x:A_0(i)}\prd{y:A_0(i)}(\idtypevar{A})_0(\pairr{i,x},y)\\
& \jdeq \sm{x:A_0(i)}\prd{y:A_0(i)}\id{x}{y}
\end{align*}
for all $i:\Gamma_0$ and, writing $H_{00}(i)$ for $\proj1 H_0(i)$ and
$H_{01}(i)$ for $\lam{y}(\proj2 H_0(i))(y)$, we have $H_1(q)$ of type
\begin{align*}
& \msm{A}{\mprd{\ctxwk{A}{A}}{\idtypevar{A}}}_1(q,H_0(i),H_0(j)) \\
& \jdeq \sm{e:A_1(q,H_{00}(i),H_{00}(j))}\mprd{\ctxwk{A}{A}}{\idtypevar{A}}_1(\pairr{q,e},H_{01}(i),H_{01}(j))\\
& \jdeq \sm{e:A_1(q,H_{00}(i),H_{00}(j))}\prd*{x:A_0(i)}*{y:A_0(j)}{d:A_1(q,x,y)}(\idtypevar{A})_1(\pairr{q,e},d,H_{01}(x),H_{01}(y))\\
& \jdeq \sm{e:A_1(q,H_{00}(i),H_{00}(j))}\prd*{x:A_0(i)}*{y:A_0(j)}{d:A_1(q,x,y)}\id{\trans{\pairr{H_{01}(x),H_{01}(y)}}{e}}{d}.
\end{align*}
By $H_0$, it follows that each $A_0(i)$ is contractible. By the contractibility
of each $A_0(i)$, it follows that the type of $H_1(q)$ is equivalent to
\begin{equation*}
\sm{e:A_1(q,H_{00}(i),H_{00}(j))}\prd{d:A_1(q,H_{00}(i),H_{00}(j))}\id{e}{d}
\end{equation*}
which asserts that $A_1(q,H_{00}(i),H_{00}(j))$ is contractible. By 
the contractibility of each $A_0(i)$, is is equivalent to the assertion
that each $A_1(q,x,y)$ is contractible.
\end{proof}

\begin{rmk}
We address the question whether it is the case that a graph $A$ in context
$\Gamma$ is contractible if and only if $\terms{A}$ is contractible. As a
consequence of \autoref{lem:contractible-graphs}, it is indeed the case
that $\terms{A}$ is contractible whenever $A$ is. However, the converse
does not hold.

To see this, we first construct a counter example to the converse of the
weak function extensionality principle, which states that there is a function
of type
\begin{equation}
\iscontr(\prd{x:X}P(x))\to\prd{x:X}\iscontr(P(x))\label{eq:wfe-converse}
\end{equation}
for any type family $P:X\to\type$. In the proof of \autoref{thm:wfe-converse}, 
we will find a family
$P:X\to\type$ with the property that $\prd{x:X}P(x)$ is contractible and for
which there is a term $x:A$ with $P(x)$ not contractible. Disproving the
converse of the weak function extensionality principle suffices for our
purposes, because if $P:X\to\type$ is a counter example to \autoref{eq:wfe-converse},
then we can take $\Gamma\defeq\pairr{\Gamma_0,\Gamma_1}$ to be given by
$\Gamma_0\defeq X$ and $\Gamma_1(i,j)\defeq\emptyt$ and we take
$A\defeq\pairr{A_0,A_1}$ to be given by $A_0\defeq P$ and $A_1(q,u,v)
\defeq\emptyt$.
\end{rmk}

We define the family $\mathcal{T}:\Sn^1\to\UU$ by
We define $\mathcal{T}(\base)\defeq\mathbf{3}$. To define $\mathcal{T}(\lloop):
\id{\mathbf{3}}{\mathbf{3}}$ we apply the univalence axiom. Hence it suffices to find an
equivalence $\eqv{\mathbf{3}}{\mathbf{3}}$, for which we take the function
$e$ defined by
\begin{equation*}
e(x)\defeq\begin{cases}
0_\mathbf{3} & \text{if }x\jdeq 0_\mathbf{3}\\
2_\mathbf{3} & \text{if }x\jdeq 1_\mathbf{3}\\
1_\mathbf{3} & \text{if }x\jdeq 2_\mathbf{3}.
\end{cases}
\end{equation*}

\begin{lem}
The type $\terms{\mathcal{T}}\defeq\prd{x:\Sn^1}\mathcal{T}(x)$ is contractible.
\end{lem}

\begin{proof}
The type of sections of $\mathcal{T}$ is equivalent to $\sm{u:\mathcal{T}(\base)}\id{e(u)}{u}$.
If we have a term $\pairr{u,\alpha}$ of the latter type, it follows by induction
on $\mathbf{3}$ that $\id{\pairr{u,\alpha}}
{\pairr{0_\mathbf{3},\refl{0_\mathbf{3}}}}$ for all $\pairr{u,\alpha}:
\sm{x:\mathcal{T}(\base)}\id{e(u)}{u}$,
which shows that the type of sections of $\mathcal{T}$ is contractible.
\end{proof}

\begin{thm}\label{thm:wfe-converse}
There is a type family $P:A\to\type$ for which
\begin{equation*}
\neg\Big(\iscontr(\prd{x:A}P(x))\to\prd{x:A}\iscontr(P(x))\Big).
\end{equation*}
\end{thm}

\begin{proof}
The type family of our counter example is $\mathcal{T}$: the fiber $\mathcal{T}(\base)$ isn't contractible.
\end{proof}

\begin{defn}
A graph morphism $f:\ctxhom{\Delta}{\Gamma}$ is an equivalence of graphs when
$\graphcharmap[\Gamma]{f}$ is a contractible graph in the context $\Gamma$.
\end{defn}

\begin{lem}
Let $A$ and $B$ be graphs in a context $\Gamma$ and let $f:A\to B$. The following are equivalent:
\begin{enumerate}
\item $f[i]:A[i]\to B[i]$ is an equivalence of graphs for every term $i:\Gamma$.
\item $\ctxext{\Gamma}f:\ctxext{\Gamma}A\to\ctxext{\Gamma}B$ is an equivalence of graphs.
\item Both $f_0(i)$ and $f_1(q,x,y)$ are always equivalences.
\item $\terms{f}:\terms{\Delta}\to\terms{\Gamma}$ is an equivalence.
\end{enumerate}
\end{lem}

\begin{rmk}
It follows that there is an equivalence
\begin{equation*}
\ctxext{\Gamma}\msm{A}{P}\simeq\ctxext({\Gamma}{A})P
\end{equation*}
for every family $P$ of graphs over a graph $A$ in context $\Gamma$.
\end{rmk}

\subsubsection{Homotopy levels}
\begin{itemize}
\item A graph $A$ in context $\Gamma$ is of homotopy level $n$ precisely when each
$A_0(i)$ and each $A_1(q,x,y)$ are of homotopy level $n$. 
\item We can name at least three different propositions in the empty context:
\begin{enumerate}
\item $\Gamma_0\defeq\emptyt$.
\item $\Gamma_0\defeq\unit$ and $\Gamma_1(\ttt,\ttt)\defeq\emptyt$.
\item $\Gamma_0\defeq\unit$ and $\Gamma_1(\ttt,\ttt)\defeq\unit$.
\end{enumerate}
Therefore $\mfGraph$ does not satisfy the law of excluded middle.
\end{itemize}



\subsubsection{Univalence for the graph model}
In the other direction, we also obtain a graph morphism $\graphcharmap[\Gamma]{f}:
\ctxhom{\Gamma}{\gobjclass}$ for every graph morphism $f:\ctxhom{\Delta}{\Gamma}$.
In \autoref{graph-object-classifier} we will prove that the maps
$\int_\Gamma:\ctxhom{\Gamma}{\gobjclass}\to\sm{\Delta:\tfGraph}\ctxhom{\Delta}{\Gamma}$
is an equivalence with iverse $\graphcharmapfunc{\Gamma}$. 

\begin{defn}
Let $f:\Delta\to\Gamma$ be a graph morphism. We define the graph morphism
$\graphcharmap[\Gamma]{f}$ by
\begin{align*}
\graphcharmap[\Gamma]{f}_0 & \defeq \lam{i}\hfib{f_0}{i}\\
\graphcharmap[\Gamma]{f}_1(i,j) & \defeq \lam{q}\hfib{f_1(i,j)}{q}.
\end{align*}
\end{defn}


\begin{thm}\label{graph-object-classifier}
Main theorem here.
\end{thm}

\begin{comment}
To describe the object classifier for graphs, we will follow Streicher. Thus
we have to look at presheaves over $I/i$ for each object $i$ of the category
$I\defeq 0{\rightrightarrows}1$ with the morphisms named $s$ and $t$ for source
and target. The category $I/0$ is the terminal category;
the category $I/1$ looks like $\cdot{\rightarrow}\cdot{\leftarrow}\cdot$.
Therefore, we have
\begin{align*}
\type^{\op{(I/0)}} & \eqv{}{\type},\\
\type^{\op{(I/1)}} & \eqv{}{\sm{X,Y,A:\type}(A\to X)\times(A\to Y)}\\
& \eqv{}{\sm{X,Y:\type}X\to Y\to\type}.
\end{align*}
The functors $\type^\op{\Sigma_s}$ and $\type^\op{\Sigma_t}$ are given by
$\pi_1$ and $\pi_2$ respectively. This leads to our following definition
of the object classifier $\gobjclass$:

\begin{defn}
Define $\gobjclass$ to be the graph consisting of
\begin{align*}
\gobjclass_0 & \defeq  \type\\
\gobjclass_1(X,Y) & \defeq  X\to Y\to\type
\end{align*}
and define $\pointed{\gobjclass}$ by
\begin{align*}
(\pointed{\gobjclass})_0 & \defeq  \pointed{\type}\\
(\pointed{\gobjclass})_1(\pairr{X,x},\pairr{Y,y}) & \defeq  \sm{R:X\to Y\to\type}R(x,y)
\end{align*}
There is the obvious forgetful graph morphism $t:\pointed{\gobjclass}\to\gobjclass$,
given by projection on the first coordinate.

For any morphism $f:\Delta\to\Gamma$ of graphs we define a morphism
$\graphcharmap(f):\Gamma\to\gobjclass$ of graphs by
\begin{align*}
\graphcharmap(f)_0(i) & \defeq  \hfiber{f_0}{i}\\
\graphcharmap(f)_1(q,\pairr{u,\alpha},\pairr{v,\beta}) & \defeq  \sm{p:\Delta_1(u,v)}
\id{\trans{\pairr{\alpha,\beta}}{f_1(p)}}{q}
\end{align*}
where $\pairr{u,\alpha}:\graphcharmap(f)_0(i)$ and $\pairr{v,\beta}:\graphcharmap(f)_0(j)$. The
morphism $\graphcharmap(f)$ is called the \emph{characteristic map of $f$}. We obtain a function
\begin{equation*}
\graphcharmap : \big(\sm{\Delta:\tfGraph }\Delta\to\Gamma\big)\to\big(\Gamma\to\gobjclass\big)
\end{equation*}
for every graph $\Gamma$.
\end{defn}

\begin{thm}\label{thm:graph-classifier1}
The function $\graphcharmap$ is an equivalence for any graph $\Gamma$.
\end{thm}

\begin{proof}
We have to find a quasi-inverse
\begin{align*}
\Sigma : (\Gamma\to\gobjclass)\to\big(\sm{\Delta:\tfGraph}\Delta\to\Gamma\big)
\end{align*}
of $\graphcharmap$. Thus, we have to define $\Sigma_0:(\Gamma\to\gobjclass)\to\tfGraph$ and
$\Sigma_1:\prd{P:\Gamma\to\gobjclass}\Sigma_0(P)\to\Gamma$. For $P:\Gamma\to\gobjclass$ we define
\begin{align*}
\Sigma_0(P)_0 & \defeq \sm{i:\Gamma_0}P_0(i)\\
\Sigma_0(P)_1(\pairr{i,u},\pairr{j,v}) & \defeq \sm{q:\Gamma_1(i,j)}P_1(q,u,v)\\
\Sigma_1(P)_0 & \defeq \proj1\\
\Sigma_1(P)_1(\pairr{i,u},\pairr{j,v}) & \defeq \proj1.
\end{align*}
\end{proof}

\begin{thm}\label{conj:graph_classifier2}
For any graph morphism $f:\Delta\to\Gamma$, the diagram
\begin{equation*}
\begin{tikzcd}
\Delta \ar{r}{} \ar{d}[swap]{f} & \pointed{\gobjclass} \ar{d}{t} \\ 
\Gamma \ar{r}[swap]{\graphcharmap(f)} & \gobjclass 
\end{tikzcd}
\end{equation*}
is a pullback square.
\end{thm}
\note{We would like it to be a pb \emph{in} the graph model}
\end{comment}

\begingroup\color{blue}
\subsubsection{The adjunctions $\tfcolim\dashv\Delta\dashv\tflim$}
We define $\Delta:\type\to\tfGraph$ by
\begin{equation*}
\Delta(X)\defeq\pairr{X,\lam{x}{x'}\id{x}{x'}}
\end{equation*}
for $X:\type$. For $A:X\to\type$ we define $\Delta(A):\mftyp(\Delta(X))$ by
\begin{align*}
\Delta(A)_0(x) & \defeq A(x)\\
\Delta(A)_1(p,a,b) & \defeq \id{\trans{p}{a}}{b}.
\end{align*}

\begin{lem}
For any type $X$ and any graph $\Gamma$ there is an equivalence
\begin{equation*}
\eqv{(X\to\terms{\Gamma})}{\terms{\Delta(X)\to\Gamma}}.
\end{equation*}
\end{lem}

\begin{proof}
We have to find functions
\begin{align*}
\varphi & : (X\to\terms{\Gamma})\to\terms{\Delta(X)\to\Gamma}\\
\psi & : \terms{\Delta(X)\to\Gamma}\to X\to\terms{\Gamma}
\end{align*}
which are each others homotopy inverse. To define $\varphi$, let
\end{proof}
\endgroup



\part{Type theory}

In this part we develop type theory from the ground up. We start with a type
theory without any of the basic constructors. This is the theory of contexts
families and terms which has the basic operations of extension, weakening,
substitution and identity terms. Type theory before type constructors has not
been studied very much. Dependent product types or even universes tend to make
an early appearance in just about any presentation of type theory. While one
could argue that in type theories like the one in \cite{Dybjer1996} the type
constructors could be left out and a study of the remnants could \emph{in
principle} be carried out, this is not the usual practice. In the type theory
of \cite{TheBook}, which seems to have won the race of introducing universes
as early as possible hands down, this seems entirely unfeasible.

Nevertheless, type theory without type constructors has received some attention
recently. In \cite{Garner2014}, Garner describes the combinatorial structure
of the type operations of the weakening, substitution and projection monads
(their projections are our identity terms) and suggests lots of further research
that can be done on type theory without constructors. Also, Joyal has been
lecturing about tribes lately, which are a categorical semantics for type theory
without type constructors. Although his slides have been going around on
mailing lists, there is as of yet no easily accessible written accound on tribes.

\section{%
  The essentially algebraic theory of contexts, families and terms%
  }
\label{tt}

In this section we give a description of dependent type theory before type
constructors. Apart from contexts, families and terms -- which provide for the
core of the language of dependent logic -- the basic ingredients
of this theory will be the operations of extension on context and on families, 
and empty context and empty families, weakening, substitution
and the identity terms. The resulting theory can be seen as a manifestation of the 
structure underlying dependent type theory.

We will
formulate the theory of contexts, families and terms in such a way that contexts aren't defined
to be lists of variable declarations. The variable-free (a.k.a.~\emph{name-free}) approach 
we take here is rather different than those appearing in 
\cite{hofmann1995extensional,TheBook} but it has appeared in the work of Coquand
and in \cite{Dybjer1996}.
The main reason we don't let variable declarations in is that we don't see them 
in the internal models either. This way we also set out to a more algebraic 
approach of type theory and higher category theory. Thirdly, we will not have to
be burdened with superficial comments about variables being bounded or not, or 
fresh or free or not occuring at all.

In the current presentation, contexts can be seen as binary
planar trees of which the leaves are (families of) contexts. 
The judgmental equality relation on contexts is an equivalence relation which 
expresses that binary planar
trees of contexts are judgmentally equal if their leaves are, taking only
(the isomorphism class of) 
the order of the leaves into account \emph{and not the actual shape of the three}.
The intuition behind this equivalence relation is indeed that unstructured
(i.e.~unbracketed) lists such as the lists of variable declarations which
usually appear in type theoretical syntax, may be regarded as contexts.

Besides contexts, families and terms there will also be a notion of `type in
a context', which one may assume.
A context is eligible to be a \emph{closed type} and over a context, a family
may be a type. We have the intuition that `being a type'
expresses the property of `being atomic or irreducible'. With a notion of types,
one has the means to require a stratification by which every context is a finite
extension of types starting at a closed type, and every family is likewise a
finite extension of types. The idea behind this is to recover the theory of
\cite{Garner2014} {\color{red}(and possibly B-systems?)}. We stress that
we will not include axioms asserting such a stratification in the E-system
and we leave discussion on the possibility of a typing judgment until the
end of this section. 

With families of contexts being the principal things of
study, we have added a bit more generality to the theory, compared to the
theory of categories with families presented in \cite{Dybjer1996}. Here are
a couple of reasons for this extra generality:
\begin{enumerate}
\item We want to be very sure that we can interpret all the rules of type theory
in a straightforward manner in any context.
\item We do want do end up with a categorical structure on the contexts, derived
from the theory by requiring that a morphism from $\Gamma$ to $\Delta$ is a
term of the weakening of $\Delta$ by $\Gamma$ in context $\Gamma$. Thus, we
must be able to lift the context $\Delta$ to become a family of contexts over
$\Gamma$ and it is not reasonable to expect that this results in a type.
\item It is very easy in this setting to assert that an operation has an action
on very high levels of families, i.e.~when $F$ is an operation taking things in
context $\Gamma$ to things in context $\Delta$, then it takes a family $A$ in
context $\Gamma$ to $F(A)$ in context $\Delta$; a family $P$ in context
$\ctxext{\Gamma}{A}$ to a family $F(P)$ in context $\ctxext{\Delta}{(F(A))}$,
and so on. The reason it becomes simple is that we don't have to explain what
`and so on' means.
\end{enumerate}

We will formulate fairly strict rules governing the judgmental equalities,
expressing that extension, weakening and substitution are combatible with
each other in a judgmental manner. This does not, however, diminish the role
of isomorphisms or of homotopies could play in the theory once identity types
are added. Indeed, types could still have non-trivial identity relations and
the category of types in a certain context could genuinely display higher
categorical structure, or so we conjecture.

Much of the rules we state are just compatibility rules of extension, weakening
and substitution with each other. In a way, these rules assert that our contexts
are just structureless lists of contexts and that likewise terms are structureless
lists of terms. They are structureless in the sense that the order in which
they are formed by pairing up is irrelevant. We note that this causes complications
in the traditional way that categorical sematics of type theory is implemented,
where contexts become objects of the category which is supposed to model type
theory. The reason for this is that context extension will not satisfy all the
compatibility rules we're about to state. The first step to resolving this is taking
the types in the empty context as the objects.


\subsection{The judgments of dependent type theory}
\label{judgments}

The theory we describe here is a theory of contexts, families of
contexts and terms thereof. The families of contexts are by some authors called
dependent contexts, but they are handled a bit differently here because they
become the primary object of study. Dependent contexts can be types; they could
be seen as atomic or indecomposable dependent contexts.

Thus we make eight kinds of judgments: ``$\Gamma$ is a context'',
``$A$ is a family of contexts over $\Gamma$'', ``$A$ is a type in context $\Gamma$''
and ``$x$ is a term of the family $A$ of contexts over $\Gamma$''. The other four
judgments are for judgmental equality.
\begin{align*}
\jalign\jctx{\Gamma} 
& \jalign\jctxeq{\Gamma}{\Gamma'}
  \\
\jalign\jfam{\Gamma}{A} 
& \jalign\jfameq{\Gamma}{A}{B}
  \\
\jalign\jterm{\Gamma}{A}{x} 
& \jalign\jtermeq{\Gamma}{A}{x}{y}.
\end{align*}

Strictly speaking, we have three different judgmental equalities in play and one
could request for a notational difference to signify that fact. For instance,
we could denote the judgmental equalities of contexts, families and terms by
$\jdeq_c$, $\jdeq_f$ and $\jdeq_t$ respectively. It will, however, always be
clear which of the three kinds of judgmental equality is meant when we assert
a judgmental equality and therefore we shall not bother to make this notational
distinction.

We note that what we call families over contexts
here could also have been named dependent contexts or telescopes, see
\cite{deBruijn1991,hofmann1995extensional}. The term family is in agreement
with the terminology scheme of \cite{TheBook}, though the reader should be
warned that the notion of familie means something slightly different there than
it does here.

The rules for judgmental equality establish that it is an equivalence relation
in all three cases (contexts, families and terms). Thus, the following inference
rules shall be required to be valid:
\bgroup\small
\begin{align*}
& \inference
  { \jctx{\Gamma}
    }
  { \jctxeq{\Gamma}{\Gamma}
    } 
& & \inference
    { \jctxeq{\Gamma}{\Delta}
      }
    { \jctxeq{\Delta}{\Gamma}
      } 
& & \inference
    { \jctxeq{\Gamma}{\Delta}
      \jctxeq{\Delta}{\greek{E}}
      }
    { \jctxeq{\Gamma}{\greek{E}}
      }
    \\
& \inference
  { \jfam{\Gamma}{A}
    }
  { \jfameq{\Gamma}{A}{A}
    } 
& & \inference
    { \jfameq{\Gamma}{A}{B}
      }
    { \jfameq{\Gamma}{B}{A}
      }
& & \inference
    { \jfameq{\Gamma}{A}{B}
      \jfameq{\Gamma}{B}{C}
      }
    { \jfameq{\Gamma}{A}{C}
      }
    \\
& \inference
  { \jterm{\Gamma}{A}{x}
    }
  { \jtermeq{\Gamma}{A}{x}{x}
    }
& & \inference
    { \jtermeq{\Gamma}{A}{x}{y}
      }
    { \jtermeq{\Gamma}{A}{y}{x}
      }
& & \inference
    { \jtermeq{\Gamma}{A}{x}{y}
      \jtermeq{\Gamma}{A}{y}{z}
      }
    { \jtermeq{\Gamma}{A}{x}{z}
      }
\end{align*}
\egroup

The following convertibility rules are responsible for the strictness
of judgmental equality, which sets it apart from equivalences or identifications:
\begin{align*}
& \inference
  { \jctxeq{\Gamma}{\Delta}
    \jfam{\Gamma}{A}
    }
  { \jfam{\Delta}{A}
    }
& & \inference
    { \jctxeq{\Gamma}{\Delta}
      \jfameq{\Gamma}{A}{B}
      }
    { \jfameq{\Delta}{A}{B}
      }
    \\
& \inference
  { \jctxeq{\Gamma}{\Delta}
    \jterm{\Gamma}{A}{x}
    }
  { \jterm{\Delta}{A}{x}
    }
& & \inference
    { \jctxeq{\Gamma}{\Delta}
      \jtermeq{\Gamma}{A}{x}{y}
      }
    { \jtermeq{\Delta}{A}{x}{y}
      }
    \\
& \inference
  { \jfameq{\Gamma}{A}{B}
    \jterm{\Gamma}{A}{x}
    }
  { \jterm{\Gamma}{B}{x}
    }
& & \inference
    { \jfameq{\Gamma}{A}{B}
      \jtermeq{\Gamma}{A}{x}{y}
      }
    { \jtermeq{\Gamma}{B}{x}{y}
      }
\end{align*}


\subsection{Extension}
\label{extension}

The operations of extension enable one to consider families over
families and higher level families alike. We will need both a context extension
operation and a family extension operation.

The inference rules introducing context and family extension are as follows:
\begin{align}
& \inference
  { \jfam{\Gamma}{A}
    }
  { \jctx{\ctxext{\Gamma}{A}}
    }
& & \inference
    { \jctxeq{\Gamma}{\Delta}
      \jfameq{\Gamma}{A}{B}
      }
    { \jctxeq{\ctxext{\Gamma}{A}}{\ctxext{\Delta}{B}}
      }
    \\
& \inference
  { \jfam{{\Gamma}{A}}{P}
    }
  { \jfam{\Gamma}{\ctxext{A}{P}}
    }
& & \inference
    { \jfameq{\Gamma}{A}{B} 
      \jfameq{{\Gamma}{A}}{P}{Q}
      }
    { \jfameq{\Gamma}{\ctxext{A}{P}}{\ctxext{B}{Q}}
      }
\end{align}
The extension operation may also be defined to act on families over families
and terms thereof. For instance, When $\jfam{{\Gamma}{A}}{P}$ is a family,
we would get a family $\jfam{{\Gamma}{A}}{{\Gamma}{P}}$ and when
$\jterm{{\Gamma}{A}}{P}{f}$ is a term, we would get a term
$\jterm{{\Gamma}{A}}{{\Gamma}{P}}{\ctxext{\Gamma}{f}}$. Then there would be
axioms stating the judgmental equalities $\ctxext{\Gamma}{P}\jdeq P$ and
$\ctxext{\Gamma}{f}\jdeq f$. Thus, it is of no actual use to introduce the
action of context extension on families and terms.

Using identity terms and substitution, it is also possible to define an 
extension operation on terms, which defines a term 
$\jterm{\Gamma}{{A}{P}}{\tmext{x}{u}}$ for every $\jterm{\Gamma}{A}{x}$ and 
$\jterm{\Gamma}{\subst{x}{P}}{u}$. Thus, there is a formal definition of the
\emph{pair} $\tmext{x}{u}$. We shall not go into this in the current paper,
but it will be very useful to have when one further develops the theory of
E-systems.

\subsubsection{Associativity of extension}
\label{comp-ee}
The inference rules asserting that extension is compatible with itself assert
that contexts are unstructured lists of type declarations. This rule is
unavoidable if we want that for a family $A$ in context $\Gamma$, a family over
$A$ is the same thing as a family over $\ctxext{\Gamma}{A}$. 
\begin{align}
& \inference
  { \jfam{\Gamma}{A}
    \jfam{{\Gamma}{A}}{P}
    }
  { \jctxeq{\ctxext{{\Gamma}{A}}{P}}{\ctxext{\Gamma}{{A}{P}}}
    }
  \\
& \inference
  { \jfam{{\Gamma}{A}}{P}
    \jfam{{{\Gamma}{A}}{P}}{Q}
    }
  { \jfameq{\Gamma}{\ctxext{A}{{P}{Q}}}{\ctxext{{A}{P}}{Q}}
    }
  \label{comp-ee-c}
\end{align}



\subsection{The empty context and the empty families}
\label{empty}
We introduce an empty context and an family over $\Gamma$ for every context $\Gamma$. 
It has been suggested by some to only include empty families and not an
empty context because an empty context is not necessary, but we do have several
reasons to include them. Having an empty context requires also rules asserting 
that a context is the same thing as a family over the empty context and this
gives the categorical structure on contexts for free once one has it for 
families. The main ingredients that will be missing from the theory once an
empty context is avoided are weakening by a context and identity morphisms from
a context to itself. Including these by hand also requires to formulate all the
compatibility rules involving weakening once more for the cases of weakening by
a context and identity terms at contexts. We prefer to state these rules once
and only once and including an empty context helps in this respect.

We also prefer to have our scheme of compatibility rules as symmetrical as
possible. The structure of type dependency should look exactly the same in the
default case as in any context. In that respect, having an empty family but not
an empty context seems a bit odd. Also, some sets of compatibility rules 
(like the rules stating that extension is compatible with the empty families)
will become assymetrical as a result of not including an empty context. Moreover,
we would eventually like to include a stratification of the theory by means of
a type judgment (asserting that a family in a context $\Gamma$ is a type) and
study closed types (i.e.~types in the empty context). It would be possible to
provide a notion of closed types without having an empty context, but this would
have to be formulated separately and we would have to restate all the rules for
types (if any) for closed types all over again.

One of the main uses of the empty context and the empty families is that we
get the property that the `action on contexts' of an operation is compatible
with its `action on families'.

The empty family over a context $\Gamma$ is introduced by the following rule
inference rule:
\begin{align}
& \inference
  { }
  { \jctx{\emptyc}
    }
  \\
& \inference
  { \jctx{\Gamma}
    }
  { \jfam{\Gamma}{\emptyf[\Gamma]}
    }
  \\
& \inference
  { \jctxeq{\Gamma}{\Gamma'}
    }
  { \jfameq{\Gamma}{\emptyf[\Gamma]}{\emptyf[\Gamma']}
    }
\end{align}

By regarding contexts as families of contexts over the empty context, we
enable ourselves also to speak of terms of contexts. A term of a context
$\Gamma$ is a term of the family $\Gamma$ over the empty context. These ideas
are captured in the following convertibility rules:
\begin{align}
& \inference
  { \jctx{\Gamma}
    }
  { \jfam{\emptyc}{i(\Gamma)}
    } 
  &
& \inference
  { \jctxeq{\Gamma}{\Delta}
    }
  { \jfameq{\emptyc}{i(\Gamma)}{i(\Delta)}
    }
\end{align}

The reader may wonder whether the empty family $\emptyf$ always has a
term. This shall follow from the rules stating the compatibility of extension
with the empty families in \autoref{comp-0e} below and from
identity terms (\autoref{identityterms}).

\subsubsection{Compatibility of extension with the empty context and families}
In the following set of inference rules we state that the empty context and
the empty family are neutral objects for both context extension (the first two
rules) and family extension (the last two rules).
\label{comp-e0}\label{comp-0e}
\begin{align}
& \inference
  { \jctx{\Gamma}
    }
  { \jctxeq{\ctxext{\emptyc}{i(\Gamma)}}{\Gamma}
    }
  \label{comp-0e-c}
  \\
& \inference
  { \jctx{\Gamma}
    }
  { \jctxeq{\ctxext{\Gamma}{\emptyf}}{\Gamma}
    }
  \label{comp-e0-c}\\
& \inference
  { \jfam{\Gamma}{A}
    }
  { \jfameq{\Gamma}{\ctxext{\emptyf}{A}}{A}
    }
  \label{comp-0e-f}
  \\
& \inference
  { \jfam{\Gamma}{A}
    }
  { \jfameq{\Gamma}{{A}{\emptyf}}{A}
    }
  \label{comp-e0-f}
  \\
& \inference
  { \jfam{\emptyc}{A}
    }
  { \jfameq{\emptyc}{i(\ctxext{\emptyc}{A})}{A}
    }
\end{align}

\subsubsection{Family extension restricted to the empty context}
Strictly speaking we should have used a different notation for context extension
as for family extension, because the following rule asserting that family extension
in the empty context is the same thing as context extension would look tautological
without a difference. So let us denote, only for the moment, context extension
of $\Gamma$ by $A$ by $(\ctxext{\Gamma}{A})^c$ and family extension of $A$ by
$P$ in context $\Gamma$ by $(\ctxext{A}{P})^\famsym$. 

Note that we may consider a context $\Gamma$ as a family over $\emptyc$ and
a family $\jfam{\Gamma}{A}$ as a family $\jfam{{\emptyc}{\Gamma}}{A}$. 
Therefore we will require the following rule:
\begin{align}
& \inference
  { \jfam{\Gamma}{A}
    }
  { \jctxeq{\ctxext{\Gamma}{A}}{\ctxext{i(\Gamma)}{A}}
    }
\end{align}
Note that this rule actually justifies that we have not utilized two different
notations for context extension and family extension.


\subsection{The theory of weakening}
\label{weakening}

When $A$ is a family in context $\Gamma$, the operation of weakening by $A$
takes a family $B$ in context $\Gamma$ and provides a family $\ctxwk{A}{B}$
in context $\ctxext{\Gamma}{A}$. The context family $\ctxwk{A}{B}$ can be seen
as the constant family over $\ctxext{\Gamma}{A}$ with value $B$. This idea will
be axiomatized in the cancellation property of weakening and substitution in
\autoref{cancellation-ws}. In \autoref{morphisms} we will take the terms 
$\unfold{\jhom{\Gamma}{A}{B}{f}}$ to be the morphisms of families from $A$ to 
$B$. These will be at the heart of the categorical structure of the theory.

The weakening operation acts on three levels: on contexts, on families and
on terms. The `action on contexts' of weakening is the action we described
above: it takes a family $B$ over $\Gamma$ to a family $\ctxwk{A}{B}$ over
$\ctxext{\Gamma}{A}$; the `action on families' of weakening takes a family
$Q$ over $\ctxext{\Gamma}{B}$ to a family $\ctxwk[\famsym]{A}{Q}$ over
$\ctxext{{\Gamma}{A}}{\ctxwk{A}{B}}$; the `action on terms' of weakening takes
a term $g$ of $Q$ to a term $\ctxwk[\tmsym]{A}{g}$ of $\ctxwk[\famsym]{A}{Q}$.
\begin{align}
& \inference
  { \jfam{\Gamma}{A}
    \jfam{\Gamma}{B}
    }
  { \jfam{{\Gamma}{A}}{\ctxwk{A}{B}}
    }
& & \inference
    { \jfameq{\Gamma}{A}{A'}
      \jfameq{\Gamma}{B}{B'}
      }
    { \jfameq{{\Gamma}{A}}{\ctxwk{A}{B}}{\ctxwk{A'}{B'}}
      }
    \\
& \inference
  { \jfam{\Gamma}{A}
    \jfam{{\Gamma}{B}}{Q}
    }
  { \jfam{{{\Gamma}{A}}{\ctxwk{A}{B}}}{\ctxwk[\famsym]{A}{Q}}
    }
& & \inference
    { \jfameq{\Gamma}{A}{A'}
      \jfameq{{\Gamma}{B}}{Q}{Q'}
      }
    { \jfameq
        {{{\Gamma}{A}}{\ctxwk{A}{B}}}
        {\ctxwk[\famsym]{A}{Q}}
        {\ctxwk[\famsym]{A'}{Q'}}
      }
    \\
& \inference
  { \jfam{\Gamma}{A}
    \jterm{{\Gamma}{B}}{Q}{g}
    }
  { \jterm{{{\Gamma}{A}}{\ctxwk{A}{B}}}{\ctxwk[\famsym]{A}{Q}}{\ctxwk[\tmsym]{A}{g}}
    }
& & \inference
    { \jfameq{\Gamma}{A}{A'}
      \jtermeq{{\Gamma}{B}}{Q}{g}{g'}
      }
    { \jtermeq
        {{{\Gamma}{A}}{\ctxwk{A}{B}}}
        {\ctxwk[\famsym]{A}{Q}}
        {\ctxwk[\tmsym]{A}{g}}
        {\ctxwk[\tmsym]{A'}{g'}}
      }
\end{align}

\subsubsection{Weakenings of extensions}
\label{comp-we}
The following rules assert that when an extended family is weakened, the
weakening distributes over the extension factors.
\begin{align}
& \inference
  { \jfam{\Gamma}{A}
    \jfam{{\Gamma}{B}}{Q}
    }
  { \jfameq
      {\ctxext{\Gamma}{A}}
      {\ctxwk{A}{\ctxext{B}{Q}}}
      {\ctxext{\ctxwk{A}{B}}{\ctxwk{A}{Q}}}
    }
  \label{comp-we-c}
  \\
& \inference
  { \jfam{\Gamma}{A}
    \jfam{{{\Gamma}{B}}{Q}}{R}
    }
  { \jfameq
      {\ctxext{{\Gamma}{A}}{\ctxwk{A}{B}}}
      {\ctxwk{A}{\ctxext{Q}{R}}}
      {\ctxext{\ctxwk{A}{Q}}{\ctxwk{A}{R}}}
    }
  \label{comp-we-f}
\end{align}
When thinking of terms of $\ctxwk{A}{B}$ as morphisms of families from $A$ to
$B$, this looks already like form of type theoretic choice. It is weaker in that
it is not stated with function types, yet it is stronger in that it states a
judgmental equality between two families.

There is also a version of this property where an extended term is weakened.
This variant is stated and proved in \autoref{comp-we-t}.

\subsubsection{Weakening of weakenings}
Weakening by a family $A$ in context $\Gamma$ brings things in context $\Gamma$
to things in context $\ctxext{\Gamma}{A}$. Since we have all the ingredients of
the theory of contexts, families and terms in the context $\Gamma$ as well it
has in particular it's own weakening by families. Suppose we have a family
$\jfam{{\Gamma}{B}}{Q}$ over $B$ in context $\Gamma$. Weakening by $Q$ brings
things from context $\ctxext{\Gamma}{B}$ to $\ctxext{{\Gamma}{B}}{Q}$. Thus
we can provide rules asserting what will happen when we first weaken by $Q$ and
then (via the action on families) by $A$. We will require the following
inference rules to be valid:
\label{comp-ww}
\begin{align}
& \inference
  { \jfam{\Gamma}{A}
    \jfam{{\Gamma}{B}}{Q}
    \jfam{{\Gamma}{B}}{R}
    }
  { \jfameq
      {{{{\Gamma}{A}}{\ctxwk{A}{B}}}{\ctxwk{A}{Q}}}
      {\ctxwk{A}{{Q}{R}}}
      {\ctxwk{{A}{Q}}{{A}{R}}}
    }
  \label{comp-ww-c}\\
& \inference
  { \jfam{\Gamma}{A}
    \jfam{{\Gamma}{B}}{Q}
    \jfam{{{\Gamma}{B}}{R}}{S}
    }
  { \jfameq
      {{{{{\Gamma}{A}}{\ctxwk{A}{B}}}{\ctxwk{A}{Q}}}{\ctxwk{A}{{Q}{R}}}}
      {\ctxwk{A}{{Q}{S}}}
      {\ctxwk{{A}{Q}}{{A}{S}}}
    }
  \label{comp-ww-f}\\
& \inference
  { \jfam{\Gamma}{A}
    \jfam{{\Gamma}{B}}{Q}
    \jterm{{{\Gamma}{B}}{R}}{S}{k}
    }
  { \jtermeq
      {{{{{\Gamma}{A}}{\ctxwk{A}{B}}}{\ctxwk{A}{Q}}}{\ctxwk{A}{{Q}{R}}}}
      {\ctxwk{A}{{Q}{S}}}
      {\ctxwk{A}{{Q}{k}}}
      {\ctxwk{{A}{Q}}{{A}{k}}}
    }
  \label{comp-ww-t}
\end{align}

\begin{rmk}
As an important special case of these inference rules we have the following
valid inference rules:
\begin{align*}
& \inference
  { \jfam{\Gamma}{A}
    \jfam{\Gamma}{B}
    \jfam{\Gamma}{C}
    }
  { \jfameq
      {{{\Gamma}{A}}{\ctxwk{A}{B}}}
      {\ctxwk{A}{{B}{C}}}
      {\ctxwk{{A}{B}}{{A}{C}}}
    }
  \\
& \inference
  { \jfam{\Gamma}{A}
    \jfam{\Gamma}{B}
    \jfam{{\Gamma}{C}}{R}
    }
  { \jfameq
      {{{{\Gamma}{A}}{\ctxwk{A}{B}}}{\ctxwk{A}{{B}{C}}}}
      {\ctxwk{A}{{B}{R}}}
      {\ctxwk{{A}{B}}{{A}{R}}}
    }
  \\
& \inference
  { \jfam{\Gamma}{A}
    \jfam{\Gamma}{B}
    \jterm{{\Gamma}{C}}{R}{h}
    }
  { \jtermeq
      {{{{\Gamma}{A}}{\ctxwk{A}{B}}}{\ctxwk{A}{{B}{C}}}}
      {\ctxwk{A}{{B}{R}}}
      {\ctxwk{A}{{B}{h}}}
      {\ctxwk{{A}{B}}{{A}{h}}}
    }
\end{align*}
Moreover, we have
\begin{equation*}
\inference
  { \jfam{\Gamma}{A}
    \jfam{\Gamma}{B}
    \jterm{\Gamma}{C}{z}
    }
  { \jtermeq
      {{{\Gamma}{A}}{\ctxwk{A}{B}}}
      {\ctxwk{A}{{B}{C}}}
      {\ctxwk{A}{{B}{z}}}
      {\ctxwk{{A}{B}}{{A}{z}}}
    }
\end{equation*}
\end{rmk}

\subsubsection{Currying for weakening}
\label{comp-ew}
The rules expressing that extension is compatible with weakening assert that
weakening by an extension is the same thing as weakening twice in the
appropriate way.
\begin{align}
& \inference
  { \jfam{\Gamma}{A}
    \jfam{{\Gamma}{A}}{P}
    \jfam{\Gamma}{B}
    }
  { \jfameq
      {{{\Gamma}{A}}{P}}
      {\ctxwk{\ctxext{A}{P}}{B}}
      {\ctxwk{P}{{A}{B}}}
    }
  \label{comp-ew-c}\\
& \inference
  { \jfam{\Gamma}{A}
    \jfam{{\Gamma}{A}}{P}
    \jfam{{\Gamma}{B}}{Q}
    }
  { \jfameq
      {{{{\Gamma}{A}}{P}}{\ctxwk{P}{{A}{B}}}}
      {\ctxwk{\ctxext{A}{P}}{Q}}
      {\ctxwk{P}{{A}{Q}}}
    }
  \label{comp-ew-f}\\
& \inference
  { \jfam{\Gamma}{A}
    \jfam{{\Gamma}{A}}{P}
    \jterm{{\Gamma}{B}}{Q}{g}
    }
  { \jtermeq
      {{{{\Gamma}{A}}{P}}{\ctxwk{P}{{A}{B}}}}
      {\ctxwk{P}{{A}{Q}}}
      {\ctxwk{\ctxext{A}{P}}{g}}
      {\ctxwk{P}{{A}{g}}}
    } 
  \label{comp-ew-t}
\end{align}


\subsection{The theory of substitution}
\label{substitution}

The theory of substitution can be formulated separately from the theory of
weakening. Therefore, we will only require the theory of extension and the
empty context and families. In \autoref{sec:esystem-equalities} we will
provide the additional judgmental equality rules that will be required when
combining the theories of substitution and of projections.

Given a family $P$ over $A$ and a term $x$ of $A$, substitution gives a way to
consider the \emph{fiber $\subst{x}{P}$ of $P$ at $x$}. As was the case with
weakening, the substitution operation comes in three parts: the `action on
contexts' of substitution is the operation just described; the `action on
families' of substitution takes a family $Q$ over $\ctxext{{\Gamma}{A}}{P}$
to a family $\subst[\famsym]{x}{Q}$ over $\ctxext{\Gamma}{\subst{x}{P}}$; the
`action on terms' of substitution takes a term $g$ of $Q$ over
$\ctxext{{\Gamma}{A}}{P}$ to a term $\subst[\tmsym]{x}{g}$ of 
$\subst[\famsym]{x}{Q}$.
\begin{align}
& \inference
  { \jterm{\Gamma}{A}{x}
    \jfam{{\Gamma}{A}}{P}
    }
  { \jfam{\Gamma}{\subst{x}{P}}
    }
& & \inference
    { \jtermeq{\Gamma}{A}{x}{x'}
      \jfameq{{\Gamma}{A}}{P}{P'}
      }
    { \jfameq{\Gamma}{\subst{x}{P}}{\subst{x'}{P'}}
      }
    \\
& \inference
  { \jterm{\Gamma}{A}{x}
    \jfam{{{\Gamma}{A}}{P}}{Q}
    }
  { \jfam{{\Gamma}{\subst{x}{P}}}{\subst[\famsym]{x}{Q}}
    }
& & \inference
    { \jtermeq{\Gamma}{A}{x}{x'}
      \jfameq{{{\Gamma}{A}}{P}}{Q}{Q'}
      }
    { \jfameq{{\Gamma}{\subst{x}{P}}}{\subst[\famsym]{x}{Q}}{\subst[\famsym]{x'}{Q'}}
      }
    \\
& \inference
  { \jterm{\Gamma}{A}{x}
    \jterm{{{\Gamma}{A}}{P}}{Q}{g}
    }
  { \jterm{{\Gamma}{\subst{x}{P}}}{\subst[\famsym]{x}{Q}}{\subst[\tmsym]{x}{g}}
    }
& & \inference
    { \jtermeq{\Gamma}{A}{x}{x'}
      \jtermeq{{{\Gamma}{A}}{P}}{Q}{g}{g'}
      }
    { \jtermeq
        {{\Gamma}{\subst{x}{P}}}
        {\subst[\famsym]{x}{Q}}
        {\subst[\tmsym]{x}{g}}
        {\subst[\tmsym]{x'}{g'}}
      }
\end{align}

\subsubsection{Fibers of extensions}
\label{comp-se}
The following inference rules assert that if we take the fiber of an extended
family at a term $x$ of $A$ in context $\Gamma$, the substitution by $x$
distributes over the factors of the extension.
\begin{align}
& \inference
  { \jterm{\Gamma}{A}{x}
    \jfam{{{\Gamma}{A}}{P}}{Q}
    }
  { \jfameq
      {\Gamma}
      {\subst{x}{\ctxext{P}{Q}}}
      {\ctxext{\subst{x}{P}}{\subst{x}{Q}}}
    }
  \label{comp-se-c}
  \\
& \inference
  { \jterm{\Gamma}{A}{x}
    \jfam{{{{\Gamma}{A}}{P}}{Q}}{R}
    }
  { \jfameq
      {{\Gamma}{\subst{x}{P}}}
      {\subst{x}{\ctxext{Q}{R}}}
      {\ctxext{\subst{x}{Q}}{\subst{x}{R}}}
    }
  \label{comp-se-f}
\end{align}
There is also a version of this statement in which extended terms are considered.
This variant is stated and proved in \autoref{comp-se-t}.

\subsubsection{Fibers of fibers}
\label{comp-ss}

The following rules explain what happens when we first substitute by a term
$g$ of $Q$ over $P$ in context $\ctxext{\Gamma}{A}$ and then by a term $x$ of
$A$:
\begin{align}
& \inference
  { \jterm{\Gamma}{A}{x}
    \jterm{{{\Gamma}{A}}{P}}{Q}{g}
    \jfam{{{{\Gamma}{A}}{P}}{Q}}{R}
    }
  { \jfameq
      {{{\Gamma}{\subst{x}{P}}}{\subst{x}{Q}}}
      {\subst{x}{{g}{R}}}
      {\subst{{x}{g}}{{x}{R}}}
    }
  \label{comp-ss-c}
  \\
& \inference
  { \jterm{\Gamma}{A}{x}
    \jterm{{{\Gamma}{A}}{P}}{Q}{g}
    \jfam{{{{{\Gamma}{A}}{P}}{Q}}{R}}{S}
    }
  { \jfameq
      {{{{\Gamma}{\subst{x}{P}}}{\subst{x}{Q}}}{\subst{x}{{g}{R}}}}
      {\subst{x}{{g}{S}}}
      {\subst{{x}{g}}{{x}{S}}}
    }
  \label{comp-ss-f}
  \\
& \inference
  { \jterm{\Gamma}{A}{x}
    \jterm{{{\Gamma}{A}}{P}}{Q}{g}
    \jterm{{{{{\Gamma}{A}}{P}}{Q}}{R}}{S}{k}
    }
  { \jtermeq
      {{{{\Gamma}{\subst{x}{P}}}{\subst{x}{Q}}}{\subst{x}{{g}{R}}}}
      {\subst{x}{{g}{S}}}
      {\subst{x}{{g}{k}}}
      {\subst{{x}{g}}{{x}{k}}}
    }
  \label{comp-ss-t}
\end{align}

\begin{rmk}
As an important special case of these inference rules we have the following
valid inference rules:
\begin{align*}
& \inference
  { \jterm{\Gamma}{A}{x}
    \jterm{{\Gamma}{A}}{P}{f}
    \jfam{{{\Gamma}{A}}{P}}{Q}
    }
  { \jfameq
      {\Gamma}
      {\subst{x}{{f}{Q}}}
      {\subst{{x}{f}}{{x}{Q}}}
    }
  \\
& \inference
  { \jterm{\Gamma}{A}{x}
    \jterm{{\Gamma}{A}}{P}{f}
    \jfam{{{{\Gamma}{A}}{P}}{Q}}{R}
    }
  { \jfameq
      {{\Gamma}{\subst{x}{{f}{Q}}}}
      {\subst{x}{{f}{R}}}
      {\subst{{x}{f}}{{x}{R}}}
    }
  \\
& \inference
  { \jterm{\Gamma}{A}{x}
    \jterm{{\Gamma}{A}}{P}{f}
    \jterm{{{{\Gamma}{A}}{P}}{Q}}{R}{h}
    }
  { \jtermeq
      {{\Gamma}{\subst{x}{{f}{Q}}}}
      {\subst{x}{{f}{R}}}
      {\subst{x}{{f}{h}}}
      {\subst{{x}{f}}{{x}{h}}}
    }
\end{align*}
Moreover, we get
\begin{equation*}
\inference
  { \jterm{\Gamma}{A}{x}
    \jterm{{\Gamma}{A}}{P}{f}
    \jterm{{{\Gamma}{A}}{P}}{Q}{g}
    }
  { \jtermeq
      {\Gamma}
      {\subst{x}{{f}{Q}}}
      {\subst{x}{{f}{g}}}
      {\subst{{x}{f}}{{x}{g}}}
    }
\end{equation*}
\end{rmk}


\subsection{Joining the theory of projections with the theory of substitution}
\label{sec:esystem-equalities}

\subsubsection{Weakening of an empty family}
The following inference rules express that when the empty family is
weakened, the result is the empty family.
\label{comp-w0}
\begin{align}
& \inference
  { \jfam{\Gamma}{A}
    }
  { \jfameq{{\Gamma}{A}}{\ctxwk{A}{\emptyf}}{\emptyf}
    }
  \label{comp-w0-c}\\
& \inference
  { \jfam{\Gamma}{A}
    \jfam{\Gamma}{B}
    }
  { \jfameq
    {{{\Gamma}{A}}{\ctxwk{A}{B}}}
    {\ctxwk[\famsym]{A}{\emptyf}}
    {\emptyf}
    }
  \label{comp-w0-f}
\end{align}

Because a family over $\Gamma$ is the same as a family over 
$\ctxext{\Gamma}{\emptyf}$ we can apply both the action on contexts and the
action on families of weakening to a family $B$ over $\Gamma$. When we apply
the action on families, we obtain a family $\ctxwk[\famsym]{A}{B}$ over the
context $\ctxext{{\Gamma}{A}}{\ctxwk{A}{\emptyf}}$. However, since we have
postulated the judgmental equalities $\ctxwk{A}{\emptyf}\jdeq\emptyf$ and
$\ctxext{{\Gamma}{A}}{\emptyf}\jdeq\ctxext{\Gamma}{A}$, we see that we can
compare $\ctxwk[\famsym]{A}{B}$ with $\ctxwk{A}{B}$. The following inference
rule postulates that these two are judgmentally equal:
\begin{equation}
\inference
{ \jfam{\Gamma}{A}
  \jfam{\Gamma}{B}
  }
{ \jfameq{{\Gamma}{A}}{\ctxwk[\famsym]{A}{B}}{\ctxwk{A}{B}}
  }
\end{equation}
Due to this rule, the action on contexts and the action on families of weakening
are compatible with each other and consequently there can be no possible
confusion when we omit the annotations $\famsym$ and $\tmsym$ alltogether. In
the future, the weakening of a family $Q$ over $\ctxext{\Gamma}{B}$ shall
be denoted just by $\ctxwk{A}{Q}$ and likewise the weakening of a term $g$ of
$Q$ shall be denoted by $\ctxwk{A}{g}$.

Because a family $B$ over $\Gamma$ can be treated as a family by the operation
of weakening, weakening also acts on the terms of $B$. The weakening of a term
$y$ of $B$ by $A$ can be seen as the constant term (or function) of the
family $\ctxwk{A}{B}$.

\subsubsection{Weakening by the empty family}
Note that we can also weaken by the empty family over $\Gamma$.
Weakening by the empty family $\emptyf$ over a context $\Gamma$ leaves families, 
their terms, families over those families and terms of those unchanged:
\label{comp-0w}
\begin{align}
& \inference
  { \jany{\Gamma}{e}
    }
  { \janyeq{\Gamma}{\ctxwk{\emptyf}{e}}{e}
    }
  \label{comp-0w-any}
\end{align}

\subsubsection{Fibers of an empty family}
The following inference rules establish that the fibers of the empty family are 
the empty families:
\label{comp-s0}
\begin{align}
& \inference
  { \jterm{\Gamma}{A}{x}
    }
  { \jfameq{\Gamma}{\subst{x}{\emptyf}}{\emptyf}
    }
  \label{comp-s0-c}
  \\
& \inference
  { \jterm{\Gamma}{A}{x}
    \jfam{{\Gamma}{A}}{P}
    }
  { \jfameq
      {{\Gamma}{\subst{x}{P}}}
      {\subst{x}{\emptyf}}
      {\emptyf}
    }
  \label{comp-s0-f}
\end{align}

We use the above rule to state the compatibility of the action on families of
substitution with the action on contexts of substitution. Note that a family
$P$ over $\ctxext{\Gamma}{A}$ may be regarded as a family over
$\ctxext{{\Gamma}{A}}{\emptyf}$. Thus, we may consider the family
$\subst[\famsym]{x}{P}$ over $\ctxext{\Gamma}{\subst{x}{\emptyf}}$. Since
$\subst{x}{\emptyf}$ is judgmentally equal to the empty family, we may compare
$\subst[\famsym]{x}{P}$ with $\subst{x}{P}$:
\begin{equation}
\inference
{ \jfam{{\Gamma}{A}}{P}
  }
{ \jfameq{\Gamma}{\subst[\famsym]{x}{P}}{\subst{x}{P}}
  }
\end{equation}
Due to this rule we need not make the annotations $\famsym$ and $\tmsym$ in
the notation for substitution anymore and thus we shall omit them from now on.
Note that because a family $P$ over $\ctxext{\Gamma}{A}$ is eligible for
application of the action on families of substitution, we may also substitute
terms of $P$. Thus, given terms $\jterm{\Gamma}{A}{x}$ and $\jterm{{\Gamma}{A}}{P}{f}$,
we get a term $\jterm{\Gamma}{\subst{x}{P}}{\subst{x}{f}}$, the \emph{value of
$f$ at $x$}.

In \autoref{morphisms} we will use a combination of weakening and substitution
to define composition of morphisms of families. However, we have to rely
on the cancellation rule stated in \autoref{cancellation-ws} before we can
meaningfully state the definition of composition.

\subsubsection{Fibers of weakenings}\label{comp-sw}
The following rules assert what happens when we first weaken by a family
$Q$ over $P$ in context $\ctxext{\Gamma}{A}$ and then substitute by a term
$x$ of $A$:
\begin{align}
& \inference
  { \jterm{\Gamma}{A}{x}
    \jfam{{{\Gamma}{A}}{P}}{Q}
    \jany{{{\Gamma}{A}}{P}}{e}
    }
  { \janyeq
      {{{\Gamma}{\subst{x}{P}}}{\subst{x}{Q}}}
      {\subst{x}{\ctxwk{Q}{e}}}
      {\ctxwk{\subst{x}{Q}}{\subst{x}{e}}}
    }
  \label{comp-sw-any}
\end{align}

\begin{rmk}
As an important special case of these inference rules we have the following
valid inference rules:
\begin{align*}
& \inference
  { \jterm{\Gamma}{A}{x}
    \jfam{{\Gamma}{A}}{P}
    \jany{{\Gamma}{A}}{e}
    }
  { \janyeq
      {{\Gamma}{\subst{x}{P}}}
      {\subst{x}{\ctxwk{P}{e}}}
      {\ctxwk{\subst{x}{P}}{\subst{x}{e}}}
    }
\end{align*}
Moreover, we get
\begin{equation*}
\inference
  { \jterm{\Gamma}{A}{x}
    \jfam{{\Gamma}{A}}{P}
    \jterm{{\Gamma}{A}}{Q}{g}
    }
  { \jtermeq
      {{\Gamma}{\subst{x}{P}}}
      {\subst{x}{\ctxwk{P}{Q}}}
      {\subst{x}{\ctxwk{P}{g}}}
      {\ctxwk{\subst{x}{P}}{\subst{x}{g}}}
    }
\end{equation*}
\end{rmk}

\subsubsection{Weakenings of fibers}
\label{comp-ws}
The following inference rules explain what happens when we first weaken by a
term $g$ of $Q$ in context $\ctxext{\Gamma}{B}$ and then weaken by a family
$A$ over $\Gamma$.
\begin{align}
& \inference
  { \jfam{\Gamma}{A}
    \jterm{{\Gamma}{B}}{Q}{g}
    \jany{{{\Gamma}{B}}{Q}}{e}
    }
  { \janyeq
      {{{\Gamma}{A}}{\ctxwk{A}{B}}}
      {\ctxwk{A}{\subst{g}{e}}}
      {\subst{\ctxwk{A}{g}}{\ctxwk{A}{e}}}
    }
  \label{comp-ws-any}
\end{align}

\begin{rmk}
As an important special case of these inference rules we have the following 
valid inference rules:
\begin{align*}
& \inference
  { \jfam{\Gamma}{A}
    \jterm{\Gamma}{B}{y}
    \jany{{\Gamma}{B}}{e}
    }
  { \janyeq
      {{\Gamma}{A}}
      {\ctxwk{A}{\subst{y}{e}}}
      {\subst{\ctxwk{A}{y}}{\ctxwk{A}{e}}}
    }
\end{align*}
Moreover, we get
\begin{equation*}
\inference
  { \jfam{\Gamma}{A}
    \jterm{\Gamma}{B}{y}
    \jterm{{\Gamma}{B}}{Q}{g}
    }
  { \jtermeq
      {{\Gamma}{A}}
      {\ctxwk{A}{\subst{y}{Q}}}
      {\ctxwk{A}{\subst{y}{g}}}
      {\subst{\ctxwk{A}{y}}{\ctxwk{A}{g}}}
    }
\end{equation*}
\end{rmk}

\subsubsection{The cancellation property of weakening and substitution}
\label{cancellation-ws}
The judgmental equalities we're about to describe assert that substituting a term
in the weakening a thing gives you the thing back. In the case of contexts we get that each fiber
$\subst{x}{\ctxwk{A}{B}}$ is just $B$ and in the case of terms we get 
that $\ctxwk{A}{y}$ is the constant function
mapping everything to $y:B$. Thus, these rules actually establish the weakening
as the weakening. After stating the rules we will describe what it means to
compose context morphisms (terms of weakened contexts).
\begin{align}
& \inference
  { \jterm{\Gamma}{A}{x}
    \jany{\Gamma}{e}
    }
  { \janyeq
      {\Gamma}
      {\subst{x}{\ctxwk{A}{e}}}
      {e}
    }
  \label{cancellation-ws-e}
\end{align}

\subsubsection{The identity term of a substituted family}
\label{comp-si}
The identity term of a substituted family is the substitution of the identity term
\begin{equation}
\inference
  { \jterm{\Gamma}{A}{x}
    \jfam{{{\Gamma}{A}}{P}}{Q}
    }
  { \unfoldall{\jhomeq
      {{\Gamma}{\subst{x}{P}}}
      {\subst{x}{Q}}
      {\subst{x}{Q}}
      {\subst{x}{\idtm{Q}}}
      {\idtm{\subst{x}{Q}}}
    }}
  \label{comp-si-t}
\end{equation}

\begin{rmk}
An important special case is the judgmental equality
\begin{equation*}
\unfoldall{\jhomeq
      {\Gamma}
      {\subst{x}{P}}
      {\subst{x}{P}}
      {\subst{x}{\idtm{P}}}
      {\idtm{\subst{x}{P}}}
    }
\end{equation*}
for a family $\jfam{{\Gamma}{A}}{P}$.
\end{rmk}

\subsubsection{The cancellation property of identity terms}
\label{cancellation-i}
Identity terms are determined by their behavior with respect to substitution combined with
weakening. The identity terms will also be subject to compatibility rules.
\begin{align}
& \inference
  { \jterm{\Gamma}{A}{x}
    }
  { \jtermeq{\Gamma}{A}{\subst{x}{\idtm{A}}}{x}
    }
  \label{cancellation-si}\\
& \inference
  { \jany{{\Gamma}{A}}{e}
    }
  { \janyeq{{\Gamma}{A}}{\subst{\idtm{A}}{\ctxwk{A}{e}}}{e}
    }
  \label{precomp-idtm-any}
\end{align}


%\subsection{%
  The possiblity of types in the theory of contexts, families and terms}
\label{types}

We have deliberately not spoken of types so far because we have taken the point
of view that a type in a context is nothing but a family in that context which
belongs to the class of types. We think of types as \emph{irreducible} families,
i.e.\ families which are neither empty nor the extension of two
families which are both not empty (in algebraic terminology: which
are both non-trivial). To allow ourselves to speak of types we introduce two
new judgments: the judgment that something is a type and the judgment that two
types are equal.
\begin{align*}
\jalign\jtype{\Gamma}{A} 
& \jalign\jtypeeq{\Gamma}{A}{B}
\end{align*}
But only families of contexts are eligible to be types. If $A$ is a type
in context $\Gamma$, then $A$ is also a family of contexts over $\Gamma$. 
Moreover, two types in context $\Gamma$ are judgmentally equal precisely when they are equal
as context families and if a family $B$ of contexts over $\Gamma$ is
judgmentally equal to a type $A$ in context $\Gamma$, then $B$ is a type in
context $\Gamma$. This is expressed by the following four inference rules:
\begin{align*}
& \inference
  { \jtype{\Gamma}{A}
    }
  { \jfam{\Gamma}{A}
    }
& & \inference
    { \jtypeeq{\Gamma}{A}{B}
      }
    { \jfameq{\Gamma}{A}{B}
      }
    \\
& \inference
  { \jtype{\Gamma}{A}
    \jfameq{\Gamma}{A}{B}
    }
  { \jtype{\Gamma}{B}
    }
& & \inference
    { \jtype{\Gamma}{A}
      \jfameq{\Gamma}{A}{B}
      }
    { \jtypeeq{\Gamma}{A}{B}
      }
\end{align*}
As pointed out at the beginning of this subsection, 
we do not assume that the empty family is a type, that would be like
assuming that the multiplicative unit of a ring is prime. 

We add rules asserting that a weakened type is again a type that 
substitution preserves the property of being a type:
\begin{align}
& \inference
  { \jfam{\Gamma}{B}
    \jtype{\ctxext{\Gamma}{B}}{Q}
    }
  { \jtype{\ctxext{{\Gamma}{A}}{\ctxwk{A}{B}}}{\ctxwk{A}{Q}}
    }
  \\
& \inference
  { \jterm{\Gamma}{A}{x}
    \jtype{\ctxext{{\Gamma}{A}}{P}}{Q}
    }
  { \jtype{\ctxext{\Gamma}{\subst{x}{P}}}{\subst{x}{Q}}
    }
\end{align}
With only the current rules, the possibility of making the judgment that
something is a type does not add much to the theory of contexts, families and
terms. Nevertheless, when studying models, having an interpretation for the
judgment that something is a type will allow for the possibility to study
conditions such as the one asserting that every family `factorizes' uniquely
as multiple applications of extension to types, analogous to the condition on
unique factorization domains in ring theory. All the tradional models of type
theory should translate to models with such a condition, simply because
traditionally contexts are viewed as lists of variable (and type) declarations.



\section{Type constructors in structural type theory}
\label{tt_constructors}
In this section we will introduce the usual type constructors to structural
type theory. We do so with the point of view that each type constructor is a
(class of) operations on type theory that should be compatible with extension,
weakening, substitution and identity morphisms. Likewise, each induction
principle of an inductively defined type constructor is going to be such an
operator, and it is therefore also required to be compatible with extension,
weakening, substitution and identity functions.

We will first state all the rules of the individual type constructors, where
only the $\wtypesym$-type constructor depends on the presence of dependent
function types (or universes too?), and hence comes with the notion that $\wtypesym$
is compatible with dependent product types and vice versa. All the other
compatibility properties are stated in \autoref{compatibility-of-type-constructors}.

\subsubsection{Issues to keep in mind}
\begin{enumerate}
\item The rule asserting that unit types are compatible with context extension makes
no sense from the point of view that contexts are lists of types and types
are contexts of length one; in other words, that types are indecomposable (i.e.~non-extended)
families. Remidies:
\begin{enumerate}
\item This point of view is wrong and should be abandoned.
\item We have, as Vladimir proposed, two kinds of judgmental equalities. One
      could be used for the very strict equalities, the other could be used
      to state the compatibility rules with. In other words, the other equality
      is a compatibility relation. If we do that, we shouldn't require that
      if a family is compatible with a type, then the family is a type. The
      compatibility relation could relate things by uniqueness up to unique
      isomorphism.
\item Don't state such compatibility rules for the type constructors.
\item Don't even give the unit type a dependent action.
\end{enumerate}
\end{enumerate}

\subsection{The unit type in structural type theory}
In this section we explore what we get if we pose compatibility conditions on
an inductively defined unit type. We will assume that not only the unit type
is compatible with extension, weakening, substitution and identity functions,
but also its induction principle should be compatible with those.

\begin{align*}
& \inference
  { \jfam{\Gamma}{A}
    }
  { \jtype{{\Gamma}{A}}{\unitc{A}}
    }
& & \inference
    { \jfameq{\Gamma}{A}{A'}
      }
    { \jtypeeq
        {{\Gamma}{A}}
        {\unitc{A}}
        {\unitc{A'}}
      }
  \\
& \inference
    { \jfam{\Gamma}{A}
      }
    { \jterm{{\Gamma}{A}}{\unitc{A}}{\unitct{A}}
      }
& & \inference
    { \jfameq{\Gamma}{A}{A'}
      }
    { \jtermeq{{\Gamma}{A}}{\unitc{A}}{\unitct{A}}{\unitct{A'}}
      }
  \\
& \inference
  { \jfam{{\Gamma}{A}}{P}
    }
  { \jtype{{{{\Gamma}{A}}{P}}{\ctxwk{P}{\unitc{A}}}}{\unitf{A}{P}}
    }
& & \inference
    { \jfameq{{\Gamma}{A}}{P}{P'}
      }
    { \jtypeeq
        {{{{\Gamma}{A}}{P}}{\ctxwk{P}{\unitc{A}}}}
        {\unitf{A}{P}}
        {\unitf{A}{P'}}
      }
  \\
& \inference
  { \jfam{{\Gamma}{A}}{P}
    }
  { \jterm
      {{{{\Gamma}{A}}{P}}{\ctxwk{P}{\unitc{A}}}}
      {\unitf{A}{P}}
      {\unitft{A}{P}}
    }
& & \inference
    { \jfameq{{\Gamma}{A}}{P}{P'}
      }
    { \jtermeq
        {{{{\Gamma}{A}}{P}}{\ctxwk{P}{\unitc{A}}}}
        {\unitf{A}{P}}
        {\unitft{A}{P}}
        {\unitft{A}{P'}}
      }
\end{align*}

We impose the following compatibility rules for the unit type:

\begin{align*}
& \inference
  { \jfam{{\Gamma}{A}}{P}
    }
  { \jtypeeq
      {{{\Gamma}{A}}{P}}
      {\ctxext{\ctxwk{P}{\unitc{A}}}{\unitf{A}{P}}}
      {\unitc{{\Gamma}{A}}}
    }\\
& \inference
  { \jfam{\Gamma}{A}
    \jfam{\Gamma}{B}
    }
  { \jtypeeq{{{\Gamma}{A}}{\ctxwk{A}{B}}}{}{}
    }  
\end{align*}

\subsection{The empty type}

\subsection{The natural numbers}

\subsection{Dependent pair types}

\subsection{Identity types}

\subsection{$\wtypesym$-types}

\subsection{The compatibility of the type constructors with each other}
\label{compatibility-of-type-constructors}

\begin{comment}
\section{Intensional algebras for the theory of contexts, families and terms}
In this section we investigate intensional algebras for the theory of contexts,
families and terms. These are algebras in a weak sense: the judgmental
equalities are replaced by propositional equalities and in some cases coherence
properties are added. The goal writing intensional algebras is twofold: first,
it allows for easy formalization in existing proof assistants such as \Coq;
second, we are shooting at an internal definition of internal weak 
$\infty$-groupoids.

\subsection{Intensional extension algebras}

\begin{defn}
An intensional extension algebra in context $\Gamma$ consists of
\begin{enumerate}
\item A type $C$ in context $\Gamma$
\item A type family $F:C\to\type$
\item A function $\epsilon_0:(\sm{x:C}F(x))\to C$.
\item A function $\epsilon_1:\prd{x:C}(\sm{y:F(x)}F(\epsilon_0(x,y)))\to F(x)$
\end{enumerate}
\end{defn}
\end{comment}


\section{The type theory of models of type theory}
In this section we pursue the idea of what a general model of type theory is by
axiomatizing what you can do with them. We have the following ideas:
\begin{itemize}
\item There are dependent models and sections thereof. Particular instances
  of sections: extension, weakening and substitution (and something for identity
  function?). And like the original
  extension, weakening and substitution, they're going to be compatible with each
  other. Therefore we state our theory of models as an extension of type theory
  without constructors.
\item Type theory without constructors is a model of itself, the canonical model $\mctx$.
\item for any (family of) model(s) $A$, there is the family model $\mfam{A}$ which
  is a family of models over $A$.
\item The terms of $\mctx$ should be precisely contexts. Terms of $\subst{\Gamma}{\mfam{\mctx}}$
  should be families over $\Gamma$.
\item if a model and a family of models over it are given, there is an extended model.
  If we extend $A$ by $\mfam{A}$, we get the Sierpinski model of $A$.
\item Likewise, models can be weakened and substituted and there are identity
  functions.
\item So the theory of models is going to be an extension of this theory with this
  data. The theory we are about to describe can be seen as an elementary theory
  of the category (with families (and terms)) of categories (with families (and terms)).
\item We would also like to remark explicitly that the valid judgments of the original type
  theory become contexts, families or terms, depending on what kind of judgments
  it was. Then, valid inferences of the original type theory become valid
  judgments here. In some sense, the theory we present here is therefore a second
  order theory over the basic theory of types.
\item Martin Escardo and Mike Shulman have been promoting the use of 
  inductive-inductive definitions for internal models. What we're doing here looks
  different, because we're writing down a type theory for internal models, but it
  might not be that different at all. The type theory can be seen as the theory
  of (dependent) algebras over the inductive-inductively defined model and the
  terms of this type theory are the (dependent) algebra homomorphisms. The
  asserted initial object of our type theory is the inductive-inductively defined
  type. 
\end{itemize}

\subsection{The basic ingredients of the type theory of models}
We first introduce the basic ingredients of our abstract theory of models.
\begin{align*}
& \inference
  { }
  { \jctx{\mctx}
    }
  \tag{the canonical model}\\
& \inference
  { \jfam{\mctx}{P}
    }
  { \jterm{\mctx}{P}{i}
    }
  \tag{initiality of $\mctx$}\\
& \inference
  { \jfam{\Gamma}{A}
    }
  { \jfam{{\Gamma}{A}}{\mfam{A}}
    }
  \tag{families}\\
& \inference
  { \jctx{\Gamma}
    }
  { \jfam{{\Gamma}{\mfam{\Gamma}}}{\mtm{\Gamma}}
    }
  \tag{terms}
\intertext{%
  The following two rules essentially make a start with saying that every term is functorial
  in the apropriate sense (there will be more rules contributing to this vies):}
& \inference
  { \jterm{{\Gamma}{A}}{P}{f}
    }
  { \jhom
      {{\Gamma}{A}}
      {\mfam{A}}
      {\subst{f}{\mfam{P}}}
      {\mfam{f}}
    }
  \\
& \inference
  { \jterm{{\Gamma}{A}}{P}{f}
    }
  { \jfhom
      {{\Gamma}{A}}
      {\mfam{A}}
      {\subst{f}{\mfam{P}}}
      {\mfam{f}}
      {\mtm{A}}
      {\subst{f}{\mtm{P}}}
      {\mtm{f}}
    }
\intertext{%
  When the situation requires clarity --  for instance when extensions
  are involved -- we will write $\mfam{f}^\Gamma$ and $\mtm{f}^\Gamma$ to make
  what we regard as the context explicit.}
\intertext{%
The empty context of a model $A$ is simply a term of $A$:}
& \inference
  { \jfam{\Gamma}{A}
    }
  { \jterm{\Gamma}{A}{\tfemp{A}}
    }
  \tag{empty context}\\
& \inference
  { \jfam{\Gamma}{A}
    }
  { \jterm{{{\Gamma}{A}}{\mfam{A}}}{\subst{\tfemp{\mfam{A}}}{\mtm{A}}}{\tft{A}}
    }
\intertext{%
  Extension is going to be a morphism from $\ctxext{A}{\mfam{A}}$ to
  $A$ in context $\Gamma$:}
& \inference
  { \jfam{\Gamma}{A}
    }
  { \jhom{\Gamma}{{A}{\mfam{A}}}{A}{\tfext{A}}
    }
  \tag{extension}
\intertext{%
  The action of weakening on families
  is a morphism from $\ctxwk{\mfam{A}}{\mfam{A}}$ to
  $\mfam{\mfam{A}}$ in context $\ctxext{{\Gamma}{A}}{\mfam{A}}$:}
& \inference
  { \jfam{\Gamma}{A}
    }
  { \jhom
      {{{\Gamma}{A}}{\mfam{A}}}
      {\ctxwk{\mfam{A}}{\mfam{A}}}
      {\mfam{\mfam{A}}}
      {\tfwk{A}^0}
    }
  \tag{weakening}
\intertext{%
  The action of weakening on terms
  is a morphism from $\ctxwk{\mfam{A}}{\mtm{A}}$ to $\mtm{\mfam{\mfam{A}}}$ over
  $\tfwk{A}^0$ in context $\ctxext{{\Gamma}{A}}{\mfam{A}}$:}
& \inference
  { \jfam{\Gamma}{A}
    }
  { \jfhom
      {{{\Gamma}{A}}{\mfam{A}}}
      {\ctxwk{\mfam{A}}{\mfam{A}}}
      {\mfam{\mfam{A}}}
      {\tfwk{A}^0}
      {\ctxwk{\mfam{A}}{\mtm{A}}}
      {\mtm{\mfam{\mfam{A}}}}
      {\tfwk{A}^1}
    }
  \tag{weakening}
\intertext{%
  The action of substitution by a term on families
  is going to be a morphism from $\ctxwk{\mtm{A}}{\mfam{\mfam{A}}}$ to 
  $\ctxwk{\mtm{A}}{{\mfam{A}}{\mfam{A}}}$
  in context $\ctxext{{{\Gamma}{A}}{\mfam{A}}}{\mtm{A}}$:}
& \inference
  { \jfam{\Gamma}{A}
    }
  { \jhom
      {{{{\Gamma}{A}}{\mfam{A}}}{\mtm{A}}}
      {\ctxwk{\mtm{A}}{\mfam{\mfam{A}}}}
      {\ctxwk{\mtm{A}}{{\mfam{A}}{\mfam{A}}}}
      {\tfsubst{A}^0}
    }
  \tag{substitution}
\intertext{%
  Likewise, the action of substitution by a term on terms of those
  families is going to be a morphism from $\ctxwk{\mtm{A}}{\mtm{\mfam{\mfam{A}}}}$
  to $\ctxwk{\mtm{A}}{{\mfam{A}}{\mtm{A}}}$ over $\tfsubst{A}^0$ in context
  $\ctxext{{{\Gamma}{A}}{\mfam{A}}}{\mtm{A}}$:}
& \inference
  { \jfam{\Gamma}{A}
    }
  { \jfhom
      {{{{\Gamma}{A}}{\mfam{A}}}{\mtm{A}}}
      {\ctxwk{\mtm{A}}{\mfam{\mfam{A}}}}
      {\ctxwk{\mtm{A}}{{\mfam{A}}{\mfam{A}}}}
      {\tfsubst{A}^0}
      {\ctxwk{\mtm{A}}{\mtm{\mfam{\mfam{A}}}}}
      {\ctxwk{\mtm{A}}{{\mfam{A}}{\mtm{A}}}}
      {\tfsubst{A}^1}
    }
  \tag{substitution}
\intertext{%
  The identity functions are coded by}
& \inference
  { \jfam{\Gamma}{A}
    }
  { \jterm
      {{{\Gamma}{A}}{\mfam{A}}}
      {\subst{{\idfunc[\mfam{A}]}{\tfwk{A}^0}}{\mtm{\mfam{\mfam{A}}}}}
      {\tfid{A}}
    }
  \tag{identity functions}
\end{align*}
We will call the terms that we introduced here the model constructors.

\subsection{The compatibility rules}
We need to do several things in this section:
\begin{itemize}
\item postulate that ordinary extension, weakening, substitution and identity
functions are compatible with the model constructors.
\item postulate that each of the model constructors is compatible with ordinary
extension, weakening, substitution and identity functions. Actually, we want
that every term of this type theory is compatible with those. I.e.~every term
is a functor/morphism of models. Also the sections.
\item postulate that the model constructors are compatible with each other,
so that they come to model ordinary extension, weakening, substitution and
identity functions respectively.
\item Note that the rules for compatibility with extension are going to explain which
model $\ctxext{A}{P}$ is by telling what the families, the terms, extension,
weakening, substitution and identity functions are. Likewise the rules for
compatibility with weakening and substitution do this for their respective
cases.
\item The easiest set of compatibility rules comes with weakening. The compatibility
rules that deal with substitution are likely going to have to do with the
Yoneda lemma, which we should be able to implement at some point.
\item The model $\subst{\tfemp{A}}{\mtm{A}}$ is going to be initial in the category of
all models over $A$. That means: whenever $P$ is a family over $\subst{\tfemp{A}}
{\mtm{A}}$ there will be a term of $P$. This means in particular that
$\subst{\tfemp{\emptyf}}{\mtm{\emptyf}}$ is going to be $\mctx$.
\end{itemize}

\subsubsection{Families of weakenings}
\begin{equation*}
\inference
  { \jfam{\Gamma}{A}
    \jfam{\Gamma}{B}
    }
  { \jfameq
      {{{\Gamma}{A}}{\ctxwk{A}{B}}}
      {\mfam{\ctxwk{A}{B}}}
      {\ctxwk{A}{\mfam{B}}}
    }.
\end{equation*}

\subsubsection{Terms of weakenings}
\begin{equation*}
\inference
  { \jfam{\Gamma}{A}
    \jfam{\Gamma}{B}
    }
  { \jfameq
    {{{{\Gamma}{A}}{\ctxwk{A}{B}}}{\mfam{\ctxwk{A}{B}}}}
    {\mtm{\ctxwk{A}{B}}}
    {\ctxwk{A}{\mtm{B}}}
    }
\end{equation*}

\subsubsection{Families over families are families over extensions}
\begin{equation*}
\inference
  { \jfam{\Gamma}{A}
    }
  { \jfameq
      {{{\Gamma}{A}}{\mfam{A}}}
      {\mfam{\mfam{A}}}
      {\jcomp{{A}{\mfam{A}}}{\tfext{A}}{\mfam{A}}}
    }
\end{equation*}
\emph{(Note: this rule might be a consequence of an explanation of what $\mfam{{A}{P}}$
is in general, but at the moment I don't see how to do this)}

\subsubsection{Terms of families over families are terms of families over extensions}
\begin{equation*}
\inference
  { \jfam{\Gamma}{A}
    }
  { \jfameq
      {{{{\Gamma}{A}}{\mfam{A}}}{\mfam{\mfam{A}}}}
      {\mtm{\mfam{A}}}
      {\jcomp{}{\tfext{A}}{\mtm{A}}}
      }
\end{equation*}

\subsubsection{Extension acts as the identity on families of families}
The family of which $\tfext{A}$ is a term, is unfolded as
\begin{equation*}
\unfold{\jhom{\Gamma}{{A}{\mfam{A}}}{A}{\tfext{A}}}
\end{equation*}
Because we have the judgmental equality $\ctxext{\Gamma}{{A}{\mfam{A}}}
\jdeq \ctxext{{\Gamma}{A}}{\mfam{A}}$, there is a term
\begin{equation*}
\jhom
  {{{\Gamma}{A}}{\mfam{A}}}
  {\mfam{\mfam{A}}}
  {\subst{\tfext{A}}{\mfam{\ctxwk{\ctxext{A}{\mfam{A}}}{A}}}}
  {\mfam{\tfext{A}}^{\ctxext{\Gamma}{A}}}
\end{equation*}
By the rule asserting that families of families are families of extensions, we
have $\mfam{\mfam{A}}\jdeq\jcomp{}{\tfext{A}}{\mfam{A}}$. By the rule that
families over weakenings are weakenings of families, we have the judgmental
equalities
\begin{equation*}
\subst{\tfext{A}}{\mfam{\ctxwk{\ctxext{A}{\mfam{A}}}{A}}}
\jdeq 
  \unfold{\jcomp{{A}{P}}{\tfext{A}}{\mfam{A}}}
\jdeq 
  \jcomp{{A}{P}}{\tfext{A}}{\mfam{A}}
\end{equation*}
Therefore, we can compare the term $\mfam{\tfext{A}}^{\ctxext{\Gamma}{A}}$
to the identity function $\idfunc[\jcomp{}{\tfext{A}}{\mfam{A}}]$. 
The compatibility
rule for the action of extension on families of families asserts that
\begin{equation*}
\inference
  { \jfam{\Gamma}{A}
    }
  { \jhomeq
      {{{\Gamma}{A}}{\mfam{A}}}
      {\jcomp{}{\tfext{A}}{\mfam{A}}}
      {\jcomp{}{\tfext{A}}{\mfam{A}}}
      {\mfam{\tfext{A}}^{\ctxext{\Gamma}{A}}}
      {\idfunc[\jcomp{}{\tfext{A}}{\mfam{A}}]}
    }
\end{equation*}
\emph{(Note: this rule might be a consequence of an explanation of what $\mfam{{A}{P}}$
is in general, but at the moment I don't see how to do this)}

\subsubsection{Extension acts as the identity on terms of families of families}
We now investigate the nature of the morphism.
\begin{equation*}
\jfhom
  {{{\Gamma}{A}}{\mfam{A}}}
  {\mfam{\mfam{A}}}
  {\subst{\tfext{A}}{\mfam{\ctxwk{\ctxext{A}{\mfam{A}}}{A}}}}
  {\mfam{\tfext{A}}^{\ctxext{\Gamma}{A}}}
  {\mtm{\mfam{A}}}
  {\subst{\tfext{A}}{\mtm{\ctxwk{\ctxext{A}{\mfam{A}}}{A}}}}
  {\mtm{\tfext{A}}^{\ctxext{\Gamma}{A}}}
\end{equation*}
Note that we have the judgmental equality 
$\mtm{\mfam{A}}\jdeq\jcomp{}{\tfext{A}}{\mtm{A}}$. Likewise, we have the
judgmental equality $\subst{\tfext{A}}{\mtm{\ctxwk{\ctxext{A}{\mfam{A}}}{A}}}
\jdeq\jcomp{}{\tfext{A}}{\mtm{A}}$. Thirdly, we have the judgmental equality
$\mfam{\tfext{A}}^{\ctxext{\Gamma}{A}}\jdeq\idfunc[\jcomp{}{\tfext{A}}{\mfam{A}}]$.
We may combine these three facts with \autoref{hom-over-id-is-hom} to see that
\begin{equation*}
\jhom
  {{{{\Gamma}{A}}{\mfam{A}}}{\jcomp{}{\tfext{A}}}{\mfam{A}}}
  {\jcomp{}{\tfext{A}}{\mtm{A}}}
  {\jcomp{}{\tfext{A}}{\mtm{A}}}
  {\mtm{\tfext{A}}^{\ctxext{\Gamma}{A}}}
\end{equation*}
Now we see that we can require that $\mtm{\tfext{A}}^{\ctxext{\Gamma}{A}}$ is
the identity function on $\jcomp{}{\tfext{A}}{\mtm{A}}$:
\begin{equation*}
\inference
  { \jfam{\Gamma}{A}
    }
  { \jhomeq
    {{{{\Gamma}{A}}{\mfam{A}}}{\jcomp{}{\tfext{A}}}{\mfam{A}}}
    {\jcomp{}{\tfext{A}}{\mtm{A}}}
    {\jcomp{}{\tfext{A}}{\mtm{A}}}
    {\mtm{\tfext{A}}^{\ctxext{\Gamma}{A}}}
    {\idfunc[\jcomp{}{\tfext{A}}{\mtm{A}}]}
    }
\end{equation*}

\subsubsection{Weakening followed by substitution is the identity}
The weakening morphism $\tfwk{A}^0$ gives a weakened morphism
\begin{equation*}
\jhom
  {{{{\Gamma}{A}}{\mfam{A}}}{\mtm{A}}}
  {\ctxwk{\mtm{A}}{{\mfam{A}}{\mfam{A}}}}
  {\ctxwk{\mtm{A}}{\mfam{\mfam{A}}}}
  {\ctxwk{\mtm{A}}{\tfwk{A}^0}}
\end{equation*}
which can be composed with the morphism $\tfsubst{A}^0$. We require that this
composition is the identity function on $\ctxwk{\mtm{A}}{{\mfam{A}}{\mfam{A}}}$:
\begin{equation*}
\inference
  { \jfam{\Gamma}{A}
    }
  { \jhomeq
      {{{{\Gamma}{A}}{\mfam{A}}}{\mtm{A}}}
      {\ctxwk{\mtm{A}}{{\mfam{A}}{\mfam{A}}}}
      {\ctxwk{\mtm{A}}{{\mfam{A}}{\mfam{A}}}}
      {\jcomp{}{\ctxwk{\mtm{A}}{\tfwk{A}^0}}{\tfsubst{A}^0}}
      {\idfunc[\ctxwk{\mtm{A}}{{\mfam{A}}{\mfam{A}}}]}
    }
\end{equation*}
Likewise, the weakening morphism $\tfwk{A}^1$ gives a weakened morphism
\begin{equation*}
\jfhom
  {{{{\Gamma}{A}}{\mfam{A}}}{\mtm{A}}}
  {\ctxwk{\mtm{A}}{{\mfam{A}}{\mfam{A}}}}
  {\ctxwk{\mtm{A}}{\mfam{\mfam{A}}}}
  {\ctxwk{\mtm{A}}{\tfwk{A}^0}}
  {\ctxwk{\mtm{A}}{\ctxwk{\mfam{A}}{\mtm{A}}}}
  {\ctxwk{\mtm{A}}{\mtm{\mfam{\mfam{A}}}}}
  {\ctxwk{\mtm{A}}{\tfwk{A}^1}}
\end{equation*}
which can be composed with the the morphism $\tfsubst{A}^1$. We require that
this composition is the identity function on 
$\ctxwk{\mtm{A}}{\ctxwk{\mfam{A}}{\mtm{A}}}$. Note that this composition gives
a morphism over $\jcomp{}{\ctxwk{\mtm{A}}{\tfwk{A}^0}}{\tfsubst{A}^0}$, which
is the identity function on $\ctxwk{\mtm{A}}{{\mfam{A}}{\mfam{A}}}$; so in fact
it is an ordinary morphism in the context
$\ctxext{{{{\Gamma}{A}}{\mfam{A}}}{\mtm{A}}}{\ctxwk{\mtm{A}}{{\mfam{A}}{\mfam{A}}}}$.
Thus the inference rule we require to become valid is:
\begin{equation*}
\inference
  { \jfam{\Gamma}{A}
    }
  { \jhomeq
      {{{{{\Gamma}{A}}{\mfam{A}}}{\mtm{A}}}{\ctxwk{\mtm{A}}{{\mfam{A}}{\mfam{A}}}}}
      {\ctxwk{\mtm{A}}{\ctxwk{\mfam{A}}{\mtm{A}}}}
      {\ctxwk{\mtm{A}}{\ctxwk{\mfam{A}}{\mtm{A}}}}
      {\jcomp{}{\ctxwk{\mtm{A}}{\tfwk{A}^1}}{\tfsubst{A}^1}}
      {\idfunc[\ctxwk{\mtm{A}}{\ctxwk{\mfam{A}}{\mtm{A}}}]}
    }
\end{equation*}

\subsubsection{Weakening by a family is precomposing with extension}
We seem to need the following rule:
\begin{equation*}
\inference
  { \jfam{{\Gamma}{A}}{P}
    }
  { \jfameq
      {{{\Gamma}{A}}{\mfam{A}}}
      {\ctxwk{\mfam{A}}{P}}
      {\jcomp{}{\tfext{A}}{P}}
    }
\end{equation*}
I don't know very well how to motivate requiring this rule. Maybe it's provable?

\subsubsection{Families of extensions}

\subsubsection{Terms of families of extensions}

\subsubsection{Terms of extensions}

\subsubsection{Families over the empty context}
\begin{equation*}
\inference
  { \jfam{\Gamma}{A}
    }
  { \jfameq{\Gamma}{\subst{\tfemp{A}}{\mfam{A}}}{A}
    }
\end{equation*}

\subsubsection{Terms of the empty family}
\begin{equation*}
\inference
  { \jfam{\Gamma}{A}
    }
  { \jfameq{{\Gamma}{A}}{\subst{\tfemp{\mfam{A}}}{\mtm{A}}}{\emptyf[A]}
    }
\end{equation*}

\subsubsection{The empty family over the empty family}
\begin{equation*}
\inference
  { \jfam{\Gamma}{A}
    }
  { \jtermeq{\Gamma}{A}{\subst{\tfemp{A}}{\tfemp{\mfam{A}}}}{\tfemp{A}}
    }
\end{equation*}

\subsubsection{Extension by the empty family}
To understand the following inference rule, recall that for $\jfam{\Gamma}{A}$
we have the judgmental equalities
\begin{equation*}
\subst{\tfemp{\mfam{A}}}{\ctxwk{\ctxext{A}{\mfam{A}}}{A}}
\jdeq
  \subst{\tfemp{\mfam{A}}}{\ctxwk{\mfam{A}}{{A}{A}}}
\jdeq
  \ctxwk{A}{A}.
\end{equation*}
Therefore, we can postulate:
\begin{equation*}
\inference
  { \jfam{\Gamma}{A}
    }
  { \jhomeq{\Gamma}{A}{A}{\subst{\tfemp{\mfam{A}}}{\tfext{A}}}{\idfunc[A]}
    }
\end{equation*}

\subsubsection{Extensions of the empty context}
To understand the following inference rule, recall that for $\jfam{\Gamma}{A}$
we have the judgmental equalities
\begin{align*}
\subst{\tfemp{A}}{\ctxwk{\ctxext{A}{\mfam{A}}}{A}}
& \jdeq 
  \subst{\tfemp{A}}{\ctxwk{\mfam{A}}{{A}{A}}}
  \\
& \jdeq 
  \ctxwk{\subst{\tfemp{A}}{\mfam{A}}}{\subst{\tfemp{A}}{\ctxwk{A}{A}}}
  \\
& \jdeq 
  \ctxwk{A}{\subst{\tfemp{A}}{\ctxwk{A}{A}}}
  \\
& \jdeq 
  \ctxwk{A}{A}
\end{align*}
Therefore, we can postulate
\begin{equation*}
\inference
  { \jfam{\Gamma}{A}
    }
  { \jhomeq{\Gamma}{A}{A}{\subst{\tfemp{A}}{\tfext{A}}}{\idfunc[A]}
    }
\end{equation*}

\subsubsection{Weakening by the empty family}

\subsubsection{Weakenings of the empty family}

\subsubsection{Substitutions of the empty family}

\subsubsection{Substitution by the term of the empty family}

\subsection{Compatibility properties of arbitrary terms}
In this subsection we will state the inference rules that assert that every
term $\jterm{{\Gamma}{A}}{P}{f}$ acts functorially. That means, every term
preserves the empty context, extension, weakening, substitution and the
identity functions. Moreover, there will be several inference rules
involving the behavior of $\mfam{f}$ and $\mtm{f}$.

\subsubsection{The action on families of families of a term is the action on
families over extensions}
Let $\jterm{{\Gamma}{A}}{P}{f}$. Then we have the term
\begin{equation*}
\jterm{{{\Gamma}{A}}{\mfam{A}}}{\ctxwk{\mfam{A}}{\subst{f}{\mfam{P}}}}{\mfam{f}}
\end{equation*}
and we have the term
\begin{equation*}
\jterm
  { {{{\Gamma}{A}}{\mfam{A}}}{\mfam{\mfam{A}}}
    }
  { \ctxwk
      {\mfam{\mfam{A}}}
      {\subst{\mfam{f}}{\mfam{\ctxwk{\mfam{A}}{\subst{f}{\mfam{P}}}}}}
    }
  { \mfam{\mfam{f}}
    }.
\end{equation*}
We have the judgmental equality $\mfam{\mfam{A}}\jdeq
\jcomp{}{\tfext{A}}{\mfam{A}}$; in this subsubsection we wish to establish a
similar judgmental equality explaining what $\mfam{\mfam{f}}$ is. Note that
we have the judgmental equalities
\begin{align*}
\ctxwk
  {\mfam{\mfam{A}}}
  {\subst{\mfam{f}}{\mfam{\ctxwk{\mfam{A}}{\subst{f}{\mfam{P}}}}}}
& \jdeq
  \ctxwk
    {\jcomp{}{\tfext{A}}{\mfam{A}}}
    {\subst{\mfam{f}}{\mfam{\ctxwk{\mfam{A}}{\subst{f}{\mfam{P}}}}}}
  \\
& \jdeq
  \ctxwk
    {\unfold{\jcomp{\ctxext{A}{\mfam{A}}}{\tfext{A}}{\mfam{A}}}}
    {\subst{\mfam{f}}{\mfam{\ctxwk{\mfam{A}}{\subst{f}{\mfam{P}}}}}}
  \\
& \jdeq
  \ctxwk
    {\subst{\tfext{A}}{\ctxwk{\mfam{A}}{{A}{\mfam{A}}}}}
    {\subst{\mfam{f}}{\mfam{\ctxwk{\mfam{A}}{\subst{f}{\mfam{P}}}}}}
  \\
& \jdeq
  \ctxwk
    { \subst{\tfext{A}}{\ctxwk{\mfam{A}}{{A}{\mfam{A}}}}
      }
    { \subst
        { \tfext{A}
          }
        { \ctxwk
            {A}
            {\subst{\mfam{f}}{\mfam{\ctxwk{\mfam{A}}{\subst{f}{\mfam{P}}}}}}}
      }
  \\
& \jdeq
  \subst
    { \tfext{A}
      }
    { \ctxwk
        { {\mfam{A}}{{A}{\mfam{A}}}
          }
        { {A}
          {\subst{\mfam{f}}{\mfam{\ctxwk{\mfam{A}}{\subst{f}{\mfam{P}}}}}}
          }
      }
  \\
& \jdeq \unfold{\jcomp{{A}{\mfam{A}}}{\tfext{A}}{\mfam{\ctxwk{\mfam{A}}{\subst{f}{\mfam{P}}}}}}
\end{align*}


\subsubsection{Every term is compatible with the empty context}
\begin{equation*}
\inference
  { \jterm{{\Gamma}{A}}{P}{f}
    }
  { \jtermeq
      {\Gamma}
      {\subst{\tfemp{A}}{P}}
      {\subst{\tfemp{A}}{f}}
      {\tfemp{\subst{\tfemp{A}}{P}}}
    }
\end{equation*}

\subsubsection{The action on families of the action on families of a term is the action
on contexts of that term}
\emph{In this subsubsection we attempted to establish a rule asserting that
$\subst{\tfemp{A}}{\mfam{f}}\jdeq f$ for any $\jterm{{\Gamma}{A}}{P}{f}$. This
is however not true.}

Consider a term $\jterm{{\Gamma}{A}}{P}{f}$. Then we have the term
\begin{equation*}
\jterm
  {{\Gamma}{\subst{\tfemp{A}}{\mfam{A}}}}
  {\subst{\tfemp{A}}{\ctxwk{\mfam{A}}{\subst{f}{\mfam{P}}}}}
  {\subst{\tfemp{A}}{\mfam{f}}}
\end{equation*}
We have required that $\subst{\tfemp{A}}{\mfam{A}}\jdeq A$. Also, we have the
judgmental equalities
\begin{align*}
\subst{\tfemp{A}}{\ctxwk{\mfam{A}}{\subst{f}{\mfam{P}}}}
& \jdeq
  \ctxwk{\subst{\tfemp{A}}{\mfam{A}}}{\subst{\tfemp{A}}{{f}{\mfam{P}}}}
  \\
& \jdeq
  \ctxwk{A}{\subst{{\tfemp{A}}{f}}{{\tfemp{A}}{\mfam{P}}}}
  \\
& \jdeq
  \ctxwk{A}{\subst{\tfemp{\subst{\tfemp{A}}{P}}}{{\tfemp{A}}{\mfam{P}}}}
  \\
& \jdeq
  \ctxwk{A}{\subst{{\tfemp{A}}{\tfemp{P}}}{{\tfemp{A}}{\mfam{P}}}}
  \\
& \jdeq
  \ctxwk{A}{\subst{\tfemp{A}}{{\tfemp{P}}{\mfam{P}}}}
  \\
& \jdeq
  \ctxwk{A}{\subst{\tfemp{A}}{P}}
\end{align*}
If we require also the following inference rule, we would get $\mfam{A}\jdeq
\ctxwk{A}{A}$.
\begin{equation*}
\inference
  { \jfam{{\Gamma}{A}}{P}
    }
  { \jfameq{{\Gamma}{A}}{P}{\ctxwk{A}{\subst{\tfemp{A}}{P}}}
    }
\end{equation*}
This rule would say that a family $P$ of models over $A$ is determined by the
model $\subst{\tfemp{A}}{P}$ and that the only possible families of models
are the constant families.

\subsubsection{Every term is compatible with extension}
Let $\jterm{{\Gamma}{A}}{P}{f}$. The rule we're about to explain is that $f$
commutes with extension.

We have the morphism
$\jhom{{\Gamma}{A}}{\mfam{A}}{\subst{f}{\mfam{P}}}{\mfam{f}}$. Let's unfold
to remind ourselves what this means:
\begin{equation*}
\unfold{\jhom{{\Gamma}{A}}{\mfam{A}}{\subst{f}{\mfam{P}}}{\mfam{f}}}
\end{equation*}
Now note that we have the judgmental equality
\begin{equation*}
  \ctxwk{\mfam{A}}{\subst{f}{\mfam{P}}}
  \jdeq
  \subst{\ctxwk{\mfam{A}}{f}}{\ctxwk{\mfam{A}}{\mfam{P}}}
\end{equation*}
and we see that we get the term
$ \tmext
    {\ctxwk{\mfam{A}}{P}}
    {\ctxwk{\mfam{A}}{\mfam{P}}}
    {\ctxwk{\mfam{A}}{f}}
    {\mfam{f}}
  $
of the family $\ctxwk{\mfam{A}}{\ctxext{P}{\mfam{P}}}$. We can substitute this
term in the morphism
\begin{equation*}
\jhom
  {{{\Gamma}{A}}{\mfam{A}}}
  {\ctxwk{\mfam{A}}{\ctxext{P}{\mfam{P}}}}
  {\ctxwk{\mfam{A}}{P}}
  {\ctxwk{\mfam{A}}{\tfext{P}}}
\end{equation*}
to obtain the term 
$ \jterm
    {{{\Gamma}{A}}{\mfam{A}}}
    {\ctxwk{\mfam{A}}{P}}
    { \subst
        { \tmext
            {\ctxwk{\mfam{A}}{P}}
            {\ctxwk{\mfam{A}}{\mfam{P}}}
            {\ctxwk{\mfam{A}}{f}}
            {\mfam{f}}
          }
        { \ctxwk{\mfam{A}}{\tfext{P}}
          }
      }.
  $
We can also compose $\tfext{A}$ with $f$ to obtain the term
\begin{equation*}
\jterm{{{\Gamma}{A}}{\mfam{f}}}{\jcomp{}{\tfext{A}}{P}}{\jcomp{}{\tfext{A}}{f}}
\end{equation*}
Note that we have (?) the judgmental equality $\jcomp{}{\tfext{A}}{P}
\jdeq\ctxwk{\mfam{A}}{P}$ and therefore
we can require:
\begin{equation*}
\inference
  { \jterm{{\Gamma}{A}}{P}{f}
    }
  { \jtermeq
       {{{\Gamma}{A}}{\mfam{A}}}
       {\ctxwk{\mfam{A}}{P}}
       { \subst
           { \tmext
               {\ctxwk{\mfam{A}}{P}}
               {\ctxwk{\mfam{A}}{\mfam{P}}}
               {\ctxwk{\mfam{A}}{f}}
               {\mfam{f}}
             }
           { \ctxwk{\mfam{A}}{\tfext{P}}
             }
         }
       {\jcomp{}{\tfext{A}}{f}}
    }
\end{equation*}

\subsubsection{Every term is compatible with weakening}
Let $\jterm{{\Gamma}{A}}{P}{f}$ be a term. We will establish an inference rule
asserting that $f$ commutes with weakening. 

\subsubsection{Every term is compatible with substitution}

\subsubsection{Every term preserves identity}


\section{Pretty type theory}
In this section we will do three things. First we explain basic type theory with
variable names, which we simply call \emph{Pretty Type Theory}. 
Then we will show how every formula in basic type theory without
variable names (shall we call it \emph{Structural Type Theory}?) 
can be interpreted in pretty type theory. Finally, we will show how every formula
in pretty type theory can be interpreted in structural type theory.

Pretty type theory is pretty as the ugly little duckling that grew up. At first, it is actually uglier than
structural type theory because one has to keep track of a variable nobody is
actually interested in. The situation improves when we introduce the operators
of extension and weakening, because from that point onwards one can draw notational
advantages from having the variable around.

Pretty type theory should be such that if you write down two contexts, families
or terms in exactly the same way, then they are the same, and there should be
notational shortcuts for extension, weakening and substitution which make
this interesting.

\subsection{The basic judgments}
The basic judgments of pretty type theory are the same as for structural type
theory. There are judgments for: ``$\Gamma$ is a context'',
``$A(i)$ over $i:\Gamma$ is a family over $\Gamma$'', ``$A(i)$ over $i:\Gamma$ 
is a type in context $\Gamma$''
and ``$x(i)$ is a term of $A(i)$ above $i:\Gamma$''. The other four
judgments are for judgmental equality. 

\begin{align*}
\jvctx*{\Gamma} & \jvctxeq*{\Gamma}{\Gamma'}\\
\jvfam*{i}{\Gamma}{A} & \jvfameq*{i}{\Gamma}{A}{B}\\
\jvtype*{i}{\Gamma}{A} & \jvtypeeq*{i}{\Gamma}{A}{B}\\
\jvterm*{i}{\Gamma}{A}{x} & \jvtermeq*{i}{\Gamma}{A}{x}{y}.
\end{align*}

We have the following basic inference rules that relate types and families:

\begin{small}
\begin{align*}
& \inference
  {\jvtype{i}{\Gamma}{A}}
  {\jvfam{i}{\Gamma}{A}}
& & \inference
    {\jvtypeeq{i}{\Gamma}{A}{B}}
    {\jvfameq{i}{\Gamma}{A}{B}}\\
& \inference
  {\jvtype{i}{\Gamma}{A}
   \jvfameq{i}{\Gamma}{A}{B}}
  {\jvtype{i}{\Gamma}{B}}
& & \inference
    {\jvtype{i}{\Gamma}{A}
     \jvfameq{i}{\Gamma}{A}{B}}
    {\jvtypeeq{i}{\Gamma}{A}{B}}
\end{align*}
\end{small}

\subsection{The basic rules for judgmental equality}
The rules for judgmental equality establish that it is an equivalence relation.
\bgroup\small
\begin{align*}
& \inference
  {\jvctx{\Gamma}}
  {\jvctxeq{\Gamma}{\Gamma}} 
& & \inference
    {\jvctxeq{\Gamma}{\Delta}}
    {\jvctxeq{\Delta}{\Gamma}} 
& & \inference
    {\jvctxeq{\Gamma}{\Delta}
     \jvctxeq{\Delta}{\greek{E}}}
    {\jvctxeq{\Gamma}{\greek{E}}}\\
& \inference
  {\jvfam{i}{\Gamma}{A}}
  {\jvfameq{i}{\Gamma}{A}{A}} 
& & \inference
    {\jvfameq{i}{\Gamma}{A}{B}}
    {\jvfameq{i}{\Gamma}{B}{A}}
& & \inference
    {\jvfameq{i}{\Gamma}{A}{B}
     \jvfameq{i}{\Gamma}{B}{C}}
    {\jvfameq{i}{\Gamma}{A}{C}}\\
& \inference
  {\jvterm{i}{\Gamma}{A}{x}}
  {\jvtermeq{i}{\Gamma}{A}{x}{x}}
& & \inference
    {\jvtermeq{i}{\Gamma}{A}{x}{y}}
    {\jvtermeq{i}{\Gamma}{A}{y}{x}}
& & \inference
    {\jvtermeq{i}{\Gamma}{A}{x}{y}
     \jvtermeq{i}{\Gamma}{A}{y}{z}}
    {\jvtermeq{i}{\Gamma}{A}{x}{z}}
\end{align*}
\egroup

The following convertibility rules are responsible for the strictness
of judgmental equality, which sets it apart from equivalences or identifications:

\begin{align*}
& \inference
  {\jvctxeq{\Gamma}{\Delta}
   \jvfam{i}{\Gamma}{A}}
  {\jvfam{i}{\Delta}{A}}
& & \inference
    {\jvctxeq{\Gamma}{\Delta}
     \jvfameq{i}{\Gamma}{A}{B}}
    {\jvfameq{i}{\Delta}{A}{B}}\\
& \inference
  {\jvctxeq{\Gamma}{\Delta}
   \jvterm{i}{\Gamma}{A}{x}}
  {\jvterm{i}{\Delta}{A}{x}}
& & \inference
    {\jvctxeq{\Gamma}{\Delta}
     \jvtermeq{i}{\Gamma}{A}{x}{y}}
    {\jvtermeq{i}{\Delta}{A}{x}{y}}\\
& \inference
  {\jvfameq{i}{\Gamma}{A}{B}
   \jvterm{i}{\Gamma}{A}{x}}
  {\jvterm{i}{\Gamma}{B}{x}}
& & \inference
    {\jvfameq{i}{\Gamma}{A}{B}
     \jvtermeq{i}{\Gamma}{A}{x}{y}}
    {\jvtermeq{i}{\Gamma}{B}{x}{y}}
\end{align*}

\subsection{The empty context}
The empty context looks a bit strange when we explicitly denote the terms. But
we will not do so anymore after this subsection.

\begin{align}
& \inference
  {}
  {\jctx{\emptyc}}\\
& \inference
  {\jctx{\Gamma}}
  {\jvfam{i}{\Gamma}{\emptyf[\Gamma]}}\\
& \inference
  {\jctx{\Gamma}}
  {\jvterm{i}{\Gamma}{\emptyf[\Gamma]}{\emptytm[\Gamma]}}\\
& \inference
  {\jvterm{i}{\Gamma}{\emptyf[\Gamma]}{x}}
  {\jvtermeq{i}{\Gamma}{\emptyf[\Gamma]}{x}{\emptytm[\Gamma]}}
\end{align}

Moreover, if $\Gamma$ is a context family over the
empty context, then $\Gamma$ is a context and every context is a context
family over the empty context. Note that this allows us to speak
of terms of contexts too.

\begin{align}
& \inference
  {\jctx{\Gamma}}
  {\jvfam{\nameless}{\emptyc}{\Gamma}} 
& & \inference
    {\jvfam{\nameless}{\emptyc}{\Gamma}}
    {\jctx{\Gamma}}\\
& \inference
  {\jctxeq{\Gamma}{\Delta}}
  {\jvfameq{\nameless}{\emptyc}{\Gamma}{\Delta}}
& & \inference
    {\jvfameq{\nameless}{\emptyc}{\Gamma}{\Delta}}
    {\jctxeq{\Gamma}{\Delta}}
\end{align}

\subsubsection{The empty context is compatible with itslef}
The empty context $\emptyc$ may be considered as a family of contexts over the empty
context. When we do this, we get $\emptyf[\emptyc]$.
\begin{equation}
\inference
  {}
  {\jvfameq{\nameless}{\emptyc}{\emptyc}{\emptyf[\emptyc]}}
\end{equation}
In the future, we shall denote $\emptyf[\Gamma]$ by $\emptyf$. The above rule
guarantees that this will not cause confusion. Likewise, we shall denote
$\emptytm[\Gamma]$ by $\emptytm$.

\subsection{Extension}
We introduce extension which not only extends a context $\Gamma$ and a family
$A$ over it to a context $\ctxext{\Gamma}{A}$, but which also extends a family $A$
in context $\Gamma$ and a family $P$ over it to a family $\ctxext{A}{P}$ over context
$\Gamma$. We do this to ensure that all of type theory can be done in a context.
For instance, we could say (1) that a context in context $\Gamma$ is the same thing
as a family over $\Gamma$; (2) When $A$ is a context in this sense, a family over
$A$ is the same thing as a family $P$ over $\ctxext{\Gamma}{A}$ and 
(3) when $P$ is a family over $A$ in this sense, a term of $P$ keeps its original meaning.

\begin{align}
& \inference
  {\jvfam{i}{\Gamma}{A}}
  {\jvfamcombi{{i}{x}}{{\Gamma}{A}}{P}}
& & \inference
    {\jctxeq{\Gamma}{\Delta}
     \jfameq{\Gamma}{A}{B}}
    {\jctxeq{\ctxext{\Gamma}{A}}{\ctxext{\Delta}{B}}}\\
& \inference
  {\jfam{{\Gamma}{A}}{P}}
  {\jfam{\Gamma}{\ctxext{A}{P}}}
& & \inference
    {\jfameq{\Gamma}{A}{B}
     \jfameq{{\Gamma}{A}}{P}{Q}}
    {\jfameq{\Gamma}{\ctxext{A}{P}}{\ctxext{B}{Q}}}
\end{align}

\subsubsection{Extension is compatible with the empty context}
The following rule asserts that extension by $\emptyc$ leaves the contexts unchanged.
\begin{align}
& \inference
  {\jctx{\Gamma}}
  {\jctxeq{\ctxext{\emptyc}{\Gamma}}{\Gamma}}\\
& \inference
  {\jctx{\Gamma}}
  {\jctxeq{\ctxext{\Gamma}{\emptyf}}{\Gamma}}\\
& \inference
  {\jfam{\Gamma}{A}}
  {\jfameq{\Gamma}{\ctxext{\emptyf}{A}}{A}}
\end{align}

\subsubsection{Extension is compatible with itself}\label{comp-ee}
The inference rules asserting that extension is compatible with itself assert
that contexts are unstructured lists of type declarations. This rule is
unavoidable if we want that for a family $A$ in context $\Gamma$, a family over
$A$ is the same thing as a family over $\ctxext{\Gamma}{A}$. 

\begin{align}
& \inference
  {\jfam{\Gamma}{A}
   \jfam{{\Gamma}{A}}{P}}
  {\jctxeq{\ctxext{{\Gamma}{A}}{P}}{\ctxext{\Gamma}{{A}{P}}}}\\
& \inference
  {\jfam{{\Gamma}{A}}{P}
   \jfam{{{\Gamma}{A}}{P}}{Q}}
  {\jfameq{\Gamma}{\ctxext{{A}{P}}{Q}}{\ctxext{A}{{P}{Q}}}}
\end{align}


\section{Chalmers type theory}
In this section we will present the type theory that Coquand and Dybjer are using.
It is a weak type theory (I think), with not so many operations and judgmental equalities.
We will show how every formula of Chalmers type theory can be interpreted in
structural type theory.


%\section{Identity types}
The identity types may now be formulated just as in ordinary Martin\nobreakdash-L\"of
type theory. However, since all model categories have path objects, we will also
introduce identity types for contexts.

We will use the symbol $\reflsym$ slightly different than the book does. For us,
$\refl{A}$ is a term of $\subst{\idfunc[A]}{\idtypevar{A}}$ in context
$\ctxext{\Gamma}{A}$ and we will write $\subst{x}{\refl{A}}$ for the reflexivity
path at $x$ (which would have been denoted by $\refl{x}$ in the book).

A notable difference in the formulation of identity types is that in our current
setting we must state the elimination rule not only for families $P$ in context
$\ctxext{{\Gamma}{\ctxwk{\Gamma}{\Gamma}}}{\idtypevar{\Gamma}}$,
but also for families $Q$ in context 
$\ctxext{{{\Gamma}{\ctxwk{\Gamma}{\Gamma}}}{\idtypevar{\Gamma}}}{P}$. The reason
is that all operations have to be closed under slicing: everything may happen
in a context. Secretly, a reason is that we don't have dependent function types.
We wouldn't even be able to find the transport maps if we didn't state the
identity elimination in an extended context.

\begin{align}
& \inference{\jctx{\Gamma}}{\jtype{\ctxext{\Gamma}{\ctxwk{\Gamma}{\Gamma}}}{\idtypevar{\Gamma}}}\\
& \inference{\jctx{\Gamma}}{\jterm{\Gamma}{\subst{\idfunc[\Gamma]}{\idtypevar{\Gamma}}}{\refl{\Gamma}}}\\
& \inference{\jtype{\ctxext{{\Gamma}{\ctxwk{\Gamma}{\Gamma}}}{\idtypevar{\Gamma}}}{P}
           \qquad
           \jterm{\Gamma}{\subst{\refl{\Gamma}}{{\idfunc[\Gamma]}{P}}}{d}}
           {\jterm{\ctxext{{\Gamma}{\ctxwk{\Gamma}{\Gamma}}}{\idtypevar{\Gamma}}}{P}{\tfJ(d)}}\\
& \inference{\jtype{\ctxext{{\Gamma}{\ctxwk{\Gamma}{\Gamma}}}{\idtypevar{\Gamma}}}{P}\qquad
\jterm{\Gamma}{\subst{\refl{\Gamma}}{{\idfunc[\Gamma]}{P}}}{d}}
{\jtermeq
  {\ctxext{{\Gamma}{\ctxwk{\Gamma}{\Gamma}}}{\idtypevar{\Gamma}}}
  {\subst{\refl{\Gamma}}{{\idfunc[\Gamma]}{P}}}
  {\subst{\refl{\Gamma}}{{\idfunc[\Gamma]}{\tfJ(d)}}}
  {d}}\\
& \inference{\jtype{\ctxext{{{\Gamma}{\ctxwk{\Gamma}{\Gamma}}}{\idtypevar{\Gamma}}}{P}}{Q}
           \qquad
           \jterm{\ctxext{\Gamma}{\subst{\refl{\Gamma}}{{\idfunc[\Gamma]}{P}}}}{\subst{\refl{\Gamma}}{{\idfunc[\Gamma]}{Q}}}{d}}
           {\jterm{\ctxext{{{\Gamma}{\ctxwk{\Gamma}{\Gamma}}}{\idtypevar{\Gamma}}}{P}}{Q}{\tfJ(d)}}\\
& \inference
  {\jtype
    {\ctxext{{{\Gamma}{\ctxwk{\Gamma}{\Gamma}}}{\idtypevar{\Gamma}}}{P}}
    {Q}
  \qquad
  \jterm
    {\ctxext{\Gamma}{\subst{\refl{\Gamma}}{{\idfunc[\Gamma]}{P}}}}
    {\subst{\refl{\Gamma}}{{\idfunc[\Gamma]}{Q}}}
    {d}}
  {\jtermeq
    {\ctxext{{{\Gamma}{\ctxwk{\Gamma}{\Gamma}}}{\idtypevar{\Gamma}}}{P}}
    {\subst{\refl{\Gamma}}{{\idfunc[\Gamma]}{Q}}}
    {\subst{\refl{\Gamma}}{{\idfunc[\Gamma]}{\tfJ(d)}}}
    {d}}
\end{align}

Likewise, we introduce identity types in a context.

\begin{align}
& \inference{\jtype{\Gamma}{A}}{\jtype{\ctxext{{\Gamma}{A}}{\ctxwk{A}{A}}}{\idtypevar{A}}}\\
& \inference{\jtype{\Gamma}{A}}{\jterm{\ctxext{\Gamma}{A}}{\subst{\idfunc[A]}{\idtypevar{A}}}{\refl{A}}}\\
& \inference{\jtype{\ctxext{{{\Gamma}{A}}{\ctxwk{A}{A}}}{\idtypevar{A}}}{P}\qquad
\jterm{\ctxext{\Gamma}{A}}{\subst{\refl{A}}{{\idfunc[A]}{P}}}{d}}
{\jterm{\ctxext{{{\Gamma}{A}}{\ctxwk{A}{A}}}{\idtypevar{A}}}{P}{\tfJ(d)}}\\
& \inference{\jtype{\ctxext{{{\Gamma}{A}}{\ctxwk{A}{A}}}{\idtypevar{A}}}{P}\qquad
\jterm{\ctxext{\Gamma}{A}}{\subst{\refl{A}}{{\idfunc[A]}{P}}}{d}}
{\jtermeq
  {\ctxext{{{\Gamma}{A}}{\ctxwk{A}{A}}}{\idtypevar{A}}}
  {\subst{\refl{A}}{{\idfunc[A]}{P}}}
  {\subst{\refl{A}}{{\idfunc[A]}{\tfJ(d)}}}
  {d}}
\end{align}

Suppose we have terms $\jterm{\Gamma}{A}{x}$ and $\jterm{\Gamma}{A}{y}$. Then
we may define $\id[A]{x}{y}\defeq\subst{y}{{x}{\idtypevar{A}}}$. A term
$\jterm{\Gamma}{\id[A]{x}{y}}{p}$ is called an identification of $x$ and $y$.

\subsubsection{Basic properties of identity types}
In this subsubsection we prove some basic properties of identity types, just to
know whether we got the type theory right.

Suppose we have a family $\jtype{\ctxext{\Gamma}{A}}{P}$. Then we can consider
the family $\jtype{\ctxext{{\Gamma}{A}}{\ctxwk{A}{A}}}{\ctxwk{A}{P}}$, which has the role of the family $\jtype{\Gamma,\,x,y:A}{P(y)}$
of ordinary Martin-L\"of type theory. We may also consider the family
$\jtype{\ctxext{{\Gamma}{A}}{\ctxwk{A}{A}}}{\ctxwk{{A}{A}}{P}}$; this one has the
role of the family $\jtype{\Gamma,\,x,y:A}{P(x)}$. Those are families in the
same context, so we have
\begin{equation*}
\jtype{\ctxext{{{\Gamma}{A}}{\ctxwk{A}{A}}}{\ctxwk{{A}{A}}{P}}}{\ctxwk{\ctxwk{{A}{A}}{P}}{{A}{P}}}
\end{equation*}

\begin{lem}
There is a term
\begin{equation*}
\jterm{\ctxext{{{{\Gamma}{A}}{\ctxwk{A}{A}}}{\idtypevar{A}}}{\ctxwk{\idtypevar{A}}{{{A}{A}}{P}}}}{\ctxwk{\idtypevar{A}}{{{{A}{A}}{P}}{{A}{P}}}}{\transfibf{P}}
\end{equation*}
\end{lem}

\begin{proof}
By identity elimination it suffices to find a term
\begin{equation*}
\jterm{\ctxext{\Gamma}{A}}{\subst{\refl{A}}{{\idfunc[A]}{\ctxwk{\idtypevar{A}}{{{{A}{A}}{P}}{{A}{P}}}}}}{t}
\end{equation*}
By the judgmental equality $\jtypeeq{\Gamma}{\subst{{x}{f}}{{x}{Q}}}{\subst{x}{{f}{Q}}}$
it follows that we have the judgmental equalities
\begin{align*}
& \subst{\refl{A}}{{\idfunc[A]}{\ctxwk{\idtypevar{A}}{{{{A}{A}}{P}}{{A}{P}}}}}\\
& \qquad \jdeq \subst{\idfunc[A]}{{\ctxwk{{A}{A}}{\refl{A}}}{\ctxwk{\idtypevar{A}}{{{{A}{A}}{P}}{{A}{P}}}}}\\
& \qquad \jdeq \subst{\idfunc[A]}{\ctxwk{{{A}{A}}{P}}{{A}{P}}}\\
& \qquad \jdeq \ctxwk{\subst{\idfunc[A]}{\ctxwk{{A}{A}}{P}}}{\subst{\idfunc[A]}{\ctxwk{A}{P}}}\\
& \qquad \jdeq \ctxwk{P}{P}
\end{align*}
in context $\ctxext{{\Gamma}{A}}{P}$. We have the term $\jterm{\ctxext{{\Gamma}{A}}{P}}{\ctxwk{P}{P}}{\idfunc[P]}$.
\end{proof}

Suppose we have a term $\jterm{\ctxext{\Gamma}{A}}{P}{f}$. 

Using identity types, we can assert that a function $\jhom{\Gamma}{\Delta}{f}$ has
a left inverse $\jhom{\Delta}{\Gamma}{g}$ by asserting that there is an identification
\begin{equation*}
..
\end{equation*}

\subsubsection{Compatibility of identity types with extension}

\subsubsection{Compatibility of identity types with weakening}
Suppose $A$ and $B$ are types in context $\Gamma$. Then we can consider the types
\begin{align*}
\jtype*{\ctxext{{{\Gamma}{A}}{\ctxwk{A}{B}}}{\ctxwk{A}{{B}{B}}}}{\ctxwk{A}{\idtypevar{B}}}\\
\jtype*{\ctxext{{{\Gamma}{A}}{\ctxwk{A}{B}}}{\ctxwk{{A}{B}}{{A}{B}}}}{\idtypevar{\ctxwk{A}{B}}}
\end{align*}
Note that we have the judgmental equality
\begin{equation*}
\jtypeeq
  {\ctxext{{\Gamma}{A}}{\ctxwk{A}{B}}}
  {\ctxwk{A}{{B}{B}}}
  {\ctxwk{{A}{B}}{{A}{B}}}
\end{equation*}
so $\ctxwk{A}{\idtypevar{B}}$ and $\idtypevar{\ctxwk{A}{B}}$ are types in the
same context. 

\begin{lem}
There is a term of type
\begin{equation*}
\jterm
  {\ctxext{{{{\Gamma}{A}}{\ctxwk{A}{B}}}{\ctxwk{{A}{B}}{{A}{B}}}}{\idtypevar{\ctxwk{A}{B}}}}
  {\ctxwk{\idtypevar{\ctxwk{A}{B}}}{\ctxwk{A}{\idtypevar{B}}}}
  {\typefont{idwktowkid}}
\end{equation*}
\end{lem}

\begin{proof}
By the identity elimination rule it suffices to find a term
\begin{equation*}
\jterm
  {\ctxext{{\Gamma}{A}}{\ctxwk{A}{B}}}
  {\subst{\refl{\ctxwk{A}{B}}}{{\idfunc[\ctxwk{A}{B}]}{\ctxwk{\idtypevar{\ctxwk{A}{B}}}{\ctxwk{A}{\idtypevar{B}}}}}}
  {i}
\end{equation*}
We may simplify the type $\subst{\refl{\ctxwk{A}{B}}}{{\idfunc[\ctxwk{A}{B}]}{\ctxwk{\idtypevar{\ctxwk{A}{B}}}{\ctxwk{A}{\idtypevar{B}}}}}$ as follows:
\begin{align*}
& \subst{\refl{\ctxwk{A}{B}}}{{\idfunc[\ctxwk{A}{B}]}{\ctxwk{\idtypevar{\ctxwk{A}{B}}}{\ctxwk{A}{\idtypevar{B}}}}}\\
& \qquad\jdeq \subst{\idfunc[\ctxwk{A}{B}]}{{\ctxwk{..}{\refl{\ctxwk{A}{B}}}}{\ctxwk{\idtypevar{\ctxwk{A}{B}}}{\ctxwk{A}{\idtypevar{B}}}}}
\end{align*}
\end{proof}

\subsubsection{Compatibility of identity types with substitution}

\subsection{Generalized identity types}

\begin{infarray}{c}
\inference{\jterm{\Gamma}{A}{x}}{\jtype{\ctxext{\Gamma}{A}}{\eqtype{x}}}\\
\inference{\jterm{\Gamma}{A}{x}}{\jterm{\Gamma}{\subst{x}{\eqtype{x}}}{\refl{x}}}\\
\inference{\jtype{\ctxext{{\Gamma}{A}}{\eqtype{x}}}{P}\qquad\jterm{\Gamma}{\subst{\refl{x}}{{x}{P}}}{d}}
          {\jterm{\ctxext{{\Gamma}{A}}{\eqtype{x}}}{P}{\ind{\eqtype{x}}(d)}}\\
\inference{\jtype{\ctxext{{\Gamma}{A}}{\eqtype{x}}}{P}\qquad\jterm{\Gamma}{\subst{\refl{x}}{{x}{P}}}{d}}
          {\jtermeq{\ctxext{{\Gamma}{A}}{\eqtype{x}}}{\subst{\refl{x}}{{x}{P}}}{\subst{\refl{x}}{{x}{\ind{\eqtype{x}}(d)}}}{d}}
\end{infarray}

We get something that looks like Paulin-Mohring equality. But we should be able to use it to
show that every $\jterm{\Gamma}{\ctxwk{\Gamma}{\Delta}}{f}$ is a trivial cofibration.

For the following conjecture, note that if we have $\jterm{\Gamma}{\ctxwk{\Gamma}{\Delta}}{f}$
Then we may consider the type $\jtype{\ctxext{{\Gamma}{\ctxwk{\Gamma}{\Delta}}}{\ctxwk{\Gamma}{{\Delta}{\Delta}}}}
{\ctxwk{\Gamma}{\idtypevar{\Delta}}}$ and we may substitute $f$ to obtain the type
\begin{equation*}
\jtype{\ctxext{\Gamma}{\subst{f}{\ctxwk{\Gamma}{{\Delta}{\Delta}}}}}
{\subst{f}{\ctxwk{\Gamma}{\idtypevar{\Delta}}}}
\end{equation*}
Note that we have the judgmental equality
\begin{equation*}
\jtypeeq{\Gamma}{\subst{f}{\ctxwk{\Gamma}{{\Delta}{\Delta}}}}{\ctxwk{\Gamma}{\Delta}}
\end{equation*}
so we obtain the type
\begin{equation*}
\jtype{\ctxext{\Gamma}{\ctxwk{\Gamma}{\Delta}}}
{\subst{f}{\ctxwk{\Gamma}{\idtypevar{\Delta}}}}
\end{equation*}
We also have $\jtype{\ctxext{\Gamma}{\ctxwk{\Gamma}{\Delta}}}{\eqtype{f}}$, so we
may ask for a term of type $\ctxwk{\eqtype{f}}{\subst{f}{\ctxwk{\Gamma}{\idtypevar{\Delta}}}}$
in context $\ctxext{\Gamma}{\ctxwk{\Gamma}{\Delta}}$ and we can ask ourselves the question
whether this is a trivial cofibration. This is the content of the following conjecture.

\begin{conj}
There is a term
\begin{equation*}
\jterm{\ctxext{{\Gamma}{\ctxwk{\Gamma}{\Delta}}}{\eqtype{f}}}{\ctxwk{\eqtype{f}}{\subst{f}{\ctxwk{\Gamma}{\idtypevar{\Delta}}}}}{\eqtoid{f}}
\end{equation*}
Moreover, $\eqtoid{f}$ is a trivial cofibration.
\end{conj}

\begin{proof}
We use the induction principle of $\eqtype{f}$ to construct $\eqtoid{f}$. Thus, we
have to find a term
\begin{equation*}
\jterm
  {\Gamma}
  {\subst{\refl{f}}{{f}{\ctxwk{\eqtype{f}}{\subst{f}{\ctxwk{\Gamma}{\idtypevar{\Delta}}}}}}}
  {\subst{\refl{f}}{{f}{\eqtoid{f}}}}
\end{equation*}
Note that we have the judgmental equalities
\begin{align*}
\subst{\refl{f}}{{f}{\ctxwk{\eqtype{f}}{\subst{f}{\ctxwk{\Gamma}{\idtypevar{\Delta}}}}}}
& \jdeq \subst{f}{{\ctxwk{{\Gamma}{\Delta}}{\refl{f}}}{\ctxwk{\eqtype{f}}{\subst{f}{\ctxwk{\Gamma}{\idtypevar{\Delta}}}}}} \\
& \jdeq \subst{f}{\ctxwk{\Gamma}{\subst{f}{\ctxwk{\Gamma}{\idtypevar{\Delta}}}}}
\end{align*}
where we find the term $\subst{f}{\ctxwk{\Gamma}{\subst{f}{\ctxwk{\Gamma}{\refl{\Delta}}}}}$.
\end{proof}

\begin{conj}
Suppose we have a type $\eqtype{x}$ for every term $\jterm{\Gamma}{A}{x}$. Then
every term $\jterm{\Gamma}{\ctxwk{\Gamma}{\Delta}}{f}$ is a trivial cofibration.
\end{conj}

\begin{proof}
Suppose that $Q$ is a type in context $\Delta$. We want to show that
\begin{equation*}
\inference{\jterm{\Gamma}{\subst{f}{\ctxwk{\Gamma}{Q}}}{g}}{\jterm{\Delta}{Q}{\tilde{g}}}
\end{equation*}
Let $g$ be a term of $\subst{f}{\ctxwk{\Gamma}{Q}}$ in context $\Gamma$. We get
the term
\begin{equation*}
\jterm{\Gamma}{\subst{\refl{f}}{{f}{\ctxwk{\eqtype{f}}{{\Gamma}{Q}}}}}{\subst{\refl{f}}{\ctxwk{\eqtype{f}}{g}}}
\end{equation*}
which is judgmentally equal to $g$; the types

For any term $\jterm{\Gamma}{\subst{f}{\ctxwk{\Gamma}{Q}}}{g}$ we get a term 
$\jterm{\Gamma}{\subst{\refl{x}}{{x}{P}}}{d}$.

$\jterm{\ctxext{{\Gamma}{\ctxwk{\Gamma}{\Delta}}}{\eqtype{f}}}{\ctxwk{\Gamma}{Q}}{\ind{\eqtype{f}}(d)}$.
\end{proof}

However, we could take other classes of terms, such as the projections
$\proj1:\ctxwk{\ctxext{\Gamma}{A}}{\Gamma}$. Let's see how that goes:

\begin{infarray}{c}
\inference{\jtype{\Gamma}{A}}{\jtype{\ctxext{{\Gamma}{A}}{\ctxwk{\ctxext{\Gamma}{A}}{\Gamma}}}{\eqtype{\proj1^A}}}\\
\inference{\jtype{\Gamma}{A}}{\jterm{\ctxext{\Gamma}{A}}{\subst{\proj1^A}{\eqtype{\proj1^A}}}{\refl{\proj1^A}}}
\end{infarray}



%\section{Introducing the type constructors}
We will now describe the rules for the type constructors that we don't assume
in our type theory by default. The first of these are the dependent function
types. We also discuss $\tfW$-types, suspensions, truncations. More general higher inductive
types will have to wait until we have introduced models, for the models of the
basic type theory because we will use them as index categories of the diagrams.

\subsection{The universal property of dependent pair types}
This section shouldn't be about extension anymore.

Using weakening and substitution we are able to state the universal property
for extension. It looks a bit more involved, since we cannot directly refer
to the variables in the contexts. On the other hand, we can now plainly see
were there were secretly weakenings going on.

We begin with stating the universal property of the extension $\ctxext{\Gamma}{A}$.
In these rules we assume we have $\jtype{\Gamma}{A}$ in the hypotheses.
\begin{align}
& \inference{}
{\jterm{\ctxext{\Gamma}{A}}{\ctxwk{A}{{\Gamma}{\ctxext{\Gamma}{A}}}}{\pair_A}}\\
& \inference{
  \jtype{\ctxext{\Gamma}{A}}{P}
  \qquad
  \jterm{\ctxext{\Gamma}{A}}{\subst{\pair_A}{\ctxwk{A}{{\Gamma}{P}}}}{f}}
  {\jterm{\ctxext{\Gamma}{A}}{P}{\ind{\tfext_\Gamma(A)}(f)}}\\
& \inference{
  \jtype{\ctxext{\Gamma}{A}}{P}
  \qquad
  \jterm{\ctxext{\Gamma}{A}}{\subst{\pair_A}{\ctxwk{A}{{\Gamma}{P}}}}{f}}
  {\jtermeq{\ctxext{\Gamma}{A}}{\subst{\pair_A}{\ctxwk{A}{{\Gamma}{P}}}}{\subst{\pair_A}{\ctxwk{A}{{\Gamma}{\ind{\tfext_\Gamma(A)}(f)}}}}{f}}
\end{align}

Under the hypothesis that $\jtypeeq{\Gamma}{A}{A'}$
we will also have the rules

\begin{align}
& \inference{}{\jtermeq{\ctxext{\Gamma}{A}}{\ctxwk{A}{{\Gamma}{\ctxext{\Gamma}{A}}}}{\pair_A}{\pair_{A'}}}\\
& \inference{\jtype{\ctxext{\Gamma}{A}}{P}\qquad\jterm{\ctxext{\Gamma}{A}}{\subst{\pair_A}{\ctxwk{A}{{\Gamma}{P}}}}{f}}
{\jtermeq{\ctxext{\Gamma}{A}}{P}{\ind{\tfext_\Gamma(A)}(f)}{\ind{\tfext_\Gamma(A')}(f)}}
\end{align}

Note that we don't need the notion of terms for contexts to state the universal
property of context extension (which is a good thing, for we don't assume to have them).

Next, we give the universal property of the extension $\ctxext{A}{P}$ in context
$\Gamma$.
In all of the following inference rules we assume that $\jtype{\ctxext{\Gamma}{A}}{P}$
is among the hypothesis. The induction principle for extension consists of three
inference rules:

\begin{align}
& \inference{}
{\jterm{\ctxext{{\Gamma}{A}}{P}}{\ctxwk{P}{{A}{\ctxext{A}{P}}}}{\pair_P}}\\
& \inference{
  \jtype{\ctxext{\Gamma}{{A}{P}}}{Q}
  \qquad
  \jterm{\ctxext{{\Gamma}{A}}{P}}{\subst{\pair_P}{\ctxwk{P}{{A}{Q}}}}{f}}
  {\jterm{\ctxext{\Gamma}{{A}{P}}}{Q}{\ind{\tfext_A(P)}(f)}}\\
& \inference{
  \jtype{\ctxext{\Gamma}{{A}{P}}}{Q}
  \qquad
  \jterm{\ctxext{{\Gamma}{A}}{P}}{\subst{\pair_P}{\ctxwk{P}{{A}{Q}}}}{f}}
  {\jtermeq{\ctxext{{\Gamma}{A}}{P}}{\subst{\pair_P}{\ctxwk{P}{{A}{Q}}}}{\subst{\pair_P}{\ctxwk{P}{{A}{\ind{\tfext_A(P)}(f)}}}}{f}}
\end{align}

As with context extension, we shall require two more inference rules stating that
$\pair_P$ and $\ind{\tfext_A(P)}$ are invariant under judgmental equality.

\subsection{The unit type}
Since we don't have a notion of terms of a context, we just say that the context
$\unit$ is the terminal context.

\begin{align}
& \inference{}{\jctx{\unit}}\\
& \inference{\jctx{\Gamma}}{\jhom{\Gamma}{\unit}{\tounit{\Gamma}}}\\
& \inference{\jctx{\Gamma}\qquad\jhom{\Gamma}{\unit}{f}}{\jhomeq{\Gamma}{\unit}{f}{\tounit{\Gamma}}}
\end{align}

Note that we don't have to require that $\jhomeq{\Gamma}{\unit}{\tounit{\Gamma}}{\tounit{\Delta}}$
whenever we have a judgmental equality $\jctxeq{\Gamma}{\Delta}$, since this already follows from the third rule.

When the context $\unit$ is present, we may use the expression $\jtype{}{\Gamma}$
as a shorthand for the judgment $\jtype{\unit}{\ctxwk{\unit}{\Gamma}}$. Likewise,
we may use the expression $\jtermc{\Gamma}{i}$ as a shorthand
for the judgment $\jhom{\unit}{\Gamma}{i}$. If we have a type $A$ in context
$\Gamma$, we may use the expression $\jtype{}{\subst{i}{A}}$ to stand for
the judgment $\jtype{\unit}{\subst{i}{\ctxwk{\unit}{A}}}$. We see that in every
respect, contexts are types in the empty context.

We have created a strictly terminal object $\unit$. This is not necessary when
we're working in a context. We introduce the unit type $\unit_\Gamma$ in context
$\Gamma$ in the familiar type theoretical way.

\begin{align}
& \inference{\jctx{\Gamma}}{\jtype{\Gamma}{\unit_\Gamma}}\\
& \inference{\jctx{\Gamma}}{\jterm{\Gamma}{\unit_\Gamma}{\ttt_\Gamma}}\\
& \inference{\jtype{\ctxext{\Gamma}{\unit_\Gamma}}{P}\qquad\jterm{\Gamma}{\subst{\ttt_\Gamma}{P}}{u}}
          {\jterm{\ctxext{\Gamma}{\unit_\Gamma}}{P}{\ind{\unit_\Gamma}(u)}}\\
& \inference{\jtype{\ctxext{\Gamma}{\unit_\Gamma}}{P}\qquad\jterm{\Gamma}{\subst{\ttt_\Gamma}{P}}{u}}
          {\jtermeq{\Gamma}{\subst{\ttt_\Gamma}{P}}{\subst{\ttt_\Gamma}{\ind{\unit_\Gamma}(u)}}{u}}
\end{align}

\subsection{Subterminal types}
The subterminal types we're about to present are strict, so they're more like \verb+Prop+
in \Coq. We can define subterminal types in two ways: the first equalizes all elements
of the subject type; the second is universal with the property that for every constant
map factors through it.

\subsubsection{Equalizing all terms}

\subsubsection{Factorizing constant maps}
Let $\jhom{\Gamma}{\Delta}{f}$. We can weaken $f$ by $\Gamma$ to obtain a term
$\jterm{\ctxext{\Gamma}{\ctxwk{\Gamma}{\Gamma}}}{\ctxwk{\Gamma}{{\Gamma}{\Delta}}}{\ctxwk{\Gamma}{f}}$.
This term is `like the function $\lam{x}{y}f(y)$'. Likewise, we can weaken $f$
by $\ctxwk{{\Gamma}{\Gamma}}$ to obtain a term
$\jterm{\ctxext{\Gamma}{\ctxwk{\Gamma}{\Gamma}}}{\ctxwk{\Gamma}{{\Gamma}{\Delta}}}{\ctxwk{{\Gamma}{\Gamma}}{f}}$,
which is `like the function $\lam{x}{y}f(x)$'. Since we have
the judgmental equality 
$\jtypeeq{\ctxext{\Gamma}{\ctxwk{\Gamma}{\Gamma}}}{\ctxwk{{\Gamma}{\Gamma}}{{\Gamma}{\Delta}}}{\ctxwk{\Gamma}{{\Gamma}{\Delta}}}$ we can consider the judgmental equality between $\ctxwk{\Gamma}{f}$
and $\ctxwk{{\Gamma}{\Gamma}}{f}$, which is exactly what we'll do in the definition
of judgmentally constant.

\begin{defn}
A term $\jhom{\Gamma}{\Delta}{f}$ is said to be \emph{judgmentally constant} if
the judgment
\begin{equation*}
\jtermeq{\ctxext{\Gamma}{\ctxwk{\Gamma}{\Gamma}}}{\ctxwk{{\Gamma}{\Gamma}}{{\Gamma}{\Delta}}}{\ctxwk{\Gamma}{f}}{\ctxwk{{\Gamma}{\Gamma}}{f}}
\end{equation*}
can be derived.
\end{defn}

\begin{defn}
\begin{align}
& \inference{\jctx{\Gamma}}{\jctx{\tau\Gamma}}\\
& \inference{\jctx{\Gamma}}{\jhom{\Gamma}{\tau\Gamma}{t}}\\
& \inference{\jctx{\Gamma}}{\jtermeq{\ctxext{\Gamma}{\ctxwk{\Gamma}{\Gamma}}}{\ctxwk{{\Gamma}{\Gamma}}{{\Gamma}{\Delta}}}{\ctxwk{\Gamma}{t}}{\ctxwk{{\Gamma}{\Gamma}}{t}}}\\
& \inference{\jhom{\Gamma}{\Delta}{f}\qquad\jtermeq{\ctxext{\Gamma}{\ctxwk{\Gamma}{\Gamma}}}{\ctxwk{{\Gamma}{\Gamma}}{{\Gamma}{\Delta}}}{\ctxwk{\Gamma}{f}}{\ctxwk{{\Gamma}{\Gamma}}{f}}}{\jhom{\tau\Gamma}{\Delta}{\tilde{f}}}\\
& \inference{\jhom{\Gamma}{\Delta}{f}\qquad\jtermeq{\ctxext{\Gamma}{\ctxwk{\Gamma}{\Gamma}}}{\ctxwk{{\Gamma}{\Gamma}}{{\Gamma}{\Delta}}}{\ctxwk{\Gamma}{f}}{\ctxwk{{\Gamma}{\Gamma}}{f}}}
{\jtermeq{\Gamma}{\ctxwk{\Gamma}{\Delta}}{\jcomp{\Gamma}{t}{\tilde{f}}}{f}}
\end{align}
\end{defn}

\begin{lem}
Any term $\jhom{\unit}{\Gamma}{i}$ is judgmentally constant.
\end{lem}

\begin{proof}

\end{proof}

\subsection{Product types}

\subsubsection{Products}
\begin{align}
& \inference{\jctx{\Gamma}\qquad\jctx{\Delta}}{\jctx{\product{\Gamma}{\Delta}}}\\
& \inference{\jctx{\Gamma}\qquad\jctx{\Delta}}{\jhom{{\Gamma}{\Delta}}{\product{\Gamma}{\Delta}}{\pair}}
\end{align}

\subsubsection{Strict products}
As with the unit type, we may use the categorical description of the product
for our type theoretical definition. If we do that, we get strict products.

\begin{align}
& \inference{\jctx{\Gamma}\qquad\jctx{\Delta}}{\jctx{\product{\Gamma}{\Delta}}}\\
& \inference{\jctx{\Gamma}\qquad\jctx{\Delta}}{\jhom{\product{\Gamma}{\Delta}}{\Gamma}{\proj1}}\\
& \inference{\jctx{\Gamma}\qquad\jctx{\Delta}}{\jhom{\product{\Gamma}{\Delta}}{\Delta}{\proj2}}\\
& \inference{\jhom{\greek{E}}{\Gamma}{f}\qquad\jhom{\greek{E}}{\Delta}{g}}
          {\jhom{\greek{E}}{\product{\Gamma}{\Delta}}{\pairp{f,g}}}\\
& \inference{\jhom{\greek{E}}{\Gamma}{f}\qquad\jhom{\greek{E}}{\Delta}{g}}
          {\jhomeq{\greek{E}}{\Gamma}{\jcomp{\greek{E}}{\pairp{f,g}}{\proj1}}{f}}\\
& \inference{\jhom{\greek{E}}{\Gamma}{f}\qquad\jhom{\greek{E}}{\Delta}{g}}
          {\jhomeq{\greek{E}}{\Delta}{\jcomp{\greek{E}}{\pairp{f,g}}{\proj2}}{g}}
\end{align}

\begin{equation}
\inference{\jhom{\greek{E}}{\product{\Gamma}{\Delta}}{h}
           \qquad
           \jhomeq{\greek{E}}{\Gamma}{\jcomp{E}{h}{\proj1}}{f}
           \qquad
           \jhomeq{\greek{E}}{\Delta}{\jcomp{E}{h}{\proj2}}{g}}
          {\jhomeq{\greek{E}}{\product{\Gamma}{\Delta}}{h}{\pairp{f,g}}}
\end{equation}

\subsection{Equalizer types}
Now that we have introduced a terminal object and products the categorical way,
we may just continue and present (strict) equalizers and pullbacks too, just to
see where we get. These notions are probably just not very useful in a univalent
setting until we got a good computational interpretation.

\begin{align}
& \inference{\jhom{\Gamma}{\Delta}{f}\qquad\jhom{\Gamma}{\Delta}{g}}{\jctx{\jequalizer{\Gamma}{f}{g}}}\\
& \inference{\jhom{\Gamma}{\Delta}{f}\qquad\jhom{\Gamma}{\Delta}{g}}{\jhom{\jequalizer{\Gamma}{f}{g}}{\Gamma}{\jequalizerin{f}{g}}}\\
& \inference{\jhom{\greek{E}}{\Gamma}{h}\qquad\jhomeq{\greek{E}}{\Delta}{\jcomp{\greek{E}}{h}{f}}{\jcomp{\greek{E}}{h}{g}}}{\jhom{\greek{E}}{\jequalizer{\Gamma}{f}{g}}{\jequalizer{h}{f}{g}}}\\
& \inference{
  {\begin{array}{l}
    \jhom{\greek{E}}{\Gamma}{h}\\
    \jhomeq{\greek{E}}{\Delta}{\jcomp{\greek{E}}{h}{f}}{\jcomp{\greek{E}}{h}{g}}
  \end{array}}
  \qquad
  {\begin{array}{l}
    \jhom{\greek{E}}{\jequalizer{\Gamma}{f}{g}}{k}\\
    \jhomeq{\greek{E}}{\Gamma}{\jcomp{\greek{E}}{k}{\jequalizerin{f}{g}}}{h}
  \end{array}}
}
  {\jhom{\greek{E}}{\jequalizer{\Gamma}{f}{g}}{\jequalizer{h}{f}{g}}}
\end{align}

\subsection{Pullback types}

\subsection{Dependent function types}
\begin{align}
& \inference{\jtype{\ctxext{\Gamma}{A}}{P}}{\jtype{\Gamma}{\mprd{A}{P}}}\\
& \inference{\jterm{\ctxext{\Gamma}{A}}{P}{f}}{\jterm{\Gamma}{\mprd{A}{P}}{\lambda(f)}}\\
& \inference{\jterm{\Gamma}{\mprd{A}{P}}{g}}{\jterm{\ctxext{\Gamma}{A}}{P}{\tfev(g)}}\\
& \inference{\jterm{\ctxext{\Gamma}{A}}{P}{f}}{\jtermeq{\Gamma}{A}{\tfev(\lambda(f))}{f}}\\
& \inference{\jtermeq{\ctxext{\Gamma}{A}}{P}{f}{f'}}{\jtermeq{\Gamma}{\mprd{A}{P}}{\lambda(f)}{\lambda(f')}}\\
& \inference{\jtermeq{\Gamma}{\mprd{A}{P}}{g}{g'}}{\jtermeq{\ctxext{\Gamma}{A}}{P}{\tfev(g)}{\tfev(g')}}\\
& \inference{\jtypeeq{\ctxext{\Gamma}{A}}{P}{P'}}{\jtypeeq{\Gamma}{\mprd{A}{P}}{\mprd{A}{P'}}}\\
\end{align}

With these rules we will not get the weak $\eta$-rule when identity types are present.
So it might be better to state that $\lambda$ is a trivial cofibration.

\subsection{Subterminal types}
Let $\jhom{\Gamma}{\Delta}{f}$. Then we may also consider the type $\ctxwk{{\Gamma}{\Gamma}}{{\Gamma}{\Delta}}$
in context $\ctxwk{\Gamma}{\Gamma}$ and we have the terms
\begin{align*}
& \jhom{\ctxext{\Gamma}{\ctxwk{\Gamma}{\Gamma}}}{\ctxwk{\Gamma}{A}}{\ctxwk{\Gamma}{f}}\\
& \jhom{\ctxext{\Gamma}{\ctxwk{\Gamma}{\Gamma}}}{\ctxwk{\Gamma}{A}}{\ctxwk{{\Gamma}{\Gamma}}{f}}
\end{align*}

\subsection{The empty type}

\subsection{Coproduct types}

\subsection{The natural numbers}

\section{A categorical explanation of the variable-free type theory}

The categorical interpretation of type theory is a category $\catfont{C}$ of
contexts, for each object $\Gamma$ of $\catfont{C}$ a class $\catfont{type}(\Gamma)$
of morphisms into $\Gamma$ and for each $A\in\catfont{type}(\Gamma)$ a category
$\catfont{term}(A)$ of sections of $A$. 

\begin{itemize}
\item The context extension $\ctxext{\Gamma}{A}$ is the domain of $A$.
\item There is a choice of $\Gamma\times\Delta$ for each $\Gamma$ and $\Delta$
      for which the first projection $\pi_1:\Gamma\times\Delta\to\Gamma$ is in $\catfont{type}(\Gamma)$;
      this is the weakening $\ctxwk{\Gamma}{\Delta}$.
\item For each $A\in\catfont{type}(\Gamma)$ and $x:A$ there is a choice of
      $\pi_1:\Gamma\times_A P\to\Gamma$ which is in $\catfont{type}(\Gamma)$.
\end{itemize}

\begin{comment}
\section{Binary trees}
We prove a little meta theorem about the type system we have so far. It is not
really of importance to the development of the theory. 

Let's say
that a binary tree of contexts is a pair $\pair{T,f}$ consisting of a binary 
tree $T$ together with a function $f$ assigning to each leaf a context. The set
of all binary trees of contexts is denoted by $B$. Such
binary trees may be defined inductively: $(\unit,f):B$ for any $f:\unit\to ctx$
and given any $(T_1,f_1)$ and $(T_2,f_2)$ in $B$ we have $(T_1,f_1)*(T_2,f_2)$
in $B$.

We simultaneously define
the following two functions:
\begin{align*}
\trext & : B\to ctx\\
\trwk_0 & : \prd{X,Y:B} typ(\trext(X))\\
\trwk_1 & : \prd{X:B}{\Gamma~ctx} typ(\Gamma)\to typ(\ctxext{\trext(X)}{\trwk_0(X,\Gamma)})
\end{align*}
Both these functions are defined by induction on binary trees. We set
\begin{align*}
\trext((\unit,f)) & \defeq f(\unit)\\
\trext((T_1,f_1)*(T_2,f_2)) & \defeq \ctxext{\trext((T_1,f_1))}
{\trwk_0((T_1,f_1),\trext((T_2,f_2)))}
\end{align*}
and for any context $\Gamma$
\begin{align*}
\trwk_0((\unit,f),(\unit,g)) & \defeq \ctxwk{f(\unit)}{g(\unit)}\\
\trwk_0((T_1,f_1)*(T_2,f_2),\Gamma) 
  & \defeq \trwk_1((T_1,f_1),\trwk_0((T_2,f_2)),\Gamma))
\end{align*}
and for any type $A$ in context $\Gamma$
\begin{align*}
\trwk_1((\unit,f),A) & \defeq \ctxwk{f(\unit)}{A}\\
\trwk_1((T_1,f_1)*(T_2,f_2),A) & \defeq
  \trwk_1((T_1,f_1),\trwk_1((T_2,f_2)),A))
\end{align*}

\begin{lem}
$\trwk_0((T,f),\Gamma)$ is a type in context $\trext((T,f))$ and
$\trwk_1((T,f),A)$ is a type in context $\ctxext{\trext((T,f))}{\trwk_0((T,f),\Gamma)}$
for any type $A$ in context $\Gamma$.
\end{lem}

\begin{proof}
It is immediate that $\trwk_0((\unit,f),\Gamma)$ is a type in context $\trext((\unit,f))$
and that $\trwk_1((\unit,f),A)$ is a type in context $\ctxext{\trext((\unit,f))}
{\trwk_0((\unit,f),\Gamma)}$ for any type $A$ in context $\Gamma$.

Suppose that $(T_1,f_1)$ and $(T_2,f_2)$ are binary trees of contexts such that
$\trwk_0((T_i,f_i),\Gamma)$ is a type in context
$\trext((T_i,f_i))$ for $i\jdeq 1$ and $i\jdeq 2$, and such that
$\trwk_1((T_i,f_1),A)$ is a type in context $\ctxext{\trext((T_i,f_i))}{\trwk_0((T_i,f_i),\Gamma)}$
for $i\jdeq 1$ and $i\jdeq 2$. Then
\begin{equation*}
\trwk_0((T_1,f_1)*(T_2,f_2),\Gamma) 
\jdeq \trwk_1((T_1,f_1),\trwk_0((T_2,f_2),\Gamma))
\end{equation*}
is a type in context
\begin{equation*}
\ctxext{\trext((T_1,f_1))}{\trwk_0((T_1,f_1),\trext((T_2,f_2)))}
\jdeq \trext((T_1,f_1)*(T_2,f_2)).
\end{equation*}
Also,
\begin{equation*}
\trwk_1((T_1,f_1)*(T_2,f_2),A) \jdeq \trwk_1((T_1,f_1),\trwk_1((T_2,f_2),A))
\end{equation*}
is a type in context
\begin{equation*}
\ctxext{\trext((T_1,f_1))}{\trwk_0((T_1,f_1),\ctxext{\trext((T_2,f_2))}{\trwk_0((T_2,f_2),\Gamma))}}
\end{equation*}
\end{proof}
\end{comment}



%\section{Internal models of type theory}
\subsection{Models of type theory without basic constructors}\label{internal-model-contexts}
\begin{defn}\label{defn:premodel}
An internal model $\mfM$ of type theory consists of the following data. 
\begin{enumerate}
\item A type $\tfctx(\mfM)$ of \emph{contexts}.
\item A function $\mftypfunc{\mfM}$ assigning to each context $\Gamma$ of $\mfM$ an internal model
$\mftyp{\mfM}{\Gamma}$ of type theory.
\begin{defn}
The type $\tfctx(\mftyp{\mfM}{\Gamma})$ is denoted by $\mftyp{\mfM}{\Gamma}$. A
term of $\mftyp{\mfM}{\Gamma}$ is called a \emph{type in context $\Gamma$}. To indicate
that $A$ is a type in context $\Gamma$ we also write $\Gamma\vdash A:\mfM$. 
When $P:\mftyp{\mftyp{\mfM}{\Gamma}}{A}$ for some type $A$ in context $\Gamma$, we
also speak of \emph{a family $P$ over $A$ in context $\Gamma$.}
\end{defn}
\item A family $\terms{\blank}:\prd*{\Gamma:\ctx(\mfM)}\mftyp{\mfM}{\Gamma}\to\type$
assigning to every type $A$ in context $\Gamma$ the type $\terms{A}$ of its
terms.
\begin{defn}
When $A$ is a type in context $\Gamma$, we define $\Gamma\vdash x:A$ 
to mean $x:\terms{A}$.
\end{defn}
\item Context extension: a morphism
\begin{equation*}
\tfext^\Gamma:\mftyp{\mfM}{\Gamma}\to\mfM
\end{equation*}
of internal models for every context $\Gamma$.
\begin{defn}
When $A$ is a type in context $\Gamma$, we also denote
the context $\tfext^\Gamma_0(A)$ of $\mfM$ by $\ctxext{\Gamma}{A}$.
We will write
$\ctxext{{\Gamma}{A}}{P}$ for $\ctxext{{\Gamma}{A}}{P}$.
\end{defn}
\item The judgmental equalities:
\begin{align*}
\mftyp{\mftyp{\mfM}{\Gamma}}{A} & \jdeq\mftyp{\mfM}{\ctxext{\Gamma}{A}}\\
\mftyp{\tfext^\Gamma}{A} & \jdeq \modelfont{id}_{\mftyp{\mfM}{\ctxext{\Gamma}{A}}}\\
\terms[{\tfext^\Gamma}]{P} & \jdeq \idfunc[\terms{P}].
\end{align*}
\begin{rmk}
In particular, we will have the judgmental equalities:
\begin{enumerate}
\item When $A$ is a type in context $\Gamma$ we have
\begin{equation*}
\mftyp{\mftyp{\mfM}{\Gamma}}{A}\jdeq\mftyp{\mfM}{\ctxext{\Gamma}{A}},
\end{equation*}
ensuring that a context in the model $\mftyp{\mftyp{\mfM}{\Gamma}}{A}$ is the same thing as a
context in the model $\mftyp{\mfM}{\ctxext{\Gamma}{A}}$.
\item If $Q$ is a family over $P$ where $P$ is a family over $A$ in context $\Gamma$, then
\begin{equation*}
\tfext^{\protect{\mftyp{\mftyp{\mfM}{\Gamma}}{A}}}(P,Q)
\jdeq
\tfext^{\protect{\mftyp{\mfM}{\ctxext{\Gamma}{A}}}}(P,Q)
\end{equation*}
ensuring that twe two possible notion of context extension are the same.
\end{enumerate}
Other judgmental equalities will be required with the ingredients that follow.
We will not list them all.
\end{rmk}
\item Weakening: a morphism
\begin{equation*}
\tfwk^A:\mftyp{\mfM}{\Gamma}\to\mftyp{\mfM}{\ctxext{\Gamma}{A}}
\end{equation*}
of internal models for every type $A$ in context $\Gamma$. When $B$ is a type
in context $\Gamma$, we denote $\tfwk^A(B)$ by $\ctxwk{A}{B}$. 
\item Substitution: a morphism
\begin{equation*}
\tfsubst^x:\mftyp{\mfM}{\ctxext{\Gamma}{A}}\to\mftyp{\mfM}{\Gamma}
\end{equation*}
of internal models for any $x:A$ in context $\Gamma$. When $P$ is a family over $A$ in context
$\Gamma$, we denote $\tfsubst^x(P)$ also by $\subst{x}{P}$. 
\item A judgmental equality
\begin{equation*}
\tfsubst^x\circ\tfwk^A\jdeq\modelfont{id}_{\mftyp{\mfM}{\Gamma}}
\end{equation*}
for any $x:A$ and $B$ in context $\Gamma$.
\item A context $\unit^\mfM:\tfctx(\mfM)$ and a term of type $\isequiv(\ctxext{\unit^\mfM}{\blank})$. We will denote
this equivalence by $e_\unit$. The context $\unit^\mfM$ is also
called the \emph{empty context}.
\begin{defn}
For any context $\Gamma$, type $\terms{\Gamma}$ is defined to mean
$\terms{e_\unit^{-1}(\Gamma)}$. 
\end{defn}
\item A section $\unit^{\blank}:\prd{\Gamma:\ctx(\mfM)}\mftyp{\mfM}{\Gamma}$ assigning
a type $\unit^\Gamma$ in context $\Gamma$ to every context $\Gamma$ and
an identification $\alpha_\unit(\Gamma):\id{\ctxext{\Gamma}{\unit^\Gamma}}{\Gamma}$
for every context $\Gamma$. We also require that there is an identification
$\id{\trans{\alpha_\unit(\Gamma)}{\ctxwk{\unit^\Gamma}{A}}}{A}$ for every
type $A$ in context $\Gamma$.
\item For any type $A$ in context $\Gamma$, a term $\idfunc[A]:\terms{\ctxwk{A}{A}}$
\end{enumerate}
\begin{flushright}
\textsl{End of \autoref{defn:premodel}.}
\end{flushright}
\end{defn}

\begin{defn}
In an internal model $\mfM$ we define
\begin{equation*}
\ctxhom{\Delta}{\Gamma}\defeq \terms{\ctxwk{e_\unit^{-1}(\Delta)}{e_\unit^{-1}(\Gamma)}}
\end{equation*}
\end{defn}

\subsection{Morphisms of internal models}
\begin{defn}\label{defn:premodel-morphism}
A morphism $f:\mfM\to \mfN$ of internal models consists of
\begin{enumerate}
\item a function $\ctx(f):\ctx(\mfM)\to\ctx(\mfN)$. The function $\ctx(f)$ is also
denoted by $f_0$.
\item a morphism $\mftyp{f}{\Gamma}:\mftyp{\mfM}{\Gamma}\to\mftyp{\mfN}{f_0(\Gamma)}$ of internal models for every
$\Gamma:\ctx(M)$.
\item the judgmental equality
\begin{equation*}
\mftyp{\mftyp{f}{\Gamma}}{A}\jdeq\mftyp{f}{\ctxext{\Gamma}{A}}
\end{equation*}
\item a function $\terms[f]{A}:\terms[\mfM]{A}\to\terms[\mfN]{\mftyp{f}{\Gamma}_0(A)}$ for
every type $A$ in context $\Gamma$. 
\item preservation of context extension: 
\begin{align*}
\alpha^f_0 & :\id{f\circ\tfext^\Gamma}{\tfext^{f_0(\Gamma)}\circ\mftyp{f}{\Gamma}}\\
\alpha^f_1 & :\id{\mftyp{f}{\Gamma}_0\circ\tfext^A}{\tfext^{\mftyp{f}{\Gamma}_0(A)}\circ\mftyp{f}{\ctxext{\Gamma}{A}}}.
\end{align*}
\item preservation of weakening: 
\begin{align*}
\beta^f_0 & :\id{\mftyp{f}{\Gamma}\circ\tfwk^\Gamma}{\tfwk^{f_0(\Gamma)}\circ f}\\
\beta^f_1 & :\id{\mftyp{f}{\ctxext{\Gamma}{A}}\circ\tfwk^A}{\tfwk^\protect{\mftyp{f}{\Gamma}_0(A)}\circ\mftyp{f}{\Gamma}}.
\end{align*}
\item preservation of substitution: 
\begin{equation*}
\gamma^f:\id{\mftyp{f}\Gamma_0\circ\tfsubst^x}{\tfsubst^\protect{\terms[f]{A}(x)}\circ\mftyp{f}{\ctxext{\Gamma}{A}}}.
\end{equation*}
\end{enumerate}
\begin{flushright}
\textsl{End of \autoref{defn:premodel-morphism}.}
\end{flushright}
\end{defn}

\begin{defn}
Suppose that $f:\mfM\to\mfN$ and $g:\mfN\to\mfN'$ are morphisms of internal models.
We define the composition $g\circ f:\mfM\to\mfN'$ to be the morphism given by
\begin{enumerate}
\item $(g\circ f)_0\defeq g_0\circ f_0$
\item $\mftypfunc{g\circ f}(\Gamma)\defeq\mftypfunc{g}(f_0(\Gamma))\circ\mftypfunc{f}(\Gamma)$
\item $\terms[g\circ f]{A}\defeq\terms[g]{\mftyp{f}{\Gamma}_0(A)}\circ\terms[f]{A}$.
\item a definition of $\alpha^{g\circ f}_0$ and $\alpha^{g\circ f}_1$.
\item a definition of $\beta^{g\circ f}_0$ and $\beta^{f\circ f}_1$.
\item a definition of $\gamma^{g\circ f}$. 
\end{enumerate}
\end{defn}

\begin{rmk}
The requirement that a morphism of models acts on terms is reminiscent of the requirement
that a functor acts on morphisms. In this fasion, the requirement that a morphism of
models preserves substitution is the counterpart of a functor preserving composition.
\end{rmk}
Context extension, weakening and substitution are required to be such morphims
of models. Thus they must be extended for this purpose.

\begin{description}
\item[Context extension] We will extend context extension to a morphism 
$\tfext(\Gamma):\mftyp{\mfM}{\Gamma}\to M$. Thus, we already have
$\tfext(\Gamma)_0\defeq\ctxext{\Gamma}{\blank}$. We also require 
\begin{align*}
\mftyp{\tfext(\Gamma)}{A} & : \mftyp{\mftyp{\mfM}{\Gamma}}{A}\to \mftyp{\mfM}{\ctxext{\Gamma}{A}}\\
\terms[\tfext(\Gamma)]{P} & : \terms[\mftyp{\mfM}{\Gamma}]{P}\to\terms[\mfM]{\mftyp{\tfext(\Gamma)}{A}_0(P)}
\end{align*}
For both of these we take the identity function.
We require furthermore that context extension preserves context extension,
weakening and substitution:
\begin{enumerate}
\item Extension preserves extension: for every family $P$ over $A$ in context
$\Gamma$ an identification $\id{\ctxext{{\Gamma}{A}}{P}}{\ctxext{\Gamma}{{A}{P}}}$.
\item Extension preserves weakening:
\item Extension preserves substitution:
Because context extension is the identity on the levels of types and terms,
the rules are easier.
\end{enumerate}
\item[Weakening] is a morphism $\tfwk:\mftyp{\mfM}{\Gamma}\to\mftyp{\mfM}{\ctxext{\Gamma}{A}}$, so we must have
\begin{align*}
\tfwk^A_0 & : \mftyp{\mfM}{\Gamma}\to\mftyp{\mfM}{\ctxext{\Gamma}{A}}
\intertext{which is denoted by $\ctxwk{A}{\blank}$,}
\mftypfunc{\tfwk^A} & : \mftyp{\mfM}{\ctxext{\Gamma}{B}}\to\mftyp{\mfM}{\ctxext{{\Gamma}{A}}{\ctxwk{A}{B}}}\\
\terms[\tfwk^A]{B} & : \terms[\mftyp{\mfM}{\Gamma}]{B}\to\terms[\mftyp{\mfM}{\ctxext{\Gamma}{A}}]{\ctxwk{A}{B}}
\end{align*}
for a term $y:B$, the term $\terms[{\tfwk^A}]{B}(y)$ is the constant map from
$A$ to $B$, assigning $y$ to every term of $A$.
\item[Substitution] is a morphism $\tfsubst^x:\mftyp{\mfM}{\ctxext{\Gamma}{A}}
\to\mftyp{\mfM}{\Gamma}$ for every $x:A$ in context $\Gamma$, so we must have
\begin{align*}
\tfsubst^x_0 & : \mftyp{\mfM}{\ctxext{\Gamma}{A}}\to\mftyp{\mfM}{\Gamma}
\intertext{which is denoted by $\subst{x}{\blank}$,}
\mftyp{\tfsubst^x}{P} & : \mftyp{\mfM}{\ctxext{{\Gamma}{A}}{P}}\to\mftyp{\mfM}{\ctxext{\Gamma}{\subst{x}{P}}}\\
\terms[\tfsubst^x]{P} & : \terms{P}\to\terms{\subst{x}{P}}
\end{align*}
The function $\terms[\tfsubst^x]{P}$ is usually denoted by $\tfev(\blank,x)$. 
\end{description}

\begin{comment}
\subsection{The basic type constructors in internal models}

\begin{defn}
\begin{enumerate}
\item A type $\mprd{A}{P}$ for every type $A$ in context $\Gamma$ and every type
$P$ in context $\ctxext{\Gamma}{A}$.
\begin{defn}
Suppose $A$ and $B$ are types in context $\Gamma$. We define
\begin{align*}
A\to B & \defeq\mprd{A}{\ctxwk{A}{B}}.
\intertext{For any two contexts $\Delta$ and $\Gamma$, we define}
\Delta\to\Gamma & \defeq e_\unit^{-1}(\Delta)\to e_\unit^{-1}(\Gamma)
\end{align*}
and furthermore we define $\ctxhom{\Delta}{\Gamma}\defeq\terms{\Delta\to\Gamma}$.
\end{defn}
\item An equivalence $\lambda:\eqv{\terms{P}}{\terms{\mprd{A}{P}}}$ for every
family $P$ over $A$ in context $\Gamma$. When $\ctxext{\Gamma}{A}\vdash 
u:P$, we call $\lambda(u)$ the \emph{$\lambda$-abstraction of $u$.}
\begin{rmk}
Note that we get an equivalence $\eqv{\terms{\unit^\Gamma\to A}}{\terms{A}}$ 
for every type $A$ in context $\Gamma$.
\end{rmk}
\begin{rmk}
Thus we see that $\eqv{\terms{A\to B}}{\terms{\ctxwk{A}{B}}}$
by $\lambda$-abstraction. 
\end{rmk}
%\item A function $\tfev:\terms{\mprd{A}{P}}\to\prd{x:\terms{A}}\terms{\subst{x}{P}}$.
\item A term $\pi^A:\ctxext{\Gamma}{A}\to \Gamma$ for every type $A$ in context
$\Gamma$.
\item A term $\iota^x:\ctxext{\Gamma}{\subst{x}{P}}\to\ctxext{{\Gamma}{A}}{P}$
for every family $P$ of types over $A$ in context $\Gamma$ and every
term $x:A$.
\item A type $A[f]$ in context $\Delta$ for every $f:\ctxhom{\Delta}{\Gamma}$ 
and every type $A$ in context $\Gamma$. {\color{blue}Could this be defined
in terms of the substitution $\subst{x}{P}$ we already have? It seems so. If this is
indeed the case we need either some identifications or we could just omit
this part of the definition.}
\item an identification $\id{B[\pi^A]}{\ctxwk{A}{B}}$ for any two types $A$
and $B$ in context $\Gamma$.
\item A type $\msm{A}{P}$ in context $\Gamma$ for every family $P$ over $A$
in context $\Gamma$, with an equivalence $\pairr{\blank,\blank}:\eqv{\sm{x:\terms{A}}\terms{\subst{x}{P}}}
{\terms{\msm{A}{P}}}$.
\item A family $\idtypevar{A}$ over $\ctxwk{A}{A}$ in context $\ctxext{\Gamma}{A}$ for
every type $A$ in context $\Gamma$ and a term of type.
\begin{rmk}
If $A$ and $B$ are types in context $\Gamma$, we may denote the type 
$\msm{A}{\ctxwk{A}{B}}$ by $A\times B$. By the end of the current definition
there will be an identification $\id{\ctxext{{\Gamma}{A}}{P}}
{\ctxext{\Gamma}{\msm{A}{P}}}$.

There is a term $\delta:\ctxhom{\ctxext{\Gamma}{A}}{\ctxext{{\Gamma}{A}}{\ctxwk{A}{A}}}$
defined by...
\end{rmk}
\item For any family $Q$ over $\idtypevar{A}$ in context $\ctxext{{\Gamma}{A}}{\ctxwk{A}{A}}$ an equivalence
$J:\eqv{\terms{Q}}{}$
\end{enumerate}
\end{defn}

\begingroup
\color{blue}
\subsubsection*{More desiderata}
Some of which hopefully follow from more elegant or general rules, but
this list is to keep them in mind:
\begin{enumerate}
\item $\id{e_\unit^{-1}(\Gamma.A)}{\msm{e_\unit^{-1}(\Gamma)}{\trans{(\eta_\Gamma)}{A}}}$, where
$\eta_\Gamma:\id{\Gamma}{e_\unit(e_\unit^{-1}(\Gamma))}$ is the unit of the
equivalence $e_\unit$.  
\item $\id{(\msm{A}{P}\to B)}{(P\to \ctxwk{A}{B})}$, expressing that $\Sigma$ is
left adjoint to weakening.
\item $\id{(\ctxwk{A}{B}\to P)}{(B\to\mprd{A}{P})}$ expressing that $\Pi$ is
right adjoint to weakening.
\item a term $\Gamma\vdash\idfunc[A]:A\to A$ for every type $A$ in context $\Gamma$.
\item a context $\UU$ for the universe.
\item an identification $\id{\pi^A\circ \pi^P\circ\iota^x}{\pi^{\subst{x}{P}}}$ for every
family $P$ of types over $A$ in context $\Gamma$ and every term $x:A$. In other
words, the diagram
\begin{equation*}
\begin{tikzcd}
\ctxext{{\Gamma}{A}}{P} \ar{r}{\pi^P} & \ctxext{\Gamma}{A} \ar{d}{\pi^A}\\
\ctxext{\Gamma}{\subst{x}{P}} \ar{u}{\iota^x} \ar{r}[swap]{\pi^{\subst{x}{P}}} & \Gamma
\end{tikzcd}
\end{equation*}
should commute.
\end{enumerate}
\endgroup

\begin{desiderata}
We should have a theorem stating that our internal model indeed interprets
the rules of type theory.
\end{desiderata}
\end{comment}

%%%%%%%%%%%%%%%%%%%%%%%%%%%%%%%%%%%%%%%%%%%%%%%%%%%%%%%%%%%%%%%%%%%%%%%%%%%%%%%%

\begin{comment}
\section{Weak $\omega$-groupoids}

\subsection{Trivial cofibrations and weak equivalences of types}
We describe a relation between types that expresses when they are weakly equivalent.
Weak equivalence is introduced because we need a weaker notion of judgmental 
equality which also makes sense when identity types are not present, since that
would allow us to state that context extension, weakening and substitution
commute with each other.

A term $f:\ctxwk{A}{B}$ is a trivial cofibration if it has the
property that for any fibration $Q$ over $B$,
to find a section of $Q$ it suffices to find a section of the fibration
$f^\ast Q$ over $A$. In our type theoretical setting, the rôle of fibrations
is played by families, the rôle of the function type $A\to B$ is played by
$\ctxwk{A}{B}$ and our version of the pullback $f^\ast Q$ is $\subst{f}{\ctxwk{A}{Q}}$.

\begin{defn}
Let $f$ be a term of $\ctxwk{A}{B}$ in context $\Gamma$.
\begin{enumerate}
\item For a family $Q$ over $B$ in context $\Gamma$ we define $f^\ast Q\jdeq\subst{f}{\ctxwk{A}{Q}}$.
\item For a term $g$ of $Q$ in context $\ctxext{\Gamma}{B}$ we define $f^\ast g\jdeq\subst{f}{\ctxwk{A}{g}}$.
\end{enumerate} 
\end{defn}
\begin{rmk}
The type $f^\ast(\ctxwk{B}{C})$ in context $\ctxext{\Gamma}{A}$ is can be viewed as the
type of functions from $A$ to $C$ which factor through $f$. Thus there should be
a function from $f^\ast(\ctxwk{B}{C})$ to $\ctxwk{A}{C}$.
\end{rmk}

\begin{rmk}
Every term $\jterm{\ctxext{\Gamma}{A}}{\ctxwk{A}{B}}{f}$ allows us to infer the following:
\begin{equation*}
\inference{\jtype{\ctxext{\Gamma}{B}}{Q}}{\jtype{\ctxext{\Gamma}{A}}{f^\ast Q}}
\qquad
\inference{\jterm{\ctxext{\Gamma}{B}}{Q}{g}}{\jterm{\ctxext{\Gamma}{A}}{f^\ast Q}{f^\ast g}}
\end{equation*}
\end{rmk}


\begin{defn}
A term $f:\ctxwk{A}{B}$ in context $\ctxext{\Gamma}{A}$ is said to be a trivial
cofibration if we can infer
\begin{equation*}
\inference{\jterm{\ctxext{\Gamma}{A}}{f^\ast Q}{t}}{\jterm{\ctxext{\Gamma}{B}}{Q}{\tilde{t}}}\qquad
\inference{\jterm{\ctxext{\Gamma}{A}}{f^\ast Q}{t}}{\jtermeq{\ctxext{\Gamma}{A}}{f^\ast Q}{f^\ast \tilde{t}}{t}}
\end{equation*}
{\color{red}This statement should be reformulated so that it only involves a single judgment...
but I don't see directly how to do that.}
\end{defn}

We have the following theorem in the type theory of \cite{TheBook}, which supports
our claim that we may indeed speak of a trivial cofibration. 

\begin{thm}
Suppose $f:A\to B$ is a function. Then $f$ is an equivalence if and only if
for every $Q:B\to\type$ and every $g:\prd{x:A}Q(f(x))$ there is a section
$h:\prd{y:B}Q(b)$ with the property that $h\circ f\htpy g$. 
\end{thm}

\begin{proof}
We can first take $Q$ to be the constant family $\lam{y}A$. Furthermore, we may
take $g\defeq\idfunc[A]$. Then we get a term of type
\begin{equation*}
\sm{h:B\to A}h\circ f\htpy \idfunc[A],
\end{equation*}
i.e.~we get a left inverse $h$ for $f$. To show that $h$ is also a right inverse
of $f$, let $Q$ be the family $\lam{y}\id{f(h(y))}{y}$. To find a section of
$Q$, which is the homotopy we aim for, it suffices to find a section of
$Q\circ f$. In other words, we have to show that $\id{f(h(f(x)))}{f(x)}$ for
every $x:A$. This follows from the fact that $h$ is a left inverse for $f$.

The reverse direction is immediate.
\end{proof}

Another approach would be to define $f:\ctxwk{A}{B}$ to be left invertible
if there is a term $\ctxext{{\Gamma}{A}}{P}$

\subsection{Some possibilities}

A weak omega groupoid is a model of type theory with $\Sigma$ and $\idtypevar{}$
in which there is a term of $\terms{A}$ for every type $A$ in context $\Gamma$.
This should work when univalence is around (so that functions can be replaced
by families).

We could also state that for every term $f:\terms{\ctxwk{A}{B}}$ and every
family $Q:\tftyp{\mfM}{\ctxext{\Gamma}{B}}$ there is a function
$\terms{\subst{f}{\ctxwk{A}{Q}}}\to\terms{Q}$ (asserting that $f$ is a trival
cofibration). This should always give that each $f$ is an equivalence and
this should be equivalent to the previous condition when univalence is around.

Another option arises when we see the identity type as an operation on a certain class of
terms. Usually, $\idtypevar{}$ is defined using the class of identity functions.
Let's take the class of all terms instead:
\end{comment}


\bibliographystyle{plain}
%\phantomsection\addcontentsline{toc}{section}{References}
\bibliography{refs}

\end{document}
