\section{Stage two: extending the basic theory}\label{stage2}
With a firm notion of internal models of type theory and several extensions of
type theory come within reach. The first of these extensions, colimits for
diagrams over graphs, we have already started to explore. The operation 
assigning the colimit to a graph is compatible with $\Sigma$ and $\idtypevar{}$
types. The first was compatibility result was originally part of the descent
theorem. We take the point of view that the descent theorem -- which asserts that
for any graph $\Gamma$, the type of equifibered families over $\Gamma$ is
equivalent in a canonical way to the type of families over $\tfcolim(\Gamma)$ --
is just one of the aspects of many compatibility results for colimits. In this
formulation, the descent theorem tells that the colimit operation acts not only
on graphs (objects) but also on families (the equifibered diagrams over graphs)
of the model of equifibered families.
The colimit operation acts furthermore on terms and is compatible with extension,
weakening and substitution. The fact that it is compatible with identity types is a new result
which essentially expresses that the initial pointed equifibered diagram over
a pointed graph (which is just another way of considering the notion of universal
cover) describes
the identity type of the colimit with respect to the base point. The methods
in this proof actually describe all the higher identity types of the colimit
as a certain higher inductive type; in particular it gives a description of the
loop spaces of the higher inductive types appearing in \cite{TheBook}, so we
should figure out which algebraic data we can extract from this description.
In summary, this group of results related to the descent theorem may be viewed as the
key to a good description of the colimit operation.

These results need to be extended. In the
first place we could ask ourselves whether the colimit operation is compatible
with $\Pi$ and the universe operation. Perhaps more importantly, the descent
theorem and the related group of results stating that the colimit operation
is compatible with various type constructors needs to be generalized to (diagrams
over) arbitrary categories. Since we have a proposal for what a category is,
namely a model of type theory without basic constructors, we seem to have the
right tools at hand.  We envision that the colimit operation becomes a type theoretical operation
(together with compatibility results) assigning a type to a diagram over a category.
In the presence of a universe operator we might also consider to precompose
the colimit operation with the universe operator, so that it acts on all families.

The result should be a fairly general theory of colimits and one of the first
applications we must seek for is the establishment of $\im(f)$ as the colimit
of the appropriate category, given a function $f:A\to B$ where $A$ and $B$
are arbitrary types. In the case where $A$ and $B$ are sets, $\im(f)$ is
the set-colimit of $\sm{x,y:A}\id{f(x)}{f(y)}\rightrightarrows A$; in general
this diagram should be more complicated, see \S 6.1 of \cite{lurie2009higher}.
It should be possible to describe, given $f:A\to B$, a diagram $K_f$ such that
its colimit is $\im(f)$ generally. In particular we shouldn't need to take the
set-colimit in the case that $f$ is a function between sets.  When we have 
come this far we should also look for a possible definition of a
(weak) higher predicative topos; the ingredients seem to be there. We hope that
this gives us access to a description of sheaf models of type theory and we
can possibly begin exploring some internal independence proofs.
