\section{Stage one: establishing internal higher categories and internal models}\label{stage1}
The project I propose here has its origins in the beginning of 2013, when I proved a version
of the descent theorem for homotopy colimits in type theory while I was working
with Bas Spitters to develop notions from higher category theory in the
univalent foundations. To arrive at a
notion of diagram general enough to capture all the higher inductive types described
in chapter 6 of \cite{TheBook} excluding the truncations we needed type
theoretical graphs. The graphs form a model of type theory and indeed we needed
the interpretations of several of the basic type constructors to give an efficient approach
to the descent property and its proof. Although it was not an issue to describe the graph model
and the sense in which it models the type constructors, 
not all of type theory is interpreted very well. To start with, if $A$ and $B$ 
are families of graphs over a graph $\Gamma$, then $B$ isn't also a family of 
graphs over the extended graph $\ctxext{\Gamma}{A}$ (interfering with a good 
interpretation of weakening). Since there are such issues, it is actually
not so straightforward to describe what it models!

Apart from all the type constructors like dependent function and pair types and
identity types, the graph model shows three basic operations that should come
before those type constructors. They are extension, weakening and substitution.
All three of them can be defined such that they act not only on contexts (the graphs), 
but also on families and terms. Moreover, they are all compatible with each 
other. In fact, any structure which has a notion of contexts, families and terms
and which models extension, weakening and substitution and which has identity
functions is in a weak sense a category, hypothesizing that
\begin{quote}
\emph{Category theory is dependent type theory without type constructors.}
\end{quote}
In \autoref{tt} we sketch how type theory without type constructors but with
\emph{explicit} extension, weakening and substitution might look like. This
theory is written down keeping the guiding principle in mind that type theory
without basic constructors should be a generalized algebraic theory of higher
categories.

Thus, models of this theory should be higher categories with some strictness
built in, such as the strictness of associativity of substitution. But the
theory really only talks about contexts, families and sections. The interpretation
of this as a category works as follows: The objects are going to be the contexts.
Then we may consider the terms of the weakening $\ctxwk{\Gamma}{\Delta}$ of a
context $\Delta$ by a context $\Gamma$: they are the morphisms of the category.
The composition of morphisms follows from the action of substitution and
weakening on terms. 

In the proposed approach, a model of type theory isn't a category by assumption. Rather,
a model of type theory \emph{becomes} a category because it interprets type
theory. In other words, type theory does all the work for us concerning the
categorical structure. In particular, we have neither started our notion of models
with AKS-categories as described in \cite{1categories} nor with Dybjer's notion
of internal categories with families \cite{Dybjer1996}. We should gain that we get rid of
truncatedness assumptions that are present in both AKS-categories, where the
type of objects has to be a $1$-type, and internal categories with families,
where the type of morphisms has to be a setoid. It needs not much arguing that any way to impose
conditions on internal models beyond the interpretations of the rules of
type theory stands in the way of a clear understanding of what an internal
model should be and it is a main goal of the project to design
an internal type theory without the described deficiencies. 

\begin{comment}
\subsection{Ideas in the definition of internal models}
An internal model of type theory is like a category with families, but we want
to avoid having to state higher coherences. In fact, we don't even start our
definition with a category of contexts; instead we just take a \emph{type} of contexts. 
The morphisms will come from the terms, evaluation of a function at a given
term will come from substitution. We recognize three basic ingredients to models:
first there is a type of contexts; second, for every context there is a model of types in
that context and third, for every type in a given context there is a type of its
terms. Then there are three basic attributes: context extension, weakening and
substitution. Context extension provides us with families over types as well as
with an interpretation of dependent pair types. We need weakening 
so that families can depend on the same type multiple times (the way the
identity type of a type depends two times on that type) and to be able
to talk about non-dependent function types,
the morphims of our category. Substitution will give us a way
to work with fibers of families as well as composition of functions and evaluation
of functions at terms.

Because we require a \emph{model} of types in a context, all the structure
which we require at the bottom level will be required to exist higher up as well.
Thus, the model of types in a given context $\Gamma$ will have a type of contexts
itself, which can be seen as the type of types in $\Gamma$; it will have its
own notion of types in a context, its own notion of terms, context extension,
weakening and substitution together with all the structure require for it. For
instance, when $A$ is a type in context $\Gamma$ in a model $\mfM$, then there
is the model of types in context $A$, which is the model of families over $A$. 
This model is required to be \emph{definitionally equal to} the model of types
in the context $\ctxext{\Gamma}{A}$, the context extension of $\Gamma$ and $A$.
In this way we protect ourselves from the need to dig an infinitely deep structure
of models when we want to consider examples.

To give the definition of a model we shall also need to consider certain morphisms
of internal models. Those should preserve all the structure: contexts are mapped
to contexts; for every context a morphism of models mapping the model of types
in that context to the model of types in the image of that context; there should
be a mapping of terms and context extension, weakening and substitution should be
preserved. We need to consider those morphisms because we require context extension,
weakening and substitution to be of that kind, thereby respecting each other
in all possible ways.

When we have this framework set up, we can interpret the basic type constructors
such as $\Pi$, $\Sigma$ and $\idtypevar{}$.
The higher categorical structure then comes from the
result that we have an interpretation of type theory.
\end{comment}

Of course there is no way of studying internal models or categories without
also studying the morphisms between them. In fact, we already have three of them
right from the start: the interpretations of extension, weakening and substitution
act on contexts, families and terms and preserve each of extension, weakening and
substitution. These have to be ingredients of morphisms of models as well. A
morphism of models should act on contexts because those are the contexts; it
should act on terms because that is how it acts on the morphisms; it should
preserve extension, weakening and substitution because that is exactly what
makes it functorial (in particular: preserve composition).

In fact, now that we have arrived at the idea that categorical structure should
come from the interpretation of type theory, the prettier (and necessary) way to go is to
describe a model of models of type theory. Thus, we should find a way to express
the notions of dependent model and terms thereof and explain what extension,
weakening and substitution mean. Then we should be able to derive that a morphism
of models coincides with a term of a weakened object.

The idea that morphisms of models preserve the basic type theoretical structure
can be carried further. Ideally, we would have the interpretations of the type
constructors as morphisms of models. Thus, when interpreting the operations 
$\Pi$, $\Sigma$, $\idtypevar{}$ or the universe operator 
(such as described in \cite{Palmgren1998}) we will require that
they act not only on objects, but also on families, terms and preserve extension,
weakening and substitution. And when two or more type constructors are interpreted,
they should furthermore be compatible with each other. In the case of
dependent pair types and identity types, this means function extensionality. It
should be investigated whether the compatibility of identity types and the
universe operator with each other indeed means univalence. We should note,
however, that we intend the compatibility with extension, weakening and substitution
should be strict, whereas the compatibility with the other type constructors will
not be strict. To give an exact meaning to these ideas is part of the proposal.

Among the examples of internal models we should have:
\begin{description}
\item[The setoid model] This is a classical one so it should be there. It might
      be a bit different in our case. We'll want to construct a setoid model
      of the basic type theory to interpret identity types without necessarily
      interpreting dependent function types (but that should remain possible).
\item[The graph model] This is the model that leads to the first version of the
      descent theorem. There are at least four variations of the graph model.
      The canonical graph model has the usual dependent graphs as families. We
      could also take left and right fibrations of graphs, although we loose
      interpretations of some type constructors by doing that. However, if we
      take the equifibered families as families we seem to get a model of type
      theory with the usual type constructors.
\item[Univalent unvirses] A univalent universe should be a model.
\item[Equifibered diagrams over a graph] should be the fibrations in some model.
\item[The model of all models] The objects in the model of models should be
      the models of type theory itself. There should be various flavors of the
      model of models. In the case of type theory without type constructors,
      this would be the model of all (small) higher categories. The model of
      models of type theory with several type constructors should interpret
      these type constructors as well. (This might get interesting in the case
      of type theory with a universe operator.)
\item[The model of weak $\omega$-groupoids] Since we think of models of the
      basic type theory as weak $\omega$-categories, it is not too hard to
      provide a condition on those models which enables us to talk about
      weak $\omega$-groupoids. In fact, we might have several options for such a
      condition. We investigate those; we also investigate how they relate to
      Brunerie's weak $\omega$-groupoids. Presumably, they form an internal
      model without too much fuss (because we already have an internal model
      of internal models at this stage) and we can ask whether this models
      the univalence axiom. In fact, the model of internal models might already
      have modeled the univalence axiom. 
\item[The set model] See \cite{RijkeSpitters:Sets}.
\item[Polynomial functors] The theory of polynomial functors is essential to the
      theory of $\tfW$-types, see \cite{MoerdijkPalmgren2000}. The theory of
      polynomial functors has been extensively developed by Kock \cite{Kock2011}.
      Polynomial functors are generalizations of spans and graphs are endospans,
      so a model of polynomial functors should generalize the graph model. 
\end{description}
