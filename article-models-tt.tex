\section{A variable-free presentation of type theory}
We have seen in the example of $\UU$ that we will not need to refer to variables
when we are manipulating contexts. In fact, it is more natural not to.
This suggests that there is a presentation
of type theory where contexts do not mention variables. The first small gain is that such a presentation
would not be burdened with comments about variables being bounded or not, or fresh or
not occuring at all.

\subsection{The basic judgments}
The judgments we wish to make are not the standad six judgments for ``$\Gamma$ is a context'', ``$A$ is
a type in context $\Gamma$'' and ``$x$ is a term of $A$ in context $\Gamma$'' and the three equality
judgments coming with them. We have those, but we also add a judgment for ``$i$ is a term of context $\Gamma$''
and ``$i$ and $j$ are equal terms of context $\Gamma$''. Thus, we have eight basic judgments we can make:

\begin{align*}
\jctx*{\Gamma} & \jctxeq*{\Gamma}{\Gamma'}\\
\jtermc*{\Gamma}{i} & \jtermceq*{\Gamma}{i}{j}\\
\jtype*{\Gamma}{A} & \jtypeeq*{\Gamma}{A}{B}\\
\jtermt*{\Gamma}{A}{x} & \jtermteq*{\Gamma}{A}{x}{y}.
\end{align*}

In contrast with standard practise, we don't assume that there is an empty context. The idea is that
if we interpret only the syntax of contexts, types and terms with extension, weakening and
substitution, we do not have a full model of type theory but we still have a 
structure resembling a category and categories need not have terminal objects. Therefore,
the rules that follow will describe only how to manipulate contexts, types and terms.
Of most of the inferences that we give there are two versions: one introducing an operation, the other asserting that the operation in question preserves judgmental equality. 

\subsection{The basic rules for judgmental equality}
The rules for judgmental equality establish that it is an equivalence relation
in all three cases (contexts, types and terms).
%\begingroup
%\renewcommand*{\arraystretch}{3}
%\begin{equation*}
%\begin{array}{ccc}
\begin{infarray}{ccc}
\inference{\jctx{\Gamma}}{\jctxeq{\Gamma}{\Gamma}} & \inference{\jctxeq{\Gamma}{\Delta}}{\jctxeq{\Delta}{\Gamma}} & \inference{\jctxeq{\Gamma}{\Delta}\qquad\jctxeq{\Delta}{\Theta}}{\jctxeq{\Gamma}{\Theta}}\\
\inference{\jtype{\Gamma}{A}}{\jtypeeq{\Gamma}{A}{A}} &
\inference{\jtypeeq{\Gamma}{A}{B}}{\jtypeeq{\Gamma}{B}{A}} & 
\inference{\jtypeeq{\Gamma}{A}{B}\qquad\jtypeeq{\Gamma}{B}{C}}{\jtypeeq{\Gamma}{A}{C}}\\
\inference{\jtermt{\Gamma}{A}{x}}{\jtermteq{\Gamma}{A}{x}{x}} & 
\inference{\jtermteq{\Gamma}{A}{x}{y}}{\jtermteq{\Gamma}{A}{y}{x}} &
\inference{\jtermteq{\Gamma}{A}{x}{y}\qquad\jtermteq{\Gamma}{A}{y}{z}}{\jtermteq{\Gamma}{A}{x}{z}}
\end{infarray}
%\end{array}
%\end{equation*}
\begin{infarray}{cc}
\inference{\jctxeq{\Gamma}{\Delta}\qquad\jtype{\Gamma}{A}}{\jtype{\Delta}{A}}
& \inference{\jctxeq{\Gamma}{\Delta}\qquad\jtypeeq{\Gamma}{A}{B}}{\jtypeeq{\Delta}{A}{B}}\\
\inference{\jctxeq{\Gamma}{\Delta}\qquad\jtermt{\Gamma}{A}{x}}{\jtermt{\Delta}{A}{x}}
& \inference{\jctxeq{\Gamma}{\Delta}\qquad\jtermteq{\Gamma}{A}{x}{y}}{\jtermteq{\Delta}{A}{x}{y}}\\
\inference{\jtypeeq{\Gamma}{A}{B}\qquad \jtermt{\Gamma}{A}{x}}{\jtermt{\Gamma}{B}{x}}
& \inference{\jtypeeq{\Gamma}{A}{B}\qquad\jtermteq{\Gamma}{A}{x}{y}}{\jtermteq{\Gamma}{B}{x}{y}}
\end{infarray}

\subsection{Extension}
We introduce extension which not only extends a context $\Gamma$ and a type
$A$ over it to a context $\ctxext{\Gamma}{A}$, but which also extends a type $A$
in context $\Gamma$ and a family $P$ over it to a type $\ctxext{A}{P}$ in context
$\Gamma$. We do this to ensure that all of type theory can be done in a context.
For instance, we could say (1) that a context in context $\Gamma$ is the same thing
as a type in context $\Gamma$; (2) When $A$ is a context in this sense, a type in
context $A$ is the same thing as a family $P$ over $A$ and (3) when $P$ is a type
in context $A$ in this sense, a term of $P$ keeps its original meaning.

Note that by introducing extension on the level of types and families,
we introduce $\Sigma$-types at a very early stage.
However, we need substitution to make this precise.
\begin{infarray}{cc}
\inference{\jtype{\Gamma}{A}}{\jctx{\ctxext{\Gamma}{A}}}
& \inference{\jctxeq{\Gamma}{\Delta}\qquad\jtypeeq{\Gamma}{A}{B}}{\jctxeq{\ctxext{\Gamma}{A}}{\ctxext{\Delta}{B}}}\\
\inference{\jtype{\ctxext{\Gamma}{A}}{P}}{\jtype{\Gamma}{\ctxext{A}{P}}}
& \inference{\jtypeeq{\Gamma}{A}{B}\qquad\jtypeeq{\ctxext{\Gamma}{A}}{P}{Q}}{\jtypeeq{\Gamma}{\ctxext{A}{P}}{\ctxext{B}{Q}}}
\end{infarray}

\subsection{Weakening}
We first define weakening by a context $\Gamma$. Since weakening by $\Gamma$
should in principle be a functor, it acts on contexts, types and terms alike.
Note that it is because of the weakening $\ctxwk{\Gamma}{\Delta}$ of a context
$\Delta$ by $\Gamma$ that we can speak of context morphisms from $\Gamma$ to $\Delta$: they are the terms
of $\ctxwk{\Gamma}{\Delta}$.
\begin{infarray}{cc}
\inference{\jctx{\Gamma}\qquad\jctx{\Delta}}{\jtype{\Gamma}{\ctxwk{\Gamma}{\Delta}}} 
& \inference{\jctxeq{\Gamma}{\Gamma'}\qquad\jctxeq{\Delta}{\Delta'}}{\jtypeeq{\Gamma}{\ctxwk{\Gamma}{\Delta}}{\ctxwk{\Gamma'}{\Delta'}}}\\
\inference{\jctx{\Gamma}\qquad\jtype{\Delta}{B}}{\jtype{\ctxext{\Gamma}{\ctxwk{\Gamma}{\Delta}}}{\ctxwk{\Gamma}{B}}} & \inference{\jctxeq{\Gamma}{\Gamma'}\qquad\jtypeeq{\Delta}{B}{B'}}{\jtypeeq{\ctxext{\Gamma}{\ctxwk{\Gamma}{\Delta}}}{\ctxwk{\Gamma}{B}}{\ctxwk{\Gamma'}{B'}}}\\
\inference{\jctx{\Gamma}\qquad\jtermt{\Delta}{B}{y}}{\jtermt{\ctxext{\Gamma}{\ctxwk{\Gamma}{\Delta}}}{\ctxwk{\Gamma}{B}}{\ctxwk{\Gamma}{y}}} 
& \inference{\jctxeq{\Gamma}{\Gamma'}\qquad\jtermteq{\Delta}{B}{y}{y'}}{\jtermteq{\ctxext{\Gamma}{\ctxwk{\Gamma}{\Delta}}}{\ctxwk{\Gamma}{B}}{\ctxwk{\Gamma}{y}}{\ctxwk{\Gamma'}{y'}}} 
\end{infarray}

A weakening operation is also defined for types. When $A$ and $B$ are both types
 in context $\Gamma$, the weakened type $\ctxwk{A}{B}$ in context $\ctxext{\Gamma}{A}$
 is the family which `doesn't really depend on $A$'. The terms of $\ctxwk{A}{B}$
 are the functions from $A$ to $B$. Likewise, the terms $\ctxwk{A}{y}$ are the
 constant maps at $y$ from $A$ to $B$, for $y:B$.
\begin{infarray}{cc}
\inference{\jtype{\Gamma}{A}\qquad\jtype{\Gamma}{B}}{\jtype{\ctxext{\Gamma}{A}}{\ctxwk{A}{B}}}
& \inference{\jtypeeq{\Gamma}{A}{A'}\qquad\jtypeeq{\Gamma}{B}{B'}}{\jtypeeq{\ctxext{\Gamma}{A}}{\ctxwk{A}{B}}{\ctxwk{A'}{B'}}}\\
\inference{\jtype{\Gamma}{A}\qquad\jtype{\ctxext{\Gamma}{B}}{Q}}
{\jtype{\ctxext{{\Gamma}{A}}{\ctxwk{A}{B}}}{\ctxwk{A}{Q}}}
& \inference{\jtypeeq{\Gamma}{A}{A'}\qquad\jtypeeq{\ctxext{\Gamma}{B}}{Q}{Q'}}
{\jtypeeq{\ctxext{{\Gamma}{A}}{\ctxwk{A}{B}}}{\ctxwk{A}{Q}}{\ctxwk{A'}{Q'}}}\\
\inference{\jtype{\Gamma}{A}\qquad\jtermt{\Gamma}{B}{y}}{\jtermt{\ctxext{\Gamma}{A}}{\ctxwk{A}{B}}{\ctxwk{A}{y}}}
& \inference{\jtypeeq{\Gamma}{A}{A'}\qquad\jtermteq{\Gamma}{B}{y}{y'}}{\jtermteq{\ctxext{\Gamma}{A}}{\ctxwk{A}{B}}{\ctxwk{A}{y}}{\ctxwk{A'}{y'}}}
\end{infarray}

\subsection{Substitution}
\begin{infarray}{cc}
\inference{\jtermt{\Gamma}{A}{x}\qquad\jtype{\ctxext{\Gamma}{A}}{P}}{\jtype{\Gamma}{\subst{x}{P}}}
& \inference{\jtermteq{\Gamma}{A}{x}{x'}\qquad\jtypeeq{\ctxext{\Gamma}{A}}{P}{P'}}{\jtypeeq{\Gamma}{\subst{x}{P}}{\subst{x'}{P'}}}\\
\inference{\jtermt{\Gamma}{A}{x}\qquad\jtype{\ctxext{{\Gamma}{A}}{P}}{Q}}{\jtype{\ctxext{\Gamma}{\subst{x}{P}}}{\subst{x}{Q}}}
& \inference{\jtermteq{\Gamma}{A}{x}{x'}\qquad\jtypeeq{\ctxext{{\Gamma}{A}}{P}}{Q}{Q'}}{\jtypeeq{\ctxext{\Gamma}{\subst{x}{P}}}{\subst{x}{Q}}{\subst{x'}{Q'}}}\\
\inference{\jtermt{\Gamma}{A}{x}\qquad\jtermt{\ctxext{\Gamma}{A}}{P}{f}}{\jtermt{\Gamma}{\subst{x}{P}}{\subst{x}{f}}}
& \inference{\jtermteq{\Gamma}{A}{x}{x'}\qquad\jtermteq{\ctxext{\Gamma}{A}}{P}{f}{f'}}{\jtermteq{\Gamma}{\subst{x}{P}}{\subst{x}{f}}{\subst{x'}{f'}}}\\
\inference{\jtermt{\Gamma}{A}{x}\qquad\jtermt{\ctxext{{\Gamma}{A}}{P}}{Q}{g}}{\jtermt{\ctxext{\Gamma}{\subst{x}{P}}}{\subst{x}{Q}}{\subst{x}{g}}}
& \inference{\jtermteq{\Gamma}{A}{x}{x'}\qquad\jtermteq{\ctxext{{\Gamma}{A}}{P}}{Q}{g}{g'}}{\jtermteq{\ctxext{\Gamma}{\subst{x}{P}}}{\subst{x}{Q}}{\subst{x}{g}}{\subst{x'}{g'}}}
\end{infarray}

\subsection{Definitional equality rules for weakening and substitution}
The judgmental equalities we're about to describe assert that weakening followed
by substitution leaves everything untouched. Thus, we get that each fiber
$\subst{x}{\ctxwk{A}{B}}$ is just $B$, that $\ctxwk{A}{y}$ is the constant function
mapping everything to $y:B$ and similar properties for families and terms thereof.

\begin{infarray}{c}
\inference{\jtype{\Gamma}{A}\qquad\jtype{\Gamma}{B}\qquad\jtermt{\Gamma}{A}{x}}{\jtypeeq{\Gamma}{\subst{x}{\ctxwk{A}{B}}}{B}}\\
\inference{\jtype{\Gamma}{A}\qquad\jtype{\ctxext{\Gamma}{B}}{Q}\qquad\jtermt{A}{x}}{\jtypeeq{\ctxext{\Gamma}{B}}{\subst{x}{\ctxwk{A}{Q}}}{Q}}\\
\inference{\jtermt{\Gamma}{A}{x}\qquad\jtermt{\Gamma}{B}{y}}{\jtermteq{\Gamma}{B}{\subst{x}{\ctxwk{A}{y}}}{y}}\\
\inference{\jtermt{\Gamma}{A}{x}\qquad\jtermt{\ctxext{\Gamma}{B}}{Q}{g}}{\jtermteq{\ctxext{\Gamma}{B}}{Q}{\subst{x}{\ctxwk{A}{g}}}{g}}
\end{infarray}

\subsection{The variable rule -- identity functions}
In the usual presentation of type theory, the variable rule asserts that ``in
the context $x_1:A_1,...,x_n:A_n$ we have a term $x_i:A_i$ when $1\leq i\leq n$''.

In our current system, we don't need this generality. Instead, it suffices for
us to assert that there are identity morphisms. We get arbitrary projections
from weakening.

\begin{infarray}{c}
\inference{\jtype{\Gamma}{A}}{\jtermt{\ctxext{\Gamma}{A}}{\ctxwk{A}{A}}{\idfunc[A]}}\\
\inference{\jtermt{\Gamma}{A}{x}}{\jtermteq{\Gamma}{A}{\subst{x}{\idfunc[A]}}{x}}\\
\inference{\jtermt{\ctxext{\Gamma}{A}}{P}{f}}{\jtermteq{\ctxext{\Gamma}{A}}{P}{\subst{\idfunc[A]}{\ctxwk{A}{f}}}{f}}\\
\inference{\jtermt{\ctxext{\Gamma}{B}}{\ctxwk{B}{A}}{g}}{\jtermteq{\ctxext{\Gamma}{B}}{\ctxwk{B}{A}}{\subst{g}{\ctxwk{B}{\idfunc[A]}}}{g}}
\end{infarray}

Using these rules we can derive that there is a judgmental equality $\jtypeeq{\ctxext{\Gamma}{A}}{\subst{x}{{\idfunc[A]}{B}}}
{\subst{x}{{x}{B}}}$.


%\begin{infarray}{c}
%\inference{\jtype{\ctxext{{\Gamma}{A}}{P}}{Q}\qquad\jtermt{\Gamma}{A}{x}\qquad\jtermt{\ctxext{A
%\end{infarray}{c}

\subsection{Compatibility of extension, weakening and substitution}
We describe the compatibility of extension, weakening and substitution. These
compatibility rules are formulated in a weak sense.

Extension is compatible with extension:
\begin{infarray}{c}
\inference
{ \jtype{\ctxext{\Gamma}{A}}{P}
}{\jtermt
  {\ctxext{\Gamma}{{A}{P}}}
  {\ctxwk{\ctxext{\Gamma}{{A}{P}}}{\ctxext{{\Gamma}{A}}{P}}}{\typefont{extext}_0(A,P)}
}\\
\inference
{ \jtype{\ctxext{{\Gamma}{A}}{P}}{Q}
}{\jtermt
  {\ctxext{\Gamma}{{A}{{P}{Q}}}}
  {\ctxwk{\ctxext{A}{{P}{Q}}}{\ctxext{{A}{P}}{Q}}}
  {\typefont{extext}_1(A,P,Q)}
}
\end{infarray}

Weakening is compatible with extension:
\begin{infarray}{c}
\inference{
  \jctx{\Gamma}
  \qquad
  \jtype{\Delta}{B}}
  {\jtermt
    {\ctxext{\Gamma}{\ctxwk{\Gamma}{\ctxext{\Delta}{B}}}}
    {\ctxwk{{\Gamma}{\ctxext{\Delta}{B}}}{\ctxext{\ctxwk{\Gamma}{\Delta}}{\ctxwk{\Gamma}{B}}}}
    {\typefont{wkext}_0(\Gamma,B)}}
\end{infarray}


Substitution is compatible with substitution:
\begin{infarray}{c}
\inference{\jtype{\ctxext{{\Gamma}{A}}{P}}{Q}\qquad\jtermt{\Gamma}{A}{x}\qquad\jtermt{\ctxext{\Gamma}{A}}{P}{f}}
{\jtypeeq{\Gamma}{\subst{x}{\subst{f}{Q}}}{\subst{{x}{f}}{\subst{x}{Q}}}}
\end{infarray}

\subsection{More stuff in the type theory}
\begin{infarray}{c}
\inference{\jtype{\Gamma}{A}}{\jtermt{\ctxext{\Gamma}{A}}{\ctxwk{A}{A}}{\idfunc[A]}}
\end{infarray}

\subsection{Leibniz equality on terms}
Occasionally we shall need a weaker version of equality than judgmental equality.
Our choice here is Leibniz equality, since it implies propositional equality when
identity types are present.

Let $P$ be a family over $A$ in context $\Gamma$. Then we can weaken
$\ctxwk{A}{P}$, which is a family over $\ctxwk{A}{A}$ in context $\ctxext{\Gamma}{A}$.
We also have the family $\ctxwk{{A}{A}}{P}$ over $\ctxwk{A}{A}$ in context
$\ctxext{\Gamma}{A}$ and thus we have the inference
\begin{equation*}
\inference{\jtype{\ctxext{\Gamma}{A}}{P}}{\jtype{\ctxext{{{\Gamma}{A}}{\ctxwk{A}{A}}}{\ctxwk{A}{P}}}{\ctxwk{{A}{P}}{\ctxwk{{A}{A}}{P}}}}
\end{equation*}
This family plays the role of the family $\jtype{\Gamma,x,y:A}{P(x)\to P(y)}$
in the type theory of \cite{TheBook}.


\subsection{Trivial cofibrations and weak equivalences of types}
We describe a relation between types that expresses when they are weakly equivalent.
Weak equivalence is introduced because we need a weaker notion of judgmental 
equality which also makes sense when identity types are not present, since that
would allow us to state that context extension, weakening and substitution
commute with each other.

A term $f:\ctxwk{A}{B}$ is a trivial cofibration if it has the
property that for any fibration $Q$ over $B$,
to find a section of $Q$ it suffices to find a section of the fibration
$f^\ast Q$ over $A$. In our type theoretical setting, the rôle of fibrations
is played by families, the rôle of the function type $A\to B$ is played by
$\ctxwk{A}{B}$ and our version of the pullback $f^\ast Q$ is $\subst{f}{\ctxwk{A}{Q}}$.

\begin{defn}
Let $f$ be a term of $\ctxwk{A}{B}$ in context $\Gamma$.
\begin{enumerate}
\item For a family $Q$ over $B$ in context $\Gamma$ we define $f^\ast Q\jdeq\subst{f}{\ctxwk{A}{Q}}$.
\item For a term $g$ of $Q$ in context $\ctxext{\Gamma}{B}$ we define $f^\ast g\jdeq\subst{f}{\ctxwk{A}{g}}$.
\end{enumerate} 
\end{defn}
\begin{rmk}
The type $f^\ast(\ctxwk{B}{C})$ in context $\ctxext{\Gamma}{A}$ is can be viewed as the
type of functions from $A$ to $C$ which factor through $f$. Thus there should be
a function from $f^\ast(\ctxwk{B}{C})$ to $\ctxwk{A}{C}$.
\end{rmk}

\begin{rmk}
Every term $\jtermt{\ctxext{\Gamma}{A}}{\ctxwk{A}{B}}{f}$ allows us to infer the following:
\begin{equation*}
\inference{\jtype{\ctxext{\Gamma}{B}}{Q}}{\jtype{\ctxext{\Gamma}{A}}{f^\ast Q}}
\qquad
\inference{\jtermt{\ctxext{\Gamma}{B}}{Q}{g}}{\jtermt{\ctxext{\Gamma}{A}}{f^\ast Q}{f^\ast g}}
\end{equation*}
\end{rmk}


\begin{defn}
A term $f:\ctxwk{A}{B}$ in context $\ctxext{\Gamma}{A}$ is said to be a trivial
cofibration if we can infer
\begin{equation*}
\inference{\jtermt{\ctxext{\Gamma}{A}}{f^\ast Q}{t}}{\jtermt{\ctxext{\Gamma}{B}}{Q}{\tilde{t}}}\qquad
\inference{\jtermt{\ctxext{\Gamma}{A}}{f^\ast Q}{t}}{\jtermteq{\ctxext{\Gamma}{A}}{f^\ast Q}{f^\ast \tilde{t}}{t}}
\end{equation*}
{\color{red}This statement should be reformulated so that it only involves a single judgment...
but I don't see directly how to do that.}
\end{defn}

We have the following theorem in the type theory of \cite{TheBook}, which supports
our claim that we may indeed speak of a trivial cofibration. 

\begin{thm}
Suppose $f:A\to B$ is a function. Then $f$ is an equivalence if and only if
for every $Q:B\to\type$ and every $g:\prd{x:A}Q(f(x))$ there is a section
$h:\prd{y:B}Q(b)$ with the property that $h\circ f\htpy g$. 
\end{thm}

\begin{proof}
We can first take $Q$ to be the constant family $\lam{y}A$. Furthermore, we may
take $g\defeq\idfunc[A]$. Then we get a term of type
\begin{equation*}
\sm{h:B\to A}h\circ f\htpy \idfunc[A],
\end{equation*}
i.e.~we get a left inverse $h$ for $f$. To show that $h$ is also a right inverse
of $f$, let $Q$ be the family $\lam{y}\id{f(h(y))}{y}$. To find a section of
$Q$, which is the homotopy we aim for, it suffices to find a section of
$Q\circ f$. In other words, we have to show that $\id{f(h(f(x)))}{f(x)}$ for
every $x:A$. This follows from the fact that $h$ is a left inverse for $f$.

The reverse direction is immediate.
\end{proof}

Another approach would be to define $f:\ctxwk{A}{B}$ to be left invertible
if there is a term $\ctxext{{\Gamma}{A}}{P}$

\subsection{The interactions of extension, weakening and substitution}

\subsection{The basic constructors of type theory}
\subsubsection{Dependent function types}
\begin{infarray}{cc}
\inference{\jtype{\ctxext{\Gamma}{A}}{P}}{\jtype{\Gamma}{\mprd{A}{P}}}
& \inference{\jtypeeq{\ctxext{\Gamma}{A}}{P}{P'}}{\jtypeeq{\Gamma}{\mprd{A}{P}}{\mprd{A}{P'}}}\\
\inference{\jtermt{\ctxext{\Gamma}{A}}{P}{f}}{\jtermt{\Gamma}{\mprd{A}{P}}{\lambda(f)}}
& \inference{\jtermteq{\ctxext{\Gamma}{A}}{P}{f}{f'}}{\jtermteq{\Gamma}{\mprd{A}{P}}{\lambda(f)}{\lambda(f')}}\\
\inference{\jtermt{\Gamma}{\mprd{A}{P}}{g}}{\jtermt{\ctxext{\Gamma}{A}}{P}{\tfev(g)}}
& \inference{\jtermteq{\Gamma}{\mprd{A}{P}}{g}{g'}}{\jtermteq{\ctxext{\Gamma}{A}}{P}{\tfev(g)}{\tfev(g')}}\\
\inference{\jtermt{\ctxext{\Gamma}{A}}{P}{f}}{\jtermteq{\Gamma}{A}{\tfev(\lambda(f))}{f}}
\end{infarray}

With these rules we will not get the weak $\eta$-rule when identity types are present.
So it might be better to state that $\lambda$ is a trivial cofibration

\subsection{Identity types}
We need to define the diagonal function $\jtermt{\ctxext{\Gamma}{A}}{\ctxwk{A}{\ctxext{A}{\ctxwk{A}{A}}}}{\delta}$
\begin{infarray}{c}
\inference{\jtype{\Gamma}{A}}{\jtype{\ctxext{{\Gamma}{A}}{\ctxwk{A}{A}}}{\idtypevar{A}}}\\
\inference{\jtype{\Gamma}{A}}{\jtermt{\ctxext{\Gamma}{A}}{\delta^\ast\idtypevar{A}}{\reflsym}}\\
\inference{\jtype{\ctxext{{{\Gamma}{A}}{\ctxwk{A}{A}}}{\idtypevar{A}}}{P}\qquad
\jtermt{\ctxext{\Gamma}{A}}{\pairr{\delta,\reflsym}^\ast P}{d}}
{\jtermt{\ctxext{{{\Gamma}{A}}{\ctxwk{A}{A}}}{\idtypevar{A}}}{P}{J}}
\end{infarray}


