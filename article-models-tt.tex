\section{Type theory before type constructors}\label{tt}
In this section we give a description of type theory before basic
constructors and with explicit extension, weakening and substitution. We will
formulate the type theory in such a way that contexts aren't lists of variable
declarations. The main reason is that we don't see variable declarations in the
models either. This way we also set out to a more algebraic approach of type
theory and higher category theory. Thirdly, we will not have to be burdened with
comments about variables being bounded or not, or fresh or not occuring at all.

We will develop a theory which is in fact a theory of contexts and families of
contexts where a family of context can be a type, behind which we have the
intuition that it is an atomic or irreducible family of contexts. We get immediately
that every construction can be made in an arbitrary context. 

It should be noted that this is just a prototype description. It may be the case
that various other elements have to be added yet, such as terms
$\jterm{{\Gamma}{\subst{x}{P}}}{\ctxwk{\subst{x}{P}}{\ctxext{A}{P}}}{i}$
including the fiber $\subst{x}{P}$ into the extended family $\ctxext{A}{P}$ over
$\Gamma$. Moreover, it hasn't been thoroughly tested whether the theory presented
here lives up to its promise.

Much of the rules we state are just compatibility rules of extension, weakening
and substitution with each other. In a way, these rules assert that our contexts
are just structureless lists of contexts and that likewise terms are structureless
lists of terms. They are structureless in the sense that the order in which
they are formed by pairing up is irrelevant. We note that this causes complications
in the traditional way that categorical sematics of type theory is implemented,
where contexts become objects of the category which is supposed to model type
theory. The reason for this is that context extension will not satisfy all the
compatibility rules we're about to state. The first step to resolving this is taking
the types in the empty context as the objects. 

To get the overview of the compatibility rules, we list the sections
where these compatibility rules are described in the following table (in this
table, the
subsection mentioned in row $X$ and column $Y$ consideres the rules of the
operation $Y\circ X$):
\begin{center}
\begin{tabular}{r|ccc}
& extension & weakening & substitution\\
\hline
extension & \autoref{comp-ee} & \autoref{comp-ew} & \\
weakening & \autoref{comp-we} & \autoref{comp-ww} & \autoref{comp-ws}\\
substitution & \autoref{comp-se} & \autoref{comp-sw} & \autoref{comp-ss}
\end{tabular}
\end{center}
Since extension by a context acts trivially on families and terms there are
no rules added for extension followed by substitution.

In \autoref{extension-on-terms} we come back to the operation of extension, but
now for terms. The rules in this section explain what it means to be a term of
an extended family, namely to be an extended term. The compatibility rules for
extended terms are then similar as the compatibility rules that have appeared
earlier: there is a compatibility rule for the extension of an extension, for
the weakening of an extended term, for the substutution of an extended term by
an arbitrary term and for substitution by an extended term. The last set of
compatibility rules is a form of currying-uncurrying -- this famous
correspondence naturally comes out of our scheme of compatibility rules.

\subsection{The basic judgments}
The type theory we describe here is a theory of contexts, families of
contexts and terms thereof. The families of contexts are by some authors called
dependent contexts, but they are handled a bit differently here because they
become the primary object of study. Dependent contexts can be types; they could
be seen as atomic or indecomposable dependent contexts.

Thus we make eight kinds of judgments in our type theory: ``$\Gamma$ is a context'',
``$A$ is a family of contexts over $\Gamma$'', ``$A$ is a type in context $\Gamma$''
and ``$x$ is a term of the family $A$ of contexts over $\Gamma$''. The other four
judgments are for judgmental equality. 

\begin{align*}
\jctx*{\Gamma} & \jctxeq*{\Gamma}{\Gamma'}\\
\jfam*{\Gamma}{A} & \jfameq*{\Gamma}{A}{B}\\
\jtype*{\Gamma}{A} & \jtypeeq*{\Gamma}{A}{B}\\
\jterm*{\Gamma}{A}{x} & \jtermeq*{\Gamma}{A}{x}{y}.
\end{align*}

If $A$ is a type
in context $\Gamma$, then $A$ is also a family of contexts over $\Gamma$. Being
a term of type $A$ means the same thing as being a term of the family $A$ of contexts.
Two types in context $\Gamma$ are judgmentally equal precisely when they are equal
as context families. Moreover, if a family $B$ of contexts over $\Gamma$ is
judgmentally equal to a type $A$ in context $\Gamma$, then $B$ is a type in
context $\Gamma$. This is expressed by the following four inference rules:

\begin{align*}
& \inference
  {\jtype{\Gamma}{A}}
  {\jfam{\Gamma}{A}}
& & \inference
    {\jtypeeq{\Gamma}{A}{B}}
    {\jfameq{\Gamma}{A}{B}}\\
& \inference
  {\jtype{\Gamma}{A}
   \jfameq{\Gamma}{A}{B}}
  {\jtype{\Gamma}{B}}
& & \inference
    {\jtype{\Gamma}{A}
     \jfameq{\Gamma}{A}{B}}
    {\jtypeeq{\Gamma}{A}{B}}
\end{align*}


\subsection{The basic rules for judgmental equality}
The rules for judgmental equality establish that it is an equivalence relation
in all three cases (contexts, types and terms).
\bgroup\small
\begin{align*}
& \inference
  {\jctx{\Gamma}}
  {\jctxeq{\Gamma}{\Gamma}} 
& & \inference
    {\jctxeq{\Gamma}{\Delta}}
    {\jctxeq{\Delta}{\Gamma}} 
& & \inference
    {\jctxeq{\Gamma}{\Delta}
     \jctxeq{\Delta}{\greek{E}}}
    {\jctxeq{\Gamma}{\greek{E}}}\\
& \inference
  {\jfam{\Gamma}{A}}
  {\jfameq{\Gamma}{A}{A}} 
& & \inference
    {\jfameq{\Gamma}{A}{B}}
    {\jfameq{\Gamma}{B}{A}}
& & \inference
    {\jfameq{\Gamma}{A}{B}
     \jfameq{\Gamma}{B}{C}}
    {\jfameq{\Gamma}{A}{C}}\\
& \inference
  {\jterm{\Gamma}{A}{x}}
  {\jtermeq{\Gamma}{A}{x}{x}}
& & \inference
    {\jtermeq{\Gamma}{A}{x}{y}}
    {\jtermeq{\Gamma}{A}{y}{x}}
& & \inference
    {\jtermeq{\Gamma}{A}{x}{y}
     \jtermeq{\Gamma}{A}{y}{z}}
    {\jtermeq{\Gamma}{A}{x}{z}}
\end{align*}
\egroup

The following convertibility rules are responsible for the strictness
of judgmental equality, which sets it apart from equivalences or identifications:

\begin{align*}
& \inference
  {\jctxeq{\Gamma}{\Delta}
   \jfam{\Gamma}{A}}
  {\jfam{\Delta}{A}}
& & \inference
    {\jctxeq{\Gamma}{\Delta}
     \jfameq{\Gamma}{A}{B}}
    {\jfameq{\Delta}{A}{B}}\\
& \inference
  {\jctxeq{\Gamma}{\Delta}
   \jterm{\Gamma}{A}{x}}
  {\jterm{\Delta}{A}{x}}
& & \inference
    {\jctxeq{\Gamma}{\Delta}
     \jtermeq{\Gamma}{A}{x}{y}}
    {\jtermeq{\Delta}{A}{x}{y}}\\
& \inference
  {\jfameq{\Gamma}{A}{B}
   \jterm{\Gamma}{A}{x}}
  {\jterm{\Gamma}{B}{x}}
& & \inference
    {\jfameq{\Gamma}{A}{B}
     \jtermeq{\Gamma}{A}{x}{y}}
    {\jtermeq{\Gamma}{B}{x}{y}}
\end{align*}

\subsection{The empty context}
There is an empty context and over any context there is an empty family of
contexts. We do not assume that the empty context is a type, that would be like
assuming that the multiplicative unit of a ring is prime. The empty family
always has a unique term. 

\begin{align}
& \inference
  {}
  {\jctx{\emptyc}}\\
& \inference
  {\jctx{\Gamma}}
  {\jfam{\Gamma}{\emptyf[\Gamma]}}\\
& \inference
  {\jctx{\Gamma}}
  {\jterm{\Gamma}{\emptyf[\Gamma]}{\emptytm[\Gamma]}}\\
& \inference
  {\jterm{\Gamma}{\emptyf[\Gamma]}{x}}
  {\jtermeq{\Gamma}{\emptyf[\Gamma]}{x}{\emptytm[\Gamma]}}
\end{align}

Moreover, if $\Gamma$ is a context family over the
empty context, then $\Gamma$ is a context and every context is a context
family over the empty context. Note that this allows us to speak
of terms of contexts too.

\begin{align}
& \inference
  {\jctx{\Gamma}}
  {\jfam{\emptyc}{\Gamma}} 
& & \inference
    {\jfam{\emptyc}{\Gamma}}
    {\jctx{\Gamma}}\\
& \inference
  {\jctxeq{\Gamma}{\Delta}}
  {\jfameq{\emptyc}{\Gamma}{\Delta}}
& & \inference
    {\jfameq{\emptyc}{\Gamma}{\Delta}}
    {\jctxeq{\Gamma}{\Delta}}
\end{align}

\subsubsection{The empty context is compatible with itslef}
The empty context $\emptyc$ may be considered as a family of contexts over the empty
context. When we do this, we get $\emptyf[\emptyc]$.
\begin{equation}
\inference
{}
{\jfameq{\emptyc}{\emptyc}{\emptyf[\emptyc]}}
\end{equation}
In the future, we shall denote $\emptyf[\Gamma]$ by $\emptyf$. The above rule
guarantees that this will not cause confusion. Likewise, we shall denote
$\emptytm[\Gamma]$ by $\emptytm$.

\subsection{Extension}
We introduce extension which not only extends a context $\Gamma$ and a family
$A$ over it to a context $\ctxext{\Gamma}{A}$, but which also extends a family $A$
in context $\Gamma$ and a family $P$ over it to a family $\ctxext{A}{P}$ over context
$\Gamma$. We do this to ensure that all of type theory can be done in a context.
For instance, we could say (1) that a context in context $\Gamma$ is the same thing
as a family over $\Gamma$; (2) When $A$ is a context in this sense, a family over
$A$ is the same thing as a family $P$ over $\ctxext{\Gamma}{A}$ and 
(3) when $P$ is a family over $A$ in this sense, a term of $P$ keeps its original meaning.

\begin{align}
& \inference
  {\jfam{\Gamma}{A}}
  {\jctx{\ctxext{\Gamma}{A}}}
& & \inference
    {\jctxeq{\Gamma}{\Delta}
     \jfameq{\Gamma}{A}{B}}
    {\jctxeq{\ctxext{\Gamma}{A}}{\ctxext{\Delta}{B}}}\\
& \inference
  {\jfam{{\Gamma}{A}}{P}}
  {\jfam{\Gamma}{\ctxext{A}{P}}}
& & \inference
    {\jfameq{\Gamma}{A}{B} 
     \jfameq{{\Gamma}{A}}{P}{Q}}
    {\jfameq{\Gamma}{\ctxext{A}{P}}{\ctxext{B}{Q}}}
\end{align}

\subsubsection{Extension is compatible with the empty context}
The following rule asserts that extension by $\emptyc$ leaves the contexts unchanged.
\begin{align}
& \inference
  {\jctx{\Gamma}}
  {\jctxeq{\ctxext{\emptyc}{\Gamma}}{\Gamma}}\\
& \inference
  {\jctx{\Gamma}}
  {\jctxeq{\ctxext{\Gamma}{\emptyf}}{\Gamma}}\\
& \inference
  {\jfam{\Gamma}{A}}
  {\jfameq{\Gamma}{\ctxext{\emptyf}{A}}{A}}
\end{align}

\subsubsection{Extension is compatible with itself}\label{comp-ee}
The inference rules asserting that extension is compatible with itself assert
that contexts are unstructured lists of type declarations. This rule is
unavoidable if we want that for a family $A$ in context $\Gamma$, a family over
$A$ is the same thing as a family over $\ctxext{\Gamma}{A}$. 

\begin{align}
& \inference
  {\jfam{\Gamma}{A}
   \jfam{{\Gamma}{A}}{P}}
  {\jctxeq{\ctxext{{\Gamma}{A}}{P}}{\ctxext{\Gamma}{{A}{P}}}}\\
& \inference
  {\jfam{{\Gamma}{A}}{P}
   \jfam{{{\Gamma}{A}}{P}}{Q}}
  {\jfameq{\Gamma}{\ctxext{{A}{P}}{Q}}{\ctxext{A}{{P}{Q}}}}
\end{align}

\subsection{The type theoretic operation of weakening}
When $A$ is a context family over a context $\Gamma$, we wish to define a weakening
operation $\ctxwk{A}{}$. The weakening operation acts on context families $B$ 
over $\Gamma$, terms thereof, context families over $B$ and terms thereof.
It weakens those, which means that it ``adds $A$ to the context''. The context
family $\ctxwk{A}{B}$ can be seen as the constant family $B$ over $\ctxext{\Gamma}{A}$.
Likewise, when $y$ is a term of $B$, the term $\ctxwk{A}{y}$ of $\ctxwk{A}{B}$
can be seen as the constant term with value $y$.
 
 In the following inference rules we assume that $\jfam{\Gamma}{A}$ and in the
 rules asserting a judgmental equality we assume furthermore that 
 $\jfameq{\Gamma}{A}{A'}$.
\begin{align}
& \inference
  {\jfam{\Gamma}{B}}
  {\jfam{{\Gamma}{A}}{\ctxwk{A}{B}}}
& & \inference
    {\jfameq{\Gamma}{B}{B'}}
    {\jfameq{{\Gamma}{A}}{\ctxwk{A}{B}}{\ctxwk{A'}{B'}}}\\
& \inference
  {\jfam{{\Gamma}{B}}{Q}}
  {\jfam{{{\Gamma}{A}}{\ctxwk{A}{B}}}{\ctxwk{A}{Q}}}
& & \inference
    {\jfameq{{\Gamma}{B}}{Q}{Q'}}
    {\jfameq{{{\Gamma}{A}}{\ctxwk{A}{B}}}{\ctxwk{A}{Q}}{\ctxwk{A'}{Q'}}}\\
& \inference
  {\jterm{\Gamma}{B}{y}}
  {\jterm{{\Gamma}{A}}{\ctxwk{A}{B}}{\ctxwk{A}{y}}}
& & \inference
    {\jtermeq{\Gamma}{B}{y}{y'}}
    {\jtermeq{{\Gamma}{A}}{\ctxwk{A}{B}}{\ctxwk{A}{y}}{\ctxwk{A'}{y'}}}\\
& \inference
  {\jterm{{\Gamma}{B}}{Q}{g}}
  {\jterm{{{\Gamma}{A}}{\ctxwk{A}{B}}}{\ctxwk{A}{Q}}{\ctxwk{A}{g}}}
& & \inference
    {\jtermeq{{\Gamma}{B}}{Q}{g}{g'}}
    {\jtermeq{{{\Gamma}{A}}{\ctxwk{A}{B}}}{\ctxwk{A}{Q}}{\ctxwk{A}{g}}{\ctxwk{A'}{g'}}}
\end{align}

We add rules asserting that a weakened type is again a type:

\begin{align}
& \inference
  {\jfam{\Gamma}{B}
   \jtype{\ctxext{\Gamma}{B}}{Q}}
  {\jtype{\ctxext{{\Gamma}{A}}{\ctxwk{A}{B}}}{\ctxwk{A}{Q}}}
\end{align}

\subsubsection{Weakening is compatible with the empty context}
The following rules express that when the empty context or context family is
weakened, the result is the empty context family.
\begin{align}
& \inference
  {\jctx{\Gamma}}
  {\jfameq{\Gamma}{\ctxwk{\Gamma}{\emptyc}}{\emptyf}}\\
& \inference
  {\jfam{\Gamma}{A}}
  {\jfameq{{\Gamma}{A}}{\ctxwk{A}{\emptyf}}{\emptyf}}
\end{align}
Weakening by the empty family $\emptyf$ over a context $\Gamma$ leaves families, 
their terms, families over those families and
terms of those unchanged:
\begin{align}
& \inference
  {\jfam{\Gamma}{B}}
  {\jfameq{\Gamma}{\ctxwk{\emptyf}{B}}{B}}\\
& \inference
  {\jterm{\Gamma}{B}{y}}
  {\jtermeq{\Gamma}{B}{\ctxwk{\emptyf}{y}}{y}}\\
& \inference
  {\jfam{{\Gamma}{B}}{Q}}
  {\jfameq{{\Gamma}{B}}{\ctxwk{\emptyf}{Q}}{Q}}\\
& \inference
  {\jterm{{\Gamma}{B}}{Q}{g}}
  {\jtermeq{{\Gamma}{B}}{Q}{\ctxwk{\emptyf}{g}}{g}}
\end{align}

\subsubsection{Weakening is compatible with extension}\label{comp-we}

The following rules assert the compatibility of extension with weakening: for
every family $A$ over $\Gamma$ and every family $Q$ over $\ctxext{\Gamma}{B}$
there is a
judgmental equality $\ctxwk{A}{\ctxext{B}{Q}}\jdeq\ctxext{\ctxwk{A}{B}}
{\ctxwk{A}{Q}}$. 

When thinking of terms of $\ctxwk{A}{B}$ as morphisms of families from $A$ to
$B$, this looks already like form of type theoretic choice. It is weaker in that
it is not stated with function types, yet it is stronger in that it states a
judgmental equality between two families. When one makes the weakening operation
notationally invisible -- as is in fact the usual practice in type theory -- the
following compatibility rules become completely obvious.

In the following inference rules we assume that $\jfam{\Gamma}{A}$.
\begin{align}
& \inference
  {\jfam{{{\Gamma}{B}}{Q}}{R}}
  {\jfameq
    {\ctxext{{\Gamma}{A}}{\ctxwk{A}{B}}}
    {\ctxwk{A}{\ctxext{Q}{R}}}
    {\ctxext{\ctxwk{A}{Q}}{\ctxwk{A}{R}}}}
\end{align}

\subsubsection{Weakening is compatible with itself}\label{comp-ww}
We state judgmental equality rules expressing
that weakening is compatible with itself. These rules state that the following
diagram commutes given any two families $A$ and $B$ in context $\Gamma$:
\begin{equation*}
\begin{tikzcd}[column sep=huge]
\jfam{\Gamma}{\blank} \ar{r}{C\mapsto\ctxwk{B}{C}} \ar{d}[swap]{C\mapsto\ctxwk{A}{C}} & \jfam{{\Gamma}{B}}{\blank} \ar{d}{Q\mapsto\ctxwk{A}{Q}}\\
\jfam{{\Gamma}{A}}{\blank} \ar{r}[swap]{P\mapsto\ctxwk{{A}{B}}{P}} & \jfam{{{\Gamma}{A}}{\ctxwk{A}{B}}}{\blank}
\end{tikzcd}
\end{equation*}
Thus, we get the following set of inference rules:
\begin{align}
& \inference
  {\jfam{\Gamma}{A}
   \jfam{\Gamma}{B}
   \jfam{{\Gamma}{C}}{R}}
  {\jfameq{{{{\Gamma}{A}}{\ctxwk{A}{B}}}{\ctxwk{{A}{B}}{{A}{C}}}}
    {\ctxwk{A}{{B}{R}}}
    {\ctxwk{{A}{B}}{{A}{R}}}}\label{comp-ww-f}\\
& \inference
  {\jfam{\Gamma}{A}
   \jfam{\Gamma}{B}
   \jterm{{\Gamma}{C}}{R}{t}}
  {\jtermeq{{{{\Gamma}{A}}{\ctxwk{A}{B}}}{\ctxwk{{A}{B}}{{A}{C}}}}
    {\ctxwk{{A}{B}}{{A}{R}}}{\ctxwk{A}{{B}{t}}}{\ctxwk{{A}{B}}{{A}{t}}}}
\label{comp-ww-t}
\end{align}

\subsubsection{Extension is compatible with weakening}\label{comp-ew}
The rules expressing that extension is compatible with weakening assert that
weakening by an extension is the same thing as weakening twice in the
appropriate way.

In the following inference rules we assume that
$\jfam{\Gamma}{A}$ and $\jfam{{\Gamma}{A}}{P}$. 
\begin{align}
& \inference
  {\jfam{{\Gamma}{B}}{Q}}
  {\jfameq{{{{\Gamma}{A}}{P}}{\ctxwk{P}{{A}{B}}}}{\ctxwk{\ctxext{A}{P}}{Q}}{\ctxwk{P}{{A}{Q}}}}
  \label{comp-ew-f}\\
& \inference
  {\jterm{{\Gamma}{B}}{Q}{g}}
  {\jtermeq{{{{\Gamma}{A}}{P}}{\ctxwk{P}{{A}{B}}}}{\ctxwk{P}{{A}{Q}}}{\ctxwk{\ctxext{A}{P}}{g}}{\ctxwk{P}{{A}{g}}}} 
  \label{comp-ew-t}
\end{align}

\subsection{The type theoretic operation of substitution}
Given a family $P$ over $A$ and a term $x$ of $A$, substitution gives a way to
consider the fiber $\subst{x}{P}$ of $P$ at $x$. Also, we get a way to evaluate
terms $f$ of $P$ at $x$. This will give us ways to compose functions too. In
this section, we shall first introduce the operations `substitution of a term $x$'
for families $P$ over $\ctxext{\Gamma}{A}$, terms $f$ of those, families $Q$ over
$\ctxext{{\Gamma}{A}}{P}$ and terms $g$ of those. 
Then we shall explain how substitution interacts
with itself, extension and weakening.

In the rules introducing the various substitutions we assume $\jterm{\Gamma}{A}{x}$;
in the rules introducing the definitional equalities we assume $\jtermeq{\Gamma}{A}{x}{x'}$.

\begin{align}
& \inference
  {\jfam{{\Gamma}{A}}{P}}
  {\jfam{\Gamma}{\subst{x}{P}}}
& & \inference
    {\jfameq{{\Gamma}{A}}{P}{P'}}
    {\jfameq{\Gamma}{\subst{x}{P}}{\subst{x'}{P'}}}\\
& \inference
  {\jfam{{{\Gamma}{A}}{P}}{Q}}
  {\jfam{{\Gamma}{\subst{x}{P}}}{\subst{x}{Q}}}
& & \inference
    {\jfameq{{{\Gamma}{A}}{P}}{Q}{Q'}}
    {\jfameq{{\Gamma}{\subst{x}{P}}}{\subst{x}{Q}}{\subst{x'}{Q'}}}\\
& \inference
  {\jterm{{\Gamma}{A}}{P}{f}}
  {\jterm{\Gamma}{\subst{x}{P}}{\subst{x}{f}}}
& & \inference
    {\jtermeq{{\Gamma}{A}}{P}{f}{f'}}
    {\jtermeq{\Gamma}{\subst{x}{P}}{\subst{x}{f}}{\subst{x'}{f'}}}\\
& \inference
  {\jterm{{{\Gamma}{A}}{P}}{Q}{g}}
  {\jterm{{\Gamma}{\subst{x}{P}}}{\subst{x}{Q}}{\subst{x}{g}}}
& & \inference
    {\jtermeq{{{\Gamma}{A}}{P}}{Q}{g}{g'}}
    {\jtermeq{{\Gamma}{\subst{x}{P}}}{\subst{x}{Q}}{\subst{x}{g}}{\subst{x'}{g'}}}
\end{align}

We also add the following rule asserting that substitution preserves the
property of being a type:
\begin{align}
& \inference
  {\jterm{\Gamma}{A}{x}
   \jtype{\ctxext{{\Gamma}{A}}{P}}{Q}}
  {\jtype{\ctxext{\Gamma}{\subst{x}{P}}}{\subst{x}{Q}}}
\end{align}

\subsubsection{Substitution is compatible with the empty context}
The fibers of the empty family are the empty families:
\begin{align}
& \inference
  {\jterm{\Gamma}{A}{x}}
  {\jfameq{\Gamma}{\subst{x}{\emptyf}}{\emptyf}}\\
& \inference
  {\jterm{\Gamma}{A}{x}
   \jfam{{\Gamma}{A}}{P}}
  {\jfameq{{\Gamma}{\subst{x}{P}}}{\subst{x}{\emptyf}}{\emptyf}}
\end{align}

The following rules assert that substituting by the term $\jterm{\Gamma}{\emptyf}{\emptytm}$
leaves everything unchanged.
\begin{align}
& \inference
  {\jfam{\Gamma}{A}}
  {\jfameq{\Gamma}{\subst{\emptytm}{A}}{A}}\\
& \inference
  {\jterm{\Gamma}{A}{x}}
  {\jtermeq{\Gamma}{A}{\subst{\emptytm}{x}}{x}}\\
& \inference
  {\jfam{{\Gamma}{A}}{P}}
  {\jfameq{{\Gamma}{A}}{\subst{\emptytm}{P}}{P}}\\
& \inference
  {\jterm{{\Gamma}{A}}{P}{f}}
  {\jtermeq{{\Gamma}{A}}{P}{\subst{\emptytm}{f}}{f}}.
\end{align}

\subsubsection{Substitution is compatible with extension}\label{comp-se}
Suppose $\jterm{\Gamma}{A}{x}$ in all of the following inference rule.
\begin{align}
& \inference
  {\jfam{{{{\Gamma}{A}}{P}}{Q}}{R}}
  {\jfameq{{\Gamma}{\subst{x}{P}}}{\subst{x}{\ctxext{Q}{R}}}{\ctxext{\subst{x}{Q}}{\subst{x}{R}}}}
\end{align}

\subsubsection{Substitution is compatible with weakening}\label{comp-sw}
The rules asserting the compatibility of substitution with weakening assert
that the following diagram commutes for any $\jterm{\Gamma}{A}{x}$ and any
$\jfam{{\Gamma}{A}}{P}$:
\begin{equation*}
\begin{tikzcd}[column sep=huge]
\jfam{{\Gamma}{A}}{\blank} \ar{d}[swap]{Q\mapsto\subst{x}{Q}} \ar{r}{Q\mapsto\ctxwk{P}{Q}} & \jfam{{{\Gamma}{A}}{P}}{\blank} \ar{d}{R\mapsto\subst{x}{R}}\\ 
\jfam{\Gamma}{\blank} \ar{r}[swap]{B\mapsto\ctxwk{\subst{x}{P}}{B}} & \jfam{{\Gamma}{\subst{x}{P}}}{\blank}
\end{tikzcd}
\end{equation*}
We plug in an extra layer of families to cover the most general case at once.
In the following inference rules we assume that $\jterm{\Gamma}{A}{x}$ and
$\jfam{{\Gamma}{A}}{P}$:
\begin{align}
& \inference
  {\jfam{{{\Gamma}{A}}{Q}}{R}}
  {\jfameq{{{\Gamma}{\subst{x}{P}}}{\subst{x}{\ctxwk{P}{Q}}}}{\subst{x}{\ctxwk{P}{R}}}{\ctxwk{\subst{x}{P}}{\subst{x}{R}}}}
  \label{comp-sw-f}\\
& \inference
  {\jterm{{{\Gamma}{A}}{Q}}{R}{h}}
  {\jtermeq
      {\ctxext{{\Gamma}{\subst{x}{P}}}{\subst{x}{\ctxwk{P}{Q}}}}
      {\subst{x}{\ctxwk{P}{R}}}
      {\subst{x}{\ctxwk{P}{h}}}
      {\ctxwk{\subst{x}{P}}{\subst{x}{h}}}}
  \label{comp-sw-t}
\end{align}

\subsubsection{Substitution is compatible with substitution}\label{comp-ss}

We require that substitution is compatible with itself, which is roughly the
assertion that substitution is associative. However, we cannot just state that
$\subst{x}{{f}{g}}\jdeq\subst{{x}{f}}{g}$ since the expression $\subst{{x}{f}}{g}$
is not well-formed. The term $\subst{x}{f}$ can be substituted in (terms of) families over
$\subst{x}{P}$; the term $\subst{x}{g}$ is such. Therefore, associativity of
substitution takes the form $\subst{x}{{f}{g}}\jdeq\subst{{x}{f}}{{x}{g}}$.
Note that the term $\subst{{x}{f}}{{x}{g}}$ may be written down more conveniently
as $\subst{x,\subst{x}{f}}{g}$, although we will not do that here.

In the following inference rules we assume
$\jterm{\Gamma}{A}{x}$ and $\jterm{{\Gamma}{A}}{P}{f}$.

\begin{align}
& \inference
  {\jfam{{{{\Gamma}{A}}{P}}{Q}}{R}}
  {\jfameq{{\Gamma}{\subst{x}{{f}{Q}}}}{\subst{x}{{f}{R}}}{\subst{{x}{f}}{{x}{R}}}}
  \label{comp-ss-f}\\
& \inference
  {\jterm{{{{\Gamma}{A}}{P}}{Q}}{R}{h}}
  {\jtermeq{{\Gamma}{\subst{x}{{f}{Q}}}}{\subst{x}{{f}{R}}}{\subst{x}{{f}{h}}}{\subst{{x}{f}}{{x}{h}}}}
  \label{comp-ss-t}
\end{align}

\subsubsection{Weakening is compatible with substitution}\label{comp-ws}
We already have rules for the compatibility of substitution with weakening, but
we still need the rules the other way around, asserting that there is a 
judgmental equality $\ctxwk{A}{\subst{y}{Q}}\jdeq\subst{\ctxwk{A}{y}}{\ctxwk{A}{Q}}$
together with all its variants.

In the following inference rules we assume that $\jfam{\Gamma}{A}$ and that
$\jterm{\Gamma}{B}{y}$.

\begin{align}
& \inference
  {\jfam{{{\Gamma}{B}}{Q}}{R}}
  {\jfameq{{{\Gamma}{A}}{\ctxwk{A}{\subst{y}{Q}}}}{\ctxwk{A}{\subst{y}{R}}}{\subst{\ctxwk{A}{y}}{\ctxwk{A}{R}}}}
  \label{comp-ws-f}\\
& \inference
  {\jterm{{{\Gamma}{B}}{Q}}{R}{h}}
  {\jtermeq{{{\Gamma}{A}}{\ctxwk{A}{\subst{y}{Q}}}}{\ctxwk{A}{\subst{y}{R}}}{\ctxwk{A}{\subst{y}{h}}}{\subst{\ctxwk{A}{y}}{\ctxwk{A}{h}}}}
  \label{comp-ws-t}
\end{align}

\subsection{Composition and identity functions}
\subsubsection{The defining property of weakening}
The judgmental equalities we're about to describe assert that substituting a term
in the weakening a thing gives you the thing back. In the case of contexts we get that each fiber
$\subst{x}{\ctxwk{A}{B}}$ is just $B$ and in the case of terms we get 
that $\ctxwk{A}{y}$ is the constant function
mapping everything to $y:B$. Thus, these rules actually establish the weakening
as the weakening. After stating the rules we will describe what it means to
compose context morphisms (terms of weakened contexts).

\begin{align}
& \inference
  {\jfam{\Gamma}{A}
   \jfam{{\Gamma}{B}}{Q}
   \jterm{\Gamma}{A}{x}}
  {\jfameq{{\Gamma}{B}}{\subst{x}{\ctxwk{A}{Q}}}{Q}}
  \label{defn-ws-3}\\
& \inference
  {\jterm{\Gamma}{A}{x}
   \jterm{{\Gamma}{B}}{Q}{g}}
  {\jtermeq{{\Gamma}{B}}{Q}{\subst{x}{\ctxwk{A}{g}}}{g}}
  \label{defn-ws-4}
\end{align}

Using the rules of the compatibility of substitution with weakening and of the
compatibility of weakening with itself, we see that we can show

\begin{lem}
The inference rule
\begin{equation*}
\inference
  {\jfam{\Gamma}{A}
   \jfam{\Gamma}{B}
   \jfam{\Gamma}{C}
   \jhom{\Gamma}{A}{B}{f}}
  {\jfameq{{\Gamma}{A}}
    {\subst{f}{\ctxwk{A}{{B}{C}}}}
    {\ctxwk{A}{C}}}
\end{equation*}
is valid.
\end{lem}

\begin{proof}
Let $\jfam{\Gamma}{A}$, $\jfam{\Gamma}{B}$, $\jfam{\Gamma}{C}$ and $\jhom{\Gamma}{A}{B}{f}$.
Then we have the judgmental equalities
\begin{align*}
\subst{f}{\ctxwk{A}{{B}{C}}}
& \jdeq \subst{f}{\ctxwk{{A}{B}}{{A}{C}}}\\
& \jdeq \ctxwk{A}{C}.
\end{align*}
\end{proof}

It follows that for $\jterm{{\Gamma}{B}}{\ctxwk{B}{C}}{g}$ we can compose $f$
with $g$ to obtain a term of $\ctxwk{A}{C}$ in context $\ctxext{\Gamma}{A}$.
In the following definition, we work with in a slightly greater generality.

\begin{defn}
For $\jhom{\Gamma}{A}{B}{f}$ and consider a term $\jterm{{\Gamma}{B}}{Q}{g}$.
We define
\begin{align*}
\jfamdefn*{{\Gamma}{A}}{\jcomp{f}{Q}}{\subst{f}{\ctxwk{A}{Q}}}\\
\jtermdefn*{{\Gamma}{A}}{\jcomp{f}{Q}}{\jcomp{f}{g}}{\subst{f}{\ctxwk{A}{g}}}.
\end{align*}
Likewise, when we have a family $\jfam{{{\Gamma}{B}}{Q}}{R}$ and a term
$\jterm{{{\Gamma}{B}}{Q}}{R}{h}$, we define
\begin{align*}
\jfamdefn*{{{\Gamma}{A}}{\jcomp{f}{Q}}}{\jcomp{f}{R}}{\subst{f}{\ctxwk{A}{R}}}\\
\jtermdefn*{{{\Gamma}{A}}{\jcomp{f}{Q}}}{\jcomp{f}{R}}{\jcomp{f}{h}}{\subst{f}{\ctxwk{A}{h}}}.
\end{align*}
\end{defn}

We have lots of compatibility properties for composition:

\begin{lem}
We have the following inference rules about the situation where something is
substituted by a composition:
\begin{align*}
& \inference
  {\jhom{\Gamma}{A}{B}{f}
   \jhom{\Gamma}{B}{C}{g}
   \jfam{{{\Gamma}{A}}{\ctxwk{A}{C}}}{R}}
  {\jfameq
    {{\Gamma}{A}}
    {\subst{\jcomp{f}{g}}{R}}
    {\subst{f}{{\ctxwk{A}{g}}{\ctxwk{{A}{B}}{R}}}}}\\
& \inference
  {\jhom{\Gamma}{A}{B}{f}
   \jhom{\Gamma}{B}{C}{g}
   \jterm{{{\Gamma}{A}}{\ctxwk{A}{C}}}{R}{h}}
  {\jfameq
    {{\Gamma}{A}}
    {\subst{\jcomp{f}{g}}{h}}
    {\subst{f}{{\ctxwk{A}{g}}{\ctxwk{{A}{B}}{h}}}}}
\end{align*}
We also have the following related inference rules, asserting that composition
is strictly associative:
\begin{align*}
& \inference
  {\jhom{\Gamma}{A}{B}{f}
   \jhom{\Gamma}{B}{C}{g}
   \jfam{{\Gamma}{C}}{R}}
  {\jfameq
    {{\Gamma}{A}}
    {\jcomp{{f}{g}}{R}}
    {\jcomp{f}{{g}{R}}}}\\
& \inference
  {\jhom{\Gamma}{A}{B}{f}
   \jhom{\Gamma}{B}{C}{g}
   \jterm{{\Gamma}{C}}{R}{h}}
  {\jtermeq
    {{\Gamma}{A}}
    {\jcomp{{f}{g}}{R}}
    {\jcomp{{f}{g}}{h}}
    {\jcomp{f}{{g}{h}}}}
\end{align*}
\end{lem}

\begin{proof}
Consider family morphisms $\jhom{\Gamma}{A}{B}{f}$ and $\jhom{\Gamma}{B}{C}{g}$
and a family $\jfam{{{\Gamma}{A}}{\ctxwk{A}{C}}}{R}$. Then we have the judgmental
equalities
\begin{align*}
\subst{\jcomp{f}{g}}{R} & \jdeq \subst{{f}{\ctxwk{A}{g}}}{R}\\
& \jdeq \subst{{f}{\ctxwk{A}{g}}}{\subst{f}{\ctxwk{{A}{B}}{R}}}\\
& \jdeq \subst{f}{{\ctxwk{A}{g}}{\ctxwk{{A}{B}}{R}}}
\end{align*}
The proof that 
$\subst{\jcomp{f}{g}}{h}\jdeq\subst{f}{{\ctxwk{A}{g}}{\ctxwk{{A}{B}}{h}}}$
is similar.

Now suppose that $\jfam{{\Gamma}{C}}{R}$ instead. Then we have
\begin{align*}
\jcomp{{f}{g}}{R} & \jdeq \subst{\jcomp{f}{g}}{\ctxwk{A}{R}}\\
& \jdeq \subst{{f}{\ctxwk{A}{g}}}{\ctxwk{A}{R}}\\
& \jdeq \subst{f}{{\ctxwk{A}{g}}{\ctxwk{{A}{B}}{{A}{R}}}}\\
& \jdeq \subst{f}{{\ctxwk{A}{g}}{\ctxwk{A}{{B}{R}}}}\\
& \jdeq \subst{f}{\ctxwk{A}{\subst{g}{\ctxwk{B}{R}}}}\\
& \jdeq \subst{f}{\ctxwk{A}{\jcomp{g}{R}}}\\
& \jdeq \jcomp{f}{{g}{R}}.
\end{align*}
Again, the proof is similar for terms $h$ of $R$ in context $\ctxext{\Gamma}{C}$.
\end{proof}

\begin{lem}
We have the following inference rules about the compatibility of composition with
weakening:
\begin{align*}
& \inference
  {\jhom{\Gamma}{A}{B}{f}
   \jhom{\Gamma}{B}{C}{g}
   \jfam{{\Gamma}{A}}{P}}
  {\jhomeq{\Gamma}{{A}{P}}{C}
    {\ctxwk{P}{\jcomp{f}{g}}}
    {\jcomp{\ctxwk{P}{f}}{g}}}\\
& \inference
  {\jterm{\Gamma}{B}{y}
   \jhom{\Gamma}{B}{C}{g}}
  {\jhomeq{\Gamma}{A}{C}
    {\jcomp{\ctxwk{A}{y}}{g}}
    {\ctxwk{A}{\subst{y}{g}}}}\\
& \inference
  {\jhom{\Gamma}{A}{B}{f}
   \jterm{\Gamma}{C}{z}}
  {\jhomeq{\Gamma}{A}{C}
    {\jcomp{f}{\ctxwk{B}{z}}}
    {\ctxwk{A}{z}}}
\end{align*}
\end{lem}

\begin{proof}
Let $\jhom{\Gamma}{A}{B}{f}$, $\jhom{\Gamma}{B}{C}{g}$ and $\jfam{{\Gamma}{A}}{P}$.
Then we have the judgmental equalities
\begin{align*}
\ctxwk{P}{\jcomp{f}{g}} 
& \jdeq \ctxwk{P}{\subst{f}{\ctxwk{A}{g}}}\\
& \jdeq \subst{\ctxwk{P}{f}}{\ctxwk{P}{{A}{g}}}\\
& \jdeq \subst{\ctxwk{P}{f}}{\ctxwk{\ctxext{A}{P}}{g}}\\
& \jdeq \jcomp{\ctxwk{P}{f}}{g}.
\end{align*}
Now let $\jterm{\Gamma}{B}{y}$ and $\jhom{\Gamma}{B}{C}{g}$. Then we have the
judgmental equalities
\begin{align*}
\jcomp{\ctxwk{A}{y}}{g}
& \jdeq \subst{\ctxwk{A}{y}}{\ctxwk{A}{g}}\\
& \jdeq \ctxwk{A}{\subst{y}{g}}.
\end{align*}
For the third assertion, let $\jhom{\Gamma}{A}{B}{f}$ and $\jterm{\Gamma}{C}{z}$.
Then we have the judgmental equalities
\begin{align*}
\jcomp{f}{\ctxwk{B}{z}} 
& \jdeq \subst{f}{\ctxwk{A}{{B}{z}}}\\
& \jdeq \subst{f}{\ctxwk{{A}{B}}{{A}{z}}}\\
& \jdeq \ctxwk{A}{z}.
\end{align*}
\end{proof}

\begin{lem}
We have the following inference rules about the compatibility of composition with
substitution:
\begin{align*}
& \inference
  {\jhom{{\Gamma}{A}}{P}{Q}{f}
   \jhom{{\Gamma}{A}}{Q}{R}{g}
   \jterm{\Gamma}{A}{x}}
  {\jhomeq{\Gamma}{\subst{x}{P}}{\subst{x}{R}}
    {\subst{x}{\jcomp{f}{g}}}
    {\jcomp{\subst{x}{f}}{\subst{x}{g}}}}\\
& \inference
  {\jhom{\Gamma}{A}{B}{f}
   \jhom{\Gamma}{B}{C}{g}
   \jterm{\Gamma}{A}{x}}
  {\jtermeq{\Gamma}{C}
    {\subst{x}{\jcomp{f}{g}}}
    {\subst{{x}{f}}{g}}}
\end{align*}
\end{lem}

\begin{proof}
Let $\jhom{{\Gamma}{A}}{P}{Q}{f}$, $\jhom{{\Gamma}{A}}{Q}{R}{g}$ and $\jterm{\Gamma}{A}{x}$.
Then we have the judgmental equalities
\begin{align*}
\subst{x}{\jcomp{f}{g}}
& \jdeq \subst{x}{{f}{\ctxwk{P}{g}}}\\
& \jdeq \subst{{x}{f}}{{x}{\ctxwk{P}{g}}}\\
& \jdeq \subst{{x}{f}}{\ctxwk{\subst{x}{P}}{\subst{x}{g}}}\\
& \jdeq \jcomp{\subst{x}{f}}{\subst{x}{g}}.
\end{align*}
Now let $\jhom{\Gamma}{A}{B}{f}$, $\jhom{\Gamma}{B}{C}{g}$ and $\jterm{\Gamma}{A}{x}$.
Then we have the judgmental equalities
\begin{align*}
\subst{x}{\jcomp{f}{g}}
& \jdeq \subst{x}{{f}{\ctxwk{A}{g}}}\\
& \jdeq \subst{{x}{f}}{{x}{\ctxwk{A}{g}}}\\
& \jdeq \subst{{x}{f}}{g}.
\end{align*}
\end{proof}

There is also a notion of morphism \emph{over} a morphism. We will develop this
notion because it will be needed in the theory of models later on.

\begin{defn}
Let $\jhom{\Gamma}{A}{B}{f}$ be a morphism from $A$ to $B$ in context $\Gamma$
and consider $\jfam{{\Gamma}{A}}{P}$ and $\jfam{{\Gamma}{B}}{Q}$. We define the
judgment
\begin{equation*}
\jfhom{\Gamma}{A}{B}{f}{P}{Q}{F},
\end{equation*}
which is pronounced as `$F$ is a morphism from $P$ to $Q$ over $f$ in context
$\Gamma$, to be the judgment
\begin{equation*}
\unfold{\jfhom{\Gamma}{A}{B}{f}{P}{Q}{F}}.
\end{equation*}
\end{defn}

We can also define composition for this version of morphism and formulate
compatibility rules for them. 

\subsubsection{Identity functions}
Without a rule explicitly asserting the existence of an identity morphism we don't
get one, hence we do that here. The identity morphism is a term which introduced
in ordinary type theory via the variable rule. The variable rule is a bit more
general: it asserts that
\begin{equation*}
\jterm{\Gamma,\,x_1:A_1,\ldots,\,x_n:A_n}{A_i}{x_i}
\end{equation*}
for every $1\leq i\leq n$. Thus, it establishes the projections. In our setting,
we get the projections from the identity morphisms together with weakening. We
already have weakening, so here it suffices to introduce the identity morphisms.
\begin{align}
& \inference
  {\jfam{\Gamma}{A}}
  {\jterm{{\Gamma}{A}}{\ctxwk{A}{A}}{\idfunc[A]}}
& & \inference
    {\jfameq{\Gamma}{A}{A'}}
    {\jtermeq{{\Gamma}{A}}{\ctxwk{A}{A}}{\idfunc[A]}{\idfunc[A']}}
\end{align}
Identity functions are determined by their behavior with respect to substitution combined with
weakening. The identity functions will also be subject to compatibility rules.
\begin{align}
& \inference
  {\jterm{\Gamma}{A}{x}}
  {\jtermeq{\Gamma}{A}{\subst{x}{\idfunc[A]}}{x}}
  \label{idfunc-subst-defn}\\
& \inference
  {\jfam{{\Gamma}{A}}{P}}
  {\jfameq{{\Gamma}{A}}{\subst{\idfunc[A]}{\ctxwk{A}{P}}}{P}}
  \label{idfunc-wk-defn}\\
& \inference
  {\jfam{{\Gamma}{A}}{P}}
  {\jfameq{{\Gamma}{A}}{\subst{\idfunc[A]}{\ctxwk{{A}{A}}{P}}}{P}}
  \label{idfunc-wk-defn2}\\
& \inference
  {\jterm{{\Gamma}{A}}{P}{f}}
  {\jtermeq{{\Gamma}{A}}{P}{\subst{\idfunc[A]}{\ctxwk{A}{f}}}{f}}
  \label{idfunc-precomp}\\
& \inference
  {\jterm{{\Gamma}{A}}{P}{f}}
  {\jtermeq{{\Gamma}{A}}{P}{\subst{\idfunc[A]}{\ctxwk{{A}{A}}{f}}}{f}}
  \label{idfunc-precomp}\\
& \inference
  {\jterm{{\Gamma}{B}}{\ctxwk{B}{A}}{g}}
  {\jtermeq{{\Gamma}{B}}{\ctxwk{B}{A}}{\subst{g}{\ctxwk{B}{\idfunc[A]}}}{g}}
  \label{idfunc-postcomp}
\end{align}

We won't state a compatibility rule stating that the identity function is
compatible with extension because we will be able to prove that. Instead, we
will just state the compatibility rules for the identity function combined with
weakening and with substitution.

The identity function of a weakened family is the weakened identity function:
\begin{equation}\label{idfunc-wk-comp}
\inference
  {\jfam{\Gamma}{A}
   \jfam{\Gamma}{B}}
  {\jtermeq{{{\Gamma}{A}}{\ctxwk{A}{B}}}{\ctxwk{{A}{B}}{{A}{B}}}{\idfunc[\ctxwk{A}{B}]}{\ctxwk{A}{\idfunc[B]}}}
\end{equation}

The identity function of a substituted family is the substitution of the identity function
\begin{equation}\label{idfunc-subst-comp}
\inference
  {\jterm{\Gamma}{A}{x}
   \jfam{{\Gamma}{A}}{P}}
  {\jtermeq{{\Gamma}{\subst{x}{P}}}{\ctxwk{\subst{x}{P}}{\subst{x}{P}}}{\idfunc[\subst{x}{P}]}{\subst{x}{\idfunc[P]}}}
\end{equation}

\subsection{Extension on terms}\label{extension-on-terms}
In this subsection we consider the notion of extension on terms, which has now
become definable inside our theory. Moreover, every compatibility rule one may
dream of is provable as well, using the compatibility rules we have introduced
earlier. \emph{This subsection contains no new assumptions.}

\begin{defn}
When $\jterm{\Gamma}{A}{x}$ and $\jterm{\Gamma}{\subst{x}{P}}{u}$ are terms,
we define 
\begin{equation*}
\jtermdefn{\Gamma}{\ctxext{A}{P}}{\ctxext{x}{u}}{\subst{u}{{x}{\idfunc[\ctxext{A}{P}]}}}.
\end{equation*} 
\end{defn}

Thus, the term $\ctxext{x}{u}$ is the pairing of $x$ and $u$. Note that because
we have the judgmental equality 
$\ctxwk{P}{{A}{\ctxext{A}{P}}}\jdeq\ctxwk{\ctxext{A}{P}}{\ctxext{A}{P}}$ in the
context $\ctxext{{\Gamma}{A}}{P}$, the
pairing function could just be defined as $\idfunc[\ctxext{A}{P}]$. 

When we substitute by an extended term we get an equal result as when we
substitute two consecutive times, like the way currying works.

\begin{lem}
The following inference rules are valid:
\begin{align*}
& \inference
  {\jterm{\Gamma}{A}{x}
   \jterm{\Gamma}{\subst{x}{P}}{u}
   \jfam{{{\Gamma}{A}}{P}}{Q}}
  {\jfameq{\Gamma}{\subst{\ctxext{x}{u}}{Q}}{\subst{u}{{x}{Q}}}}\\
& \inference
  {\jterm{\Gamma}{A}{x}
   \jterm{\Gamma}{\subst{x}{P}}{u}
   \jterm{{{\Gamma}{A}}{P}}{Q}{g}}
  {\jtermeq{\Gamma}{\subst{u}{{x}{Q}}}{\subst{\ctxext{x}{u}}{g}}{\subst{u}{{x}{g}}}}\\
& \inference
  {\jterm{\Gamma}{A}{x}
   \jterm{\Gamma}{\subst{x}{P}}{u}
   \jfam{{{{\Gamma}{A}}{P}}{Q}}{R}}
  {\jfameq{{\Gamma}{\subst{u}{{x}{Q}}}}{\subst{\ctxext{x}{u}}{R}}{\subst{u}{{x}{R}}}}\\
& \inference
  {\jterm{\Gamma}{A}{x}
   \jterm{\Gamma}{\subst{x}{P}}{u}
   \jterm{{{{\Gamma}{A}}{P}}{Q}}{R}{t}}
  {\jtermeq{{\Gamma}{\subst{u}{{x}{Q}}}}{\subst{u}{{x}{R}}}{\subst{\ctxext{x}{u}}{t}}{\subst{u}{{x}{t}}}}
\end{align*}
\end{lem}

\begin{proof}
We prove only the first judgmental equality. All the others are similar.
Let $\jterm{\Gamma}{A}{x}$ and $\jterm{\Gamma}{\subst{x}{P}}{u}$
be terms and let $\jfam{{{\Gamma}{A}}{P}}{Q}$ be a family. Then we have
\begin{align*}
\subst{\ctxext{x}{u}}{Q} 
& \jdeq \subst{{u}{{x}{\idfunc[\ctxext{A}{P}]}}}{Q} \tag{by definition}\\
& \jdeq \subst{{u}{{x}{\idfunc[\ctxext{A}{P}]}}}{{x}{\ctxwk{A}{Q}}}\tag{by \autoref{defn-ws-3}}\\
& \jdeq \subst{{u}{{x}{\idfunc[\ctxext{A}{P}]}}}{{u}{\ctxwk{\subst{x}{P}}{\subst{x}{\ctxwk{A}{Q}}}}}\tag{by \autoref{defn-ws-3}}\\
& \jdeq \subst{{u}{{x}{\idfunc[\ctxext{A}{P}]}}}{{u}{{x}{\ctxwk{P}{{A}{Q}}}}}\tag{by \autoref{comp-sw-f}}\\
& \jdeq \subst{u}{{{x}{\idfunc[\ctxext{A}{P}]}}{{x}{\ctxwk{{P}{{A}{Q}}}}}}\tag{by \autoref{comp-ss-f}}\\
& \jdeq \subst{u}{{x}{{\idfunc[\ctxext{A}{P}]}{\ctxwk{P}{{A}{Q}}}}}\tag{by \autoref{comp-ss-f}}\\
& \jdeq \subst{u}{{x}{{\idfunc[\ctxext{A}{P}]}{\ctxwk{\ctxext{A}{P}}{Q}}}}\tag{by \autoref{comp-ew-f}}\\
& \jdeq \subst{u}{{x}{Q}}\tag{by \autoref{idfunc-wk-defn}}
\end{align*}
\end{proof}

We have seen above that the pairing function into $\ctxext{A}{P}$ is just the identity function on
$\ctxext{A}{P}$. To analyze the pairing functin a little further, we will also
need the projection maps from $\ctxext{A}{P}$ to $A$ and from $\ctxext{A}{P}$
to $P$. We will now define these and see that the identity function of an
extended family is the extension (or pairing) of the identity
functions on the components in the apropriate way.

To find out what the
apropriate way is, note that
\begin{align*}
\ctxwk{\ctxext{A}{P}}{\ctxext{A}{P}} & \jdeq \ctxwk{P}{{A}{\ctxext{A}{P}}}\\
& \jdeq \ctxext{\ctxwk{P}{{A}{A}}}{\ctxwk{P}{{A}{P}}}
\end{align*}
We have the term $\jterm{{\Gamma}{{A}{P}}}{\ctxwk{P}{A}}{\ctxwk{P}{\idfunc[A]}}$.
Thus we need to find out what $\subst{\ctxwk{P}{\idfunc[A]}}{\ctxwk{P}{{A}{P}}}$ is:
\begin{align*}
\subst{\ctxwk{P}{\idfunc[A]}}{\ctxwk{P}{{A}{P}}} & \jdeq \ctxwk{P}{\subst{\idfunc[A]}{\ctxwk{A}{P}}}\\
& \jdeq \ctxwk{P}{P},
\end{align*}
where we find the term $\idfunc[P]$. Therefore we define:

\begin{defn}
Let $\jfam{\Gamma}{A}$ and $\jfam{{\Gamma}{A}}{P}$ be families. We define
\begin{align*}
\jtermdefn*{\ctxext{\Gamma}{{A}{P}}}{\ctxwk{P}{{A}{A}}}{\cprojfstf{A}{P}}{\ctxwk{P}{\idfunc[A]}}\\
\jtermdefn*{\ctxext{\Gamma}{{A}{P}}}{\ctxwk{P}{P}}{\cprojsndf{A}{P}}{\idfunc[P]}
\end{align*}
\end{defn}

We find the following inference rule, which expresses that the identity function
is compatible with extension:

\begin{lem}
For any $\jfam{\Gamma}{A}$ and $\jfam{{\Gamma}{A}}{P}$ we have
\begin{equation}\label{idfunc-ext-comp}
\inference
  {\jfam{\Gamma}{A}
   \jfam{{\Gamma}{A}}{P}}
  {\jtermeq
    {{\Gamma}{{A}{P}}}
    {\ctxwk{\ctxext{A}{P}}{\ctxext{A}{P}}}
    {\idfunc[\ctxext{A}{P}]}
    {\ctxext{\cprojfstf{A}{P}}{\cprojsndf{A}{P}}}}
\end{equation}
\end{lem}

\begin{proof}
Consider the families $\jfam{\Gamma}{A}$ and $\jfam{{\Gamma}{A}}{P}$. Then
we have the judgmental equalities
\begin{align*}
\ctxext{\cprojfstf{A}{P}}{\cprojsndf{A}{P}}
& \jdeq \subst{\cprojsndf{A}{P}}{{\cprojfstf{A}{P}}{\idfunc[\ctxext{\ctxwk{\ctxext{A}{P}}{A}}{\ctxwk{\ctxext{A}{P}}{P}}]}}\\
& \jdeq \subst
          {\cprojsndf{A}{P}}
          { {\cprojfstf{A}{P}}
            {\idfunc[\ctxwk{\ctxext{A}{P}}{\ctxext{A}{P}}]}}\\
& \jdeq \subst
          {\idfunc[P]}
          { {\ctxwk{P}{\idfunc[A]}}
            {\idfunc[\ctxwk{\ctxext{A}{P}}{\ctxext{A}{P}}]}}\\
& \jdeq \subst
          {\idfunc[P]}
          { {\ctxwk{P}{\idfunc[A]}}
            {\ctxwk{\ctxext{A}{P}}{\idfunc[\ctxext{A}{P}]}}}\\
& \jdeq \subst
          {\idfunc[P]}
          { {\ctxwk{P}{\idfunc[A]}}
            {\ctxwk{P}{{A}{\idfunc[\ctxext{A}{P}]}}}}\\
& \jdeq \subst
          {\idfunc[P]}
          {\ctxwk
            {P}
            {\subst
              {\idfunc[A]}
              {\ctxwk{A}{\idfunc[\ctxext{A}{P}]}}}}\\
& \jdeq \subst
          {\idfunc[A]}
          {\ctxwk{A}{\idfunc[\ctxext{A}{P}]}}\\
& \jdeq \idfunc[\ctxext{A}{P}].
\end{align*}
\end{proof}

\begin{lem}
The following inference is valid:
\begin{equation*}
\inference
  {\jfam{{\Gamma}{A}}{P}
   \jfam{{\Gamma}{A}}{Q}}
  {\jfameq{{\Gamma}{{A}{P}}}{\subst{\cprojfstf{A}{P}}{\ctxwk{\ctxext{A}{P}}{Q}}}{\ctxwk{P}{Q}}}
\end{equation*}
\end{lem}

\begin{proof}
Let $\jfam{{\Gamma}{A}}{P}$ and $\jfam{{\Gamma}{A}}{Q}$ be
families. Then we have
\begin{align*}
\subst{\cprojfstf{A}{P}}{\ctxwk{\ctxext{A}{P}}{Q}} 
& \jdeq \subst{\ctxwk{P}{\idfunc[A]}}{\ctxwk{\ctxext{A}{P}}{Q}} \tag{by definition}\\
& \jdeq \subst{\ctxwk{P}{\idfunc[A]}}{\ctxwk{P}{{A}{Q}}} \tag{by \autoref{comp-ww-f}}\\
& \jdeq \ctxwk{P}{\subst{\idfunc[A]}{\ctxwk{A}{Q}}} \tag{by \autoref{comp-ws-f}}\\
& \jdeq \ctxwk{P}{Q} \tag{by \autoref{idfunc-wk-defn}}
\end{align*}
\end{proof}

The constructions of the terms $\ctxext{x}{u}$ and $\cprojfst{A}{P}{w}$ and
$\cprojsnd{A}{P}{w}$ are subject to various rules, with all of them being
consequences of earlier introduced inference rules.

\begin{lem} The following inference rules expressing that pairing is a strict
inverse to the combination of decompositions, are valid:
\begin{align*}
& \inference
  {\jterm{\Gamma}{\ctxext{A}{P}}{w}}
  {\jtermeq{\Gamma}{\ctxext{A}{P}}{\ctxext{\cprojfst{A}{P}{w}}{\cprojsnd{A}{P}{w}}}{w}}\\
& \inference
  {\jterm{\Gamma}{A}{x}
   \jterm{\Gamma}{\subst{x}{P}}{u}}
  {\jtermeq{\Gamma}{A}{\cprojfst{A}{P}{\ctxext{x}{u}}}{x}}\\
& \inference
  {\jterm{\Gamma}{A}{x}
   \jterm{\Gamma}{\subst{x}{P}}{u}}
  {\jtermeq{\Gamma}{\subst{x}{P}}{\cprojsnd{A}{P}{\ctxext{x}{u}}}{u}}
\end{align*}
\end{lem}

\begin{proof}
To prove the first judgmental equality, note that
\begin{align*}
w & \jdeq \subst{w}{\idfunc[\ctxext{A}{P}]} \tag{by \autoref{idfunc-subst-defn}}\\
& \jdeq \subst{w}{\subst{\idfunc[P]}{{\ctxwk{P}{\idfunc[A]}}{\idfunc[\ctxwk{\ctxext{A}{P}}{\ctxext{A}{P}}]}}}
  \tag{by \autoref{idfunc-ext-comp}}\\
& \jdeq \subst{{w}{\idfunc[P]}}{\subst{w}{{\ctxwk{P}{\idfunc[A]}}{\idfunc[\ctxwk{\ctxext{A}{P}}{\ctxext{A}{P}}]}}}\tag{by \autoref{comp-ss-t}}\\
& \jdeq \subst{{w}{\idfunc[P]}}{{\subst{w}{\ctxwk{P}{\idfunc[A]}}}{\subst{w}{\idfunc[\ctxwk{\ctxext{A}{P}}{\ctxext{A}{P}}]}}}\tag{by \autoref{comp-ss-t}}\\
& \jdeq \subst{{w}{\idfunc[P]}}{{\subst{w}{\ctxwk{P}{\idfunc[A]}}}{\subst{w}{\ctxwk{\ctxext{A}{P}}{\idfunc[\ctxext{A}{P}]}}}}\tag{by \autoref{idfunc-wk-comp}}\\
& \jdeq \subst{{w}{\idfunc[P]}}{{\subst{w}{\ctxwk{P}{\idfunc[A]}}}{\idfunc[\ctxext{A}{P}]}}\tag{by \autoref{defn-ws-4}}\\
& \jdeq \ctxext{\cprojfst{A}{P}{w}}{\cprojsnd{A}{P}{w}}\tag{by definition}
\end{align*}
To prove the second judgmental equality, let $\jterm{\Gamma}{A}{x}$ and
$\jterm{\Gamma}{\subst{x}{P}}{u}$. Then we have
\begin{align*}
\cprojfst{A}{P}{\ctxext{x}{u}}
& \jdeq \subst{\ctxext{x}{u}}{\ctxwk{P}{\idfunc[A]}}\\
& \jdeq \subst{u}{{x}{\ctxwk{P}{\idfunc[A]}}} \\
& \jdeq \subst{u}{\ctxwk{\subst{x}{P}}{\subst{x}{\idfunc[A]}}}\\
& \jdeq \subst{x}{\idfunc[A]}\\
& \jdeq x.
\end{align*}
To prove the third judgmental equality, note that
\begin{align*}
\cprojsnd{A}{P}{\ctxext{x}{u}}
& \jdeq \subst{\ctxext{x}{u}}{\idfunc[P]}\\
& \jdeq \subst{u}{{x}{\idfunc[P]}}\\
& \jdeq \subst{u}{\idfunc[\subst{x}{P}]}\\
& \jdeq u.
\end{align*}
\end{proof}

\begin{lem}
The following compatibility rules for two consecutive term extensions is valid:
\begin{align*}
& \inference
  {\jterm{\Gamma}{A}{x}
   \jterm{\Gamma}{\subst{x}{P}}{u}
   \jterm{\Gamma}{\subst{\ctxext{x}{u}}{Q}}{v}}
  {\jtermeq{\Gamma}{\ctxext{{A}{P}}{Q}}{\ctxext{x}{{u}{v}}}{\ctxext{{x}{u}}{v}}}\\
& \inference
  {\jterm{\Gamma}{\ctxext{{A}{P}}{Q}}{w}}
  {\jtermeq{\Gamma}{A}{\cprojfst{A}{P}{\cprojfst{\ctxext{A}{P}}{Q}{w}}}{\cprojfst{A}{\ctxext{P}{Q}}{w}}}\\
& \inference
  {\jterm{\Gamma}{\ctxext{{A}{P}}{Q}}{w}}
  {\jtermeq{\Gamma}{\subst{\cprojfst{A}{\ctxext{P}{Q}}{w}}{P}}
    {\cprojsnd{A}{P}{\cprojfst{\ctxext{A}{P}}{Q}{w}}}
    {\cprojfst{P}{Q}{\cprojsnd{A}{\ctxext{P}{Q}}{w}}}}\\
& \inference
  {\jterm{\Gamma}{\ctxext{{A}{P}}{Q}}{w}}
  {\jtermeq{\Gamma}{\subst{\cprojfst{\ctxext{A}{P}}{Q}{w}}{Q}}
    {\cprojsnd{P}{Q}{\cprojsnd{A}{\ctxext{P}{Q}}{w}}}
    {\cprojsnd{\ctxext{A}{P}}{Q}{w}}}
\end{align*}
\end{lem}

\begin{proof}
Consider terms $\jterm{\Gamma}{A}{x}$, $\jterm{\Gamma}{\subst{x}{P}}{u}$ and
$\jterm{\Gamma}{\subst{u}{{x}{Q}}}{v}$. Then we have
\begin{align*}
\ctxext{x}{{u}{v}} & \jdeq \subst{\ctxext{u}{v}}{{x}{\idfunc[\ctxext{A}{{P}{Q}}]}}\\
& \jdeq \subst{v}{{u}{{x}{\idfunc[\ctxext{A}{{P}{Q}}]}}}\\
& \jdeq \subst{v}{{u}{{x}{\idfunc[\ctxext{{A}{P}}{Q}]}}}\\
& \jdeq \subst{v}{{\ctxext{x}{u}}{\idfunc[\ctxext{{A}{P}}{Q}]}}\\
& \jdeq \ctxext{{x}{u}}{v}.
\end{align*}
Now consider a term $\jterm{\Gamma}{\ctxext{A}{{P}{Q}}}{w}$. Then we have
\begin{align*}
w 
& \jdeq \ctxext{\cprojfst{A}{\ctxext{P}{Q}}{w}}{\cprojsnd{A}{\ctxext{P}{Q}}{w}}\\
& \jdeq \ctxext{\cprojfst{A}{\ctxext{P}{Q}}{w}}{{\cprojfst{P}{Q}{\cprojsnd{A}{\ctxext{P}{Q}}{w}}}{\cprojsnd{P}{Q}{\cprojsnd{A}{\ctxext{P}{Q}}{w}}}}\\
& \jdeq \ctxext{{\cprojfst{A}{\ctxext{P}{Q}}{w}}{\cprojfst{P}{Q}{\cprojsnd{A}{\ctxext{P}{Q}}{w}}}}{\cprojsnd{P}{Q}{\cprojsnd{A}{\ctxext{P}{Q}}{w}}}
\end{align*}
Thus we see that 
\begin{align*}
\cprojfst{\ctxext{A}{P}}{Q}{w} & \jdeq \ctxext{\cprojfst{A}{\ctxext{P}{Q}}{w}}{\cprojfst{P}{Q}{\cprojsnd{A}{\ctxext{P}{Q}}{w}}}\\ 
\cprojsnd{\ctxext{A}{P}}{Q}{w} & \jdeq \cprojsnd{P}{Q}{\cprojsnd{A}{\ctxext{P}{Q}}{w}},
\end{align*}
proving the fourth judgmental equality, and therefore also that
\begin{align*}
\cprojfst{A}{P}{\cprojfst{\ctxext{A}{P}}{Q}{w}} & \jdeq \cprojfst{A}{\ctxext{P}{Q}}{w}\\
\cprojsnd{A}{P}{\cprojfst{\ctxext{A}{P}}{Q}{w}} & \jdeq \cprojfst{P}{Q}{\cprojsnd{A}{\ctxext{P}{Q}}{w}},
\end{align*}
proving the second and the third judgmental equalities.
\end{proof}

\begin{lem}
When we weaken a term $\ctxext{y}{v}$ of $\ctxext{B}{Q}$ in context $\Gamma$ by
a family $A$, the term that we get is $\ctxext{\ctxwk{A}{y}}{\ctxwk{A}{v}}$. More
precisely, the following inference rules are valid:
\begin{align*}
& \inference
  {\jterm{{\Gamma}{B}}{Q}{g}
   \jterm{{\Gamma}{B}}{\subst{g}{R}}{t}}
  {\jtermeq
    {{{\Gamma}{A}}{\ctxwk{A}{B}}}
    {\ctxwk{A}{\ctxext{Q}{R}}}
    {\ctxwk{A}{\ctxext{g}{t}}}
    {\ctxext{\ctxwk{A}{g}}{\ctxwk{A}{t}}}}
% \\
%& \inference{\jterm{\Gamma}{\ctxwk{A}{\ctxext{B}{Q}}}{\ctxwk{A}{w}}}{\jtermeq{{\Gamma}{A}}{\ctxwk{A}{B}}{\cprojfst{\ctxwk{A}{B}}{\ctxwk{A}{Q}}{\ctxwk{A}{w}}}{\ctxwk{A}{(\cprojfst{B}{Q}{w})}}}\\
\end{align*}
\end{lem}

\begin{proof}
Consider $\jterm{{\Gamma}{B}}{Q}{g}$ and $\jterm{{\Gamma}{B}}{\subst{g}{R}}{t}$.
Then we have the judgmental equalities
\begin{align*}
\ctxwk{A}{\ctxext{g}{t}}
& \jdeq \ctxwk{A}{\subst{t}{{g}{\idfunc[\ctxext{Q}{R}]}}}\\
& \jdeq \subst{\ctxwk{A}{t}}{\ctxwk{A}{\subst{g}{\idfunc[\ctxext{Q}{R}]}}}\\
& \jdeq \subst{\ctxwk{A}{t}}{{\ctxwk{A}{g}}{\ctxwk{A}{\idfunc[\ctxext{Q}{R}]}}}\\
& \jdeq \subst{\ctxwk{A}{t}}{{\ctxwk{A}{g}}{\idfunc[\ctxwk{A}{\ctxext{Q}{R}}]}}\\
& \jdeq \subst{\ctxwk{A}{t}}{{\ctxwk{A}{g}}{\idfunc[\ctxext{\ctxwk{A}{Q}}{\ctxwk{A}{R}}]}}\\
& \jdeq \ctxext{\ctxwk{A}{g}}{\ctxwk{A}{t}}
\end{align*}
\end{proof}

\begin{lem}
When we substitute an extended term $\ctxext{f}{g}$ of $\ctxext{P}{Q}$ by a term
$x$ of $A$, the term that we get is $\ctxext{\subst{x}{f}}{\subst{x}{g}}$.
More precisely, the following inference rules are valid:
\begin{align*}
& \inference
  {\jterm{{{\Gamma}{A}}{P}}{Q}{g}
   \jterm{{{\Gamma}{A}}{P}}{\subst{g}{R}}{t}}
  {\jtermeq
    {{\Gamma}{\subst{x}{P}}}
    {\ctxext{\subst{x}{Q}}{\subst{x}{R}}}
    {\subst{x}{\ctxext{g}{t}}}
    {\ctxext{\subst{x}{g}}{\subst{x}{t}}}}
\end{align*}
\end{lem}

\begin{proof}
Consider $\jterm{{\Gamma}{B}}{Q}{g}$ and $\jterm{{\Gamma}{B}}{\subst{g}{R}}{t}$.
Then we have the judgmental equalities
\begin{align*}
\subst{x}{\ctxext{g}{t}}
& \jdeq \subst{x}{{t}{{g}{\idfunc[\ctxext{Q}{R}]}}}\\
& \jdeq \subst{{x}{t}}{{x}{{g}{\idfunc[\ctxext{Q}{R}]}}}\\
& \jdeq \subst{{x}{t}}{{{x}{g}}{{x}{\idfunc[\ctxext{Q}{R}]}}}\\
& \jdeq \subst{{x}{t}}{{{x}{g}}{\idfunc[\subst{x}{\ctxext{Q}{R}}]}}\\
& \jdeq \subst{{x}{t}}{{{x}{g}}{\idfunc[\ctxext{\subst{x}{Q}}{\subst{x}{R}}]}}\\
& \jdeq \ctxext{\subst{x}{g}}{\subst{x}{t}}.
\end{align*}
\end{proof}

We also have the following lemma about the compatibility of pairing and composition:

\begin{lem}
The following inference rule is valid
\begin{align*}
& \inference
  {\jhom{\Gamma}{A}{B}{f}
   \jhom{\Gamma}{B}{C}{g}
   \jfam{{\Gamma}{C}}{R}
   \jterm{{\Gamma}{B}}{\subst{g}{\ctxwk{B}{R}}}{w}}
  {\jhomeq{\Gamma}{A}{{C}{R}}
    {\jcomp{f}{\ctxext{g}{w}}}
    {\ctxext{\jcomp{f}{g}}{\jcomp{f}{w}}}}
\end{align*}
\end{lem}

\begin{proof}
Let $\jhom{\Gamma}{A}{B}{f}$, $\jhom{\Gamma}{B}{C}{g}$, $\jfam{{\Gamma}{C}}{R}$
and $\jterm{{\Gamma}{B}}{\subst{g}{\ctxwk{B}{R}}}{w}$. Then we have the
judgmental equalities
\begin{align*}
\jcomp{f}{\ctxext{g}{w}}
& \jdeq \subst{f}{\ctxwk{A}{\ctxext{g}{w}}}\\
& \jdeq \subst{f}{\ctxext{\ctxwk{A}{g}}{\ctxwk{A}{w}}}\\
& \jdeq \ctxext{\subst{f}{\ctxwk{A}{g}}}{\subst{f}{\ctxwk{A}{w}}}\\
& \jdeq \ctxext{\jcomp{f}{g}}{\jcomp{f}{w}}.
\end{align*}
\end{proof}
%\end{document}

\section{Type constructors in structural type theory}
In this section we will introduce the usual type constructors to structural
type theory. We do so with the point of view that each type constructor is a
(class of) operations on type theory that should be compatible with extension,
weakening, substitution and identity morphisms. Likewise, each induction
principle of an inductively defined type constructor is going to be such an
operator, and it is therefore also required to be compatible with extension,
weakening, substitution and identity functions.

We will first state all the rules of the individual type constructors, where
only the $\wtypesym$-type constructor depends on the presence of dependent
function types (or universes too?), and hence comes with the notion that $\wtypesym$
is compatible with dependent product types and vice versa. All the other
compatibility properties are stated in \autoref{compatibility-of-type-constructors}.

\subsubsection{Issues to keep in mind}
\begin{enumerate}
\item The rule asserting that unit types are compatible with context extension makes
no sense from the point of view that contexts are lists of types and types
are contexts of length one; in other words, that types are indecomposable (i.e.~non-extended)
families. Remidies:
\begin{enumerate}
\item This point of view is wrong and should be abandoned.
\item We have, as Vladimir proposed, two kinds of judgmental equalities. One
      could be used for the very strict equalities, the other could be used
      to state the compatibility rules with. In other words, the other equality
      is a compatibility relation. If we do that, we shouldn't require that
      if a family is compatible with a type, then the family is a type. The
      compatibility relation could relate things by uniqueness up to unique
      isomorphism.
\item Don't state such compatibility rules for the type constructors.
\item Don't even give the unit type a dependent action.
\end{enumerate}
\end{enumerate}

\subsection{The unit type in structural type theory}
In this section we explore what we get if we pose compatibility conditions on
an inductively defined unit type. We will assume that not only the unit type
is compatible with extension, weakening, substitution and identity functions,
but also its induction principle should be compatible with those.

\begin{align*}
& \inference
  {\jfam{\Gamma}{A}}
  {\jtype{{\Gamma}{A}}{\unitc{A}}}
& & \inference
    {\jfameq{\Gamma}{A}{A'}}
    {\jtypeeq{{\Gamma}{A}}
      {\unitc{A}}
      {\unitc{A'}}}\\
& \inference
  {\jfam{\Gamma}{A}}
  {\jterm{{\Gamma}{A}}{\unitc{A}}{\unitct{A}}}
& & \inference
    {\jfameq{\Gamma}{A}{A'}}
    {\jtermeq{{\Gamma}{A}}{\unitc{A}}{\unitct{A}}{\unitct{A'}}}\\
& \inference
  {\jfam{{\Gamma}{A}}{P}}
  {\jtype{{{{\Gamma}{A}}{P}}{\ctxwk{P}{\unitc{A}}}}{\unitf{A}{P}}}
& & \inference
    {\jfameq{{\Gamma}{A}}{P}{P'}}
    {\jtypeeq{{{{\Gamma}{A}}{P}}{\ctxwk{P}{\unitc{A}}}}
      {\unitf{A}{P}}
      {\unitf{A}{P'}}}\\
& \inference
  {\jfam{{\Gamma}{A}}{P}}
  {\jterm{{{{\Gamma}{A}}{P}}{\ctxwk{P}{\unitc{A}}}}{\unitf{A}{P}}{\unitft{A}{P}}}
& & \inference
    {\jfameq{{\Gamma}{A}}{P}{P'}}
    {\jtermeq{{{{\Gamma}{A}}{P}}{\ctxwk{P}{\unitc{A}}}}{\unitf{A}{P}}
      {\unitft{A}{P}}
      {\unitft{A}{P'}}}
\end{align*}

We impose the following compatibility rules for the unit type:

\begin{align*}
& \inference
  {\jfam{{\Gamma}{A}}{P}}
  {\jtypeeq{{{\Gamma}{A}}{P}}{\ctxext{\ctxwk{P}{\unitc{A}}}{\unitf{A}{P}}}{\unitc{{\Gamma}{A}}}}\\
& \inference
  {\jfam{\Gamma}{A}
   \jfam{\Gamma}{B}}
  {\jtypeeq{{{\Gamma}{A}}{\ctxwk{A}{B}}}{}{}}  
\end{align*}

\subsection{The empty type}

\subsection{The natural numbers}

\subsection{Dependent pair types}

\subsection{Identity types}

\subsection{Dependent function types}

\subsection{$\wtypesym$-types}

\subsection{Universes}

\subsection{The compatibility of the type constructors with each other}
\label{compatibility-of-type-constructors}

\section{The type theory of models of type theory}
In this section we pursue the idea of what a general model of type theory is by
axiomatizing what you can do with them. We have the following ideas:
\begin{enumerate}
\item There are dependent models and sections thereof. Particular instances
of sections: extension, weakening and substitution (and something for identity
function?). And like the original
extension, weakening and substitution, they're going to be compatible with each
other. Therefore we state our theory of models as an extension of type theory
without constructors.
\item Type theory without constructors is a model of itself, the canonical model $\mctx$.
\item for any (family of) model(s) $A$, there is the family model $\mfam{A}$ which
is a family of models over $A$.
\item The terms of $\mctx$ should be precisely contexts. Terms of $\subst{\Gamma}{\mfam{\mctx}}$
should be families over $\Gamma$.
\item if a model and a family of models over it are given, there is an extended model.
If we extend $A$ by $\mfam{A}$, we get the Sierpinski model of $A$.
\item Likewise, models can be weakened and substituted and there are identity
functions.
\end{enumerate}
So the theory of models is going to be an extension of this theory with this
data. The theory we are about to describe can be seen as an elementary theory
of the category (with families (and terms)) of categories (with families (and terms)).

\subsection{The basic ingredients of the type theory of models}
We first introduce the basic ingredients of our abstract theory of models.
\begin{align*}
& \inference
  {}
  {\jctx{\mctx}}
  \tag{the canonical model}\\
& \inference
  {\jfam{\Gamma}{A}}
  {\jfam{{\Gamma}{A}}{\mfam{A}}}
  \tag{families}\\
& \inference
  {\jctx{\Gamma}}
  {\jfam{{\Gamma}{\mfam{\Gamma}}}{\mtm{\Gamma}}}
  \tag{terms}
\intertext{The following two rules essentially make a start with saying that every term is functorial
in the apropriate sense (there will be more rules contributing to this vies):}
& \inference
  {\jterm{{\Gamma}{A}}{P}{f}}
  {\jterm{{{\Gamma}{A}}{\mfam{A}}}{\subst{f}{\mfam{P}}}{\mfam{f}}}\\
& \inference
  {\jterm{{\Gamma}{A}}{P}{f}}
  {\jterm{{{{\Gamma}{A}}{\mfam{A}}}{\mtm{A}}}{\subst{\mfam{f}}{{f}{\mtm{P}}}}{\mtm{f}}}
\intertext{The empty context of a model $A$ is simply a term of $A$:}
& \inference
  {\jfam{\Gamma}{A}}
  {\jterm{\Gamma}{A}{\tfemp{A}}}
  \tag{empty context}
\intertext{Extension is going to be a morphism from $\ctxext{A}{\mfam{A}}$ to
$A$ in context $\Gamma$:}
& \inference
  {\jfam{\Gamma}{A}}
  {\jhom{\Gamma}{{A}{\mfam{A}}}{A}{\tfext{A}}}
  \tag{extension}
\intertext{The action of weakening on families
is a morphism from $\ctxwk{\mfam{A}}{\mfam{A}}$ to
$\mfam{\mfam{A}}$ in context $\ctxext{{\Gamma}{A}}{\mfam{A}}$:}
& \inference
  {\jfam{\Gamma}{A}}
  {\jhom{{{\Gamma}{A}}{\mfam{A}}}{\ctxwk{\mfam{A}}{\mfam{A}}}
    {\mfam{\mfam{A}}}
    {\tfwk{A}^0}}
  \tag{weakening}
\intertext{The action of weakening on terms
is a morphism from $\ctxwk{\mfam{A}}{\mtm{A}}$ to $\mtm{\mfam{\mfam{A}}}$ over
$\tfwk{A}^0$ in context $\ctxext{{\Gamma}{A}}{\mfam{A}}$:}
& \inference
  {\jfam{\Gamma}{A}}
  {\jfhom
    {{{\Gamma}{A}}{\mfam{A}}}
    {\ctxwk{\mfam{A}}{\mfam{A}}}
    {\mfam{\mfam{A}}}
    {\tfwk{A}^0}
    {\ctxwk{\mfam{A}}{\mtm{A}}}
    {\mtm{\mfam{\mfam{A}}}}
    {\tfwk{A}^1}}
  \tag{weakening}
\intertext{The action of substitution by a term on families
is going to be a morphism from $\ctxwk{\mtm{A}}{\mfam{\mfam{A}}}$ to 
$\ctxwk{\mtm{A}}{{\mfam{A}}{\mfam{A}}}$
in context $\ctxext{{{\Gamma}{A}}{\mfam{A}}}{\mtm{A}}$:}
& \inference
  {\jfam{\Gamma}{A}}
  {\jhom
  {{{{\Gamma}{A}}{\mfam{A}}}{\mtm{A}}}
  {\ctxwk{\mtm{A}}{\mfam{\mfam{A}}}}
  {\ctxwk{\mtm{A}}{{\mfam{A}}{\mfam{A}}}}
  {\tfsubst{A}^0}}
\intertext{Likewise, the action of substitution by a term on terms of those
families is going to be a morphism from $\ctxwk{\mtm{A}}{\mtm{\mfam{\mfam{A}}}}$
to $\ctxwk{\mtm{A}}{{\mfam{A}}{\mtm{A}}}$ over $\tfsubst{A}^0$ in context
$\ctxext{{{\Gamma}{A}}{\mfam{A}}}{\mtm{A}}$:}
& \inference
  {\jfam{\Gamma}{A}}
  {\jfhom
  {{{{\Gamma}{A}}{\mfam{A}}}{\mtm{A}}}
  {\ctxwk{\mtm{A}}{\mfam{\mfam{A}}}}
  {\ctxwk{\mtm{A}}{{\mfam{A}}{\mfam{A}}}}
  {\tfsubst{A}^0}
  {\ctxwk{\mtm{A}}{\mtm{\mfam{\mfam{A}}}}}
  {\ctxwk{\mtm{A}}{{\mfam{A}}{\mtm{A}}}}
  {\tfsubst{A}^1}}
\intertext{The identity functions are coded by}
& \inference
  {\jfam{\Gamma}{A}}
  {\jterm
   {{{\Gamma}{A}}{\mfam{A}}}
   {\subst{{\idfunc[\mfam{A}]}{\tfwk{A}^0}}{\mtm{\mfam{\mfam{A}}}}}
   {\tfid{A}}}
\end{align*}
We will call the terms that we introduced here the model constructors.

\subsection{The compatibility rules}
We need to do several things in this section:
\begin{enumerate}
\item postulate that ordinary extension, weakening, substitution and identity
functions are compatible with the model constructors.
\item postulate that each of the model constructors is compatible with ordinary
extension, weakening, substitution and identity functions.
\item postulate that the model constructors are compatible with each other,
so that they come to model ordinary extension, weakening, substitution and
identity functions respectively.
\end{enumerate}
Note that the rules for compatibility with extension are going to explain which
model $\ctxext{A}{P}$ is by telling what the families, the terms, extension,
weakening, substitution and identity functions are. Likewise the rules for
compatibility with weakening and substitution do this for their respective
cases.

\subsubsection{Families of extensions}

\subsubsection{Terms of families of extensions}

\subsubsection{Terms of extensions}

\subsubsection{Families over the empty context}
\begin{equation*}
\inference
  {\jfam{\Gamma}{A}}
  {\jfameq{\Gamma}{\subst{\tfemp{A}}{\mfam{A}}}{A}}
\end{equation*}

\subsubsection{Terms of the empty family}
\begin{equation*}
\inference
  {\jfam{\Gamma}{A}}
  {\jfameq{{\Gamma}{A}}{\subst{\tfemp{\mfam{A}}}{\mtm{A}}}{\emptyf[A]}}
\end{equation*}

\section{Pretty type theory}
In this section we will do three things. First we explain basic type theory with
variable names, which we simply call \emph{Pretty Type Theory}. 
Then we will show how every formula in basic type theory without
variable names (shall we call it \emph{Structural Type Theory}?) 
can be interpreted in pretty type theory. Finally, we will show how every formula
in pretty type theory can be interpreted in structural type theory.

Pretty type theory is pretty as the ugly little duckling that grew up. At first, it is actually uglier than
structural type theory because one has to keep track of a variable nobody is
actually interested in. The situation improves when we introduce the operators
of extension and weakening, because from that point onwards one can draw notational
advantages from having the variable around.

Pretty type theory should be such that if you write down two contexts, families
or terms in exactly the same way, then they are the same, and there should be
notational shortcuts for extension, weakening and substitution which make
this interesting.

\subsection{The basic judgments}
The basic judgments of pretty type theory are the same as for structural type
theory. There are judgments for: ``$\Gamma$ is a context'',
``$A(i)$ over $i:\Gamma$ is a family over $\Gamma$'', ``$A(i)$ over $i:\Gamma$ 
is a type in context $\Gamma$''
and ``$x(i)$ is a term of $A(i)$ above $i:\Gamma$''. The other four
judgments are for judgmental equality. 

\begin{align*}
\jvctx*{\Gamma} & \jvctxeq*{\Gamma}{\Gamma'}\\
\jvfam*{i}{\Gamma}{A} & \jvfameq*{i}{\Gamma}{A}{B}\\
\jvtype*{i}{\Gamma}{A} & \jvtypeeq*{i}{\Gamma}{A}{B}\\
\jvterm*{i}{\Gamma}{A}{x} & \jvtermeq*{i}{\Gamma}{A}{x}{y}.
\end{align*}

We have the following basic inference rules that relate types and families:

\begin{small}
\begin{align*}
& \inference
  {\jvtype{i}{\Gamma}{A}}
  {\jvfam{i}{\Gamma}{A}}
& & \inference
    {\jvtypeeq{i}{\Gamma}{A}{B}}
    {\jvfameq{i}{\Gamma}{A}{B}}\\
& \inference
  {\jvtype{i}{\Gamma}{A}
   \jvfameq{i}{\Gamma}{A}{B}}
  {\jvtype{i}{\Gamma}{B}}
& & \inference
    {\jvtype{i}{\Gamma}{A}
     \jvfameq{i}{\Gamma}{A}{B}}
    {\jvtypeeq{i}{\Gamma}{A}{B}}
\end{align*}
\end{small}

\subsection{The basic rules for judgmental equality}
The rules for judgmental equality establish that it is an equivalence relation.
\bgroup\small
\begin{align*}
& \inference
  {\jvctx{\Gamma}}
  {\jvctxeq{\Gamma}{\Gamma}} 
& & \inference
    {\jvctxeq{\Gamma}{\Delta}}
    {\jvctxeq{\Delta}{\Gamma}} 
& & \inference
    {\jvctxeq{\Gamma}{\Delta}
     \jvctxeq{\Delta}{\greek{E}}}
    {\jvctxeq{\Gamma}{\greek{E}}}\\
& \inference
  {\jvfam{i}{\Gamma}{A}}
  {\jvfameq{i}{\Gamma}{A}{A}} 
& & \inference
    {\jvfameq{i}{\Gamma}{A}{B}}
    {\jvfameq{i}{\Gamma}{B}{A}}
& & \inference
    {\jvfameq{i}{\Gamma}{A}{B}
     \jvfameq{i}{\Gamma}{B}{C}}
    {\jvfameq{i}{\Gamma}{A}{C}}\\
& \inference
  {\jvterm{i}{\Gamma}{A}{x}}
  {\jvtermeq{i}{\Gamma}{A}{x}{x}}
& & \inference
    {\jvtermeq{i}{\Gamma}{A}{x}{y}}
    {\jvtermeq{i}{\Gamma}{A}{y}{x}}
& & \inference
    {\jvtermeq{i}{\Gamma}{A}{x}{y}
     \jvtermeq{i}{\Gamma}{A}{y}{z}}
    {\jvtermeq{i}{\Gamma}{A}{x}{z}}
\end{align*}
\egroup

The following convertibility rules are responsible for the strictness
of judgmental equality, which sets it apart from equivalences or identifications:

\begin{align*}
& \inference
  {\jvctxeq{\Gamma}{\Delta}
   \jvfam{i}{\Gamma}{A}}
  {\jvfam{i}{\Delta}{A}}
& & \inference
    {\jvctxeq{\Gamma}{\Delta}
     \jvfameq{i}{\Gamma}{A}{B}}
    {\jvfameq{i}{\Delta}{A}{B}}\\
& \inference
  {\jvctxeq{\Gamma}{\Delta}
   \jvterm{i}{\Gamma}{A}{x}}
  {\jvterm{i}{\Delta}{A}{x}}
& & \inference
    {\jvctxeq{\Gamma}{\Delta}
     \jvtermeq{i}{\Gamma}{A}{x}{y}}
    {\jvtermeq{i}{\Delta}{A}{x}{y}}\\
& \inference
  {\jvfameq{i}{\Gamma}{A}{B}
   \jvterm{i}{\Gamma}{A}{x}}
  {\jvterm{i}{\Gamma}{B}{x}}
& & \inference
    {\jvfameq{i}{\Gamma}{A}{B}
     \jvtermeq{i}{\Gamma}{A}{x}{y}}
    {\jvtermeq{i}{\Gamma}{B}{x}{y}}
\end{align*}

\subsection{The empty context}
The empty context looks a bit strange when we explicitly denote the terms. But
we will not do so anymore after this subsection.

\begin{align}
& \inference
  {}
  {\jctx{\emptyc}}\\
& \inference
  {\jctx{\Gamma}}
  {\jvfam{i}{\Gamma}{\emptyf[\Gamma]}}\\
& \inference
  {\jctx{\Gamma}}
  {\jvterm{i}{\Gamma}{\emptyf[\Gamma]}{\emptytm[\Gamma]}}\\
& \inference
  {\jvterm{i}{\Gamma}{\emptyf[\Gamma]}{x}}
  {\jvtermeq{i}{\Gamma}{\emptyf[\Gamma]}{x}{\emptytm[\Gamma]}}
\end{align}

Moreover, if $\Gamma$ is a context family over the
empty context, then $\Gamma$ is a context and every context is a context
family over the empty context. Note that this allows us to speak
of terms of contexts too.

\begin{align}
& \inference
  {\jctx{\Gamma}}
  {\jvfam{\nameless}{\emptyc}{\Gamma}} 
& & \inference
    {\jvfam{\nameless}{\emptyc}{\Gamma}}
    {\jctx{\Gamma}}\\
& \inference
  {\jctxeq{\Gamma}{\Delta}}
  {\jvfameq{\nameless}{\emptyc}{\Gamma}{\Delta}}
& & \inference
    {\jvfameq{\nameless}{\emptyc}{\Gamma}{\Delta}}
    {\jctxeq{\Gamma}{\Delta}}
\end{align}

\subsubsection{The empty context is compatible with itslef}
The empty context $\emptyc$ may be considered as a family of contexts over the empty
context. When we do this, we get $\emptyf[\emptyc]$.
\begin{equation}
\inference
  {}
  {\jvfameq{\nameless}{\emptyc}{\emptyc}{\emptyf[\emptyc]}}
\end{equation}
In the future, we shall denote $\emptyf[\Gamma]$ by $\emptyf$. The above rule
guarantees that this will not cause confusion. Likewise, we shall denote
$\emptytm[\Gamma]$ by $\emptytm$.

\subsection{Extension}
We introduce extension which not only extends a context $\Gamma$ and a family
$A$ over it to a context $\ctxext{\Gamma}{A}$, but which also extends a family $A$
in context $\Gamma$ and a family $P$ over it to a family $\ctxext{A}{P}$ over context
$\Gamma$. We do this to ensure that all of type theory can be done in a context.
For instance, we could say (1) that a context in context $\Gamma$ is the same thing
as a family over $\Gamma$; (2) When $A$ is a context in this sense, a family over
$A$ is the same thing as a family $P$ over $\ctxext{\Gamma}{A}$ and 
(3) when $P$ is a family over $A$ in this sense, a term of $P$ keeps its original meaning.

\begin{align}
& \inference
  {\jvfam{i}{\Gamma}{A}}
  {\jvfamcombi{{i}{x}}{{\Gamma}{A}}{P}}
& & \inference
    {\jctxeq{\Gamma}{\Delta}
     \jfameq{\Gamma}{A}{B}}
    {\jctxeq{\ctxext{\Gamma}{A}}{\ctxext{\Delta}{B}}}\\
& \inference
  {\jfam{{\Gamma}{A}}{P}}
  {\jfam{\Gamma}{\ctxext{A}{P}}}
& & \inference
    {\jfameq{\Gamma}{A}{B}
     \jfameq{{\Gamma}{A}}{P}{Q}}
    {\jfameq{\Gamma}{\ctxext{A}{P}}{\ctxext{B}{Q}}}
\end{align}

\subsubsection{Extension is compatible with the empty context}
The following rule asserts that extension by $\emptyc$ leaves the contexts unchanged.
\begin{align}
& \inference
  {\jctx{\Gamma}}
  {\jctxeq{\ctxext{\emptyc}{\Gamma}}{\Gamma}}\\
& \inference
  {\jctx{\Gamma}}
  {\jctxeq{\ctxext{\Gamma}{\emptyf}}{\Gamma}}\\
& \inference
  {\jfam{\Gamma}{A}}
  {\jfameq{\Gamma}{\ctxext{\emptyf}{A}}{A}}
\end{align}

\subsubsection{Extension is compatible with itself}\label{comp-ee}
The inference rules asserting that extension is compatible with itself assert
that contexts are unstructured lists of type declarations. This rule is
unavoidable if we want that for a family $A$ in context $\Gamma$, a family over
$A$ is the same thing as a family over $\ctxext{\Gamma}{A}$. 

\begin{align}
& \inference
  {\jfam{\Gamma}{A}
   \jfam{{\Gamma}{A}}{P}}
  {\jctxeq{\ctxext{{\Gamma}{A}}{P}}{\ctxext{\Gamma}{{A}{P}}}}\\
& \inference
  {\jfam{{\Gamma}{A}}{P}
   \jfam{{{\Gamma}{A}}{P}}{Q}}
  {\jfameq{\Gamma}{\ctxext{{A}{P}}{Q}}{\ctxext{A}{{P}{Q}}}}
\end{align}


\section{Chalmers type theory}
In this section we will present the type theory that Coquand and Dybjer are using.
It is a weak type theory (I think), with not so many operations and judgmental equalities.
We will show how every formula of Chalmers type theory can be interpreted in
structural type theory.
