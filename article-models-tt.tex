\section{A variable-free presentation of type theory}
We have seen in the example of $\UU$ that we will not need to refer to variables
when we are manipulating contexts. In fact, it is more natural not to.
This suggests that there is a presentation
of type theory where contexts do not mention variables. The first small gain is that such a presentation
would not be burdened with comments about variables being bounded or not, or fresh or
not occuring at all.

\subsection{The basic judgments}
The judgments we wish to make are not the standad six judgments for ``$\Gamma$ is a context'', ``$A$ is
a type in context $\Gamma$'' and ``$x$ is a term of $A$ in context $\Gamma$'' and the three equality
judgments coming with them. We have those, but we also add a judgment for ``$i$ is a term of context $\Gamma$''
and ``$i$ and $j$ are equal terms of context $\Gamma$''. Thus, we have eight basic judgments we can make:

\begin{align*}
\jctx*{\Gamma} & \jctxeq*{\Gamma}{\Gamma'}\\
\jtermc*{\Gamma}{i} & \jtermceq*{\Gamma}{i}{j}\\
\jtype*{\Gamma}{A} & \jtypeeq*{\Gamma}{A}{B}\\
\jtermt*{\Gamma}{A}{x} & \jtermteq*{\Gamma}{A}{x}{y}.
\end{align*}

In contrast with standard practise, we don't assume that there is an empty context. The idea is that
if we interpret only the syntax of contexts, types and terms with extension, weakening and
substitution, we do not have a full model of type theory but we still have a 
structure resembling a category and categories need not have terminal objects. Therefore,
the rules that follow will describe only how to manipulate contexts, types and terms.
Of most of the inferences that we give there are two versions: one introducing an operation, the other asserting that the operation in question preserves judgmental equality. 

\subsection{The basic rules for judgmental equality}
The rules for judgmental equality establish that it is an equivalence relation
in all three cases (contexts, types and terms).
%\begingroup
%\renewcommand*{\arraystretch}{3}
%\begin{equation*}
%\begin{array}{ccc}
\begin{infarray}{ccc}
\inference{\jctx{\Gamma}}{\jctxeq{\Gamma}{\Gamma}} & \inference{\jctxeq{\Gamma}{\Delta}}{\jctxeq{\Delta}{\Gamma}} & \inference{\jctxeq{\Gamma}{\Delta}\qquad\jctxeq{\Delta}{\Theta}}{\jctxeq{\Gamma}{\Theta}}\\
\inference{\jtype{\Gamma}{A}}{\jtypeeq{\Gamma}{A}{A}} &
\inference{\jtypeeq{\Gamma}{A}{B}}{\jtypeeq{\Gamma}{B}{A}} & 
\inference{\jtypeeq{\Gamma}{A}{B}\qquad\jtypeeq{\Gamma}{B}{C}}{\jtypeeq{\Gamma}{A}{C}}\\
\inference{\jtermt{\Gamma}{A}{x}}{\jtermteq{\Gamma}{A}{x}{x}} & 
\inference{\jtermteq{\Gamma}{A}{x}{y}}{\jtermteq{\Gamma}{A}{y}{x}} &
\inference{\jtermteq{\Gamma}{A}{x}{y}\qquad\jtermteq{\Gamma}{A}{y}{z}}{\jtermteq{\Gamma}{A}{x}{z}}
\end{infarray}
%\end{array}
%\end{equation*}
\begin{infarray}{cc}
\inference{\jctxeq{\Gamma}{\Delta}\qquad\jtype{\Gamma}{A}}{\jtype{\Delta}{A}}
& \inference{\jctxeq{\Gamma}{\Delta}\qquad\jtypeeq{\Gamma}{A}{B}}{\jtypeeq{\Delta}{A}{B}}\\
\inference{\jctxeq{\Gamma}{\Delta}\qquad\jtermt{\Gamma}{A}{x}}{\jtermt{\Delta}{A}{x}}
& \inference{\jctxeq{\Gamma}{\Delta}\qquad\jtermteq{\Gamma}{A}{x}{y}}{\jtermteq{\Delta}{A}{x}{y}}\\
\inference{\jtypeeq{\Gamma}{A}{B}\qquad \jtermt{\Gamma}{A}{x}}{\jtermt{\Gamma}{B}{x}}
& \inference{\jtypeeq{\Gamma}{A}{B}\qquad\jtermteq{\Gamma}{A}{x}{y}}{\jtermteq{\Gamma}{B}{x}{y}}
\end{infarray}

\subsection{Extension}
We introduce extension which not only extends a context $\Gamma$ and a type
$A$ over it to a context $\ctxext{\Gamma}{A}$, but which also extends a type $A$
in context $\Gamma$ and a family $P$ over it to a type $\ctxext{A}{P}$ in context
$\Gamma$. We do this to ensure that all of type theory can be done in a context.
For instance, we could say (1) that a context in context $\Gamma$ is the same thing
as a type in context $\Gamma$; (2) When $A$ is a context in this sense, a type in
context $A$ is the same thing as a family $P$ over $A$ and (3) when $P$ is a type
in context $A$ in this sense, a term of $P$ keeps its original meaning.

Note that by introducing extension on the level of types and families,
we introduce $\Sigma$-types at a very early stage.
However, we need substitution to make this precise.
\begin{infarray}{cc}
\inference{\jtype{\Gamma}{A}}{\jctx{\ctxext{\Gamma}{A}}}
& \inference{\jctxeq{\Gamma}{\Delta}\qquad\jtypeeq{\Gamma}{A}{B}}{\jctxeq{\ctxext{\Gamma}{A}}{\ctxext{\Delta}{B}}}\\
\inference{\jtype{\ctxext{\Gamma}{A}}{P}}{\jtype{\Gamma}{\ctxext{A}{P}}}
& \inference{\jtypeeq{\Gamma}{A}{B}\qquad\jtypeeq{\ctxext{\Gamma}{A}}{P}{Q}}{\jtypeeq{\Gamma}{\ctxext{A}{P}}{\ctxext{B}{Q}}}
\end{infarray}

\subsection{Weakening}
We first define weakening by a context $\Gamma$. Since weakening by $\Gamma$
should in principle be a functor, it acts on contexts, types and terms alike.
Note that it is because of the weakening $\ctxwk{\Gamma}{\Delta}$ of a context
$\Delta$ by $\Gamma$ that we can speak of context morphisms from $\Gamma$ to $\Delta$: they are the terms
of $\ctxwk{\Gamma}{\Delta}$.
\begin{infarray}{cc}
\inference{\jctx{\Gamma}\qquad\jctx{\Delta}}{\jtype{\Gamma}{\ctxwk{\Gamma}{\Delta}}} 
& \inference{\jctxeq{\Gamma}{\Gamma'}\qquad\jctxeq{\Delta}{\Delta'}}{\jtypeeq{\Gamma}{\ctxwk{\Gamma}{\Delta}}{\ctxwk{\Gamma'}{\Delta'}}}\\
\inference{\jctx{\Gamma}\qquad\jtype{\Delta}{B}}{\jtype{\ctxext{\Gamma}{\ctxwk{\Gamma}{\Delta}}}{\ctxwk{\Gamma}{B}}} & \inference{\jctxeq{\Gamma}{\Gamma'}\qquad\jtypeeq{\Delta}{B}{B'}}{\jtypeeq{\ctxext{\Gamma}{\ctxwk{\Gamma}{\Delta}}}{\ctxwk{\Gamma}{B}}{\ctxwk{\Gamma'}{B'}}}\\
\inference{\jctx{\Gamma}\qquad\jtermt{\Delta}{B}{y}}{\jtermt{\ctxext{\Gamma}{\ctxwk{\Gamma}{\Delta}}}{\ctxwk{\Gamma}{B}}{\ctxwk{\Gamma}{y}}} 
& \inference{\jctxeq{\Gamma}{\Gamma'}\qquad\jtermteq{\Delta}{B}{y}{y'}}{\jtermteq{\ctxext{\Gamma}{\ctxwk{\Gamma}{\Delta}}}{\ctxwk{\Gamma}{B}}{\ctxwk{\Gamma}{y}}{\ctxwk{\Gamma'}{y'}}} 
\end{infarray}

A weakening operation is also defined for types. When $A$ and $B$ are both types
 in context $\Gamma$, the weakened type $\ctxwk{A}{B}$ in context $\ctxext{\Gamma}{A}$
 is the family which `doesn't really depend on $A$'. The terms of $\ctxwk{A}{B}$
 are the functions from $A$ to $B$. Likewise, the terms $\ctxwk{A}{y}$ are the
 constant maps at $y$ from $A$ to $B$, for $y:B$.
\begin{infarray}{cc}
\inference{\jtype{\Gamma}{A}\qquad\jtype{\Gamma}{B}}{\jtype{\ctxext{\Gamma}{A}}{\ctxwk{A}{B}}}
& \inference{\jtypeeq{\Gamma}{A}{A'}\qquad\jtypeeq{\Gamma}{B}{B'}}{\jtypeeq{\ctxext{\Gamma}{A}}{\ctxwk{A}{B}}{\ctxwk{A'}{B'}}}\\
\inference{\jtype{\Gamma}{A}\qquad\jtype{\ctxext{\Gamma}{B}}{Q}}
{\jtype{\ctxext{{\Gamma}{A}}{\ctxwk{A}{B}}}{\ctxwk{A}{Q}}}
& \inference{\jtypeeq{\Gamma}{A}{A'}\qquad\jtypeeq{\ctxext{\Gamma}{B}}{Q}{Q'}}
{\jtypeeq{\ctxext{{\Gamma}{A}}{\ctxwk{A}{B}}}{\ctxwk{A}{Q}}{\ctxwk{A'}{Q'}}}\\
\inference{\jtype{\Gamma}{A}\qquad\jtermt{\Gamma}{B}{y}}{\jtermt{\ctxext{\Gamma}{A}}{\ctxwk{A}{B}}{\ctxwk{A}{y}}}
& \inference{\jtypeeq{\Gamma}{A}{A'}\qquad\jtermteq{\Gamma}{B}{y}{y'}}{\jtermteq{\ctxext{\Gamma}{A}}{\ctxwk{A}{B}}{\ctxwk{A}{y}}{\ctxwk{A'}{y'}}}
\end{infarray}

\subsubsection{Weakening is compatible with itself}
To conclude the rules for weakening, we state judgmental equality rules expressing
that weakening is compatible with itself. These rules state that the following
diagram commutes given any two contexts $\Gamma$ and $\Delta$:
\begin{equation*}
\begin{tikzcd}[column sep=huge]
\jctx{\blank} \ar{r}{\greek{E}\mapsto\ctxwk{\Delta}{E}} \ar{d}[swap]{\greek{E}\mapsto\ctxwk{\Gamma}{E}} & \jtype{\Delta}{\blank} \ar{d}{B\mapsto\ctxwk{\Gamma}{B}}\\
\jtype{\Gamma}{\blank} \ar{r}[swap]{A\mapsto\ctxwk{{\Gamma}{\Delta}}{A}} & \jtype{\ctxext{\Gamma}{\ctxwk{\Gamma}{\Delta}}}{\blank}
\end{tikzcd}
\end{equation*}
There is also a version of this diagram in which all happens in a context. Thus,
we get two sets of inference rules. For weakening by contexts we get:

\begin{infarray}{c}
\inference{\jctx{\Gamma}\qquad\jctx{\Delta}\qquad\jctx{\greek{E}}}
          {\jtypeeq{\ctxext{\Gamma}{\ctxwk{\Gamma}{\Delta}}}{\ctxwk{\Gamma}{{\Delta}{\greek{E}}}}
            {\ctxwk{{\Gamma}{\Delta}}{{\Gamma}{\greek{E}}}}}\\
\inference{\jctx{\Gamma}\qquad\jctx{\Delta}\qquad\jtype{\greek{E}}{C}}
          {\jtypeeq{\ctxext{{\Gamma}{\ctxwk{\Gamma}{\Delta}}}{\ctxwk{{\Gamma}{\Delta}}{{\Gamma}{\greek{E}}}}}
            {\ctxwk{\Gamma}{{\Delta}{C}}}
            {\ctxwk{{\Gamma}{\Delta}}{{\Gamma}{C}}}}\\
\inference{\jctx{\Gamma}\qquad\jctx{\Delta}\qquad\jtermt{\greek{E}}{C}{t}}
          {\jtermteq{\ctxext{{\Gamma}{\ctxwk{\Gamma}{\Delta}}}{\ctxwk{{\Gamma}{\Delta}}{{\Gamma}{\greek{E}}}}}
            {\ctxwk{{\Gamma}{\Delta}}{{\Gamma}{C}}}{\ctxwk{\Gamma}{{\Delta}{t}}}{\ctxwk{{\Gamma}{\Delta}}{{\Gamma}{t}}}}
\end{infarray}

For weakening by types we get:

\begin{infarray}{c}
\inference{\jtype{\Gamma}{A}\qquad\jtype{\Gamma}{B}\qquad\jtype{\Gamma}{C}}
          {\jtypeeq{\ctxext{{\Gamma}{A}}{\ctxwk{A}{B}}}{\ctxwk{A}{{B}{C}}}
            {\ctxwk{{A}{B}}{{A}{C}}}}\\
\inference{\jtype{\Gamma}{A}\qquad\jtype{\Gamma}{B}\qquad\jtype{\ctxext{\Gamma}{C}}{R}}
          {\jtypeeq{\ctxext{{{\Gamma}{A}}{\ctxwk{A}{B}}}{\ctxwk{{A}{B}}{{A}{C}}}}
            {\ctxwk{A}{{B}{R}}}
            {\ctxwk{{A}{B}}{{A}{R}}}}\\
\inference{\jtype{\Gamma}{A}\qquad\jtype{\Gamma}{B}\qquad\jtermt{\Gamma}{R}{t}}
          {\jtermteq{\ctxext{{{\Gamma}{A}}{\ctxwk{A}{B}}}{\ctxwk{{A}{B}}{{A}{C}}}}
            {\ctxwk{{A}{B}}{{A}{R}}}{\ctxwk{A}{{B}{t}}}{\ctxwk{{A}{B}}{{A}{t}}}}
\end{infarray}

\subsection{Substitution}
Given a family $P$ over $A$ and a term $x$ of $A$, substitution gives a way to
consider the fiber $\subst{x}{P}$ of $P$ at $x$. Also, we get a way to evaluate
terms $f$ of $P$ at $x$. This will give us ways to compose functions too. In
this section, we shall first introduce the operations `substitution of a term $x$'
for types, terms and families. Then we shall explain how substitution interacts
with itself, extension and weakening.

In the rules introducing the various substitutions we assume $\jtermt{\Gamma}{A}{x}$;
in the rules introducing the definitional equalities we assume $\jtermteq{\Gamma}{A}{x}{x'}$.

\begin{infarray}{cc}
\inference{\jtype{\ctxext{\Gamma}{A}}{P}}{\jtype{\Gamma}{\subst{x}{P}}}
& \inference{\jtypeeq{\ctxext{\Gamma}{A}}{P}{P'}}{\jtypeeq{\Gamma}{\subst{x}{P}}{\subst{x'}{P'}}}\\
\inference{\jtype{\ctxext{{\Gamma}{A}}{P}}{Q}}{\jtype{\ctxext{\Gamma}{\subst{x}{P}}}{\subst{x}{Q}}}
& \inference{\jtypeeq{\ctxext{{\Gamma}{A}}{P}}{Q}{Q'}}{\jtypeeq{\ctxext{\Gamma}{\subst{x}{P}}}{\subst{x}{Q}}{\subst{x'}{Q'}}}\\
\inference{\jtermt{\ctxext{\Gamma}{A}}{P}{f}}{\jtermt{\Gamma}{\subst{x}{P}}{\subst{x}{f}}}
& \inference{\jtermteq{\ctxext{\Gamma}{A}}{P}{f}{f'}}{\jtermteq{\Gamma}{\subst{x}{P}}{\subst{x}{f}}{\subst{x'}{f'}}}\\
\inference{\jtermt{\ctxext{{\Gamma}{A}}{P}}{Q}{g}}{\jtermt{\ctxext{\Gamma}{\subst{x}{P}}}{\subst{x}{Q}}{\subst{x}{g}}}
& \inference{\jtermteq{\ctxext{{\Gamma}{A}}{P}}{Q}{g}{g'}}{\jtermteq{\ctxext{\Gamma}{\subst{x}{P}}}{\subst{x}{Q}}{\subst{x}{g}}{\subst{x'}{g'}}}
\end{infarray}

\subsubsection{Substitution is compatible with substitution}

We require that substitution is compatible with itself, which is roughly the
assertion that substitution is associative. However, we cannot just state that
$\subst{x}{{f}{g}}\jdeq\subst{{x}{f}}{g}$ since the expression $\subst{{x}{f}}{g}$
is not well-formed. The term $\subst{x}{f}$ can be substituted in (terms of) families over
$\subst{x}{P}$; the term $\subst{x}{g}$ is such. Therefore, associativity of
substitution takes the form $\subst{x}{{f}{g}}\jdeq\subst{{x}{f}}{{x}{g}}$.

In the following inference rules we assume
$\jtermt{\Gamma}{A}{x}$ and $\jtermt{\ctxext{\Gamma}{A}}{P}{f}$.

\begin{infarray}{c}
\inference{\jtype{\ctxext{{\Gamma}{A}}{P}}{Q}}
{\jtypeeq{\Gamma}{\subst{x}{{f}{Q}}}{\subst{{x}{f}}{{x}{Q}}}}\\
\inference{\jtype{\ctxext{{{\Gamma}{A}}{P}}{Q}}{R}}
{\jtypeeq{\ctxext{\Gamma}{\subst{x}{{f}{Q}}}}{\subst{x}{{f}{R}}}{\subst{{x}{f}}{{x}{R}}}}\\
\inference{\jtermt{\ctxext{{\Gamma}{A}}{P}}{Q}{g}}
{\jtermteq{\Gamma}{\subst{x}{{f}{Q}}}{\subst{x}{{f}{g}}}{\subst{{x}{f}}{{x}{g}}}}\\
\inference{\jtermt{\ctxext{{{\Gamma}{A}}{P}}{Q}}{R}{h}}
{\jtermteq{\ctxext{\Gamma}{\subst{x}{{f}{Q}}}}{\subst{x}{{f}{R}}}{\subst{x}{{f}{h}}}{\subst{{x}{f}}{{x}{h}}}}
\end{infarray}

\subsubsection{Substitution is compatible with extension}
Suppose $\jtermt{\Gamma}{A}{x}$ in all of the following inference rules.
\begin{infarray}{c}
\inference{\jtype{\ctxext{{\Gamma}{A}}{P}}{Q}}{\jtypeeq{\Gamma}{\subst{x}{\ctxext{P}{Q}}}{\ctxext{\subst{x}{P}}{\subst{x}{Q}}}}
\end{infarray}

\subsubsection{Substitution is compatible with weakening}
The judgmental equalities we're about to describe assert that weakening followed
by substitution leaves everything untouched. Thus, we get that each fiber
$\subst{x}{\ctxwk{A}{B}}$ is just $B$, that $\ctxwk{A}{y}$ is the constant function
mapping everything to $y:B$ and similar properties for families and terms thereof.

\begin{infarray}{c}
\inference{\jtype{\Gamma}{A}\qquad\jtype{\Gamma}{B}\qquad\jtermt{\Gamma}{A}{x}}{\jtypeeq{\Gamma}{\subst{x}{\ctxwk{A}{B}}}{B}}\\
\inference{\jtype{\Gamma}{A}\qquad\jtype{\ctxext{\Gamma}{B}}{Q}\qquad\jtermt{\Gamma}{A}{x}}{\jtypeeq{\ctxext{\Gamma}{B}}{\subst{x}{\ctxwk{A}{Q}}}{Q}}\\
\inference{\jtermt{\Gamma}{A}{x}\qquad\jtermt{\Gamma}{B}{y}}{\jtermteq{\Gamma}{B}{\subst{x}{\ctxwk{A}{y}}}{y}}\\
\inference{\jtermt{\Gamma}{A}{x}\qquad\jtermt{\ctxext{\Gamma}{B}}{Q}{g}}{\jtermteq{\ctxext{\Gamma}{B}}{Q}{\subst{x}{\ctxwk{A}{g}}}{g}}
\end{infarray}

\subsection{The universal property of extension}

Using weakening and substitution we are able to state the universal property
for extension. It looks a bit more involved, since we cannot directly refer
to the variables in the contexts. On the other hand, we can now plainly see
were there were secretly weakenings going on.

We begin with stating the universal property of the extension $\ctxext{\Gamma}{A}$.
In these rules we assume we have $\jtype{\Gamma}{A}$ in the hypotheses.

\begin{infarray}{c}
\inference{}
{\jtermt{\ctxext{\Gamma}{A}}{\ctxwk{A}{{\Gamma}{\ctxext{\Gamma}{A}}}}{\pair_A}}\\
\inference{
  \jtype{\ctxext{\Gamma}{A}}{P}
  \qquad
  \jtermt{\ctxext{\Gamma}{A}}{\subst{\pair_A}{\ctxwk{A}{{\Gamma}{P}}}}{f}}
  {\jtermt{\ctxext{\Gamma}{A}}{P}{\ind{\tfext_\Gamma(A)}(f)}}\\
\inference{
  \jtype{\ctxext{\Gamma}{A}}{P}
  \qquad
  \jtermt{\ctxext{\Gamma}{A}}{\subst{\pair_A}{\ctxwk{A}{{\Gamma}{P}}}}{f}}
  {\jtermteq{\ctxext{\Gamma}{A}}{\subst{\pair_A}{\ctxwk{A}{{\Gamma}{P}}}}{\subst{\pair_A}{\ctxwk{A}{{\Gamma}{\ind{\tfext_\Gamma(A)}(f)}}}}{f}}
\end{infarray}

Under the hypothesis that $\jtypeeq{\Gamma}{A}{A'}$
we will also have the rules

\begin{infarray}{c}
\inference{}{\jtermteq{\ctxext{\Gamma}{A}}{\ctxwk{A}{{\Gamma}{\ctxext{\Gamma}{A}}}}{\pair_A}{\pair_{A'}}}\\
\inference{\jtype{\ctxext{\Gamma}{A}}{P}\qquad\jtermt{\ctxext{\Gamma}{A}}{\subst{\pair_A}{\ctxwk{A}{{\Gamma}{P}}}}{f}}
{\jtermteq{\ctxext{\Gamma}{A}}{P}{\ind{\tfext_\Gamma(A)}(f)}{\ind{\tfext_\Gamma(A')}(f)}}
\end{infarray}

Note that we don't need the notion of terms for contexts to state the universal
property of context extension (which is a good thing, for we don't assume to have them).

Next, we give the universal property of the extension $\ctxext{A}{P}$ in context
$\Gamma$.
In all of the following inference rules we assume that $\jtype{\ctxext{\Gamma}{A}}{P}$
is among the hypothesis. The induction principle for extension consists of three
inference rules:

\begin{infarray}{c}
\inference{}
{\jtermt{\ctxext{{\Gamma}{A}}{P}}{\ctxwk{P}{{A}{\ctxext{A}{P}}}}{\pair_P}}\\
\inference{
  \jtype{\ctxext{\Gamma}{{A}{P}}}{Q}
  \qquad
  \jtermt{\ctxext{{\Gamma}{A}}{P}}{\subst{\pair_P}{\ctxwk{P}{{A}{Q}}}}{f}}
  {\jtermt{\ctxext{\Gamma}{{A}{P}}}{Q}{\ind{\tfext_A(P)}(f)}}\\
\inference{
  \jtype{\ctxext{\Gamma}{{A}{P}}}{Q}
  \qquad
  \jtermt{\ctxext{{\Gamma}{A}}{P}}{\subst{\pair_P}{\ctxwk{P}{{A}{Q}}}}{f}}
  {\jtermteq{\ctxext{{\Gamma}{A}}{P}}{\subst{\pair_P}{\ctxwk{P}{{A}{Q}}}}{\subst{\pair_P}{\ctxwk{P}{{A}{\ind{\tfext_A(P)}(f)}}}}{f}}
\end{infarray}

As with context extension, we shall require two more inference rules stating that
$\pair_P$ and $\ind{\tfext_A(P)}$ are invariant under judgmental equality.

\subsection{The variable rule -- identity functions}
The variable rule for contexts:
\begin{infarray}{c}
\inference{\jctx{\Gamma}}{\jtermt{\Gamma}{\ctxwk{\Gamma}{\Gamma}}{\idfunc[\Gamma]}}\\
\inference{\jtermt{\Gamma}{A}{x}}{\jtermteq{\Gamma}{A}{\subst{\idfunc[\Gamma]}{\ctxwk{\Gamma}{x}}}{x}}\\
\inference{\jtermt{\Delta}{\ctxwk{\Delta}{\Gamma}}{g}}{\jtermteq{\Delta}{\ctxwk{\Delta}{\Gamma}}{\subst{g}{\ctxwk{\Delta}{\idfunc[\Gamma]}}}{g}}
\end{infarray}


The variable rule for types:
\begin{infarray}{c}
\inference{\jtype{\Gamma}{A}}{\jtermt{\ctxext{\Gamma}{A}}{\ctxwk{A}{A}}{\idfunc[A]}}\\
\inference{\jtermt{\Gamma}{A}{x}}{\jtermteq{\Gamma}{A}{\subst{x}{\idfunc[A]}}{x}}\\
\inference{\jtermt{\ctxext{\Gamma}{A}}{P}{f}}{\jtermteq{\ctxext{\Gamma}{A}}{P}{\subst{\idfunc[A]}{\ctxwk{A}{f}}}{f}}\\
\inference{\jtermt{\ctxext{\Gamma}{B}}{\ctxwk{B}{A}}{g}}{\jtermteq{\ctxext{\Gamma}{B}}{\ctxwk{B}{A}}{\subst{g}{\ctxwk{B}{\idfunc[A]}}}{g}}
\end{infarray}

Using these rules we can derive that there is a judgmental equality $\jtypeeq{\ctxext{\Gamma}{A}}{\subst{x}{{\idfunc[A]}{B}}}
{\subst{x}{{x}{B}}}$.

\subsection{Identity types}
The identity types may now be formulated just as in ordinary Martin\nobreakdash-L\"of
type theory. However, since all model categories have path objects, we will also
introduce identity types for contexts.

We will use the symbol $\reflsym$ slightly different than the book does. For us,
$\refl{A}$ is a term of $\subst{\idfunc[A]}{\idtypevar{A}}$ in context
$\ctxext{\Gamma}{A}$ and we will write $\subst{x}{\refl{A}}$ for the reflexivity
path at $x$ (which would have been denoted by $\refl{x}$ in the book).

\begin{infarray}{c}
\inference{\jctx{\Gamma}}{\jtype{\ctxext{\Gamma}{\ctxwk{\Gamma}{\Gamma}}}{\idtypevar{\Gamma}}}\\
\inference{\jctx{\Gamma}}{\jtermt{\Gamma}{\subst{\idfunc[\Gamma]}{\idtypevar{\Gamma}}}{\refl{\Gamma}}}\\
\inference{\jtype{\ctxext{{\Gamma}{\ctxwk{\Gamma}{\Gamma}}}{\idtypevar{\Gamma}}}{P}
           \qquad
           \jtermt{\Gamma}{\subst{\refl{\Gamma}}{{\idfunc[\Gamma]}{P}}}{d}}
           {\jtermt{\ctxext{{\Gamma}{\ctxwk{\Gamma}{\Gamma}}}{\idtypevar{\Gamma}}}{P}{\tfJ(d)}}\\
\inference{\jtype{\ctxext{{\Gamma}{\ctxwk{\Gamma}{\Gamma}}}{\idtypevar{\Gamma}}}{P}\qquad
\jtermt{\Gamma}{\subst{\refl{\Gamma}}{{\idfunc[\Gamma]}{P}}}{d}}
{\jtermteq
  {\ctxext{{\Gamma}{\ctxwk{\Gamma}{\Gamma}}}{\idtypevar{\Gamma}}}
  {\subst{\refl{\Gamma}}{{\idfunc[\Gamma]}{P}}}
  {\subst{\refl{\Gamma}}{{\idfunc[\Gamma]}{\tfJ(d)}}}
  {d}}
\end{infarray}

\begin{infarray}{c}
\inference{\jtype{\Gamma}{A}}{\jtype{\ctxext{{\Gamma}{A}}{\ctxwk{A}{A}}}{\idtypevar{A}}}\\
\inference{\jtype{\Gamma}{A}}{\jtermt{\ctxext{\Gamma}{A}}{\subst{\idfunc[A]}{\idtypevar{A}}}{\refl{A}}}\\
\inference{\jtype{\ctxext{{{\Gamma}{A}}{\ctxwk{A}{A}}}{\idtypevar{A}}}{P}\qquad
\jtermt{\ctxext{\Gamma}{A}}{\subst{\refl{A}}{{\idfunc[A]}{P}}}{d}}
{\jtermt{\ctxext{{{\Gamma}{A}}{\ctxwk{A}{A}}}{\idtypevar{A}}}{P}{\tfJ(d)}}\\
\inference{\jtype{\ctxext{{{\Gamma}{A}}{\ctxwk{A}{A}}}{\idtypevar{A}}}{P}\qquad
\jtermt{\ctxext{\Gamma}{A}}{\subst{\refl{A}}{{\idfunc[A]}{P}}}{d}}
{\jtermteq
  {\ctxext{{{\Gamma}{A}}{\ctxwk{A}{A}}}{\idtypevar{A}}}
  {\subst{\refl{A}}{{\idfunc[A]}{P}}}
  {\subst{\refl{A}}{{\idfunc[A]}{\tfJ(d)}}}
  {d}}
\end{infarray}

\subsection{Other type constructors}
We will now describe the rules for the type constructors that we don't assume
in our type theory by default. The first of these are the dependent function
types. We also discuss $\tfW$-types, suspensions, truncations. More general higher inductive
types will have to wait until we have introduced models, for the models of the
basic type theory because we will use them as index categories of the diagrams.

\subsubsection{Dependent function types}
\begin{infarray}{cc}
\inference{\jtype{\ctxext{\Gamma}{A}}{P}}{\jtype{\Gamma}{\mprd{A}{P}}}
& \inference{\jtypeeq{\ctxext{\Gamma}{A}}{P}{P'}}{\jtypeeq{\Gamma}{\mprd{A}{P}}{\mprd{A}{P'}}}\\
\inference{\jtermt{\ctxext{\Gamma}{A}}{P}{f}}{\jtermt{\Gamma}{\mprd{A}{P}}{\lambda(f)}}
& \inference{\jtermteq{\ctxext{\Gamma}{A}}{P}{f}{f'}}{\jtermteq{\Gamma}{\mprd{A}{P}}{\lambda(f)}{\lambda(f')}}\\
\inference{\jtermt{\Gamma}{\mprd{A}{P}}{g}}{\jtermt{\ctxext{\Gamma}{A}}{P}{\tfev(g)}}
& \inference{\jtermteq{\Gamma}{\mprd{A}{P}}{g}{g'}}{\jtermteq{\ctxext{\Gamma}{A}}{P}{\tfev(g)}{\tfev(g')}}\\
\inference{\jtermt{\ctxext{\Gamma}{A}}{P}{f}}{\jtermteq{\Gamma}{A}{\tfev(\lambda(f))}{f}}
\end{infarray}

With these rules we will not get the weak $\eta$-rule when identity types are present.
So it might be better to state that $\lambda$ is a trivial cofibration.
