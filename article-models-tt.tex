\section{Variable-free type theory -- the basic theory}
We have seen in the example of $\UU$ that we will not need to refer to variables
when we are manipulating contexts. In fact, it is more natural not to.
This suggests that there is a presentation
of type theory where contexts do not mention variables. The first small gain is that such a presentation
would not be burdened with comments about variables being bounded or not, or fresh or
not occuring at all.

\subsection{The basic judgments}
The judgments we wish to make are not the standad six judgments for ``$\Gamma$ is a context'', ``$A$ is
a type in context $\Gamma$'' and ``$x$ is a term of $A$ in context $\Gamma$'' and the three equality
judgments coming with them. We have those, but we also add a judgment for ``$i$ is a term of context $\Gamma$''
and ``$i$ and $j$ are equal terms of context $\Gamma$''. Thus, we have eight basic judgments we can make:

\begin{align*}
\jctx*{\Gamma} & \jctxeq*{\Gamma}{\Gamma'}\\
\jtermc*{\Gamma}{i} & \jtermceq*{\Gamma}{i}{j}\\
\jtype*{\Gamma}{A} & \jtypeeq*{\Gamma}{A}{B}\\
\jterm*{\Gamma}{A}{x} & \jtermeq*{\Gamma}{A}{x}{y}.
\end{align*}

In contrast with standard practise, we don't assume that there is an empty context. The idea is that
if we interpret only the syntax of contexts, types and terms with extension, weakening and
substitution, we do not have a full model of type theory but we still have a 
structure resembling a category and categories need not have terminal objects. Therefore,
the rules that follow will describe only how to manipulate contexts, types and terms.
Of most of the inferences that we give there are two versions: one introducing an operation, the other asserting that the operation in question preserves judgmental equality. 

\subsection{The basic rules for judgmental equality}
The rules for judgmental equality establish that it is an equivalence relation
in all three cases (contexts, types and terms).
%\begingroup
%\renewcommand*{\arraystretch}{3}
%\begin{equation*}
%\begin{array}{ccc}
\begin{infarray}{ccc}
\inference{\jctx{\Gamma}}{\jctxeq{\Gamma}{\Gamma}} & \inference{\jctxeq{\Gamma}{\Delta}}{\jctxeq{\Delta}{\Gamma}} & \inference{\jctxeq{\Gamma}{\Delta}\qquad\jctxeq{\Delta}{\Theta}}{\jctxeq{\Gamma}{\Theta}}\\
\inference{\jtype{\Gamma}{A}}{\jtypeeq{\Gamma}{A}{A}} &
\inference{\jtypeeq{\Gamma}{A}{B}}{\jtypeeq{\Gamma}{B}{A}} & 
\inference{\jtypeeq{\Gamma}{A}{B}\qquad\jtypeeq{\Gamma}{B}{C}}{\jtypeeq{\Gamma}{A}{C}}\\
\inference{\jterm{\Gamma}{A}{x}}{\jtermeq{\Gamma}{A}{x}{x}} & 
\inference{\jtermeq{\Gamma}{A}{x}{y}}{\jtermeq{\Gamma}{A}{y}{x}} &
\inference{\jtermeq{\Gamma}{A}{x}{y}\qquad\jtermeq{\Gamma}{A}{y}{z}}{\jtermeq{\Gamma}{A}{x}{z}}
\end{infarray}
%\end{array}
%\end{equation*}
\begin{infarray}{cc}
\inference{\jctxeq{\Gamma}{\Delta}\qquad\jtype{\Gamma}{A}}{\jtype{\Delta}{A}}
& \inference{\jctxeq{\Gamma}{\Delta}\qquad\jtypeeq{\Gamma}{A}{B}}{\jtypeeq{\Delta}{A}{B}}\\
\inference{\jctxeq{\Gamma}{\Delta}\qquad\jterm{\Gamma}{A}{x}}{\jterm{\Delta}{A}{x}}
& \inference{\jctxeq{\Gamma}{\Delta}\qquad\jtermeq{\Gamma}{A}{x}{y}}{\jtermeq{\Delta}{A}{x}{y}}\\
\inference{\jtypeeq{\Gamma}{A}{B}\qquad \jterm{\Gamma}{A}{x}}{\jterm{\Gamma}{B}{x}}
& \inference{\jtypeeq{\Gamma}{A}{B}\qquad\jtermeq{\Gamma}{A}{x}{y}}{\jtermeq{\Gamma}{B}{x}{y}}
\end{infarray}

\subsection{Extension}
We introduce extension which not only extends a context $\Gamma$ and a type
$A$ over it to a context $\ctxext{\Gamma}{A}$, but which also extends a type $A$
in context $\Gamma$ and a family $P$ over it to a type $\ctxext{A}{P}$ in context
$\Gamma$. We do this to ensure that all of type theory can be done in a context.
For instance, we could say (1) that a context in context $\Gamma$ is the same thing
as a type in context $\Gamma$; (2) When $A$ is a context in this sense, a type in
context $A$ is the same thing as a family $P$ over $A$ and (3) when $P$ is a type
in context $A$ in this sense, a term of $P$ keeps its original meaning.

Note that by introducing extension on the level of types and families,
we introduce $\Sigma$-types at a very early stage.
However, we need substitution to make this precise.
\begin{infarray}{cc}
\inference{\jtype{\Gamma}{A}}{\jctx{\ctxext{\Gamma}{A}}}
& \inference{\jctxeq{\Gamma}{\Delta}\qquad\jtypeeq{\Gamma}{A}{B}}{\jctxeq{\ctxext{\Gamma}{A}}{\ctxext{\Delta}{B}}}\\
\inference{\jtype{\ctxext{\Gamma}{A}}{P}}{\jtype{\Gamma}{\ctxext{A}{P}}}
& \inference{\jtypeeq{\Gamma}{A}{B}\qquad\jtypeeq{\ctxext{\Gamma}{A}}{P}{Q}}{\jtypeeq{\Gamma}{\ctxext{A}{P}}{\ctxext{B}{Q}}}
\end{infarray}

\subsection{Weakening}
We first define weakening by a context $\Gamma$. Since weakening by $\Gamma$
should in principle be a functor, it acts on contexts, types and terms alike.
Note that it is because of the weakening $\ctxwk{\Gamma}{\Delta}$ of a context
$\Delta$ by $\Gamma$ that we can speak of context morphisms from $\Gamma$ to $\Delta$: they are the terms
of $\ctxwk{\Gamma}{\Delta}$.
\begin{infarray}{cc}
\inference{\jctx{\Gamma}\qquad\jctx{\Delta}}{\jtype{\Gamma}{\ctxwk{\Gamma}{\Delta}}} 
& \inference{\jctxeq{\Gamma}{\Gamma'}\qquad\jctxeq{\Delta}{\Delta'}}{\jtypeeq{\Gamma}{\ctxwk{\Gamma}{\Delta}}{\ctxwk{\Gamma'}{\Delta'}}}\\
\inference{\jctx{\Gamma}\qquad\jtype{\Delta}{B}}{\jtype{\ctxext{\Gamma}{\ctxwk{\Gamma}{\Delta}}}{\ctxwk{\Gamma}{B}}} & \inference{\jctxeq{\Gamma}{\Gamma'}\qquad\jtypeeq{\Delta}{B}{B'}}{\jtypeeq{\ctxext{\Gamma}{\ctxwk{\Gamma}{\Delta}}}{\ctxwk{\Gamma}{B}}{\ctxwk{\Gamma'}{B'}}}\\
\inference{\jctx{\Gamma}\qquad\jterm{\Delta}{B}{y}}{\jterm{\ctxext{\Gamma}{\ctxwk{\Gamma}{\Delta}}}{\ctxwk{\Gamma}{B}}{\ctxwk{\Gamma}{y}}} 
& \inference{\jctxeq{\Gamma}{\Gamma'}\qquad\jtermeq{\Delta}{B}{y}{y'}}{\jtermeq{\ctxext{\Gamma}{\ctxwk{\Gamma}{\Delta}}}{\ctxwk{\Gamma}{B}}{\ctxwk{\Gamma}{y}}{\ctxwk{\Gamma'}{y'}}} 
\end{infarray}

A weakening operation is also defined for types. When $A$ and $B$ are both types
 in context $\Gamma$, the weakened type $\ctxwk{A}{B}$ in context $\ctxext{\Gamma}{A}$
 is the family which `doesn't really depend on $A$'. The terms of $\ctxwk{A}{B}$
 are the functions from $A$ to $B$. Likewise, the terms $\ctxwk{A}{y}$ are the
 constant maps at $y$ from $A$ to $B$, for $y:B$.
\begin{infarray}{cc}
\inference{\jtype{\Gamma}{A}\qquad\jtype{\Gamma}{B}}{\jtype{\ctxext{\Gamma}{A}}{\ctxwk{A}{B}}}
& \inference{\jtypeeq{\Gamma}{A}{A'}\qquad\jtypeeq{\Gamma}{B}{B'}}{\jtypeeq{\ctxext{\Gamma}{A}}{\ctxwk{A}{B}}{\ctxwk{A'}{B'}}}\\
\inference{\jtype{\Gamma}{A}\qquad\jtype{\ctxext{\Gamma}{B}}{Q}}
{\jtype{\ctxext{{\Gamma}{A}}{\ctxwk{A}{B}}}{\ctxwk{A}{Q}}}
& \inference{\jtypeeq{\Gamma}{A}{A'}\qquad\jtypeeq{\ctxext{\Gamma}{B}}{Q}{Q'}}
{\jtypeeq{\ctxext{{\Gamma}{A}}{\ctxwk{A}{B}}}{\ctxwk{A}{Q}}{\ctxwk{A'}{Q'}}}\\
\inference{\jtype{\Gamma}{A}\qquad\jterm{\Gamma}{B}{y}}{\jterm{\ctxext{\Gamma}{A}}{\ctxwk{A}{B}}{\ctxwk{A}{y}}}
& \inference{\jtypeeq{\Gamma}{A}{A'}\qquad\jtermeq{\Gamma}{B}{y}{y'}}{\jtermeq{\ctxext{\Gamma}{A}}{\ctxwk{A}{B}}{\ctxwk{A}{y}}{\ctxwk{A'}{y'}}}
\end{infarray}

\subsubsection{Weakening is compatible with itself}
To conclude the rules for weakening, we state judgmental equality rules expressing
that weakening is compatible with itself. These rules state that the following
diagram commutes given any two contexts $\Gamma$ and $\Delta$:
\begin{equation*}
\begin{tikzcd}[column sep=huge]
\jctx{\blank} \ar{r}{\greek{E}\mapsto\ctxwk{\Delta}{E}} \ar{d}[swap]{\greek{E}\mapsto\ctxwk{\Gamma}{E}} & \jtype{\Delta}{\blank} \ar{d}{B\mapsto\ctxwk{\Gamma}{B}}\\
\jtype{\Gamma}{\blank} \ar{r}[swap]{A\mapsto\ctxwk{{\Gamma}{\Delta}}{A}} & \jtype{\ctxext{\Gamma}{\ctxwk{\Gamma}{\Delta}}}{\blank}
\end{tikzcd}
\end{equation*}
There is also a version of this diagram in which all happens in a context. Thus,
we get two sets of inference rules. For weakening by contexts we get:

\begin{infarray}{c}
\inference{\jctx{\Gamma}\qquad\jctx{\Delta}\qquad\jctx{\greek{E}}}
          {\jtypeeq{\ctxext{\Gamma}{\ctxwk{\Gamma}{\Delta}}}{\ctxwk{\Gamma}{{\Delta}{\greek{E}}}}
            {\ctxwk{{\Gamma}{\Delta}}{{\Gamma}{\greek{E}}}}}\\
\inference{\jctx{\Gamma}\qquad\jctx{\Delta}\qquad\jtype{\greek{E}}{C}}
          {\jtypeeq{\ctxext{{\Gamma}{\ctxwk{\Gamma}{\Delta}}}{\ctxwk{{\Gamma}{\Delta}}{{\Gamma}{\greek{E}}}}}
            {\ctxwk{\Gamma}{{\Delta}{C}}}
            {\ctxwk{{\Gamma}{\Delta}}{{\Gamma}{C}}}}\\
\inference{\jctx{\Gamma}\qquad\jctx{\Delta}\qquad\jterm{\greek{E}}{C}{t}}
          {\jtermeq{\ctxext{{\Gamma}{\ctxwk{\Gamma}{\Delta}}}{\ctxwk{{\Gamma}{\Delta}}{{\Gamma}{\greek{E}}}}}
            {\ctxwk{{\Gamma}{\Delta}}{{\Gamma}{C}}}{\ctxwk{\Gamma}{{\Delta}{t}}}{\ctxwk{{\Gamma}{\Delta}}{{\Gamma}{t}}}}
\end{infarray}

For weakening by types we get:

\begin{infarray}{c}
\inference{\jtype{\Gamma}{A}\qquad\jtype{\Gamma}{B}\qquad\jtype{\Gamma}{C}}
          {\jtypeeq{\ctxext{{\Gamma}{A}}{\ctxwk{A}{B}}}{\ctxwk{A}{{B}{C}}}
            {\ctxwk{{A}{B}}{{A}{C}}}}\\
\inference{\jtype{\Gamma}{A}\qquad\jtype{\Gamma}{B}\qquad\jtype{\ctxext{\Gamma}{C}}{R}}
          {\jtypeeq{\ctxext{{{\Gamma}{A}}{\ctxwk{A}{B}}}{\ctxwk{{A}{B}}{{A}{C}}}}
            {\ctxwk{A}{{B}{R}}}
            {\ctxwk{{A}{B}}{{A}{R}}}}\\
\inference{\jtype{\Gamma}{A}\qquad\jtype{\Gamma}{B}\qquad\jterm{\Gamma}{R}{t}}
          {\jtermeq{\ctxext{{{\Gamma}{A}}{\ctxwk{A}{B}}}{\ctxwk{{A}{B}}{{A}{C}}}}
            {\ctxwk{{A}{B}}{{A}{R}}}{\ctxwk{A}{{B}{t}}}{\ctxwk{{A}{B}}{{A}{t}}}}
\end{infarray}

\subsubsection{A game of trees}
We prove a little meta theorem about the type system we have so far. It is not
really of importance to the development of the theory. 

Let's say
that a binary tree of contexts is a pair $\pair{T,f}$ consisting of a binary 
tree $T$ together with a function $f$ assigning to each leaf a context. The set
of all binary trees of contexts is denoted by $B$. Such
binary trees may be defined inductively: $(\unit,f):B$ for any $f:\unit\to ctx$
and given any $(T_1,f_1)$ and $(T_2,f_2)$ in $B$ we have $(T_1,f_1)*(T_2,f_2)$
in $B$.

We simultaneously define
the following two functions:
\begin{align*}
\trext & : B\to ctx\\
\trwk_0 & : \prd{X,Y:B} typ(\trext(X))\\
\trwk_1 & : \prd{X:B}{\Gamma~ctx} typ(\Gamma)\to typ(\ctxext{\trext(X)}{\trwk_0(X,\Gamma)})
\end{align*}
Both these functions are defined by induction on binary trees. We set
\begin{align*}
\trext((\unit,f)) & \defeq f(\unit)\\
\trext((T_1,f_1)*(T_2,f_2)) & \defeq \ctxext{\trext((T_1,f_1))}
{\trwk_0((T_1,f_1),\trext((T_2,f_2)))}
\end{align*}
and for any context $\Gamma$
\begin{align*}
\trwk_0((\unit,f),(\unit,g)) & \defeq \ctxwk{f(\unit)}{g(\unit)}\\
\trwk_0((T_1,f_1)*(T_2,f_2),\Gamma) 
  & \defeq \trwk_1((T_1,f_1),\trwk_0((T_2,f_2)),\Gamma))
\end{align*}
and for any type $A$ in context $\Gamma$
\begin{align*}
\trwk_1((\unit,f),A) & \defeq \ctxwk{f(\unit)}{A}\\
\trwk_1((T_1,f_1)*(T_2,f_2),A) & \defeq
  \trwk_1((T_1,f_1),\trwk_1((T_2,f_2)),A))
\end{align*}

\begin{lem}
$\trwk_0((T,f),\Gamma)$ is a type in context $\trext((T,f))$ and
$\trwk_1((T,f),A)$ is a type in context $\ctxext{\trext((T,f))}{\trwk_0((T,f),\Gamma)}$
for any type $A$ in context $\Gamma$.
\end{lem}

\begin{proof}
It is immediate that $\trwk_0((\unit,f),\Gamma)$ is a type in context $\trext((\unit,f))$
and that $\trwk_1((\unit,f),A)$ is a type in context $\ctxext{\trext((\unit,f))}
{\trwk_0((\unit,f),\Gamma)}$ for any type $A$ in context $\Gamma$.

Suppose that $(T_1,f_1)$ and $(T_2,f_2)$ are binary trees of contexts such that
$\trwk_0((T_i,f_i),\Gamma)$ is a type in context
$\trext((T_i,f_i))$ for $i\jdeq 1$ and $i\jdeq 2$, and such that
$\trwk_1((T_i,f_1),A)$ is a type in context $\ctxext{\trext((T_i,f_i))}{\trwk_0((T_i,f_i),\Gamma)}$
for $i\jdeq 1$ and $i\jdeq 2$. Then
\begin{equation*}
\trwk_0((T_1,f_1)*(T_2,f_2),\Gamma) 
\jdeq \trwk_1((T_1,f_1),\trwk_0((T_2,f_2),\Gamma))
\end{equation*}
is a type in context
\begin{equation*}
\ctxext{\trext((T_1,f_1))}{\trwk_0((T_1,f_1),\trext((T_2,f_2)))}
\jdeq \trext((T_1,f_1)*(T_2,f_2)).
\end{equation*}
Also,
\begin{equation*}
\trwk_1((T_1,f_1)*(T_2,f_2),A) \jdeq \trwk_1((T_1,f_1),\trwk_1((T_2,f_2),A))
\end{equation*}
is a type in context
\begin{equation*}
\ctxext{\trext((T_1,f_1))}{\trwk_0((T_1,f_1),\ctxext{\trext((T_2,f_2))}{\trwk_0((T_2,f_2),\Gamma))}}
\end{equation*}
\end{proof}

\subsection{Substitution}
Given a family $P$ over $A$ and a term $x$ of $A$, substitution gives a way to
consider the fiber $\subst{x}{P}$ of $P$ at $x$. Also, we get a way to evaluate
terms $f$ of $P$ at $x$. This will give us ways to compose functions too. In
this section, we shall first introduce the operations `substitution of a term $x$'
for types, terms and families. Then we shall explain how substitution interacts
with itself, extension and weakening.

In the rules introducing the various substitutions we assume $\jterm{\Gamma}{A}{x}$;
in the rules introducing the definitional equalities we assume $\jtermeq{\Gamma}{A}{x}{x'}$.

\begin{infarray}{cc}
\inference{\jtype{\ctxext{\Gamma}{A}}{P}}{\jtype{\Gamma}{\subst{x}{P}}}
& \inference{\jtypeeq{\ctxext{\Gamma}{A}}{P}{P'}}{\jtypeeq{\Gamma}{\subst{x}{P}}{\subst{x'}{P'}}}\\
\inference{\jtype{\ctxext{{\Gamma}{A}}{P}}{Q}}{\jtype{\ctxext{\Gamma}{\subst{x}{P}}}{\subst{x}{Q}}}
& \inference{\jtypeeq{\ctxext{{\Gamma}{A}}{P}}{Q}{Q'}}{\jtypeeq{\ctxext{\Gamma}{\subst{x}{P}}}{\subst{x}{Q}}{\subst{x'}{Q'}}}\\
\inference{\jterm{\ctxext{\Gamma}{A}}{P}{f}}{\jterm{\Gamma}{\subst{x}{P}}{\subst{x}{f}}}
& \inference{\jtermeq{\ctxext{\Gamma}{A}}{P}{f}{f'}}{\jtermeq{\Gamma}{\subst{x}{P}}{\subst{x}{f}}{\subst{x'}{f'}}}\\
\inference{\jterm{\ctxext{{\Gamma}{A}}{P}}{Q}{g}}{\jterm{\ctxext{\Gamma}{\subst{x}{P}}}{\subst{x}{Q}}{\subst{x}{g}}}
& \inference{\jtermeq{\ctxext{{\Gamma}{A}}{P}}{Q}{g}{g'}}{\jtermeq{\ctxext{\Gamma}{\subst{x}{P}}}{\subst{x}{Q}}{\subst{x}{g}}{\subst{x'}{g'}}}
\end{infarray}

\subsubsection{Substitution is compatible with substitution}

We require that substitution is compatible with itself, which is roughly the
assertion that substitution is associative. However, we cannot just state that
$\subst{x}{{f}{g}}\jdeq\subst{{x}{f}}{g}$ since the expression $\subst{{x}{f}}{g}$
is not well-formed. The term $\subst{x}{f}$ can be substituted in (terms of) families over
$\subst{x}{P}$; the term $\subst{x}{g}$ is such. Therefore, associativity of
substitution takes the form $\subst{x}{{f}{g}}\jdeq\subst{{x}{f}}{{x}{g}}$.

In the following inference rules we assume
$\jterm{\Gamma}{A}{x}$ and $\jterm{\ctxext{\Gamma}{A}}{P}{f}$.

\begin{infarray}{c}
\inference{\jtype{\ctxext{{\Gamma}{A}}{P}}{Q}}
{\jtypeeq{\Gamma}{\subst{x}{{f}{Q}}}{\subst{{x}{f}}{{x}{Q}}}}\\
\inference{\jtype{\ctxext{{{\Gamma}{A}}{P}}{Q}}{R}}
{\jtypeeq{\ctxext{\Gamma}{\subst{x}{{f}{Q}}}}{\subst{x}{{f}{R}}}{\subst{{x}{f}}{{x}{R}}}}\\
\inference{\jterm{\ctxext{{\Gamma}{A}}{P}}{Q}{g}}
{\jtermeq{\Gamma}{\subst{x}{{f}{Q}}}{\subst{x}{{f}{g}}}{\subst{{x}{f}}{{x}{g}}}}\\
\inference{\jterm{\ctxext{{{\Gamma}{A}}{P}}{Q}}{R}{h}}
{\jtermeq{\ctxext{\Gamma}{\subst{x}{{f}{Q}}}}{\subst{x}{{f}{R}}}{\subst{x}{{f}{h}}}{\subst{{x}{f}}{{x}{h}}}}
\end{infarray}

\subsubsection{Substitution is compatible with extension}
Suppose $\jterm{\Gamma}{A}{x}$ in all of the following inference rules.
\begin{infarray}{c}
\inference{\jtype{\ctxext{{\Gamma}{A}}{P}}{Q}}{\jtypeeq{\Gamma}{\subst{x}{\ctxext{P}{Q}}}{\ctxext{\subst{x}{P}}{\subst{x}{Q}}}}
\end{infarray}

\subsubsection{Substitution is compatible with weakening}
\begin{equation*}
\inference{\jterm{\Gamma}{A}{x}\qquad\jtype{\ctxext{\Gamma}{A}}{P}\qquad\jtype{\ctxext{\Gamma}{A}}{Q}}
          {\jtypeeq{\ctxext{\Gamma}{\subst{x}{P}}}{\subst{x}{\ctxwk{P}{Q}}}{\ctxwk{\subst{x}{P}}{\subst{x}{Q}}}}
\end{equation*}

The judgmental equalities we're about to describe assert that weakening followed
by substitution leaves everything untouched. Thus, we get that each fiber
$\subst{x}{\ctxwk{A}{B}}$ is just $B$, that $\ctxwk{A}{y}$ is the constant function
mapping everything to $y:B$ and similar properties for families and terms thereof.

\begin{infarray}{c}
\inference{\jtype{\Gamma}{A}\qquad\jtype{\Gamma}{B}\qquad\jterm{\Gamma}{A}{x}}{\jtypeeq{\Gamma}{\subst{x}{\ctxwk{A}{B}}}{B}}\\
\inference{\jtype{\Gamma}{A}\qquad\jtype{\ctxext{\Gamma}{B}}{Q}\qquad\jterm{\Gamma}{A}{x}}{\jtypeeq{\ctxext{\Gamma}{B}}{\subst{x}{\ctxwk{A}{Q}}}{Q}}\\
\inference{\jterm{\Gamma}{A}{x}\qquad\jterm{\Gamma}{B}{y}}{\jtermeq{\Gamma}{B}{\subst{x}{\ctxwk{A}{y}}}{y}}\\
\inference{\jterm{\Gamma}{A}{x}\qquad\jterm{\ctxext{\Gamma}{B}}{Q}{g}}{\jtermeq{\ctxext{\Gamma}{B}}{Q}{\subst{x}{\ctxwk{A}{g}}}{g}}
\end{infarray}

Using the rules of the compatibility of substitution with weakening and of the
compatibility of weakening with itself, we see that we can show
\begin{equation*}
\jtypeeq{\Gamma}{\subst{f}{\ctxwk{\Gamma}{{\Delta}{\greek{E}}}}}{\ctxwk{\Gamma}{E}}
\end{equation*}
for any three contexts $\Gamma$, $\Delta$ and $\greek{E}$ and any $\jhom{\Gamma}{\Delta}{f}$.
It follows that for $\jhom{\Delta}{\greek{E}}{g}$ we have
\begin{equation*}
\jhom{\Gamma}{\greek{E}}{\subst{f}{\ctxwk{\Gamma}{g}}}
\end{equation*}
The term $\jcomp{\Gamma}{f}{g}$ is the composition of $g$ with $f$.

\subsection{The universal property of extension}

Using weakening and substitution we are able to state the universal property
for extension. It looks a bit more involved, since we cannot directly refer
to the variables in the contexts. On the other hand, we can now plainly see
were there were secretly weakenings going on.

We begin with stating the universal property of the extension $\ctxext{\Gamma}{A}$.
In these rules we assume we have $\jtype{\Gamma}{A}$ in the hypotheses.

\begin{infarray}{c}
\inference{}
{\jterm{\ctxext{\Gamma}{A}}{\ctxwk{A}{{\Gamma}{\ctxext{\Gamma}{A}}}}{\pair_A}}\\
\inference{
  \jtype{\ctxext{\Gamma}{A}}{P}
  \qquad
  \jterm{\ctxext{\Gamma}{A}}{\subst{\pair_A}{\ctxwk{A}{{\Gamma}{P}}}}{f}}
  {\jterm{\ctxext{\Gamma}{A}}{P}{\ind{\tfext_\Gamma(A)}(f)}}\\
\inference{
  \jtype{\ctxext{\Gamma}{A}}{P}
  \qquad
  \jterm{\ctxext{\Gamma}{A}}{\subst{\pair_A}{\ctxwk{A}{{\Gamma}{P}}}}{f}}
  {\jtermeq{\ctxext{\Gamma}{A}}{\subst{\pair_A}{\ctxwk{A}{{\Gamma}{P}}}}{\subst{\pair_A}{\ctxwk{A}{{\Gamma}{\ind{\tfext_\Gamma(A)}(f)}}}}{f}}
\end{infarray}

Under the hypothesis that $\jtypeeq{\Gamma}{A}{A'}$
we will also have the rules

\begin{infarray}{c}
\inference{}{\jtermeq{\ctxext{\Gamma}{A}}{\ctxwk{A}{{\Gamma}{\ctxext{\Gamma}{A}}}}{\pair_A}{\pair_{A'}}}\\
\inference{\jtype{\ctxext{\Gamma}{A}}{P}\qquad\jterm{\ctxext{\Gamma}{A}}{\subst{\pair_A}{\ctxwk{A}{{\Gamma}{P}}}}{f}}
{\jtermeq{\ctxext{\Gamma}{A}}{P}{\ind{\tfext_\Gamma(A)}(f)}{\ind{\tfext_\Gamma(A')}(f)}}
\end{infarray}

Note that we don't need the notion of terms for contexts to state the universal
property of context extension (which is a good thing, for we don't assume to have them).

Next, we give the universal property of the extension $\ctxext{A}{P}$ in context
$\Gamma$.
In all of the following inference rules we assume that $\jtype{\ctxext{\Gamma}{A}}{P}$
is among the hypothesis. The induction principle for extension consists of three
inference rules:

\begin{infarray}{c}
\inference{}
{\jterm{\ctxext{{\Gamma}{A}}{P}}{\ctxwk{P}{{A}{\ctxext{A}{P}}}}{\pair_P}}\\
\inference{
  \jtype{\ctxext{\Gamma}{{A}{P}}}{Q}
  \qquad
  \jterm{\ctxext{{\Gamma}{A}}{P}}{\subst{\pair_P}{\ctxwk{P}{{A}{Q}}}}{f}}
  {\jterm{\ctxext{\Gamma}{{A}{P}}}{Q}{\ind{\tfext_A(P)}(f)}}\\
\inference{
  \jtype{\ctxext{\Gamma}{{A}{P}}}{Q}
  \qquad
  \jterm{\ctxext{{\Gamma}{A}}{P}}{\subst{\pair_P}{\ctxwk{P}{{A}{Q}}}}{f}}
  {\jtermeq{\ctxext{{\Gamma}{A}}{P}}{\subst{\pair_P}{\ctxwk{P}{{A}{Q}}}}{\subst{\pair_P}{\ctxwk{P}{{A}{\ind{\tfext_A(P)}(f)}}}}{f}}
\end{infarray}

As with context extension, we shall require two more inference rules stating that
$\pair_P$ and $\ind{\tfext_A(P)}$ are invariant under judgmental equality.

\subsection{The variable rule -- identity functions}
The variable rule for contexts:
\begin{infarray}{c}
\inference{\jctx{\Gamma}}{\jterm{\Gamma}{\ctxwk{\Gamma}{\Gamma}}{\idfunc[\Gamma]}}\\
\inference{\jterm{\Gamma}{A}{x}}{\jtermeq{\Gamma}{A}{\subst{\idfunc[\Gamma]}{\ctxwk{\Gamma}{x}}}{x}}\\
\inference{\jterm{\Delta}{\ctxwk{\Delta}{\Gamma}}{g}}{\jtermeq{\Delta}{\ctxwk{\Delta}{\Gamma}}{\subst{g}{\ctxwk{\Delta}{\idfunc[\Gamma]}}}{g}}
\end{infarray}


The variable rule for types:
\begin{infarray}{c}
\inference{\jtype{\Gamma}{A}}{\jterm{\ctxext{\Gamma}{A}}{\ctxwk{A}{A}}{\idfunc[A]}}\\
\inference{\jterm{\Gamma}{A}{x}}{\jtermeq{\Gamma}{A}{\subst{x}{\idfunc[A]}}{x}}\\
\inference{\jtype{\ctxext{\Gamma}{A}}{P}}{\jtypeeq{\ctxext{\Gamma}{A}}{\subst{\idfunc[A]}{\ctxwk{A}{P}}}{P}}\\
\inference{\jterm{\ctxext{\Gamma}{A}}{P}{f}}{\jtermeq{\ctxext{\Gamma}{A}}{P}{\subst{\idfunc[A]}{\ctxwk{A}{f}}}{f}}\\
\inference{\jterm{\ctxext{\Gamma}{B}}{\ctxwk{B}{A}}{g}}{\jtermeq{\ctxext{\Gamma}{B}}{\ctxwk{B}{A}}{\subst{g}{\ctxwk{B}{\idfunc[A]}}}{g}}
\end{infarray}

Using these rules we can derive that there is a judgmental equality $\jtypeeq{\ctxext{\Gamma}{A}}{\subst{x}{{\idfunc[A]}{B}}}
{\subst{x}{{x}{B}}}$.

\subsection{Identity types}
The identity types may now be formulated just as in ordinary Martin\nobreakdash-L\"of
type theory. However, since all model categories have path objects, we will also
introduce identity types for contexts.

We will use the symbol $\reflsym$ slightly different than the book does. For us,
$\refl{A}$ is a term of $\subst{\idfunc[A]}{\idtypevar{A}}$ in context
$\ctxext{\Gamma}{A}$ and we will write $\subst{x}{\refl{A}}$ for the reflexivity
path at $x$ (which would have been denoted by $\refl{x}$ in the book).

A notable difference in the formulation of identity types is that in our current
setting we must state the elimination rule not only for families $P$ in context
$\ctxext{{\Gamma}{\ctxwk{\Gamma}{\Gamma}}}{\idtypevar{\Gamma}}$,
but also for families $Q$ in context 
$\ctxext{{{\Gamma}{\ctxwk{\Gamma}{\Gamma}}}{\idtypevar{\Gamma}}}{P}$. The reason
is that all operations have to be closed under slicing: everything may happen
in a context. Secretly, a reason is that we don't have dependent function types.
We wouldn't even be able to find the transport maps if we didn't state the
identity elimination in an extended context.

\begin{infarray}{c}
\inference{\jctx{\Gamma}}{\jtype{\ctxext{\Gamma}{\ctxwk{\Gamma}{\Gamma}}}{\idtypevar{\Gamma}}}\\
\inference{\jctx{\Gamma}}{\jterm{\Gamma}{\subst{\idfunc[\Gamma]}{\idtypevar{\Gamma}}}{\refl{\Gamma}}}\\
\inference{\jtype{\ctxext{{\Gamma}{\ctxwk{\Gamma}{\Gamma}}}{\idtypevar{\Gamma}}}{P}
           \qquad
           \jterm{\Gamma}{\subst{\refl{\Gamma}}{{\idfunc[\Gamma]}{P}}}{d}}
           {\jterm{\ctxext{{\Gamma}{\ctxwk{\Gamma}{\Gamma}}}{\idtypevar{\Gamma}}}{P}{\tfJ(d)}}\\
\inference{\jtype{\ctxext{{\Gamma}{\ctxwk{\Gamma}{\Gamma}}}{\idtypevar{\Gamma}}}{P}\qquad
\jterm{\Gamma}{\subst{\refl{\Gamma}}{{\idfunc[\Gamma]}{P}}}{d}}
{\jtermeq
  {\ctxext{{\Gamma}{\ctxwk{\Gamma}{\Gamma}}}{\idtypevar{\Gamma}}}
  {\subst{\refl{\Gamma}}{{\idfunc[\Gamma]}{P}}}
  {\subst{\refl{\Gamma}}{{\idfunc[\Gamma]}{\tfJ(d)}}}
  {d}}\\
\inference{\jtype{\ctxext{{{\Gamma}{\ctxwk{\Gamma}{\Gamma}}}{\idtypevar{\Gamma}}}{P}}{Q}
           \qquad
           \jterm{\ctxext{\Gamma}{\subst{\refl{\Gamma}}{{\idfunc[\Gamma]}{P}}}}{\subst{\refl{\Gamma}}{{\idfunc[\Gamma]}{Q}}}{d}}
           {\jterm{\ctxext{{{\Gamma}{\ctxwk{\Gamma}{\Gamma}}}{\idtypevar{\Gamma}}}{P}}{Q}{\tfJ(d)}}\\
\inference
  {\jtype
    {\ctxext{{{\Gamma}{\ctxwk{\Gamma}{\Gamma}}}{\idtypevar{\Gamma}}}{P}}
    {Q}
  \qquad
  \jterm
    {\ctxext{\Gamma}{\subst{\refl{\Gamma}}{{\idfunc[\Gamma]}{P}}}}
    {\subst{\refl{\Gamma}}{{\idfunc[\Gamma]}{Q}}}
    {d}}
  {\jtermeq
    {\ctxext{{{\Gamma}{\ctxwk{\Gamma}{\Gamma}}}{\idtypevar{\Gamma}}}{P}}
    {\subst{\refl{\Gamma}}{{\idfunc[\Gamma]}{Q}}}
    {\subst{\refl{\Gamma}}{{\idfunc[\Gamma]}{\tfJ(d)}}}
    {d}}
\end{infarray}

Likewise, we introduce identity types in a context.

\begin{infarray}{c}
\inference{\jtype{\Gamma}{A}}{\jtype{\ctxext{{\Gamma}{A}}{\ctxwk{A}{A}}}{\idtypevar{A}}}\\
\inference{\jtype{\Gamma}{A}}{\jterm{\ctxext{\Gamma}{A}}{\subst{\idfunc[A]}{\idtypevar{A}}}{\refl{A}}}\\
\inference{\jtype{\ctxext{{{\Gamma}{A}}{\ctxwk{A}{A}}}{\idtypevar{A}}}{P}\qquad
\jterm{\ctxext{\Gamma}{A}}{\subst{\refl{A}}{{\idfunc[A]}{P}}}{d}}
{\jterm{\ctxext{{{\Gamma}{A}}{\ctxwk{A}{A}}}{\idtypevar{A}}}{P}{\tfJ(d)}}\\
\inference{\jtype{\ctxext{{{\Gamma}{A}}{\ctxwk{A}{A}}}{\idtypevar{A}}}{P}\qquad
\jterm{\ctxext{\Gamma}{A}}{\subst{\refl{A}}{{\idfunc[A]}{P}}}{d}}
{\jtermeq
  {\ctxext{{{\Gamma}{A}}{\ctxwk{A}{A}}}{\idtypevar{A}}}
  {\subst{\refl{A}}{{\idfunc[A]}{P}}}
  {\subst{\refl{A}}{{\idfunc[A]}{\tfJ(d)}}}
  {d}}
\end{infarray}

Suppose we have terms $\jterm{\Gamma}{A}{x}$ and $\jterm{\Gamma}{A}{y}$. Then
we may define $\id[A]{x}{y}\defeq\subst{y}{{x}{\idtypevar{A}}}$. A term
$\jterm{\Gamma}{\id[A]{x}{y}}{p}$ is called an identification of $x$ and $y$.

\subsubsection{Basic properties of identity types}
In this subsubsection we prove some basic properties of identity types, just to
know whether we got the type theory right.

Suppose we have a family $\jtype{\ctxext{\Gamma}{A}}{P}$. Then we can consider
the family $\jtype{\ctxext{{\Gamma}{A}}{\ctxwk{A}{A}}}{\ctxwk{A}{P}}$, which has the role of the family $\jtype{\Gamma,\,x,y:A}{P(y)}$
of ordinary Martin-L\"of type theory. We may also consider the family
$\jtype{\ctxext{{\Gamma}{A}}{\ctxwk{A}{A}}}{\ctxwk{{A}{A}}{P}}$; this one has the
role of the family $\jtype{\Gamma,\,x,y:A}{P(x)}$. Those are families in the
same context, so we have
\begin{equation*}
\jtype{\ctxext{{{\Gamma}{A}}{\ctxwk{A}{A}}}{\ctxwk{{A}{A}}{P}}}{\ctxwk{\ctxwk{{A}{A}}{P}}{{A}{P}}}
\end{equation*}

\begin{lem}
There is a term
\begin{equation*}
\jterm{\ctxext{{{{\Gamma}{A}}{\ctxwk{A}{A}}}{\idtypevar{A}}}{\ctxwk{\idtypevar{A}}{{{A}{A}}{P}}}}{\ctxwk{\idtypevar{A}}{{{{A}{A}}{P}}{{A}{P}}}}{\transfibf{P}}
\end{equation*}
\end{lem}

\begin{proof}
By identity elimination it suffices to find a term
\begin{equation*}
\jterm{\ctxext{\Gamma}{A}}{\subst{\refl{A}}{{\idfunc[A]}{\ctxwk{\idtypevar{A}}{{{{A}{A}}{P}}{{A}{P}}}}}}{t}
\end{equation*}
By the judgmental equality $\jtypeeq{\Gamma}{\subst{{x}{f}}{{x}{Q}}}{\subst{x}{{f}{Q}}}$
it follows that we have the judgmental equalities
\begin{align*}
& \subst{\refl{A}}{{\idfunc[A]}{\ctxwk{\idtypevar{A}}{{{{A}{A}}{P}}{{A}{P}}}}}\\
& \qquad \jdeq \subst{\idfunc[A]}{{\ctxwk{{A}{A}}{\refl{A}}}{\ctxwk{\idtypevar{A}}{{{{A}{A}}{P}}{{A}{P}}}}}\\
& \qquad \jdeq \subst{\idfunc[A]}{\ctxwk{{{A}{A}}{P}}{{A}{P}}}\\
& \qquad \jdeq \ctxwk{\subst{\idfunc[A]}{\ctxwk{{A}{A}}{P}}}{\subst{\idfunc[A]}{\ctxwk{A}{P}}}\\
& \qquad \jdeq \ctxwk{P}{P}
\end{align*}
in context $\ctxext{{\Gamma}{A}}{P}$. We have the term $\jterm{\ctxext{{\Gamma}{A}}{P}}{\ctxwk{P}{P}}{\idfunc[P]}$.
\end{proof}

Suppose we have a term $\jterm{\ctxext{\Gamma}{A}}{P}{f}$. 

Using identity types, we can assert that a function $\jhom{\Gamma}{\Delta}{f}$ has
a left inverse $\jhom{\Delta}{\Gamma}{g}$ by asserting that there is an identification
\begin{equation*}
..
\end{equation*}

\section{Variable-free type theory -- other type constructors}
We will now describe the rules for the type constructors that we don't assume
in our type theory by default. The first of these are the dependent function
types. We also discuss $\tfW$-types, suspensions, truncations. More general higher inductive
types will have to wait until we have introduced models, for the models of the
basic type theory because we will use them as index categories of the diagrams.

\subsection{The unit type}
Since we don't have a notion of terms of a context, we just say that the context
$\unit$ is the terminal context.

\begin{infarray}{c}
\inference{}{\jctx{\unit}}\\
\inference{\jctx{\Gamma}}{\jhom{\Gamma}{\unit}{\tounit{\Gamma}}}\\
\inference{\jctx{\Gamma}\qquad\jhom{\Gamma}{\unit}{f}}{\jhomeq{\Gamma}{\unit}{f}{\tounit{\Gamma}}}
\end{infarray}

Note that we don't have to require that $\jhomeq{\Gamma}{\unit}{\tounit{\Gamma}}{\tounit{\Delta}}$
whenever we have a judgmental equality $\jctxeq{\Gamma}{\Delta}$, since this already follows from the third rule.

When the context $\unit$ is present, we may use the expression $\jtype{}{\Gamma}$
as a shorthand for the judgment $\jtype{\unit}{\ctxwk{\unit}{\Gamma}}$. Likewise,
we may use the expression $\jtermc{\Gamma}{i}$ as a shorthand
for the judgment $\jhom{\unit}{\Gamma}{i}$. If we have a type $A$ in context
$\Gamma$, we may use the expression $\jtype{}{\subst{i}{A}}$ to stand for
the judgment $\jtype{\unit}{\subst{i}{\ctxwk{\unit}{A}}}$. We see that in every
respect, contexts are types in the empty context.

We have created a strictly terminal object $\unit$. This is not necessary when
we're working in a context. We introduce the unit type $\unit_\Gamma$ in context
$\Gamma$ in the familiar type theoretical way.

\begin{infarray}{c}
\inference{\jctx{\Gamma}}{\jtype{\Gamma}{\unit_\Gamma}}\\
\inference{\jctx{\Gamma}}{\jterm{\Gamma}{\unit_\Gamma}{\ttt_\Gamma}}\\
\inference{\jtype{\ctxext{\Gamma}{\unit_\Gamma}}{P}\qquad\jterm{\Gamma}{\subst{\ttt_\Gamma}{P}}{u}}
          {\jterm{\ctxext{\Gamma}{\unit_\Gamma}}{P}{\ind{\unit_\Gamma}(u)}}\\
\inference{\jtype{\ctxext{\Gamma}{\unit_\Gamma}}{P}\qquad\jterm{\Gamma}{\subst{\ttt_\Gamma}{P}}{u}}
          {\jtermeq{\Gamma}{\subst{\ttt_\Gamma}{P}}{\subst{\ttt_\Gamma}{\ind{\unit_\Gamma}(u)}}{u}}
\end{infarray}

\subsection{Subterminal types}
The subterminal types we're about to present are strict, so they're more like \verb+Prop+
in \Coq. We can define subterminal types in two ways: the first equalizes all elements
of the subject type; the second is universal with the property that for every constant
map factors through it.

\subsubsection{Equalizing all terms}

\subsubsection{Factorizing constant maps}
Let $\jhom{\Gamma}{\Delta}{f}$. We can weaken $f$ by $\Gamma$ to obtain a term
$\jterm{\ctxext{\Gamma}{\ctxwk{\Gamma}{\Gamma}}}{\ctxwk{\Gamma}{{\Gamma}{\Delta}}}{\ctxwk{\Gamma}{f}}$.
This term is `like the function $\lam{x}{y}f(y)$'. Likewise, we can weaken $f$
by $\ctxwk{{\Gamma}{\Gamma}}$ to obtain a term
$\jterm{\ctxext{\Gamma}{\ctxwk{\Gamma}{\Gamma}}}{\ctxwk{\Gamma}{{\Gamma}{\Delta}}}{\ctxwk{{\Gamma}{\Gamma}}{f}}$,
which is `like the function $\lam{x}{y}f(x)$'. Since we have
the judgmental equality 
$\jtypeeq{\ctxext{\Gamma}{\ctxwk{\Gamma}{\Gamma}}}{\ctxwk{{\Gamma}{\Gamma}}{{\Gamma}{\Delta}}}{\ctxwk{\Gamma}{{\Gamma}{\Delta}}}$ we can consider the judgmental equality between $\ctxwk{\Gamma}{f}$
and $\ctxwk{{\Gamma}{\Gamma}}{f}$, which is exactly what we'll do in the definition
of judgmentally constant.

\begin{defn}
A term $\jhom{\Gamma}{\Delta}{f}$ is said to be \emph{judgmentally constant} if
the judgment
\begin{equation*}
\jtermeq{\ctxext{\Gamma}{\ctxwk{\Gamma}{\Gamma}}}{\ctxwk{{\Gamma}{\Gamma}}{{\Gamma}{\Delta}}}{\ctxwk{\Gamma}{f}}{\ctxwk{{\Gamma}{\Gamma}}{f}}
\end{equation*}
can be derived.
\end{defn}

\begin{defn}
\begin{infarray}{c}
\inference{\jctx{\Gamma}}{\jctx{\tau\Gamma}}\\
\inference{\jctx{\Gamma}}{\jhom{\Gamma}{\tau\Gamma}{t}}\\
\inference{\jctx{\Gamma}}{\jtermeq{\ctxext{\Gamma}{\ctxwk{\Gamma}{\Gamma}}}{\ctxwk{{\Gamma}{\Gamma}}{{\Gamma}{\Delta}}}{\ctxwk{\Gamma}{t}}{\ctxwk{{\Gamma}{\Gamma}}{t}}}\\
\inference{\jhom{\Gamma}{\Delta}{f}\qquad\jtermeq{\ctxext{\Gamma}{\ctxwk{\Gamma}{\Gamma}}}{\ctxwk{{\Gamma}{\Gamma}}{{\Gamma}{\Delta}}}{\ctxwk{\Gamma}{f}}{\ctxwk{{\Gamma}{\Gamma}}{f}}}{\jhom{\tau\Gamma}{\Delta}{\tilde{f}}}\\
\inference{\jhom{\Gamma}{\Delta}{f}\qquad\jtermeq{\ctxext{\Gamma}{\ctxwk{\Gamma}{\Gamma}}}{\ctxwk{{\Gamma}{\Gamma}}{{\Gamma}{\Delta}}}{\ctxwk{\Gamma}{f}}{\ctxwk{{\Gamma}{\Gamma}}{f}}}
{\jtermeq{\Gamma}{\ctxwk{\Gamma}{\Delta}}{\jcomp{\Gamma}{t}{\tilde{f}}}{f}}
\end{infarray}
\end{defn}

\begin{lem}
Any term $\jhom{\unit}{\Gamma}{i}$ is judgmentally constant.
\end{lem}

\begin{proof}

\end{proof}

\subsection{Product types}

\subsubsection{Products}
\begin{infarray}{c}
\inference{\jctx{\Gamma}\qquad\jctx{\Delta}}{\jctx{\product{\Gamma}{\Delta}}}\\
\inference{\jctx{\Gamma}\qquad\jctx{\Delta}}{\jhom{{\Gamma}{\Delta}}{\product{\Gamma}{\Delta}}{\pair}}
\end{infarray}

\subsubsection{Strict products}
As with the unit type, we may use the categorical description of the product
for our type theoretical definition. If we do that, we get strict products.

\begin{infarray}{cc}
\inference{\jctx{\Gamma}\qquad\jctx{\Delta}}{\jctx{\product{\Gamma}{\Delta}}}
 & \inference{\jhom{\greek{E}}{\Gamma}{f}\qquad\jhom{\greek{E}}{\Delta}{g}}
          {\jhom{\greek{E}}{\product{\Gamma}{\Delta}}{\pairp{f,g}}}\\
\inference{\jctx{\Gamma}\qquad\jctx{\Delta}}{\jhom{\product{\Gamma}{\Delta}}{\Gamma}{\proj1}}
 & \inference{\jhom{\greek{E}}{\Gamma}{f}\qquad\jhom{\greek{E}}{\Delta}{g}}
          {\jhomeq{\greek{E}}{\Gamma}{\jcomp{\greek{E}}{\pairp{f,g}}{\proj1}}{f}}\\
\inference{\jctx{\Gamma}\qquad\jctx{\Delta}}{\jhom{\product{\Gamma}{\Delta}}{\Delta}{\proj2}} 
 & \inference{\jhom{\greek{E}}{\Gamma}{f}\qquad\jhom{\greek{E}}{\Delta}{g}}
          {\jhomeq{\greek{E}}{\Delta}{\jcomp{\greek{E}}{\pairp{f,g}}{\proj2}}{g}}
\end{infarray}
\begin{equation*}
\inference{\jhom{\greek{E}}{\product{\Gamma}{\Delta}}{h}
           \qquad
           \jhomeq{\greek{E}}{\Gamma}{\jcomp{E}{h}{\proj1}}{f}
           \qquad
           \jhomeq{\greek{E}}{\Delta}{\jcomp{E}{h}{\proj2}}{g}}
          {\jhomeq{\greek{E}}{\product{\Gamma}{\Delta}}{h}{\pairp{f,g}}}
\end{equation*}

\subsection{Equalizer types}
Now that we have introduced a terminal object and products the categorical way,
we may just continue and present (strict) equalizers and pullbacks too, just to
see where we get. These notions are probably just not very useful in a univalent
setting until we got a good computational interpretation.

\begin{infarray}{c}
\inference{\jhom{\Gamma}{\Delta}{f}\qquad\jhom{\Gamma}{\Delta}{g}}{\jctx{\jequalizer{\Gamma}{f}{g}}}\\
\inference{\jhom{\Gamma}{\Delta}{f}\qquad\jhom{\Gamma}{\Delta}{g}}{\jhom{\jequalizer{\Gamma}{f}{g}}{\Gamma}{\jequalizerin{f}{g}}}\\
\inference{\jhom{\greek{E}}{\Gamma}{h}\qquad\jhomeq{\greek{E}}{\Delta}{\jcomp{\greek{E}}{h}{f}}{\jcomp{\greek{E}}{h}{g}}}{\jhom{\greek{E}}{\jequalizer{\Gamma}{f}{g}}{\jequalizer{h}{f}{g}}}\\
\inference{
  {\begin{array}{l}
    \jhom{\greek{E}}{\Gamma}{h}\\
    \jhomeq{\greek{E}}{\Delta}{\jcomp{\greek{E}}{h}{f}}{\jcomp{\greek{E}}{h}{g}}
  \end{array}}
  \qquad
  {\begin{array}{l}
    \jhom{\greek{E}}{\jequalizer{\Gamma}{f}{g}}{k}\\
    \jhomeq{\greek{E}}{\Gamma}{\jcomp{\greek{E}}{k}{\jequalizerin{f}{g}}}{h}
  \end{array}}
}
  {\jhom{\greek{E}}{\jequalizer{\Gamma}{f}{g}}{\jequalizer{h}{f}{g}}}
\end{infarray}

\subsection{Pullback types}

\subsection{Dependent function types}
\begin{infarray}{cc}
\inference{\jtype{\ctxext{\Gamma}{A}}{P}}{\jtype{\Gamma}{\mprd{A}{P}}}
& \inference{\jtypeeq{\ctxext{\Gamma}{A}}{P}{P'}}{\jtypeeq{\Gamma}{\mprd{A}{P}}{\mprd{A}{P'}}}\\
\inference{\jterm{\ctxext{\Gamma}{A}}{P}{f}}{\jterm{\Gamma}{\mprd{A}{P}}{\lambda(f)}}
& \inference{\jtermeq{\ctxext{\Gamma}{A}}{P}{f}{f'}}{\jtermeq{\Gamma}{\mprd{A}{P}}{\lambda(f)}{\lambda(f')}}\\
\inference{\jterm{\Gamma}{\mprd{A}{P}}{g}}{\jterm{\ctxext{\Gamma}{A}}{P}{\tfev(g)}}
& \inference{\jtermeq{\Gamma}{\mprd{A}{P}}{g}{g'}}{\jtermeq{\ctxext{\Gamma}{A}}{P}{\tfev(g)}{\tfev(g')}}\\
\inference{\jterm{\ctxext{\Gamma}{A}}{P}{f}}{\jtermeq{\Gamma}{A}{\tfev(\lambda(f))}{f}}
\end{infarray}

With these rules we will not get the weak $\eta$-rule when identity types are present.
So it might be better to state that $\lambda$ is a trivial cofibration.

\subsection{Subterminal types}
Let $\jhom{\Gamma}{\Delta}{f}$. Then we may also consider the type $\ctxwk{{\Gamma}{\Gamma}}{{\Gamma}{\Delta}}$
in context $\ctxwk{\Gamma}{\Gamma}$ and we have the terms
\begin{align*}
& \jhom{\ctxext{\Gamma}{\ctxwk{\Gamma}{\Gamma}}}{\ctxwk{\Gamma}{A}}{\ctxwk{\Gamma}{f}}\\
& \jhom{\ctxext{\Gamma}{\ctxwk{\Gamma}{\Gamma}}}{\ctxwk{\Gamma}{A}}{\ctxwk{{\Gamma}{\Gamma}}{f}}
\end{align*}

\subsection{The empty type}

\subsection{Coproduct types}

\subsection{The natural numbers}
